\chapter{Manual de instalación}
En este anexo se presentan algunos aspectos tecnicos del software desarrollado,
la forma correcta de obtenerlo, ademas de brindar una guia acerca de como debe
ser instalado y puesto en marcha.

\section{Licencia de Software}
Desde la misma concepción de este proyecto se considero la creación de una
herramienta de software libre, y asi es como se ha mantenido desde entonces.

Yachay es software libre bajo la licencia GPL version 3\footnote{Puede ahondarse
en esta definición en:
http://es.wikipedia.org/wiki/GNU\_General\_Public\_License}.

\section{Obtención del codigo fuente}
Inicialmente el proyecto estuvo alojado sobre un repositorio privado y
versionado bajo el sistema de control de versiones \emph{mercurial}, poco
despues de tener un primer prototipo estable, se cambio al un repositorio
publico en el sitio \emph{github}, y paso ser versionado bajo el sistema de
control de versiones \emph{git}; actualmente puede conseguir una copia del
software desde este servidor.

La dirección oficial para la obtencion del codigo fuente es:
\begin{quote}
https://github.com/ccaballero/yachay.git
\end{quote}

Desde este dirección web, puede obtenerse una copia en formato zip
(recomandable para implantación), o tambien puede clonarse una copia versionada
con un cliente GIT.

\section{Requisitos de software}
El software requiere ser desplegado en un servidor HTTP, que soporte la
interpretación del lenguaje de programación PHP, siendo recomendables dos de
ellos: Apache Web Server o EngineX. En esta guia Se presentarán ambas
posibilidades.

Ademas de un servidor HTTP, se requiere tener acceso a una base de datos MySQL o
MariaDB, para el almacenamiento de la información.

Respecto al lenguaje de programación, es requerido tener instalado PHP en su
version 5.3 o superior. Ademas de necesitarse tener las librerias del framework
Zend 1 (disponibles en el sitio web:
http://framework.zend.com/downloads/latest).

\section{Instalación}
Cabe recalcar que toda la hermenéutica descrita, se restringe a cualquier
distribución dal sistema operativo GNU/Linux, no siendo probado o corregido
para ningún otro sistema operativo.

\subsection{Archivo de hosts}
El primer paso para la instalación del sistema es la creacion de un host virtual
para esto se requiere crear un dominio, dependiendo de la mecanica de resolucion
de nombre de dominio esto puede cambiar, si no se utiliza un servidor DNS, lo
mas simple es modificar el archivo \emph{/etc/hosts}, para agregar una
resolución de dominio simple.

\begin{figure}
\begin{verbatim}
127.0.0.1    yachay.local    yachay
\end{verbatim}
\caption{Ejemplo de archivo de hosts}
\end{figure}

En este caso \emph{yachay.local} es el dominio sobre el que estamos instalando.

\subsection{Host virtual}
Hecho esto, se pasa a configurar un host virtual para el servidor de
aplicaciones web, para el caso del servidor apache, este se realizó de la
siguiente manera:

\small
\begin{verbatim}
<VirtualHost *:80>
    ServerName yachay.local
    ServerAlias *.yachay.local
    DocumentRoot /var/www/yachay/public

    SetEnv APPLICATION_ENV "production"

    LogLevel info
    ErrorLog "/var/www/yachay/logs/error.log"
    CustomLog "/var/www/yachay/logs/user-agent.log" "%{User-agent}i"
    CustomLog "/var/www/yachay/logs/referer.log" "%{Referer}i"
    CustomLog "/var/www/yachay/logs/resume.log" "%v %m %U%q"

    <Directory /var/www/yachay/public>
        DirectoryIndex index.php
        AllowOverride All
        Order allow,deny
        Allow from all
    </Directory>
</VirtualHost>
\end{verbatim}

Adicionalmente se realizaron pruebas sobre el servidor nginx, basados en el
siguiente archivo de configuracion:

\small
\begin{verbatim}
server {
    listen 80;
    server_name yachay.local;
    root /var/www/yachay/public;
    index index.php;

    client_max_body_size 40m;
    client_body_buffer_size 128k;

    access_log /var/log/nginx/yachay.access_log main;
    error_log /var/log/nginx/yachay.error_log debug;

    include /etc/nginx/drop.conf;
    location / {
        if (!-f $request_filename) {
            rewrite ^(.*)$ /index.php last;
            break;
        }
    }
    location ~ \.php$ {
        include /etc/nginx/fastcgi.conf;
        fastcgi_pass 127.0.0.1:9000;
        fastcgi_index /index.php;
        fastcgi_param APPLICATION_ENV development;
    }
}
\end{verbatim}

En ambos casos \emph{/var/www/yachay} es el sitio donde esta instalado.

\subsection{Archivo .htaccess}
Como parte de la definición de la arquitectura diseñada. se detalla también el
archivo .htaccess utilizado, para manejar varios formatos de respuesta de la
petición, y asegurar la calidad de las url, se define como se muestra a
continuación:

\begin{verbatim}
RewriteEngine on
RewriteCond %{SCRIPT_FILENAME} !-f
RewriteCond %{HTTP_HOST} ^([a-z]*)\.yachay\.local$ [NC]
RewriteRule ^(.*)$ index.php?format=%1 [L,QSA]
RewriteCond %{SCRIPT_FILENAME} !-f
RewriteCond %{HTTP_HOST} ^yachay\.local$ [NC]
RewriteRule ^(.*)$ index.php?format=www [L,QSA]
\end{verbatim}

Pueden verse dos reglas; la primera para cualquier subdominio que los paquetes
requieran; la segunda para el dominio principal, de modo tal que en cualquier
caso se envié por petición GET la variable "format" que define el tipo de
respuesta que se espera.

\section{Sobre módulos, controladores y acciones}
Todo los recursos (no importando de que tipo, es decir, en su mayor grado de
abstracción) poseen en esencia dos controladores iniciales:

\begin{description}
\item [index (list)] Que presenta una lista de los recursos disponibles.
\item [manager (admin)] Que presenta el conjunto de funciones disponibles sobre
los recursos.
\end{description}

Estas funciones sobre los recursos, dependiendo del paquete, pueden ser:

\begin{description}
\item [new (put)] Función de creación de un nuevo recurso.
\item [view (get)] Función de presentación de la información de un recurso.
\item [edit (post)] Función de edición del recurso.
\item [delete (delete)] Función de eliminación del recurso.
\item [lock] Función de bloqueo o deshabilitación del recurso.
\item [unlock] Función de desbloqueo o habilitación del recurso.
\end{description}

\section{Paquetes construidos}
Si bien, ya existían una gran cantidad de paquetes (antes denominados módulos),
contemplando varios aspectos, como la modularidad, el acoplamiento, entre otros;
fueron readecuados a los nuevos requerimientos, muchos fueron eliminados, otros
nuevos se construyeron, o fusionaron. En el cuadro~\ref{modulos_actuales} se
detallan los paquetes de los que se dispone en la actualidad y la función que
desempeñan.

El objetivo de los siguientes capítulos tratan exclusivamente de la explicación detallada de estos paquetes, detallando sus especificaciones y justificación de existencia, enfatizando en como están relacionados con los objetivos del proyecto.

Adicionalmente, en el cuadro~\ref{modulos_eliminados} se detallan los que antes eran módulos y ahora no están presentes, también se justifica su eliminación.

\begin{table}
\begin{tabular}{l|l}
Modulo & Detalle \\
\hline
frontpage & Sus funciones fueron incluidas en el paquete \emph{spaces}. \\
menus & Sus funciones fueron incluidas en el paquete \emph{templates}. \\
paginator & Sus funciones fueron incluidas en el paquete \emph{spaces}. \\
regions & Sus funciones fueron incluidas en el paquete \emph{templates}. \\
toolbar & Sus funciones fueron incluidas en el paquete \emph{templates}. \\
\end{tabular}
\caption{Módulos que no llegaron a ser paquetes.}
\label{modulos_eliminados}
\end{table}

