\chapter{Manual de instalación}
En este anexo se presentan algunos aspectos técnicos del software desarrollado,
la forma correcta de obtenerlo, además de brindar una guía acerca de como debe
ser instalado y puesto en marcha.

\section{Licencia de Software}
Desde la misma concepción de este proyecto se considero la creación de una
herramienta de software libre, y así es como se ha mantenido desde entonces.

Yachay es software libre bajo la licencia GPL versión 3\footnote{Puede ahondarse
en esta definición en:
http://es.wikipedia.org/wiki/GNU\_General\_Public\_License}.

\section{Obtención del código fuente}
Inicialmente el proyecto estuvo alojado sobre un repositorio privado y
versionado bajo el sistema de control de versiones \emph{mercurial}, poco
después de tener un primer prototipo estable, se migro a un repositorio
publico en el sitio \emph{github.com}, y paso ser versionado bajo el sistema de
control de versiones \emph{git}; actualmente se puede conseguir una copia del
software desde este servidor.

La dirección oficial para la obtención del código fuente es:
\begin{quote}
https://github.com/ccaballero/yachay.git
\end{quote}

Desde esta dirección web, puede obtenerse una copia en formato zip
(recomendable para implantación), o también puede clonarse una copia versionada
con un cliente GIT (recomendable para modificación).

\section{Requisitos de software}
El software requiere ser desplegado en un servidor HTTP, que soporte la
interpretación del lenguaje de programación PHP, siendo recomendables dos de
ellos: Apache Web Server o Nginx. En esta guia Se presentarán ambas
posibilidades.

Además de un servidor HTTP, se requiere tener acceso a una base de datos MySQL o
MariaDB, para el almacenamiento de la información.

Respecto al lenguaje de programación, es requerido tener instalado PHP en su
versión 5.3 o superior. Además de necesitarse tener las librerías del framework
Zend 1 (disponibles en el sitio web:
http://framework.zend.com/downloads/latest).

\section{Instalación}
Ahora describiremos todo el proceso necesario para la instalación de la
aplicación web, este proceso consta de los siguientes pasos:

\begin{enumerate}
\item Registro de un dominio.
\item Configuracion de un host virtual en el servidor web.
\item Copiado de los archivos en un directorio accesible por el servidor web.
\item Creación de una base de datos con permisos de creacion de tablas.
\item Creación del esquema de base de datos provisto por la aplicación web.
\item Configuración de la base de datos en la aplicacion web.
\item Primer acceso y configuración de la cuenta administrativa.
\end{enumerate}

Cabe recalcar que toda la hermenéutica descrita, se restringe a cualquier
distribución del sistema operativo GNU/Linux, no siendo probado o corregido
para ningún otro sistema operativo.

\subsection{Registro de dominio}
El primer paso para la instalación del sistema es la creación de un host virtual
para esto se requiere crear un dominio, dependiendo de la mecánica de resolución
de nombre de dominio esto puede cambiar, si no se utiliza un servidor DNS, lo
mas simple es modificar el archivo \emph{/etc/hosts}, para agregar una
resolución de dominio simple, en la figura \ref{config_hosts} puede verse un
ejemplo de archivo \emph{hosts}, en el cual se agrego la linea para resolución
del dominio \emph{yachay.local}.

\begin{figure}
\begin{verbatim}
# /etc/hosts: Local Host Database
#
# This file describes a number of aliases-to-address mappings for the for 
# local hosts that share this file.
#
# In the presence of the domain name service or NIS, this file may not be 
# consulted at all; see /etc/host.conf for the resolution order.
#

# IPv4 and IPv6 localhost aliases
127.0.0.1       localhost
::1             localhost

# Virtual hosts
127.0.0.1       yachay.local        yachay
\end{verbatim}
\caption{Ejemplo de archivo de hosts}
\label{config_hosts}
\end{figure}

\subsection{Host virtual}
Una vez agregado un nombre de dominio para resolver, es tiempo de configurar el
Servidor HTTP que se vaya a utilizar, dependiendo si el servidor sea Apache Web
Server o Nginx, esto es diferente.

Se presentarán ambas configuraciones, para lo cual se debe tener los ficheros
de la aplicación web en la dirección \emph{/var/www/yachay}, esta dirección es
la que se utilizará en los ejemplos.

\subsubsection{Apache Web Server 2.2}
Para configurar un hosts virtual en el servidor web Apache, es necesario añadir
una directiva, dependiendo de la distribución esto puede ser realizado de muchas
maneras, se recomienda revisar cual es la especifica para la distribución que se
este utilizando.

En la figura \ref{config_apache} se muestra la directiva a añadir y la
configuración que esta debe contener, para el dominio que se registro
anteriormente.

\begin{figure}
\begin{verbatim}
<VirtualHost *:80>
    ServerName yachay.local
    ServerAlias *.yachay.local
    DocumentRoot /var/www/yachay/public

    SetEnv APPLICATION_ENV "production"

    LogLevel info
    ErrorLog "/var/www/yachay/logs/error.log"
    CustomLog "/var/www/yachay/logs/user-agent.log" "%{User-agent}i"
    CustomLog "/var/www/yachay/logs/referer.log" "%{Referer}i"
    CustomLog "/var/www/yachay/logs/resume.log" "%v %m %U%q"

    <Directory /var/www/yachay/public>
        DirectoryIndex index.php
        AllowOverride All
        Order allow,deny
        Allow from all
    </Directory>
</VirtualHost>
\end{verbatim}
\caption{Host virtual para Apache Web Server}
\label{config_apache}
\end{figure}

\subsubsection{Nginx}
Adicionalmente se realizaron pruebas sobre el servidor nginx, para el uso de un
host virtual se debe añadir la configuración presentada en la figura
\ref{config_nginx}.

\begin{figure}
\begin{verbatim}
server {
    listen 80;
    server_name yachay.local;
    root /var/www/yachay/public;
    index index.php;

    client_max_body_size 40m;
    client_body_buffer_size 128k;

    access_log /var/log/nginx/yachay.access_log main;
    error_log /var/log/nginx/yachay.error_log debug;

    include /etc/nginx/drop.conf;
    location / {
        if (!-f $request_filename) {
            rewrite ^(.*)$ /index.php last;
            break;
        }
    }
    location ~ \.php$ {
        include /etc/nginx/fastcgi.conf;
        fastcgi_pass 127.0.0.1:9000;
        fastcgi_index /index.php;
        fastcgi_param APPLICATION_ENV development;
    }
}
\end{verbatim}
\caption{Host virtual para Nginx}
\label{config_nginx}
\end{figure}

