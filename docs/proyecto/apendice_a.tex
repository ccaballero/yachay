\chapter{Manual de instalación}
En este anexo se presentan algunos aspectos técnicos del software desarrollado,
la forma correcta de obtenerlo, además de brindar una guía acerca de como debe
ser instalado y puesto en marcha.

\section{Licencia de Software}
Desde la misma concepción de este proyecto se considero la creación de una
herramienta de software libre, y así es como se ha mantenido desde entonces.

Yachay es software libre bajo la licencia GPL versión 3\footnote{Puede ahondarse
en esta definición en:
http://es.wikipedia.org/wiki/GNU\_General\_Public\_License}.

\section{Obtención del código fuente}
Inicialmente el proyecto estuvo alojado sobre un repositorio privado y
versionado bajo el sistema de control de versiones \emph{mercurial}, poco
después de tener un primer prototipo estable, se migro a un repositorio
publico en el sitio \emph{github.com}, y paso ser versionado bajo el sistema de
control de versiones \emph{git}; actualmente se puede conseguir una copia del
software desde este servidor donde a la fecha existen dos releases estables, se
recomienda usar el release 0.1.

La dirección oficial para la obtención del código fuente es:
\begin{quote}
https://github.com/ccaballero/yachay.git
\end{quote}

Desde esta dirección web, puede obtenerse una copia en formato zip
(recomendable para implantación), o también puede clonarse una copia versionada
con un cliente GIT (recomendable para modificación).

\section{Requisitos de software}
El software requiere ser desplegado en un servidor HTTP, que soporte la
interpretación del lenguaje de programación PHP, siendo probados dos de
ellos: Apache Web Server o Nginx. En esta guia se presentarán ambas
posibilidades.

Además de un servidor HTTP, se requiere tener acceso a una base de datos MySQL o
MariaDB, para el almacenamiento de la información.

Respecto al lenguaje de programación, es requerido tener instalado PHP en su
versión 5.3 o superior. Además de necesitarse tener las librerías del framework
Zend 1 (disponibles en el sitio web:
http://framework.zend.com/downloads/latest).

\section{Instalación}
Ahora describiremos el proceso necesario para la instalación de la aplicación
web, este proceso consta de los siguientes pasos:

\begin{enumerate}
\item Copiado de los archivos en un directorio accesible por el servidor web.
\item Registro de dominio.
\item Configuracion de host virtual en el servidor web.
\item Creación de la base de datos y configuración de permisos.
\item Creación del esquema de base de datos provisto por la aplicación web.
\item Configuración de la aplicacion web.
\item Creación de permisos especiales para la aplicación web.
\item Primer acceso y configuración de la cuenta administrativa.
\end{enumerate}

Cabe recalcar que toda la hermenéutica descrita, se restringe a cualquier
distribución del sistema operativo GNU/Linux, no siendo probado o corregido
para ningún otro sistema operativo.

\subsection{Copiado de archivos}
El primer paso es copiar el codigo fuente obtenido a un carpeta accesible al
servidor web, en la mayoria de los casos esta carpeta se encuentra en
\emph{/var/www/}, si se obtuvo una copia en formato zip, esta debe ser
descompresa en ese directorio, si se clono el repositorio, debe clonarse en el
mismo directorio; de forma tal que exista una carpeta \emph{yachay}.

\subsection{Registro de dominio}
A continuación debe procederse a el registro de un dominio para el despliegue de
la aplicación web, dependiendo de la mecánica de resolución de nombres de
dominio y su infraestructura IT esto puede cambiar, si no se utiliza un servidor
DNS, lo mas simple es modificar el archivo \emph{/etc/hosts}, para agregar una
resolución de dominio simple, en la figura \ref{config_hosts} puede verse un
ejemplo de archivo \emph{hosts}, en el cual se agrego la linea para resolución
del dominio \emph{yachay.local}.

\begin{figure}
\begin{verbatim}
# /etc/hosts: Local Host Database
#
# This file describes a number of aliases-to-address mappings for the for 
# local hosts that share this file.
#
# In the presence of the domain name service or NIS, this file may not be 
# consulted at all; see /etc/host.conf for the resolution order.
#

# IPv4 and IPv6 localhost aliases
127.0.0.1       localhost
::1             localhost

# Virtual hosts
127.0.0.1       yachay.local        yachay
\end{verbatim}
\caption{Ejemplo de archivo de hosts}
\label{config_hosts}
\end{figure}

\subsection{Creación de host virtual}
Una vez agregado un nombre de dominio, es tiempo de configurar el Servidor HTTP
que se vaya a utilizar, dependiendo si el servidor sea Apache Web Server o
Nginx, este es un proceso diferente.

El término Hosting Virtual se refiere a hacer funcionar más de un sitio web en
una sola máquina. Los sitios web virtuales pueden estar ``basados en direcciones
IP'', lo que significa que cada sitio web tiene una dirección IP diferente, o
``basados en nombres diferentes'', lo que significa que con una sola dirección
IP están funcionando sitios web con diferentes nombres (de dominio). El hecho de
que estén funcionando en la misma máquina física pasa completamente
desapercibido para el usuario que visita esos sitios web\cite{Apache}.

\subsubsection{Apache Web Server 2.2}
Para configurar un hosts virtual en el servidor web Apache, es necesario añadir
una directiva, dependiendo de la distribución esto puede ser realizado de muchas
maneras, se recomienda revisar cual es la especifica para la distribución que se
este utilizando.

En la figura \ref{config_apache} se muestra la directiva a añadir y la
configuración que esta debe contener, para el dominio que se registro
anteriormente.

\begin{figure}
\begin{verbatim}
<VirtualHost *:80>
    ServerName yachay.local
    ServerAlias *.yachay.local
    DocumentRoot /var/www/yachay/public

    SetEnv APPLICATION_ENV "production"

    LogLevel info
    ErrorLog "/var/www/yachay/logs/error.log"
    CustomLog "/var/www/yachay/logs/user-agent.log" "%{User-agent}i"
    CustomLog "/var/www/yachay/logs/referer.log" "%{Referer}i"
    CustomLog "/var/www/yachay/logs/resume.log" "%v %m %U%q"

    <Directory /var/www/yachay/public>
        DirectoryIndex index.php
        AllowOverride All
        Order allow,deny
        Allow from all
    </Directory>
</VirtualHost>
\end{verbatim}
\caption{Host virtual para Apache Web Server}
\label{config_apache}
\end{figure}

\subsubsection{Nginx}
Nginx es un servidor HTTP ligero de alto rendimiento, y actualmente es el
segundo mas utilizado despues de Apache\footnote{Datos extraido de:
http://w3techs.com/technologies/overview/web\_server/all}.

Inicialmente el proyecto fue desarrollado y probado para la ejecución en el
servidor Apache, posteriormente se hicieron pruebas de despliegue sobre el
servidor Nginx siendo estas satisfactorias, en la actualidad el sistema esta
soportado y probado para las dos tecnologias, para el uso de un host virtual se
debe añadir la configuración presentada en la figura \ref{config_nginx}.

\begin{figure}
\begin{verbatim}
server {
    listen 80;
    server_name yachay.local;
    root /var/www/yachay/public;
    index index.php;

    client_max_body_size 40m;
    client_body_buffer_size 128k;

    access_log /var/log/nginx/yachay.access_log main;
    error_log /var/log/nginx/yachay.error_log debug;

    include /etc/nginx/drop.conf;
    location / {
        if (!-f $request_filename) {
            rewrite ^(.*)$ /index.php last;
            break;
        }
    }
    location ~ \.php$ {
        include /etc/nginx/fastcgi.conf;
        fastcgi_pass 127.0.0.1:9000;
        fastcgi_index /index.php;
        fastcgi_param APPLICATION_ENV development;
    }
}
\end{verbatim}
\caption{Host virtual para Nginx}
\label{config_nginx}
\end{figure}

\subsection{Creación y configuración de la base de datos}
Yachay fue construido unicamente para soportar al servidor de base de datos
MySQL, posteriormente se hicieron pruebas de compatibilidad sobre MariaDB que
resultaron satisfactorias.

Para este paso debe ser creada una base de datos con codificación de caracteres
UTF8 sobre un servidor de base de datos que sea accesible desde el servidor
HTTP, en la figura \ref{config_mysql1} pueden apreciarse los comandos necesarios
para la creacion y asignación de privilegios a un usuario llamado \emph{carlos}
e identificado con la contraseña \emph{asdf}.

\begin{figure}
\begin{verbatim}
~ $ mysql --user=root --host=127.0.0.1 -p
Enter password: 
Welcome to the MariaDB monitor.  Commands end with ; or \g.
Your MariaDB connection id is 6
Server version: 10.0.14-MariaDB-log Source distribution

Copyright (c) 2000, 2014, Oracle, SkySQL Ab and others.

Type 'help;' or '\h' for help. Type '\c' to clear the current input statement.

MariaDB [(none)]> CREATE DATABASE yachay CHARACTER SET UTF8;
Query OK, 1 row affected (0,02 sec)

MariaDB [(none)]> GRANT all ON yachay.* TO 'carlos'@'127.0.0.1' IDENTIFIED BY 'asdf';
Query OK, 0 rows affected (0,01 sec)

MariaDB [(none)]> use yachay
Database changed
MariaDB [yachay]> show tables;
Empty set (0,00 sec)

MariaDB [yachay]> 
\end{verbatim}
\caption{Configuración del servidor de base de datos}
\label{config_mysql1}
\end{figure}

\subsection{Creación del esquema de base de datos}
Una vez creada y configurada la base de datos, se debe ejecutar el script SQL,
para la creación de las tablas y los datos básicos en la nueva base de datos.

Los scripts necesarios se encuentran el directorio \emph{/data/sql}, estos deben
ser ejecutados como se presenta en la figura \ref{config_mysql2}.

\begin{figure}
\begin{verbatim}
$ cd /var/www/yachay/data/sql/
/var/www/yachay/data/sql
$ mysql --user=carlos --host=127.0.0.1 --database=yachay -p < _install.sql 
Enter password: 
base
packages
routes
widgets
privileges
roles
spaces
users
login
profile
settings
friends
invitations
gestions
subjects
areas
careers
groups
teams
communities
groupsets
resources
notes
links
files
events
photos
videos
evaluations
califications
feedback
comments
ratings
valorations
tags
templates
analytics
register
extra
\end{verbatim}
\caption{Creación del esquema de base de datos}
\label{config_mysql2}
\end{figure}

\subsection{Configuración de la aplicación web}
Una vez creados el esquema y los datos básicos para la aplicación, es necesario
pasar los datos de acceso a la base de datos a la aplicación web.

En la aplicación web se cuenta con dos archivos para la configuración, ambos
ubicados en la carpeta \emph{/configs/}, estos archivos son:
\emph{application.ini} donde se encuentras los valores por defecto de todas las
posibles variables que parametrizan a la aplicación web, \emph{local.ini} es el
archivo donde deben configurarse los parametros que se deseen cambiar. Estos
archivos siguen las reglas de configuración de los archivos INI, y pueden ser
configurados de formas muy complejas, incluso hacerse multiples instancias del
mismo sistema pero con configuraciones diferentes\footnote{Pueden verse la
estructura y formato en: https://en.wikipedia.org/wiki/INI\_file}.

En la figura \ref{yachay_config1} pueden apreciarse algunas de las variables que
comunmente son modificadas, y las cuales son detalladas a continuación:

\begin{figure}
\begin{verbatim}
[production]
resources.frontController.baseUrl = 

resources.layout.layout = webarte

resources.db.adapter = Mysqli
resources.db.params.charset = utf8
resources.db.params.host = localhost
resources.db.params.username = carlos
resources.db.params.password = asdf
resources.db.params.dbname = yachay

resources.mail.transport.username = 
resources.mail.transport.password = 

system.email_direction = 
system.email_name = 

system.title = " ~ yachay ~ "
system.url = http://yachay.local
system.key = 
system.timezone = "America/La_Paz"
system.locale = "es_BO"

[staging : production]

[testing : production]

[development : production]
\end{verbatim}
\caption{Archivo de configuración local.ini}
\label{yachay_config1}
\end{figure}

\begin{description}
\item [resources.frontController.baseUrl] Utilizada para la instalación
sobre un hosting compartido.
\item [resources.layout.layout] Utilizada para la configuración de la
plantilla por defecto para todo el sistema.
\item [resources.db.adapter] Segun el tipo de conector de PHP-MySQL se tenga
este debe tener unos de los siguientes valores: \emph{mysql}, \emph{mysqli}, o
\emph{pdo\_mysql}.
\item [resources.db.params.charset] Determina la codificación de caracteres que
se utilizará para la conexion a la base de datos, se recomienda siempre el uso
de la codificación UTF8;
\item [resources.db.params.host] Determina la IP o URL donde se localiza el
servidor de base de datos.
\item [resources.db.params.username] Determina el nombre del usuario autorizado
en el servidor de base de datos, con el que se manipulara la información.
\item [resources.db.params.password] Determina la contraseña del usuario
autorizado en el servidor de base de datos, con el que se manipulara la
información.
\item [resources.db.params.dbname] Determina el nombre de la base de datos donde
se haya instalado el esquema de base de datos para la aplicacion web.
\item [resources.mail.transport.username] Determina el nombre de usuario para el
servidor SMTP, encargado del manejo de correo electronico.
\item [resources.mail.transport.password] Determina la contraseña a utilizar
para el servidor SMTP, encargado del manejo de correo electronico.
\item [system.email\_direction] Determina la dirección de correo electronico que
será el remitente de todo el correo electronico enviado.
\item [system.email\_name] Determina el nombre para el correo electronico que
será el remitente de todo el correo electronico enviado.
\item [system.title] Variable que determina el nombre del sistema que se
mostrará en el navegador.
\item [system.url] Variable que determina la dirección del sitio web donde se
instalo la aplicacion web, este es muy usada en el envio de correo electronico.
\item [system.key] Hash utilizado para la identificación univoca de la
aplicacion web.
\item [system.timezone] Determina la zona horaria del lugar geografico donde se
encuentra el servidor que aloja la aplicacion web.
\item [system.locale] Un \emph{locale} es un conjunto de parámetros que define
el idioma, país y cualquier otra preferencia especial que el usuario desee ver
en su interfaz de usuario. Se utiliza mucho en el framework Zend.
\end{description}

