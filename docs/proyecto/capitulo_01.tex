\chapter{Introducción}

Con el auge de los últimos años con respecto a la red social
Facebook\cite{Jeria}, se ha notado un gran cambio en la mentalidad de las
personas con respecto a su entorno, compartiendo recursos e intercambiando
ideas, se han abierto grandes posibilidades para un salto en las viejas
concepciones respecto a lo que concierne a las formas de aprendizaje y la
gestión del conocimiento.

Aunque los cambios han sido positivos, aún pueden concebirse nuevas e
innovadoras maneras para obtener una gran retroalimentación entre los
estudiantes, con una forma más de asistir a la educación en las aulas.

En este documento se detalla todo el proceso de construcción de una red social
orientada a tópicos netamente académicos, intentando de alguna manera reducir
los métodos estrictamente formales en la relación entre el educador y sus
alumnos. Y de esta forma obtener una mayor integración entre estudiantes, y
docentes, fomentando de esa forma la interacción, comunicación y colaboración
entre todos los involucrados.

En este capitulo se definen los problemas, los objetivos fundamentales, y los
factores que despertaron el interés por resolverlos.

\section{Antecedentes}
Con la creciente accesibilidad de las personas al uso de Internet, es bastante
claro que el rol ha cambiado, se ha pasado de un conjunto amplio de simples
consumidores de recursos, a ser participes en tareas de creación, publicación,
categorización, y valoración de los recursos, es decir “Pasar de ser
consumidores de información en Internet a ser productores de contenidos,
información y conocimiento”\cite{Rodriguez}.

Todo esto ha abierto un nuevo camino hacia nuevas formas de interrelación
social, que ofrecen una inmejorable oportunidad en el campo de lo educativo,
colaborando en el apoyo y mejora de los métodos de aprendizaje. Aprovechando
oportunidades como el despertar de la Web 2.0, que es una “Revolución social
más que tecnológica, que da un énfasis especial al intercambio abierto del
conocimiento”\cite{Rodriguez}. Redes sociales como Hi5, Facebook, MySpace,
Orkut, LinkedIn entre otras, permiten a sus usuarios almacenar, organizar y
compartir recursos como fotos, vídeos, etc. Además de crear comunidades de
personas agrupadas por un interés común.

También existen otras posibilidades, que son mas orientadas a asistir al
aprendizaje, como ser: Moodle o Elgg; grandes sistemas que cuentan con el apoyo
de muchas instituciones educativas y desarrolladores, “que permiten al docente
contextualizar al aula, la utilización de las diferentes herramientas
tecnológicas que tendrá a su disposición, para atender las necesidades
específicas de aprendizaje, que previamente haya identificado en su labor
docente”\cite{Gonzalez}.

\section{Definición del problema}
Se ha observado que los docentes se ven sobrecargados de actividades, que en
parte podrían ser simplificadas, ya sea en manejar toda la logística de un
espacio virtual para su materia, o en la misma atención que debe brindar a los
estudiantes.

“El tutor debe atender a un elevado número de alumnos, ante la imposibilidad de
atender este trabajo se recurre a dejar de lado a aquellos alumnos que no
insisten, se utilizan mensajes genéricos o fragmentos de textos copiados y
pegados sin excesivo cuidado, se leen los mensajes de los alumnos de modo
rápido, ignorando aspectos o matices importantes”\cite{Bartolome}.

Además de notar que los estudiantes al ver el modelo actual que deben seguir en
sus estudios superiores, van perdiendo progresivamente el interés por compartir
sus ideas y experiencias; conocimiento que podría servir a otros estudiantes en
la construcción de sus propios criterios.

Pero en los estudiantes que ya poseen una solida rutina de participación la
dificultad viene sumida en la amplia variedad de sitios orientados a la
provisión de recursos, despertando una necesidad de centralizar todos estos
recursos en un solo lugar.

Por lo mencionado se define el problema como:

\emph{“La escasa interacción académica entre docentes, y estudiantes conduce al
uso de métodos deficientes de adquisición del conocimiento”}.

\section{Objetivos}

\subsection{Objetivo General}
Promover el intercambio de información entre los estudiantes, mediante el uso
de una red social para mejorar los métodos de adquisición del conocimiento.

\subsection{Objetivos Específicos}
\begin{itemize}
\item Agilizar la creación de espacios virtuales para incrementar la cantidad y
variabilidad de estos.
\item Facilitar el intercambio de recursos entre los estudiantes para acelerar
la adquisición de experiencia.
\item Facilitar el intercambio de recursos entre distintas instancias del
sistema para mejorar la disponibilidad de recursos.
\item Mejorar los canales de comunicación entre estudiantes y docentes para
facilitar la retroalimentación.
\item Planear estrategias que fomenten la participación para mantener activo el
sistema.
\end{itemize}

\section{Justificación}
La construcción de una red social por definición está inmersa en ese mundo de
vida propia, que es Internet; por tanto se nutre de todo lo que ella puede
proveer, y todo lo que en ella se pueda construir.

Se intenta también posibilitar el gran ahorro de tiempo, tanto para los
estudiantes, que podrán reutilizar contenidos de otras personas, además de
tenerlos a disposición en cualquier momento; como para los docentes, que se
verán apoyados en su misión de enseñanza por nuevos canales de comunicación,
facilitando así todo el proceso de enseñanza-aprendizaje.

En el aspecto social, promueve la comunicación y fomenta la comunión entre
personas con distintos grados de conocimiento, haciendo que unos puedan conocer
y decidir que caminos pueden seguir, y a otros mostrando las ventajas y/o
desventajas que pueden encontrar en el camino a sus objetivos.

Sin una manera de captar, promover y transferir todo el aprendizaje que pueden
ocurrir dentro de un amplia ecología de aprendizaje conectado, estamos
limitándola, al desalentar el aprendizaje participativo, por lo que las
habilidades críticas son poco atractivas o inaccesibles, al aislar o ignorar
los esfuerzos de calidad y las interacciones\cite{Santamaria}.

\section{Innovación tecnológica}
Se plantea utilizar la tecnología provista por las librerías del framework
Zend, para desarrollar en el lenguaje de programación PHP, de modo que la
herramienta pueda ser fácilmente instalada en el común de los servidores de
Internet.

\section{Alcance}
El desarrollo de este sistema considera toda la interacción entre distintas
instancias del sistema, es decir, en lo que respecta a: autenticación,
consumo y provisión de recursos, y control de privilegios. Es necesario
mencionar también que escapan de las funciones de este sistema la interacción
entre el sistema desarrollado y otras redes sociales, sea para provisión o
consumo de recursos.

Otra restricción impuesta será el registro cerrado para usuarios, esta será
exclusivamente por medio de invitaciones, todo esto para crear una red social
de conexiones lo menos dispersas posibles.

