\chapter{Introducción}

Con el auge de los últimos años con respecto a la red social Facebook\cite{Jeria}, se ha notado un gran cambio en la mentalidad de las personas con respecto a su entorno, compartiendo recursos e intercambiando ideas, se han abierto grandes posibilidades para un salto en las viejas concepciones respecto a lo que concierne a las formas de aprendizaje y la gestión del conocimiento.

Aunque los cambios han sido positivos, aún pueden concebirse nuevas e innovadoras maneras para obtener una gran retroalimentación entre los estudiantes, con una forma más de asistir a la educación en las aulas.

En este documento se detalla todo el proceso de construcción de una red social orientada a tópicos netamente académicos, intentando de alguna manera reducir los métodos estrictamente formales en la relación entre el educador y sus alumnos. Y de esta forma obtener una mayor integración entre estudiantes, docentes, fomentando de esa forma la interacción, comunicación y colaboración entre las partes.

En este capitulo se definen los problemas, los objetivos fundamentales, y los factores que despertaron el interés por resolverlos; posteriormente se da a conocer algunos detalles claves con respecto a la teoría del aprendizaje social, que es una de las justificaciones teóricas base sobre las que se basa toda actividad realizada.

\section{Antecedentes}
Con la creciente accesibilidad de las personas al uso de Internet, es bastante claro que el rol ha cambiado, se ha pasado de un conjunto amplio de simples consumidores de recursos, a ser participes en tareas de creación, publicación, categorización, y valoración de los recursos, es decir “Pasar de ser consumidores de información en Internet a ser productores de contenidos, información y conocimiento”\cite{Rodriguez}.

Todo esto ha abierto un nuevo camino hacia nuevas formas de interrelación social, que ofrecen una inmejorable oportunidad en el campo de lo educativo, colaborando en el apoyo y mejora de los métodos de aprendizaje. Aprovechando oportunidades como el despertar de la Web 2.0, que es una “Revolución social más que tecnológica, que da un énfasis especial al intercambio abierto del conocimiento”\cite{Rodriguez}. Redes sociales como Hi5, Facebook, MySpace, Orkut, LinkedIn entre otras, permiten a sus usuarios almacenar, organizar y compartir recursos como fotos, videos, etc. Además de crear comunidades por entorno a intereses comunes de propósito general.

También existen otras posibilidades, que son mas orientadas a asistir al aprendizaje, como ser: Moodle o Elgg; grandes sistemas que cuentan con el apoyo de muchas instituciones educativas y desarrolladores, “que permiten al docente contextualizar al aula, la utilización de las diferentes herramientas tecnológicas que tendrá a su disposición, para atender las  necesidades específicas de aprendizaje, que previamente haya identificado en su labor docente”\cite{Gonzalez}.

\section{Definición del problema}
Se ha observado que los docentes se ven sobrecargados de actividades, que en parte podrían ser simplificadas, ya sea en manejar toda la logística de un espacio virtual para su materia, o en la misma atención que debe brindar a los estudiantes.

“El tutor debe atender a un elevado número de alumnos, ante la imposibilidad de atender este trabajo se recurre a dejar de lado a aquellos alumnos que no insisten, se utilizan mensajes genéricos o fragmentos de textos copiados y pegados sin excesivo cuidado, se leen los mensajes de los alumnos de modo rápido, ignorando aspectos o matices importantes”\cite{Bartolome}.

Además de notar que los estudiantes al ver el modelo actual que deben seguir en sus estudios superiores, van perdiendo progresivamente el interés por compartir sus ideas y experiencias; conocimiento que podría servir a otros estudiantes en la construcción de sus propios criterios.

Pero en los estudiantes que ya poseen una solida rutina de participación la dificultad viene sumida en la amplia variedad de sitios orientados a la provisión de recursos, despertando una necesidad de centralizar todos estos recursos en un solo lugar.

Por lo mencionado se define el problema como:

\emph{“La escasa interacción académica entre los estudiantes conduce al uso de métodos deficientes de adquisición del conocimiento.”}

\section{Objetivos}
Para apalear este problema, se han considerado varias actividades, entre ellas muchas son informáticas y otras son mas de cambiar la cultura de las personal. En este capítulo se describen las cuestiones informáticas, y se deja para los capítulos posteriores las actividades de análisis y reflexión del contexto sobre el que se pretende trabajar.

\subsection{Objetivo General}
Promover el intercambio de información entre los estudiantes, mediante el uso de una red social para mejorar los métodos de adquisición del conocimiento.

\subsection{Objetivos Específicos}
\begin{itemize}
\item Agilizar la creación de espacios virtuales para incrementar la cantidad y variabilidad de estos.
\item Facilitar el intercambio de recursos entre los estudiantes para acelerar la adquisición de experiencia.
\item Facilitar el intercambio de recursos entre distintas instancias del sistema para mejorar la disponibilidad de recursos.
\item Mejorar los canales de comunicación entre estudiantes y docentes para facilitar la retroalimentación.
\item Planear estrategias que fomenten la participación para mantener activo el sistema.
\end{itemize}

\section{Justificación}
La construcción de una red social por definición está inmersa en ese mundo de vida propia, que es Internet; por tanto se nutre de todo lo que ella puede proveer, y todo lo que en ella se pueda construir.

Se intentá también posibilitar el gran ahorro de tiempo, tanto para los estudiantes, que podrán reutilizar contenidos de otras personas, además de tenerlos a disposición en cualquier momento; como para los docentes, que se verán apoyados en su misión de enseñanza por nuevos canales de comunicación, facilitando así todo el proceso de enseñanza-aprendizaje.

En el aspecto social, promueve la comunicación y fomenta la comunión entre personas con distintos grados de conocimiento, haciendo que unos puedan conocer y decidir que caminos pueden seguir, y a otros mostrando las ventajas y/o desventajas que pueden encontrar en el camino a sus objetivos.

\section{Teoría del aprendizaje social}
Inicialmente se partirá de esta teoría, sus fundamentos, definiciones y condiciones, aunque este sección parezca irrelevante, es importante mencionar que es el molde sobre el que se plantean las estrategias del presente trabajo.

La \emph{Teoría del aprendizaje social} es un término utilizado en psicología, educación y comunicación, plantea que parte de la adquisición de conocimiento de un individuo puede estar directamente relacionado con la observación de los demás en el contexto de las interacciones sociales, las experiencias y los medios de comunicación influyentes en el exterior.\footnote{Definición extraída de: http://en.wikipedia.org/wiki/Social\_learning\_theory}

Se propone que el aprendizaje social se produce a través de cuatro etapas principales:

\begin{enumerate}
\item Contacto cercano.
\item Imitación de los superiores.
\item Comprensión de los conceptos.
\item Comportamiento del modelo a seguir.
\end{enumerate}

Albert Bandura, concluye que el ambiente causa el comportamiento, pero que el comportamiento causa el ambiente también, esto lo definió con el nombre de \emph{determinismo reciproco}. El mundo y el comportamiento de una persona se causan mutuamente; a partir de esto empezó a considerar a la personalidad como una interacción entre tres cosas:

\begin{enumerate}
\item El ambiente.
\item El comportamiento.
\item Los procesos psicológicos de la persona.
\end{enumerate}

En definitiva el comportamiento depende del ambiente así como de los factores personales como: motivación, atención, retención y reproducción.\footnote{Extraido de http://socialpsychology43.lacoctelera.net/post/2008/07/21/aprendizaje-social-teorias-albert-bandura}

Aclarados los factores a tomar en cuenta, podríamos definir este proyecto como:

\emph{Construir un sistema de refuerzo a la educación, ademas de brindarle las herramientas necesarias para que sea usado, útil, y provechoso para todos sus usuarios, apoyado por funciones que demostraron alguna efectividad en sistemas ampliamente utilizados, que comparten patrones comunes con el nuestro; y la adaptación de otras funciones, cuya utilidad estará justificada ya sea por alguna teoría sociológica, o por el aprovechamiento de algunas oportunidades en las áreas sobre los que experimenta las ciencias de la educación.}
