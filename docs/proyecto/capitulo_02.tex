\chapter{Aprendizaje}

En este capitulo se desglosará, y analizará los conceptos utilizados para la
fundamentación del proyecto. Primero se tratará el concepto de
\emph{aprendizaje}, veremos los avances que se realizaron a lo largo de los
años, además de las tendencias y las diversas corrientes de pensamiento que han
aportado a una mejor comprensión acerca de como los seres humanos aprendemos.

A partir de ahí, describiremos las diversas teorías de aprendizaje, y los nuevos
paradigmas encaminados a abordar como la tecnología esta cambiando las formas
clásicas de aprendizaje.

Para terminar mencionaremos la cultura de adhesión, describiremos las categorías
de usuarios presentes en cualquier medio social, y plantearemos una estrategia
integral y adecuada de solución.

\section{Definición}

El aprendizaje\footnote{Definición extraída de
http://es.wikipedia.org/wiki/Aprendizaje} es el proceso a través del cual se
adquieren o modifican habilidades, destrezas, conocimientos, conductas o valores
como resultado del estudio, la experiencia, la instrucción, el razonamiento y la
observación.

El aprendizaje humano resulta de la interacción de la persona con el medio
ambiente. Es el resultado de la experiencia, del contacto del hombre con su
entorno. Este proceso, inicialmente es natural, nace en el entorno familiar y
social; luego, simultáneamente, se hace deliberado (previamente planificado).
La evidencia de un nuevo aprendizaje se manifiesta cuando la persona expresa
una respuesta adecuada interna o externamente\cite{Rojas}.

Basados en estas definiciones, se considerará al aprendizaje como un proceso
natural, que puede ser reforzado con técnicas especificas para el dominio del 
conocimiento requerido. Esta claro que el objetivo final del proyecto es mejorar
las técnicas de adquisición de conocimiento por parte de los estudiantes, para
esto es necesario primero analizar las diferentes corrientes desarrolladas en el
ámbito de la teoría del aprendizaje.

\section{Teorías del aprendizaje}

Las teorías de aprendizaje tratan los procesos de adquisición de conocimiento,
en el último siglo estas se han hecho cada vez mas importantes, desarrollándose
a partir de los descubrimientos realizados en los campos de la psicología, la
pedagogía, y la misma informática.

De las diversas corrientes postuladas en los últimos tiempos, describiremos
aquellas que se han hecho fundamentales para la comprensión del concepto de
aprendizaje.

\subsection{Conductismo}

Se conoce como conductismo a la corriente que dentro de la psicología fue
desarrollada primeramente por el psicólogo John B. Watson hacia finales del
siglo XIX y que consiste en el empleo de procedimientos estrictamente
experimentales para estudiar el comportamiento humano observable, es decir,
lisa y llanamente la conducta que despliega una persona y lo hará entendiendo
al entorno de esta como un conjunto de estímulos-respuesta\cite{ABC}.

El conductismo surgió como oposición directa al énfasis que había puesto el
psicoanálisis en los impulsos ocultos e inconscientes. El problema era que tales
impulsos no podían estudiarse y cuantificarse, lo que implicaba que la
psicología parecía no ser científica. A comienzos del siglo XX, John Watson
(1878-1958) expuso que, para que la psicología fuera considerada una ciencia,
los psicólogos debían examinar solo lo que pudieran ver y medir: la conducta y
no los pensamientos y los impulsos ocultos\footnote{Definición extraída de
https://es.wikipedia.org/wiki/Conductismo}.

El conductismo introducirá el concepto de \emph{repertorios básicos de
conducta}, como principal herramienta para explicar la conducta humana. Para
esta corriente, el proceso de aprendizaje que tiene lugar a lo largo de la
historia individual es acumulativo y jerárquico, esto quiere decir que las
conductas aprendidas tienden a acumularse con el paso del tiempo y se organizan
de modo que algunas tendrán más preeminencia sobre otras.

Los repertorios básicos son la base para la adquisición de otras conductas. Son
repertorios básicos: la atención, la imitación, y el seguimiento de
instrucciones. Estos repertorios básicos son el requisito o el repertorio de
entrada para la aplicación de cualquier otro programa. Es obvio que el sujeto
que carece de los repertorios básicos no posee tampoco los demás. También es
obvio que un sujeto que posee repertorios básicos, sociales y verbales tiene un
grado de adaptación muy elevado, y que el que carece de los básicos tiene una
mayor discapacidad\cite{Glez}.

\subsection{Cognitivismo}

Las teorías cognitivas se focalizan en el estudio de los procesos internos que
conducen al aprendizaje. Se interesa por los fenómenos y procesos internos que
ocurren en el individuo cuando aprende, como ingresa la información a aprender,
como se transforma en el individuo, considera al aprendizaje como un proceso en
el cual cambian las estructuras cognoscitivas, debido a su interacción con los
factores del medio ambiente.

El cognitivismo explica los procesos cognitivos en términos de procesamiento de
la información y considera que la mente, o al menos la parte cognitiva de ésta,
es susceptible de entenderse como un gran ordenador. Desde esta perspectiva, ha
sido frecuente considerar los procesos mentales como una serie de manipulaciones
de símbolos, con una estructura sintáctica y semántica, de acuerdo con reglas
computacionales.

Aunque la ciencia cognitiva es hoy por hoy un saber interdisciplinar, lo cierto
es que cuando se habla de cognitivismo se está haciendo frecuentemente
referencia al ámbito de la psicología cognitiva, que es el campo en el que
converge el fruto de las investigaciones del resto de las ciencias que se llaman
“cognitivas”. Por ello, un tratamiento más extenso del significado de este
concepto puede encontrarse en las voces psicología cognitiva e inteligencia
artificial\footnote{Definiciones extraídas de:
http://www.lahistoriaconmapas.com/historia/historia2\
/definicion-de-cognitivismo/}.

El aprendizaje bajo esta concepción, no se limita a una conducta observable; es
conocimiento, significativo, sentimiento, creatividad, pensamientos. Los
educadores y psicólogos que estudian el aprendizaje humano están interesados en
explicar como éste tiene lugar y como se recupera la información almacenada en
la memoria\cite{Rojas}.

\subsection{Constructivismo}

Hasta ahora, los dos enfoque anteriores, tienden a presentar el aprendizaje como
un ente objetivo y real. Es decir, una vez procesada la información, podemos
verificar el aprendizaje a partir de los resultados externos.

No obstante, algunos psicólogos cognoscitivos plantean que la persona construye
significado a partir de sus propias experiencias. Se trata de una postura que
intenta explicar cómo el ser humano conoce y cómo modifica lo
conocido\cite{Rojas}.

El constructivismo está basado en los postulados de Jean Piaget. Este psicólogo
señaló que el desarrollo de las habilidades de la inteligencia es impulsado por
la propia persona mediante sus interacciones con el medio.

Además de este citado autor también hay que subrayar el relevante papel que
ejercieron otros dentro de esta rama del constructivismo tales como Lev
Vygotsky. En su caso la principal idea que emana de sus teorías y 
planteamientos es que el ser humano y en concreto su desarrollo sólo puede ser
explicado desde el punto de vista de la interacción social\footnote{Definición
extraída de: http://definicion.de/constructivismo/}.


\subsection{Conectivismo}

El conectivismo fue presentado como una teoría del aprendizaje basado en la
premisa de que el conocimiento existe en el mundo en lugar de encontrarse en la
cabeza de un individuo. El Conectivismo propone una perspectiva similar a la
teoría de la actividad de Vygotsky\footnote{Wikipedia: La Teoría de la actividad
es una meta-teoría, paradigma, o marco de estudio no psicológico, con raíces
dadas por la psicología histórica-cultural del psicólogo soviético Lev
Vygotsky.}, ya que se refiere al conocimiento que existe dentro de los sistemas
que se accede a través de las personas que participan en las actividades.

El artículo “Conectivismo: una teoría del aprendizaje para la era digital”,
propuesto por George Siemens define que el conectivismo es la integración de
principios explorados por las teorías de caos, redes, complejidad y auto-
organización. El aprendizaje es un proceso que ocurre al interior de ambientes
difusos de elementos centrales cambiantes - que no están por completo bajo
control del individuo. El aprendizaje (definido como conocimiento aplicable)
puede residir fuera de nosotros (al interior de una organización o una base de
datos), está enfocado en conectar conjuntos de información especializada, y las
conexiones que nos permiten aprender más tienen mayor importancia que nuestro
estado actual de conocimiento.

El conectivismo es orientado por la comprensión que las decisiones están basadas
en principios que cambian rápidamente. Continuamente se está adquiriendo nueva
información. La habilidad de realizar distinciones entre la información
importante y no importante resulta vital. También es crítica la habilidad de
reconocer cuándo una nueva información altera un entorno basado en las
decisiones tomadas anteriormente.

\subsubsection{Principios del conectivismo}

\begin{itemize}
\item El aprendizaje y el conocimiento dependen de la diversidad de opiniones.
\item El aprendizaje es un proceso de conectar nodos o fuentes de información
especializados.
\item El aprendizaje puede residir en dispositivos no humanos.
\item La capacidad de saber más es más crítica que aquello que se sabe en un
momento dado.
\item La alimentación y mantenimiento de las conexiones es necesaria para
facilitar el aprendizaje continuo.
\item La habilidad de ver conexiones entre áreas, ideas y conceptos es una
habilidad clave.
\item La actualización (conocimiento preciso y actual) es la intención de todas
las actividades conectivistas de aprendizaje.
\item La toma de decisiones es, en sí misma, un proceso de aprendizaje. El acto
de escoger qué aprender y el significado de la información que se recibe, es
visto a través del lente de una realidad cambiante. Una decisión correcta hoy,
puede estar equivocada mañana debido a alteraciones en el entorno informativo
que afecta la decisión. 
\end{itemize}

\section{Teoría del aprendizaje social}

La \emph{teoría del aprendizaje social} es un término utilizado en psicología,
educación y comunicación, plantea que parte de la adquisición de conocimiento
de un individuo puede estar directamente relacionado con la observación de los
demás en el contexto de las interacciones sociales, las experiencias y los
medios de comunicación influyentes en el exterior\footnote{Definición extraída
de: http://en.wikipedia.org/wiki/Social\_learning\_theory}.

La teoría del aprendizaje social se deriva de la obra de Albert Bandura, que
propuso que el aprendizaje social se produce a través de cuatro etapas
principales de la imitación:

\begin{enumerate}
\item Contacto cercano.
\item Imitación de los superiores.
\item Comprensión de los conceptos.
\item Comportamiento del modelo a seguir.
\end{enumerate}

Albert Bandura, concluye que el ambiente causa el comportamiento, pero que el
comportamiento causa el ambiente también, esto lo definió con el nombre de
\emph{determinismo reciproco}. El mundo y el comportamiento de una persona se
causan mutuamente; a partir de esto empezó a considerar a la personalidad como
una interacción entre tres cosas:

\begin{enumerate}
\item El ambiente.
\item El comportamiento.
\item Los procesos psicológicos de la persona.
\end{enumerate}

En definitiva el comportamiento depende del ambiente así como de los factores
personales como: motivación, atención, retención, y
reproducción\footnote{Extraído de http://socialpsychology43.lacoctelera.net\
/post/2008/07/21/aprendizaje-social-teorias-albert-bandura}.

Julian Rotter sugiere que el efecto de la conducta tiene un impacto en la
motivación de la gente a participar en ese comportamiento específico. La gente
quiere evitar consecuencias negativas, mientras que desean resultados positivos
o efectos. Si uno espera un resultado positivo de una conducta, o cree que hay
una alta probabilidad de un resultado positivo, entonces serán más propensos a
involucrarse en este comportamiento.

El comportamiento se ve reforzado, con resultados positivos, lo que lleva a una
persona a repetir el comportamiento. Esta teoría del aprendizaje social sugiere
que el comportamiento está influenciado por estos factores ambientales o
estímulos, y no solo los factores psicológicos.

\section{Ludificación}

Otro concepto importante para los objetivos del proyecto, es el concepto de
ludificación, que se refiere a la aplicación de mecánicas de juego a entornos
no lúdicos\footnote{Definición extraída de: http://www.fundeu.es/recomendacion\
/ludificacion-mejor-que-gamificacion-como-traduccion-de-gamification-1390/}.

A partir de las ideas establecidas por Abraham Maslow en su obra: \emph{Una
teoría sobre la motivación humana}, se definen un conjunto de necesidades de
alto nivel, entre estas las necesidades sociales (relacionamientos,
participación, y aceptación), y las necesidades de estima\footnote{Puede verse
mas a fondo el concepto en:
http://es.wikipedia.org/wiki/Pirámide\_de\_Maslow}; se destaca el sentido de
pertenencia como la forma de creación de competencia entre diferentes
individuos en una colectividad\cite{Venegas}.

El aprendizaje no es sólo “el tiempo de seguridad” en las escuelas y encerrados,
sino que se extiende a través de múltiples contextos, experiencias e
interacciones. Ya no es sólo un concepto aislado o individual, sino que es
incluyente, social, informal, participativo, creativo y para toda la vida.

Por lo tanto, crear reconocimientos personales (insignias o badges) puede
desempeñar un papel crucial en la ecología de aprendizaje conectado al actuar
como un puente entre el contexto y estos canales alternativos de aprendizaje,
las habilidades y los tipos de aprendizaje que pueden ser más viables,
portátiles e impactantes. Las insignias puede ser otorgadas por un conjunto
potencialmente infinito de las propias capacidades individuales, 
independientemente de dónde se desarrolla cada habilidad, y la colección de
insignias pueden servir como una hoja de vida virtual de las competencias y
habilidades de las partes interesadas clave, como sus compañeros, escuelas o
posibles empleadores\cite{Santamaria}.

En concreto, podrían tener repercusión y apoyo en:

\begin{itemize}
\item Captura de la ruta de aprendizaje.
\item Señalización de un logro.
\item Motivación.
\item Apoyo a la innovación y flexibilidad.
\item La identidad y la construcción de la reputación.
\item Construcción de una comunidad.
\end{itemize}

\subsection{Técnicas utilizadas}

Existen muchas técnicas utilizadas por redes sociales, entre estas están:

\begin{itemize}
\item Plasmar los niveles de logro.
\item Tener la plataforma tablas de clasificación.
\item Disponer de una barra de progreso o de otros elementos visuales para
      indicar qué tan cerca está de completar una tarea de una empresa o de
      superar a tal persona.
\item Algunos disponen de una moneda virtual.
\item Con sistemas para la concesión, regalos y posible intercambio de puntos
      entre usuarios.
\item Posibilidad de intercambiar objetos y utensilios entre los usuarios.
\item Posibilidad también de solucionar los problemas entre ellos con un buena
      retroalimentación de lo ocurrido.
\item Incorporación de pequeños juegos ocasionales en otras actividades.
\end{itemize}

\section{Cultura de la adhesión}

De los muchos inconvenientes acerca de las redes sociales, el mas critico debe
ser definitivamente, el concepto de “cultura de adhesión”, que consiste en la
sustitución de los espacios distribuidos de deliberación, por espacios
centralizados en los que las personas sólo pueden mostrar su adhesión a una
figura, causa o convocatoria\cite{LasIndias}.

\section{Tipos de usuarios}

En la creación de espacios de participación, pueden encontrarse varios tipos de
categorías\cite{Santamaria2}:

\begin{itemize}
\item \textbf{Creadores:} Aquellos que crean contenidos y medios en la red.
\item \textbf{Críticos}: Aquellos que comentan, valoran, y contribuyen en la
      edición de contenido existente en una red.
\item \textbf{Colectores}: Aquellos que catalogan y organizan las contenidos de
      la red.
\item \textbf{Sociables}: Aquellos cuya su principal objetivo es la
      sociabilización en las redes sociales.
\item \textbf{Espectadores}: Principalmente se dedica a consumir los recursos y
      medios disponibles en la red.
\item \textbf{Inactivos}: Aquella que no participa de ninguna de las maneras
      antes mencionadas.
\end{itemize}

Es de vital importancia considerar tales categorías, para aprovechar las
fortalezas que ofrece cada tipo de usuario.

\section{Estrategias a tomar en cuenta}

Ahora que se han revisado las teorías de aprendizaje, se plantea un conjunto de
medidas a tomar en cuenta el óptimo aprovechamiento de los objetivos del
proyecto. Aprovechando las metodologías que estas corrientes en el aprendizaje han experimentado, pasamos a describir las estrategias ha plantearse:

\begin{itemize}
\item Construcción de indicadores de medición cuantificables, de
modo que el conjunto de observaciones, experimentaciones. y habilidades básicas 
necesarias pueda ser medido con algún grado de precisión, aseverando el enfoque
conductista del aprendizaje.
\item Para enfocar el aprendizaje como un proceso, siguiendo el enfoque
cognitivista, se maximizaran los métodos de interacción en las relaciones
docente-estudiante, y estudiante-estudiante.
\item Como un forma para incrementar los métodos de adquisición de experiencia,
se utilizará fomentará el uso de los conceptos inherentes al concepto de 
\emph{ludificación}, para lo cual se plantea la construcción de una plataforma,
en la cual puedan incluirse aplicaciones que sigan este concepto.
\item Se plantea la construcción de un sistema b-learning\footnote{Blended
learning: El aprendizaje semi-presencial es el aprendizaje facilitado a través
de la combinación eficiente de diferentes métodos de impartición, modelos de
enseñanza y estilos de aprendizaje, y basado en una comunicación transparente
de todas las áreas implicadas en el curso.}, que no sustituya la educación
presencial, sino mas bien asista a esta, además que facilite la comunicación
entre los involucrados a través de áreas de interés comunes.
\item Maximizar la interacción entre usuarios del sistema, para fomentar la
tendencia hacia la imitación de comportamientos, y la creación automática de
modelos a seguir.
\item Reforzamiento de conductas apropiadas a partir de la construcción de pseudo-jerarquías en el sistema, es decir, un sistema de reputación.
\end{itemize}
