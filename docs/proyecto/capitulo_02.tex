\chapter{El modelo base}

Como parte del análisis requerido para la desarrollo del sistema, en este capítulo describiremos las características y
limitaciones del sistema Yeah!, antecesor del cuál se basaron las primeras ideas e intenciones del presente trabajo.

\section{Reseña histórica}
El proyecto Yeah! se inicia en febrero del 2009, con la intención de desarrollar un sistema web de administración de
cursos y notas, pasadas varias situaciones, se llega a una versión estable en junio del mismo año, dándose por terminado
el desarrollo e intenciones iniciales (versión 0.1).

Transcurridos varios años, se reactiva el proyecto en mayo del 2010, con la intención inicial de mejorar las funciones
inicialmente propuestas, además de complementar con ideas que nacieron en el transcurso del tiempo. Llegando a un nuevo
producto estable en noviembre del 2010, y cambiando el nombre a proyecto Yachay  (versión 0.2).

En este año 2011, se fueron realizando varias adaptaciones, necesarias para la utilización en un contexto real
(instalación de la primera instancia en colaboración con el centro MEMI), que ha su vez fueron develando muchas
necesidades de reestructuración; el resultado final fue la versión 0.3, siendo esta la que esta siendo evaluada y
utilizada.

\section{Primeras intenciones}
El objetivo del proyecto Yeah!, estaba definido como:

\emph{Desarrollar una plataforma social basada en tecnologías web 2.0, que posea fines educativos, de manera que mejore
el proceso enseñanza-aprendizaje y la interacción entre docentes y estudiantes.}

\section{Funciones que moldean el sistema}
\begin{description}
\item [Gestión de usuarios] Componente encargado de la gestión de usuarios y roles de usuarios.
\item [Gestión de espacios virtuales] Componente encargado de la gestión de espacios de materias, espacios de discusión
y espacios privados.
\item [Gestión de equipos de trabajo] Componente encargado de la gestión de equipos de trabajo entre los usuarios del
sistema.
\item [Sub-sistema de administración del sistema] Componente encargado de facilitar la administración del sistema.
\item [Sub-sistema de administración de módulos] Componente encargado de la gestión de los módulos del sistema.
\end{itemize}

\section{Estructura}
La estructura del sistema siempre estuvo regida por las recomendaciones de la librería Zend en su versión 1.8.x, desde
ese entonces hasta ahora (versión 1.11.x), varios métodos y aspectos han cambiado, con la salida de nuevas versiones, lo
que hizó que la estructura requiera cambios importantes\footnote{Todos los cambios requeridos, estan basados en la
documentación oficial alojada en http://framework.zend.com/manual/en/learning.quickstart.create-project.html}.

En la gráfica~\ref{estructura_base} se muestra la estructura antigua, mucho mas simple que la nueva que puede verse en la
gráfica~\ref{estructura_actual}, la estructura ahora utilizada frecuentemente por la librería Zend, que está por demás
demostrado posee una mejor adecuación a los nuevos requerimientos en el desarrollo web, ademas de ofrecer mayor control
y seguridad que la versiones mas antiguas\cite{Gilmore}.

\begin{figure}
\small
\dirtree{%
.1 /.
.2 libs \ldots{} \begin{minipage}[t]{8cm} Carpeta contenedora de las clases php.\end{minipage}.
.3 File.
.3 Xcel.
.3 Yeah \ldots{} \begin{minipage}[t]{8cm} Librerías propias del sistema.\end{minipage}.
.4 Helpers.
.4 Model.
.5 Row.
.6 Validation.php \ldots{} \begin{minipage}[t]{8cm} Validador automático de modelos.\end{minipage}.
.5 Table.php \ldots{} \begin{minipage}[t]{8cm} Superclase base de todos los adaptadores de base de datos.\end{minipage}.
.4 Regions.
.5 \ldots{}.
.4 Settings.
.5 Config.php \ldots{} \begin{minipage}[t]{8cm} Archivo de configuración del sistema.\end{minipage}.
.5 Database.php \ldots{} \begin{minipage}[t]{8cm} Archivo de configuración de la base de datos.\end{minipage}.
.5 \ldots{}.
.4 Validators.
.5 \ldots{}.
.4 Acl.php.
.4 Action.php \ldots{} \begin{minipage}[t]{8cm} Superclase base de todos los controladores.\end{minipage}.
.4 Adapter.php.
.4 Bootstrap.php \ldots{} \begin{minipage}[t]{8cm} Inicializador del motor de Zend Framework.\end{minipage}.
.4 Init.php.
.4 Loader.php \ldots{} \begin{minipage}[t]{8cm} Cargador automático de modelos.\end{minipage}.
.4 Utils.php.
.3 Zend.
.4 \ldots{}.
.2 media.
.3 upload.
.3 \ldots{}.
.2 modules \ldots{} \begin{minipage}[t]{8cm} Módulos instalados.\end{minipage}.
.3 \ldots{}.
.2 templates \ldots{} \begin{minipage}[t]{8cm} Plantillas instaladas.\end{minipage}.
.3 \ldots{}.
.2 .htaccess.
.2 index.php.
.2 yeah.log \ldots{} \begin{minipage}[t]{8cm} Archivo de log de rutas visitadas.\end{minipage}.
}
\caption{Estructura de ficheros del modelo base.}
\label{estructura_base}
\end{figure}

\begin{figure}
\small
\dirtree{%
.1 /.
.2 application \ldots{} \begin{minipage}[t]{8cm} Librerías especificas del sistema.\end{minipage}.
.3 configs.
.4 application.ini \ldots{} \begin{minipage}[t]{8cm} Archivo de configuración del sistema.\end{minipage}.
.3 modules \ldots{} \begin{minipage}[t]{8cm} Módulos habilitados.\end{minipage}.
.4 \ldots{}.
.3 Bootstrap.php.
.2 data \ldots{} \begin{minipage}[t]{8cm} Carpetas para información especifica de la instancia instalada.\end{minipage}.
.3 dtd.
.3 sql \ldots{} \begin{minipage}[t]{8cm} Plantillas SQL de instalación de los módulos.\end{minipage}.
.3 src \ldots{} \begin{minipage}[t]{8cm} Paquetes disponibles para uso.\end{minipage}.
.3 \ldots{}.
.2 library.
.3 Yachay.
.4 Controller.
.5 Plugins.
.6 Formatter.php.
.5 Templates.
.6 \ldots{}.
.5 Action.php.
.4 Db.
.5 Table.
.6 Row.php.
.5 Table.php.
.4 Model.
.5 Loader.php.
.4 Console.php.
.3 Zend.
.4 \ldots{}.
.3 \ldots{}.
.2 logs.
.3 \ldots{}.
.2 public \ldots{} \begin{minipage}[t]{8cm} Carpeta para archivos estáticos.\end{minipage}.
.3 media.
.4 \ldots{}.
.3 index.php.
.2 shell \ldots{} \begin{minipage}[t]{8cm} Herramientas del sistema por consola.\end{minipage}.
.3 \_\_header.php.
.3 installer.php.
.3 \ldots{}.
}
\caption{Estructura de ficheros del modelo actual.}
\label{estructura_actual}
\end{figure}

\section{Módulos}
El proyecto tenía desarrollados varios módulos, cuyas funciones iban desde la manipulación de menús, hasta el manejo de
comunidades. Una de las máximas desventajas del modelo base fue su alto acoplamiento, que hizó a los módulos
tremendamente interdependientes, haciendo de todo el sistema algo complejo de mantener.

En el cuadro~\ref{modulos_base} se detallan los módulos que ya estaban construidos con anterioridad y la función que
desempeñan.

\begin{table}
\begin{tabular}{l|l}
Módulo & Descripción \\
\hline
areas & Administración de áreas temáticas. \\
califications & Manejo de notas en los cursos. \\
careers & Administración de carreras. \\
comments & Manejo de comentarios. \\
communities & Administración de comunidades. \\
evaluations & Manejo de las formas de evaluación en los cursos. \\
events & Manejo de recursos tipo evento. \\
feedback & Manejo de mensajes de retroalimentación. \\
files & Manejo de recursos tipo archivo. \\
friends & Administración de la red de contactos. \\
frontpage & Manejo de pagina principal. \\
gestions & Administración de periodos académicos. \\
groups & Administración de grupos de estudio. \\
groupsets & Utilitario de manejo de múltiples grupos. \\
invitations & Manejo de invitaciones de registro. \\
links & Manejo de recursos tipo enlace. \\
login & Autentificación de usuarios. \\
menus & Manejo de regiones en plantilla. \\
modules & Administración de módulos. \\
notes & Manejo de recursos tipo nota. \\
pages & Administración de paginas. \\
paginator & Utilidad de paginación de recursos. \\
photos & Manejo de recursos tipo fotografía. \\
privileges & Administración de privilegios del sistema. \\
profile & Utilidad de configuración de información de usuario. \\
ratings & Manejo de rankings sobre recursos. \\
regions & Manejo genérico de regiones. \\
resources & Administración genérica de recursos. \\
roles & Administración de grupos de privilegios. \\
settings & Utilidad de configuración de usuario. \\
subjects & Administración de materias. \\
tags & Manejo de etiquetas sobre recursos. \\
teams & Administración de equipos de trabajo. \\
templates & Manejo de plantillas. \\
toolbar & Manejo de la region toolbar en plantilla. \\
users & Administración de usuarios. \\
valorations & Manejo de valoraciones en los usuarios. \\
videos & Manejo de recursos tipo video. \\
widgets & Manejo de widgets en plantilla. \\
\end{tabular}
\caption{Módulos existentes en el modelo base.}
\label{modulos_base}
\end{table}

\section{Críticas al modelo establecido}
Aunque se han discutido ya los problemas concernientes a la ya casi caduca estructuración del sistema, y el problema del
alto acoplamiento de los módulos; existían otros problemas menores que requerían plantearse.

\subsection{Sistema de plantillas}
Se tenían creadas dos plantillas, una de ellas, construida en html, y la otra, en xhtml con hojas de estilo (css2). Si
bien era posible crear plantillas que aprovechen toda la potencia de html, css, y javascript, resultaba este, un proceso
altamente costoso, debido al uso de paginas muy personalizadas, lo que conducía al casi imposible uso de la tecnología
ajax, haciendo de toda la arquitectura de plantillas implementada, cuestión de imperioso reemplazo.

\subsection{Instalación}
El sistema no poseía un método cómodo o claramente establecido para la instalación, lo que complicaba el uso de esta
herramienta a los administradores.

\subsection{Conectitividad}
Una de las características más llamativas para el sistema, es la conexión con otros sistemas, su bien las redes sociales
populares tienden hacia la centralización, el objetivo prioritario en este sistema será la descentralización de la
información, una cuestión muy análoga a lo referente a las tecnologías p2p.

\subsection{Modelos muy teoricos}
El problema final encontrado era la falta de roce del sistema con la vida real, lo que resultó un modelo de soluciones a
un campo de visión muy estrecho, siendo prioridad para este nuevo recorrido escuchar a los usuarios en todo su espectro
y categorización.
