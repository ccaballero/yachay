\chapter{El producto desarrollado}

En este capítulo describiremos las características y limitaciones del sistema.

\section{Arquitectura del sistema}

\section{Plataforma}
\subsection{El gestor de paquetes \emph{packages}}
\subsection{El manejo de privilegios \emph{privileges}}
\subsection{El manejo de rutas y navegación \emph{routes}}
\subsection{El sistema de plantillas \emph{templates}}

\section{Conectitividad}
\subsection{La inteligencia colectiva \emph{axon}}

\section{Personas}
\subsection{El espacio personal \emph{users}}
\subsection{El control de las funciones \emph{roles}}

\section{Red de contactos}
\subsection{Las redes sociales \emph{contacts}}
\subsection{Los círculos \emph{sets}}
\subsection{La propagación \emph{invitations}}

\section{Espacios virtuales}
\subsection{El espacio genérico \emph{spaces}}
\subsection{Los espacios formales \emph{gestions}, \emph{careers},
\emph{areas}}
\subsection{El espacio informal \emph{communities}}
\subsection{Los espacios jerárquicos \emph{subjects}, \emph{groups},
\emph{teams}}
\subsection{Las utilidades sobre grupos \emph{groupsets}}

\section{B-learning}
\subsection{Los sistemas de evaluación \emph{evaluations}}
\subsection{Las calificaciones \emph{califications}}

\section{Recursos}
\subsection{Los componentes genéricos \emph{resources}}
\subsection{El recurso mas básico \emph{notes}}
\subsection{Los archivos en general \emph{files}}
\subsection{Los archivos especiales \emph{photos}, \emph{videos}}
\subsection{Los recursos espacio-temporales \emph{events}}
\subsection{La reenderizacion personalizada \emph{links}}
\subsection{Las sugerencias \emph{feedback}}

\section{Web 2.0}
\subsection{Los comentarios \emph{comments}}
\subsection{La calidad del recurso \emph{ratings}}
\subsection{Las nuevas interpretaciones \emph{tags}}
\subsection{Los sistemas de reputación \emph{valorations}}

\section{Sistemas de control}
\subsection{Los indicadores medibles \emph{stats}}
\subsection{El panel de control \emph{panels}}
