\chapter{Desarrollo del proyecto}

En este capítulo, trataremos los asuntos concernientes a la construcción de las
funciones del sistema, sobre las que recaerán el control de los recursos, y la
expansibilidad que pueda darse a todo el proyecto. Si bien en el anterior
capitulo el tema fundamental era el \emph{proceso de desarrollo}, el objeto
central de este capitulo es el \emph{producto de software}.

\section{Base funcional del sistema}
Una de las características deseables a tomar en cuenta para el desarrollo del
sistema, era obtener un perfecto equilibrio entre modularidad y rendimiento,
pero sin incrementar la complejidad del sistema de modo apreciable.

Para conseguir tal característica, se opto por utilizar una arquitectura basada
en capas, muy similar al diseño de los sistemas operativos, pero sin llegar a la
complejidad que estos mismos poseen. En la capa mas básica de la arquitectura
del sistema, se encuentran tres paquetes que son fundamentales para cualquier
función que el sistema quiera desempeñar. En esta sección se tratan estos tres
paquetes, además de la solución que se plantea para proveer al usuario final de
una personalización mas atractiva.

\subsection{\emph{packages}: Manejador de paquetes}
Las principales funciones de este paquete son:

\begin{itemize}
\item Instalación de paquetes en el sistema.
\item Manejo de dependencias entre paquetes del sistema.
\item Establecimiento de rutas de acceso para un paquete determinado.
\end{itemize}

\subsection{\emph{privileges}: Manejador de privilegios}
Las principales funciones de este paquete son:

\begin{itemize}
\item Registro de privilegios reservados por cada paquete.
\item Control de acceso a recursos y acciones especificas.
\end{itemize}

\subsection{\emph{routes}: Manejador de rutas de navegación}
Las principales funciones de este paquete son:

\begin{itemize}
\item Registro de rutas reservadas por cada paquete.
\end{itemize}

\subsection{\emph{templates}: Manejador de plantillas}
Las principales funciones de este paquete son:

\begin{itemize}
\item Gestión de la presentación del sistema.
\item Administración de las utilidades adicionales que presentan las paginas
del sistema.
\end{itemize}

En la figura (figura), pueden verse los relacionamientos entre estas entidades a
nivel de base de datos, el proceso de creacion de un paquete nuevo consiste de
los siguientes pasos:

\begin{itemize}
\item Creacion de un archivo SQL con la definicion de las tablas necesarias para
el nuevo paquete, estas deben estar prefijadas con el nombre del paquete.
\item Inserción un nuevo registro de paquete.
\item Inserción de los registros de privilegios para el paquete nuevo.
\item Inserción de los registros de ruta para el paquete nuevo.
\item Registro de los permisos necesarios por ruta creada.
\end{itemize}

En la figura (figura), puede verse un ejemplo, a partir de ultimo paquete creado
en el sistema (feedback); este registra cuatro diferentes tipos de privilegios;
cuatro diferentes tipos de rutas de acceso (aqui puede notarse que una ruta no
necesariamente atiende exclusivamente a la peticion GET, sino a cualquier tipo
de petición que sea necesario realizar (ya sea POST, PUT, DELETE, etc); ademas
puede verse el establecimiento de privilegios para el acceso a las rutas.

Esta diseño esta basado en la forma de control que puede verse en cualquier
sistema administrador de contenido básico (CMS).

\section{Construcción de espacios virtuales}
Uno de los puntos fundamentales en la construcción del sistema, es el control y
manejo organizado de los espacios disponibles del sistema, estos espacios
constituyen los lugares, de intercambio, producción, y discusión de los
recursos que posea el sistema.

Para proporcionar las funcionalidades requeridas, se han clasificado los
espacios segun su orden temporal; siendo estos de dos tipos:

\begin{description}
\item [Temporales] Es un espacio temporal todo aquel que depende de la gestión
en la que uno se encuentre; tanto su acceso, como su visibilidad estan
delimitadas por la gestión que este presentandose, estos espacios son lo que se
contruyeron inicialmente, ejemplos de estos son: materias, grupos, etc.
\item [Atemporales] Es un espacio atemporal aquel que no esta englobado en una
gestión determinada, su acceso y visibilidad es independiente, ejemplos de
estos son: comunidades, areas, etc.
\end{description}

En la figura (figura) pueden apreciarse los paquetes que se han construido para
el manejo de espacios virtuales, remarcando su caracteristica temporal.

\subsection{\emph{spaces}: La abstracción de todos los espacios}
Para facilitar el manejo de multiples tipos de espacios, se ha optado por
abstraer muchas de las funcionalidades comunes (ya sean de presentación, o de
funcionalidad), es asi como se ha creado este paquete que acumula la mayor parte
de la funcionalidad para los demas espacios.

Las principales funciones de este paquete son:

\begin{itemize}
\item Concentrar las funciones sobre los recursos del sistema.
\item Generación de una capa de abstracción para la creación de espacios con
condiciones de uso diferente.
\item Administración de los espacios (creacion, modificacion, presentación,
eliminación).
\item Control de permisos sobre los espacios.
\end{itemize}

\subsection{\emph{gestions}, \emph{careers}, \emph{areas}: Los espacios
formales}
Al comienzo del diseño se observo que una gran parte de los espacios vitales que
eran acordes al modelo al uso en la Universidad, eran dependientes de la
gestión, es decir, que una materia/grupo acumula recursos unicamente validos por
un determinado periodo de tiempo, es asi como se crearon estos espacios.

\begin{description}
\item [Gestion] Es el espacio que determina el inicio y final de la validez de
un recurso temporal, esta muy asociado a la idea que se tiene de gestion
academica en la Universidad.
\item [Área] Un área es un tipo de espacio temporal, se creo para agrupar otros
tipos de espacio de forma transversal, y de esta forma flexibilizar las
jerarquias que poseia el model tradicional; un ejemplo de esto se ve en las
diferentes relaciones que existen entre diferentes materias o grupos en el
modelo.
\item [Carrera] Se creo este espacio para agrupar los espacios que tienen como
caracteristica principal estar dentro de una malla curricular especifica.
\end{description}

Las principales funciones de estos paquetes son:

\begin{itemize}
\item Conformación de los espacios organizados acorde a la estructura real
observada en el contexto del sistema.
\item Agrupación de otros espacios que comparten algun grado de relacionamiento
entre si.
\end{itemize}

\subsection{\emph{subjects}, \emph{groups}, \emph{teams}: Los espacios
jerárquicos}
Estos espacios representan el eje central sobre el que se ha construido el
sistema, tambien estan modelados acorde a la estructura observada en el contexto
del sistema, estas poseen una estructura de tipo jerarquico (es decir, cada una
esta contenida dentro de otra).

\begin{description}
\item [Materia] Representa el espacio de materia y a su vez puede poseer varios
grupos en su propio contexto.
\item [Grupo] Representa el espacio de trabajo de un grupo especifico de una
materia, esta a su vez puede contener varios equipos, segun la organización y
metodos que utilize el instructor del grupo.
\item [Equipo] Representa el minimo espacio organizacional en un grupo, se
construyo para facilitar la organización de estudiantes dentro de un grupo.
\end{description}

Las principales funciones de estos paquetes son:

\begin{itemize}
\item Creación de la estructura jerárquica que es también un reflejo de la
estructura real observada en el contexto del sistema.
\item Asignación de usuarios (con algun rol especifico), a los sub-espacios
establecidos.
\end{itemize}

\subsection{\emph{communities}: El espacio informal}
Como una forma de crear transversales al modelo jerarquico presentado
anteriormente, se construyo el espacio de comunidad, que agrupa usuarios a
partir de criterios no relacionados a una malla curricular, es decir, como una
forma alternativa de crear relaciones entre usuarios, a partir de un interes
particular. Este es un espacio atemporal, es decir, esta no depende de la
gestión.

Las principales funciones de estos paquetes son:

\begin{itemize}
\item Creación de un espacio común, transversal a las estructuras del contexto
del sistema.
\item Manejo de permisos para el espacio (una comunidad puede ser abierta a
cualquier usuario; o cerrada, de forma que un usuario pueda ser aceptado o
rechazado en la comunidad).
\end{itemize}

\section{B-learning}
Para facilitar el manejo por parte del docente asignado a un grupo, se crearon
dos paquetes que administren las formas de calificación, y la presentación de
calificaciones.

\subsection{\emph{evaluations}: Los sistemas de evaluación}
En un comienzo se notó que si bien, los docentes siempre presentan las
calificaciones con un formato unico (Primer parcial, Segundo parcial, Promedio
de los parciales, Examen final y Segunda instancia), estas solo son la parte
final de un proceso aun mas complejo, por lo cual se creo un paquete que pueda
manejar criterios de evaluacion mas elaborados, segun el docente que imparta la
materia.

Las principales funciones de este paquete son:

\begin{itemize}
\item Creación y edición de criterios de evaluación.
\item Publicación y aplicación de un criterio de evaluación sobre un grupo
determinado por un docente.
\item Aplicacion de calificaciones segun un criterio de evaluacion especifico.
\end{itemize}

\subsection{\emph{califications}: Las calificaciones}
A partir del paquete que administra y proporciona diferentes tipos de evaluacion
para un grupo, se han creado las funcionalidades necesarias para establecer las
calificaciones de un grupo.

Las principales funciones de este paquete son:

\begin{itemize}
\item Edicion de las calificaciones de los alumnos de un determinado grupo,
segun un criterio de evaluacion establecido.
\item Importación automatica a partir de una archivo en formato CSV, de las
calificaciones de una grupo.
\item Exportacion a formato CSV de las calificaciones de una grupo, segun un
criterio de evaluacion establecido.
\end{itemize}

\section{Recursos}
Ya creados los espacios virtuales, se han construido los paquetes necesarios
para la publicacion e intercambio de recursos en el sistema.

\subsection{\emph{resources}: La abstracción de todos los recursos}
Al igual que en los espacios virtuales, los recursos tambien comparten un gran
conjunto de funcionalidad comun a todos ellos (ya sean de presentacion, como de
funcionalidad), es asi como se han abstraido estas funcionalides en este
paquete.

Las principales funciones de este paquete son:

\begin{itemize}
\item Concentrar las funciones de valoracion sobre los recursos.
\item Generación de una capa de abstracción para la creación de recursos con
condiciones de uso diferente.
\item Administración de los recursos (creacion, modificacion, presentación,
valoración, eliminación).
\item Control de permisos sobre los recursos.
\end{itemize}

\subsection{\emph{notes}: Los recursos mas básicos}
El tipo de recurso mas básico es la nota, que representa esencialmente texto.
Este recurso ademas se diseño para ser utilizado por otros tipo de recursos mas
enriquecidos y especializados.

\subsection{\emph{links}: El administrador de marcadores}
Este recurso fue construido para compartir recursos, a partir de enlaces sobre
la red de Internet, se penso para trabajos posterior, hacer de este tipo de
recurso una funcionalidad de reconocimiento de recurso y una reenderizacion mas
apropiada.

\subsection{\emph{files}: El administrador de archivos}
Este recurso fue construido para compartir archivos en un espacio virtual,
tambien se penso para trabajo posterior, una especializacion sobre la
reenderizacion y reconocimiento de los formatos.

\subsection{\emph{photos}, \emph{videos}: Los recursos especiales}
Estos recursos estan basados en el recurso tipo archivo, pero poseen
caracteristicas que les permiten ser reenderizados apropiadamente.

\subsection{\emph{events}: El recurso espacio-temporales}
Este recurso representa un evento sobre un espacio virtual, a partir de este
recurso, se considero para trabajo posterior, el que sea tanto un recurso, como
un espacio virtual independiente, ademas de facilitar el manejo de
notificaciones.

\subsection{\emph{feedback}: El manejador de sugerencias}
Este fue el ultimo paquete que se construyo, y fue creado exclusivamente para
que los usuarios del sistema puedan hacer sus sugerencias sobre nueva
funcionalidad que podrian implementarse sobre el sistema.

\section{Los usuarios y su red de contactos}
Una vez establecidos los conceptos de espacio virtual, y recurso, se han
construido los paquetes necesarios para la representación y manejo de usuarios
del sistema.

\subsection{\emph{users}: El espacio personal}
Representa al usuario final del sistema (docentes, estudiantes, auxiliares,
etc.), estos ademas estan provistos de las funcionalidades para espacios
virtuales.

Las principales funciones de este paquete son:

\begin{itemize}
\item Creación de un espacio virtual propio del usuario.
\item Manejo y visualización de sus valoraciones en el sistema.
\item Administración de los datos personales del usuario.
\item Manejo de los recursos creados por el usuario sobre los diferentes
espacios virtuales a los que tiene acceso.
\end{itemize}

\subsection{\emph{roles}: El controlador de las privilegios}
Un rol representa el conjunto de privilegios que posee un usuario sobre los
paquetes del sistema.
Inicialmente de considero categorizar a los usuarios de forma estatica, segun
los criterios del modelo de la Universidad, estos son:

\begin{itemize}
\item Visitante
\item Invitado
\item Estudiante
\item Auxiliar
\item Docente
\item Moderador
\item Desarrollador
\item Administrador
\end{itemize}

Pero para facilitar la adaptación a otros tipos de organización, se vio
conveniente crear roles dinamicamente, asi como poder administrar el conjunto
de privilegios que estos posean.

Las principales funciones de este paquete son:

\begin{itemize}
\item Administración de roles (creación, visualización, edición, eliminación).
\item Asignación de privilegios dinamicamente.
\item Control de acceso a los espacios, recursos, y rutas segun un rol
establecido.
\end{itemize}

\subsection{\emph{contacts}: Las redes sociales}
Para proveer las caracteristicas de una red social, se construyo un paquete que
pudiese manejar las relaciones entre diferentes tipos de usuarios.

Las principales funciones de este paquete son:

\begin{itemize}
\item Agregacion de un contacto.
\item Eliminación de un contacto.
\item Visualización de los recursos compartidos por los contactos.
\end{itemize}

\subsection{\emph{invitations}: Estrategia de propagación del sistema}
Se considero que la creación de cuentas en el sistema nunca sea abierta, con el
objetivo de hacer que los usuarios en el sistema, tengan al menos un contacto
con otro usuario, por tal motivo, se crearon las invitaciones.

Las principales funciones de este paquete son:

\begin{itemize}
\item Envio de una invitación de un usuario a un correo electronico.
\item Gestion de la caducidad de una invitacion.
\item Manejo del contacto con el usuario que atiende a la invitación.
\end{itemize}

\section{Fomento a la participación}
Como medidas para fomentar la participación de parte de los usuarios, se han
implementado algunas funcionalidade propias de la web 2.0.

\subsection{\emph{comments}: Los comentarios}
\subsection{\emph{ratings}: La calidad del recurso}
\subsection{\emph{tags}: Las nuevas interpretaciones}
\subsection{\emph{valorations}: Los sistemas de reputación}

\section{Sistemas de control}
\subsection{\emph{stats}: Los indicadores medibles}

