\chapter{Desarrollo del proyecto}

En este capítulo, trataremos los asuntos concernientes a la construcción de las
funciones del sistema, sobre las que recaerán el control de los recursos, y la
expansibilidad que pueda darse a todo el proyecto. Si bien en el anterior
capitulo el tema fundamental era el \emph{proceso de desarrollo}, el objeto
central de este capitulo es el \emph{producto de software}.

\section{Base funcional del sistema}
Una de las características deseables a tomar en cuenta para el desarrollo del
sistema, era obtener un perfecto equilibrio entre modularidad y rendimiento,
pero sin incrementar la complejidad del sistema de modo apreciable.

Para conseguir tal característica, se opto por utilizar una arquitectura basada
en capas, muy similar al diseño de los sistemas operativos, pero sin llegar a la
complejidad que estos mismos poseen. En la capa mas básica de la arquitectura
del sistema, se encuentran tres paquetes que son fundamentales para cualquier
función que el sistema quiera desempeñar. En esta sección se tratan estos tres
paquetes, además de la solución que se plantea para proveer al usuario final de
una personalización mas atractiva.

\begin{figure}
\centering
%LaTeX with PSTricks extensions
%%Creator: inkscape 0.48.5
%%Please note this file requires PSTricks extensions
\psset{xunit=.5pt,yunit=.5pt,runit=.5pt}
\begin{pspicture}(650,500)
{
\newrgbcolor{curcolor}{0 0 0}
\pscustom[linewidth=1.84937644,linecolor=curcolor]
{
\newpath
\moveto(54.07824701,405.22349423)
\lineto(202.22888177,405.22349423)
\lineto(202.22888177,357.07287091)
\lineto(54.07824701,357.07287091)
\closepath
}
}
{
\newrgbcolor{curcolor}{0 0 0}
\pscustom[linestyle=none,fillstyle=solid,fillcolor=curcolor]
{
\newpath
\moveto(80.0492695,381.65818158)
\lineto(83.1992695,381.65818158)
\curveto(84.15926101,381.65817036)(84.98926018,381.99817002)(85.6892695,382.67818158)
\curveto(86.38925878,383.35816866)(86.73925843,384.40816761)(86.7392695,385.82818158)
\curveto(86.73925843,387.02816499)(86.44925872,387.96816405)(85.8692695,388.64818158)
\curveto(85.28925988,389.34816267)(84.33926083,389.69816232)(83.0192695,389.69818158)
\lineto(80.0492695,389.69818158)
\lineto(80.0492695,381.65818158)
\moveto(77.4092695,391.85818158)
\lineto(82.8692695,391.85818158)
\curveto(83.169262,391.85816016)(83.53926163,391.84816017)(83.9792695,391.82818158)
\curveto(84.43926073,391.80816021)(84.90926026,391.73816028)(85.3892695,391.61818158)
\curveto(85.88925928,391.5181605)(86.37925879,391.33816068)(86.8592695,391.07818158)
\curveto(87.35925781,390.83816118)(87.79925737,390.48816153)(88.1792695,390.02818158)
\curveto(88.57925659,389.56816245)(88.89925627,388.98816303)(89.1392695,388.28818158)
\curveto(89.37925579,387.58816443)(89.49925567,386.72816529)(89.4992695,385.70818158)
\curveto(89.49925567,384.70816731)(89.34925582,383.8181682)(89.0492695,383.03818158)
\curveto(88.74925642,382.27816974)(88.31925685,381.62817039)(87.7592695,381.08818158)
\curveto(87.21925795,380.56817145)(86.5692586,380.16817185)(85.8092695,379.88818158)
\curveto(85.04926012,379.62817239)(84.21926095,379.49817252)(83.3192695,379.49818158)
\lineto(80.0492695,379.49818158)
\lineto(80.0492695,370.43818158)
\lineto(77.4092695,370.43818158)
\lineto(77.4092695,391.85818158)
}
}
{
\newrgbcolor{curcolor}{0 0 0}
\pscustom[linestyle=none,fillstyle=solid,fillcolor=curcolor]
{
\newpath
\moveto(93.64934762,378.47818158)
\lineto(99.40934762,378.47818158)
\lineto(96.67934762,388.82818158)
\lineto(96.61934762,388.82818158)
\lineto(93.64934762,378.47818158)
\moveto(94.93934762,391.85818158)
\lineto(98.47934762,391.85818158)
\lineto(104.23934762,370.43818158)
\lineto(101.47934762,370.43818158)
\lineto(99.94934762,376.31818158)
\lineto(93.10934762,376.31818158)
\lineto(91.51934762,370.43818158)
\lineto(88.75934762,370.43818158)
\lineto(94.93934762,391.85818158)
}
}
{
\newrgbcolor{curcolor}{0 0 0}
\pscustom[linestyle=none,fillstyle=solid,fillcolor=curcolor]
{
\newpath
\moveto(118.8786445,377.78818158)
\curveto(118.81862971,376.76817525)(118.65862987,375.78817623)(118.3986445,374.84818158)
\curveto(118.15863037,373.92817809)(117.78863074,373.10817891)(117.2886445,372.38818158)
\curveto(116.78863174,371.66818035)(116.1286324,371.08818093)(115.3086445,370.64818158)
\curveto(114.50863402,370.22818179)(113.528635,370.018182)(112.3686445,370.01818158)
\curveto(110.84863768,370.018182)(109.6286389,370.33818168)(108.7086445,370.97818158)
\curveto(107.80864072,371.63818038)(107.11864141,372.49817952)(106.6386445,373.55818158)
\curveto(106.15864237,374.6181774)(105.83864269,375.80817621)(105.6786445,377.12818158)
\curveto(105.53864299,378.44817357)(105.46864306,379.78817223)(105.4686445,381.14818158)
\curveto(105.46864306,382.48816953)(105.54864298,383.8181682)(105.7086445,385.13818158)
\curveto(105.86864266,386.47816554)(106.19864233,387.67816434)(106.6986445,388.73818158)
\curveto(107.19864133,389.79816222)(107.89864063,390.64816137)(108.7986445,391.28818158)
\curveto(109.69863883,391.94816007)(110.88863764,392.27815974)(112.3686445,392.27818158)
\curveto(114.54863398,392.27815974)(116.1286324,391.69816032)(117.1086445,390.53818158)
\curveto(118.10863042,389.37816264)(118.63862989,387.73816428)(118.6986445,385.61818158)
\lineto(115.9386445,385.61818158)
\curveto(115.91863261,386.2181658)(115.84863268,386.78816523)(115.7286445,387.32818158)
\curveto(115.60863292,387.88816413)(115.40863312,388.36816365)(115.1286445,388.76818158)
\curveto(114.86863366,389.18816283)(114.50863402,389.5181625)(114.0486445,389.75818158)
\curveto(113.60863492,389.99816202)(113.04863548,390.1181619)(112.3686445,390.11818158)
\curveto(111.44863708,390.1181619)(110.71863781,389.87816214)(110.1786445,389.39818158)
\curveto(109.63863889,388.93816308)(109.21863931,388.29816372)(108.9186445,387.47818158)
\curveto(108.63863989,386.67816534)(108.44864008,385.72816629)(108.3486445,384.62818158)
\curveto(108.26864026,383.54816847)(108.2286403,382.38816963)(108.2286445,381.14818158)
\curveto(108.2286403,379.90817211)(108.26864026,378.73817328)(108.3486445,377.63818158)
\curveto(108.44864008,376.55817546)(108.63863989,375.60817641)(108.9186445,374.78818158)
\curveto(109.21863931,373.98817803)(109.63863889,373.34817867)(110.1786445,372.86818158)
\curveto(110.71863781,372.40817961)(111.44863708,372.17817984)(112.3686445,372.17818158)
\curveto(113.16863536,372.17817984)(113.80863472,372.34817967)(114.2886445,372.68818158)
\curveto(114.76863376,373.02817899)(115.13863339,373.46817855)(115.3986445,374.00818158)
\curveto(115.67863285,374.54817747)(115.85863267,375.14817687)(115.9386445,375.80818158)
\curveto(116.03863249,376.46817555)(116.09863243,377.12817489)(116.1186445,377.78818158)
\lineto(118.8786445,377.78818158)
}
}
{
\newrgbcolor{curcolor}{0 0 0}
\pscustom[linestyle=none,fillstyle=solid,fillcolor=curcolor]
{
\newpath
\moveto(121.354582,391.85818158)
\lineto(123.994582,391.85818158)
\lineto(123.994582,381.47818158)
\lineto(124.054582,381.47818158)
\lineto(131.464582,391.85818158)
\lineto(134.404582,391.85818158)
\lineto(127.924582,382.91818158)
\lineto(134.944582,370.43818158)
\lineto(132.004582,370.43818158)
\lineto(126.214582,380.75818158)
\lineto(123.994582,377.72818158)
\lineto(123.994582,370.43818158)
\lineto(121.354582,370.43818158)
\lineto(121.354582,391.85818158)
}
}
{
\newrgbcolor{curcolor}{0 0 0}
\pscustom[linestyle=none,fillstyle=solid,fillcolor=curcolor]
{
\newpath
\moveto(139.264582,378.47818158)
\lineto(145.024582,378.47818158)
\lineto(142.294582,388.82818158)
\lineto(142.234582,388.82818158)
\lineto(139.264582,378.47818158)
\moveto(140.554582,391.85818158)
\lineto(144.094582,391.85818158)
\lineto(149.854582,370.43818158)
\lineto(147.094582,370.43818158)
\lineto(145.564582,376.31818158)
\lineto(138.724582,376.31818158)
\lineto(137.134582,370.43818158)
\lineto(134.374582,370.43818158)
\lineto(140.554582,391.85818158)
}
}
{
\newrgbcolor{curcolor}{0 0 0}
\pscustom[linestyle=none,fillstyle=solid,fillcolor=curcolor]
{
\newpath
\moveto(161.524582,385.91818158)
\curveto(161.48457013,386.47816554)(161.39457022,387.00816501)(161.254582,387.50818158)
\curveto(161.13457048,388.02816399)(160.93457068,388.47816354)(160.654582,388.85818158)
\curveto(160.39457122,389.23816278)(160.04457157,389.53816248)(159.604582,389.75818158)
\curveto(159.16457245,389.99816202)(158.614573,390.1181619)(157.954582,390.11818158)
\curveto(157.03457458,390.1181619)(156.30457531,389.87816214)(155.764582,389.39818158)
\curveto(155.22457639,388.93816308)(154.80457681,388.29816372)(154.504582,387.47818158)
\curveto(154.22457739,386.67816534)(154.03457758,385.72816629)(153.934582,384.62818158)
\curveto(153.85457776,383.54816847)(153.8145778,382.38816963)(153.814582,381.14818158)
\curveto(153.8145778,379.90817211)(153.85457776,378.73817328)(153.934582,377.63818158)
\curveto(154.03457758,376.55817546)(154.22457739,375.60817641)(154.504582,374.78818158)
\curveto(154.80457681,373.98817803)(155.22457639,373.34817867)(155.764582,372.86818158)
\curveto(156.30457531,372.40817961)(157.03457458,372.17817984)(157.954582,372.17818158)
\curveto(158.87457274,372.17817984)(159.59457202,372.4181796)(160.114582,372.89818158)
\curveto(160.63457098,373.39817862)(161.02457059,374.00817801)(161.284582,374.72818158)
\curveto(161.54457007,375.44817657)(161.70456991,376.22817579)(161.764582,377.06818158)
\curveto(161.84456977,377.90817411)(161.88456973,378.68817333)(161.884582,379.40818158)
\lineto(157.654582,379.40818158)
\lineto(157.654582,381.56818158)
\lineto(164.284582,381.56818158)
\lineto(164.284582,370.43818158)
\lineto(162.304582,370.43818158)
\lineto(162.304582,373.34818158)
\lineto(162.244582,373.34818158)
\curveto(161.96456965,372.42817959)(161.42457019,371.63818038)(160.624582,370.97818158)
\curveto(159.82457179,370.33818168)(158.82457279,370.018182)(157.624582,370.01818158)
\curveto(156.22457539,370.018182)(155.09457652,370.32818169)(154.234582,370.94818158)
\curveto(153.37457824,371.56818045)(152.70457891,372.38817963)(152.224582,373.40818158)
\curveto(151.76457985,374.44817757)(151.45458016,375.63817638)(151.294582,376.97818158)
\curveto(151.13458048,378.3181737)(151.05458056,379.70817231)(151.054582,381.14818158)
\curveto(151.05458056,382.48816953)(151.13458048,383.8181682)(151.294582,385.13818158)
\curveto(151.45458016,386.47816554)(151.78457983,387.67816434)(152.284582,388.73818158)
\curveto(152.78457883,389.79816222)(153.48457813,390.64816137)(154.384582,391.28818158)
\curveto(155.28457633,391.94816007)(156.47457514,392.27815974)(157.954582,392.27818158)
\curveto(158.97457264,392.27815974)(159.83457178,392.14815987)(160.534582,391.88818158)
\curveto(161.25457036,391.62816039)(161.84456977,391.28816073)(162.304582,390.86818158)
\curveto(162.78456883,390.46816155)(163.15456846,390.018162)(163.414582,389.51818158)
\curveto(163.67456794,389.018163)(163.86456775,388.52816349)(163.984582,388.04818158)
\curveto(164.12456749,387.58816443)(164.20456741,387.15816486)(164.224582,386.75818158)
\curveto(164.26456735,386.37816564)(164.28456733,386.09816592)(164.284582,385.91818158)
\lineto(161.524582,385.91818158)
}
}
{
\newrgbcolor{curcolor}{0 0 0}
\pscustom[linestyle=none,fillstyle=solid,fillcolor=curcolor]
{
\newpath
\moveto(167.46786325,391.85818158)
\lineto(178.53786325,391.85818158)
\lineto(178.53786325,389.51818158)
\lineto(170.10786325,389.51818158)
\lineto(170.10786325,382.79818158)
\lineto(178.05786325,382.79818158)
\lineto(178.05786325,380.45818158)
\lineto(170.10786325,380.45818158)
\lineto(170.10786325,372.77818158)
\lineto(178.89786325,372.77818158)
\lineto(178.89786325,370.43818158)
\lineto(167.46786325,370.43818158)
\lineto(167.46786325,391.85818158)
}
}
{
\newrgbcolor{curcolor}{0 0 0}
\pscustom[linestyle=none,fillstyle=solid,fillcolor=curcolor]
{
\newpath
\moveto(367.32515495,381.65818529)
\lineto(370.47515495,381.65818529)
\curveto(371.43514646,381.65817407)(372.26514563,381.99817373)(372.96515495,382.67818529)
\curveto(373.66514423,383.35817237)(374.01514388,384.40817132)(374.01515495,385.82818529)
\curveto(374.01514388,387.0281687)(373.72514417,387.96816776)(373.14515495,388.64818529)
\curveto(372.56514533,389.34816638)(371.61514628,389.69816603)(370.29515495,389.69818529)
\lineto(367.32515495,389.69818529)
\lineto(367.32515495,381.65818529)
\moveto(364.68515495,391.85818529)
\lineto(370.14515495,391.85818529)
\curveto(370.44514745,391.85816387)(370.81514708,391.84816388)(371.25515495,391.82818529)
\curveto(371.71514618,391.80816392)(372.18514571,391.73816399)(372.66515495,391.61818529)
\curveto(373.16514473,391.51816421)(373.65514424,391.33816439)(374.13515495,391.07818529)
\curveto(374.63514326,390.83816489)(375.07514282,390.48816524)(375.45515495,390.02818529)
\curveto(375.85514204,389.56816616)(376.17514172,388.98816674)(376.41515495,388.28818529)
\curveto(376.65514124,387.58816814)(376.77514112,386.728169)(376.77515495,385.70818529)
\curveto(376.77514112,384.70817102)(376.62514127,383.81817191)(376.32515495,383.03818529)
\curveto(376.02514187,382.27817345)(375.5951423,381.6281741)(375.03515495,381.08818529)
\curveto(374.4951434,380.56817516)(373.84514405,380.16817556)(373.08515495,379.88818529)
\curveto(372.32514557,379.6281761)(371.4951464,379.49817623)(370.59515495,379.49818529)
\lineto(367.32515495,379.49818529)
\lineto(367.32515495,370.43818529)
\lineto(364.68515495,370.43818529)
\lineto(364.68515495,391.85818529)
}
}
{
\newrgbcolor{curcolor}{0 0 0}
\pscustom[linestyle=none,fillstyle=solid,fillcolor=curcolor]
{
\newpath
\moveto(381.73921745,382.13818529)
\lineto(384.34921745,382.13818529)
\curveto(384.72921008,382.13817359)(385.16920964,382.15817357)(385.66921745,382.19818529)
\curveto(386.18920862,382.23817349)(386.67920813,382.38817334)(387.13921745,382.64818529)
\curveto(387.59920721,382.90817282)(387.97920683,383.31817241)(388.27921745,383.87818529)
\curveto(388.59920621,384.43817129)(388.75920605,385.23817049)(388.75921745,386.27818529)
\curveto(388.75920605,387.35816837)(388.42920638,388.19816753)(387.76921745,388.79818529)
\curveto(387.1092077,389.39816633)(386.14920866,389.69816603)(384.88921745,389.69818529)
\lineto(381.73921745,389.69818529)
\lineto(381.73921745,382.13818529)
\moveto(379.09921745,391.85818529)
\lineto(386.02921745,391.85818529)
\curveto(387.72920708,391.85816387)(389.06920574,391.38816434)(390.04921745,390.44818529)
\curveto(391.02920378,389.50816622)(391.51920329,388.18816754)(391.51921745,386.48818529)
\curveto(391.51920329,385.90816982)(391.45920335,385.3281704)(391.33921745,384.74818529)
\curveto(391.23920357,384.16817156)(391.05920375,383.6281721)(390.79921745,383.12818529)
\curveto(390.53920427,382.64817308)(390.19920461,382.21817351)(389.77921745,381.83818529)
\curveto(389.35920545,381.47817425)(388.83920597,381.2281745)(388.21921745,381.08818529)
\lineto(388.21921745,381.02818529)
\curveto(389.15920565,380.9281748)(389.87920493,380.53817519)(390.37921745,379.85818529)
\curveto(390.89920391,379.17817655)(391.18920362,378.37817735)(391.24921745,377.45818529)
\lineto(391.42921745,373.79818529)
\curveto(391.44920336,373.19818253)(391.48920332,372.70818302)(391.54921745,372.32818529)
\curveto(391.6092032,371.94818378)(391.68920312,371.6281841)(391.78921745,371.36818529)
\curveto(391.88920292,371.1281846)(391.99920281,370.93818479)(392.11921745,370.79818529)
\curveto(392.25920255,370.65818507)(392.4092024,370.53818519)(392.56921745,370.43818529)
\lineto(389.38921745,370.43818529)
\curveto(389.26920554,370.55818517)(389.16920564,370.73818499)(389.08921745,370.97818529)
\curveto(389.0092058,371.21818451)(388.93920587,371.47818425)(388.87921745,371.75818529)
\curveto(388.81920599,372.05818367)(388.76920604,372.35818337)(388.72921745,372.65818529)
\curveto(388.7092061,372.97818275)(388.68920612,373.26818246)(388.66921745,373.52818529)
\lineto(388.48921745,376.85818529)
\curveto(388.42920638,377.59817813)(388.28920652,378.16817756)(388.06921745,378.56818529)
\curveto(387.86920694,378.98817674)(387.61920719,379.29817643)(387.31921745,379.49818529)
\curveto(387.01920779,379.71817601)(386.68920812,379.84817588)(386.32921745,379.88818529)
\curveto(385.98920882,379.94817578)(385.64920916,379.97817575)(385.30921745,379.97818529)
\lineto(381.73921745,379.97818529)
\lineto(381.73921745,370.43818529)
\lineto(379.09921745,370.43818529)
\lineto(379.09921745,391.85818529)
}
}
{
\newrgbcolor{curcolor}{0 0 0}
\pscustom[linestyle=none,fillstyle=solid,fillcolor=curcolor]
{
\newpath
\moveto(394.68515495,391.85818529)
\lineto(397.32515495,391.85818529)
\lineto(397.32515495,370.43818529)
\lineto(394.68515495,370.43818529)
\lineto(394.68515495,391.85818529)
}
}
{
\newrgbcolor{curcolor}{0 0 0}
\pscustom[linestyle=none,fillstyle=solid,fillcolor=curcolor]
{
\newpath
\moveto(398.97890495,391.85818529)
\lineto(401.73890495,391.85818529)
\lineto(405.93890495,373.46818529)
\lineto(405.99890495,373.46818529)
\lineto(410.19890495,391.85818529)
\lineto(412.95890495,391.85818529)
\lineto(407.55890495,370.43818529)
\lineto(404.19890495,370.43818529)
\lineto(398.97890495,391.85818529)
}
}
{
\newrgbcolor{curcolor}{0 0 0}
\pscustom[linestyle=none,fillstyle=solid,fillcolor=curcolor]
{
\newpath
\moveto(414.6656237,391.85818529)
\lineto(417.3056237,391.85818529)
\lineto(417.3056237,370.43818529)
\lineto(414.6656237,370.43818529)
\lineto(414.6656237,391.85818529)
}
}
{
\newrgbcolor{curcolor}{0 0 0}
\pscustom[linestyle=none,fillstyle=solid,fillcolor=curcolor]
{
\newpath
\moveto(420.7593737,391.85818529)
\lineto(423.3993737,391.85818529)
\lineto(423.3993737,372.77818529)
\lineto(432.0993737,372.77818529)
\lineto(432.0993737,370.43818529)
\lineto(420.7593737,370.43818529)
\lineto(420.7593737,391.85818529)
}
}
{
\newrgbcolor{curcolor}{0 0 0}
\pscustom[linestyle=none,fillstyle=solid,fillcolor=curcolor]
{
\newpath
\moveto(434.06015495,391.85818529)
\lineto(445.13015495,391.85818529)
\lineto(445.13015495,389.51818529)
\lineto(436.70015495,389.51818529)
\lineto(436.70015495,382.79818529)
\lineto(444.65015495,382.79818529)
\lineto(444.65015495,380.45818529)
\lineto(436.70015495,380.45818529)
\lineto(436.70015495,372.77818529)
\lineto(445.49015495,372.77818529)
\lineto(445.49015495,370.43818529)
\lineto(434.06015495,370.43818529)
\lineto(434.06015495,391.85818529)
}
}
{
\newrgbcolor{curcolor}{0 0 0}
\pscustom[linestyle=none,fillstyle=solid,fillcolor=curcolor]
{
\newpath
\moveto(458.1168737,385.91818529)
\curveto(458.07686183,386.47816925)(457.98686192,387.00816872)(457.8468737,387.50818529)
\curveto(457.72686218,388.0281677)(457.52686238,388.47816725)(457.2468737,388.85818529)
\curveto(456.98686292,389.23816649)(456.63686327,389.53816619)(456.1968737,389.75818529)
\curveto(455.75686415,389.99816573)(455.2068647,390.11816561)(454.5468737,390.11818529)
\curveto(453.62686628,390.11816561)(452.89686701,389.87816585)(452.3568737,389.39818529)
\curveto(451.81686809,388.93816679)(451.39686851,388.29816743)(451.0968737,387.47818529)
\curveto(450.81686909,386.67816905)(450.62686928,385.72817)(450.5268737,384.62818529)
\curveto(450.44686946,383.54817218)(450.4068695,382.38817334)(450.4068737,381.14818529)
\curveto(450.4068695,379.90817582)(450.44686946,378.73817699)(450.5268737,377.63818529)
\curveto(450.62686928,376.55817917)(450.81686909,375.60818012)(451.0968737,374.78818529)
\curveto(451.39686851,373.98818174)(451.81686809,373.34818238)(452.3568737,372.86818529)
\curveto(452.89686701,372.40818332)(453.62686628,372.17818355)(454.5468737,372.17818529)
\curveto(455.46686444,372.17818355)(456.18686372,372.41818331)(456.7068737,372.89818529)
\curveto(457.22686268,373.39818233)(457.61686229,374.00818172)(457.8768737,374.72818529)
\curveto(458.13686177,375.44818028)(458.29686161,376.2281795)(458.3568737,377.06818529)
\curveto(458.43686147,377.90817782)(458.47686143,378.68817704)(458.4768737,379.40818529)
\lineto(454.2468737,379.40818529)
\lineto(454.2468737,381.56818529)
\lineto(460.8768737,381.56818529)
\lineto(460.8768737,370.43818529)
\lineto(458.8968737,370.43818529)
\lineto(458.8968737,373.34818529)
\lineto(458.8368737,373.34818529)
\curveto(458.55686135,372.4281833)(458.01686189,371.63818409)(457.2168737,370.97818529)
\curveto(456.41686349,370.33818539)(455.41686449,370.01818571)(454.2168737,370.01818529)
\curveto(452.81686709,370.01818571)(451.68686822,370.3281854)(450.8268737,370.94818529)
\curveto(449.96686994,371.56818416)(449.29687061,372.38818334)(448.8168737,373.40818529)
\curveto(448.35687155,374.44818128)(448.04687186,375.63818009)(447.8868737,376.97818529)
\curveto(447.72687218,378.31817741)(447.64687226,379.70817602)(447.6468737,381.14818529)
\curveto(447.64687226,382.48817324)(447.72687218,383.81817191)(447.8868737,385.13818529)
\curveto(448.04687186,386.47816925)(448.37687153,387.67816805)(448.8768737,388.73818529)
\curveto(449.37687053,389.79816593)(450.07686983,390.64816508)(450.9768737,391.28818529)
\curveto(451.87686803,391.94816378)(453.06686684,392.27816345)(454.5468737,392.27818529)
\curveto(455.56686434,392.27816345)(456.42686348,392.14816358)(457.1268737,391.88818529)
\curveto(457.84686206,391.6281641)(458.43686147,391.28816444)(458.8968737,390.86818529)
\curveto(459.37686053,390.46816526)(459.74686016,390.01816571)(460.0068737,389.51818529)
\curveto(460.26685964,389.01816671)(460.45685945,388.5281672)(460.5768737,388.04818529)
\curveto(460.71685919,387.58816814)(460.79685911,387.15816857)(460.8168737,386.75818529)
\curveto(460.85685905,386.37816935)(460.87685903,386.09816963)(460.8768737,385.91818529)
\lineto(458.1168737,385.91818529)
}
}
{
\newrgbcolor{curcolor}{0 0 0}
\pscustom[linestyle=none,fillstyle=solid,fillcolor=curcolor]
{
\newpath
\moveto(464.06015495,391.85818529)
\lineto(475.13015495,391.85818529)
\lineto(475.13015495,389.51818529)
\lineto(466.70015495,389.51818529)
\lineto(466.70015495,382.79818529)
\lineto(474.65015495,382.79818529)
\lineto(474.65015495,380.45818529)
\lineto(466.70015495,380.45818529)
\lineto(466.70015495,372.77818529)
\lineto(475.49015495,372.77818529)
\lineto(475.49015495,370.43818529)
\lineto(464.06015495,370.43818529)
\lineto(464.06015495,391.85818529)
}
}
{
\newrgbcolor{curcolor}{0 0 0}
\pscustom[linewidth=1.84937644,linecolor=curcolor]
{
\newpath
\moveto(346.01232413,405.22349413)
\lineto(494.1629589,405.22349413)
\lineto(494.1629589,357.07287081)
\lineto(346.01232413,357.07287081)
\closepath
}
}
{
\newrgbcolor{curcolor}{0 0 0}
\pscustom[linestyle=none,fillstyle=solid,fillcolor=curcolor]
{
\newpath
\moveto(372.00287084,248.23967999)
\lineto(374.61287084,248.23967999)
\curveto(374.99286347,248.23966829)(375.43286303,248.25966827)(375.93287084,248.29967999)
\curveto(376.45286201,248.33966819)(376.94286152,248.48966804)(377.40287084,248.74967999)
\curveto(377.8628606,249.00966752)(378.24286022,249.41966711)(378.54287084,249.97967999)
\curveto(378.8628596,250.53966599)(379.02285944,251.33966519)(379.02287084,252.37967999)
\curveto(379.02285944,253.45966307)(378.69285977,254.29966223)(378.03287084,254.89967999)
\curveto(377.37286109,255.49966103)(376.41286205,255.79966073)(375.15287084,255.79967999)
\lineto(372.00287084,255.79967999)
\lineto(372.00287084,248.23967999)
\moveto(369.36287084,257.95967999)
\lineto(376.29287084,257.95967999)
\curveto(377.99286047,257.95965857)(379.33285913,257.48965904)(380.31287084,256.54967999)
\curveto(381.29285717,255.60966092)(381.78285668,254.28966224)(381.78287084,252.58967999)
\curveto(381.78285668,252.00966452)(381.72285674,251.4296651)(381.60287084,250.84967999)
\curveto(381.50285696,250.26966626)(381.32285714,249.7296668)(381.06287084,249.22967999)
\curveto(380.80285766,248.74966778)(380.462858,248.31966821)(380.04287084,247.93967999)
\curveto(379.62285884,247.57966895)(379.10285936,247.3296692)(378.48287084,247.18967999)
\lineto(378.48287084,247.12967999)
\curveto(379.42285904,247.0296695)(380.14285832,246.63966989)(380.64287084,245.95967999)
\curveto(381.1628573,245.27967125)(381.45285701,244.47967205)(381.51287084,243.55967999)
\lineto(381.69287084,239.89967999)
\curveto(381.71285675,239.29967723)(381.75285671,238.80967772)(381.81287084,238.42967999)
\curveto(381.87285659,238.04967848)(381.95285651,237.7296788)(382.05287084,237.46967999)
\curveto(382.15285631,237.2296793)(382.2628562,237.03967949)(382.38287084,236.89967999)
\curveto(382.52285594,236.75967977)(382.67285579,236.63967989)(382.83287084,236.53967999)
\lineto(379.65287084,236.53967999)
\curveto(379.53285893,236.65967987)(379.43285903,236.83967969)(379.35287084,237.07967999)
\curveto(379.27285919,237.31967921)(379.20285926,237.57967895)(379.14287084,237.85967999)
\curveto(379.08285938,238.15967837)(379.03285943,238.45967807)(378.99287084,238.75967999)
\curveto(378.97285949,239.07967745)(378.95285951,239.36967716)(378.93287084,239.62967999)
\lineto(378.75287084,242.95967999)
\curveto(378.69285977,243.69967283)(378.55285991,244.26967226)(378.33287084,244.66967999)
\curveto(378.13286033,245.08967144)(377.88286058,245.39967113)(377.58287084,245.59967999)
\curveto(377.28286118,245.81967071)(376.95286151,245.94967058)(376.59287084,245.98967999)
\curveto(376.25286221,246.04967048)(375.91286255,246.07967045)(375.57287084,246.07967999)
\lineto(372.00287084,246.07967999)
\lineto(372.00287084,236.53967999)
\lineto(369.36287084,236.53967999)
\lineto(369.36287084,257.95967999)
}
}
{
\newrgbcolor{curcolor}{0 0 0}
\pscustom[linestyle=none,fillstyle=solid,fillcolor=curcolor]
{
\newpath
\moveto(391.54880834,256.21967999)
\curveto(390.62880092,256.21966031)(389.89880165,255.97966055)(389.35880834,255.49967999)
\curveto(388.81880273,255.03966149)(388.39880315,254.39966213)(388.09880834,253.57967999)
\curveto(387.81880373,252.77966375)(387.62880392,251.8296647)(387.52880834,250.72967999)
\curveto(387.4488041,249.64966688)(387.40880414,248.48966804)(387.40880834,247.24967999)
\curveto(387.40880414,246.00967052)(387.4488041,244.83967169)(387.52880834,243.73967999)
\curveto(387.62880392,242.65967387)(387.81880373,241.70967482)(388.09880834,240.88967999)
\curveto(388.39880315,240.08967644)(388.81880273,239.44967708)(389.35880834,238.96967999)
\curveto(389.89880165,238.50967802)(390.62880092,238.27967825)(391.54880834,238.27967999)
\curveto(392.46879908,238.27967825)(393.19879835,238.50967802)(393.73880834,238.96967999)
\curveto(394.27879727,239.44967708)(394.68879686,240.08967644)(394.96880834,240.88967999)
\curveto(395.26879628,241.70967482)(395.45879609,242.65967387)(395.53880834,243.73967999)
\curveto(395.63879591,244.83967169)(395.68879586,246.00967052)(395.68880834,247.24967999)
\curveto(395.68879586,248.48966804)(395.63879591,249.64966688)(395.53880834,250.72967999)
\curveto(395.45879609,251.8296647)(395.26879628,252.77966375)(394.96880834,253.57967999)
\curveto(394.68879686,254.39966213)(394.27879727,255.03966149)(393.73880834,255.49967999)
\curveto(393.19879835,255.97966055)(392.46879908,256.21966031)(391.54880834,256.21967999)
\moveto(391.54880834,258.37967999)
\curveto(393.02879852,258.37965815)(394.21879733,258.04965848)(395.11880834,257.38967999)
\curveto(396.01879553,256.74965978)(396.71879483,255.89966063)(397.21880834,254.83967999)
\curveto(397.71879383,253.77966275)(398.0487935,252.57966395)(398.20880834,251.23967999)
\curveto(398.36879318,249.91966661)(398.4487931,248.58966794)(398.44880834,247.24967999)
\curveto(398.4487931,245.88967064)(398.36879318,244.54967198)(398.20880834,243.22967999)
\curveto(398.0487935,241.90967462)(397.71879383,240.71967581)(397.21880834,239.65967999)
\curveto(396.71879483,238.59967793)(396.01879553,237.73967879)(395.11880834,237.07967999)
\curveto(394.21879733,236.43968009)(393.02879852,236.11968041)(391.54880834,236.11967999)
\curveto(390.06880148,236.11968041)(388.87880267,236.43968009)(387.97880834,237.07967999)
\curveto(387.07880447,237.73967879)(386.37880517,238.59967793)(385.87880834,239.65967999)
\curveto(385.37880617,240.71967581)(385.0488065,241.90967462)(384.88880834,243.22967999)
\curveto(384.72880682,244.54967198)(384.6488069,245.88967064)(384.64880834,247.24967999)
\curveto(384.6488069,248.58966794)(384.72880682,249.91966661)(384.88880834,251.23967999)
\curveto(385.0488065,252.57966395)(385.37880617,253.77966275)(385.87880834,254.83967999)
\curveto(386.37880517,255.89966063)(387.07880447,256.74965978)(387.97880834,257.38967999)
\curveto(388.87880267,258.04965848)(390.06880148,258.37965815)(391.54880834,258.37967999)
}
}
{
\newrgbcolor{curcolor}{0 0 0}
\pscustom[linestyle=none,fillstyle=solid,fillcolor=curcolor]
{
\newpath
\moveto(401.40802709,257.95967999)
\lineto(404.04802709,257.95967999)
\lineto(404.04802709,242.89967999)
\curveto(404.04802295,241.31967521)(404.31802268,240.14967638)(404.85802709,239.38967999)
\curveto(405.41802158,238.64967788)(406.35802064,238.27967825)(407.67802709,238.27967999)
\curveto(409.07801792,238.27967825)(410.03801696,238.66967786)(410.55802709,239.44967999)
\curveto(411.07801592,240.24967628)(411.33801566,241.39967513)(411.33802709,242.89967999)
\lineto(411.33802709,257.95967999)
\lineto(413.97802709,257.95967999)
\lineto(413.97802709,242.89967999)
\curveto(413.97801302,240.83967569)(413.44801355,239.18967734)(412.38802709,237.94967999)
\curveto(411.34801565,236.7296798)(409.77801722,236.11968041)(407.67802709,236.11967999)
\curveto(405.4980215,236.11968041)(403.90802309,236.68967984)(402.90802709,237.82967999)
\curveto(401.90802509,238.96967756)(401.40802559,240.65967587)(401.40802709,242.89967999)
\lineto(401.40802709,257.95967999)
}
}
{
\newrgbcolor{curcolor}{0 0 0}
\pscustom[linestyle=none,fillstyle=solid,fillcolor=curcolor]
{
\newpath
\moveto(423.74396459,236.53967999)
\lineto(421.10396459,236.53967999)
\lineto(421.10396459,255.61967999)
\lineto(415.73396459,255.61967999)
\lineto(415.73396459,257.95967999)
\lineto(429.14396459,257.95967999)
\lineto(429.14396459,255.61967999)
\lineto(423.74396459,255.61967999)
\lineto(423.74396459,236.53967999)
}
}
{
\newrgbcolor{curcolor}{0 0 0}
\pscustom[linestyle=none,fillstyle=solid,fillcolor=curcolor]
{
\newpath
\moveto(431.12068334,257.95967999)
\lineto(442.19068334,257.95967999)
\lineto(442.19068334,255.61967999)
\lineto(433.76068334,255.61967999)
\lineto(433.76068334,248.89967999)
\lineto(441.71068334,248.89967999)
\lineto(441.71068334,246.55967999)
\lineto(433.76068334,246.55967999)
\lineto(433.76068334,238.87967999)
\lineto(442.55068334,238.87967999)
\lineto(442.55068334,236.53967999)
\lineto(431.12068334,236.53967999)
\lineto(431.12068334,257.95967999)
}
}
{
\newrgbcolor{curcolor}{0 0 0}
\pscustom[linestyle=none,fillstyle=solid,fillcolor=curcolor]
{
\newpath
\moveto(443.26740209,234.28967999)
\lineto(458.26740209,234.28967999)
\lineto(458.26740209,232.78967999)
\lineto(443.26740209,232.78967999)
\lineto(443.26740209,234.28967999)
}
}
{
\newrgbcolor{curcolor}{0 0 0}
\pscustom[linestyle=none,fillstyle=solid,fillcolor=curcolor]
{
\newpath
\moveto(462.64740209,247.75967999)
\lineto(465.79740209,247.75967999)
\curveto(466.7573936,247.75966877)(467.58739277,248.09966843)(468.28740209,248.77967999)
\curveto(468.98739137,249.45966707)(469.33739102,250.50966602)(469.33740209,251.92967999)
\curveto(469.33739102,253.1296634)(469.04739131,254.06966246)(468.46740209,254.74967999)
\curveto(467.88739247,255.44966108)(466.93739342,255.79966073)(465.61740209,255.79967999)
\lineto(462.64740209,255.79967999)
\lineto(462.64740209,247.75967999)
\moveto(460.00740209,257.95967999)
\lineto(465.46740209,257.95967999)
\curveto(465.76739459,257.95965857)(466.13739422,257.94965858)(466.57740209,257.92967999)
\curveto(467.03739332,257.90965862)(467.50739285,257.83965869)(467.98740209,257.71967999)
\curveto(468.48739187,257.61965891)(468.97739138,257.43965909)(469.45740209,257.17967999)
\curveto(469.9573904,256.93965959)(470.39738996,256.58965994)(470.77740209,256.12967999)
\curveto(471.17738918,255.66966086)(471.49738886,255.08966144)(471.73740209,254.38967999)
\curveto(471.97738838,253.68966284)(472.09738826,252.8296637)(472.09740209,251.80967999)
\curveto(472.09738826,250.80966572)(471.94738841,249.91966661)(471.64740209,249.13967999)
\curveto(471.34738901,248.37966815)(470.91738944,247.7296688)(470.35740209,247.18967999)
\curveto(469.81739054,246.66966986)(469.16739119,246.26967026)(468.40740209,245.98967999)
\curveto(467.64739271,245.7296708)(466.81739354,245.59967093)(465.91740209,245.59967999)
\lineto(462.64740209,245.59967999)
\lineto(462.64740209,236.53967999)
\lineto(460.00740209,236.53967999)
\lineto(460.00740209,257.95967999)
}
}
{
\newrgbcolor{curcolor}{0 0 0}
\pscustom[linestyle=none,fillstyle=solid,fillcolor=curcolor]
{
\newpath
\moveto(477.06146459,248.23967999)
\lineto(479.67146459,248.23967999)
\curveto(480.05145722,248.23966829)(480.49145678,248.25966827)(480.99146459,248.29967999)
\curveto(481.51145576,248.33966819)(482.00145527,248.48966804)(482.46146459,248.74967999)
\curveto(482.92145435,249.00966752)(483.30145397,249.41966711)(483.60146459,249.97967999)
\curveto(483.92145335,250.53966599)(484.08145319,251.33966519)(484.08146459,252.37967999)
\curveto(484.08145319,253.45966307)(483.75145352,254.29966223)(483.09146459,254.89967999)
\curveto(482.43145484,255.49966103)(481.4714558,255.79966073)(480.21146459,255.79967999)
\lineto(477.06146459,255.79967999)
\lineto(477.06146459,248.23967999)
\moveto(474.42146459,257.95967999)
\lineto(481.35146459,257.95967999)
\curveto(483.05145422,257.95965857)(484.39145288,257.48965904)(485.37146459,256.54967999)
\curveto(486.35145092,255.60966092)(486.84145043,254.28966224)(486.84146459,252.58967999)
\curveto(486.84145043,252.00966452)(486.78145049,251.4296651)(486.66146459,250.84967999)
\curveto(486.56145071,250.26966626)(486.38145089,249.7296668)(486.12146459,249.22967999)
\curveto(485.86145141,248.74966778)(485.52145175,248.31966821)(485.10146459,247.93967999)
\curveto(484.68145259,247.57966895)(484.16145311,247.3296692)(483.54146459,247.18967999)
\lineto(483.54146459,247.12967999)
\curveto(484.48145279,247.0296695)(485.20145207,246.63966989)(485.70146459,245.95967999)
\curveto(486.22145105,245.27967125)(486.51145076,244.47967205)(486.57146459,243.55967999)
\lineto(486.75146459,239.89967999)
\curveto(486.7714505,239.29967723)(486.81145046,238.80967772)(486.87146459,238.42967999)
\curveto(486.93145034,238.04967848)(487.01145026,237.7296788)(487.11146459,237.46967999)
\curveto(487.21145006,237.2296793)(487.32144995,237.03967949)(487.44146459,236.89967999)
\curveto(487.58144969,236.75967977)(487.73144954,236.63967989)(487.89146459,236.53967999)
\lineto(484.71146459,236.53967999)
\curveto(484.59145268,236.65967987)(484.49145278,236.83967969)(484.41146459,237.07967999)
\curveto(484.33145294,237.31967921)(484.26145301,237.57967895)(484.20146459,237.85967999)
\curveto(484.14145313,238.15967837)(484.09145318,238.45967807)(484.05146459,238.75967999)
\curveto(484.03145324,239.07967745)(484.01145326,239.36967716)(483.99146459,239.62967999)
\lineto(483.81146459,242.95967999)
\curveto(483.75145352,243.69967283)(483.61145366,244.26967226)(483.39146459,244.66967999)
\curveto(483.19145408,245.08967144)(482.94145433,245.39967113)(482.64146459,245.59967999)
\curveto(482.34145493,245.81967071)(482.01145526,245.94967058)(481.65146459,245.98967999)
\curveto(481.31145596,246.04967048)(480.9714563,246.07967045)(480.63146459,246.07967999)
\lineto(477.06146459,246.07967999)
\lineto(477.06146459,236.53967999)
\lineto(474.42146459,236.53967999)
\lineto(474.42146459,257.95967999)
}
}
{
\newrgbcolor{curcolor}{0 0 0}
\pscustom[linestyle=none,fillstyle=solid,fillcolor=curcolor]
{
\newpath
\moveto(490.00740209,257.95967999)
\lineto(492.64740209,257.95967999)
\lineto(492.64740209,236.53967999)
\lineto(490.00740209,236.53967999)
\lineto(490.00740209,257.95967999)
}
}
{
\newrgbcolor{curcolor}{0 0 0}
\pscustom[linestyle=none,fillstyle=solid,fillcolor=curcolor]
{
\newpath
\moveto(494.30115209,257.95967999)
\lineto(497.06115209,257.95967999)
\lineto(501.26115209,239.56967999)
\lineto(501.32115209,239.56967999)
\lineto(505.52115209,257.95967999)
\lineto(508.28115209,257.95967999)
\lineto(502.88115209,236.53967999)
\lineto(499.52115209,236.53967999)
\lineto(494.30115209,257.95967999)
}
}
{
\newrgbcolor{curcolor}{0 0 0}
\pscustom[linestyle=none,fillstyle=solid,fillcolor=curcolor]
{
\newpath
\moveto(509.98787084,257.95967999)
\lineto(512.62787084,257.95967999)
\lineto(512.62787084,236.53967999)
\lineto(509.98787084,236.53967999)
\lineto(509.98787084,257.95967999)
}
}
{
\newrgbcolor{curcolor}{0 0 0}
\pscustom[linestyle=none,fillstyle=solid,fillcolor=curcolor]
{
\newpath
\moveto(516.08162084,257.95967999)
\lineto(518.72162084,257.95967999)
\lineto(518.72162084,238.87967999)
\lineto(527.42162084,238.87967999)
\lineto(527.42162084,236.53967999)
\lineto(516.08162084,236.53967999)
\lineto(516.08162084,257.95967999)
}
}
{
\newrgbcolor{curcolor}{0 0 0}
\pscustom[linestyle=none,fillstyle=solid,fillcolor=curcolor]
{
\newpath
\moveto(529.38240209,257.95967999)
\lineto(540.45240209,257.95967999)
\lineto(540.45240209,255.61967999)
\lineto(532.02240209,255.61967999)
\lineto(532.02240209,248.89967999)
\lineto(539.97240209,248.89967999)
\lineto(539.97240209,246.55967999)
\lineto(532.02240209,246.55967999)
\lineto(532.02240209,238.87967999)
\lineto(540.81240209,238.87967999)
\lineto(540.81240209,236.53967999)
\lineto(529.38240209,236.53967999)
\lineto(529.38240209,257.95967999)
}
}
{
\newrgbcolor{curcolor}{0 0 0}
\pscustom[linestyle=none,fillstyle=solid,fillcolor=curcolor]
{
\newpath
\moveto(553.43912084,252.01967999)
\curveto(553.39910897,252.57966395)(553.30910906,253.10966342)(553.16912084,253.60967999)
\curveto(553.04910932,254.1296624)(552.84910952,254.57966195)(552.56912084,254.95967999)
\curveto(552.30911006,255.33966119)(551.95911041,255.63966089)(551.51912084,255.85967999)
\curveto(551.07911129,256.09966043)(550.52911184,256.21966031)(549.86912084,256.21967999)
\curveto(548.94911342,256.21966031)(548.21911415,255.97966055)(547.67912084,255.49967999)
\curveto(547.13911523,255.03966149)(546.71911565,254.39966213)(546.41912084,253.57967999)
\curveto(546.13911623,252.77966375)(545.94911642,251.8296647)(545.84912084,250.72967999)
\curveto(545.7691166,249.64966688)(545.72911664,248.48966804)(545.72912084,247.24967999)
\curveto(545.72911664,246.00967052)(545.7691166,244.83967169)(545.84912084,243.73967999)
\curveto(545.94911642,242.65967387)(546.13911623,241.70967482)(546.41912084,240.88967999)
\curveto(546.71911565,240.08967644)(547.13911523,239.44967708)(547.67912084,238.96967999)
\curveto(548.21911415,238.50967802)(548.94911342,238.27967825)(549.86912084,238.27967999)
\curveto(550.78911158,238.27967825)(551.50911086,238.51967801)(552.02912084,238.99967999)
\curveto(552.54910982,239.49967703)(552.93910943,240.10967642)(553.19912084,240.82967999)
\curveto(553.45910891,241.54967498)(553.61910875,242.3296742)(553.67912084,243.16967999)
\curveto(553.75910861,244.00967252)(553.79910857,244.78967174)(553.79912084,245.50967999)
\lineto(549.56912084,245.50967999)
\lineto(549.56912084,247.66967999)
\lineto(556.19912084,247.66967999)
\lineto(556.19912084,236.53967999)
\lineto(554.21912084,236.53967999)
\lineto(554.21912084,239.44967999)
\lineto(554.15912084,239.44967999)
\curveto(553.87910849,238.529678)(553.33910903,237.73967879)(552.53912084,237.07967999)
\curveto(551.73911063,236.43968009)(550.73911163,236.11968041)(549.53912084,236.11967999)
\curveto(548.13911423,236.11968041)(547.00911536,236.4296801)(546.14912084,237.04967999)
\curveto(545.28911708,237.66967886)(544.61911775,238.48967804)(544.13912084,239.50967999)
\curveto(543.67911869,240.54967598)(543.369119,241.73967479)(543.20912084,243.07967999)
\curveto(543.04911932,244.41967211)(542.9691194,245.80967072)(542.96912084,247.24967999)
\curveto(542.9691194,248.58966794)(543.04911932,249.91966661)(543.20912084,251.23967999)
\curveto(543.369119,252.57966395)(543.69911867,253.77966275)(544.19912084,254.83967999)
\curveto(544.69911767,255.89966063)(545.39911697,256.74965978)(546.29912084,257.38967999)
\curveto(547.19911517,258.04965848)(548.38911398,258.37965815)(549.86912084,258.37967999)
\curveto(550.88911148,258.37965815)(551.74911062,258.24965828)(552.44912084,257.98967999)
\curveto(553.1691092,257.7296588)(553.75910861,257.38965914)(554.21912084,256.96967999)
\curveto(554.69910767,256.56965996)(555.0691073,256.11966041)(555.32912084,255.61967999)
\curveto(555.58910678,255.11966141)(555.77910659,254.6296619)(555.89912084,254.14967999)
\curveto(556.03910633,253.68966284)(556.11910625,253.25966327)(556.13912084,252.85967999)
\curveto(556.17910619,252.47966405)(556.19910617,252.19966433)(556.19912084,252.01967999)
\lineto(553.43912084,252.01967999)
}
}
{
\newrgbcolor{curcolor}{0 0 0}
\pscustom[linestyle=none,fillstyle=solid,fillcolor=curcolor]
{
\newpath
\moveto(559.38240209,257.95967999)
\lineto(570.45240209,257.95967999)
\lineto(570.45240209,255.61967999)
\lineto(562.02240209,255.61967999)
\lineto(562.02240209,248.89967999)
\lineto(569.97240209,248.89967999)
\lineto(569.97240209,246.55967999)
\lineto(562.02240209,246.55967999)
\lineto(562.02240209,238.87967999)
\lineto(570.81240209,238.87967999)
\lineto(570.81240209,236.53967999)
\lineto(559.38240209,236.53967999)
\lineto(559.38240209,257.95967999)
}
}
{
\newrgbcolor{curcolor}{0 0 0}
\pscustom[linewidth=2.37778282,linecolor=curcolor]
{
\newpath
\moveto(346.27652685,269.39578961)
\lineto(593.89874975,269.39578961)
\lineto(593.89874975,221.77357052)
\lineto(346.27652685,221.77357052)
\closepath
}
}
{
\newrgbcolor{curcolor}{0 0 0}
\pscustom[linestyle=none,fillstyle=solid,fillcolor=curcolor]
{
\newpath
\moveto(135.15906759,246.57468354)
\lineto(137.76906759,246.57468354)
\curveto(138.14906022,246.57467184)(138.58905978,246.59467182)(139.08906759,246.63468354)
\curveto(139.60905876,246.67467174)(140.09905827,246.82467159)(140.55906759,247.08468354)
\curveto(141.01905735,247.34467107)(141.39905697,247.75467066)(141.69906759,248.31468354)
\curveto(142.01905635,248.87466954)(142.17905619,249.67466874)(142.17906759,250.71468354)
\curveto(142.17905619,251.79466662)(141.84905652,252.63466578)(141.18906759,253.23468354)
\curveto(140.52905784,253.83466458)(139.5690588,254.13466428)(138.30906759,254.13468354)
\lineto(135.15906759,254.13468354)
\lineto(135.15906759,246.57468354)
\moveto(132.51906759,256.29468354)
\lineto(139.44906759,256.29468354)
\curveto(141.14905722,256.29466212)(142.48905588,255.82466259)(143.46906759,254.88468354)
\curveto(144.44905392,253.94466447)(144.93905343,252.62466579)(144.93906759,250.92468354)
\curveto(144.93905343,250.34466807)(144.87905349,249.76466865)(144.75906759,249.18468354)
\curveto(144.65905371,248.60466981)(144.47905389,248.06467035)(144.21906759,247.56468354)
\curveto(143.95905441,247.08467133)(143.61905475,246.65467176)(143.19906759,246.27468354)
\curveto(142.77905559,245.9146725)(142.25905611,245.66467275)(141.63906759,245.52468354)
\lineto(141.63906759,245.46468354)
\curveto(142.57905579,245.36467305)(143.29905507,244.97467344)(143.79906759,244.29468354)
\curveto(144.31905405,243.6146748)(144.60905376,242.8146756)(144.66906759,241.89468354)
\lineto(144.84906759,238.23468354)
\curveto(144.8690535,237.63468078)(144.90905346,237.14468127)(144.96906759,236.76468354)
\curveto(145.02905334,236.38468203)(145.10905326,236.06468235)(145.20906759,235.80468354)
\curveto(145.30905306,235.56468285)(145.41905295,235.37468304)(145.53906759,235.23468354)
\curveto(145.67905269,235.09468332)(145.82905254,234.97468344)(145.98906759,234.87468354)
\lineto(142.80906759,234.87468354)
\curveto(142.68905568,234.99468342)(142.58905578,235.17468324)(142.50906759,235.41468354)
\curveto(142.42905594,235.65468276)(142.35905601,235.9146825)(142.29906759,236.19468354)
\curveto(142.23905613,236.49468192)(142.18905618,236.79468162)(142.14906759,237.09468354)
\curveto(142.12905624,237.414681)(142.10905626,237.70468071)(142.08906759,237.96468354)
\lineto(141.90906759,241.29468354)
\curveto(141.84905652,242.03467638)(141.70905666,242.60467581)(141.48906759,243.00468354)
\curveto(141.28905708,243.42467499)(141.03905733,243.73467468)(140.73906759,243.93468354)
\curveto(140.43905793,244.15467426)(140.10905826,244.28467413)(139.74906759,244.32468354)
\curveto(139.40905896,244.38467403)(139.0690593,244.414674)(138.72906759,244.41468354)
\lineto(135.15906759,244.41468354)
\lineto(135.15906759,234.87468354)
\lineto(132.51906759,234.87468354)
\lineto(132.51906759,256.29468354)
}
}
{
\newrgbcolor{curcolor}{0 0 0}
\pscustom[linestyle=none,fillstyle=solid,fillcolor=curcolor]
{
\newpath
\moveto(154.70500509,254.55468354)
\curveto(153.78499767,254.55466386)(153.0549984,254.3146641)(152.51500509,253.83468354)
\curveto(151.97499948,253.37466504)(151.5549999,252.73466568)(151.25500509,251.91468354)
\curveto(150.97500048,251.1146673)(150.78500067,250.16466825)(150.68500509,249.06468354)
\curveto(150.60500085,247.98467043)(150.56500089,246.82467159)(150.56500509,245.58468354)
\curveto(150.56500089,244.34467407)(150.60500085,243.17467524)(150.68500509,242.07468354)
\curveto(150.78500067,240.99467742)(150.97500048,240.04467837)(151.25500509,239.22468354)
\curveto(151.5549999,238.42467999)(151.97499948,237.78468063)(152.51500509,237.30468354)
\curveto(153.0549984,236.84468157)(153.78499767,236.6146818)(154.70500509,236.61468354)
\curveto(155.62499583,236.6146818)(156.3549951,236.84468157)(156.89500509,237.30468354)
\curveto(157.43499402,237.78468063)(157.84499361,238.42467999)(158.12500509,239.22468354)
\curveto(158.42499303,240.04467837)(158.61499284,240.99467742)(158.69500509,242.07468354)
\curveto(158.79499266,243.17467524)(158.84499261,244.34467407)(158.84500509,245.58468354)
\curveto(158.84499261,246.82467159)(158.79499266,247.98467043)(158.69500509,249.06468354)
\curveto(158.61499284,250.16466825)(158.42499303,251.1146673)(158.12500509,251.91468354)
\curveto(157.84499361,252.73466568)(157.43499402,253.37466504)(156.89500509,253.83468354)
\curveto(156.3549951,254.3146641)(155.62499583,254.55466386)(154.70500509,254.55468354)
\moveto(154.70500509,256.71468354)
\curveto(156.18499527,256.7146617)(157.37499408,256.38466203)(158.27500509,255.72468354)
\curveto(159.17499228,255.08466333)(159.87499158,254.23466418)(160.37500509,253.17468354)
\curveto(160.87499058,252.1146663)(161.20499025,250.9146675)(161.36500509,249.57468354)
\curveto(161.52498993,248.25467016)(161.60498985,246.92467149)(161.60500509,245.58468354)
\curveto(161.60498985,244.22467419)(161.52498993,242.88467553)(161.36500509,241.56468354)
\curveto(161.20499025,240.24467817)(160.87499058,239.05467936)(160.37500509,237.99468354)
\curveto(159.87499158,236.93468148)(159.17499228,236.07468234)(158.27500509,235.41468354)
\curveto(157.37499408,234.77468364)(156.18499527,234.45468396)(154.70500509,234.45468354)
\curveto(153.22499823,234.45468396)(152.03499942,234.77468364)(151.13500509,235.41468354)
\curveto(150.23500122,236.07468234)(149.53500192,236.93468148)(149.03500509,237.99468354)
\curveto(148.53500292,239.05467936)(148.20500325,240.24467817)(148.04500509,241.56468354)
\curveto(147.88500357,242.88467553)(147.80500365,244.22467419)(147.80500509,245.58468354)
\curveto(147.80500365,246.92467149)(147.88500357,248.25467016)(148.04500509,249.57468354)
\curveto(148.20500325,250.9146675)(148.53500292,252.1146663)(149.03500509,253.17468354)
\curveto(149.53500192,254.23466418)(150.23500122,255.08466333)(151.13500509,255.72468354)
\curveto(152.03499942,256.38466203)(153.22499823,256.7146617)(154.70500509,256.71468354)
}
}
{
\newrgbcolor{curcolor}{0 0 0}
\pscustom[linestyle=none,fillstyle=solid,fillcolor=curcolor]
{
\newpath
\moveto(164.56422384,256.29468354)
\lineto(167.20422384,256.29468354)
\lineto(167.20422384,241.23468354)
\curveto(167.2042197,239.65467876)(167.47421943,238.48467993)(168.01422384,237.72468354)
\curveto(168.57421833,236.98468143)(169.51421739,236.6146818)(170.83422384,236.61468354)
\curveto(172.23421467,236.6146818)(173.19421371,237.00468141)(173.71422384,237.78468354)
\curveto(174.23421267,238.58467983)(174.49421241,239.73467868)(174.49422384,241.23468354)
\lineto(174.49422384,256.29468354)
\lineto(177.13422384,256.29468354)
\lineto(177.13422384,241.23468354)
\curveto(177.13420977,239.17467924)(176.6042103,237.52468089)(175.54422384,236.28468354)
\curveto(174.5042124,235.06468335)(172.93421397,234.45468396)(170.83422384,234.45468354)
\curveto(168.65421825,234.45468396)(167.06421984,235.02468339)(166.06422384,236.16468354)
\curveto(165.06422184,237.30468111)(164.56422234,238.99467942)(164.56422384,241.23468354)
\lineto(164.56422384,256.29468354)
}
}
{
\newrgbcolor{curcolor}{0 0 0}
\pscustom[linestyle=none,fillstyle=solid,fillcolor=curcolor]
{
\newpath
\moveto(186.90016134,234.87468354)
\lineto(184.26016134,234.87468354)
\lineto(184.26016134,253.95468354)
\lineto(178.89016134,253.95468354)
\lineto(178.89016134,256.29468354)
\lineto(192.30016134,256.29468354)
\lineto(192.30016134,253.95468354)
\lineto(186.90016134,253.95468354)
\lineto(186.90016134,234.87468354)
}
}
{
\newrgbcolor{curcolor}{0 0 0}
\pscustom[linestyle=none,fillstyle=solid,fillcolor=curcolor]
{
\newpath
\moveto(194.27688009,256.29468354)
\lineto(205.34688009,256.29468354)
\lineto(205.34688009,253.95468354)
\lineto(196.91688009,253.95468354)
\lineto(196.91688009,247.23468354)
\lineto(204.86688009,247.23468354)
\lineto(204.86688009,244.89468354)
\lineto(196.91688009,244.89468354)
\lineto(196.91688009,237.21468354)
\lineto(205.70688009,237.21468354)
\lineto(205.70688009,234.87468354)
\lineto(194.27688009,234.87468354)
\lineto(194.27688009,256.29468354)
}
}
{
\newrgbcolor{curcolor}{0 0 0}
\pscustom[linewidth=1.84937644,linecolor=curcolor]
{
\newpath
\moveto(95.03764303,269.65999237)
\lineto(243.1882778,269.65999237)
\lineto(243.1882778,221.50936905)
\lineto(95.03764303,221.50936905)
\closepath
}
}
{
\newrgbcolor{curcolor}{0 0 0}
\pscustom[linestyle=none,fillstyle=solid,fillcolor=curcolor]
{
\newpath
\moveto(397.15931505,169.65084021)
\lineto(399.76931505,169.65084021)
\curveto(400.14930768,169.65082851)(400.58930724,169.67082849)(401.08931505,169.71084021)
\curveto(401.60930622,169.75082841)(402.09930573,169.90082826)(402.55931505,170.16084021)
\curveto(403.01930481,170.42082774)(403.39930443,170.83082733)(403.69931505,171.39084021)
\curveto(404.01930381,171.95082621)(404.17930365,172.75082541)(404.17931505,173.79084021)
\curveto(404.17930365,174.87082329)(403.84930398,175.71082245)(403.18931505,176.31084021)
\curveto(402.5293053,176.91082125)(401.56930626,177.21082095)(400.30931505,177.21084021)
\lineto(397.15931505,177.21084021)
\lineto(397.15931505,169.65084021)
\moveto(394.51931505,179.37084021)
\lineto(401.44931505,179.37084021)
\curveto(403.14930468,179.37081879)(404.48930334,178.90081926)(405.46931505,177.96084021)
\curveto(406.44930138,177.02082114)(406.93930089,175.70082246)(406.93931505,174.00084021)
\curveto(406.93930089,173.42082474)(406.87930095,172.84082532)(406.75931505,172.26084021)
\curveto(406.65930117,171.68082648)(406.47930135,171.14082702)(406.21931505,170.64084021)
\curveto(405.95930187,170.160828)(405.61930221,169.73082843)(405.19931505,169.35084021)
\curveto(404.77930305,168.99082917)(404.25930357,168.74082942)(403.63931505,168.60084021)
\lineto(403.63931505,168.54084021)
\curveto(404.57930325,168.44082972)(405.29930253,168.05083011)(405.79931505,167.37084021)
\curveto(406.31930151,166.69083147)(406.60930122,165.89083227)(406.66931505,164.97084021)
\lineto(406.84931505,161.31084021)
\curveto(406.86930096,160.71083745)(406.90930092,160.22083794)(406.96931505,159.84084021)
\curveto(407.0293008,159.4608387)(407.10930072,159.14083902)(407.20931505,158.88084021)
\curveto(407.30930052,158.64083952)(407.41930041,158.45083971)(407.53931505,158.31084021)
\curveto(407.67930015,158.17083999)(407.8293,158.05084011)(407.98931505,157.95084021)
\lineto(404.80931505,157.95084021)
\curveto(404.68930314,158.07084009)(404.58930324,158.25083991)(404.50931505,158.49084021)
\curveto(404.4293034,158.73083943)(404.35930347,158.99083917)(404.29931505,159.27084021)
\curveto(404.23930359,159.57083859)(404.18930364,159.87083829)(404.14931505,160.17084021)
\curveto(404.1293037,160.49083767)(404.10930372,160.78083738)(404.08931505,161.04084021)
\lineto(403.90931505,164.37084021)
\curveto(403.84930398,165.11083305)(403.70930412,165.68083248)(403.48931505,166.08084021)
\curveto(403.28930454,166.50083166)(403.03930479,166.81083135)(402.73931505,167.01084021)
\curveto(402.43930539,167.23083093)(402.10930572,167.3608308)(401.74931505,167.40084021)
\curveto(401.40930642,167.4608307)(401.06930676,167.49083067)(400.72931505,167.49084021)
\lineto(397.15931505,167.49084021)
\lineto(397.15931505,157.95084021)
\lineto(394.51931505,157.95084021)
\lineto(394.51931505,179.37084021)
}
}
{
\newrgbcolor{curcolor}{0 0 0}
\pscustom[linestyle=none,fillstyle=solid,fillcolor=curcolor]
{
\newpath
\moveto(416.70525255,177.63084021)
\curveto(415.78524513,177.63082053)(415.05524586,177.39082077)(414.51525255,176.91084021)
\curveto(413.97524694,176.45082171)(413.55524736,175.81082235)(413.25525255,174.99084021)
\curveto(412.97524794,174.19082397)(412.78524813,173.24082492)(412.68525255,172.14084021)
\curveto(412.60524831,171.0608271)(412.56524835,169.90082826)(412.56525255,168.66084021)
\curveto(412.56524835,167.42083074)(412.60524831,166.25083191)(412.68525255,165.15084021)
\curveto(412.78524813,164.07083409)(412.97524794,163.12083504)(413.25525255,162.30084021)
\curveto(413.55524736,161.50083666)(413.97524694,160.8608373)(414.51525255,160.38084021)
\curveto(415.05524586,159.92083824)(415.78524513,159.69083847)(416.70525255,159.69084021)
\curveto(417.62524329,159.69083847)(418.35524256,159.92083824)(418.89525255,160.38084021)
\curveto(419.43524148,160.8608373)(419.84524107,161.50083666)(420.12525255,162.30084021)
\curveto(420.42524049,163.12083504)(420.6152403,164.07083409)(420.69525255,165.15084021)
\curveto(420.79524012,166.25083191)(420.84524007,167.42083074)(420.84525255,168.66084021)
\curveto(420.84524007,169.90082826)(420.79524012,171.0608271)(420.69525255,172.14084021)
\curveto(420.6152403,173.24082492)(420.42524049,174.19082397)(420.12525255,174.99084021)
\curveto(419.84524107,175.81082235)(419.43524148,176.45082171)(418.89525255,176.91084021)
\curveto(418.35524256,177.39082077)(417.62524329,177.63082053)(416.70525255,177.63084021)
\moveto(416.70525255,179.79084021)
\curveto(418.18524273,179.79081837)(419.37524154,179.4608187)(420.27525255,178.80084021)
\curveto(421.17523974,178.16082)(421.87523904,177.31082085)(422.37525255,176.25084021)
\curveto(422.87523804,175.19082297)(423.20523771,173.99082417)(423.36525255,172.65084021)
\curveto(423.52523739,171.33082683)(423.60523731,170.00082816)(423.60525255,168.66084021)
\curveto(423.60523731,167.30083086)(423.52523739,165.9608322)(423.36525255,164.64084021)
\curveto(423.20523771,163.32083484)(422.87523804,162.13083603)(422.37525255,161.07084021)
\curveto(421.87523904,160.01083815)(421.17523974,159.15083901)(420.27525255,158.49084021)
\curveto(419.37524154,157.85084031)(418.18524273,157.53084063)(416.70525255,157.53084021)
\curveto(415.22524569,157.53084063)(414.03524688,157.85084031)(413.13525255,158.49084021)
\curveto(412.23524868,159.15083901)(411.53524938,160.01083815)(411.03525255,161.07084021)
\curveto(410.53525038,162.13083603)(410.20525071,163.32083484)(410.04525255,164.64084021)
\curveto(409.88525103,165.9608322)(409.80525111,167.30083086)(409.80525255,168.66084021)
\curveto(409.80525111,170.00082816)(409.88525103,171.33082683)(410.04525255,172.65084021)
\curveto(410.20525071,173.99082417)(410.53525038,175.19082297)(411.03525255,176.25084021)
\curveto(411.53524938,177.31082085)(412.23524868,178.16082)(413.13525255,178.80084021)
\curveto(414.03524688,179.4608187)(415.22524569,179.79081837)(416.70525255,179.79084021)
}
}
{
\newrgbcolor{curcolor}{0 0 0}
\pscustom[linestyle=none,fillstyle=solid,fillcolor=curcolor]
{
\newpath
\moveto(426.5644713,179.37084021)
\lineto(429.2044713,179.37084021)
\lineto(429.2044713,164.31084021)
\curveto(429.20446716,162.73083543)(429.47446689,161.5608366)(430.0144713,160.80084021)
\curveto(430.57446579,160.0608381)(431.51446485,159.69083847)(432.8344713,159.69084021)
\curveto(434.23446213,159.69083847)(435.19446117,160.08083808)(435.7144713,160.86084021)
\curveto(436.23446013,161.6608365)(436.49445987,162.81083535)(436.4944713,164.31084021)
\lineto(436.4944713,179.37084021)
\lineto(439.1344713,179.37084021)
\lineto(439.1344713,164.31084021)
\curveto(439.13445723,162.25083591)(438.60445776,160.60083756)(437.5444713,159.36084021)
\curveto(436.50445986,158.14084002)(434.93446143,157.53084063)(432.8344713,157.53084021)
\curveto(430.65446571,157.53084063)(429.0644673,158.10084006)(428.0644713,159.24084021)
\curveto(427.0644693,160.38083778)(426.5644698,162.07083609)(426.5644713,164.31084021)
\lineto(426.5644713,179.37084021)
}
}
{
\newrgbcolor{curcolor}{0 0 0}
\pscustom[linestyle=none,fillstyle=solid,fillcolor=curcolor]
{
\newpath
\moveto(448.9004088,157.95084021)
\lineto(446.2604088,157.95084021)
\lineto(446.2604088,177.03084021)
\lineto(440.8904088,177.03084021)
\lineto(440.8904088,179.37084021)
\lineto(454.3004088,179.37084021)
\lineto(454.3004088,177.03084021)
\lineto(448.9004088,177.03084021)
\lineto(448.9004088,157.95084021)
}
}
{
\newrgbcolor{curcolor}{0 0 0}
\pscustom[linestyle=none,fillstyle=solid,fillcolor=curcolor]
{
\newpath
\moveto(456.27712755,179.37084021)
\lineto(467.34712755,179.37084021)
\lineto(467.34712755,177.03084021)
\lineto(458.91712755,177.03084021)
\lineto(458.91712755,170.31084021)
\lineto(466.86712755,170.31084021)
\lineto(466.86712755,167.97084021)
\lineto(458.91712755,167.97084021)
\lineto(458.91712755,160.29084021)
\lineto(467.70712755,160.29084021)
\lineto(467.70712755,157.95084021)
\lineto(456.27712755,157.95084021)
\lineto(456.27712755,179.37084021)
}
}
{
\newrgbcolor{curcolor}{0 0 0}
\pscustom[linestyle=none,fillstyle=solid,fillcolor=curcolor]
{
\newpath
\moveto(468.4238463,155.70084021)
\lineto(483.4238463,155.70084021)
\lineto(483.4238463,154.20084021)
\lineto(468.4238463,154.20084021)
\lineto(468.4238463,155.70084021)
}
}
{
\newrgbcolor{curcolor}{0 0 0}
\pscustom[linestyle=none,fillstyle=solid,fillcolor=curcolor]
{
\newpath
\moveto(485.2238463,179.37084021)
\lineto(489.6638463,179.37084021)
\lineto(493.9538463,162.39084021)
\lineto(494.0138463,162.39084021)
\lineto(498.3038463,179.37084021)
\lineto(502.7438463,179.37084021)
\lineto(502.7438463,157.95084021)
\lineto(500.1038463,157.95084021)
\lineto(500.1038463,176.67084021)
\lineto(500.0438463,176.67084021)
\lineto(495.3038463,157.95084021)
\lineto(492.6638463,157.95084021)
\lineto(487.9238463,176.67084021)
\lineto(487.8638463,176.67084021)
\lineto(487.8638463,157.95084021)
\lineto(485.2238463,157.95084021)
\lineto(485.2238463,179.37084021)
}
}
{
\newrgbcolor{curcolor}{0 0 0}
\pscustom[linestyle=none,fillstyle=solid,fillcolor=curcolor]
{
\newpath
\moveto(506.2575963,179.37084021)
\lineto(517.3275963,179.37084021)
\lineto(517.3275963,177.03084021)
\lineto(508.8975963,177.03084021)
\lineto(508.8975963,170.31084021)
\lineto(516.8475963,170.31084021)
\lineto(516.8475963,167.97084021)
\lineto(508.8975963,167.97084021)
\lineto(508.8975963,160.29084021)
\lineto(517.6875963,160.29084021)
\lineto(517.6875963,157.95084021)
\lineto(506.2575963,157.95084021)
\lineto(506.2575963,179.37084021)
}
}
{
\newrgbcolor{curcolor}{0 0 0}
\pscustom[linestyle=none,fillstyle=solid,fillcolor=curcolor]
{
\newpath
\moveto(520.14431505,179.37084021)
\lineto(522.75431505,179.37084021)
\lineto(523.56431505,179.37084021)
\lineto(530.64431505,161.49084021)
\lineto(530.70431505,161.49084021)
\lineto(530.70431505,179.37084021)
\lineto(533.34431505,179.37084021)
\lineto(533.34431505,157.95084021)
\lineto(530.73431505,157.95084021)
\lineto(529.71431505,157.95084021)
\lineto(522.84431505,175.29084021)
\lineto(522.78431505,175.29084021)
\lineto(522.78431505,157.95084021)
\lineto(520.14431505,157.95084021)
\lineto(520.14431505,179.37084021)
}
}
{
\newrgbcolor{curcolor}{0 0 0}
\pscustom[linestyle=none,fillstyle=solid,fillcolor=curcolor]
{
\newpath
\moveto(536.6035338,179.37084021)
\lineto(539.2435338,179.37084021)
\lineto(539.2435338,164.31084021)
\curveto(539.24352966,162.73083543)(539.51352939,161.5608366)(540.0535338,160.80084021)
\curveto(540.61352829,160.0608381)(541.55352735,159.69083847)(542.8735338,159.69084021)
\curveto(544.27352463,159.69083847)(545.23352367,160.08083808)(545.7535338,160.86084021)
\curveto(546.27352263,161.6608365)(546.53352237,162.81083535)(546.5335338,164.31084021)
\lineto(546.5335338,179.37084021)
\lineto(549.1735338,179.37084021)
\lineto(549.1735338,164.31084021)
\curveto(549.17351973,162.25083591)(548.64352026,160.60083756)(547.5835338,159.36084021)
\curveto(546.54352236,158.14084002)(544.97352393,157.53084063)(542.8735338,157.53084021)
\curveto(540.69352821,157.53084063)(539.1035298,158.10084006)(538.1035338,159.24084021)
\curveto(537.1035318,160.38083778)(536.6035323,162.07083609)(536.6035338,164.31084021)
\lineto(536.6035338,179.37084021)
}
}
{
\newrgbcolor{curcolor}{0 0 0}
\pscustom[linewidth=2.37778306,linecolor=curcolor]
{
\newpath
\moveto(348.03532701,190.80691932)
\lineto(595.65754991,190.80691932)
\lineto(595.65754991,143.18470405)
\lineto(348.03532701,143.18470405)
\closepath
}
}
{
\newrgbcolor{curcolor}{0 0 0}
\pscustom[linestyle=none,fillstyle=solid,fillcolor=curcolor]
{
\newpath
\moveto(382.82284273,96.89941627)
\lineto(385.46284273,96.89941627)
\lineto(388.73284273,79.37941627)
\lineto(388.79284273,79.37941627)
\lineto(391.82284273,96.89941627)
\lineto(395.00284273,96.89941627)
\lineto(398.03284273,79.37941627)
\lineto(398.09284273,79.37941627)
\lineto(401.36284273,96.89941627)
\lineto(404.00284273,96.89941627)
\lineto(399.59284273,75.47941627)
\lineto(396.38284273,75.47941627)
\lineto(393.44284273,92.81941627)
\lineto(393.38284273,92.81941627)
\lineto(390.29284273,75.47941627)
\lineto(387.08284273,75.47941627)
\lineto(382.82284273,96.89941627)
}
}
{
\newrgbcolor{curcolor}{0 0 0}
\pscustom[linestyle=none,fillstyle=solid,fillcolor=curcolor]
{
\newpath
\moveto(406.00253023,96.89941627)
\lineto(408.64253023,96.89941627)
\lineto(408.64253023,75.47941627)
\lineto(406.00253023,75.47941627)
\lineto(406.00253023,96.89941627)
}
}
{
\newrgbcolor{curcolor}{0 0 0}
\pscustom[linestyle=none,fillstyle=solid,fillcolor=curcolor]
{
\newpath
\moveto(412.36628023,96.89941627)
\lineto(418.09628023,96.89941627)
\curveto(419.75627083,96.89939485)(421.07626951,96.61939513)(422.05628023,96.05941627)
\curveto(423.05626753,95.49939625)(423.81626677,94.72939702)(424.33628023,93.74941627)
\curveto(424.85626573,92.78939896)(425.19626539,91.65940009)(425.35628023,90.35941627)
\curveto(425.51626507,89.05940269)(425.59626499,87.66940408)(425.59628023,86.18941627)
\curveto(425.59626499,84.82940692)(425.49626509,83.50940824)(425.29628023,82.22941627)
\curveto(425.09626549,80.9494108)(424.72626586,79.80941194)(424.18628023,78.80941627)
\curveto(423.64626694,77.80941394)(422.90626768,76.99941475)(421.96628023,76.37941627)
\curveto(421.02626956,75.77941597)(419.82627076,75.47941627)(418.36628023,75.47941627)
\lineto(412.36628023,75.47941627)
\lineto(412.36628023,96.89941627)
\moveto(415.00628023,77.63941627)
\lineto(417.76628023,77.63941627)
\curveto(418.90627168,77.63941411)(419.80627078,77.89941385)(420.46628023,78.41941627)
\curveto(421.14626944,78.95941279)(421.65626893,79.63941211)(421.99628023,80.45941627)
\curveto(422.35626823,81.27941047)(422.586268,82.18940956)(422.68628023,83.18941627)
\curveto(422.7862678,84.20940754)(422.83626775,85.19940655)(422.83628023,86.15941627)
\curveto(422.83626775,87.19940455)(422.79626779,88.22940352)(422.71628023,89.24941627)
\curveto(422.63626795,90.26940148)(422.42626816,91.17940057)(422.08628023,91.97941627)
\curveto(421.76626882,92.79939895)(421.26626932,93.45939829)(420.58628023,93.95941627)
\curveto(419.90627068,94.47939727)(418.96627162,94.73939701)(417.76628023,94.73941627)
\lineto(415.00628023,94.73941627)
\lineto(415.00628023,77.63941627)
}
}
{
\newrgbcolor{curcolor}{0 0 0}
\pscustom[linestyle=none,fillstyle=solid,fillcolor=curcolor]
{
\newpath
\moveto(438.96549898,90.95941627)
\curveto(438.92548711,91.51940023)(438.8354872,92.0493997)(438.69549898,92.54941627)
\curveto(438.57548746,93.06939868)(438.37548766,93.51939823)(438.09549898,93.89941627)
\curveto(437.8354882,94.27939747)(437.48548855,94.57939717)(437.04549898,94.79941627)
\curveto(436.60548943,95.03939671)(436.05548998,95.15939659)(435.39549898,95.15941627)
\curveto(434.47549156,95.15939659)(433.74549229,94.91939683)(433.20549898,94.43941627)
\curveto(432.66549337,93.97939777)(432.24549379,93.33939841)(431.94549898,92.51941627)
\curveto(431.66549437,91.71940003)(431.47549456,90.76940098)(431.37549898,89.66941627)
\curveto(431.29549474,88.58940316)(431.25549478,87.42940432)(431.25549898,86.18941627)
\curveto(431.25549478,84.9494068)(431.29549474,83.77940797)(431.37549898,82.67941627)
\curveto(431.47549456,81.59941015)(431.66549437,80.6494111)(431.94549898,79.82941627)
\curveto(432.24549379,79.02941272)(432.66549337,78.38941336)(433.20549898,77.90941627)
\curveto(433.74549229,77.4494143)(434.47549156,77.21941453)(435.39549898,77.21941627)
\curveto(436.31548972,77.21941453)(437.035489,77.45941429)(437.55549898,77.93941627)
\curveto(438.07548796,78.43941331)(438.46548757,79.0494127)(438.72549898,79.76941627)
\curveto(438.98548705,80.48941126)(439.14548689,81.26941048)(439.20549898,82.10941627)
\curveto(439.28548675,82.9494088)(439.32548671,83.72940802)(439.32549898,84.44941627)
\lineto(435.09549898,84.44941627)
\lineto(435.09549898,86.60941627)
\lineto(441.72549898,86.60941627)
\lineto(441.72549898,75.47941627)
\lineto(439.74549898,75.47941627)
\lineto(439.74549898,78.38941627)
\lineto(439.68549898,78.38941627)
\curveto(439.40548663,77.46941428)(438.86548717,76.67941507)(438.06549898,76.01941627)
\curveto(437.26548877,75.37941637)(436.26548977,75.05941669)(435.06549898,75.05941627)
\curveto(433.66549237,75.05941669)(432.5354935,75.36941638)(431.67549898,75.98941627)
\curveto(430.81549522,76.60941514)(430.14549589,77.42941432)(429.66549898,78.44941627)
\curveto(429.20549683,79.48941226)(428.89549714,80.67941107)(428.73549898,82.01941627)
\curveto(428.57549746,83.35940839)(428.49549754,84.749407)(428.49549898,86.18941627)
\curveto(428.49549754,87.52940422)(428.57549746,88.85940289)(428.73549898,90.17941627)
\curveto(428.89549714,91.51940023)(429.22549681,92.71939903)(429.72549898,93.77941627)
\curveto(430.22549581,94.83939691)(430.92549511,95.68939606)(431.82549898,96.32941627)
\curveto(432.72549331,96.98939476)(433.91549212,97.31939443)(435.39549898,97.31941627)
\curveto(436.41548962,97.31939443)(437.27548876,97.18939456)(437.97549898,96.92941627)
\curveto(438.69548734,96.66939508)(439.28548675,96.32939542)(439.74549898,95.90941627)
\curveto(440.22548581,95.50939624)(440.59548544,95.05939669)(440.85549898,94.55941627)
\curveto(441.11548492,94.05939769)(441.30548473,93.56939818)(441.42549898,93.08941627)
\curveto(441.56548447,92.62939912)(441.64548439,92.19939955)(441.66549898,91.79941627)
\curveto(441.70548433,91.41940033)(441.72548431,91.13940061)(441.72549898,90.95941627)
\lineto(438.96549898,90.95941627)
}
}
{
\newrgbcolor{curcolor}{0 0 0}
\pscustom[linestyle=none,fillstyle=solid,fillcolor=curcolor]
{
\newpath
\moveto(444.90878023,96.89941627)
\lineto(455.97878023,96.89941627)
\lineto(455.97878023,94.55941627)
\lineto(447.54878023,94.55941627)
\lineto(447.54878023,87.83941627)
\lineto(455.49878023,87.83941627)
\lineto(455.49878023,85.49941627)
\lineto(447.54878023,85.49941627)
\lineto(447.54878023,77.81941627)
\lineto(456.33878023,77.81941627)
\lineto(456.33878023,75.47941627)
\lineto(444.90878023,75.47941627)
\lineto(444.90878023,96.89941627)
}
}
{
\newrgbcolor{curcolor}{0 0 0}
\pscustom[linestyle=none,fillstyle=solid,fillcolor=curcolor]
{
\newpath
\moveto(465.30549898,75.47941627)
\lineto(462.66549898,75.47941627)
\lineto(462.66549898,94.55941627)
\lineto(457.29549898,94.55941627)
\lineto(457.29549898,96.89941627)
\lineto(470.70549898,96.89941627)
\lineto(470.70549898,94.55941627)
\lineto(465.30549898,94.55941627)
\lineto(465.30549898,75.47941627)
}
}
{
\newrgbcolor{curcolor}{0 0 0}
\pscustom[linestyle=none,fillstyle=solid,fillcolor=curcolor]
{
\newpath
\moveto(470.94221773,73.22941627)
\lineto(485.94221773,73.22941627)
\lineto(485.94221773,71.72941627)
\lineto(470.94221773,71.72941627)
\lineto(470.94221773,73.22941627)
}
}
{
\newrgbcolor{curcolor}{0 0 0}
\pscustom[linestyle=none,fillstyle=solid,fillcolor=curcolor]
{
\newpath
\moveto(490.32221773,87.17941627)
\lineto(492.93221773,87.17941627)
\curveto(493.31221036,87.17940457)(493.75220992,87.19940455)(494.25221773,87.23941627)
\curveto(494.7722089,87.27940447)(495.26220841,87.42940432)(495.72221773,87.68941627)
\curveto(496.18220749,87.9494038)(496.56220711,88.35940339)(496.86221773,88.91941627)
\curveto(497.18220649,89.47940227)(497.34220633,90.27940147)(497.34221773,91.31941627)
\curveto(497.34220633,92.39939935)(497.01220666,93.23939851)(496.35221773,93.83941627)
\curveto(495.69220798,94.43939731)(494.73220894,94.73939701)(493.47221773,94.73941627)
\lineto(490.32221773,94.73941627)
\lineto(490.32221773,87.17941627)
\moveto(487.68221773,96.89941627)
\lineto(494.61221773,96.89941627)
\curveto(496.31220736,96.89939485)(497.65220602,96.42939532)(498.63221773,95.48941627)
\curveto(499.61220406,94.5493972)(500.10220357,93.22939852)(500.10221773,91.52941627)
\curveto(500.10220357,90.9494008)(500.04220363,90.36940138)(499.92221773,89.78941627)
\curveto(499.82220385,89.20940254)(499.64220403,88.66940308)(499.38221773,88.16941627)
\curveto(499.12220455,87.68940406)(498.78220489,87.25940449)(498.36221773,86.87941627)
\curveto(497.94220573,86.51940523)(497.42220625,86.26940548)(496.80221773,86.12941627)
\lineto(496.80221773,86.06941627)
\curveto(497.74220593,85.96940578)(498.46220521,85.57940617)(498.96221773,84.89941627)
\curveto(499.48220419,84.21940753)(499.7722039,83.41940833)(499.83221773,82.49941627)
\lineto(500.01221773,78.83941627)
\curveto(500.03220364,78.23941351)(500.0722036,77.749414)(500.13221773,77.36941627)
\curveto(500.19220348,76.98941476)(500.2722034,76.66941508)(500.37221773,76.40941627)
\curveto(500.4722032,76.16941558)(500.58220309,75.97941577)(500.70221773,75.83941627)
\curveto(500.84220283,75.69941605)(500.99220268,75.57941617)(501.15221773,75.47941627)
\lineto(497.97221773,75.47941627)
\curveto(497.85220582,75.59941615)(497.75220592,75.77941597)(497.67221773,76.01941627)
\curveto(497.59220608,76.25941549)(497.52220615,76.51941523)(497.46221773,76.79941627)
\curveto(497.40220627,77.09941465)(497.35220632,77.39941435)(497.31221773,77.69941627)
\curveto(497.29220638,78.01941373)(497.2722064,78.30941344)(497.25221773,78.56941627)
\lineto(497.07221773,81.89941627)
\curveto(497.01220666,82.63940911)(496.8722068,83.20940854)(496.65221773,83.60941627)
\curveto(496.45220722,84.02940772)(496.20220747,84.33940741)(495.90221773,84.53941627)
\curveto(495.60220807,84.75940699)(495.2722084,84.88940686)(494.91221773,84.92941627)
\curveto(494.5722091,84.98940676)(494.23220944,85.01940673)(493.89221773,85.01941627)
\lineto(490.32221773,85.01941627)
\lineto(490.32221773,75.47941627)
\lineto(487.68221773,75.47941627)
\lineto(487.68221773,96.89941627)
}
}
{
\newrgbcolor{curcolor}{0 0 0}
\pscustom[linestyle=none,fillstyle=solid,fillcolor=curcolor]
{
\newpath
\moveto(509.86815523,95.15941627)
\curveto(508.94814781,95.15939659)(508.21814854,94.91939683)(507.67815523,94.43941627)
\curveto(507.13814962,93.97939777)(506.71815004,93.33939841)(506.41815523,92.51941627)
\curveto(506.13815062,91.71940003)(505.94815081,90.76940098)(505.84815523,89.66941627)
\curveto(505.76815099,88.58940316)(505.72815103,87.42940432)(505.72815523,86.18941627)
\curveto(505.72815103,84.9494068)(505.76815099,83.77940797)(505.84815523,82.67941627)
\curveto(505.94815081,81.59941015)(506.13815062,80.6494111)(506.41815523,79.82941627)
\curveto(506.71815004,79.02941272)(507.13814962,78.38941336)(507.67815523,77.90941627)
\curveto(508.21814854,77.4494143)(508.94814781,77.21941453)(509.86815523,77.21941627)
\curveto(510.78814597,77.21941453)(511.51814524,77.4494143)(512.05815523,77.90941627)
\curveto(512.59814416,78.38941336)(513.00814375,79.02941272)(513.28815523,79.82941627)
\curveto(513.58814317,80.6494111)(513.77814298,81.59941015)(513.85815523,82.67941627)
\curveto(513.9581428,83.77940797)(514.00814275,84.9494068)(514.00815523,86.18941627)
\curveto(514.00814275,87.42940432)(513.9581428,88.58940316)(513.85815523,89.66941627)
\curveto(513.77814298,90.76940098)(513.58814317,91.71940003)(513.28815523,92.51941627)
\curveto(513.00814375,93.33939841)(512.59814416,93.97939777)(512.05815523,94.43941627)
\curveto(511.51814524,94.91939683)(510.78814597,95.15939659)(509.86815523,95.15941627)
\moveto(509.86815523,97.31941627)
\curveto(511.34814541,97.31939443)(512.53814422,96.98939476)(513.43815523,96.32941627)
\curveto(514.33814242,95.68939606)(515.03814172,94.83939691)(515.53815523,93.77941627)
\curveto(516.03814072,92.71939903)(516.36814039,91.51940023)(516.52815523,90.17941627)
\curveto(516.68814007,88.85940289)(516.76813999,87.52940422)(516.76815523,86.18941627)
\curveto(516.76813999,84.82940692)(516.68814007,83.48940826)(516.52815523,82.16941627)
\curveto(516.36814039,80.8494109)(516.03814072,79.65941209)(515.53815523,78.59941627)
\curveto(515.03814172,77.53941421)(514.33814242,76.67941507)(513.43815523,76.01941627)
\curveto(512.53814422,75.37941637)(511.34814541,75.05941669)(509.86815523,75.05941627)
\curveto(508.38814837,75.05941669)(507.19814956,75.37941637)(506.29815523,76.01941627)
\curveto(505.39815136,76.67941507)(504.69815206,77.53941421)(504.19815523,78.59941627)
\curveto(503.69815306,79.65941209)(503.36815339,80.8494109)(503.20815523,82.16941627)
\curveto(503.04815371,83.48940826)(502.96815379,84.82940692)(502.96815523,86.18941627)
\curveto(502.96815379,87.52940422)(503.04815371,88.85940289)(503.20815523,90.17941627)
\curveto(503.36815339,91.51940023)(503.69815306,92.71939903)(504.19815523,93.77941627)
\curveto(504.69815206,94.83939691)(505.39815136,95.68939606)(506.29815523,96.32941627)
\curveto(507.19814956,96.98939476)(508.38814837,97.31939443)(509.86815523,97.31941627)
}
}
{
\newrgbcolor{curcolor}{0 0 0}
\pscustom[linestyle=none,fillstyle=solid,fillcolor=curcolor]
{
\newpath
\moveto(519.72737398,96.89941627)
\lineto(522.36737398,96.89941627)
\lineto(522.36737398,81.83941627)
\curveto(522.36736984,80.25941149)(522.63736957,79.08941266)(523.17737398,78.32941627)
\curveto(523.73736847,77.58941416)(524.67736753,77.21941453)(525.99737398,77.21941627)
\curveto(527.39736481,77.21941453)(528.35736385,77.60941414)(528.87737398,78.38941627)
\curveto(529.39736281,79.18941256)(529.65736255,80.33941141)(529.65737398,81.83941627)
\lineto(529.65737398,96.89941627)
\lineto(532.29737398,96.89941627)
\lineto(532.29737398,81.83941627)
\curveto(532.29735991,79.77941197)(531.76736044,78.12941362)(530.70737398,76.88941627)
\curveto(529.66736254,75.66941608)(528.09736411,75.05941669)(525.99737398,75.05941627)
\curveto(523.81736839,75.05941669)(522.22736998,75.62941612)(521.22737398,76.76941627)
\curveto(520.22737198,77.90941384)(519.72737248,79.59941215)(519.72737398,81.83941627)
\lineto(519.72737398,96.89941627)
}
}
{
\newrgbcolor{curcolor}{0 0 0}
\pscustom[linestyle=none,fillstyle=solid,fillcolor=curcolor]
{
\newpath
\moveto(542.06331148,75.47941627)
\lineto(539.42331148,75.47941627)
\lineto(539.42331148,94.55941627)
\lineto(534.05331148,94.55941627)
\lineto(534.05331148,96.89941627)
\lineto(547.46331148,96.89941627)
\lineto(547.46331148,94.55941627)
\lineto(542.06331148,94.55941627)
\lineto(542.06331148,75.47941627)
}
}
{
\newrgbcolor{curcolor}{0 0 0}
\pscustom[linestyle=none,fillstyle=solid,fillcolor=curcolor]
{
\newpath
\moveto(549.44003023,96.89941627)
\lineto(560.51003023,96.89941627)
\lineto(560.51003023,94.55941627)
\lineto(552.08003023,94.55941627)
\lineto(552.08003023,87.83941627)
\lineto(560.03003023,87.83941627)
\lineto(560.03003023,85.49941627)
\lineto(552.08003023,85.49941627)
\lineto(552.08003023,77.81941627)
\lineto(560.87003023,77.81941627)
\lineto(560.87003023,75.47941627)
\lineto(549.44003023,75.47941627)
\lineto(549.44003023,96.89941627)
}
}
{
\newrgbcolor{curcolor}{0 0 0}
\pscustom[linewidth=2.37778306,linecolor=curcolor]
{
\newpath
\moveto(348.03532687,108.33549537)
\lineto(595.65754977,108.33549537)
\lineto(595.65754977,60.7132801)
\lineto(348.03532687,60.7132801)
\closepath
}
}
{
\newrgbcolor{curcolor}{0 0 0}
\pscustom[linestyle=none,fillstyle=solid,fillcolor=curcolor]
{
\newpath
\moveto(125.17163234,95.23435559)
\lineto(127.81163234,95.23435559)
\lineto(131.08163234,77.71435559)
\lineto(131.14163234,77.71435559)
\lineto(134.17163234,95.23435559)
\lineto(137.35163234,95.23435559)
\lineto(140.38163234,77.71435559)
\lineto(140.44163234,77.71435559)
\lineto(143.71163234,95.23435559)
\lineto(146.35163234,95.23435559)
\lineto(141.94163234,73.81435559)
\lineto(138.73163234,73.81435559)
\lineto(135.79163234,91.15435559)
\lineto(135.73163234,91.15435559)
\lineto(132.64163234,73.81435559)
\lineto(129.43163234,73.81435559)
\lineto(125.17163234,95.23435559)
}
}
{
\newrgbcolor{curcolor}{0 0 0}
\pscustom[linestyle=none,fillstyle=solid,fillcolor=curcolor]
{
\newpath
\moveto(148.35131984,95.23435559)
\lineto(150.99131984,95.23435559)
\lineto(150.99131984,73.81435559)
\lineto(148.35131984,73.81435559)
\lineto(148.35131984,95.23435559)
}
}
{
\newrgbcolor{curcolor}{0 0 0}
\pscustom[linestyle=none,fillstyle=solid,fillcolor=curcolor]
{
\newpath
\moveto(154.71506984,95.23435559)
\lineto(160.44506984,95.23435559)
\curveto(162.10506044,95.23433417)(163.42505912,94.95433445)(164.40506984,94.39435559)
\curveto(165.40505714,93.83433557)(166.16505638,93.06433634)(166.68506984,92.08435559)
\curveto(167.20505534,91.12433828)(167.545055,89.99433941)(167.70506984,88.69435559)
\curveto(167.86505468,87.39434201)(167.9450546,86.0043434)(167.94506984,84.52435559)
\curveto(167.9450546,83.16434624)(167.8450547,81.84434756)(167.64506984,80.56435559)
\curveto(167.4450551,79.28435012)(167.07505547,78.14435126)(166.53506984,77.14435559)
\curveto(165.99505655,76.14435326)(165.25505729,75.33435407)(164.31506984,74.71435559)
\curveto(163.37505917,74.11435529)(162.17506037,73.81435559)(160.71506984,73.81435559)
\lineto(154.71506984,73.81435559)
\lineto(154.71506984,95.23435559)
\moveto(157.35506984,75.97435559)
\lineto(160.11506984,75.97435559)
\curveto(161.25506129,75.97435343)(162.15506039,76.23435317)(162.81506984,76.75435559)
\curveto(163.49505905,77.29435211)(164.00505854,77.97435143)(164.34506984,78.79435559)
\curveto(164.70505784,79.61434979)(164.93505761,80.52434888)(165.03506984,81.52435559)
\curveto(165.13505741,82.54434686)(165.18505736,83.53434587)(165.18506984,84.49435559)
\curveto(165.18505736,85.53434387)(165.1450574,86.56434284)(165.06506984,87.58435559)
\curveto(164.98505756,88.6043408)(164.77505777,89.51433989)(164.43506984,90.31435559)
\curveto(164.11505843,91.13433827)(163.61505893,91.79433761)(162.93506984,92.29435559)
\curveto(162.25506029,92.81433659)(161.31506123,93.07433633)(160.11506984,93.07435559)
\lineto(157.35506984,93.07435559)
\lineto(157.35506984,75.97435559)
}
}
{
\newrgbcolor{curcolor}{0 0 0}
\pscustom[linestyle=none,fillstyle=solid,fillcolor=curcolor]
{
\newpath
\moveto(181.31428859,89.29435559)
\curveto(181.27427672,89.85433955)(181.18427681,90.38433902)(181.04428859,90.88435559)
\curveto(180.92427707,91.404338)(180.72427727,91.85433755)(180.44428859,92.23435559)
\curveto(180.18427781,92.61433679)(179.83427816,92.91433649)(179.39428859,93.13435559)
\curveto(178.95427904,93.37433603)(178.40427959,93.49433591)(177.74428859,93.49435559)
\curveto(176.82428117,93.49433591)(176.0942819,93.25433615)(175.55428859,92.77435559)
\curveto(175.01428298,92.31433709)(174.5942834,91.67433773)(174.29428859,90.85435559)
\curveto(174.01428398,90.05433935)(173.82428417,89.1043403)(173.72428859,88.00435559)
\curveto(173.64428435,86.92434248)(173.60428439,85.76434364)(173.60428859,84.52435559)
\curveto(173.60428439,83.28434612)(173.64428435,82.11434729)(173.72428859,81.01435559)
\curveto(173.82428417,79.93434947)(174.01428398,78.98435042)(174.29428859,78.16435559)
\curveto(174.5942834,77.36435204)(175.01428298,76.72435268)(175.55428859,76.24435559)
\curveto(176.0942819,75.78435362)(176.82428117,75.55435385)(177.74428859,75.55435559)
\curveto(178.66427933,75.55435385)(179.38427861,75.79435361)(179.90428859,76.27435559)
\curveto(180.42427757,76.77435263)(180.81427718,77.38435202)(181.07428859,78.10435559)
\curveto(181.33427666,78.82435058)(181.4942765,79.6043498)(181.55428859,80.44435559)
\curveto(181.63427636,81.28434812)(181.67427632,82.06434734)(181.67428859,82.78435559)
\lineto(177.44428859,82.78435559)
\lineto(177.44428859,84.94435559)
\lineto(184.07428859,84.94435559)
\lineto(184.07428859,73.81435559)
\lineto(182.09428859,73.81435559)
\lineto(182.09428859,76.72435559)
\lineto(182.03428859,76.72435559)
\curveto(181.75427624,75.8043536)(181.21427678,75.01435439)(180.41428859,74.35435559)
\curveto(179.61427838,73.71435569)(178.61427938,73.39435601)(177.41428859,73.39435559)
\curveto(176.01428198,73.39435601)(174.88428311,73.7043557)(174.02428859,74.32435559)
\curveto(173.16428483,74.94435446)(172.4942855,75.76435364)(172.01428859,76.78435559)
\curveto(171.55428644,77.82435158)(171.24428675,79.01435039)(171.08428859,80.35435559)
\curveto(170.92428707,81.69434771)(170.84428715,83.08434632)(170.84428859,84.52435559)
\curveto(170.84428715,85.86434354)(170.92428707,87.19434221)(171.08428859,88.51435559)
\curveto(171.24428675,89.85433955)(171.57428642,91.05433835)(172.07428859,92.11435559)
\curveto(172.57428542,93.17433623)(173.27428472,94.02433538)(174.17428859,94.66435559)
\curveto(175.07428292,95.32433408)(176.26428173,95.65433375)(177.74428859,95.65435559)
\curveto(178.76427923,95.65433375)(179.62427837,95.52433388)(180.32428859,95.26435559)
\curveto(181.04427695,95.0043344)(181.63427636,94.66433474)(182.09428859,94.24435559)
\curveto(182.57427542,93.84433556)(182.94427505,93.39433601)(183.20428859,92.89435559)
\curveto(183.46427453,92.39433701)(183.65427434,91.9043375)(183.77428859,91.42435559)
\curveto(183.91427408,90.96433844)(183.994274,90.53433887)(184.01428859,90.13435559)
\curveto(184.05427394,89.75433965)(184.07427392,89.47433993)(184.07428859,89.29435559)
\lineto(181.31428859,89.29435559)
}
}
{
\newrgbcolor{curcolor}{0 0 0}
\pscustom[linestyle=none,fillstyle=solid,fillcolor=curcolor]
{
\newpath
\moveto(187.25756984,95.23435559)
\lineto(198.32756984,95.23435559)
\lineto(198.32756984,92.89435559)
\lineto(189.89756984,92.89435559)
\lineto(189.89756984,86.17435559)
\lineto(197.84756984,86.17435559)
\lineto(197.84756984,83.83435559)
\lineto(189.89756984,83.83435559)
\lineto(189.89756984,76.15435559)
\lineto(198.68756984,76.15435559)
\lineto(198.68756984,73.81435559)
\lineto(187.25756984,73.81435559)
\lineto(187.25756984,95.23435559)
}
}
{
\newrgbcolor{curcolor}{0 0 0}
\pscustom[linestyle=none,fillstyle=solid,fillcolor=curcolor]
{
\newpath
\moveto(207.65428859,73.81435559)
\lineto(205.01428859,73.81435559)
\lineto(205.01428859,92.89435559)
\lineto(199.64428859,92.89435559)
\lineto(199.64428859,95.23435559)
\lineto(213.05428859,95.23435559)
\lineto(213.05428859,92.89435559)
\lineto(207.65428859,92.89435559)
\lineto(207.65428859,73.81435559)
}
}
{
\newrgbcolor{curcolor}{0 0 0}
\pscustom[linewidth=1.84937644,linecolor=curcolor]
{
\newpath
\moveto(95.03764278,108.59969494)
\lineto(243.18827754,108.59969494)
\lineto(243.18827754,60.44907162)
\lineto(95.03764278,60.44907162)
\closepath
}
}
{
\newrgbcolor{curcolor}{0 0 0}
\pscustom[linewidth=2,linecolor=curcolor]
{
\newpath
\moveto(209.843326,376.71835)
\lineto(209.843326,387.0755)
}
}
{
\newrgbcolor{curcolor}{1 1 1}
\pscustom[linestyle=none,fillstyle=solid,fillcolor=curcolor]
{
\newpath
\moveto(202.527406,381.81691)
\lineto(345.655026,381.81691)
}
}
{
\newrgbcolor{curcolor}{0 0 0}
\pscustom[linewidth=2,linecolor=curcolor]
{
\newpath
\moveto(202.527406,381.81691)
\lineto(345.655026,381.81691)
}
}
{
\newrgbcolor{curcolor}{1 1 1}
\pscustom[linestyle=none,fillstyle=solid,fillcolor=curcolor]
{
\newpath
\moveto(337.646279,376.57861)
\lineto(329.123117,381.75564)
\lineto(337.646279,386.93267)
\closepath
}
}
{
\newrgbcolor{curcolor}{0 0 0}
\pscustom[linewidth=2,linecolor=curcolor]
{
\newpath
\moveto(337.646279,376.57861)
\lineto(329.123117,381.75564)
\lineto(337.646279,386.93267)
\closepath
}
}
{
\newrgbcolor{curcolor}{1 1 1}
\pscustom[linestyle=none,fillstyle=solid,fillcolor=curcolor]
{
\newpath
\moveto(326.73273481,384.69090758)
\curveto(328.40931357,384.69090758)(329.7684492,383.33177196)(329.7684492,381.6551932)
\curveto(329.7684492,379.97861443)(328.40931357,378.61947881)(326.73273481,378.61947881)
\curveto(325.05615605,378.61947881)(323.69702043,379.97861443)(323.69702043,381.6551932)
\curveto(323.69702043,383.33177196)(325.05615605,384.69090758)(326.73273481,384.69090758)
\closepath
}
}
{
\newrgbcolor{curcolor}{0 0 0}
\pscustom[linewidth=2,linecolor=curcolor]
{
\newpath
\moveto(326.73273481,384.69090758)
\curveto(328.40931357,384.69090758)(329.7684492,383.33177196)(329.7684492,381.6551932)
\curveto(329.7684492,379.97861443)(328.40931357,378.61947881)(326.73273481,378.61947881)
\curveto(325.05615605,378.61947881)(323.69702043,379.97861443)(323.69702043,381.6551932)
\curveto(323.69702043,383.33177196)(325.05615605,384.69090758)(326.73273481,384.69090758)
\closepath
}
}
{
\newrgbcolor{curcolor}{1 1 1}
\pscustom[linestyle=none,fillstyle=solid,fillcolor=curcolor]
{
\newpath
\moveto(337.714872,385.14226)
\lineto(345.0363,385.14226)
}
}
{
\newrgbcolor{curcolor}{0 0 0}
\pscustom[linewidth=2,linecolor=curcolor]
{
\newpath
\moveto(337.714872,385.14226)
\lineto(345.0363,385.14226)
}
}
{
\newrgbcolor{curcolor}{1 1 1}
\pscustom[linestyle=none,fillstyle=solid,fillcolor=curcolor]
{
\newpath
\moveto(337.741024,378.34673)
\lineto(345.330309,378.34673)
}
}
{
\newrgbcolor{curcolor}{0 0 0}
\pscustom[linewidth=2,linecolor=curcolor]
{
\newpath
\moveto(337.741024,378.34673)
\lineto(345.330309,378.34673)
}
}
{
\newrgbcolor{curcolor}{0 0 0}
\pscustom[linewidth=2,linecolor=curcolor]
{
\newpath
\moveto(62.739852,406.140582)
\lineto(62.739852,439.475616)
\lineto(181.937856,439.475616)
\lineto(181.937856,405.509237)
}
}
{
\newrgbcolor{curcolor}{1 1 1}
\pscustom[linestyle=none,fillstyle=solid,fillcolor=curcolor]
{
\newpath
\moveto(65.73982038,413.87047971)
\curveto(65.73982038,412.19390095)(64.38068476,410.83476532)(62.704106,410.83476532)
\curveto(61.02752723,410.83476532)(59.66839161,412.19390095)(59.66839161,413.87047971)
\curveto(59.66839161,415.54705847)(61.02752723,416.90619409)(62.704106,416.90619409)
\curveto(64.38068476,416.90619409)(65.73982038,415.54705847)(65.73982038,413.87047971)
\closepath
}
}
{
\newrgbcolor{curcolor}{0 0 0}
\pscustom[linewidth=2,linecolor=curcolor]
{
\newpath
\moveto(65.73982038,413.87047971)
\curveto(65.73982038,412.19390095)(64.38068476,410.83476532)(62.704106,410.83476532)
\curveto(61.02752723,410.83476532)(59.66839161,412.19390095)(59.66839161,413.87047971)
\curveto(59.66839161,415.54705847)(61.02752723,416.90619409)(62.704106,416.90619409)
\curveto(64.38068476,416.90619409)(65.73982038,415.54705847)(65.73982038,413.87047971)
\closepath
}
}
{
\newrgbcolor{curcolor}{1 1 1}
\pscustom[linestyle=none,fillstyle=solid,fillcolor=curcolor]
{
\newpath
\moveto(184.78565358,417.05368069)
\curveto(184.78565358,415.37710193)(183.42651796,414.0179663)(181.7499392,414.0179663)
\curveto(180.07336043,414.0179663)(178.71422481,415.37710193)(178.71422481,417.05368069)
\curveto(178.71422481,418.73025945)(180.07336043,420.08939507)(181.7499392,420.08939507)
\curveto(183.42651796,420.08939507)(184.78565358,418.73025945)(184.78565358,417.05368069)
\closepath
}
}
{
\newrgbcolor{curcolor}{0 0 0}
\pscustom[linewidth=2,linecolor=curcolor]
{
\newpath
\moveto(184.78565358,417.05368069)
\curveto(184.78565358,415.37710193)(183.42651796,414.0179663)(181.7499392,414.0179663)
\curveto(180.07336043,414.0179663)(178.71422481,415.37710193)(178.71422481,417.05368069)
\curveto(178.71422481,418.73025945)(180.07336043,420.08939507)(181.7499392,420.08939507)
\curveto(183.42651796,420.08939507)(184.78565358,418.73025945)(184.78565358,417.05368069)
\closepath
}
}
{
\newrgbcolor{curcolor}{0 0 0}
\pscustom[linewidth=2,linecolor=curcolor]
{
\newpath
\moveto(176.625206,405.531035)
\lineto(181.802236,414.054208)
\lineto(186.979266,405.531035)
}
}
{
\newrgbcolor{curcolor}{0 0 0}
\pscustom[linewidth=2,linecolor=curcolor]
{
\newpath
\moveto(107.739852,271.082162)
\lineto(107.739852,304.417196)
\lineto(226.937856,304.417196)
\lineto(226.937856,270.450817)
}
}
{
\newrgbcolor{curcolor}{1 1 1}
\pscustom[linestyle=none,fillstyle=solid,fillcolor=curcolor]
{
\newpath
\moveto(110.73982038,278.81205971)
\curveto(110.73982038,277.13548095)(109.38068476,275.77634532)(107.704106,275.77634532)
\curveto(106.02752723,275.77634532)(104.66839161,277.13548095)(104.66839161,278.81205971)
\curveto(104.66839161,280.48863847)(106.02752723,281.84777409)(107.704106,281.84777409)
\curveto(109.38068476,281.84777409)(110.73982038,280.48863847)(110.73982038,278.81205971)
\closepath
}
}
{
\newrgbcolor{curcolor}{0 0 0}
\pscustom[linewidth=2,linecolor=curcolor]
{
\newpath
\moveto(110.73982038,278.81205971)
\curveto(110.73982038,277.13548095)(109.38068476,275.77634532)(107.704106,275.77634532)
\curveto(106.02752723,275.77634532)(104.66839161,277.13548095)(104.66839161,278.81205971)
\curveto(104.66839161,280.48863847)(106.02752723,281.84777409)(107.704106,281.84777409)
\curveto(109.38068476,281.84777409)(110.73982038,280.48863847)(110.73982038,278.81205971)
\closepath
}
}
{
\newrgbcolor{curcolor}{1 1 1}
\pscustom[linestyle=none,fillstyle=solid,fillcolor=curcolor]
{
\newpath
\moveto(229.78565358,281.99526069)
\curveto(229.78565358,280.31868193)(228.42651796,278.9595463)(226.7499392,278.9595463)
\curveto(225.07336043,278.9595463)(223.71422481,280.31868193)(223.71422481,281.99526069)
\curveto(223.71422481,283.67183945)(225.07336043,285.03097507)(226.7499392,285.03097507)
\curveto(228.42651796,285.03097507)(229.78565358,283.67183945)(229.78565358,281.99526069)
\closepath
}
}
{
\newrgbcolor{curcolor}{0 0 0}
\pscustom[linewidth=2,linecolor=curcolor]
{
\newpath
\moveto(229.78565358,281.99526069)
\curveto(229.78565358,280.31868193)(228.42651796,278.9595463)(226.7499392,278.9595463)
\curveto(225.07336043,278.9595463)(223.71422481,280.31868193)(223.71422481,281.99526069)
\curveto(223.71422481,283.67183945)(225.07336043,285.03097507)(226.7499392,285.03097507)
\curveto(228.42651796,285.03097507)(229.78565358,283.67183945)(229.78565358,281.99526069)
\closepath
}
}
{
\newrgbcolor{curcolor}{0 0 0}
\pscustom[linewidth=2,linecolor=curcolor]
{
\newpath
\moveto(221.625206,270.472615)
\lineto(226.802236,278.995788)
\lineto(231.979266,270.472615)
}
}
{
\newrgbcolor{curcolor}{0 0 0}
\pscustom[linewidth=2,linecolor=curcolor]
{
\newpath
\moveto(63.954116,357.88834)
\lineto(63.954116,245.0312)
\lineto(93.954116,245.0312)
}
}
{
\newrgbcolor{curcolor}{0 0 0}
\pscustom[linewidth=2,linecolor=curcolor]
{
\newpath
\moveto(58.239823,349.13834)
\lineto(68.596986,349.13834)
}
}
{
\newrgbcolor{curcolor}{1 1 1}
\pscustom[linestyle=none,fillstyle=solid,fillcolor=curcolor]
{
\newpath
\moveto(86.623371,239.95547)
\lineto(78.100209,245.1325)
\lineto(86.623371,250.30953)
\closepath
}
}
{
\newrgbcolor{curcolor}{0 0 0}
\pscustom[linewidth=2,linecolor=curcolor]
{
\newpath
\moveto(86.623371,239.95547)
\lineto(78.100209,245.1325)
\lineto(86.623371,250.30953)
\closepath
}
}
{
\newrgbcolor{curcolor}{1 1 1}
\pscustom[linestyle=none,fillstyle=solid,fillcolor=curcolor]
{
\newpath
\moveto(75.70982681,248.06776758)
\curveto(77.38640557,248.06776758)(78.7455412,246.70863196)(78.7455412,245.0320532)
\curveto(78.7455412,243.35547443)(77.38640557,241.99633881)(75.70982681,241.99633881)
\curveto(74.03324805,241.99633881)(72.67411243,243.35547443)(72.67411243,245.0320532)
\curveto(72.67411243,246.70863196)(74.03324805,248.06776758)(75.70982681,248.06776758)
\closepath
}
}
{
\newrgbcolor{curcolor}{0 0 0}
\pscustom[linewidth=2,linecolor=curcolor]
{
\newpath
\moveto(75.70982681,248.06776758)
\curveto(77.38640557,248.06776758)(78.7455412,246.70863196)(78.7455412,245.0320532)
\curveto(78.7455412,243.35547443)(77.38640557,241.99633881)(75.70982681,241.99633881)
\curveto(74.03324805,241.99633881)(72.67411243,243.35547443)(72.67411243,245.0320532)
\curveto(72.67411243,246.70863196)(74.03324805,248.06776758)(75.70982681,248.06776758)
\closepath
}
}
{
\newrgbcolor{curcolor}{1 1 1}
\pscustom[linestyle=none,fillstyle=solid,fillcolor=curcolor]
{
\newpath
\moveto(86.691964,248.51912)
\lineto(94.013392,248.51912)
}
}
{
\newrgbcolor{curcolor}{0 0 0}
\pscustom[linewidth=2,linecolor=curcolor]
{
\newpath
\moveto(86.691964,248.51912)
\lineto(94.013392,248.51912)
}
}
{
\newrgbcolor{curcolor}{1 1 1}
\pscustom[linestyle=none,fillstyle=solid,fillcolor=curcolor]
{
\newpath
\moveto(86.718116,241.72359)
\lineto(94.307401,241.72359)
}
}
{
\newrgbcolor{curcolor}{0 0 0}
\pscustom[linewidth=2,linecolor=curcolor]
{
\newpath
\moveto(86.718116,241.72359)
\lineto(94.307401,241.72359)
}
}
{
\newrgbcolor{curcolor}{0 0 0}
\pscustom[linewidth=2,linecolor=curcolor]
{
\newpath
\moveto(243.557156,244.0111)
\lineto(345.582566,244.0111)
}
}
{
\newrgbcolor{curcolor}{0 0 0}
\pscustom[linewidth=2,linecolor=curcolor]
{
\newpath
\moveto(412.252636,356.64048)
\lineto(412.252636,270.27506)
}
}
{
\newrgbcolor{curcolor}{1 1 1}
\pscustom[linestyle=none,fillstyle=solid,fillcolor=curcolor]
{
\newpath
\moveto(337.646289,238.94531)
\lineto(329.123127,244.12234)
\lineto(337.646289,249.29937)
\closepath
}
}
{
\newrgbcolor{curcolor}{0 0 0}
\pscustom[linewidth=2,linecolor=curcolor]
{
\newpath
\moveto(337.646289,238.94531)
\lineto(329.123127,244.12234)
\lineto(337.646289,249.29937)
\closepath
}
}
{
\newrgbcolor{curcolor}{1 1 1}
\pscustom[linestyle=none,fillstyle=solid,fillcolor=curcolor]
{
\newpath
\moveto(326.73274481,247.05760758)
\curveto(328.40932357,247.05760758)(329.7684592,245.69847196)(329.7684592,244.0218932)
\curveto(329.7684592,242.34531443)(328.40932357,240.98617881)(326.73274481,240.98617881)
\curveto(325.05616605,240.98617881)(323.69703043,242.34531443)(323.69703043,244.0218932)
\curveto(323.69703043,245.69847196)(325.05616605,247.05760758)(326.73274481,247.05760758)
\closepath
}
}
{
\newrgbcolor{curcolor}{0 0 0}
\pscustom[linewidth=2,linecolor=curcolor]
{
\newpath
\moveto(326.73274481,247.05760758)
\curveto(328.40932357,247.05760758)(329.7684592,245.69847196)(329.7684592,244.0218932)
\curveto(329.7684592,242.34531443)(328.40932357,240.98617881)(326.73274481,240.98617881)
\curveto(325.05616605,240.98617881)(323.69703043,242.34531443)(323.69703043,244.0218932)
\curveto(323.69703043,245.69847196)(325.05616605,247.05760758)(326.73274481,247.05760758)
\closepath
}
}
{
\newrgbcolor{curcolor}{1 1 1}
\pscustom[linestyle=none,fillstyle=solid,fillcolor=curcolor]
{
\newpath
\moveto(337.714882,247.50896)
\lineto(345.03631,247.50896)
}
}
{
\newrgbcolor{curcolor}{0 0 0}
\pscustom[linewidth=2,linecolor=curcolor]
{
\newpath
\moveto(337.714882,247.50896)
\lineto(345.03631,247.50896)
}
}
{
\newrgbcolor{curcolor}{1 1 1}
\pscustom[linestyle=none,fillstyle=solid,fillcolor=curcolor]
{
\newpath
\moveto(337.741034,240.71343)
\lineto(345.330319,240.71343)
}
}
{
\newrgbcolor{curcolor}{0 0 0}
\pscustom[linewidth=2,linecolor=curcolor]
{
\newpath
\moveto(337.741034,240.71343)
\lineto(345.330319,240.71343)
}
}
{
\newrgbcolor{curcolor}{0 0 0}
\pscustom[linewidth=2,linecolor=curcolor]
{
\newpath
\moveto(250.249426,239.3376)
\lineto(250.249426,249.69475)
}
}
{
\newrgbcolor{curcolor}{0 0 0}
\pscustom[linewidth=2,linecolor=curcolor]
{
\newpath
\moveto(406.742476,349.13834)
\lineto(417.099636,349.13834)
}
}
{
\newrgbcolor{curcolor}{1 1 1}
\pscustom[linestyle=none,fillstyle=solid,fillcolor=curcolor]
{
\newpath
\moveto(407.118376,278.112817)
\lineto(412.295406,286.635979)
\lineto(417.472436,278.112817)
\closepath
}
}
{
\newrgbcolor{curcolor}{0 0 0}
\pscustom[linewidth=2,linecolor=curcolor]
{
\newpath
\moveto(407.118376,278.112817)
\lineto(412.295406,286.635979)
\lineto(417.472436,278.112817)
\closepath
}
}
{
\newrgbcolor{curcolor}{1 1 1}
\pscustom[linestyle=none,fillstyle=solid,fillcolor=curcolor]
{
\newpath
\moveto(415.23067358,289.02636119)
\curveto(415.23067358,287.34978243)(413.87153796,285.9906468)(412.1949592,285.9906468)
\curveto(410.51838043,285.9906468)(409.15924481,287.34978243)(409.15924481,289.02636119)
\curveto(409.15924481,290.70293995)(410.51838043,292.06207557)(412.1949592,292.06207557)
\curveto(413.87153796,292.06207557)(415.23067358,290.70293995)(415.23067358,289.02636119)
\closepath
}
}
{
\newrgbcolor{curcolor}{0 0 0}
\pscustom[linewidth=2,linecolor=curcolor]
{
\newpath
\moveto(415.23067358,289.02636119)
\curveto(415.23067358,287.34978243)(413.87153796,285.9906468)(412.1949592,285.9906468)
\curveto(410.51838043,285.9906468)(409.15924481,287.34978243)(409.15924481,289.02636119)
\curveto(409.15924481,290.70293995)(410.51838043,292.06207557)(412.1949592,292.06207557)
\curveto(413.87153796,292.06207557)(415.23067358,290.70293995)(415.23067358,289.02636119)
\closepath
}
}
{
\newrgbcolor{curcolor}{1 1 1}
\pscustom[linestyle=none,fillstyle=solid,fillcolor=curcolor]
{
\newpath
\moveto(415.682026,278.044224)
\lineto(415.682026,270.722796)
}
}
{
\newrgbcolor{curcolor}{0 0 0}
\pscustom[linewidth=2,linecolor=curcolor]
{
\newpath
\moveto(415.682026,278.044224)
\lineto(415.682026,270.722796)
}
}
{
\newrgbcolor{curcolor}{1 1 1}
\pscustom[linestyle=none,fillstyle=solid,fillcolor=curcolor]
{
\newpath
\moveto(408.886496,278.018072)
\lineto(408.886496,270.428787)
}
}
{
\newrgbcolor{curcolor}{0 0 0}
\pscustom[linewidth=2,linecolor=curcolor]
{
\newpath
\moveto(408.886496,278.018072)
\lineto(408.886496,270.428787)
}
}
{
\newrgbcolor{curcolor}{0 0 0}
\pscustom[linewidth=2,linecolor=curcolor]
{
\newpath
\moveto(220.527756,220.50759)
\lineto(220.527756,166.49184)
\lineto(347.754936,166.49184)
}
}
{
\newrgbcolor{curcolor}{0 0 0}
\pscustom[linewidth=2,linecolor=curcolor]
{
\newpath
\moveto(214.813466,211.75759)
\lineto(225.170626,211.75759)
}
}
{
\newrgbcolor{curcolor}{1 1 1}
\pscustom[linestyle=none,fillstyle=solid,fillcolor=curcolor]
{
\newpath
\moveto(339.919109,161.41611)
\lineto(331.395947,166.59314)
\lineto(339.919109,171.77017)
\closepath
}
}
{
\newrgbcolor{curcolor}{0 0 0}
\pscustom[linewidth=2,linecolor=curcolor]
{
\newpath
\moveto(339.919109,161.41611)
\lineto(331.395947,166.59314)
\lineto(339.919109,171.77017)
\closepath
}
}
{
\newrgbcolor{curcolor}{1 1 1}
\pscustom[linestyle=none,fillstyle=solid,fillcolor=curcolor]
{
\newpath
\moveto(329.00556481,169.52840758)
\curveto(330.68214357,169.52840758)(332.0412792,168.16927196)(332.0412792,166.4926932)
\curveto(332.0412792,164.81611443)(330.68214357,163.45697881)(329.00556481,163.45697881)
\curveto(327.32898605,163.45697881)(325.96985043,164.81611443)(325.96985043,166.4926932)
\curveto(325.96985043,168.16927196)(327.32898605,169.52840758)(329.00556481,169.52840758)
\closepath
}
}
{
\newrgbcolor{curcolor}{0 0 0}
\pscustom[linewidth=2,linecolor=curcolor]
{
\newpath
\moveto(329.00556481,169.52840758)
\curveto(330.68214357,169.52840758)(332.0412792,168.16927196)(332.0412792,166.4926932)
\curveto(332.0412792,164.81611443)(330.68214357,163.45697881)(329.00556481,163.45697881)
\curveto(327.32898605,163.45697881)(325.96985043,164.81611443)(325.96985043,166.4926932)
\curveto(325.96985043,168.16927196)(327.32898605,169.52840758)(329.00556481,169.52840758)
\closepath
}
}
{
\newrgbcolor{curcolor}{1 1 1}
\pscustom[linestyle=none,fillstyle=solid,fillcolor=curcolor]
{
\newpath
\moveto(339.987702,169.97976)
\lineto(347.30913,169.97976)
}
}
{
\newrgbcolor{curcolor}{0 0 0}
\pscustom[linewidth=2,linecolor=curcolor]
{
\newpath
\moveto(339.987702,169.97976)
\lineto(347.30913,169.97976)
}
}
{
\newrgbcolor{curcolor}{1 1 1}
\pscustom[linestyle=none,fillstyle=solid,fillcolor=curcolor]
{
\newpath
\moveto(340.013854,163.18423)
\lineto(347.603139,163.18423)
}
}
{
\newrgbcolor{curcolor}{0 0 0}
\pscustom[linewidth=2,linecolor=curcolor]
{
\newpath
\moveto(340.013854,163.18423)
\lineto(347.603139,163.18423)
}
}
{
\newrgbcolor{curcolor}{0 0 0}
\pscustom[linewidth=2,linecolor=curcolor]
{
\newpath
\moveto(245.0723848,83.29681)
\lineto(347.0977948,83.29681)
}
}
{
\newrgbcolor{curcolor}{1 1 1}
\pscustom[linestyle=none,fillstyle=solid,fillcolor=curcolor]
{
\newpath
\moveto(339.1615178,78.23102)
\lineto(330.6383558,83.40805)
\lineto(339.1615178,88.58508)
\closepath
}
}
{
\newrgbcolor{curcolor}{0 0 0}
\pscustom[linewidth=2,linecolor=curcolor]
{
\newpath
\moveto(339.1615178,78.23102)
\lineto(330.6383558,83.40805)
\lineto(339.1615178,88.58508)
\closepath
}
}
{
\newrgbcolor{curcolor}{1 1 1}
\pscustom[linestyle=none,fillstyle=solid,fillcolor=curcolor]
{
\newpath
\moveto(328.24797361,86.34331758)
\curveto(329.92455237,86.34331758)(331.283688,84.98418196)(331.283688,83.3076032)
\curveto(331.283688,81.63102443)(329.92455237,80.27188881)(328.24797361,80.27188881)
\curveto(326.57139485,80.27188881)(325.21225923,81.63102443)(325.21225923,83.3076032)
\curveto(325.21225923,84.98418196)(326.57139485,86.34331758)(328.24797361,86.34331758)
\closepath
}
}
{
\newrgbcolor{curcolor}{0 0 0}
\pscustom[linewidth=2,linecolor=curcolor]
{
\newpath
\moveto(328.24797361,86.34331758)
\curveto(329.92455237,86.34331758)(331.283688,84.98418196)(331.283688,83.3076032)
\curveto(331.283688,81.63102443)(329.92455237,80.27188881)(328.24797361,80.27188881)
\curveto(326.57139485,80.27188881)(325.21225923,81.63102443)(325.21225923,83.3076032)
\curveto(325.21225923,84.98418196)(326.57139485,86.34331758)(328.24797361,86.34331758)
\closepath
}
}
{
\newrgbcolor{curcolor}{1 1 1}
\pscustom[linestyle=none,fillstyle=solid,fillcolor=curcolor]
{
\newpath
\moveto(339.2301108,86.79467)
\lineto(346.5515388,86.79467)
}
}
{
\newrgbcolor{curcolor}{0 0 0}
\pscustom[linewidth=2,linecolor=curcolor]
{
\newpath
\moveto(339.2301108,86.79467)
\lineto(346.5515388,86.79467)
}
}
{
\newrgbcolor{curcolor}{1 1 1}
\pscustom[linestyle=none,fillstyle=solid,fillcolor=curcolor]
{
\newpath
\moveto(339.2562628,79.99914)
\lineto(346.8455478,79.99914)
}
}
{
\newrgbcolor{curcolor}{0 0 0}
\pscustom[linewidth=2,linecolor=curcolor]
{
\newpath
\moveto(339.2562628,79.99914)
\lineto(346.8455478,79.99914)
}
}
{
\newrgbcolor{curcolor}{0 0 0}
\pscustom[linewidth=2,linecolor=curcolor]
{
\newpath
\moveto(251.7646548,78.62331)
\lineto(251.7646548,88.98046)
}
}
{
\newrgbcolor{curcolor}{0 0 0}
\pscustom[linewidth=2,linecolor=curcolor]
{
\newpath
\moveto(63.954116,357.88834)
\lineto(63.954116,84.16441)
\lineto(93.954116,84.16441)
}
}
{
\newrgbcolor{curcolor}{0 0 0}
\pscustom[linewidth=2,linecolor=curcolor]
{
\newpath
\moveto(58.239823,349.13834)
\lineto(68.596986,349.13834)
}
}
{
\newrgbcolor{curcolor}{1 1 1}
\pscustom[linestyle=none,fillstyle=solid,fillcolor=curcolor]
{
\newpath
\moveto(86.623371,79.08868)
\lineto(78.100209,84.26571)
\lineto(86.623371,89.44274)
\closepath
}
}
{
\newrgbcolor{curcolor}{0 0 0}
\pscustom[linewidth=2,linecolor=curcolor]
{
\newpath
\moveto(86.623371,79.08868)
\lineto(78.100209,84.26571)
\lineto(86.623371,89.44274)
\closepath
}
}
{
\newrgbcolor{curcolor}{1 1 1}
\pscustom[linestyle=none,fillstyle=solid,fillcolor=curcolor]
{
\newpath
\moveto(75.70982681,87.20097758)
\curveto(77.38640557,87.20097758)(78.7455412,85.84184196)(78.7455412,84.1652632)
\curveto(78.7455412,82.48868443)(77.38640557,81.12954881)(75.70982681,81.12954881)
\curveto(74.03324805,81.12954881)(72.67411243,82.48868443)(72.67411243,84.1652632)
\curveto(72.67411243,85.84184196)(74.03324805,87.20097758)(75.70982681,87.20097758)
\closepath
}
}
{
\newrgbcolor{curcolor}{0 0 0}
\pscustom[linewidth=2,linecolor=curcolor]
{
\newpath
\moveto(75.70982681,87.20097758)
\curveto(77.38640557,87.20097758)(78.7455412,85.84184196)(78.7455412,84.1652632)
\curveto(78.7455412,82.48868443)(77.38640557,81.12954881)(75.70982681,81.12954881)
\curveto(74.03324805,81.12954881)(72.67411243,82.48868443)(72.67411243,84.1652632)
\curveto(72.67411243,85.84184196)(74.03324805,87.20097758)(75.70982681,87.20097758)
\closepath
}
}
{
\newrgbcolor{curcolor}{1 1 1}
\pscustom[linestyle=none,fillstyle=solid,fillcolor=curcolor]
{
\newpath
\moveto(86.691964,87.65233)
\lineto(94.013392,87.65233)
}
}
{
\newrgbcolor{curcolor}{0 0 0}
\pscustom[linewidth=2,linecolor=curcolor]
{
\newpath
\moveto(86.691964,87.65233)
\lineto(94.013392,87.65233)
}
}
{
\newrgbcolor{curcolor}{1 1 1}
\pscustom[linestyle=none,fillstyle=solid,fillcolor=curcolor]
{
\newpath
\moveto(86.718116,80.8568)
\lineto(94.307401,80.8568)
}
}
{
\newrgbcolor{curcolor}{0 0 0}
\pscustom[linewidth=2,linecolor=curcolor]
{
\newpath
\moveto(86.718116,80.8568)
\lineto(94.307401,80.8568)
}
}
\end{pspicture}

\caption{Modelo Entidad-Relación de la base funcional del sistema.}
\label{modelo1}
\end{figure}

En la figura (\ref{modelo1}), se presenta una parte del modelo entidad-relación
del sistema, que comprende a las entidades tratadas en esta sección. Puede
notarse la gran cantidad de relaciones de entidad débil, que será muy
característico a lo largo de todo el sistema.

\subsection{\emph{packages}: Manejador de paquetes}
Las principales funciones de este paquete son:

\begin{itemize}
\item Instalación de paquetes en el sistema.
\item Manejo de dependencias entre paquetes del sistema.
\item Establecimiento de rutas de acceso para un paquete determinado.
\end{itemize}

\subsection{\emph{privileges}: Manejador de privilegios}
Las principales funciones de este paquete son:

\begin{itemize}
\item Registro de privilegios reservados por cada paquete.
\item Control de acceso a recursos y acciones especificas.
\end{itemize}

\subsection{\emph{routes}: Manejador de rutas de navegación}
Las principales funciones de este paquete son:

\begin{itemize}
\item Registro de rutas reservadas por cada paquete.
\end{itemize}

\subsection{Creación de un paquete del sistema}
Después de construidas las funcionalidades antes mencionadas se llego a una
definición precisa en la construcción de funcionalidad nueva, ahora detallaremos
tal proceso a partir del ultimo paquete construido en el sistema, el paquete de
sugerencias.

Puede verse en el ejemplo, que este registra cuatro diferentes tipos de
privilegios; cuatro diferentes tipos de rutas de acceso, aquí puede notarse que
una ruta no necesariamente atiende exclusivamente a la petición GET, sino a
cualquier tipo de petición que sea necesario realizar (ya sea POST, PUT, DELETE,
etc); además puede verse el establecimiento de privilegios para el acceso a las
rutas.

Este proceso consiste en los siguientes pasos:

\begin{figure}
\centering
\begin{SQL}
CREATE TABLE `feedback` (
    `resource`    int unsigned NOT NULL,
    `description` text         NOT NULL,
    `resolved`    boolean      NOT NULL DEFAULT FALSE,
    `mark`        boolean      NOT NULL DEFAULT FALSE,
    PRIMARY KEY (`resource`),
    INDEX (`resource`),
    FOREIGN KEY (`resource`) REFERENCES `resource`(`ident`)
        ON UPDATE CASCADE ON DELETE RESTRICT
) DEFAULT CHARACTER SET UTF8;
\end{SQL}
\caption{Definición de datos para el modulo de sugerencias}
\label{code1}
\end{figure}

\begin{figure}
\centering
\begin{SQL}
INSERT INTO `package` (
    `label`, `url`, `type`, `parent`,
    `tsregister`, `description`)
VALUES
    ('feedback', 'feedback', 'app', 'notes',
    UNIX_TIMESTAMP(),
   'Modulo de registro de sugerencias del sistema');
\end{SQL}
\caption{Inserción del paquete en el registro}
\label{code2}
\end{figure}

\begin{figure}
\centering
\begin{SQL}
INSERT INTO `privilege` (
    `description`, `package`, `label`)
VALUES
    ('Ver sugerencias',                 'feedback', 'list'),
    ('Marcar sugerencias solucionadas', 'feedback', 'resolv'),
    ('Marcar sugerencias interesantes', 'feedback', 'mark'),
    ('Eliminar sugerencias inutiles',   'feedback', 'delete');
\end{SQL}
\caption{Inserciones en el registro de privilegios}
\label{code3}
\end{figure}

\begin{figure}
\centering
\begin{SQL}
INSERT INTO `route` (
    `label`,
    `type`, `parent`, `route`,
    `mapping`,
    `module`, `controller`, `action`)
VALUES
    ('Lista de sugerencias',
     'list', '', 'feedback_list',
     'feedback',
     'feedback', 'index',   'index'),
    ('Administrador de sugerencias',
     'list', '', 'feedback_manager',
     'feedback/manager',
     'feedback', 'manager', 'index'),
    ('Nueva sugerencia',
     'view', '', 'feedback_new',
     'feedback/new',
     'feedback', 'manager', 'new'),
    ('Sugerencia: $entry',
     'view', '', 'feedback_entry_view',
     'feedback/:entry',
     'feedback', 'entry', 'view'),
    ('Editar: $entry',
     'view', '', 'feedback_entry_edit',
     'feedback/:entry/edit',
     'feedback', 'entry', 'edit'),
    ('', 'action', '', 'feedback_entry_resolv',
     'feedback/:entry/resolv',
     'feedback', 'entry', 'resolv'),
    ('', 'action', '', 'feedback_entry_unresolv',
     'feedback/:entry/unresolv',
     'feedback', 'entry', 'unresolv'),
    ('', 'action', '', 'feedback_entry_mark',
     'feedback/:entry/mark',
     'feedback', 'entry', 'mark'),
    ('', 'action', '', 'feedback_entry_unmark',
     'feedback/:entry/unmark',
     'feedback', 'entry', 'unmark'),
    ('', 'action', '', 'feedback_entry_delete',
     'feedback/:entry/delete',
     'feedback', 'entry', 'delete'),
    ('', 'action', '', 'feedback_entry_drop',
     'feedback/:entry/drop',
     'feedback', 'entry', 'drop');
\end{SQL}
\caption{Inserciones en el registro de rutas}
\label{code4}
\end{figure}

\begin{figure}
\centering
\begin{SQL}
INSERT INTO `route_privilege`
(`route`, `package`, `privilege`)
VALUES
('feedback_list',    'feedback', 'list'),
('feedback_manager', 'feedback', 'resolv'),
('feedback_manager', 'feedback', 'mark'),
('feedback_manager', 'feedback', 'delete');
\end{SQL}
\caption{Inserciones en el registro de rutas-privilegio}
\label{code5}
\end{figure}

\begin{itemize}
\item Creación de un archivo SQL con la definición de las tablas necesarias para
el nuevo paquete, estas deben estar prefijadas con el nombre del paquete
(figura \ref{code1}).
\item Inserción un nuevo registro de paquete (figura \ref{code2}).
\item Inserción de los registros de privilegios para el paquete nuevo
(figura \ref{code3}).
\item Inserción de los registros de ruta para el paquete nuevo
(figura \ref{code4}).
\item Registro de los permisos necesarios por ruta creada
(figura \ref{code5}).
\end{itemize}

Esta diseño esta basado en la forma de control que puede verse en cualquier
sistema administrador de contenido básico (CMS).

\subsection{\emph{templates}: Manejador de plantillas}
Las principales funciones de este paquete son:

\begin{itemize}
\item Gestión de la presentación del sistema.
\item Gestión de la presentación personalizada del usuario final.
\item Administración de las utilidades adicionales que presentan las paginas
del sistema.
\end{itemize}

\section{Construcción de espacios virtuales}
Una vez terminada la construcción de los paquetes que componen la base funcional
del sistema, se encaminó el proyecto a la construcción de espacios virtuales,
como se menciono en el capitulo anterior, han sido definido varios tipos de
espacios, en esta sección se tratan las funcionalidades que poseen cada uno de
ellos.

\begin{figure}
\centering
%LaTeX with PSTricks extensions
%%Creator: inkscape 0.48.5
%%Please note this file requires PSTricks extensions
\psset{xunit=.5pt,yunit=.5pt,runit=.5pt}
\begin{pspicture}(800,1120)
{
\newrgbcolor{curcolor}{0 0 0}
\pscustom[linewidth=2,linecolor=curcolor]
{
\newpath
\moveto(149.7723,1035.418408)
\lineto(149.7723,807.14284)
\lineto(304.1473,807.14284)
}
}
{
\newrgbcolor{curcolor}{0 0 0}
\pscustom[linewidth=2,linecolor=curcolor]
{
\newpath
\moveto(304.94869995,829.9588254)
\lineto(453.09933472,829.9588254)
\lineto(453.09933472,781.80820208)
\lineto(304.94869995,781.80820208)
\closepath
}
}
{
\newrgbcolor{curcolor}{0 0 0}
\pscustom[linestyle=none,fillstyle=solid,fillcolor=curcolor]
{
\newpath
\moveto(347.27955322,809.30351107)
\curveto(347.27954467,810.32349868)(347.10954484,811.1034979)(346.76955322,811.64351107)
\curveto(346.4495455,812.18349682)(345.82954612,812.45349655)(344.90955322,812.45351107)
\curveto(344.70954724,812.45349655)(344.45954749,812.42349658)(344.15955322,812.36351107)
\curveto(343.85954809,812.32349668)(343.56954838,812.22349678)(343.28955322,812.06351107)
\curveto(343.02954892,811.9034971)(342.79954915,811.65349735)(342.59955322,811.31351107)
\curveto(342.39954955,810.99349801)(342.29954965,810.55349845)(342.29955322,809.99351107)
\curveto(342.29954965,809.51349949)(342.40954954,809.12349988)(342.62955322,808.82351107)
\curveto(342.86954908,808.52350048)(343.16954878,808.26350074)(343.52955322,808.04351107)
\curveto(343.90954804,807.84350116)(344.32954762,807.67350133)(344.78955322,807.53351107)
\curveto(345.26954668,807.39350161)(345.75954619,807.24350176)(346.25955322,807.08351107)
\curveto(346.73954521,806.92350208)(347.20954474,806.74350226)(347.66955322,806.54351107)
\curveto(348.1495438,806.36350264)(348.56954338,806.1035029)(348.92955322,805.76351107)
\curveto(349.30954264,805.44350356)(349.60954234,805.02350398)(349.82955322,804.50351107)
\curveto(350.06954188,804.003505)(350.18954176,803.35350565)(350.18955322,802.55351107)
\curveto(350.18954176,801.71350729)(350.05954189,800.97350803)(349.79955322,800.33351107)
\curveto(349.53954241,799.71350929)(349.17954277,799.19350981)(348.71955322,798.77351107)
\curveto(348.25954369,798.35351065)(347.70954424,798.04351096)(347.06955322,797.84351107)
\curveto(346.42954552,797.62351138)(345.73954621,797.51351149)(344.99955322,797.51351107)
\curveto(343.63954831,797.51351149)(342.57954937,797.73351127)(341.81955322,798.17351107)
\curveto(341.07955087,798.61351039)(340.52955142,799.13350987)(340.16955322,799.73351107)
\curveto(339.82955212,800.35350865)(339.62955232,800.98350802)(339.56955322,801.62351107)
\curveto(339.50955244,802.26350674)(339.47955247,802.79350621)(339.47955322,803.21351107)
\lineto(341.99955322,803.21351107)
\curveto(341.99954995,802.73350627)(342.03954991,802.26350674)(342.11955322,801.80351107)
\curveto(342.19954975,801.36350764)(342.3495496,800.96350804)(342.56955322,800.60351107)
\curveto(342.78954916,800.26350874)(343.08954886,799.99350901)(343.46955322,799.79351107)
\curveto(343.86954808,799.59350941)(344.37954757,799.49350951)(344.99955322,799.49351107)
\curveto(345.19954675,799.49350951)(345.4495465,799.52350948)(345.74955322,799.58351107)
\curveto(346.0495459,799.64350936)(346.33954561,799.77350923)(346.61955322,799.97351107)
\curveto(346.91954503,800.17350883)(347.16954478,800.44350856)(347.36955322,800.78351107)
\curveto(347.56954438,801.12350788)(347.66954428,801.58350742)(347.66955322,802.16351107)
\curveto(347.66954428,802.7035063)(347.5495444,803.14350586)(347.30955322,803.48351107)
\curveto(347.08954486,803.84350516)(346.78954516,804.13350487)(346.40955322,804.35351107)
\curveto(346.0495459,804.59350441)(345.62954632,804.79350421)(345.14955322,804.95351107)
\curveto(344.68954726,805.11350389)(344.21954773,805.27350373)(343.73955322,805.43351107)
\curveto(343.25954869,805.59350341)(342.77954917,805.76350324)(342.29955322,805.94351107)
\curveto(341.81955013,806.14350286)(341.38955056,806.4035026)(341.00955322,806.72351107)
\curveto(340.6495513,807.04350196)(340.3495516,807.46350154)(340.10955322,807.98351107)
\curveto(339.88955206,808.5035005)(339.77955217,809.17349983)(339.77955322,809.99351107)
\curveto(339.77955217,810.73349827)(339.90955204,811.38349762)(340.16955322,811.94351107)
\curveto(340.4495515,812.5034965)(340.81955113,812.96349604)(341.27955322,813.32351107)
\curveto(341.75955019,813.7034953)(342.30954964,813.98349502)(342.92955322,814.16351107)
\curveto(343.5495484,814.34349466)(344.20954774,814.43349457)(344.90955322,814.43351107)
\curveto(346.06954588,814.43349457)(346.97954497,814.25349475)(347.63955322,813.89351107)
\curveto(348.29954365,813.53349547)(348.78954316,813.09349591)(349.10955322,812.57351107)
\curveto(349.42954252,812.05349695)(349.61954233,811.49349751)(349.67955322,810.89351107)
\curveto(349.75954219,810.31349869)(349.79954215,809.78349922)(349.79955322,809.30351107)
\lineto(347.27955322,809.30351107)
}
}
{
\newrgbcolor{curcolor}{0 0 0}
\pscustom[linestyle=none,fillstyle=solid,fillcolor=curcolor]
{
\newpath
\moveto(363.15705322,797.93351107)
\lineto(360.75705322,797.93351107)
\lineto(360.75705322,799.82351107)
\lineto(360.69705322,799.82351107)
\curveto(360.35704378,799.08350992)(359.81704432,798.51351049)(359.07705322,798.11351107)
\curveto(358.3370458,797.71351129)(357.57704656,797.51351149)(356.79705322,797.51351107)
\curveto(355.7370484,797.51351149)(354.91704922,797.68351132)(354.33705322,798.02351107)
\curveto(353.77705036,798.38351062)(353.35705078,798.82351018)(353.07705322,799.34351107)
\curveto(352.81705132,799.86350914)(352.66705147,800.41350859)(352.62705322,800.99351107)
\curveto(352.58705155,801.59350741)(352.56705157,802.13350687)(352.56705322,802.61351107)
\lineto(352.56705322,814.01351107)
\lineto(355.08705322,814.01351107)
\lineto(355.08705322,802.91351107)
\curveto(355.08704905,802.61350639)(355.09704904,802.27350673)(355.11705322,801.89351107)
\curveto(355.15704898,801.51350749)(355.25704888,801.15350785)(355.41705322,800.81351107)
\curveto(355.57704856,800.49350851)(355.80704833,800.22350878)(356.10705322,800.00351107)
\curveto(356.42704771,799.78350922)(356.87704726,799.67350933)(357.45705322,799.67351107)
\curveto(357.79704634,799.67350933)(358.14704599,799.73350927)(358.50705322,799.85351107)
\curveto(358.88704525,799.97350903)(359.2370449,800.16350884)(359.55705322,800.42351107)
\curveto(359.87704426,800.68350832)(360.137044,801.01350799)(360.33705322,801.41351107)
\curveto(360.5370436,801.83350717)(360.6370435,802.33350667)(360.63705322,802.91351107)
\lineto(360.63705322,814.01351107)
\lineto(363.15705322,814.01351107)
\lineto(363.15705322,797.93351107)
}
}
{
\newrgbcolor{curcolor}{0 0 0}
\pscustom[linestyle=none,fillstyle=solid,fillcolor=curcolor]
{
\newpath
\moveto(366.45377197,819.35351107)
\lineto(368.97377197,819.35351107)
\lineto(368.97377197,812.18351107)
\lineto(369.03377197,812.18351107)
\curveto(369.31376746,812.88349612)(369.79376698,813.43349557)(370.47377197,813.83351107)
\curveto(371.15376562,814.23349477)(371.91376486,814.43349457)(372.75377197,814.43351107)
\curveto(373.83376294,814.43349457)(374.69376208,814.14349486)(375.33377197,813.56351107)
\curveto(375.99376078,813.003496)(376.49376028,812.3034967)(376.83377197,811.46351107)
\curveto(377.1737596,810.62349838)(377.39375938,809.7034993)(377.49377197,808.70351107)
\curveto(377.59375918,807.72350128)(377.64375913,806.81350219)(377.64377197,805.97351107)
\curveto(377.64375913,804.83350417)(377.54375923,803.75350525)(377.34377197,802.73351107)
\curveto(377.14375963,801.71350729)(376.83375994,800.81350819)(376.41377197,800.03351107)
\curveto(375.99376078,799.27350973)(375.44376133,798.66351034)(374.76377197,798.20351107)
\curveto(374.10376267,797.74351126)(373.31376346,797.51351149)(372.39377197,797.51351107)
\curveto(371.95376482,797.51351149)(371.54376523,797.58351142)(371.16377197,797.72351107)
\curveto(370.78376599,797.86351114)(370.43376634,798.04351096)(370.11377197,798.26351107)
\curveto(369.81376696,798.48351052)(369.55376722,798.73351027)(369.33377197,799.01351107)
\curveto(369.13376764,799.31350969)(368.99376778,799.61350939)(368.91377197,799.91351107)
\lineto(368.85377197,799.91351107)
\lineto(368.85377197,797.93351107)
\lineto(366.45377197,797.93351107)
\lineto(366.45377197,819.35351107)
\moveto(368.82377197,805.97351107)
\curveto(368.82376795,805.15350385)(368.86376791,804.36350464)(368.94377197,803.60351107)
\curveto(369.02376775,802.84350616)(369.18376759,802.17350683)(369.42377197,801.59351107)
\curveto(369.66376711,801.01350799)(369.99376678,800.54350846)(370.41377197,800.18351107)
\curveto(370.83376594,799.84350916)(371.38376539,799.67350933)(372.06377197,799.67351107)
\curveto(372.64376413,799.67350933)(373.12376365,799.82350918)(373.50377197,800.12351107)
\curveto(373.88376289,800.44350856)(374.18376259,800.88350812)(374.40377197,801.44351107)
\curveto(374.62376215,802.003507)(374.773762,802.66350634)(374.85377197,803.42351107)
\curveto(374.95376182,804.2035048)(375.00376177,805.05350395)(375.00377197,805.97351107)
\curveto(375.00376177,806.93350207)(374.95376182,807.8035012)(374.85377197,808.58351107)
\curveto(374.773762,809.36349964)(374.62376215,810.02349898)(374.40377197,810.56351107)
\curveto(374.18376259,811.1034979)(373.88376289,811.52349748)(373.50377197,811.82351107)
\curveto(373.12376365,812.12349688)(372.64376413,812.27349673)(372.06377197,812.27351107)
\curveto(371.38376539,812.27349673)(370.83376594,812.09349691)(370.41377197,811.73351107)
\curveto(369.99376678,811.37349763)(369.66376711,810.89349811)(369.42377197,810.29351107)
\curveto(369.18376759,809.69349931)(369.02376775,809.01349999)(368.94377197,808.25351107)
\curveto(368.86376791,807.51350149)(368.82376795,806.75350225)(368.82377197,805.97351107)
}
}
{
\newrgbcolor{curcolor}{0 0 0}
\pscustom[linestyle=none,fillstyle=solid,fillcolor=curcolor]
{
\newpath
\moveto(383.01049072,816.47351107)
\lineto(380.49049072,816.47351107)
\lineto(380.49049072,819.35351107)
\lineto(383.01049072,819.35351107)
\lineto(383.01049072,816.47351107)
\moveto(383.01049072,796.37351107)
\curveto(383.0104864,795.05351395)(382.7104867,794.06351494)(382.11049072,793.40351107)
\curveto(381.53048788,792.74351626)(380.55048886,792.41351659)(379.17049072,792.41351107)
\curveto(378.97049044,792.41351659)(378.78049063,792.42351658)(378.60049072,792.44351107)
\lineto(378.00049072,792.50351107)
\lineto(378.00049072,794.66351107)
\lineto(378.48049072,794.60351107)
\curveto(378.62049079,794.58351442)(378.77049064,794.57351443)(378.93049072,794.57351107)
\curveto(379.3104901,794.57351443)(379.6104898,794.64351436)(379.83049072,794.78351107)
\curveto(380.05048936,794.92351408)(380.20048921,795.11351389)(380.28049072,795.35351107)
\curveto(380.38048903,795.59351341)(380.44048897,795.88351312)(380.46049072,796.22351107)
\curveto(380.48048893,796.54351246)(380.49048892,796.89351211)(380.49049072,797.27351107)
\lineto(380.49049072,814.01351107)
\lineto(383.01049072,814.01351107)
\lineto(383.01049072,796.37351107)
}
}
{
\newrgbcolor{curcolor}{0 0 0}
\pscustom[linestyle=none,fillstyle=solid,fillcolor=curcolor]
{
\newpath
\moveto(388.53424072,805.67351107)
\curveto(388.53423697,805.05350395)(388.54423696,804.38350462)(388.56424072,803.66351107)
\curveto(388.6042369,802.94350606)(388.72423678,802.27350673)(388.92424072,801.65351107)
\curveto(389.12423638,801.03350797)(389.42423608,800.51350849)(389.82424072,800.09351107)
\curveto(390.24423526,799.69350931)(390.84423466,799.49350951)(391.62424072,799.49351107)
\curveto(392.22423328,799.49350951)(392.7042328,799.62350938)(393.06424072,799.88351107)
\curveto(393.42423208,800.16350884)(393.69423181,800.49350851)(393.87424072,800.87351107)
\curveto(394.07423143,801.27350773)(394.2042313,801.68350732)(394.26424072,802.10351107)
\curveto(394.32423118,802.54350646)(394.35423115,802.91350609)(394.35424072,803.21351107)
\lineto(396.87424072,803.21351107)
\curveto(396.87422863,802.79350621)(396.79422871,802.25350675)(396.63424072,801.59351107)
\curveto(396.49422901,800.95350805)(396.22422928,800.32350868)(395.82424072,799.70351107)
\curveto(395.42423008,799.1035099)(394.87423063,798.58351042)(394.17424072,798.14351107)
\curveto(393.47423203,797.72351128)(392.57423293,797.51351149)(391.47424072,797.51351107)
\curveto(389.49423601,797.51351149)(388.06423744,798.19351081)(387.18424072,799.55351107)
\curveto(386.32423918,800.91350809)(385.89423961,802.98350602)(385.89424072,805.76351107)
\curveto(385.89423961,806.76350224)(385.95423955,807.77350123)(386.07424072,808.79351107)
\curveto(386.21423929,809.83349917)(386.48423902,810.76349824)(386.88424072,811.58351107)
\curveto(387.3042382,812.42349658)(387.88423762,813.1034959)(388.62424072,813.62351107)
\curveto(389.38423612,814.16349484)(390.38423512,814.43349457)(391.62424072,814.43351107)
\curveto(392.84423266,814.43349457)(393.81423169,814.19349481)(394.53424072,813.71351107)
\curveto(395.25423025,813.23349577)(395.79422971,812.61349639)(396.15424072,811.85351107)
\curveto(396.51422899,811.11349789)(396.74422876,810.28349872)(396.84424072,809.36351107)
\curveto(396.94422856,808.44350056)(396.99422851,807.55350145)(396.99424072,806.69351107)
\lineto(396.99424072,805.67351107)
\lineto(388.53424072,805.67351107)
\moveto(394.35424072,807.65351107)
\lineto(394.35424072,808.52351107)
\curveto(394.35423115,808.96350004)(394.31423119,809.41349959)(394.23424072,809.87351107)
\curveto(394.15423135,810.35349865)(394.0042315,810.78349822)(393.78424072,811.16351107)
\curveto(393.58423192,811.54349746)(393.3042322,811.85349715)(392.94424072,812.09351107)
\curveto(392.58423292,812.33349667)(392.12423338,812.45349655)(391.56424072,812.45351107)
\curveto(390.9042346,812.45349655)(390.37423513,812.27349673)(389.97424072,811.91351107)
\curveto(389.59423591,811.57349743)(389.3042362,811.17349783)(389.10424072,810.71351107)
\curveto(388.9042366,810.25349875)(388.77423673,809.78349922)(388.71424072,809.30351107)
\curveto(388.65423685,808.84350016)(388.62423688,808.49350051)(388.62424072,808.25351107)
\lineto(388.62424072,807.65351107)
\lineto(394.35424072,807.65351107)
}
}
{
\newrgbcolor{curcolor}{0 0 0}
\pscustom[linestyle=none,fillstyle=solid,fillcolor=curcolor]
{
\newpath
\moveto(407.38502197,809.12351107)
\curveto(407.38501267,809.5034995)(407.33501272,809.89349911)(407.23502197,810.29351107)
\curveto(407.1550129,810.69349831)(407.01501304,811.05349795)(406.81502197,811.37351107)
\curveto(406.63501342,811.69349731)(406.37501368,811.95349705)(406.03502197,812.15351107)
\curveto(405.71501434,812.35349665)(405.31501474,812.45349655)(404.83502197,812.45351107)
\curveto(404.43501562,812.45349655)(404.04501601,812.38349662)(403.66502197,812.24351107)
\curveto(403.30501675,812.1034969)(402.97501708,811.79349721)(402.67502197,811.31351107)
\curveto(402.37501768,810.85349815)(402.13501792,810.18349882)(401.95502197,809.30351107)
\curveto(401.77501828,808.42350058)(401.68501837,807.25350175)(401.68502197,805.79351107)
\curveto(401.68501837,805.27350373)(401.69501836,804.65350435)(401.71502197,803.93351107)
\curveto(401.7550183,803.21350579)(401.87501818,802.52350648)(402.07502197,801.86351107)
\curveto(402.27501778,801.2035078)(402.57501748,800.64350836)(402.97502197,800.18351107)
\curveto(403.39501666,799.72350928)(403.98501607,799.49350951)(404.74502197,799.49351107)
\curveto(405.28501477,799.49350951)(405.72501433,799.62350938)(406.06502197,799.88351107)
\curveto(406.40501365,800.14350886)(406.67501338,800.46350854)(406.87502197,800.84351107)
\curveto(407.07501298,801.24350776)(407.20501285,801.67350733)(407.26502197,802.13351107)
\curveto(407.34501271,802.61350639)(407.38501267,803.07350593)(407.38502197,803.51351107)
\lineto(409.90502197,803.51351107)
\curveto(409.90501015,802.87350613)(409.81501024,802.2035068)(409.63502197,801.50351107)
\curveto(409.47501058,800.8035082)(409.18501087,800.15350885)(408.76502197,799.55351107)
\curveto(408.36501169,798.97351003)(407.82501223,798.48351052)(407.14502197,798.08351107)
\curveto(406.46501359,797.7035113)(405.62501443,797.51351149)(404.62502197,797.51351107)
\curveto(402.64501741,797.51351149)(401.21501884,798.19351081)(400.33502197,799.55351107)
\curveto(399.47502058,800.91350809)(399.04502101,802.98350602)(399.04502197,805.76351107)
\curveto(399.04502101,806.76350224)(399.10502095,807.77350123)(399.22502197,808.79351107)
\curveto(399.36502069,809.83349917)(399.63502042,810.76349824)(400.03502197,811.58351107)
\curveto(400.4550196,812.42349658)(401.03501902,813.1034959)(401.77502197,813.62351107)
\curveto(402.53501752,814.16349484)(403.53501652,814.43349457)(404.77502197,814.43351107)
\curveto(405.87501418,814.43349457)(406.7550133,814.24349476)(407.41502197,813.86351107)
\curveto(408.09501196,813.48349552)(408.61501144,813.01349599)(408.97502197,812.45351107)
\curveto(409.3550107,811.91349709)(409.60501045,811.33349767)(409.72502197,810.71351107)
\curveto(409.84501021,810.11349889)(409.90501015,809.58349942)(409.90502197,809.12351107)
\lineto(407.38502197,809.12351107)
}
}
{
\newrgbcolor{curcolor}{0 0 0}
\pscustom[linestyle=none,fillstyle=solid,fillcolor=curcolor]
{
\newpath
\moveto(413.25845947,818.69351107)
\lineto(415.77845947,818.69351107)
\lineto(415.77845947,814.01351107)
\lineto(418.56845947,814.01351107)
\lineto(418.56845947,812.03351107)
\lineto(415.77845947,812.03351107)
\lineto(415.77845947,801.71351107)
\curveto(415.77845455,801.07350793)(415.88845444,800.61350839)(416.10845947,800.33351107)
\curveto(416.328454,800.05350895)(416.76845356,799.91350909)(417.42845947,799.91351107)
\curveto(417.70845262,799.91350909)(417.9284524,799.92350908)(418.08845947,799.94351107)
\curveto(418.24845208,799.96350904)(418.39845193,799.98350902)(418.53845947,800.00351107)
\lineto(418.53845947,797.93351107)
\curveto(418.37845195,797.89351111)(418.1284522,797.85351115)(417.78845947,797.81351107)
\curveto(417.44845288,797.77351123)(417.01845331,797.75351125)(416.49845947,797.75351107)
\curveto(415.83845449,797.75351125)(415.29845503,797.81351119)(414.87845947,797.93351107)
\curveto(414.45845587,798.07351093)(414.1284562,798.27351073)(413.88845947,798.53351107)
\curveto(413.64845668,798.81351019)(413.47845685,799.15350985)(413.37845947,799.55351107)
\curveto(413.29845703,799.95350905)(413.25845707,800.41350859)(413.25845947,800.93351107)
\lineto(413.25845947,812.03351107)
\lineto(410.91845947,812.03351107)
\lineto(410.91845947,814.01351107)
\lineto(413.25845947,814.01351107)
\lineto(413.25845947,818.69351107)
}
}
{
\newrgbcolor{curcolor}{0 0 0}
\pscustom[linewidth=2,linecolor=curcolor]
{
\newpath
\moveto(291.57174683,1084.15015154)
\lineto(439.72238159,1084.15015154)
\lineto(439.72238159,1035.99952822)
\lineto(291.57174683,1035.99952822)
\closepath
}
}
{
\newrgbcolor{curcolor}{0 0 0}
\pscustom[linestyle=none,fillstyle=solid,fillcolor=curcolor]
{
\newpath
\moveto(339.26603394,1063.22483989)
\curveto(339.26602464,1063.60482832)(339.21602469,1063.99482793)(339.11603394,1064.39483989)
\curveto(339.03602487,1064.79482713)(338.89602501,1065.15482677)(338.69603394,1065.47483989)
\curveto(338.51602539,1065.79482613)(338.25602565,1066.05482587)(337.91603394,1066.25483989)
\curveto(337.59602631,1066.45482547)(337.19602671,1066.55482537)(336.71603394,1066.55483989)
\curveto(336.31602759,1066.55482537)(335.92602798,1066.48482544)(335.54603394,1066.34483989)
\curveto(335.18602872,1066.20482572)(334.85602905,1065.89482603)(334.55603394,1065.41483989)
\curveto(334.25602965,1064.95482697)(334.01602989,1064.28482764)(333.83603394,1063.40483989)
\curveto(333.65603025,1062.5248294)(333.56603034,1061.35483057)(333.56603394,1059.89483989)
\curveto(333.56603034,1059.37483255)(333.57603033,1058.75483317)(333.59603394,1058.03483989)
\curveto(333.63603027,1057.31483461)(333.75603015,1056.6248353)(333.95603394,1055.96483989)
\curveto(334.15602975,1055.30483662)(334.45602945,1054.74483718)(334.85603394,1054.28483989)
\curveto(335.27602863,1053.8248381)(335.86602804,1053.59483833)(336.62603394,1053.59483989)
\curveto(337.16602674,1053.59483833)(337.6060263,1053.7248382)(337.94603394,1053.98483989)
\curveto(338.28602562,1054.24483768)(338.55602535,1054.56483736)(338.75603394,1054.94483989)
\curveto(338.95602495,1055.34483658)(339.08602482,1055.77483615)(339.14603394,1056.23483989)
\curveto(339.22602468,1056.71483521)(339.26602464,1057.17483475)(339.26603394,1057.61483989)
\lineto(341.78603394,1057.61483989)
\curveto(341.78602212,1056.97483495)(341.69602221,1056.30483562)(341.51603394,1055.60483989)
\curveto(341.35602255,1054.90483702)(341.06602284,1054.25483767)(340.64603394,1053.65483989)
\curveto(340.24602366,1053.07483885)(339.7060242,1052.58483934)(339.02603394,1052.18483989)
\curveto(338.34602556,1051.80484012)(337.5060264,1051.61484031)(336.50603394,1051.61483989)
\curveto(334.52602938,1051.61484031)(333.09603081,1052.29483963)(332.21603394,1053.65483989)
\curveto(331.35603255,1055.01483691)(330.92603298,1057.08483484)(330.92603394,1059.86483989)
\curveto(330.92603298,1060.86483106)(330.98603292,1061.87483005)(331.10603394,1062.89483989)
\curveto(331.24603266,1063.93482799)(331.51603239,1064.86482706)(331.91603394,1065.68483989)
\curveto(332.33603157,1066.5248254)(332.91603099,1067.20482472)(333.65603394,1067.72483989)
\curveto(334.41602949,1068.26482366)(335.41602849,1068.53482339)(336.65603394,1068.53483989)
\curveto(337.75602615,1068.53482339)(338.63602527,1068.34482358)(339.29603394,1067.96483989)
\curveto(339.97602393,1067.58482434)(340.49602341,1067.11482481)(340.85603394,1066.55483989)
\curveto(341.23602267,1066.01482591)(341.48602242,1065.43482649)(341.60603394,1064.81483989)
\curveto(341.72602218,1064.21482771)(341.78602212,1063.68482824)(341.78603394,1063.22483989)
\lineto(339.26603394,1063.22483989)
}
}
{
\newrgbcolor{curcolor}{0 0 0}
\pscustom[linestyle=none,fillstyle=solid,fillcolor=curcolor]
{
\newpath
\moveto(351.76947144,1060.79483989)
\curveto(351.52946265,1060.55483137)(351.21946296,1060.35483157)(350.83947144,1060.19483989)
\curveto(350.4794637,1060.03483189)(350.08946409,1059.88483204)(349.66947144,1059.74483989)
\curveto(349.26946491,1059.6248323)(348.86946531,1059.49483243)(348.46947144,1059.35483989)
\curveto(348.08946609,1059.23483269)(347.75946642,1059.08483284)(347.47947144,1058.90483989)
\curveto(347.0794671,1058.64483328)(346.76946741,1058.3248336)(346.54947144,1057.94483989)
\curveto(346.34946783,1057.58483434)(346.24946793,1057.05483487)(346.24947144,1056.35483989)
\curveto(346.24946793,1055.51483641)(346.40946777,1054.84483708)(346.72947144,1054.34483989)
\curveto(347.06946711,1053.84483808)(347.66946651,1053.59483833)(348.52947144,1053.59483989)
\curveto(348.94946523,1053.59483833)(349.34946483,1053.67483825)(349.72947144,1053.83483989)
\curveto(350.12946405,1053.99483793)(350.4794637,1054.21483771)(350.77947144,1054.49483989)
\curveto(351.0794631,1054.77483715)(351.31946286,1055.09483683)(351.49947144,1055.45483989)
\curveto(351.6794625,1055.83483609)(351.76946241,1056.23483569)(351.76947144,1056.65483989)
\lineto(351.76947144,1060.79483989)
\moveto(343.99947144,1063.25483989)
\curveto(343.99947018,1065.07482685)(344.41946976,1066.40482552)(345.25947144,1067.24483989)
\curveto(346.09946808,1068.10482382)(347.4794667,1068.53482339)(349.39947144,1068.53483989)
\curveto(350.61946356,1068.53482339)(351.55946262,1068.37482355)(352.21947144,1068.05483989)
\curveto(352.8794613,1067.73482419)(353.35946082,1067.33482459)(353.65947144,1066.85483989)
\curveto(353.9794602,1066.37482555)(354.15946002,1065.86482606)(354.19947144,1065.32483989)
\curveto(354.25945992,1064.80482712)(354.28945989,1064.33482759)(354.28947144,1063.91483989)
\lineto(354.28947144,1054.94483989)
\curveto(354.28945989,1054.60483732)(354.31945986,1054.30483762)(354.37947144,1054.04483989)
\curveto(354.43945974,1053.78483814)(354.66945951,1053.65483827)(355.06947144,1053.65483989)
\curveto(355.22945895,1053.65483827)(355.34945883,1053.66483826)(355.42947144,1053.68483989)
\curveto(355.52945865,1053.7248382)(355.60945857,1053.76483816)(355.66947144,1053.80483989)
\lineto(355.66947144,1052.00483989)
\curveto(355.56945861,1051.98483994)(355.3794588,1051.95483997)(355.09947144,1051.91483989)
\curveto(354.81945936,1051.87484005)(354.51945966,1051.85484007)(354.19947144,1051.85483989)
\curveto(353.95946022,1051.85484007)(353.70946047,1051.86484006)(353.44947144,1051.88483989)
\curveto(353.20946097,1051.90484002)(352.96946121,1051.97483995)(352.72947144,1052.09483989)
\curveto(352.50946167,1052.23483969)(352.31946186,1052.44483948)(352.15947144,1052.72483989)
\curveto(352.01946216,1053.00483892)(351.93946224,1053.40483852)(351.91947144,1053.92483989)
\lineto(351.85947144,1053.92483989)
\curveto(351.45946272,1053.20483872)(350.89946328,1052.63483929)(350.17947144,1052.21483989)
\curveto(349.4794647,1051.81484011)(348.74946543,1051.61484031)(347.98947144,1051.61483989)
\curveto(346.48946769,1051.61484031)(345.3794688,1052.03483989)(344.65947144,1052.87483989)
\curveto(343.95947022,1053.71483821)(343.60947057,1054.85483707)(343.60947144,1056.29483989)
\curveto(343.60947057,1057.43483449)(343.84947033,1058.37483355)(344.32947144,1059.11483989)
\curveto(344.82946935,1059.87483205)(345.59946858,1060.41483151)(346.63947144,1060.73483989)
\lineto(350.02947144,1061.75483989)
\curveto(350.48946369,1061.89483003)(350.83946334,1062.03482989)(351.07947144,1062.17483989)
\curveto(351.33946284,1062.33482959)(351.51946266,1062.50482942)(351.61947144,1062.68483989)
\curveto(351.73946244,1062.86482906)(351.80946237,1063.07482885)(351.82947144,1063.31483989)
\curveto(351.84946233,1063.55482837)(351.85946232,1063.84482808)(351.85947144,1064.18483989)
\curveto(351.85946232,1065.76482616)(350.99946318,1066.55482537)(349.27947144,1066.55483989)
\curveto(348.5794656,1066.55482537)(348.03946614,1066.41482551)(347.65947144,1066.13483989)
\curveto(347.29946688,1065.87482605)(347.02946715,1065.56482636)(346.84947144,1065.20483989)
\curveto(346.68946749,1064.84482708)(346.58946759,1064.48482744)(346.54947144,1064.12483989)
\curveto(346.52946765,1063.78482814)(346.51946766,1063.54482838)(346.51947144,1063.40483989)
\lineto(346.51947144,1063.25483989)
\lineto(343.99947144,1063.25483989)
}
}
{
\newrgbcolor{curcolor}{0 0 0}
\pscustom[linestyle=none,fillstyle=solid,fillcolor=curcolor]
{
\newpath
\moveto(357.69025269,1068.11483989)
\lineto(360.21025269,1068.11483989)
\lineto(360.21025269,1065.71483989)
\lineto(360.27025269,1065.71483989)
\curveto(360.45024828,1066.09482583)(360.64024809,1066.45482547)(360.84025269,1066.79483989)
\curveto(361.06024767,1067.13482479)(361.31024742,1067.43482449)(361.59025269,1067.69483989)
\curveto(361.87024686,1067.95482397)(362.18024655,1068.15482377)(362.52025269,1068.29483989)
\curveto(362.88024585,1068.45482347)(363.29024544,1068.53482339)(363.75025269,1068.53483989)
\curveto(364.25024448,1068.53482339)(364.62024411,1068.47482345)(364.86025269,1068.35483989)
\lineto(364.86025269,1065.89483989)
\curveto(364.74024399,1065.91482601)(364.58024415,1065.93482599)(364.38025269,1065.95483989)
\curveto(364.20024453,1065.99482593)(363.91024482,1066.01482591)(363.51025269,1066.01483989)
\curveto(363.19024554,1066.01482591)(362.84024589,1065.93482599)(362.46025269,1065.77483989)
\curveto(362.08024665,1065.63482629)(361.72024701,1065.40482652)(361.38025269,1065.08483989)
\curveto(361.06024767,1064.78482714)(360.78024795,1064.39482753)(360.54025269,1063.91483989)
\curveto(360.32024841,1063.43482849)(360.21024852,1062.86482906)(360.21025269,1062.20483989)
\lineto(360.21025269,1052.03483989)
\lineto(357.69025269,1052.03483989)
\lineto(357.69025269,1068.11483989)
}
}
{
\newrgbcolor{curcolor}{0 0 0}
\pscustom[linestyle=none,fillstyle=solid,fillcolor=curcolor]
{
\newpath
\moveto(368.69650269,1059.77483989)
\curveto(368.69649894,1059.15483277)(368.70649893,1058.48483344)(368.72650269,1057.76483989)
\curveto(368.76649887,1057.04483488)(368.88649875,1056.37483555)(369.08650269,1055.75483989)
\curveto(369.28649835,1055.13483679)(369.58649805,1054.61483731)(369.98650269,1054.19483989)
\curveto(370.40649723,1053.79483813)(371.00649663,1053.59483833)(371.78650269,1053.59483989)
\curveto(372.38649525,1053.59483833)(372.86649477,1053.7248382)(373.22650269,1053.98483989)
\curveto(373.58649405,1054.26483766)(373.85649378,1054.59483733)(374.03650269,1054.97483989)
\curveto(374.2364934,1055.37483655)(374.36649327,1055.78483614)(374.42650269,1056.20483989)
\curveto(374.48649315,1056.64483528)(374.51649312,1057.01483491)(374.51650269,1057.31483989)
\lineto(377.03650269,1057.31483989)
\curveto(377.0364906,1056.89483503)(376.95649068,1056.35483557)(376.79650269,1055.69483989)
\curveto(376.65649098,1055.05483687)(376.38649125,1054.4248375)(375.98650269,1053.80483989)
\curveto(375.58649205,1053.20483872)(375.0364926,1052.68483924)(374.33650269,1052.24483989)
\curveto(373.636494,1051.8248401)(372.7364949,1051.61484031)(371.63650269,1051.61483989)
\curveto(369.65649798,1051.61484031)(368.22649941,1052.29483963)(367.34650269,1053.65483989)
\curveto(366.48650115,1055.01483691)(366.05650158,1057.08483484)(366.05650269,1059.86483989)
\curveto(366.05650158,1060.86483106)(366.11650152,1061.87483005)(366.23650269,1062.89483989)
\curveto(366.37650126,1063.93482799)(366.64650099,1064.86482706)(367.04650269,1065.68483989)
\curveto(367.46650017,1066.5248254)(368.04649959,1067.20482472)(368.78650269,1067.72483989)
\curveto(369.54649809,1068.26482366)(370.54649709,1068.53482339)(371.78650269,1068.53483989)
\curveto(373.00649463,1068.53482339)(373.97649366,1068.29482363)(374.69650269,1067.81483989)
\curveto(375.41649222,1067.33482459)(375.95649168,1066.71482521)(376.31650269,1065.95483989)
\curveto(376.67649096,1065.21482671)(376.90649073,1064.38482754)(377.00650269,1063.46483989)
\curveto(377.10649053,1062.54482938)(377.15649048,1061.65483027)(377.15650269,1060.79483989)
\lineto(377.15650269,1059.77483989)
\lineto(368.69650269,1059.77483989)
\moveto(374.51650269,1061.75483989)
\lineto(374.51650269,1062.62483989)
\curveto(374.51649312,1063.06482886)(374.47649316,1063.51482841)(374.39650269,1063.97483989)
\curveto(374.31649332,1064.45482747)(374.16649347,1064.88482704)(373.94650269,1065.26483989)
\curveto(373.74649389,1065.64482628)(373.46649417,1065.95482597)(373.10650269,1066.19483989)
\curveto(372.74649489,1066.43482549)(372.28649535,1066.55482537)(371.72650269,1066.55483989)
\curveto(371.06649657,1066.55482537)(370.5364971,1066.37482555)(370.13650269,1066.01483989)
\curveto(369.75649788,1065.67482625)(369.46649817,1065.27482665)(369.26650269,1064.81483989)
\curveto(369.06649857,1064.35482757)(368.9364987,1063.88482804)(368.87650269,1063.40483989)
\curveto(368.81649882,1062.94482898)(368.78649885,1062.59482933)(368.78650269,1062.35483989)
\lineto(368.78650269,1061.75483989)
\lineto(374.51650269,1061.75483989)
}
}
{
\newrgbcolor{curcolor}{0 0 0}
\pscustom[linestyle=none,fillstyle=solid,fillcolor=curcolor]
{
\newpath
\moveto(381.99728394,1059.77483989)
\curveto(381.99728019,1059.15483277)(382.00728018,1058.48483344)(382.02728394,1057.76483989)
\curveto(382.06728012,1057.04483488)(382.18728,1056.37483555)(382.38728394,1055.75483989)
\curveto(382.5872796,1055.13483679)(382.8872793,1054.61483731)(383.28728394,1054.19483989)
\curveto(383.70727848,1053.79483813)(384.30727788,1053.59483833)(385.08728394,1053.59483989)
\curveto(385.6872765,1053.59483833)(386.16727602,1053.7248382)(386.52728394,1053.98483989)
\curveto(386.8872753,1054.26483766)(387.15727503,1054.59483733)(387.33728394,1054.97483989)
\curveto(387.53727465,1055.37483655)(387.66727452,1055.78483614)(387.72728394,1056.20483989)
\curveto(387.7872744,1056.64483528)(387.81727437,1057.01483491)(387.81728394,1057.31483989)
\lineto(390.33728394,1057.31483989)
\curveto(390.33727185,1056.89483503)(390.25727193,1056.35483557)(390.09728394,1055.69483989)
\curveto(389.95727223,1055.05483687)(389.6872725,1054.4248375)(389.28728394,1053.80483989)
\curveto(388.8872733,1053.20483872)(388.33727385,1052.68483924)(387.63728394,1052.24483989)
\curveto(386.93727525,1051.8248401)(386.03727615,1051.61484031)(384.93728394,1051.61483989)
\curveto(382.95727923,1051.61484031)(381.52728066,1052.29483963)(380.64728394,1053.65483989)
\curveto(379.7872824,1055.01483691)(379.35728283,1057.08483484)(379.35728394,1059.86483989)
\curveto(379.35728283,1060.86483106)(379.41728277,1061.87483005)(379.53728394,1062.89483989)
\curveto(379.67728251,1063.93482799)(379.94728224,1064.86482706)(380.34728394,1065.68483989)
\curveto(380.76728142,1066.5248254)(381.34728084,1067.20482472)(382.08728394,1067.72483989)
\curveto(382.84727934,1068.26482366)(383.84727834,1068.53482339)(385.08728394,1068.53483989)
\curveto(386.30727588,1068.53482339)(387.27727491,1068.29482363)(387.99728394,1067.81483989)
\curveto(388.71727347,1067.33482459)(389.25727293,1066.71482521)(389.61728394,1065.95483989)
\curveto(389.97727221,1065.21482671)(390.20727198,1064.38482754)(390.30728394,1063.46483989)
\curveto(390.40727178,1062.54482938)(390.45727173,1061.65483027)(390.45728394,1060.79483989)
\lineto(390.45728394,1059.77483989)
\lineto(381.99728394,1059.77483989)
\moveto(387.81728394,1061.75483989)
\lineto(387.81728394,1062.62483989)
\curveto(387.81727437,1063.06482886)(387.77727441,1063.51482841)(387.69728394,1063.97483989)
\curveto(387.61727457,1064.45482747)(387.46727472,1064.88482704)(387.24728394,1065.26483989)
\curveto(387.04727514,1065.64482628)(386.76727542,1065.95482597)(386.40728394,1066.19483989)
\curveto(386.04727614,1066.43482549)(385.5872766,1066.55482537)(385.02728394,1066.55483989)
\curveto(384.36727782,1066.55482537)(383.83727835,1066.37482555)(383.43728394,1066.01483989)
\curveto(383.05727913,1065.67482625)(382.76727942,1065.27482665)(382.56728394,1064.81483989)
\curveto(382.36727982,1064.35482757)(382.23727995,1063.88482804)(382.17728394,1063.40483989)
\curveto(382.11728007,1062.94482898)(382.0872801,1062.59482933)(382.08728394,1062.35483989)
\lineto(382.08728394,1061.75483989)
\lineto(387.81728394,1061.75483989)
}
}
{
\newrgbcolor{curcolor}{0 0 0}
\pscustom[linestyle=none,fillstyle=solid,fillcolor=curcolor]
{
\newpath
\moveto(393.19806519,1068.11483989)
\lineto(395.71806519,1068.11483989)
\lineto(395.71806519,1065.71483989)
\lineto(395.77806519,1065.71483989)
\curveto(395.95806078,1066.09482583)(396.14806059,1066.45482547)(396.34806519,1066.79483989)
\curveto(396.56806017,1067.13482479)(396.81805992,1067.43482449)(397.09806519,1067.69483989)
\curveto(397.37805936,1067.95482397)(397.68805905,1068.15482377)(398.02806519,1068.29483989)
\curveto(398.38805835,1068.45482347)(398.79805794,1068.53482339)(399.25806519,1068.53483989)
\curveto(399.75805698,1068.53482339)(400.12805661,1068.47482345)(400.36806519,1068.35483989)
\lineto(400.36806519,1065.89483989)
\curveto(400.24805649,1065.91482601)(400.08805665,1065.93482599)(399.88806519,1065.95483989)
\curveto(399.70805703,1065.99482593)(399.41805732,1066.01482591)(399.01806519,1066.01483989)
\curveto(398.69805804,1066.01482591)(398.34805839,1065.93482599)(397.96806519,1065.77483989)
\curveto(397.58805915,1065.63482629)(397.22805951,1065.40482652)(396.88806519,1065.08483989)
\curveto(396.56806017,1064.78482714)(396.28806045,1064.39482753)(396.04806519,1063.91483989)
\curveto(395.82806091,1063.43482849)(395.71806102,1062.86482906)(395.71806519,1062.20483989)
\lineto(395.71806519,1052.03483989)
\lineto(393.19806519,1052.03483989)
\lineto(393.19806519,1068.11483989)
}
}
{
\newrgbcolor{curcolor}{0 0 0}
\pscustom[linewidth=2,linecolor=curcolor]
{
\newpath
\moveto(245.6098175,962.45398836)
\lineto(485.68431091,962.45398836)
\lineto(485.68431091,914.3033536)
\lineto(245.6098175,914.3033536)
\closepath
}
}
{
\newrgbcolor{curcolor}{0 0 0}
\pscustom[linestyle=none,fillstyle=solid,fillcolor=curcolor]
{
\newpath
\moveto(291.803479,941.6186664)
\curveto(291.8034697,941.99865483)(291.75346975,942.38865444)(291.653479,942.7886664)
\curveto(291.57346993,943.18865364)(291.43347007,943.54865328)(291.233479,943.8686664)
\curveto(291.05347045,944.18865264)(290.79347071,944.44865238)(290.453479,944.6486664)
\curveto(290.13347137,944.84865198)(289.73347177,944.94865188)(289.253479,944.9486664)
\curveto(288.85347265,944.94865188)(288.46347304,944.87865195)(288.083479,944.7386664)
\curveto(287.72347378,944.59865223)(287.39347411,944.28865254)(287.093479,943.8086664)
\curveto(286.79347471,943.34865348)(286.55347495,942.67865415)(286.373479,941.7986664)
\curveto(286.19347531,940.91865591)(286.1034754,939.74865708)(286.103479,938.2886664)
\curveto(286.1034754,937.76865906)(286.11347539,937.14865968)(286.133479,936.4286664)
\curveto(286.17347533,935.70866112)(286.29347521,935.01866181)(286.493479,934.3586664)
\curveto(286.69347481,933.69866313)(286.99347451,933.13866369)(287.393479,932.6786664)
\curveto(287.81347369,932.21866461)(288.4034731,931.98866484)(289.163479,931.9886664)
\curveto(289.7034718,931.98866484)(290.14347136,932.11866471)(290.483479,932.3786664)
\curveto(290.82347068,932.63866419)(291.09347041,932.95866387)(291.293479,933.3386664)
\curveto(291.49347001,933.73866309)(291.62346988,934.16866266)(291.683479,934.6286664)
\curveto(291.76346974,935.10866172)(291.8034697,935.56866126)(291.803479,936.0086664)
\lineto(294.323479,936.0086664)
\curveto(294.32346718,935.36866146)(294.23346727,934.69866213)(294.053479,933.9986664)
\curveto(293.89346761,933.29866353)(293.6034679,932.64866418)(293.183479,932.0486664)
\curveto(292.78346872,931.46866536)(292.24346926,930.97866585)(291.563479,930.5786664)
\curveto(290.88347062,930.19866663)(290.04347146,930.00866682)(289.043479,930.0086664)
\curveto(287.06347444,930.00866682)(285.63347587,930.68866614)(284.753479,932.0486664)
\curveto(283.89347761,933.40866342)(283.46347804,935.47866135)(283.463479,938.2586664)
\curveto(283.46347804,939.25865757)(283.52347798,940.26865656)(283.643479,941.2886664)
\curveto(283.78347772,942.3286545)(284.05347745,943.25865357)(284.453479,944.0786664)
\curveto(284.87347663,944.91865191)(285.45347605,945.59865123)(286.193479,946.1186664)
\curveto(286.95347455,946.65865017)(287.95347355,946.9286499)(289.193479,946.9286664)
\curveto(290.29347121,946.9286499)(291.17347033,946.73865009)(291.833479,946.3586664)
\curveto(292.51346899,945.97865085)(293.03346847,945.50865132)(293.393479,944.9486664)
\curveto(293.77346773,944.40865242)(294.02346748,943.828653)(294.143479,943.2086664)
\curveto(294.26346724,942.60865422)(294.32346718,942.07865475)(294.323479,941.6186664)
\lineto(291.803479,941.6186664)
}
}
{
\newrgbcolor{curcolor}{0 0 0}
\pscustom[linestyle=none,fillstyle=solid,fillcolor=curcolor]
{
\newpath
\moveto(304.3069165,939.1886664)
\curveto(304.06690771,938.94865788)(303.75690802,938.74865808)(303.3769165,938.5886664)
\curveto(303.01690876,938.4286584)(302.62690915,938.27865855)(302.2069165,938.1386664)
\curveto(301.80690997,938.01865881)(301.40691037,937.88865894)(301.0069165,937.7486664)
\curveto(300.62691115,937.6286592)(300.29691148,937.47865935)(300.0169165,937.2986664)
\curveto(299.61691216,937.03865979)(299.30691247,936.71866011)(299.0869165,936.3386664)
\curveto(298.88691289,935.97866085)(298.78691299,935.44866138)(298.7869165,934.7486664)
\curveto(298.78691299,933.90866292)(298.94691283,933.23866359)(299.2669165,932.7386664)
\curveto(299.60691217,932.23866459)(300.20691157,931.98866484)(301.0669165,931.9886664)
\curveto(301.48691029,931.98866484)(301.88690989,932.06866476)(302.2669165,932.2286664)
\curveto(302.66690911,932.38866444)(303.01690876,932.60866422)(303.3169165,932.8886664)
\curveto(303.61690816,933.16866366)(303.85690792,933.48866334)(304.0369165,933.8486664)
\curveto(304.21690756,934.2286626)(304.30690747,934.6286622)(304.3069165,935.0486664)
\lineto(304.3069165,939.1886664)
\moveto(296.5369165,941.6486664)
\curveto(296.53691524,943.46865336)(296.95691482,944.79865203)(297.7969165,945.6386664)
\curveto(298.63691314,946.49865033)(300.01691176,946.9286499)(301.9369165,946.9286664)
\curveto(303.15690862,946.9286499)(304.09690768,946.76865006)(304.7569165,946.4486664)
\curveto(305.41690636,946.1286507)(305.89690588,945.7286511)(306.1969165,945.2486664)
\curveto(306.51690526,944.76865206)(306.69690508,944.25865257)(306.7369165,943.7186664)
\curveto(306.79690498,943.19865363)(306.82690495,942.7286541)(306.8269165,942.3086664)
\lineto(306.8269165,933.3386664)
\curveto(306.82690495,932.99866383)(306.85690492,932.69866413)(306.9169165,932.4386664)
\curveto(306.9769048,932.17866465)(307.20690457,932.04866478)(307.6069165,932.0486664)
\curveto(307.76690401,932.04866478)(307.88690389,932.05866477)(307.9669165,932.0786664)
\curveto(308.06690371,932.11866471)(308.14690363,932.15866467)(308.2069165,932.1986664)
\lineto(308.2069165,930.3986664)
\curveto(308.10690367,930.37866645)(307.91690386,930.34866648)(307.6369165,930.3086664)
\curveto(307.35690442,930.26866656)(307.05690472,930.24866658)(306.7369165,930.2486664)
\curveto(306.49690528,930.24866658)(306.24690553,930.25866657)(305.9869165,930.2786664)
\curveto(305.74690603,930.29866653)(305.50690627,930.36866646)(305.2669165,930.4886664)
\curveto(305.04690673,930.6286662)(304.85690692,930.83866599)(304.6969165,931.1186664)
\curveto(304.55690722,931.39866543)(304.4769073,931.79866503)(304.4569165,932.3186664)
\lineto(304.3969165,932.3186664)
\curveto(303.99690778,931.59866523)(303.43690834,931.0286658)(302.7169165,930.6086664)
\curveto(302.01690976,930.20866662)(301.28691049,930.00866682)(300.5269165,930.0086664)
\curveto(299.02691275,930.00866682)(297.91691386,930.4286664)(297.1969165,931.2686664)
\curveto(296.49691528,932.10866472)(296.14691563,933.24866358)(296.1469165,934.6886664)
\curveto(296.14691563,935.828661)(296.38691539,936.76866006)(296.8669165,937.5086664)
\curveto(297.36691441,938.26865856)(298.13691364,938.80865802)(299.1769165,939.1286664)
\lineto(302.5669165,940.1486664)
\curveto(303.02690875,940.28865654)(303.3769084,940.4286564)(303.6169165,940.5686664)
\curveto(303.8769079,940.7286561)(304.05690772,940.89865593)(304.1569165,941.0786664)
\curveto(304.2769075,941.25865557)(304.34690743,941.46865536)(304.3669165,941.7086664)
\curveto(304.38690739,941.94865488)(304.39690738,942.23865459)(304.3969165,942.5786664)
\curveto(304.39690738,944.15865267)(303.53690824,944.94865188)(301.8169165,944.9486664)
\curveto(301.11691066,944.94865188)(300.5769112,944.80865202)(300.1969165,944.5286664)
\curveto(299.83691194,944.26865256)(299.56691221,943.95865287)(299.3869165,943.5986664)
\curveto(299.22691255,943.23865359)(299.12691265,942.87865395)(299.0869165,942.5186664)
\curveto(299.06691271,942.17865465)(299.05691272,941.93865489)(299.0569165,941.7986664)
\lineto(299.0569165,941.6486664)
\lineto(296.5369165,941.6486664)
}
}
{
\newrgbcolor{curcolor}{0 0 0}
\pscustom[linestyle=none,fillstyle=solid,fillcolor=curcolor]
{
\newpath
\moveto(310.22769775,946.5086664)
\lineto(312.74769775,946.5086664)
\lineto(312.74769775,944.1086664)
\lineto(312.80769775,944.1086664)
\curveto(312.98769334,944.48865234)(313.17769315,944.84865198)(313.37769775,945.1886664)
\curveto(313.59769273,945.5286513)(313.84769248,945.828651)(314.12769775,946.0886664)
\curveto(314.40769192,946.34865048)(314.71769161,946.54865028)(315.05769775,946.6886664)
\curveto(315.41769091,946.84864998)(315.8276905,946.9286499)(316.28769775,946.9286664)
\curveto(316.78768954,946.9286499)(317.15768917,946.86864996)(317.39769775,946.7486664)
\lineto(317.39769775,944.2886664)
\curveto(317.27768905,944.30865252)(317.11768921,944.3286525)(316.91769775,944.3486664)
\curveto(316.73768959,944.38865244)(316.44768988,944.40865242)(316.04769775,944.4086664)
\curveto(315.7276906,944.40865242)(315.37769095,944.3286525)(314.99769775,944.1686664)
\curveto(314.61769171,944.0286528)(314.25769207,943.79865303)(313.91769775,943.4786664)
\curveto(313.59769273,943.17865365)(313.31769301,942.78865404)(313.07769775,942.3086664)
\curveto(312.85769347,941.828655)(312.74769358,941.25865557)(312.74769775,940.5986664)
\lineto(312.74769775,930.4286664)
\lineto(310.22769775,930.4286664)
\lineto(310.22769775,946.5086664)
}
}
{
\newrgbcolor{curcolor}{0 0 0}
\pscustom[linestyle=none,fillstyle=solid,fillcolor=curcolor]
{
\newpath
\moveto(321.23394775,938.1686664)
\curveto(321.233944,937.54865928)(321.24394399,936.87865995)(321.26394775,936.1586664)
\curveto(321.30394393,935.43866139)(321.42394381,934.76866206)(321.62394775,934.1486664)
\curveto(321.82394341,933.5286633)(322.12394311,933.00866382)(322.52394775,932.5886664)
\curveto(322.94394229,932.18866464)(323.54394169,931.98866484)(324.32394775,931.9886664)
\curveto(324.92394031,931.98866484)(325.40393983,932.11866471)(325.76394775,932.3786664)
\curveto(326.12393911,932.65866417)(326.39393884,932.98866384)(326.57394775,933.3686664)
\curveto(326.77393846,933.76866306)(326.90393833,934.17866265)(326.96394775,934.5986664)
\curveto(327.02393821,935.03866179)(327.05393818,935.40866142)(327.05394775,935.7086664)
\lineto(329.57394775,935.7086664)
\curveto(329.57393566,935.28866154)(329.49393574,934.74866208)(329.33394775,934.0886664)
\curveto(329.19393604,933.44866338)(328.92393631,932.81866401)(328.52394775,932.1986664)
\curveto(328.12393711,931.59866523)(327.57393766,931.07866575)(326.87394775,930.6386664)
\curveto(326.17393906,930.21866661)(325.27393996,930.00866682)(324.17394775,930.0086664)
\curveto(322.19394304,930.00866682)(320.76394447,930.68866614)(319.88394775,932.0486664)
\curveto(319.02394621,933.40866342)(318.59394664,935.47866135)(318.59394775,938.2586664)
\curveto(318.59394664,939.25865757)(318.65394658,940.26865656)(318.77394775,941.2886664)
\curveto(318.91394632,942.3286545)(319.18394605,943.25865357)(319.58394775,944.0786664)
\curveto(320.00394523,944.91865191)(320.58394465,945.59865123)(321.32394775,946.1186664)
\curveto(322.08394315,946.65865017)(323.08394215,946.9286499)(324.32394775,946.9286664)
\curveto(325.54393969,946.9286499)(326.51393872,946.68865014)(327.23394775,946.2086664)
\curveto(327.95393728,945.7286511)(328.49393674,945.10865172)(328.85394775,944.3486664)
\curveto(329.21393602,943.60865322)(329.44393579,942.77865405)(329.54394775,941.8586664)
\curveto(329.64393559,940.93865589)(329.69393554,940.04865678)(329.69394775,939.1886664)
\lineto(329.69394775,938.1686664)
\lineto(321.23394775,938.1686664)
\moveto(327.05394775,940.1486664)
\lineto(327.05394775,941.0186664)
\curveto(327.05393818,941.45865537)(327.01393822,941.90865492)(326.93394775,942.3686664)
\curveto(326.85393838,942.84865398)(326.70393853,943.27865355)(326.48394775,943.6586664)
\curveto(326.28393895,944.03865279)(326.00393923,944.34865248)(325.64394775,944.5886664)
\curveto(325.28393995,944.828652)(324.82394041,944.94865188)(324.26394775,944.9486664)
\curveto(323.60394163,944.94865188)(323.07394216,944.76865206)(322.67394775,944.4086664)
\curveto(322.29394294,944.06865276)(322.00394323,943.66865316)(321.80394775,943.2086664)
\curveto(321.60394363,942.74865408)(321.47394376,942.27865455)(321.41394775,941.7986664)
\curveto(321.35394388,941.33865549)(321.32394391,940.98865584)(321.32394775,940.7486664)
\lineto(321.32394775,940.1486664)
\lineto(327.05394775,940.1486664)
}
}
{
\newrgbcolor{curcolor}{0 0 0}
\pscustom[linestyle=none,fillstyle=solid,fillcolor=curcolor]
{
\newpath
\moveto(334.534729,938.1686664)
\curveto(334.53472525,937.54865928)(334.54472524,936.87865995)(334.564729,936.1586664)
\curveto(334.60472518,935.43866139)(334.72472506,934.76866206)(334.924729,934.1486664)
\curveto(335.12472466,933.5286633)(335.42472436,933.00866382)(335.824729,932.5886664)
\curveto(336.24472354,932.18866464)(336.84472294,931.98866484)(337.624729,931.9886664)
\curveto(338.22472156,931.98866484)(338.70472108,932.11866471)(339.064729,932.3786664)
\curveto(339.42472036,932.65866417)(339.69472009,932.98866384)(339.874729,933.3686664)
\curveto(340.07471971,933.76866306)(340.20471958,934.17866265)(340.264729,934.5986664)
\curveto(340.32471946,935.03866179)(340.35471943,935.40866142)(340.354729,935.7086664)
\lineto(342.874729,935.7086664)
\curveto(342.87471691,935.28866154)(342.79471699,934.74866208)(342.634729,934.0886664)
\curveto(342.49471729,933.44866338)(342.22471756,932.81866401)(341.824729,932.1986664)
\curveto(341.42471836,931.59866523)(340.87471891,931.07866575)(340.174729,930.6386664)
\curveto(339.47472031,930.21866661)(338.57472121,930.00866682)(337.474729,930.0086664)
\curveto(335.49472429,930.00866682)(334.06472572,930.68866614)(333.184729,932.0486664)
\curveto(332.32472746,933.40866342)(331.89472789,935.47866135)(331.894729,938.2586664)
\curveto(331.89472789,939.25865757)(331.95472783,940.26865656)(332.074729,941.2886664)
\curveto(332.21472757,942.3286545)(332.4847273,943.25865357)(332.884729,944.0786664)
\curveto(333.30472648,944.91865191)(333.8847259,945.59865123)(334.624729,946.1186664)
\curveto(335.3847244,946.65865017)(336.3847234,946.9286499)(337.624729,946.9286664)
\curveto(338.84472094,946.9286499)(339.81471997,946.68865014)(340.534729,946.2086664)
\curveto(341.25471853,945.7286511)(341.79471799,945.10865172)(342.154729,944.3486664)
\curveto(342.51471727,943.60865322)(342.74471704,942.77865405)(342.844729,941.8586664)
\curveto(342.94471684,940.93865589)(342.99471679,940.04865678)(342.994729,939.1886664)
\lineto(342.994729,938.1686664)
\lineto(334.534729,938.1686664)
\moveto(340.354729,940.1486664)
\lineto(340.354729,941.0186664)
\curveto(340.35471943,941.45865537)(340.31471947,941.90865492)(340.234729,942.3686664)
\curveto(340.15471963,942.84865398)(340.00471978,943.27865355)(339.784729,943.6586664)
\curveto(339.5847202,944.03865279)(339.30472048,944.34865248)(338.944729,944.5886664)
\curveto(338.5847212,944.828652)(338.12472166,944.94865188)(337.564729,944.9486664)
\curveto(336.90472288,944.94865188)(336.37472341,944.76865206)(335.974729,944.4086664)
\curveto(335.59472419,944.06865276)(335.30472448,943.66865316)(335.104729,943.2086664)
\curveto(334.90472488,942.74865408)(334.77472501,942.27865455)(334.714729,941.7986664)
\curveto(334.65472513,941.33865549)(334.62472516,940.98865584)(334.624729,940.7486664)
\lineto(334.624729,940.1486664)
\lineto(340.354729,940.1486664)
}
}
{
\newrgbcolor{curcolor}{0 0 0}
\pscustom[linestyle=none,fillstyle=solid,fillcolor=curcolor]
{
\newpath
\moveto(345.73551025,946.5086664)
\lineto(348.25551025,946.5086664)
\lineto(348.25551025,944.1086664)
\lineto(348.31551025,944.1086664)
\curveto(348.49550584,944.48865234)(348.68550565,944.84865198)(348.88551025,945.1886664)
\curveto(349.10550523,945.5286513)(349.35550498,945.828651)(349.63551025,946.0886664)
\curveto(349.91550442,946.34865048)(350.22550411,946.54865028)(350.56551025,946.6886664)
\curveto(350.92550341,946.84864998)(351.335503,946.9286499)(351.79551025,946.9286664)
\curveto(352.29550204,946.9286499)(352.66550167,946.86864996)(352.90551025,946.7486664)
\lineto(352.90551025,944.2886664)
\curveto(352.78550155,944.30865252)(352.62550171,944.3286525)(352.42551025,944.3486664)
\curveto(352.24550209,944.38865244)(351.95550238,944.40865242)(351.55551025,944.4086664)
\curveto(351.2355031,944.40865242)(350.88550345,944.3286525)(350.50551025,944.1686664)
\curveto(350.12550421,944.0286528)(349.76550457,943.79865303)(349.42551025,943.4786664)
\curveto(349.10550523,943.17865365)(348.82550551,942.78865404)(348.58551025,942.3086664)
\curveto(348.36550597,941.828655)(348.25550608,941.25865557)(348.25551025,940.5986664)
\lineto(348.25551025,930.4286664)
\lineto(345.73551025,930.4286664)
\lineto(345.73551025,946.5086664)
}
}
{
\newrgbcolor{curcolor}{0 0 0}
\pscustom[linestyle=none,fillstyle=solid,fillcolor=curcolor]
{
\newpath
\moveto(352.99176025,928.1786664)
\lineto(367.99176025,928.1786664)
\lineto(367.99176025,926.6786664)
\lineto(352.99176025,926.6786664)
\lineto(352.99176025,928.1786664)
}
}
{
\newrgbcolor{curcolor}{0 0 0}
\pscustom[linestyle=none,fillstyle=solid,fillcolor=curcolor]
{
\newpath
\moveto(376.54176025,941.7986664)
\curveto(376.5417517,942.81865401)(376.37175187,943.59865323)(376.03176025,944.1386664)
\curveto(375.71175253,944.67865215)(375.09175315,944.94865188)(374.17176025,944.9486664)
\curveto(373.97175427,944.94865188)(373.72175452,944.91865191)(373.42176025,944.8586664)
\curveto(373.12175512,944.81865201)(372.83175541,944.71865211)(372.55176025,944.5586664)
\curveto(372.29175595,944.39865243)(372.06175618,944.14865268)(371.86176025,943.8086664)
\curveto(371.66175658,943.48865334)(371.56175668,943.04865378)(371.56176025,942.4886664)
\curveto(371.56175668,942.00865482)(371.67175657,941.61865521)(371.89176025,941.3186664)
\curveto(372.13175611,941.01865581)(372.43175581,940.75865607)(372.79176025,940.5386664)
\curveto(373.17175507,940.33865649)(373.59175465,940.16865666)(374.05176025,940.0286664)
\curveto(374.53175371,939.88865694)(375.02175322,939.73865709)(375.52176025,939.5786664)
\curveto(376.00175224,939.41865741)(376.47175177,939.23865759)(376.93176025,939.0386664)
\curveto(377.41175083,938.85865797)(377.83175041,938.59865823)(378.19176025,938.2586664)
\curveto(378.57174967,937.93865889)(378.87174937,937.51865931)(379.09176025,936.9986664)
\curveto(379.33174891,936.49866033)(379.45174879,935.84866098)(379.45176025,935.0486664)
\curveto(379.45174879,934.20866262)(379.32174892,933.46866336)(379.06176025,932.8286664)
\curveto(378.80174944,932.20866462)(378.4417498,931.68866514)(377.98176025,931.2686664)
\curveto(377.52175072,930.84866598)(376.97175127,930.53866629)(376.33176025,930.3386664)
\curveto(375.69175255,930.11866671)(375.00175324,930.00866682)(374.26176025,930.0086664)
\curveto(372.90175534,930.00866682)(371.8417564,930.2286666)(371.08176025,930.6686664)
\curveto(370.3417579,931.10866572)(369.79175845,931.6286652)(369.43176025,932.2286664)
\curveto(369.09175915,932.84866398)(368.89175935,933.47866335)(368.83176025,934.1186664)
\curveto(368.77175947,934.75866207)(368.7417595,935.28866154)(368.74176025,935.7086664)
\lineto(371.26176025,935.7086664)
\curveto(371.26175698,935.2286616)(371.30175694,934.75866207)(371.38176025,934.2986664)
\curveto(371.46175678,933.85866297)(371.61175663,933.45866337)(371.83176025,933.0986664)
\curveto(372.05175619,932.75866407)(372.35175589,932.48866434)(372.73176025,932.2886664)
\curveto(373.13175511,932.08866474)(373.6417546,931.98866484)(374.26176025,931.9886664)
\curveto(374.46175378,931.98866484)(374.71175353,932.01866481)(375.01176025,932.0786664)
\curveto(375.31175293,932.13866469)(375.60175264,932.26866456)(375.88176025,932.4686664)
\curveto(376.18175206,932.66866416)(376.43175181,932.93866389)(376.63176025,933.2786664)
\curveto(376.83175141,933.61866321)(376.93175131,934.07866275)(376.93176025,934.6586664)
\curveto(376.93175131,935.19866163)(376.81175143,935.63866119)(376.57176025,935.9786664)
\curveto(376.35175189,936.33866049)(376.05175219,936.6286602)(375.67176025,936.8486664)
\curveto(375.31175293,937.08865974)(374.89175335,937.28865954)(374.41176025,937.4486664)
\curveto(373.95175429,937.60865922)(373.48175476,937.76865906)(373.00176025,937.9286664)
\curveto(372.52175572,938.08865874)(372.0417562,938.25865857)(371.56176025,938.4386664)
\curveto(371.08175716,938.63865819)(370.65175759,938.89865793)(370.27176025,939.2186664)
\curveto(369.91175833,939.53865729)(369.61175863,939.95865687)(369.37176025,940.4786664)
\curveto(369.15175909,940.99865583)(369.0417592,941.66865516)(369.04176025,942.4886664)
\curveto(369.0417592,943.2286536)(369.17175907,943.87865295)(369.43176025,944.4386664)
\curveto(369.71175853,944.99865183)(370.08175816,945.45865137)(370.54176025,945.8186664)
\curveto(371.02175722,946.19865063)(371.57175667,946.47865035)(372.19176025,946.6586664)
\curveto(372.81175543,946.83864999)(373.47175477,946.9286499)(374.17176025,946.9286664)
\curveto(375.33175291,946.9286499)(376.241752,946.74865008)(376.90176025,946.3886664)
\curveto(377.56175068,946.0286508)(378.05175019,945.58865124)(378.37176025,945.0686664)
\curveto(378.69174955,944.54865228)(378.88174936,943.98865284)(378.94176025,943.3886664)
\curveto(379.02174922,942.80865402)(379.06174918,942.27865455)(379.06176025,941.7986664)
\lineto(376.54176025,941.7986664)
}
}
{
\newrgbcolor{curcolor}{0 0 0}
\pscustom[linestyle=none,fillstyle=solid,fillcolor=curcolor]
{
\newpath
\moveto(392.41926025,930.4286664)
\lineto(390.01926025,930.4286664)
\lineto(390.01926025,932.3186664)
\lineto(389.95926025,932.3186664)
\curveto(389.61925081,931.57866525)(389.07925135,931.00866582)(388.33926025,930.6086664)
\curveto(387.59925283,930.20866662)(386.83925359,930.00866682)(386.05926025,930.0086664)
\curveto(384.99925543,930.00866682)(384.17925625,930.17866665)(383.59926025,930.5186664)
\curveto(383.03925739,930.87866595)(382.61925781,931.31866551)(382.33926025,931.8386664)
\curveto(382.07925835,932.35866447)(381.9292585,932.90866392)(381.88926025,933.4886664)
\curveto(381.84925858,934.08866274)(381.8292586,934.6286622)(381.82926025,935.1086664)
\lineto(381.82926025,946.5086664)
\lineto(384.34926025,946.5086664)
\lineto(384.34926025,935.4086664)
\curveto(384.34925608,935.10866172)(384.35925607,934.76866206)(384.37926025,934.3886664)
\curveto(384.41925601,934.00866282)(384.51925591,933.64866318)(384.67926025,933.3086664)
\curveto(384.83925559,932.98866384)(385.06925536,932.71866411)(385.36926025,932.4986664)
\curveto(385.68925474,932.27866455)(386.13925429,932.16866466)(386.71926025,932.1686664)
\curveto(387.05925337,932.16866466)(387.40925302,932.2286646)(387.76926025,932.3486664)
\curveto(388.14925228,932.46866436)(388.49925193,932.65866417)(388.81926025,932.9186664)
\curveto(389.13925129,933.17866365)(389.39925103,933.50866332)(389.59926025,933.9086664)
\curveto(389.79925063,934.3286625)(389.89925053,934.828662)(389.89926025,935.4086664)
\lineto(389.89926025,946.5086664)
\lineto(392.41926025,946.5086664)
\lineto(392.41926025,930.4286664)
}
}
{
\newrgbcolor{curcolor}{0 0 0}
\pscustom[linestyle=none,fillstyle=solid,fillcolor=curcolor]
{
\newpath
\moveto(395.715979,951.8486664)
\lineto(398.235979,951.8486664)
\lineto(398.235979,944.6786664)
\lineto(398.295979,944.6786664)
\curveto(398.57597449,945.37865145)(399.05597401,945.9286509)(399.735979,946.3286664)
\curveto(400.41597265,946.7286501)(401.17597189,946.9286499)(402.015979,946.9286664)
\curveto(403.09596997,946.9286499)(403.95596911,946.63865019)(404.595979,946.0586664)
\curveto(405.25596781,945.49865133)(405.75596731,944.79865203)(406.095979,943.9586664)
\curveto(406.43596663,943.11865371)(406.65596641,942.19865463)(406.755979,941.1986664)
\curveto(406.85596621,940.21865661)(406.90596616,939.30865752)(406.905979,938.4686664)
\curveto(406.90596616,937.3286595)(406.80596626,936.24866058)(406.605979,935.2286664)
\curveto(406.40596666,934.20866262)(406.09596697,933.30866352)(405.675979,932.5286664)
\curveto(405.25596781,931.76866506)(404.70596836,931.15866567)(404.025979,930.6986664)
\curveto(403.3659697,930.23866659)(402.57597049,930.00866682)(401.655979,930.0086664)
\curveto(401.21597185,930.00866682)(400.80597226,930.07866675)(400.425979,930.2186664)
\curveto(400.04597302,930.35866647)(399.69597337,930.53866629)(399.375979,930.7586664)
\curveto(399.07597399,930.97866585)(398.81597425,931.2286656)(398.595979,931.5086664)
\curveto(398.39597467,931.80866502)(398.25597481,932.10866472)(398.175979,932.4086664)
\lineto(398.115979,932.4086664)
\lineto(398.115979,930.4286664)
\lineto(395.715979,930.4286664)
\lineto(395.715979,951.8486664)
\moveto(398.085979,938.4686664)
\curveto(398.08597498,937.64865918)(398.12597494,936.85865997)(398.205979,936.0986664)
\curveto(398.28597478,935.33866149)(398.44597462,934.66866216)(398.685979,934.0886664)
\curveto(398.92597414,933.50866332)(399.25597381,933.03866379)(399.675979,932.6786664)
\curveto(400.09597297,932.33866449)(400.64597242,932.16866466)(401.325979,932.1686664)
\curveto(401.90597116,932.16866466)(402.38597068,932.31866451)(402.765979,932.6186664)
\curveto(403.14596992,932.93866389)(403.44596962,933.37866345)(403.665979,933.9386664)
\curveto(403.88596918,934.49866233)(404.03596903,935.15866167)(404.115979,935.9186664)
\curveto(404.21596885,936.69866013)(404.2659688,937.54865928)(404.265979,938.4686664)
\curveto(404.2659688,939.4286574)(404.21596885,940.29865653)(404.115979,941.0786664)
\curveto(404.03596903,941.85865497)(403.88596918,942.51865431)(403.665979,943.0586664)
\curveto(403.44596962,943.59865323)(403.14596992,944.01865281)(402.765979,944.3186664)
\curveto(402.38597068,944.61865221)(401.90597116,944.76865206)(401.325979,944.7686664)
\curveto(400.64597242,944.76865206)(400.09597297,944.58865224)(399.675979,944.2286664)
\curveto(399.25597381,943.86865296)(398.92597414,943.38865344)(398.685979,942.7886664)
\curveto(398.44597462,942.18865464)(398.28597478,941.50865532)(398.205979,940.7486664)
\curveto(398.12597494,940.00865682)(398.08597498,939.24865758)(398.085979,938.4686664)
}
}
{
\newrgbcolor{curcolor}{0 0 0}
\pscustom[linestyle=none,fillstyle=solid,fillcolor=curcolor]
{
\newpath
\moveto(412.27269775,948.9686664)
\lineto(409.75269775,948.9686664)
\lineto(409.75269775,951.8486664)
\lineto(412.27269775,951.8486664)
\lineto(412.27269775,948.9686664)
\moveto(412.27269775,928.8686664)
\curveto(412.27269343,927.54866928)(411.97269373,926.55867027)(411.37269775,925.8986664)
\curveto(410.79269491,925.23867159)(409.81269589,924.90867192)(408.43269775,924.9086664)
\curveto(408.23269747,924.90867192)(408.04269766,924.91867191)(407.86269775,924.9386664)
\lineto(407.26269775,924.9986664)
\lineto(407.26269775,927.1586664)
\lineto(407.74269775,927.0986664)
\curveto(407.88269782,927.07866975)(408.03269767,927.06866976)(408.19269775,927.0686664)
\curveto(408.57269713,927.06866976)(408.87269683,927.13866969)(409.09269775,927.2786664)
\curveto(409.31269639,927.41866941)(409.46269624,927.60866922)(409.54269775,927.8486664)
\curveto(409.64269606,928.08866874)(409.702696,928.37866845)(409.72269775,928.7186664)
\curveto(409.74269596,929.03866779)(409.75269595,929.38866744)(409.75269775,929.7686664)
\lineto(409.75269775,946.5086664)
\lineto(412.27269775,946.5086664)
\lineto(412.27269775,928.8686664)
}
}
{
\newrgbcolor{curcolor}{0 0 0}
\pscustom[linestyle=none,fillstyle=solid,fillcolor=curcolor]
{
\newpath
\moveto(417.79644775,938.1686664)
\curveto(417.796444,937.54865928)(417.80644399,936.87865995)(417.82644775,936.1586664)
\curveto(417.86644393,935.43866139)(417.98644381,934.76866206)(418.18644775,934.1486664)
\curveto(418.38644341,933.5286633)(418.68644311,933.00866382)(419.08644775,932.5886664)
\curveto(419.50644229,932.18866464)(420.10644169,931.98866484)(420.88644775,931.9886664)
\curveto(421.48644031,931.98866484)(421.96643983,932.11866471)(422.32644775,932.3786664)
\curveto(422.68643911,932.65866417)(422.95643884,932.98866384)(423.13644775,933.3686664)
\curveto(423.33643846,933.76866306)(423.46643833,934.17866265)(423.52644775,934.5986664)
\curveto(423.58643821,935.03866179)(423.61643818,935.40866142)(423.61644775,935.7086664)
\lineto(426.13644775,935.7086664)
\curveto(426.13643566,935.28866154)(426.05643574,934.74866208)(425.89644775,934.0886664)
\curveto(425.75643604,933.44866338)(425.48643631,932.81866401)(425.08644775,932.1986664)
\curveto(424.68643711,931.59866523)(424.13643766,931.07866575)(423.43644775,930.6386664)
\curveto(422.73643906,930.21866661)(421.83643996,930.00866682)(420.73644775,930.0086664)
\curveto(418.75644304,930.00866682)(417.32644447,930.68866614)(416.44644775,932.0486664)
\curveto(415.58644621,933.40866342)(415.15644664,935.47866135)(415.15644775,938.2586664)
\curveto(415.15644664,939.25865757)(415.21644658,940.26865656)(415.33644775,941.2886664)
\curveto(415.47644632,942.3286545)(415.74644605,943.25865357)(416.14644775,944.0786664)
\curveto(416.56644523,944.91865191)(417.14644465,945.59865123)(417.88644775,946.1186664)
\curveto(418.64644315,946.65865017)(419.64644215,946.9286499)(420.88644775,946.9286664)
\curveto(422.10643969,946.9286499)(423.07643872,946.68865014)(423.79644775,946.2086664)
\curveto(424.51643728,945.7286511)(425.05643674,945.10865172)(425.41644775,944.3486664)
\curveto(425.77643602,943.60865322)(426.00643579,942.77865405)(426.10644775,941.8586664)
\curveto(426.20643559,940.93865589)(426.25643554,940.04865678)(426.25644775,939.1886664)
\lineto(426.25644775,938.1686664)
\lineto(417.79644775,938.1686664)
\moveto(423.61644775,940.1486664)
\lineto(423.61644775,941.0186664)
\curveto(423.61643818,941.45865537)(423.57643822,941.90865492)(423.49644775,942.3686664)
\curveto(423.41643838,942.84865398)(423.26643853,943.27865355)(423.04644775,943.6586664)
\curveto(422.84643895,944.03865279)(422.56643923,944.34865248)(422.20644775,944.5886664)
\curveto(421.84643995,944.828652)(421.38644041,944.94865188)(420.82644775,944.9486664)
\curveto(420.16644163,944.94865188)(419.63644216,944.76865206)(419.23644775,944.4086664)
\curveto(418.85644294,944.06865276)(418.56644323,943.66865316)(418.36644775,943.2086664)
\curveto(418.16644363,942.74865408)(418.03644376,942.27865455)(417.97644775,941.7986664)
\curveto(417.91644388,941.33865549)(417.88644391,940.98865584)(417.88644775,940.7486664)
\lineto(417.88644775,940.1486664)
\lineto(423.61644775,940.1486664)
}
}
{
\newrgbcolor{curcolor}{0 0 0}
\pscustom[linestyle=none,fillstyle=solid,fillcolor=curcolor]
{
\newpath
\moveto(436.647229,941.6186664)
\curveto(436.6472197,941.99865483)(436.59721975,942.38865444)(436.497229,942.7886664)
\curveto(436.41721993,943.18865364)(436.27722007,943.54865328)(436.077229,943.8686664)
\curveto(435.89722045,944.18865264)(435.63722071,944.44865238)(435.297229,944.6486664)
\curveto(434.97722137,944.84865198)(434.57722177,944.94865188)(434.097229,944.9486664)
\curveto(433.69722265,944.94865188)(433.30722304,944.87865195)(432.927229,944.7386664)
\curveto(432.56722378,944.59865223)(432.23722411,944.28865254)(431.937229,943.8086664)
\curveto(431.63722471,943.34865348)(431.39722495,942.67865415)(431.217229,941.7986664)
\curveto(431.03722531,940.91865591)(430.9472254,939.74865708)(430.947229,938.2886664)
\curveto(430.9472254,937.76865906)(430.95722539,937.14865968)(430.977229,936.4286664)
\curveto(431.01722533,935.70866112)(431.13722521,935.01866181)(431.337229,934.3586664)
\curveto(431.53722481,933.69866313)(431.83722451,933.13866369)(432.237229,932.6786664)
\curveto(432.65722369,932.21866461)(433.2472231,931.98866484)(434.007229,931.9886664)
\curveto(434.5472218,931.98866484)(434.98722136,932.11866471)(435.327229,932.3786664)
\curveto(435.66722068,932.63866419)(435.93722041,932.95866387)(436.137229,933.3386664)
\curveto(436.33722001,933.73866309)(436.46721988,934.16866266)(436.527229,934.6286664)
\curveto(436.60721974,935.10866172)(436.6472197,935.56866126)(436.647229,936.0086664)
\lineto(439.167229,936.0086664)
\curveto(439.16721718,935.36866146)(439.07721727,934.69866213)(438.897229,933.9986664)
\curveto(438.73721761,933.29866353)(438.4472179,932.64866418)(438.027229,932.0486664)
\curveto(437.62721872,931.46866536)(437.08721926,930.97866585)(436.407229,930.5786664)
\curveto(435.72722062,930.19866663)(434.88722146,930.00866682)(433.887229,930.0086664)
\curveto(431.90722444,930.00866682)(430.47722587,930.68866614)(429.597229,932.0486664)
\curveto(428.73722761,933.40866342)(428.30722804,935.47866135)(428.307229,938.2586664)
\curveto(428.30722804,939.25865757)(428.36722798,940.26865656)(428.487229,941.2886664)
\curveto(428.62722772,942.3286545)(428.89722745,943.25865357)(429.297229,944.0786664)
\curveto(429.71722663,944.91865191)(430.29722605,945.59865123)(431.037229,946.1186664)
\curveto(431.79722455,946.65865017)(432.79722355,946.9286499)(434.037229,946.9286664)
\curveto(435.13722121,946.9286499)(436.01722033,946.73865009)(436.677229,946.3586664)
\curveto(437.35721899,945.97865085)(437.87721847,945.50865132)(438.237229,944.9486664)
\curveto(438.61721773,944.40865242)(438.86721748,943.828653)(438.987229,943.2086664)
\curveto(439.10721724,942.60865422)(439.16721718,942.07865475)(439.167229,941.6186664)
\lineto(436.647229,941.6186664)
}
}
{
\newrgbcolor{curcolor}{0 0 0}
\pscustom[linestyle=none,fillstyle=solid,fillcolor=curcolor]
{
\newpath
\moveto(442.5206665,951.1886664)
\lineto(445.0406665,951.1886664)
\lineto(445.0406665,946.5086664)
\lineto(447.8306665,946.5086664)
\lineto(447.8306665,944.5286664)
\lineto(445.0406665,944.5286664)
\lineto(445.0406665,934.2086664)
\curveto(445.04066158,933.56866326)(445.15066147,933.10866372)(445.3706665,932.8286664)
\curveto(445.59066103,932.54866428)(446.03066059,932.40866442)(446.6906665,932.4086664)
\curveto(446.97065965,932.40866442)(447.19065943,932.41866441)(447.3506665,932.4386664)
\curveto(447.51065911,932.45866437)(447.66065896,932.47866435)(447.8006665,932.4986664)
\lineto(447.8006665,930.4286664)
\curveto(447.64065898,930.38866644)(447.39065923,930.34866648)(447.0506665,930.3086664)
\curveto(446.71065991,930.26866656)(446.28066034,930.24866658)(445.7606665,930.2486664)
\curveto(445.10066152,930.24866658)(444.56066206,930.30866652)(444.1406665,930.4286664)
\curveto(443.7206629,930.56866626)(443.39066323,930.76866606)(443.1506665,931.0286664)
\curveto(442.91066371,931.30866552)(442.74066388,931.64866518)(442.6406665,932.0486664)
\curveto(442.56066406,932.44866438)(442.5206641,932.90866392)(442.5206665,933.4286664)
\lineto(442.5206665,944.5286664)
\lineto(440.1806665,944.5286664)
\lineto(440.1806665,946.5086664)
\lineto(442.5206665,946.5086664)
\lineto(442.5206665,951.1886664)
}
}
{
\newrgbcolor{curcolor}{0 0 0}
\pscustom[linewidth=2,linecolor=curcolor]
{
\newpath
\moveto(585.02105713,1084.15015154)
\lineto(733.17169189,1084.15015154)
\lineto(733.17169189,1035.99952822)
\lineto(585.02105713,1035.99952822)
\closepath
}
}
{
\newrgbcolor{curcolor}{0 0 0}
\pscustom[linestyle=none,fillstyle=solid,fillcolor=curcolor]
{
\newpath
\moveto(643.47244385,1060.79483989)
\curveto(643.23243506,1060.55483137)(642.92243537,1060.35483157)(642.54244385,1060.19483989)
\curveto(642.18243611,1060.03483189)(641.7924365,1059.88483204)(641.37244385,1059.74483989)
\curveto(640.97243732,1059.6248323)(640.57243772,1059.49483243)(640.17244385,1059.35483989)
\curveto(639.7924385,1059.23483269)(639.46243883,1059.08483284)(639.18244385,1058.90483989)
\curveto(638.78243951,1058.64483328)(638.47243982,1058.3248336)(638.25244385,1057.94483989)
\curveto(638.05244024,1057.58483434)(637.95244034,1057.05483487)(637.95244385,1056.35483989)
\curveto(637.95244034,1055.51483641)(638.11244018,1054.84483708)(638.43244385,1054.34483989)
\curveto(638.77243952,1053.84483808)(639.37243892,1053.59483833)(640.23244385,1053.59483989)
\curveto(640.65243764,1053.59483833)(641.05243724,1053.67483825)(641.43244385,1053.83483989)
\curveto(641.83243646,1053.99483793)(642.18243611,1054.21483771)(642.48244385,1054.49483989)
\curveto(642.78243551,1054.77483715)(643.02243527,1055.09483683)(643.20244385,1055.45483989)
\curveto(643.38243491,1055.83483609)(643.47243482,1056.23483569)(643.47244385,1056.65483989)
\lineto(643.47244385,1060.79483989)
\moveto(635.70244385,1063.25483989)
\curveto(635.70244259,1065.07482685)(636.12244217,1066.40482552)(636.96244385,1067.24483989)
\curveto(637.80244049,1068.10482382)(639.18243911,1068.53482339)(641.10244385,1068.53483989)
\curveto(642.32243597,1068.53482339)(643.26243503,1068.37482355)(643.92244385,1068.05483989)
\curveto(644.58243371,1067.73482419)(645.06243323,1067.33482459)(645.36244385,1066.85483989)
\curveto(645.68243261,1066.37482555)(645.86243243,1065.86482606)(645.90244385,1065.32483989)
\curveto(645.96243233,1064.80482712)(645.9924323,1064.33482759)(645.99244385,1063.91483989)
\lineto(645.99244385,1054.94483989)
\curveto(645.9924323,1054.60483732)(646.02243227,1054.30483762)(646.08244385,1054.04483989)
\curveto(646.14243215,1053.78483814)(646.37243192,1053.65483827)(646.77244385,1053.65483989)
\curveto(646.93243136,1053.65483827)(647.05243124,1053.66483826)(647.13244385,1053.68483989)
\curveto(647.23243106,1053.7248382)(647.31243098,1053.76483816)(647.37244385,1053.80483989)
\lineto(647.37244385,1052.00483989)
\curveto(647.27243102,1051.98483994)(647.08243121,1051.95483997)(646.80244385,1051.91483989)
\curveto(646.52243177,1051.87484005)(646.22243207,1051.85484007)(645.90244385,1051.85483989)
\curveto(645.66243263,1051.85484007)(645.41243288,1051.86484006)(645.15244385,1051.88483989)
\curveto(644.91243338,1051.90484002)(644.67243362,1051.97483995)(644.43244385,1052.09483989)
\curveto(644.21243408,1052.23483969)(644.02243427,1052.44483948)(643.86244385,1052.72483989)
\curveto(643.72243457,1053.00483892)(643.64243465,1053.40483852)(643.62244385,1053.92483989)
\lineto(643.56244385,1053.92483989)
\curveto(643.16243513,1053.20483872)(642.60243569,1052.63483929)(641.88244385,1052.21483989)
\curveto(641.18243711,1051.81484011)(640.45243784,1051.61484031)(639.69244385,1051.61483989)
\curveto(638.1924401,1051.61484031)(637.08244121,1052.03483989)(636.36244385,1052.87483989)
\curveto(635.66244263,1053.71483821)(635.31244298,1054.85483707)(635.31244385,1056.29483989)
\curveto(635.31244298,1057.43483449)(635.55244274,1058.37483355)(636.03244385,1059.11483989)
\curveto(636.53244176,1059.87483205)(637.30244099,1060.41483151)(638.34244385,1060.73483989)
\lineto(641.73244385,1061.75483989)
\curveto(642.1924361,1061.89483003)(642.54243575,1062.03482989)(642.78244385,1062.17483989)
\curveto(643.04243525,1062.33482959)(643.22243507,1062.50482942)(643.32244385,1062.68483989)
\curveto(643.44243485,1062.86482906)(643.51243478,1063.07482885)(643.53244385,1063.31483989)
\curveto(643.55243474,1063.55482837)(643.56243473,1063.84482808)(643.56244385,1064.18483989)
\curveto(643.56243473,1065.76482616)(642.70243559,1066.55482537)(640.98244385,1066.55483989)
\curveto(640.28243801,1066.55482537)(639.74243855,1066.41482551)(639.36244385,1066.13483989)
\curveto(639.00243929,1065.87482605)(638.73243956,1065.56482636)(638.55244385,1065.20483989)
\curveto(638.3924399,1064.84482708)(638.29244,1064.48482744)(638.25244385,1064.12483989)
\curveto(638.23244006,1063.78482814)(638.22244007,1063.54482838)(638.22244385,1063.40483989)
\lineto(638.22244385,1063.25483989)
\lineto(635.70244385,1063.25483989)
}
}
{
\newrgbcolor{curcolor}{0 0 0}
\pscustom[linestyle=none,fillstyle=solid,fillcolor=curcolor]
{
\newpath
\moveto(649.3932251,1068.11483989)
\lineto(651.9132251,1068.11483989)
\lineto(651.9132251,1065.71483989)
\lineto(651.9732251,1065.71483989)
\curveto(652.15322069,1066.09482583)(652.3432205,1066.45482547)(652.5432251,1066.79483989)
\curveto(652.76322008,1067.13482479)(653.01321983,1067.43482449)(653.2932251,1067.69483989)
\curveto(653.57321927,1067.95482397)(653.88321896,1068.15482377)(654.2232251,1068.29483989)
\curveto(654.58321826,1068.45482347)(654.99321785,1068.53482339)(655.4532251,1068.53483989)
\curveto(655.95321689,1068.53482339)(656.32321652,1068.47482345)(656.5632251,1068.35483989)
\lineto(656.5632251,1065.89483989)
\curveto(656.4432164,1065.91482601)(656.28321656,1065.93482599)(656.0832251,1065.95483989)
\curveto(655.90321694,1065.99482593)(655.61321723,1066.01482591)(655.2132251,1066.01483989)
\curveto(654.89321795,1066.01482591)(654.5432183,1065.93482599)(654.1632251,1065.77483989)
\curveto(653.78321906,1065.63482629)(653.42321942,1065.40482652)(653.0832251,1065.08483989)
\curveto(652.76322008,1064.78482714)(652.48322036,1064.39482753)(652.2432251,1063.91483989)
\curveto(652.02322082,1063.43482849)(651.91322093,1062.86482906)(651.9132251,1062.20483989)
\lineto(651.9132251,1052.03483989)
\lineto(649.3932251,1052.03483989)
\lineto(649.3932251,1068.11483989)
}
}
{
\newrgbcolor{curcolor}{0 0 0}
\pscustom[linestyle=none,fillstyle=solid,fillcolor=curcolor]
{
\newpath
\moveto(660.3994751,1059.77483989)
\curveto(660.39947135,1059.15483277)(660.40947134,1058.48483344)(660.4294751,1057.76483989)
\curveto(660.46947128,1057.04483488)(660.58947116,1056.37483555)(660.7894751,1055.75483989)
\curveto(660.98947076,1055.13483679)(661.28947046,1054.61483731)(661.6894751,1054.19483989)
\curveto(662.10946964,1053.79483813)(662.70946904,1053.59483833)(663.4894751,1053.59483989)
\curveto(664.08946766,1053.59483833)(664.56946718,1053.7248382)(664.9294751,1053.98483989)
\curveto(665.28946646,1054.26483766)(665.55946619,1054.59483733)(665.7394751,1054.97483989)
\curveto(665.93946581,1055.37483655)(666.06946568,1055.78483614)(666.1294751,1056.20483989)
\curveto(666.18946556,1056.64483528)(666.21946553,1057.01483491)(666.2194751,1057.31483989)
\lineto(668.7394751,1057.31483989)
\curveto(668.73946301,1056.89483503)(668.65946309,1056.35483557)(668.4994751,1055.69483989)
\curveto(668.35946339,1055.05483687)(668.08946366,1054.4248375)(667.6894751,1053.80483989)
\curveto(667.28946446,1053.20483872)(666.73946501,1052.68483924)(666.0394751,1052.24483989)
\curveto(665.33946641,1051.8248401)(664.43946731,1051.61484031)(663.3394751,1051.61483989)
\curveto(661.35947039,1051.61484031)(659.92947182,1052.29483963)(659.0494751,1053.65483989)
\curveto(658.18947356,1055.01483691)(657.75947399,1057.08483484)(657.7594751,1059.86483989)
\curveto(657.75947399,1060.86483106)(657.81947393,1061.87483005)(657.9394751,1062.89483989)
\curveto(658.07947367,1063.93482799)(658.3494734,1064.86482706)(658.7494751,1065.68483989)
\curveto(659.16947258,1066.5248254)(659.749472,1067.20482472)(660.4894751,1067.72483989)
\curveto(661.2494705,1068.26482366)(662.2494695,1068.53482339)(663.4894751,1068.53483989)
\curveto(664.70946704,1068.53482339)(665.67946607,1068.29482363)(666.3994751,1067.81483989)
\curveto(667.11946463,1067.33482459)(667.65946409,1066.71482521)(668.0194751,1065.95483989)
\curveto(668.37946337,1065.21482671)(668.60946314,1064.38482754)(668.7094751,1063.46483989)
\curveto(668.80946294,1062.54482938)(668.85946289,1061.65483027)(668.8594751,1060.79483989)
\lineto(668.8594751,1059.77483989)
\lineto(660.3994751,1059.77483989)
\moveto(666.2194751,1061.75483989)
\lineto(666.2194751,1062.62483989)
\curveto(666.21946553,1063.06482886)(666.17946557,1063.51482841)(666.0994751,1063.97483989)
\curveto(666.01946573,1064.45482747)(665.86946588,1064.88482704)(665.6494751,1065.26483989)
\curveto(665.4494663,1065.64482628)(665.16946658,1065.95482597)(664.8094751,1066.19483989)
\curveto(664.4494673,1066.43482549)(663.98946776,1066.55482537)(663.4294751,1066.55483989)
\curveto(662.76946898,1066.55482537)(662.23946951,1066.37482555)(661.8394751,1066.01483989)
\curveto(661.45947029,1065.67482625)(661.16947058,1065.27482665)(660.9694751,1064.81483989)
\curveto(660.76947098,1064.35482757)(660.63947111,1063.88482804)(660.5794751,1063.40483989)
\curveto(660.51947123,1062.94482898)(660.48947126,1062.59482933)(660.4894751,1062.35483989)
\lineto(660.4894751,1061.75483989)
\lineto(666.2194751,1061.75483989)
}
}
{
\newrgbcolor{curcolor}{0 0 0}
\pscustom[linestyle=none,fillstyle=solid,fillcolor=curcolor]
{
\newpath
\moveto(678.98025635,1060.79483989)
\curveto(678.74024756,1060.55483137)(678.43024787,1060.35483157)(678.05025635,1060.19483989)
\curveto(677.69024861,1060.03483189)(677.300249,1059.88483204)(676.88025635,1059.74483989)
\curveto(676.48024982,1059.6248323)(676.08025022,1059.49483243)(675.68025635,1059.35483989)
\curveto(675.300251,1059.23483269)(674.97025133,1059.08483284)(674.69025635,1058.90483989)
\curveto(674.29025201,1058.64483328)(673.98025232,1058.3248336)(673.76025635,1057.94483989)
\curveto(673.56025274,1057.58483434)(673.46025284,1057.05483487)(673.46025635,1056.35483989)
\curveto(673.46025284,1055.51483641)(673.62025268,1054.84483708)(673.94025635,1054.34483989)
\curveto(674.28025202,1053.84483808)(674.88025142,1053.59483833)(675.74025635,1053.59483989)
\curveto(676.16025014,1053.59483833)(676.56024974,1053.67483825)(676.94025635,1053.83483989)
\curveto(677.34024896,1053.99483793)(677.69024861,1054.21483771)(677.99025635,1054.49483989)
\curveto(678.29024801,1054.77483715)(678.53024777,1055.09483683)(678.71025635,1055.45483989)
\curveto(678.89024741,1055.83483609)(678.98024732,1056.23483569)(678.98025635,1056.65483989)
\lineto(678.98025635,1060.79483989)
\moveto(671.21025635,1063.25483989)
\curveto(671.21025509,1065.07482685)(671.63025467,1066.40482552)(672.47025635,1067.24483989)
\curveto(673.31025299,1068.10482382)(674.69025161,1068.53482339)(676.61025635,1068.53483989)
\curveto(677.83024847,1068.53482339)(678.77024753,1068.37482355)(679.43025635,1068.05483989)
\curveto(680.09024621,1067.73482419)(680.57024573,1067.33482459)(680.87025635,1066.85483989)
\curveto(681.19024511,1066.37482555)(681.37024493,1065.86482606)(681.41025635,1065.32483989)
\curveto(681.47024483,1064.80482712)(681.5002448,1064.33482759)(681.50025635,1063.91483989)
\lineto(681.50025635,1054.94483989)
\curveto(681.5002448,1054.60483732)(681.53024477,1054.30483762)(681.59025635,1054.04483989)
\curveto(681.65024465,1053.78483814)(681.88024442,1053.65483827)(682.28025635,1053.65483989)
\curveto(682.44024386,1053.65483827)(682.56024374,1053.66483826)(682.64025635,1053.68483989)
\curveto(682.74024356,1053.7248382)(682.82024348,1053.76483816)(682.88025635,1053.80483989)
\lineto(682.88025635,1052.00483989)
\curveto(682.78024352,1051.98483994)(682.59024371,1051.95483997)(682.31025635,1051.91483989)
\curveto(682.03024427,1051.87484005)(681.73024457,1051.85484007)(681.41025635,1051.85483989)
\curveto(681.17024513,1051.85484007)(680.92024538,1051.86484006)(680.66025635,1051.88483989)
\curveto(680.42024588,1051.90484002)(680.18024612,1051.97483995)(679.94025635,1052.09483989)
\curveto(679.72024658,1052.23483969)(679.53024677,1052.44483948)(679.37025635,1052.72483989)
\curveto(679.23024707,1053.00483892)(679.15024715,1053.40483852)(679.13025635,1053.92483989)
\lineto(679.07025635,1053.92483989)
\curveto(678.67024763,1053.20483872)(678.11024819,1052.63483929)(677.39025635,1052.21483989)
\curveto(676.69024961,1051.81484011)(675.96025034,1051.61484031)(675.20025635,1051.61483989)
\curveto(673.7002526,1051.61484031)(672.59025371,1052.03483989)(671.87025635,1052.87483989)
\curveto(671.17025513,1053.71483821)(670.82025548,1054.85483707)(670.82025635,1056.29483989)
\curveto(670.82025548,1057.43483449)(671.06025524,1058.37483355)(671.54025635,1059.11483989)
\curveto(672.04025426,1059.87483205)(672.81025349,1060.41483151)(673.85025635,1060.73483989)
\lineto(677.24025635,1061.75483989)
\curveto(677.7002486,1061.89483003)(678.05024825,1062.03482989)(678.29025635,1062.17483989)
\curveto(678.55024775,1062.33482959)(678.73024757,1062.50482942)(678.83025635,1062.68483989)
\curveto(678.95024735,1062.86482906)(679.02024728,1063.07482885)(679.04025635,1063.31483989)
\curveto(679.06024724,1063.55482837)(679.07024723,1063.84482808)(679.07025635,1064.18483989)
\curveto(679.07024723,1065.76482616)(678.21024809,1066.55482537)(676.49025635,1066.55483989)
\curveto(675.79025051,1066.55482537)(675.25025105,1066.41482551)(674.87025635,1066.13483989)
\curveto(674.51025179,1065.87482605)(674.24025206,1065.56482636)(674.06025635,1065.20483989)
\curveto(673.9002524,1064.84482708)(673.8002525,1064.48482744)(673.76025635,1064.12483989)
\curveto(673.74025256,1063.78482814)(673.73025257,1063.54482838)(673.73025635,1063.40483989)
\lineto(673.73025635,1063.25483989)
\lineto(671.21025635,1063.25483989)
}
}
{
\newrgbcolor{curcolor}{0 0 0}
\pscustom[linewidth=2,linecolor=curcolor]
{
\newpath
\moveto(539.05914307,962.45398836)
\lineto(779.13363647,962.45398836)
\lineto(779.13363647,914.3033536)
\lineto(539.05914307,914.3033536)
\closepath
}
}
{
\newrgbcolor{curcolor}{0 0 0}
\pscustom[linestyle=none,fillstyle=solid,fillcolor=curcolor]
{
\newpath
\moveto(595.86764648,939.1886664)
\curveto(595.62763769,938.94865788)(595.317638,938.74865808)(594.93764648,938.5886664)
\curveto(594.57763874,938.4286584)(594.18763913,938.27865855)(593.76764648,938.1386664)
\curveto(593.36763995,938.01865881)(592.96764035,937.88865894)(592.56764648,937.7486664)
\curveto(592.18764113,937.6286592)(591.85764146,937.47865935)(591.57764648,937.2986664)
\curveto(591.17764214,937.03865979)(590.86764245,936.71866011)(590.64764648,936.3386664)
\curveto(590.44764287,935.97866085)(590.34764297,935.44866138)(590.34764648,934.7486664)
\curveto(590.34764297,933.90866292)(590.50764281,933.23866359)(590.82764648,932.7386664)
\curveto(591.16764215,932.23866459)(591.76764155,931.98866484)(592.62764648,931.9886664)
\curveto(593.04764027,931.98866484)(593.44763987,932.06866476)(593.82764648,932.2286664)
\curveto(594.22763909,932.38866444)(594.57763874,932.60866422)(594.87764648,932.8886664)
\curveto(595.17763814,933.16866366)(595.4176379,933.48866334)(595.59764648,933.8486664)
\curveto(595.77763754,934.2286626)(595.86763745,934.6286622)(595.86764648,935.0486664)
\lineto(595.86764648,939.1886664)
\moveto(588.09764648,941.6486664)
\curveto(588.09764522,943.46865336)(588.5176448,944.79865203)(589.35764648,945.6386664)
\curveto(590.19764312,946.49865033)(591.57764174,946.9286499)(593.49764648,946.9286664)
\curveto(594.7176386,946.9286499)(595.65763766,946.76865006)(596.31764648,946.4486664)
\curveto(596.97763634,946.1286507)(597.45763586,945.7286511)(597.75764648,945.2486664)
\curveto(598.07763524,944.76865206)(598.25763506,944.25865257)(598.29764648,943.7186664)
\curveto(598.35763496,943.19865363)(598.38763493,942.7286541)(598.38764648,942.3086664)
\lineto(598.38764648,933.3386664)
\curveto(598.38763493,932.99866383)(598.4176349,932.69866413)(598.47764648,932.4386664)
\curveto(598.53763478,932.17866465)(598.76763455,932.04866478)(599.16764648,932.0486664)
\curveto(599.32763399,932.04866478)(599.44763387,932.05866477)(599.52764648,932.0786664)
\curveto(599.62763369,932.11866471)(599.70763361,932.15866467)(599.76764648,932.1986664)
\lineto(599.76764648,930.3986664)
\curveto(599.66763365,930.37866645)(599.47763384,930.34866648)(599.19764648,930.3086664)
\curveto(598.9176344,930.26866656)(598.6176347,930.24866658)(598.29764648,930.2486664)
\curveto(598.05763526,930.24866658)(597.80763551,930.25866657)(597.54764648,930.2786664)
\curveto(597.30763601,930.29866653)(597.06763625,930.36866646)(596.82764648,930.4886664)
\curveto(596.60763671,930.6286662)(596.4176369,930.83866599)(596.25764648,931.1186664)
\curveto(596.1176372,931.39866543)(596.03763728,931.79866503)(596.01764648,932.3186664)
\lineto(595.95764648,932.3186664)
\curveto(595.55763776,931.59866523)(594.99763832,931.0286658)(594.27764648,930.6086664)
\curveto(593.57763974,930.20866662)(592.84764047,930.00866682)(592.08764648,930.0086664)
\curveto(590.58764273,930.00866682)(589.47764384,930.4286664)(588.75764648,931.2686664)
\curveto(588.05764526,932.10866472)(587.70764561,933.24866358)(587.70764648,934.6886664)
\curveto(587.70764561,935.828661)(587.94764537,936.76866006)(588.42764648,937.5086664)
\curveto(588.92764439,938.26865856)(589.69764362,938.80865802)(590.73764648,939.1286664)
\lineto(594.12764648,940.1486664)
\curveto(594.58763873,940.28865654)(594.93763838,940.4286564)(595.17764648,940.5686664)
\curveto(595.43763788,940.7286561)(595.6176377,940.89865593)(595.71764648,941.0786664)
\curveto(595.83763748,941.25865557)(595.90763741,941.46865536)(595.92764648,941.7086664)
\curveto(595.94763737,941.94865488)(595.95763736,942.23865459)(595.95764648,942.5786664)
\curveto(595.95763736,944.15865267)(595.09763822,944.94865188)(593.37764648,944.9486664)
\curveto(592.67764064,944.94865188)(592.13764118,944.80865202)(591.75764648,944.5286664)
\curveto(591.39764192,944.26865256)(591.12764219,943.95865287)(590.94764648,943.5986664)
\curveto(590.78764253,943.23865359)(590.68764263,942.87865395)(590.64764648,942.5186664)
\curveto(590.62764269,942.17865465)(590.6176427,941.93865489)(590.61764648,941.7986664)
\lineto(590.61764648,941.6486664)
\lineto(588.09764648,941.6486664)
}
}
{
\newrgbcolor{curcolor}{0 0 0}
\pscustom[linestyle=none,fillstyle=solid,fillcolor=curcolor]
{
\newpath
\moveto(601.78842773,946.5086664)
\lineto(604.30842773,946.5086664)
\lineto(604.30842773,944.1086664)
\lineto(604.36842773,944.1086664)
\curveto(604.54842332,944.48865234)(604.73842313,944.84865198)(604.93842773,945.1886664)
\curveto(605.15842271,945.5286513)(605.40842246,945.828651)(605.68842773,946.0886664)
\curveto(605.9684219,946.34865048)(606.27842159,946.54865028)(606.61842773,946.6886664)
\curveto(606.97842089,946.84864998)(607.38842048,946.9286499)(607.84842773,946.9286664)
\curveto(608.34841952,946.9286499)(608.71841915,946.86864996)(608.95842773,946.7486664)
\lineto(608.95842773,944.2886664)
\curveto(608.83841903,944.30865252)(608.67841919,944.3286525)(608.47842773,944.3486664)
\curveto(608.29841957,944.38865244)(608.00841986,944.40865242)(607.60842773,944.4086664)
\curveto(607.28842058,944.40865242)(606.93842093,944.3286525)(606.55842773,944.1686664)
\curveto(606.17842169,944.0286528)(605.81842205,943.79865303)(605.47842773,943.4786664)
\curveto(605.15842271,943.17865365)(604.87842299,942.78865404)(604.63842773,942.3086664)
\curveto(604.41842345,941.828655)(604.30842356,941.25865557)(604.30842773,940.5986664)
\lineto(604.30842773,930.4286664)
\lineto(601.78842773,930.4286664)
\lineto(601.78842773,946.5086664)
}
}
{
\newrgbcolor{curcolor}{0 0 0}
\pscustom[linestyle=none,fillstyle=solid,fillcolor=curcolor]
{
\newpath
\moveto(612.79467773,938.1686664)
\curveto(612.79467398,937.54865928)(612.80467397,936.87865995)(612.82467773,936.1586664)
\curveto(612.86467391,935.43866139)(612.98467379,934.76866206)(613.18467773,934.1486664)
\curveto(613.38467339,933.5286633)(613.68467309,933.00866382)(614.08467773,932.5886664)
\curveto(614.50467227,932.18866464)(615.10467167,931.98866484)(615.88467773,931.9886664)
\curveto(616.48467029,931.98866484)(616.96466981,932.11866471)(617.32467773,932.3786664)
\curveto(617.68466909,932.65866417)(617.95466882,932.98866384)(618.13467773,933.3686664)
\curveto(618.33466844,933.76866306)(618.46466831,934.17866265)(618.52467773,934.5986664)
\curveto(618.58466819,935.03866179)(618.61466816,935.40866142)(618.61467773,935.7086664)
\lineto(621.13467773,935.7086664)
\curveto(621.13466564,935.28866154)(621.05466572,934.74866208)(620.89467773,934.0886664)
\curveto(620.75466602,933.44866338)(620.48466629,932.81866401)(620.08467773,932.1986664)
\curveto(619.68466709,931.59866523)(619.13466764,931.07866575)(618.43467773,930.6386664)
\curveto(617.73466904,930.21866661)(616.83466994,930.00866682)(615.73467773,930.0086664)
\curveto(613.75467302,930.00866682)(612.32467445,930.68866614)(611.44467773,932.0486664)
\curveto(610.58467619,933.40866342)(610.15467662,935.47866135)(610.15467773,938.2586664)
\curveto(610.15467662,939.25865757)(610.21467656,940.26865656)(610.33467773,941.2886664)
\curveto(610.4746763,942.3286545)(610.74467603,943.25865357)(611.14467773,944.0786664)
\curveto(611.56467521,944.91865191)(612.14467463,945.59865123)(612.88467773,946.1186664)
\curveto(613.64467313,946.65865017)(614.64467213,946.9286499)(615.88467773,946.9286664)
\curveto(617.10466967,946.9286499)(618.0746687,946.68865014)(618.79467773,946.2086664)
\curveto(619.51466726,945.7286511)(620.05466672,945.10865172)(620.41467773,944.3486664)
\curveto(620.774666,943.60865322)(621.00466577,942.77865405)(621.10467773,941.8586664)
\curveto(621.20466557,940.93865589)(621.25466552,940.04865678)(621.25467773,939.1886664)
\lineto(621.25467773,938.1686664)
\lineto(612.79467773,938.1686664)
\moveto(618.61467773,940.1486664)
\lineto(618.61467773,941.0186664)
\curveto(618.61466816,941.45865537)(618.5746682,941.90865492)(618.49467773,942.3686664)
\curveto(618.41466836,942.84865398)(618.26466851,943.27865355)(618.04467773,943.6586664)
\curveto(617.84466893,944.03865279)(617.56466921,944.34865248)(617.20467773,944.5886664)
\curveto(616.84466993,944.828652)(616.38467039,944.94865188)(615.82467773,944.9486664)
\curveto(615.16467161,944.94865188)(614.63467214,944.76865206)(614.23467773,944.4086664)
\curveto(613.85467292,944.06865276)(613.56467321,943.66865316)(613.36467773,943.2086664)
\curveto(613.16467361,942.74865408)(613.03467374,942.27865455)(612.97467773,941.7986664)
\curveto(612.91467386,941.33865549)(612.88467389,940.98865584)(612.88467773,940.7486664)
\lineto(612.88467773,940.1486664)
\lineto(618.61467773,940.1486664)
}
}
{
\newrgbcolor{curcolor}{0 0 0}
\pscustom[linestyle=none,fillstyle=solid,fillcolor=curcolor]
{
\newpath
\moveto(631.37545898,939.1886664)
\curveto(631.13545019,938.94865788)(630.8254505,938.74865808)(630.44545898,938.5886664)
\curveto(630.08545124,938.4286584)(629.69545163,938.27865855)(629.27545898,938.1386664)
\curveto(628.87545245,938.01865881)(628.47545285,937.88865894)(628.07545898,937.7486664)
\curveto(627.69545363,937.6286592)(627.36545396,937.47865935)(627.08545898,937.2986664)
\curveto(626.68545464,937.03865979)(626.37545495,936.71866011)(626.15545898,936.3386664)
\curveto(625.95545537,935.97866085)(625.85545547,935.44866138)(625.85545898,934.7486664)
\curveto(625.85545547,933.90866292)(626.01545531,933.23866359)(626.33545898,932.7386664)
\curveto(626.67545465,932.23866459)(627.27545405,931.98866484)(628.13545898,931.9886664)
\curveto(628.55545277,931.98866484)(628.95545237,932.06866476)(629.33545898,932.2286664)
\curveto(629.73545159,932.38866444)(630.08545124,932.60866422)(630.38545898,932.8886664)
\curveto(630.68545064,933.16866366)(630.9254504,933.48866334)(631.10545898,933.8486664)
\curveto(631.28545004,934.2286626)(631.37544995,934.6286622)(631.37545898,935.0486664)
\lineto(631.37545898,939.1886664)
\moveto(623.60545898,941.6486664)
\curveto(623.60545772,943.46865336)(624.0254573,944.79865203)(624.86545898,945.6386664)
\curveto(625.70545562,946.49865033)(627.08545424,946.9286499)(629.00545898,946.9286664)
\curveto(630.2254511,946.9286499)(631.16545016,946.76865006)(631.82545898,946.4486664)
\curveto(632.48544884,946.1286507)(632.96544836,945.7286511)(633.26545898,945.2486664)
\curveto(633.58544774,944.76865206)(633.76544756,944.25865257)(633.80545898,943.7186664)
\curveto(633.86544746,943.19865363)(633.89544743,942.7286541)(633.89545898,942.3086664)
\lineto(633.89545898,933.3386664)
\curveto(633.89544743,932.99866383)(633.9254474,932.69866413)(633.98545898,932.4386664)
\curveto(634.04544728,932.17866465)(634.27544705,932.04866478)(634.67545898,932.0486664)
\curveto(634.83544649,932.04866478)(634.95544637,932.05866477)(635.03545898,932.0786664)
\curveto(635.13544619,932.11866471)(635.21544611,932.15866467)(635.27545898,932.1986664)
\lineto(635.27545898,930.3986664)
\curveto(635.17544615,930.37866645)(634.98544634,930.34866648)(634.70545898,930.3086664)
\curveto(634.4254469,930.26866656)(634.1254472,930.24866658)(633.80545898,930.2486664)
\curveto(633.56544776,930.24866658)(633.31544801,930.25866657)(633.05545898,930.2786664)
\curveto(632.81544851,930.29866653)(632.57544875,930.36866646)(632.33545898,930.4886664)
\curveto(632.11544921,930.6286662)(631.9254494,930.83866599)(631.76545898,931.1186664)
\curveto(631.6254497,931.39866543)(631.54544978,931.79866503)(631.52545898,932.3186664)
\lineto(631.46545898,932.3186664)
\curveto(631.06545026,931.59866523)(630.50545082,931.0286658)(629.78545898,930.6086664)
\curveto(629.08545224,930.20866662)(628.35545297,930.00866682)(627.59545898,930.0086664)
\curveto(626.09545523,930.00866682)(624.98545634,930.4286664)(624.26545898,931.2686664)
\curveto(623.56545776,932.10866472)(623.21545811,933.24866358)(623.21545898,934.6886664)
\curveto(623.21545811,935.828661)(623.45545787,936.76866006)(623.93545898,937.5086664)
\curveto(624.43545689,938.26865856)(625.20545612,938.80865802)(626.24545898,939.1286664)
\lineto(629.63545898,940.1486664)
\curveto(630.09545123,940.28865654)(630.44545088,940.4286564)(630.68545898,940.5686664)
\curveto(630.94545038,940.7286561)(631.1254502,940.89865593)(631.22545898,941.0786664)
\curveto(631.34544998,941.25865557)(631.41544991,941.46865536)(631.43545898,941.7086664)
\curveto(631.45544987,941.94865488)(631.46544986,942.23865459)(631.46545898,942.5786664)
\curveto(631.46544986,944.15865267)(630.60545072,944.94865188)(628.88545898,944.9486664)
\curveto(628.18545314,944.94865188)(627.64545368,944.80865202)(627.26545898,944.5286664)
\curveto(626.90545442,944.26865256)(626.63545469,943.95865287)(626.45545898,943.5986664)
\curveto(626.29545503,943.23865359)(626.19545513,942.87865395)(626.15545898,942.5186664)
\curveto(626.13545519,942.17865465)(626.1254552,941.93865489)(626.12545898,941.7986664)
\lineto(626.12545898,941.6486664)
\lineto(623.60545898,941.6486664)
}
}
{
\newrgbcolor{curcolor}{0 0 0}
\pscustom[linestyle=none,fillstyle=solid,fillcolor=curcolor]
{
\newpath
\moveto(635.64624023,928.1786664)
\lineto(650.64624023,928.1786664)
\lineto(650.64624023,926.6786664)
\lineto(635.64624023,926.6786664)
\lineto(635.64624023,928.1786664)
}
}
{
\newrgbcolor{curcolor}{0 0 0}
\pscustom[linestyle=none,fillstyle=solid,fillcolor=curcolor]
{
\newpath
\moveto(659.19624023,941.7986664)
\curveto(659.19623168,942.81865401)(659.02623185,943.59865323)(658.68624023,944.1386664)
\curveto(658.36623251,944.67865215)(657.74623313,944.94865188)(656.82624023,944.9486664)
\curveto(656.62623425,944.94865188)(656.3762345,944.91865191)(656.07624023,944.8586664)
\curveto(655.7762351,944.81865201)(655.48623539,944.71865211)(655.20624023,944.5586664)
\curveto(654.94623593,944.39865243)(654.71623616,944.14865268)(654.51624023,943.8086664)
\curveto(654.31623656,943.48865334)(654.21623666,943.04865378)(654.21624023,942.4886664)
\curveto(654.21623666,942.00865482)(654.32623655,941.61865521)(654.54624023,941.3186664)
\curveto(654.78623609,941.01865581)(655.08623579,940.75865607)(655.44624023,940.5386664)
\curveto(655.82623505,940.33865649)(656.24623463,940.16865666)(656.70624023,940.0286664)
\curveto(657.18623369,939.88865694)(657.6762332,939.73865709)(658.17624023,939.5786664)
\curveto(658.65623222,939.41865741)(659.12623175,939.23865759)(659.58624023,939.0386664)
\curveto(660.06623081,938.85865797)(660.48623039,938.59865823)(660.84624023,938.2586664)
\curveto(661.22622965,937.93865889)(661.52622935,937.51865931)(661.74624023,936.9986664)
\curveto(661.98622889,936.49866033)(662.10622877,935.84866098)(662.10624023,935.0486664)
\curveto(662.10622877,934.20866262)(661.9762289,933.46866336)(661.71624023,932.8286664)
\curveto(661.45622942,932.20866462)(661.09622978,931.68866514)(660.63624023,931.2686664)
\curveto(660.1762307,930.84866598)(659.62623125,930.53866629)(658.98624023,930.3386664)
\curveto(658.34623253,930.11866671)(657.65623322,930.00866682)(656.91624023,930.0086664)
\curveto(655.55623532,930.00866682)(654.49623638,930.2286666)(653.73624023,930.6686664)
\curveto(652.99623788,931.10866572)(652.44623843,931.6286652)(652.08624023,932.2286664)
\curveto(651.74623913,932.84866398)(651.54623933,933.47866335)(651.48624023,934.1186664)
\curveto(651.42623945,934.75866207)(651.39623948,935.28866154)(651.39624023,935.7086664)
\lineto(653.91624023,935.7086664)
\curveto(653.91623696,935.2286616)(653.95623692,934.75866207)(654.03624023,934.2986664)
\curveto(654.11623676,933.85866297)(654.26623661,933.45866337)(654.48624023,933.0986664)
\curveto(654.70623617,932.75866407)(655.00623587,932.48866434)(655.38624023,932.2886664)
\curveto(655.78623509,932.08866474)(656.29623458,931.98866484)(656.91624023,931.9886664)
\curveto(657.11623376,931.98866484)(657.36623351,932.01866481)(657.66624023,932.0786664)
\curveto(657.96623291,932.13866469)(658.25623262,932.26866456)(658.53624023,932.4686664)
\curveto(658.83623204,932.66866416)(659.08623179,932.93866389)(659.28624023,933.2786664)
\curveto(659.48623139,933.61866321)(659.58623129,934.07866275)(659.58624023,934.6586664)
\curveto(659.58623129,935.19866163)(659.46623141,935.63866119)(659.22624023,935.9786664)
\curveto(659.00623187,936.33866049)(658.70623217,936.6286602)(658.32624023,936.8486664)
\curveto(657.96623291,937.08865974)(657.54623333,937.28865954)(657.06624023,937.4486664)
\curveto(656.60623427,937.60865922)(656.13623474,937.76865906)(655.65624023,937.9286664)
\curveto(655.1762357,938.08865874)(654.69623618,938.25865857)(654.21624023,938.4386664)
\curveto(653.73623714,938.63865819)(653.30623757,938.89865793)(652.92624023,939.2186664)
\curveto(652.56623831,939.53865729)(652.26623861,939.95865687)(652.02624023,940.4786664)
\curveto(651.80623907,940.99865583)(651.69623918,941.66865516)(651.69624023,942.4886664)
\curveto(651.69623918,943.2286536)(651.82623905,943.87865295)(652.08624023,944.4386664)
\curveto(652.36623851,944.99865183)(652.73623814,945.45865137)(653.19624023,945.8186664)
\curveto(653.6762372,946.19865063)(654.22623665,946.47865035)(654.84624023,946.6586664)
\curveto(655.46623541,946.83864999)(656.12623475,946.9286499)(656.82624023,946.9286664)
\curveto(657.98623289,946.9286499)(658.89623198,946.74865008)(659.55624023,946.3886664)
\curveto(660.21623066,946.0286508)(660.70623017,945.58865124)(661.02624023,945.0686664)
\curveto(661.34622953,944.54865228)(661.53622934,943.98865284)(661.59624023,943.3886664)
\curveto(661.6762292,942.80865402)(661.71622916,942.27865455)(661.71624023,941.7986664)
\lineto(659.19624023,941.7986664)
}
}
{
\newrgbcolor{curcolor}{0 0 0}
\pscustom[linestyle=none,fillstyle=solid,fillcolor=curcolor]
{
\newpath
\moveto(675.07374023,930.4286664)
\lineto(672.67374023,930.4286664)
\lineto(672.67374023,932.3186664)
\lineto(672.61374023,932.3186664)
\curveto(672.27373079,931.57866525)(671.73373133,931.00866582)(670.99374023,930.6086664)
\curveto(670.25373281,930.20866662)(669.49373357,930.00866682)(668.71374023,930.0086664)
\curveto(667.65373541,930.00866682)(666.83373623,930.17866665)(666.25374023,930.5186664)
\curveto(665.69373737,930.87866595)(665.27373779,931.31866551)(664.99374023,931.8386664)
\curveto(664.73373833,932.35866447)(664.58373848,932.90866392)(664.54374023,933.4886664)
\curveto(664.50373856,934.08866274)(664.48373858,934.6286622)(664.48374023,935.1086664)
\lineto(664.48374023,946.5086664)
\lineto(667.00374023,946.5086664)
\lineto(667.00374023,935.4086664)
\curveto(667.00373606,935.10866172)(667.01373605,934.76866206)(667.03374023,934.3886664)
\curveto(667.07373599,934.00866282)(667.17373589,933.64866318)(667.33374023,933.3086664)
\curveto(667.49373557,932.98866384)(667.72373534,932.71866411)(668.02374023,932.4986664)
\curveto(668.34373472,932.27866455)(668.79373427,932.16866466)(669.37374023,932.1686664)
\curveto(669.71373335,932.16866466)(670.063733,932.2286646)(670.42374023,932.3486664)
\curveto(670.80373226,932.46866436)(671.15373191,932.65866417)(671.47374023,932.9186664)
\curveto(671.79373127,933.17866365)(672.05373101,933.50866332)(672.25374023,933.9086664)
\curveto(672.45373061,934.3286625)(672.55373051,934.828662)(672.55374023,935.4086664)
\lineto(672.55374023,946.5086664)
\lineto(675.07374023,946.5086664)
\lineto(675.07374023,930.4286664)
}
}
{
\newrgbcolor{curcolor}{0 0 0}
\pscustom[linestyle=none,fillstyle=solid,fillcolor=curcolor]
{
\newpath
\moveto(678.37045898,951.8486664)
\lineto(680.89045898,951.8486664)
\lineto(680.89045898,944.6786664)
\lineto(680.95045898,944.6786664)
\curveto(681.23045447,945.37865145)(681.71045399,945.9286509)(682.39045898,946.3286664)
\curveto(683.07045263,946.7286501)(683.83045187,946.9286499)(684.67045898,946.9286664)
\curveto(685.75044995,946.9286499)(686.61044909,946.63865019)(687.25045898,946.0586664)
\curveto(687.91044779,945.49865133)(688.41044729,944.79865203)(688.75045898,943.9586664)
\curveto(689.09044661,943.11865371)(689.31044639,942.19865463)(689.41045898,941.1986664)
\curveto(689.51044619,940.21865661)(689.56044614,939.30865752)(689.56045898,938.4686664)
\curveto(689.56044614,937.3286595)(689.46044624,936.24866058)(689.26045898,935.2286664)
\curveto(689.06044664,934.20866262)(688.75044695,933.30866352)(688.33045898,932.5286664)
\curveto(687.91044779,931.76866506)(687.36044834,931.15866567)(686.68045898,930.6986664)
\curveto(686.02044968,930.23866659)(685.23045047,930.00866682)(684.31045898,930.0086664)
\curveto(683.87045183,930.00866682)(683.46045224,930.07866675)(683.08045898,930.2186664)
\curveto(682.700453,930.35866647)(682.35045335,930.53866629)(682.03045898,930.7586664)
\curveto(681.73045397,930.97866585)(681.47045423,931.2286656)(681.25045898,931.5086664)
\curveto(681.05045465,931.80866502)(680.91045479,932.10866472)(680.83045898,932.4086664)
\lineto(680.77045898,932.4086664)
\lineto(680.77045898,930.4286664)
\lineto(678.37045898,930.4286664)
\lineto(678.37045898,951.8486664)
\moveto(680.74045898,938.4686664)
\curveto(680.74045496,937.64865918)(680.78045492,936.85865997)(680.86045898,936.0986664)
\curveto(680.94045476,935.33866149)(681.1004546,934.66866216)(681.34045898,934.0886664)
\curveto(681.58045412,933.50866332)(681.91045379,933.03866379)(682.33045898,932.6786664)
\curveto(682.75045295,932.33866449)(683.3004524,932.16866466)(683.98045898,932.1686664)
\curveto(684.56045114,932.16866466)(685.04045066,932.31866451)(685.42045898,932.6186664)
\curveto(685.8004499,932.93866389)(686.1004496,933.37866345)(686.32045898,933.9386664)
\curveto(686.54044916,934.49866233)(686.69044901,935.15866167)(686.77045898,935.9186664)
\curveto(686.87044883,936.69866013)(686.92044878,937.54865928)(686.92045898,938.4686664)
\curveto(686.92044878,939.4286574)(686.87044883,940.29865653)(686.77045898,941.0786664)
\curveto(686.69044901,941.85865497)(686.54044916,942.51865431)(686.32045898,943.0586664)
\curveto(686.1004496,943.59865323)(685.8004499,944.01865281)(685.42045898,944.3186664)
\curveto(685.04045066,944.61865221)(684.56045114,944.76865206)(683.98045898,944.7686664)
\curveto(683.3004524,944.76865206)(682.75045295,944.58865224)(682.33045898,944.2286664)
\curveto(681.91045379,943.86865296)(681.58045412,943.38865344)(681.34045898,942.7886664)
\curveto(681.1004546,942.18865464)(680.94045476,941.50865532)(680.86045898,940.7486664)
\curveto(680.78045492,940.00865682)(680.74045496,939.24865758)(680.74045898,938.4686664)
}
}
{
\newrgbcolor{curcolor}{0 0 0}
\pscustom[linestyle=none,fillstyle=solid,fillcolor=curcolor]
{
\newpath
\moveto(694.92717773,948.9686664)
\lineto(692.40717773,948.9686664)
\lineto(692.40717773,951.8486664)
\lineto(694.92717773,951.8486664)
\lineto(694.92717773,948.9686664)
\moveto(694.92717773,928.8686664)
\curveto(694.92717341,927.54866928)(694.62717371,926.55867027)(694.02717773,925.8986664)
\curveto(693.44717489,925.23867159)(692.46717587,924.90867192)(691.08717773,924.9086664)
\curveto(690.88717745,924.90867192)(690.69717764,924.91867191)(690.51717773,924.9386664)
\lineto(689.91717773,924.9986664)
\lineto(689.91717773,927.1586664)
\lineto(690.39717773,927.0986664)
\curveto(690.5371778,927.07866975)(690.68717765,927.06866976)(690.84717773,927.0686664)
\curveto(691.22717711,927.06866976)(691.52717681,927.13866969)(691.74717773,927.2786664)
\curveto(691.96717637,927.41866941)(692.11717622,927.60866922)(692.19717773,927.8486664)
\curveto(692.29717604,928.08866874)(692.35717598,928.37866845)(692.37717773,928.7186664)
\curveto(692.39717594,929.03866779)(692.40717593,929.38866744)(692.40717773,929.7686664)
\lineto(692.40717773,946.5086664)
\lineto(694.92717773,946.5086664)
\lineto(694.92717773,928.8686664)
}
}
{
\newrgbcolor{curcolor}{0 0 0}
\pscustom[linestyle=none,fillstyle=solid,fillcolor=curcolor]
{
\newpath
\moveto(700.45092773,938.1686664)
\curveto(700.45092398,937.54865928)(700.46092397,936.87865995)(700.48092773,936.1586664)
\curveto(700.52092391,935.43866139)(700.64092379,934.76866206)(700.84092773,934.1486664)
\curveto(701.04092339,933.5286633)(701.34092309,933.00866382)(701.74092773,932.5886664)
\curveto(702.16092227,932.18866464)(702.76092167,931.98866484)(703.54092773,931.9886664)
\curveto(704.14092029,931.98866484)(704.62091981,932.11866471)(704.98092773,932.3786664)
\curveto(705.34091909,932.65866417)(705.61091882,932.98866384)(705.79092773,933.3686664)
\curveto(705.99091844,933.76866306)(706.12091831,934.17866265)(706.18092773,934.5986664)
\curveto(706.24091819,935.03866179)(706.27091816,935.40866142)(706.27092773,935.7086664)
\lineto(708.79092773,935.7086664)
\curveto(708.79091564,935.28866154)(708.71091572,934.74866208)(708.55092773,934.0886664)
\curveto(708.41091602,933.44866338)(708.14091629,932.81866401)(707.74092773,932.1986664)
\curveto(707.34091709,931.59866523)(706.79091764,931.07866575)(706.09092773,930.6386664)
\curveto(705.39091904,930.21866661)(704.49091994,930.00866682)(703.39092773,930.0086664)
\curveto(701.41092302,930.00866682)(699.98092445,930.68866614)(699.10092773,932.0486664)
\curveto(698.24092619,933.40866342)(697.81092662,935.47866135)(697.81092773,938.2586664)
\curveto(697.81092662,939.25865757)(697.87092656,940.26865656)(697.99092773,941.2886664)
\curveto(698.1309263,942.3286545)(698.40092603,943.25865357)(698.80092773,944.0786664)
\curveto(699.22092521,944.91865191)(699.80092463,945.59865123)(700.54092773,946.1186664)
\curveto(701.30092313,946.65865017)(702.30092213,946.9286499)(703.54092773,946.9286664)
\curveto(704.76091967,946.9286499)(705.7309187,946.68865014)(706.45092773,946.2086664)
\curveto(707.17091726,945.7286511)(707.71091672,945.10865172)(708.07092773,944.3486664)
\curveto(708.430916,943.60865322)(708.66091577,942.77865405)(708.76092773,941.8586664)
\curveto(708.86091557,940.93865589)(708.91091552,940.04865678)(708.91092773,939.1886664)
\lineto(708.91092773,938.1686664)
\lineto(700.45092773,938.1686664)
\moveto(706.27092773,940.1486664)
\lineto(706.27092773,941.0186664)
\curveto(706.27091816,941.45865537)(706.2309182,941.90865492)(706.15092773,942.3686664)
\curveto(706.07091836,942.84865398)(705.92091851,943.27865355)(705.70092773,943.6586664)
\curveto(705.50091893,944.03865279)(705.22091921,944.34865248)(704.86092773,944.5886664)
\curveto(704.50091993,944.828652)(704.04092039,944.94865188)(703.48092773,944.9486664)
\curveto(702.82092161,944.94865188)(702.29092214,944.76865206)(701.89092773,944.4086664)
\curveto(701.51092292,944.06865276)(701.22092321,943.66865316)(701.02092773,943.2086664)
\curveto(700.82092361,942.74865408)(700.69092374,942.27865455)(700.63092773,941.7986664)
\curveto(700.57092386,941.33865549)(700.54092389,940.98865584)(700.54092773,940.7486664)
\lineto(700.54092773,940.1486664)
\lineto(706.27092773,940.1486664)
}
}
{
\newrgbcolor{curcolor}{0 0 0}
\pscustom[linestyle=none,fillstyle=solid,fillcolor=curcolor]
{
\newpath
\moveto(719.30170898,941.6186664)
\curveto(719.30169968,941.99865483)(719.25169973,942.38865444)(719.15170898,942.7886664)
\curveto(719.07169991,943.18865364)(718.93170005,943.54865328)(718.73170898,943.8686664)
\curveto(718.55170043,944.18865264)(718.29170069,944.44865238)(717.95170898,944.6486664)
\curveto(717.63170135,944.84865198)(717.23170175,944.94865188)(716.75170898,944.9486664)
\curveto(716.35170263,944.94865188)(715.96170302,944.87865195)(715.58170898,944.7386664)
\curveto(715.22170376,944.59865223)(714.89170409,944.28865254)(714.59170898,943.8086664)
\curveto(714.29170469,943.34865348)(714.05170493,942.67865415)(713.87170898,941.7986664)
\curveto(713.69170529,940.91865591)(713.60170538,939.74865708)(713.60170898,938.2886664)
\curveto(713.60170538,937.76865906)(713.61170537,937.14865968)(713.63170898,936.4286664)
\curveto(713.67170531,935.70866112)(713.79170519,935.01866181)(713.99170898,934.3586664)
\curveto(714.19170479,933.69866313)(714.49170449,933.13866369)(714.89170898,932.6786664)
\curveto(715.31170367,932.21866461)(715.90170308,931.98866484)(716.66170898,931.9886664)
\curveto(717.20170178,931.98866484)(717.64170134,932.11866471)(717.98170898,932.3786664)
\curveto(718.32170066,932.63866419)(718.59170039,932.95866387)(718.79170898,933.3386664)
\curveto(718.99169999,933.73866309)(719.12169986,934.16866266)(719.18170898,934.6286664)
\curveto(719.26169972,935.10866172)(719.30169968,935.56866126)(719.30170898,936.0086664)
\lineto(721.82170898,936.0086664)
\curveto(721.82169716,935.36866146)(721.73169725,934.69866213)(721.55170898,933.9986664)
\curveto(721.39169759,933.29866353)(721.10169788,932.64866418)(720.68170898,932.0486664)
\curveto(720.2816987,931.46866536)(719.74169924,930.97866585)(719.06170898,930.5786664)
\curveto(718.3817006,930.19866663)(717.54170144,930.00866682)(716.54170898,930.0086664)
\curveto(714.56170442,930.00866682)(713.13170585,930.68866614)(712.25170898,932.0486664)
\curveto(711.39170759,933.40866342)(710.96170802,935.47866135)(710.96170898,938.2586664)
\curveto(710.96170802,939.25865757)(711.02170796,940.26865656)(711.14170898,941.2886664)
\curveto(711.2817077,942.3286545)(711.55170743,943.25865357)(711.95170898,944.0786664)
\curveto(712.37170661,944.91865191)(712.95170603,945.59865123)(713.69170898,946.1186664)
\curveto(714.45170453,946.65865017)(715.45170353,946.9286499)(716.69170898,946.9286664)
\curveto(717.79170119,946.9286499)(718.67170031,946.73865009)(719.33170898,946.3586664)
\curveto(720.01169897,945.97865085)(720.53169845,945.50865132)(720.89170898,944.9486664)
\curveto(721.27169771,944.40865242)(721.52169746,943.828653)(721.64170898,943.2086664)
\curveto(721.76169722,942.60865422)(721.82169716,942.07865475)(721.82170898,941.6186664)
\lineto(719.30170898,941.6186664)
}
}
{
\newrgbcolor{curcolor}{0 0 0}
\pscustom[linestyle=none,fillstyle=solid,fillcolor=curcolor]
{
\newpath
\moveto(725.17514648,951.1886664)
\lineto(727.69514648,951.1886664)
\lineto(727.69514648,946.5086664)
\lineto(730.48514648,946.5086664)
\lineto(730.48514648,944.5286664)
\lineto(727.69514648,944.5286664)
\lineto(727.69514648,934.2086664)
\curveto(727.69514156,933.56866326)(727.80514145,933.10866372)(728.02514648,932.8286664)
\curveto(728.24514101,932.54866428)(728.68514057,932.40866442)(729.34514648,932.4086664)
\curveto(729.62513963,932.40866442)(729.84513941,932.41866441)(730.00514648,932.4386664)
\curveto(730.16513909,932.45866437)(730.31513894,932.47866435)(730.45514648,932.4986664)
\lineto(730.45514648,930.4286664)
\curveto(730.29513896,930.38866644)(730.04513921,930.34866648)(729.70514648,930.3086664)
\curveto(729.36513989,930.26866656)(728.93514032,930.24866658)(728.41514648,930.2486664)
\curveto(727.7551415,930.24866658)(727.21514204,930.30866652)(726.79514648,930.4286664)
\curveto(726.37514288,930.56866626)(726.04514321,930.76866606)(725.80514648,931.0286664)
\curveto(725.56514369,931.30866552)(725.39514386,931.64866518)(725.29514648,932.0486664)
\curveto(725.21514404,932.44866438)(725.17514408,932.90866392)(725.17514648,933.4286664)
\lineto(725.17514648,944.5286664)
\lineto(722.83514648,944.5286664)
\lineto(722.83514648,946.5086664)
\lineto(725.17514648,946.5086664)
\lineto(725.17514648,951.1886664)
}
}
{
\newrgbcolor{curcolor}{0 0 0}
\pscustom[linewidth=2,linecolor=curcolor]
{
\newpath
\moveto(77.92449951,1084.15015154)
\lineto(226.07513428,1084.15015154)
\lineto(226.07513428,1035.99952822)
\lineto(77.92449951,1035.99952822)
\closepath
}
}
{
\newrgbcolor{curcolor}{0 0 0}
\pscustom[linestyle=none,fillstyle=solid,fillcolor=curcolor]
{
\newpath
\moveto(124.31357025,1053.14483988)
\curveto(124.31355801,1051.94484006)(124.17355815,1050.93484107)(123.89357025,1050.11483988)
\curveto(123.63355869,1049.27484273)(123.26355906,1048.59484341)(122.78357025,1048.07483988)
\curveto(122.30356002,1047.55484445)(121.7235606,1047.18484482)(121.04357025,1046.96483988)
\curveto(120.38356194,1046.72484528)(119.65356267,1046.6048454)(118.85357025,1046.60483988)
\curveto(118.61356371,1046.6048454)(118.21356411,1046.62484538)(117.65357025,1046.66483988)
\curveto(117.11356521,1046.7048453)(116.55356577,1046.85484515)(115.97357025,1047.11483988)
\curveto(115.41356691,1047.35484465)(114.89356743,1047.75484425)(114.41357025,1048.31483988)
\curveto(113.95356837,1048.85484315)(113.67356865,1049.62484238)(113.57357025,1050.62483988)
\lineto(116.09357025,1050.62483988)
\curveto(116.15356617,1049.88484212)(116.43356589,1049.36484264)(116.93357025,1049.06483988)
\curveto(117.43356489,1048.74484326)(118.01356431,1048.58484342)(118.67357025,1048.58483988)
\curveto(119.47356285,1048.58484342)(120.08356224,1048.72484328)(120.50357025,1049.00483988)
\curveto(120.94356138,1049.26484274)(121.25356107,1049.58484242)(121.43357025,1049.96483988)
\curveto(121.63356069,1050.34484166)(121.74356058,1050.74484126)(121.76357025,1051.16483988)
\curveto(121.78356054,1051.56484044)(121.79356053,1051.9048401)(121.79357025,1052.18483988)
\lineto(121.79357025,1054.28483988)
\lineto(121.73357025,1054.28483988)
\curveto(121.45356087,1053.6048384)(120.97356135,1053.07483893)(120.29357025,1052.69483988)
\curveto(119.63356269,1052.31483969)(118.90356342,1052.12483988)(118.10357025,1052.12483988)
\curveto(117.323565,1052.12483988)(116.65356567,1052.27483973)(116.09357025,1052.57483988)
\curveto(115.55356677,1052.87483913)(115.09356723,1053.26483874)(114.71357025,1053.74483988)
\curveto(114.35356797,1054.22483778)(114.06356826,1054.75483725)(113.84357025,1055.33483988)
\curveto(113.64356868,1055.93483607)(113.48356884,1056.53483547)(113.36357025,1057.13483988)
\curveto(113.26356906,1057.73483427)(113.19356913,1058.29483371)(113.15357025,1058.81483988)
\curveto(113.13356919,1059.35483265)(113.1235692,1059.8048322)(113.12357025,1060.16483988)
\curveto(113.1235692,1061.24483076)(113.20356912,1062.29482971)(113.36357025,1063.31483988)
\curveto(113.5235688,1064.33482767)(113.80356852,1065.23482677)(114.20357025,1066.01483988)
\curveto(114.6235677,1066.81482519)(115.16356716,1067.44482456)(115.82357025,1067.90483988)
\curveto(116.50356582,1068.38482362)(117.35356497,1068.62482338)(118.37357025,1068.62483988)
\curveto(118.81356351,1068.62482338)(119.2235631,1068.55482345)(119.60357025,1068.41483988)
\curveto(120.00356232,1068.27482373)(120.35356197,1068.09482391)(120.65357025,1067.87483988)
\curveto(120.95356137,1067.65482435)(121.20356112,1067.39482461)(121.40357025,1067.09483988)
\curveto(121.6235607,1066.79482521)(121.77356055,1066.48482552)(121.85357025,1066.16483988)
\lineto(121.91357025,1066.16483988)
\lineto(121.91357025,1068.20483988)
\lineto(124.31357025,1068.20483988)
\lineto(124.31357025,1053.14483988)
\moveto(118.70357025,1066.46483988)
\curveto(118.1235642,1066.46482554)(117.64356468,1066.31482569)(117.26357025,1066.01483988)
\curveto(116.88356544,1065.71482629)(116.58356574,1065.29482671)(116.36357025,1064.75483988)
\curveto(116.14356618,1064.21482779)(115.98356634,1063.55482845)(115.88357025,1062.77483988)
\curveto(115.80356652,1061.99483001)(115.76356656,1061.12483088)(115.76357025,1060.16483988)
\curveto(115.76356656,1059.5048325)(115.79356653,1058.82483318)(115.85357025,1058.12483988)
\curveto(115.93356639,1057.44483456)(116.07356625,1056.81483519)(116.27357025,1056.23483988)
\curveto(116.47356585,1055.67483633)(116.76356556,1055.2048368)(117.14357025,1054.82483988)
\curveto(117.5235648,1054.46483754)(118.03356429,1054.28483772)(118.67357025,1054.28483988)
\curveto(119.35356297,1054.28483772)(119.90356242,1054.43483757)(120.32357025,1054.73483988)
\curveto(120.76356156,1055.03483697)(121.10356122,1055.44483656)(121.34357025,1055.96483988)
\curveto(121.58356074,1056.5048355)(121.74356058,1057.13483487)(121.82357025,1057.85483988)
\curveto(121.90356042,1058.57483343)(121.94356038,1059.34483266)(121.94357025,1060.16483988)
\curveto(121.94356038,1060.94483106)(121.90356042,1061.7048303)(121.82357025,1062.44483988)
\curveto(121.74356058,1063.2048288)(121.58356074,1063.88482812)(121.34357025,1064.48483988)
\curveto(121.10356122,1065.08482692)(120.77356155,1065.56482644)(120.35357025,1065.92483988)
\curveto(119.93356239,1066.28482572)(119.38356294,1066.46482554)(118.70357025,1066.46483988)
}
}
{
\newrgbcolor{curcolor}{0 0 0}
\pscustom[linestyle=none,fillstyle=solid,fillcolor=curcolor]
{
\newpath
\moveto(129.710289,1059.86483988)
\curveto(129.71028525,1059.24483276)(129.72028524,1058.57483343)(129.740289,1057.85483988)
\curveto(129.78028518,1057.13483487)(129.90028506,1056.46483554)(130.100289,1055.84483988)
\curveto(130.30028466,1055.22483678)(130.60028436,1054.7048373)(131.000289,1054.28483988)
\curveto(131.42028354,1053.88483812)(132.02028294,1053.68483832)(132.800289,1053.68483988)
\curveto(133.40028156,1053.68483832)(133.88028108,1053.81483819)(134.240289,1054.07483988)
\curveto(134.60028036,1054.35483765)(134.87028009,1054.68483732)(135.050289,1055.06483988)
\curveto(135.25027971,1055.46483654)(135.38027958,1055.87483613)(135.440289,1056.29483988)
\curveto(135.50027946,1056.73483527)(135.53027943,1057.1048349)(135.530289,1057.40483988)
\lineto(138.050289,1057.40483988)
\curveto(138.05027691,1056.98483502)(137.97027699,1056.44483556)(137.810289,1055.78483988)
\curveto(137.67027729,1055.14483686)(137.40027756,1054.51483749)(137.000289,1053.89483988)
\curveto(136.60027836,1053.29483871)(136.05027891,1052.77483923)(135.350289,1052.33483988)
\curveto(134.65028031,1051.91484009)(133.75028121,1051.7048403)(132.650289,1051.70483988)
\curveto(130.67028429,1051.7048403)(129.24028572,1052.38483962)(128.360289,1053.74483988)
\curveto(127.50028746,1055.1048369)(127.07028789,1057.17483483)(127.070289,1059.95483988)
\curveto(127.07028789,1060.95483105)(127.13028783,1061.96483004)(127.250289,1062.98483988)
\curveto(127.39028757,1064.02482798)(127.6602873,1064.95482705)(128.060289,1065.77483988)
\curveto(128.48028648,1066.61482539)(129.0602859,1067.29482471)(129.800289,1067.81483988)
\curveto(130.5602844,1068.35482365)(131.5602834,1068.62482338)(132.800289,1068.62483988)
\curveto(134.02028094,1068.62482338)(134.99027997,1068.38482362)(135.710289,1067.90483988)
\curveto(136.43027853,1067.42482458)(136.97027799,1066.8048252)(137.330289,1066.04483988)
\curveto(137.69027727,1065.3048267)(137.92027704,1064.47482753)(138.020289,1063.55483988)
\curveto(138.12027684,1062.63482937)(138.17027679,1061.74483026)(138.170289,1060.88483988)
\lineto(138.170289,1059.86483988)
\lineto(129.710289,1059.86483988)
\moveto(135.530289,1061.84483988)
\lineto(135.530289,1062.71483988)
\curveto(135.53027943,1063.15482885)(135.49027947,1063.6048284)(135.410289,1064.06483988)
\curveto(135.33027963,1064.54482746)(135.18027978,1064.97482703)(134.960289,1065.35483988)
\curveto(134.7602802,1065.73482627)(134.48028048,1066.04482596)(134.120289,1066.28483988)
\curveto(133.7602812,1066.52482548)(133.30028166,1066.64482536)(132.740289,1066.64483988)
\curveto(132.08028288,1066.64482536)(131.55028341,1066.46482554)(131.150289,1066.10483988)
\curveto(130.77028419,1065.76482624)(130.48028448,1065.36482664)(130.280289,1064.90483988)
\curveto(130.08028488,1064.44482756)(129.95028501,1063.97482803)(129.890289,1063.49483988)
\curveto(129.83028513,1063.03482897)(129.80028516,1062.68482932)(129.800289,1062.44483988)
\lineto(129.800289,1061.84483988)
\lineto(135.530289,1061.84483988)
}
}
{
\newrgbcolor{curcolor}{0 0 0}
\pscustom[linestyle=none,fillstyle=solid,fillcolor=curcolor]
{
\newpath
\moveto(147.81107025,1063.49483988)
\curveto(147.8110617,1064.51482749)(147.64106187,1065.29482671)(147.30107025,1065.83483988)
\curveto(146.98106253,1066.37482563)(146.36106315,1066.64482536)(145.44107025,1066.64483988)
\curveto(145.24106427,1066.64482536)(144.99106452,1066.61482539)(144.69107025,1066.55483988)
\curveto(144.39106512,1066.51482549)(144.10106541,1066.41482559)(143.82107025,1066.25483988)
\curveto(143.56106595,1066.09482591)(143.33106618,1065.84482616)(143.13107025,1065.50483988)
\curveto(142.93106658,1065.18482682)(142.83106668,1064.74482726)(142.83107025,1064.18483988)
\curveto(142.83106668,1063.7048283)(142.94106657,1063.31482869)(143.16107025,1063.01483988)
\curveto(143.40106611,1062.71482929)(143.70106581,1062.45482955)(144.06107025,1062.23483988)
\curveto(144.44106507,1062.03482997)(144.86106465,1061.86483014)(145.32107025,1061.72483988)
\curveto(145.80106371,1061.58483042)(146.29106322,1061.43483057)(146.79107025,1061.27483988)
\curveto(147.27106224,1061.11483089)(147.74106177,1060.93483107)(148.20107025,1060.73483988)
\curveto(148.68106083,1060.55483145)(149.10106041,1060.29483171)(149.46107025,1059.95483988)
\curveto(149.84105967,1059.63483237)(150.14105937,1059.21483279)(150.36107025,1058.69483988)
\curveto(150.60105891,1058.19483381)(150.72105879,1057.54483446)(150.72107025,1056.74483988)
\curveto(150.72105879,1055.9048361)(150.59105892,1055.16483684)(150.33107025,1054.52483988)
\curveto(150.07105944,1053.9048381)(149.7110598,1053.38483862)(149.25107025,1052.96483988)
\curveto(148.79106072,1052.54483946)(148.24106127,1052.23483977)(147.60107025,1052.03483988)
\curveto(146.96106255,1051.81484019)(146.27106324,1051.7048403)(145.53107025,1051.70483988)
\curveto(144.17106534,1051.7048403)(143.1110664,1051.92484008)(142.35107025,1052.36483988)
\curveto(141.6110679,1052.8048392)(141.06106845,1053.32483868)(140.70107025,1053.92483988)
\curveto(140.36106915,1054.54483746)(140.16106935,1055.17483683)(140.10107025,1055.81483988)
\curveto(140.04106947,1056.45483555)(140.0110695,1056.98483502)(140.01107025,1057.40483988)
\lineto(142.53107025,1057.40483988)
\curveto(142.53106698,1056.92483508)(142.57106694,1056.45483555)(142.65107025,1055.99483988)
\curveto(142.73106678,1055.55483645)(142.88106663,1055.15483685)(143.10107025,1054.79483988)
\curveto(143.32106619,1054.45483755)(143.62106589,1054.18483782)(144.00107025,1053.98483988)
\curveto(144.40106511,1053.78483822)(144.9110646,1053.68483832)(145.53107025,1053.68483988)
\curveto(145.73106378,1053.68483832)(145.98106353,1053.71483829)(146.28107025,1053.77483988)
\curveto(146.58106293,1053.83483817)(146.87106264,1053.96483804)(147.15107025,1054.16483988)
\curveto(147.45106206,1054.36483764)(147.70106181,1054.63483737)(147.90107025,1054.97483988)
\curveto(148.10106141,1055.31483669)(148.20106131,1055.77483623)(148.20107025,1056.35483988)
\curveto(148.20106131,1056.89483511)(148.08106143,1057.33483467)(147.84107025,1057.67483988)
\curveto(147.62106189,1058.03483397)(147.32106219,1058.32483368)(146.94107025,1058.54483988)
\curveto(146.58106293,1058.78483322)(146.16106335,1058.98483302)(145.68107025,1059.14483988)
\curveto(145.22106429,1059.3048327)(144.75106476,1059.46483254)(144.27107025,1059.62483988)
\curveto(143.79106572,1059.78483222)(143.3110662,1059.95483205)(142.83107025,1060.13483988)
\curveto(142.35106716,1060.33483167)(141.92106759,1060.59483141)(141.54107025,1060.91483988)
\curveto(141.18106833,1061.23483077)(140.88106863,1061.65483035)(140.64107025,1062.17483988)
\curveto(140.42106909,1062.69482931)(140.3110692,1063.36482864)(140.31107025,1064.18483988)
\curveto(140.3110692,1064.92482708)(140.44106907,1065.57482643)(140.70107025,1066.13483988)
\curveto(140.98106853,1066.69482531)(141.35106816,1067.15482485)(141.81107025,1067.51483988)
\curveto(142.29106722,1067.89482411)(142.84106667,1068.17482383)(143.46107025,1068.35483988)
\curveto(144.08106543,1068.53482347)(144.74106477,1068.62482338)(145.44107025,1068.62483988)
\curveto(146.60106291,1068.62482338)(147.511062,1068.44482356)(148.17107025,1068.08483988)
\curveto(148.83106068,1067.72482428)(149.32106019,1067.28482472)(149.64107025,1066.76483988)
\curveto(149.96105955,1066.24482576)(150.15105936,1065.68482632)(150.21107025,1065.08483988)
\curveto(150.29105922,1064.5048275)(150.33105918,1063.97482803)(150.33107025,1063.49483988)
\lineto(147.81107025,1063.49483988)
}
}
{
\newrgbcolor{curcolor}{0 0 0}
\pscustom[linestyle=none,fillstyle=solid,fillcolor=curcolor]
{
\newpath
\moveto(153.84857025,1072.88483988)
\lineto(156.36857025,1072.88483988)
\lineto(156.36857025,1068.20483988)
\lineto(159.15857025,1068.20483988)
\lineto(159.15857025,1066.22483988)
\lineto(156.36857025,1066.22483988)
\lineto(156.36857025,1055.90483988)
\curveto(156.36856533,1055.26483674)(156.47856522,1054.8048372)(156.69857025,1054.52483988)
\curveto(156.91856478,1054.24483776)(157.35856434,1054.1048379)(158.01857025,1054.10483988)
\curveto(158.2985634,1054.1048379)(158.51856318,1054.11483789)(158.67857025,1054.13483988)
\curveto(158.83856286,1054.15483785)(158.98856271,1054.17483783)(159.12857025,1054.19483988)
\lineto(159.12857025,1052.12483988)
\curveto(158.96856273,1052.08483992)(158.71856298,1052.04483996)(158.37857025,1052.00483988)
\curveto(158.03856366,1051.96484004)(157.60856409,1051.94484006)(157.08857025,1051.94483988)
\curveto(156.42856527,1051.94484006)(155.88856581,1052.00484)(155.46857025,1052.12483988)
\curveto(155.04856665,1052.26483974)(154.71856698,1052.46483954)(154.47857025,1052.72483988)
\curveto(154.23856746,1053.004839)(154.06856763,1053.34483866)(153.96857025,1053.74483988)
\curveto(153.88856781,1054.14483786)(153.84856785,1054.6048374)(153.84857025,1055.12483988)
\lineto(153.84857025,1066.22483988)
\lineto(151.50857025,1066.22483988)
\lineto(151.50857025,1068.20483988)
\lineto(153.84857025,1068.20483988)
\lineto(153.84857025,1072.88483988)
}
}
{
\newrgbcolor{curcolor}{0 0 0}
\pscustom[linestyle=none,fillstyle=solid,fillcolor=curcolor]
{
\newpath
\moveto(161.041539,1073.54483988)
\lineto(163.561539,1073.54483988)
\lineto(163.561539,1070.66483988)
\lineto(161.041539,1070.66483988)
\lineto(161.041539,1073.54483988)
\moveto(161.041539,1068.20483988)
\lineto(163.561539,1068.20483988)
\lineto(163.561539,1052.12483988)
\lineto(161.041539,1052.12483988)
\lineto(161.041539,1068.20483988)
}
}
{
\newrgbcolor{curcolor}{0 0 0}
\pscustom[linestyle=none,fillstyle=solid,fillcolor=curcolor]
{
\newpath
\moveto(166.205289,1060.16483988)
\curveto(166.20528813,1061.3048307)(166.28528805,1062.38482962)(166.445289,1063.40483988)
\curveto(166.62528771,1064.42482758)(166.92528741,1065.31482669)(167.345289,1066.07483988)
\curveto(167.78528655,1066.85482515)(168.37528596,1067.47482453)(169.115289,1067.93483988)
\curveto(169.87528446,1068.39482361)(170.8352835,1068.62482338)(171.995289,1068.62483988)
\curveto(173.15528118,1068.62482338)(174.10528023,1068.39482361)(174.845289,1067.93483988)
\curveto(175.60527873,1067.47482453)(176.19527814,1066.85482515)(176.615289,1066.07483988)
\curveto(177.05527728,1065.31482669)(177.35527698,1064.42482758)(177.515289,1063.40483988)
\curveto(177.69527664,1062.38482962)(177.78527655,1061.3048307)(177.785289,1060.16483988)
\curveto(177.78527655,1059.02483298)(177.69527664,1057.94483406)(177.515289,1056.92483988)
\curveto(177.35527698,1055.9048361)(177.05527728,1055.004837)(176.615289,1054.22483988)
\curveto(176.17527816,1053.46483854)(175.58527875,1052.85483915)(174.845289,1052.39483988)
\curveto(174.10528023,1051.93484007)(173.15528118,1051.7048403)(171.995289,1051.70483988)
\curveto(170.8352835,1051.7048403)(169.87528446,1051.93484007)(169.115289,1052.39483988)
\curveto(168.37528596,1052.85483915)(167.78528655,1053.46483854)(167.345289,1054.22483988)
\curveto(166.92528741,1055.004837)(166.62528771,1055.9048361)(166.445289,1056.92483988)
\curveto(166.28528805,1057.94483406)(166.20528813,1059.02483298)(166.205289,1060.16483988)
\moveto(171.935289,1053.68483988)
\curveto(172.59528174,1053.68483832)(173.1352812,1053.85483815)(173.555289,1054.19483988)
\curveto(173.97528036,1054.55483745)(174.30528003,1055.02483698)(174.545289,1055.60483988)
\curveto(174.78527955,1056.2048358)(174.94527939,1056.89483511)(175.025289,1057.67483988)
\curveto(175.10527923,1058.45483355)(175.14527919,1059.28483272)(175.145289,1060.16483988)
\curveto(175.14527919,1061.02483098)(175.10527923,1061.84483016)(175.025289,1062.62483988)
\curveto(174.94527939,1063.42482858)(174.78527955,1064.11482789)(174.545289,1064.69483988)
\curveto(174.32528001,1065.29482671)(174.00528033,1065.76482624)(173.585289,1066.10483988)
\curveto(173.16528117,1066.46482554)(172.61528172,1066.64482536)(171.935289,1066.64483988)
\curveto(171.29528304,1066.64482536)(170.77528356,1066.46482554)(170.375289,1066.10483988)
\curveto(169.97528436,1065.76482624)(169.65528468,1065.29482671)(169.415289,1064.69483988)
\curveto(169.19528514,1064.11482789)(169.04528529,1063.42482858)(168.965289,1062.62483988)
\curveto(168.88528545,1061.84483016)(168.84528549,1061.02483098)(168.845289,1060.16483988)
\curveto(168.84528549,1059.28483272)(168.88528545,1058.45483355)(168.965289,1057.67483988)
\curveto(169.04528529,1056.89483511)(169.19528514,1056.2048358)(169.415289,1055.60483988)
\curveto(169.6352847,1055.02483698)(169.94528439,1054.55483745)(170.345289,1054.19483988)
\curveto(170.76528357,1053.85483815)(171.29528304,1053.68483832)(171.935289,1053.68483988)
}
}
{
\newrgbcolor{curcolor}{0 0 0}
\pscustom[linestyle=none,fillstyle=solid,fillcolor=curcolor]
{
\newpath
\moveto(180.28607025,1068.20483988)
\lineto(182.68607025,1068.20483988)
\lineto(182.68607025,1066.31483988)
\lineto(182.74607025,1066.31483988)
\curveto(183.0860658,1067.05482495)(183.62606526,1067.62482438)(184.36607025,1068.02483988)
\curveto(185.10606378,1068.42482358)(185.86606302,1068.62482338)(186.64607025,1068.62483988)
\curveto(187.70606118,1068.62482338)(188.51606037,1068.44482356)(189.07607025,1068.08483988)
\curveto(189.65605923,1067.74482426)(190.07605881,1067.31482469)(190.33607025,1066.79483988)
\curveto(190.61605827,1066.27482573)(190.77605811,1065.71482629)(190.81607025,1065.11483988)
\curveto(190.85605803,1064.53482747)(190.87605801,1064.004828)(190.87607025,1063.52483988)
\lineto(190.87607025,1052.12483988)
\lineto(188.35607025,1052.12483988)
\lineto(188.35607025,1063.22483988)
\curveto(188.35606053,1063.52482848)(188.33606055,1063.86482814)(188.29607025,1064.24483988)
\curveto(188.27606061,1064.62482738)(188.1860607,1064.97482703)(188.02607025,1065.29483988)
\curveto(187.86606102,1065.63482637)(187.62606126,1065.91482609)(187.30607025,1066.13483988)
\curveto(186.9860619,1066.35482565)(186.54606234,1066.46482554)(185.98607025,1066.46483988)
\curveto(185.64606324,1066.46482554)(185.2860636,1066.4048256)(184.90607025,1066.28483988)
\curveto(184.54606434,1066.16482584)(184.20606468,1065.97482603)(183.88607025,1065.71483988)
\curveto(183.56606532,1065.45482655)(183.30606558,1065.11482689)(183.10607025,1064.69483988)
\curveto(182.90606598,1064.29482771)(182.80606608,1063.8048282)(182.80607025,1063.22483988)
\lineto(182.80607025,1052.12483988)
\lineto(180.28607025,1052.12483988)
\lineto(180.28607025,1068.20483988)
}
}
{
\newrgbcolor{curcolor}{0 0 0}
\pscustom[linewidth=2,linecolor=curcolor]
{
\newpath
\moveto(467.96582031,741.14299898)
\lineto(616.11645508,741.14299898)
\lineto(616.11645508,692.99237566)
\lineto(467.96582031,692.99237566)
\closepath
}
}
{
\newrgbcolor{curcolor}{0 0 0}
\pscustom[linestyle=none,fillstyle=solid,fillcolor=curcolor]
{
\newpath
\moveto(522.345896,712.59769137)
\curveto(522.34588376,711.39769155)(522.2058839,710.38769256)(521.925896,709.56769137)
\curveto(521.66588444,708.72769422)(521.29588481,708.0476949)(520.815896,707.52769137)
\curveto(520.33588577,707.00769594)(519.75588635,706.63769631)(519.075896,706.41769137)
\curveto(518.41588769,706.17769677)(517.68588842,706.05769689)(516.885896,706.05769137)
\curveto(516.64588946,706.05769689)(516.24588986,706.07769687)(515.685896,706.11769137)
\curveto(515.14589096,706.15769679)(514.58589152,706.30769664)(514.005896,706.56769137)
\curveto(513.44589266,706.80769614)(512.92589318,707.20769574)(512.445896,707.76769137)
\curveto(511.98589412,708.30769464)(511.7058944,709.07769387)(511.605896,710.07769137)
\lineto(514.125896,710.07769137)
\curveto(514.18589192,709.33769361)(514.46589164,708.81769413)(514.965896,708.51769137)
\curveto(515.46589064,708.19769475)(516.04589006,708.03769491)(516.705896,708.03769137)
\curveto(517.5058886,708.03769491)(518.11588799,708.17769477)(518.535896,708.45769137)
\curveto(518.97588713,708.71769423)(519.28588682,709.03769391)(519.465896,709.41769137)
\curveto(519.66588644,709.79769315)(519.77588633,710.19769275)(519.795896,710.61769137)
\curveto(519.81588629,711.01769193)(519.82588628,711.35769159)(519.825896,711.63769137)
\lineto(519.825896,713.73769137)
\lineto(519.765896,713.73769137)
\curveto(519.48588662,713.05768989)(519.0058871,712.52769042)(518.325896,712.14769137)
\curveto(517.66588844,711.76769118)(516.93588917,711.57769137)(516.135896,711.57769137)
\curveto(515.35589075,711.57769137)(514.68589142,711.72769122)(514.125896,712.02769137)
\curveto(513.58589252,712.32769062)(513.12589298,712.71769023)(512.745896,713.19769137)
\curveto(512.38589372,713.67768927)(512.09589401,714.20768874)(511.875896,714.78769137)
\curveto(511.67589443,715.38768756)(511.51589459,715.98768696)(511.395896,716.58769137)
\curveto(511.29589481,717.18768576)(511.22589488,717.7476852)(511.185896,718.26769137)
\curveto(511.16589494,718.80768414)(511.15589495,719.25768369)(511.155896,719.61769137)
\curveto(511.15589495,720.69768225)(511.23589487,721.7476812)(511.395896,722.76769137)
\curveto(511.55589455,723.78767916)(511.83589427,724.68767826)(512.235896,725.46769137)
\curveto(512.65589345,726.26767668)(513.19589291,726.89767605)(513.855896,727.35769137)
\curveto(514.53589157,727.83767511)(515.38589072,728.07767487)(516.405896,728.07769137)
\curveto(516.84588926,728.07767487)(517.25588885,728.00767494)(517.635896,727.86769137)
\curveto(518.03588807,727.72767522)(518.38588772,727.5476754)(518.685896,727.32769137)
\curveto(518.98588712,727.10767584)(519.23588687,726.8476761)(519.435896,726.54769137)
\curveto(519.65588645,726.2476767)(519.8058863,725.93767701)(519.885896,725.61769137)
\lineto(519.945896,725.61769137)
\lineto(519.945896,727.65769137)
\lineto(522.345896,727.65769137)
\lineto(522.345896,712.59769137)
\moveto(516.735896,725.91769137)
\curveto(516.15588995,725.91767703)(515.67589043,725.76767718)(515.295896,725.46769137)
\curveto(514.91589119,725.16767778)(514.61589149,724.7476782)(514.395896,724.20769137)
\curveto(514.17589193,723.66767928)(514.01589209,723.00767994)(513.915896,722.22769137)
\curveto(513.83589227,721.4476815)(513.79589231,720.57768237)(513.795896,719.61769137)
\curveto(513.79589231,718.95768399)(513.82589228,718.27768467)(513.885896,717.57769137)
\curveto(513.96589214,716.89768605)(514.105892,716.26768668)(514.305896,715.68769137)
\curveto(514.5058916,715.12768782)(514.79589131,714.65768829)(515.175896,714.27769137)
\curveto(515.55589055,713.91768903)(516.06589004,713.73768921)(516.705896,713.73769137)
\curveto(517.38588872,713.73768921)(517.93588817,713.88768906)(518.355896,714.18769137)
\curveto(518.79588731,714.48768846)(519.13588697,714.89768805)(519.375896,715.41769137)
\curveto(519.61588649,715.95768699)(519.77588633,716.58768636)(519.855896,717.30769137)
\curveto(519.93588617,718.02768492)(519.97588613,718.79768415)(519.975896,719.61769137)
\curveto(519.97588613,720.39768255)(519.93588617,721.15768179)(519.855896,721.89769137)
\curveto(519.77588633,722.65768029)(519.61588649,723.33767961)(519.375896,723.93769137)
\curveto(519.13588697,724.53767841)(518.8058873,725.01767793)(518.385896,725.37769137)
\curveto(517.96588814,725.73767721)(517.41588869,725.91767703)(516.735896,725.91769137)
}
}
{
\newrgbcolor{curcolor}{0 0 0}
\pscustom[linestyle=none,fillstyle=solid,fillcolor=curcolor]
{
\newpath
\moveto(525.64261475,727.65769137)
\lineto(528.16261475,727.65769137)
\lineto(528.16261475,725.25769137)
\lineto(528.22261475,725.25769137)
\curveto(528.40261034,725.63767731)(528.59261015,725.99767695)(528.79261475,726.33769137)
\curveto(529.01260973,726.67767627)(529.26260948,726.97767597)(529.54261475,727.23769137)
\curveto(529.82260892,727.49767545)(530.13260861,727.69767525)(530.47261475,727.83769137)
\curveto(530.83260791,727.99767495)(531.2426075,728.07767487)(531.70261475,728.07769137)
\curveto(532.20260654,728.07767487)(532.57260617,728.01767493)(532.81261475,727.89769137)
\lineto(532.81261475,725.43769137)
\curveto(532.69260605,725.45767749)(532.53260621,725.47767747)(532.33261475,725.49769137)
\curveto(532.15260659,725.53767741)(531.86260688,725.55767739)(531.46261475,725.55769137)
\curveto(531.1426076,725.55767739)(530.79260795,725.47767747)(530.41261475,725.31769137)
\curveto(530.03260871,725.17767777)(529.67260907,724.947678)(529.33261475,724.62769137)
\curveto(529.01260973,724.32767862)(528.73261001,723.93767901)(528.49261475,723.45769137)
\curveto(528.27261047,722.97767997)(528.16261058,722.40768054)(528.16261475,721.74769137)
\lineto(528.16261475,711.57769137)
\lineto(525.64261475,711.57769137)
\lineto(525.64261475,727.65769137)
}
}
{
\newrgbcolor{curcolor}{0 0 0}
\pscustom[linestyle=none,fillstyle=solid,fillcolor=curcolor]
{
\newpath
\moveto(533.76886475,719.61769137)
\curveto(533.76886388,720.75768219)(533.8488638,721.83768111)(534.00886475,722.85769137)
\curveto(534.18886346,723.87767907)(534.48886316,724.76767818)(534.90886475,725.52769137)
\curveto(535.3488623,726.30767664)(535.93886171,726.92767602)(536.67886475,727.38769137)
\curveto(537.43886021,727.8476751)(538.39885925,728.07767487)(539.55886475,728.07769137)
\curveto(540.71885693,728.07767487)(541.66885598,727.8476751)(542.40886475,727.38769137)
\curveto(543.16885448,726.92767602)(543.75885389,726.30767664)(544.17886475,725.52769137)
\curveto(544.61885303,724.76767818)(544.91885273,723.87767907)(545.07886475,722.85769137)
\curveto(545.25885239,721.83768111)(545.3488523,720.75768219)(545.34886475,719.61769137)
\curveto(545.3488523,718.47768447)(545.25885239,717.39768555)(545.07886475,716.37769137)
\curveto(544.91885273,715.35768759)(544.61885303,714.45768849)(544.17886475,713.67769137)
\curveto(543.73885391,712.91769003)(543.1488545,712.30769064)(542.40886475,711.84769137)
\curveto(541.66885598,711.38769156)(540.71885693,711.15769179)(539.55886475,711.15769137)
\curveto(538.39885925,711.15769179)(537.43886021,711.38769156)(536.67886475,711.84769137)
\curveto(535.93886171,712.30769064)(535.3488623,712.91769003)(534.90886475,713.67769137)
\curveto(534.48886316,714.45768849)(534.18886346,715.35768759)(534.00886475,716.37769137)
\curveto(533.8488638,717.39768555)(533.76886388,718.47768447)(533.76886475,719.61769137)
\moveto(539.49886475,713.13769137)
\curveto(540.15885749,713.13768981)(540.69885695,713.30768964)(541.11886475,713.64769137)
\curveto(541.53885611,714.00768894)(541.86885578,714.47768847)(542.10886475,715.05769137)
\curveto(542.3488553,715.65768729)(542.50885514,716.3476866)(542.58886475,717.12769137)
\curveto(542.66885498,717.90768504)(542.70885494,718.73768421)(542.70886475,719.61769137)
\curveto(542.70885494,720.47768247)(542.66885498,721.29768165)(542.58886475,722.07769137)
\curveto(542.50885514,722.87768007)(542.3488553,723.56767938)(542.10886475,724.14769137)
\curveto(541.88885576,724.7476782)(541.56885608,725.21767773)(541.14886475,725.55769137)
\curveto(540.72885692,725.91767703)(540.17885747,726.09767685)(539.49886475,726.09769137)
\curveto(538.85885879,726.09767685)(538.33885931,725.91767703)(537.93886475,725.55769137)
\curveto(537.53886011,725.21767773)(537.21886043,724.7476782)(536.97886475,724.14769137)
\curveto(536.75886089,723.56767938)(536.60886104,722.87768007)(536.52886475,722.07769137)
\curveto(536.4488612,721.29768165)(536.40886124,720.47768247)(536.40886475,719.61769137)
\curveto(536.40886124,718.73768421)(536.4488612,717.90768504)(536.52886475,717.12769137)
\curveto(536.60886104,716.3476866)(536.75886089,715.65768729)(536.97886475,715.05769137)
\curveto(537.19886045,714.47768847)(537.50886014,714.00768894)(537.90886475,713.64769137)
\curveto(538.32885932,713.30768964)(538.85885879,713.13768981)(539.49886475,713.13769137)
}
}
{
\newrgbcolor{curcolor}{0 0 0}
\pscustom[linestyle=none,fillstyle=solid,fillcolor=curcolor]
{
\newpath
\moveto(558.439646,711.57769137)
\lineto(556.039646,711.57769137)
\lineto(556.039646,713.46769137)
\lineto(555.979646,713.46769137)
\curveto(555.63963656,712.72769022)(555.0996371,712.15769079)(554.359646,711.75769137)
\curveto(553.61963858,711.35769159)(552.85963934,711.15769179)(552.079646,711.15769137)
\curveto(551.01964118,711.15769179)(550.199642,711.32769162)(549.619646,711.66769137)
\curveto(549.05964314,712.02769092)(548.63964356,712.46769048)(548.359646,712.98769137)
\curveto(548.0996441,713.50768944)(547.94964425,714.05768889)(547.909646,714.63769137)
\curveto(547.86964433,715.23768771)(547.84964435,715.77768717)(547.849646,716.25769137)
\lineto(547.849646,727.65769137)
\lineto(550.369646,727.65769137)
\lineto(550.369646,716.55769137)
\curveto(550.36964183,716.25768669)(550.37964182,715.91768703)(550.399646,715.53769137)
\curveto(550.43964176,715.15768779)(550.53964166,714.79768815)(550.699646,714.45769137)
\curveto(550.85964134,714.13768881)(551.08964111,713.86768908)(551.389646,713.64769137)
\curveto(551.70964049,713.42768952)(552.15964004,713.31768963)(552.739646,713.31769137)
\curveto(553.07963912,713.31768963)(553.42963877,713.37768957)(553.789646,713.49769137)
\curveto(554.16963803,713.61768933)(554.51963768,713.80768914)(554.839646,714.06769137)
\curveto(555.15963704,714.32768862)(555.41963678,714.65768829)(555.619646,715.05769137)
\curveto(555.81963638,715.47768747)(555.91963628,715.97768697)(555.919646,716.55769137)
\lineto(555.919646,727.65769137)
\lineto(558.439646,727.65769137)
\lineto(558.439646,711.57769137)
}
}
{
\newrgbcolor{curcolor}{0 0 0}
\pscustom[linestyle=none,fillstyle=solid,fillcolor=curcolor]
{
\newpath
\moveto(561.73636475,727.65769137)
\lineto(564.13636475,727.65769137)
\lineto(564.13636475,725.67769137)
\lineto(564.19636475,725.67769137)
\curveto(564.27636056,725.97767697)(564.41636042,726.26767668)(564.61636475,726.54769137)
\curveto(564.83636,726.8476761)(565.09635974,727.10767584)(565.39636475,727.32769137)
\curveto(565.71635912,727.5476754)(566.06635877,727.72767522)(566.44636475,727.86769137)
\curveto(566.82635801,728.00767494)(567.2363576,728.07767487)(567.67636475,728.07769137)
\curveto(568.6363562,728.07767487)(569.44635539,727.87767507)(570.10636475,727.47769137)
\curveto(570.76635407,727.07767587)(571.30635353,726.51767643)(571.72636475,725.79769137)
\curveto(572.14635269,725.07767787)(572.44635239,724.21767873)(572.62636475,723.21769137)
\curveto(572.82635201,722.21768073)(572.92635191,721.11768183)(572.92636475,719.91769137)
\curveto(572.92635191,718.99768395)(572.84635199,718.02768492)(572.68636475,717.00769137)
\curveto(572.52635231,715.98768696)(572.24635259,715.03768791)(571.84636475,714.15769137)
\curveto(571.46635337,713.29768965)(570.9363539,712.57769037)(570.25636475,711.99769137)
\curveto(569.57635526,711.43769151)(568.71635612,711.15769179)(567.67636475,711.15769137)
\curveto(566.95635788,711.15769179)(566.28635855,711.35769159)(565.66636475,711.75769137)
\curveto(565.04635979,712.15769079)(564.59636024,712.70769024)(564.31636475,713.40769137)
\lineto(564.25636475,713.40769137)
\lineto(564.25636475,706.29769137)
\lineto(561.73636475,706.29769137)
\lineto(561.73636475,727.65769137)
\moveto(564.10636475,719.61769137)
\curveto(564.10636073,718.83768411)(564.14636069,718.06768488)(564.22636475,717.30769137)
\curveto(564.30636053,716.56768638)(564.46636037,715.89768705)(564.70636475,715.29769137)
\curveto(564.94635989,714.69768825)(565.27635956,714.21768873)(565.69636475,713.85769137)
\curveto(566.11635872,713.49768945)(566.66635817,713.31768963)(567.34636475,713.31769137)
\curveto(567.92635691,713.31768963)(568.40635643,713.46768948)(568.78636475,713.76769137)
\curveto(569.16635567,714.06768888)(569.46635537,714.49768845)(569.68636475,715.05769137)
\curveto(569.90635493,715.61768733)(570.05635478,716.30768664)(570.13636475,717.12769137)
\curveto(570.2363546,717.947685)(570.28635455,718.87768407)(570.28636475,719.91769137)
\curveto(570.28635455,720.79768215)(570.2363546,721.60768134)(570.13636475,722.34769137)
\curveto(570.05635478,723.08767986)(569.90635493,723.71767923)(569.68636475,724.23769137)
\curveto(569.46635537,724.77767817)(569.16635567,725.18767776)(568.78636475,725.46769137)
\curveto(568.40635643,725.76767718)(567.92635691,725.91767703)(567.34636475,725.91769137)
\curveto(566.64635819,725.91767703)(566.08635875,725.75767719)(565.66636475,725.43769137)
\curveto(565.24635959,725.13767781)(564.91635992,724.70767824)(564.67636475,724.14769137)
\curveto(564.45636038,723.58767936)(564.30636053,722.91768003)(564.22636475,722.13769137)
\curveto(564.14636069,721.37768157)(564.10636073,720.53768241)(564.10636475,719.61769137)
}
}
{
\newrgbcolor{curcolor}{0 0 0}
\pscustom[linewidth=2,linecolor=curcolor]
{
\newpath
\moveto(630.98297119,652.32717257)
\lineto(779.13360596,652.32717257)
\lineto(779.13360596,604.17654925)
\lineto(630.98297119,604.17654925)
\closepath
}
}
{
\newrgbcolor{curcolor}{0 0 0}
\pscustom[linestyle=none,fillstyle=solid,fillcolor=curcolor]
{
\newpath
\moveto(680.34604492,638.84187228)
\lineto(682.86604492,638.84187228)
\lineto(682.86604492,634.16187228)
\lineto(685.65604492,634.16187228)
\lineto(685.65604492,632.18187228)
\lineto(682.86604492,632.18187228)
\lineto(682.86604492,621.86187228)
\curveto(682.86604,621.22186914)(682.97603989,620.7618696)(683.19604492,620.48187228)
\curveto(683.41603945,620.20187016)(683.85603901,620.0618703)(684.51604492,620.06187228)
\curveto(684.79603807,620.0618703)(685.01603785,620.07187029)(685.17604492,620.09187228)
\curveto(685.33603753,620.11187025)(685.48603738,620.13187023)(685.62604492,620.15187228)
\lineto(685.62604492,618.08187228)
\curveto(685.4660374,618.04187232)(685.21603765,618.00187236)(684.87604492,617.96187228)
\curveto(684.53603833,617.92187244)(684.10603876,617.90187246)(683.58604492,617.90187228)
\curveto(682.92603994,617.90187246)(682.38604048,617.9618724)(681.96604492,618.08187228)
\curveto(681.54604132,618.22187214)(681.21604165,618.42187194)(680.97604492,618.68187228)
\curveto(680.73604213,618.9618714)(680.5660423,619.30187106)(680.46604492,619.70187228)
\curveto(680.38604248,620.10187026)(680.34604252,620.5618698)(680.34604492,621.08187228)
\lineto(680.34604492,632.18187228)
\lineto(678.00604492,632.18187228)
\lineto(678.00604492,634.16187228)
\lineto(680.34604492,634.16187228)
\lineto(680.34604492,638.84187228)
}
}
{
\newrgbcolor{curcolor}{0 0 0}
\pscustom[linestyle=none,fillstyle=solid,fillcolor=curcolor]
{
\newpath
\moveto(689.48901367,625.82187228)
\curveto(689.48900992,625.20186516)(689.49900991,624.53186583)(689.51901367,623.81187228)
\curveto(689.55900985,623.09186727)(689.67900973,622.42186794)(689.87901367,621.80187228)
\curveto(690.07900933,621.18186918)(690.37900903,620.6618697)(690.77901367,620.24187228)
\curveto(691.19900821,619.84187052)(691.79900761,619.64187072)(692.57901367,619.64187228)
\curveto(693.17900623,619.64187072)(693.65900575,619.77187059)(694.01901367,620.03187228)
\curveto(694.37900503,620.31187005)(694.64900476,620.64186972)(694.82901367,621.02187228)
\curveto(695.02900438,621.42186894)(695.15900425,621.83186853)(695.21901367,622.25187228)
\curveto(695.27900413,622.69186767)(695.3090041,623.0618673)(695.30901367,623.36187228)
\lineto(697.82901367,623.36187228)
\curveto(697.82900158,622.94186742)(697.74900166,622.40186796)(697.58901367,621.74187228)
\curveto(697.44900196,621.10186926)(697.17900223,620.47186989)(696.77901367,619.85187228)
\curveto(696.37900303,619.25187111)(695.82900358,618.73187163)(695.12901367,618.29187228)
\curveto(694.42900498,617.87187249)(693.52900588,617.6618727)(692.42901367,617.66187228)
\curveto(690.44900896,617.6618727)(689.01901039,618.34187202)(688.13901367,619.70187228)
\curveto(687.27901213,621.0618693)(686.84901256,623.13186723)(686.84901367,625.91187228)
\curveto(686.84901256,626.91186345)(686.9090125,627.92186244)(687.02901367,628.94187228)
\curveto(687.16901224,629.98186038)(687.43901197,630.91185945)(687.83901367,631.73187228)
\curveto(688.25901115,632.57185779)(688.83901057,633.25185711)(689.57901367,633.77187228)
\curveto(690.33900907,634.31185605)(691.33900807,634.58185578)(692.57901367,634.58187228)
\curveto(693.79900561,634.58185578)(694.76900464,634.34185602)(695.48901367,633.86187228)
\curveto(696.2090032,633.38185698)(696.74900266,632.7618576)(697.10901367,632.00187228)
\curveto(697.46900194,631.2618591)(697.69900171,630.43185993)(697.79901367,629.51187228)
\curveto(697.89900151,628.59186177)(697.94900146,627.70186266)(697.94901367,626.84187228)
\lineto(697.94901367,625.82187228)
\lineto(689.48901367,625.82187228)
\moveto(695.30901367,627.80187228)
\lineto(695.30901367,628.67187228)
\curveto(695.3090041,629.11186125)(695.26900414,629.5618608)(695.18901367,630.02187228)
\curveto(695.1090043,630.50185986)(694.95900445,630.93185943)(694.73901367,631.31187228)
\curveto(694.53900487,631.69185867)(694.25900515,632.00185836)(693.89901367,632.24187228)
\curveto(693.53900587,632.48185788)(693.07900633,632.60185776)(692.51901367,632.60187228)
\curveto(691.85900755,632.60185776)(691.32900808,632.42185794)(690.92901367,632.06187228)
\curveto(690.54900886,631.72185864)(690.25900915,631.32185904)(690.05901367,630.86187228)
\curveto(689.85900955,630.40185996)(689.72900968,629.93186043)(689.66901367,629.45187228)
\curveto(689.6090098,628.99186137)(689.57900983,628.64186172)(689.57901367,628.40187228)
\lineto(689.57901367,627.80187228)
\lineto(695.30901367,627.80187228)
}
}
{
\newrgbcolor{curcolor}{0 0 0}
\pscustom[linestyle=none,fillstyle=solid,fillcolor=curcolor]
{
\newpath
\moveto(708.06979492,626.84187228)
\curveto(707.82978613,626.60186376)(707.51978644,626.40186396)(707.13979492,626.24187228)
\curveto(706.77978718,626.08186428)(706.38978757,625.93186443)(705.96979492,625.79187228)
\curveto(705.56978839,625.67186469)(705.16978879,625.54186482)(704.76979492,625.40187228)
\curveto(704.38978957,625.28186508)(704.0597899,625.13186523)(703.77979492,624.95187228)
\curveto(703.37979058,624.69186567)(703.06979089,624.37186599)(702.84979492,623.99187228)
\curveto(702.64979131,623.63186673)(702.54979141,623.10186726)(702.54979492,622.40187228)
\curveto(702.54979141,621.5618688)(702.70979125,620.89186947)(703.02979492,620.39187228)
\curveto(703.36979059,619.89187047)(703.96978999,619.64187072)(704.82979492,619.64187228)
\curveto(705.24978871,619.64187072)(705.64978831,619.72187064)(706.02979492,619.88187228)
\curveto(706.42978753,620.04187032)(706.77978718,620.2618701)(707.07979492,620.54187228)
\curveto(707.37978658,620.82186954)(707.61978634,621.14186922)(707.79979492,621.50187228)
\curveto(707.97978598,621.88186848)(708.06978589,622.28186808)(708.06979492,622.70187228)
\lineto(708.06979492,626.84187228)
\moveto(700.29979492,629.30187228)
\curveto(700.29979366,631.12185924)(700.71979324,632.45185791)(701.55979492,633.29187228)
\curveto(702.39979156,634.15185621)(703.77979018,634.58185578)(705.69979492,634.58187228)
\curveto(706.91978704,634.58185578)(707.8597861,634.42185594)(708.51979492,634.10187228)
\curveto(709.17978478,633.78185658)(709.6597843,633.38185698)(709.95979492,632.90187228)
\curveto(710.27978368,632.42185794)(710.4597835,631.91185845)(710.49979492,631.37187228)
\curveto(710.5597834,630.85185951)(710.58978337,630.38185998)(710.58979492,629.96187228)
\lineto(710.58979492,620.99187228)
\curveto(710.58978337,620.65186971)(710.61978334,620.35187001)(710.67979492,620.09187228)
\curveto(710.73978322,619.83187053)(710.96978299,619.70187066)(711.36979492,619.70187228)
\curveto(711.52978243,619.70187066)(711.64978231,619.71187065)(711.72979492,619.73187228)
\curveto(711.82978213,619.77187059)(711.90978205,619.81187055)(711.96979492,619.85187228)
\lineto(711.96979492,618.05187228)
\curveto(711.86978209,618.03187233)(711.67978228,618.00187236)(711.39979492,617.96187228)
\curveto(711.11978284,617.92187244)(710.81978314,617.90187246)(710.49979492,617.90187228)
\curveto(710.2597837,617.90187246)(710.00978395,617.91187245)(709.74979492,617.93187228)
\curveto(709.50978445,617.95187241)(709.26978469,618.02187234)(709.02979492,618.14187228)
\curveto(708.80978515,618.28187208)(708.61978534,618.49187187)(708.45979492,618.77187228)
\curveto(708.31978564,619.05187131)(708.23978572,619.45187091)(708.21979492,619.97187228)
\lineto(708.15979492,619.97187228)
\curveto(707.7597862,619.25187111)(707.19978676,618.68187168)(706.47979492,618.26187228)
\curveto(705.77978818,617.8618725)(705.04978891,617.6618727)(704.28979492,617.66187228)
\curveto(702.78979117,617.6618727)(701.67979228,618.08187228)(700.95979492,618.92187228)
\curveto(700.2597937,619.7618706)(699.90979405,620.90186946)(699.90979492,622.34187228)
\curveto(699.90979405,623.48186688)(700.14979381,624.42186594)(700.62979492,625.16187228)
\curveto(701.12979283,625.92186444)(701.89979206,626.4618639)(702.93979492,626.78187228)
\lineto(706.32979492,627.80187228)
\curveto(706.78978717,627.94186242)(707.13978682,628.08186228)(707.37979492,628.22187228)
\curveto(707.63978632,628.38186198)(707.81978614,628.55186181)(707.91979492,628.73187228)
\curveto(708.03978592,628.91186145)(708.10978585,629.12186124)(708.12979492,629.36187228)
\curveto(708.14978581,629.60186076)(708.1597858,629.89186047)(708.15979492,630.23187228)
\curveto(708.1597858,631.81185855)(707.29978666,632.60185776)(705.57979492,632.60187228)
\curveto(704.87978908,632.60185776)(704.33978962,632.4618579)(703.95979492,632.18187228)
\curveto(703.59979036,631.92185844)(703.32979063,631.61185875)(703.14979492,631.25187228)
\curveto(702.98979097,630.89185947)(702.88979107,630.53185983)(702.84979492,630.17187228)
\curveto(702.82979113,629.83186053)(702.81979114,629.59186077)(702.81979492,629.45187228)
\lineto(702.81979492,629.30187228)
\lineto(700.29979492,629.30187228)
}
}
{
\newrgbcolor{curcolor}{0 0 0}
\pscustom[linestyle=none,fillstyle=solid,fillcolor=curcolor]
{
\newpath
\moveto(714.23057617,634.16187228)
\lineto(716.63057617,634.16187228)
\lineto(716.63057617,632.27187228)
\lineto(716.69057617,632.27187228)
\curveto(717.03057148,633.01185735)(717.57057094,633.58185678)(718.31057617,633.98187228)
\curveto(719.05056946,634.38185598)(719.8105687,634.58185578)(720.59057617,634.58187228)
\curveto(721.55056696,634.58185578)(722.30056621,634.38185598)(722.84057617,633.98187228)
\curveto(723.40056511,633.60185676)(723.8105647,632.95185741)(724.07057617,632.03187228)
\curveto(724.43056408,632.73185763)(724.94056357,633.33185703)(725.60057617,633.83187228)
\curveto(726.28056223,634.33185603)(727.04056147,634.58185578)(727.88057617,634.58187228)
\curveto(728.94055957,634.58185578)(729.75055876,634.40185596)(730.31057617,634.04187228)
\curveto(730.89055762,633.70185666)(731.3105572,633.27185709)(731.57057617,632.75187228)
\curveto(731.85055666,632.23185813)(732.0105565,631.67185869)(732.05057617,631.07187228)
\curveto(732.09055642,630.49185987)(732.1105564,629.9618604)(732.11057617,629.48187228)
\lineto(732.11057617,618.08187228)
\lineto(729.59057617,618.08187228)
\lineto(729.59057617,629.18187228)
\curveto(729.59055892,629.48186088)(729.57055894,629.82186054)(729.53057617,630.20187228)
\curveto(729.510559,630.58185978)(729.43055908,630.93185943)(729.29057617,631.25187228)
\curveto(729.15055936,631.59185877)(728.93055958,631.87185849)(728.63057617,632.09187228)
\curveto(728.35056016,632.31185805)(727.95056056,632.42185794)(727.43057617,632.42187228)
\curveto(727.13056138,632.42185794)(726.80056171,632.37185799)(726.44057617,632.27187228)
\curveto(726.10056241,632.17185819)(725.78056273,631.99185837)(725.48057617,631.73187228)
\curveto(725.18056333,631.49185887)(724.93056358,631.1618592)(724.73057617,630.74187228)
\curveto(724.53056398,630.32186004)(724.43056408,629.80186056)(724.43057617,629.18187228)
\lineto(724.43057617,618.08187228)
\lineto(721.91057617,618.08187228)
\lineto(721.91057617,629.18187228)
\curveto(721.9105666,629.48186088)(721.89056662,629.82186054)(721.85057617,630.20187228)
\curveto(721.83056668,630.58185978)(721.75056676,630.93185943)(721.61057617,631.25187228)
\curveto(721.47056704,631.59185877)(721.25056726,631.87185849)(720.95057617,632.09187228)
\curveto(720.67056784,632.31185805)(720.27056824,632.42185794)(719.75057617,632.42187228)
\curveto(719.45056906,632.42185794)(719.12056939,632.37185799)(718.76057617,632.27187228)
\curveto(718.42057009,632.17185819)(718.10057041,631.99185837)(717.80057617,631.73187228)
\curveto(717.50057101,631.49185887)(717.25057126,631.1618592)(717.05057617,630.74187228)
\curveto(716.85057166,630.32186004)(716.75057176,629.80186056)(716.75057617,629.18187228)
\lineto(716.75057617,618.08187228)
\lineto(714.23057617,618.08187228)
\lineto(714.23057617,634.16187228)
}
}
{
\newrgbcolor{curcolor}{0 0 0}
\pscustom[linewidth=2,linecolor=curcolor]
{
\newpath
\moveto(19.35304832,753.51134615)
\lineto(194.28938866,753.51134615)
\lineto(194.28938866,705.36071139)
\lineto(19.35304832,705.36071139)
\closepath
}
}
{
\newrgbcolor{curcolor}{0 0 0}
\pscustom[linestyle=none,fillstyle=solid,fillcolor=curcolor]
{
\newpath
\moveto(54.15434341,732.55602145)
\curveto(54.15433411,732.93600988)(54.10433416,733.32600949)(54.00434341,733.72602145)
\curveto(53.92433434,734.12600869)(53.78433448,734.48600833)(53.58434341,734.80602145)
\curveto(53.40433486,735.12600769)(53.14433512,735.38600743)(52.80434341,735.58602145)
\curveto(52.48433578,735.78600703)(52.08433618,735.88600693)(51.60434341,735.88602145)
\curveto(51.20433706,735.88600693)(50.81433745,735.816007)(50.43434341,735.67602145)
\curveto(50.07433819,735.53600728)(49.74433852,735.22600759)(49.44434341,734.74602145)
\curveto(49.14433912,734.28600853)(48.90433936,733.6160092)(48.72434341,732.73602145)
\curveto(48.54433972,731.85601096)(48.45433981,730.68601213)(48.45434341,729.22602145)
\curveto(48.45433981,728.70601411)(48.4643398,728.08601473)(48.48434341,727.36602145)
\curveto(48.52433974,726.64601617)(48.64433962,725.95601686)(48.84434341,725.29602145)
\curveto(49.04433922,724.63601818)(49.34433892,724.07601874)(49.74434341,723.61602145)
\curveto(50.1643381,723.15601966)(50.75433751,722.92601989)(51.51434341,722.92602145)
\curveto(52.05433621,722.92601989)(52.49433577,723.05601976)(52.83434341,723.31602145)
\curveto(53.17433509,723.57601924)(53.44433482,723.89601892)(53.64434341,724.27602145)
\curveto(53.84433442,724.67601814)(53.97433429,725.10601771)(54.03434341,725.56602145)
\curveto(54.11433415,726.04601677)(54.15433411,726.50601631)(54.15434341,726.94602145)
\lineto(56.67434341,726.94602145)
\curveto(56.67433159,726.30601651)(56.58433168,725.63601718)(56.40434341,724.93602145)
\curveto(56.24433202,724.23601858)(55.95433231,723.58601923)(55.53434341,722.98602145)
\curveto(55.13433313,722.40602041)(54.59433367,721.9160209)(53.91434341,721.51602145)
\curveto(53.23433503,721.13602168)(52.39433587,720.94602187)(51.39434341,720.94602145)
\curveto(49.41433885,720.94602187)(47.98434028,721.62602119)(47.10434341,722.98602145)
\curveto(46.24434202,724.34601847)(45.81434245,726.4160164)(45.81434341,729.19602145)
\curveto(45.81434245,730.19601262)(45.87434239,731.20601161)(45.99434341,732.22602145)
\curveto(46.13434213,733.26600955)(46.40434186,734.19600862)(46.80434341,735.01602145)
\curveto(47.22434104,735.85600696)(47.80434046,736.53600628)(48.54434341,737.05602145)
\curveto(49.30433896,737.59600522)(50.30433796,737.86600495)(51.54434341,737.86602145)
\curveto(52.64433562,737.86600495)(53.52433474,737.67600514)(54.18434341,737.29602145)
\curveto(54.8643334,736.9160059)(55.38433288,736.44600637)(55.74434341,735.88602145)
\curveto(56.12433214,735.34600747)(56.37433189,734.76600805)(56.49434341,734.14602145)
\curveto(56.61433165,733.54600927)(56.67433159,733.0160098)(56.67434341,732.55602145)
\lineto(54.15434341,732.55602145)
}
}
{
\newrgbcolor{curcolor}{0 0 0}
\pscustom[linestyle=none,fillstyle=solid,fillcolor=curcolor]
{
\newpath
\moveto(58.49778091,729.40602145)
\curveto(58.49778004,730.54601227)(58.57777996,731.62601119)(58.73778091,732.64602145)
\curveto(58.91777962,733.66600915)(59.21777932,734.55600826)(59.63778091,735.31602145)
\curveto(60.07777846,736.09600672)(60.66777787,736.7160061)(61.40778091,737.17602145)
\curveto(62.16777637,737.63600518)(63.12777541,737.86600495)(64.28778091,737.86602145)
\curveto(65.44777309,737.86600495)(66.39777214,737.63600518)(67.13778091,737.17602145)
\curveto(67.89777064,736.7160061)(68.48777005,736.09600672)(68.90778091,735.31602145)
\curveto(69.34776919,734.55600826)(69.64776889,733.66600915)(69.80778091,732.64602145)
\curveto(69.98776855,731.62601119)(70.07776846,730.54601227)(70.07778091,729.40602145)
\curveto(70.07776846,728.26601455)(69.98776855,727.18601563)(69.80778091,726.16602145)
\curveto(69.64776889,725.14601767)(69.34776919,724.24601857)(68.90778091,723.46602145)
\curveto(68.46777007,722.70602011)(67.87777066,722.09602072)(67.13778091,721.63602145)
\curveto(66.39777214,721.17602164)(65.44777309,720.94602187)(64.28778091,720.94602145)
\curveto(63.12777541,720.94602187)(62.16777637,721.17602164)(61.40778091,721.63602145)
\curveto(60.66777787,722.09602072)(60.07777846,722.70602011)(59.63778091,723.46602145)
\curveto(59.21777932,724.24601857)(58.91777962,725.14601767)(58.73778091,726.16602145)
\curveto(58.57777996,727.18601563)(58.49778004,728.26601455)(58.49778091,729.40602145)
\moveto(64.22778091,722.92602145)
\curveto(64.88777365,722.92601989)(65.42777311,723.09601972)(65.84778091,723.43602145)
\curveto(66.26777227,723.79601902)(66.59777194,724.26601855)(66.83778091,724.84602145)
\curveto(67.07777146,725.44601737)(67.2377713,726.13601668)(67.31778091,726.91602145)
\curveto(67.39777114,727.69601512)(67.4377711,728.52601429)(67.43778091,729.40602145)
\curveto(67.4377711,730.26601255)(67.39777114,731.08601173)(67.31778091,731.86602145)
\curveto(67.2377713,732.66601015)(67.07777146,733.35600946)(66.83778091,733.93602145)
\curveto(66.61777192,734.53600828)(66.29777224,735.00600781)(65.87778091,735.34602145)
\curveto(65.45777308,735.70600711)(64.90777363,735.88600693)(64.22778091,735.88602145)
\curveto(63.58777495,735.88600693)(63.06777547,735.70600711)(62.66778091,735.34602145)
\curveto(62.26777627,735.00600781)(61.94777659,734.53600828)(61.70778091,733.93602145)
\curveto(61.48777705,733.35600946)(61.3377772,732.66601015)(61.25778091,731.86602145)
\curveto(61.17777736,731.08601173)(61.1377774,730.26601255)(61.13778091,729.40602145)
\curveto(61.1377774,728.52601429)(61.17777736,727.69601512)(61.25778091,726.91602145)
\curveto(61.3377772,726.13601668)(61.48777705,725.44601737)(61.70778091,724.84602145)
\curveto(61.92777661,724.26601855)(62.2377763,723.79601902)(62.63778091,723.43602145)
\curveto(63.05777548,723.09601972)(63.58777495,722.92601989)(64.22778091,722.92602145)
}
}
{
\newrgbcolor{curcolor}{0 0 0}
\pscustom[linestyle=none,fillstyle=solid,fillcolor=curcolor]
{
\newpath
\moveto(72.81856216,737.44602145)
\lineto(75.21856216,737.44602145)
\lineto(75.21856216,735.55602145)
\lineto(75.27856216,735.55602145)
\curveto(75.61855747,736.29600652)(76.15855693,736.86600595)(76.89856216,737.26602145)
\curveto(77.63855545,737.66600515)(78.39855469,737.86600495)(79.17856216,737.86602145)
\curveto(80.13855295,737.86600495)(80.8885522,737.66600515)(81.42856216,737.26602145)
\curveto(81.9885511,736.88600593)(82.39855069,736.23600658)(82.65856216,735.31602145)
\curveto(83.01855007,736.0160068)(83.52854956,736.6160062)(84.18856216,737.11602145)
\curveto(84.86854822,737.6160052)(85.62854746,737.86600495)(86.46856216,737.86602145)
\curveto(87.52854556,737.86600495)(88.33854475,737.68600513)(88.89856216,737.32602145)
\curveto(89.47854361,736.98600583)(89.89854319,736.55600626)(90.15856216,736.03602145)
\curveto(90.43854265,735.5160073)(90.59854249,734.95600786)(90.63856216,734.35602145)
\curveto(90.67854241,733.77600904)(90.69854239,733.24600957)(90.69856216,732.76602145)
\lineto(90.69856216,721.36602145)
\lineto(88.17856216,721.36602145)
\lineto(88.17856216,732.46602145)
\curveto(88.17854491,732.76601005)(88.15854493,733.10600971)(88.11856216,733.48602145)
\curveto(88.09854499,733.86600895)(88.01854507,734.2160086)(87.87856216,734.53602145)
\curveto(87.73854535,734.87600794)(87.51854557,735.15600766)(87.21856216,735.37602145)
\curveto(86.93854615,735.59600722)(86.53854655,735.70600711)(86.01856216,735.70602145)
\curveto(85.71854737,735.70600711)(85.3885477,735.65600716)(85.02856216,735.55602145)
\curveto(84.6885484,735.45600736)(84.36854872,735.27600754)(84.06856216,735.01602145)
\curveto(83.76854932,734.77600804)(83.51854957,734.44600837)(83.31856216,734.02602145)
\curveto(83.11854997,733.60600921)(83.01855007,733.08600973)(83.01856216,732.46602145)
\lineto(83.01856216,721.36602145)
\lineto(80.49856216,721.36602145)
\lineto(80.49856216,732.46602145)
\curveto(80.49855259,732.76601005)(80.47855261,733.10600971)(80.43856216,733.48602145)
\curveto(80.41855267,733.86600895)(80.33855275,734.2160086)(80.19856216,734.53602145)
\curveto(80.05855303,734.87600794)(79.83855325,735.15600766)(79.53856216,735.37602145)
\curveto(79.25855383,735.59600722)(78.85855423,735.70600711)(78.33856216,735.70602145)
\curveto(78.03855505,735.70600711)(77.70855538,735.65600716)(77.34856216,735.55602145)
\curveto(77.00855608,735.45600736)(76.6885564,735.27600754)(76.38856216,735.01602145)
\curveto(76.088557,734.77600804)(75.83855725,734.44600837)(75.63856216,734.02602145)
\curveto(75.43855765,733.60600921)(75.33855775,733.08600973)(75.33856216,732.46602145)
\lineto(75.33856216,721.36602145)
\lineto(72.81856216,721.36602145)
\lineto(72.81856216,737.44602145)
}
}
{
\newrgbcolor{curcolor}{0 0 0}
\pscustom[linestyle=none,fillstyle=solid,fillcolor=curcolor]
{
\newpath
\moveto(94.49824966,737.44602145)
\lineto(96.89824966,737.44602145)
\lineto(96.89824966,735.55602145)
\lineto(96.95824966,735.55602145)
\curveto(97.29824497,736.29600652)(97.83824443,736.86600595)(98.57824966,737.26602145)
\curveto(99.31824295,737.66600515)(100.07824219,737.86600495)(100.85824966,737.86602145)
\curveto(101.81824045,737.86600495)(102.5682397,737.66600515)(103.10824966,737.26602145)
\curveto(103.6682386,736.88600593)(104.07823819,736.23600658)(104.33824966,735.31602145)
\curveto(104.69823757,736.0160068)(105.20823706,736.6160062)(105.86824966,737.11602145)
\curveto(106.54823572,737.6160052)(107.30823496,737.86600495)(108.14824966,737.86602145)
\curveto(109.20823306,737.86600495)(110.01823225,737.68600513)(110.57824966,737.32602145)
\curveto(111.15823111,736.98600583)(111.57823069,736.55600626)(111.83824966,736.03602145)
\curveto(112.11823015,735.5160073)(112.27822999,734.95600786)(112.31824966,734.35602145)
\curveto(112.35822991,733.77600904)(112.37822989,733.24600957)(112.37824966,732.76602145)
\lineto(112.37824966,721.36602145)
\lineto(109.85824966,721.36602145)
\lineto(109.85824966,732.46602145)
\curveto(109.85823241,732.76601005)(109.83823243,733.10600971)(109.79824966,733.48602145)
\curveto(109.77823249,733.86600895)(109.69823257,734.2160086)(109.55824966,734.53602145)
\curveto(109.41823285,734.87600794)(109.19823307,735.15600766)(108.89824966,735.37602145)
\curveto(108.61823365,735.59600722)(108.21823405,735.70600711)(107.69824966,735.70602145)
\curveto(107.39823487,735.70600711)(107.0682352,735.65600716)(106.70824966,735.55602145)
\curveto(106.3682359,735.45600736)(106.04823622,735.27600754)(105.74824966,735.01602145)
\curveto(105.44823682,734.77600804)(105.19823707,734.44600837)(104.99824966,734.02602145)
\curveto(104.79823747,733.60600921)(104.69823757,733.08600973)(104.69824966,732.46602145)
\lineto(104.69824966,721.36602145)
\lineto(102.17824966,721.36602145)
\lineto(102.17824966,732.46602145)
\curveto(102.17824009,732.76601005)(102.15824011,733.10600971)(102.11824966,733.48602145)
\curveto(102.09824017,733.86600895)(102.01824025,734.2160086)(101.87824966,734.53602145)
\curveto(101.73824053,734.87600794)(101.51824075,735.15600766)(101.21824966,735.37602145)
\curveto(100.93824133,735.59600722)(100.53824173,735.70600711)(100.01824966,735.70602145)
\curveto(99.71824255,735.70600711)(99.38824288,735.65600716)(99.02824966,735.55602145)
\curveto(98.68824358,735.45600736)(98.3682439,735.27600754)(98.06824966,735.01602145)
\curveto(97.7682445,734.77600804)(97.51824475,734.44600837)(97.31824966,734.02602145)
\curveto(97.11824515,733.60600921)(97.01824525,733.08600973)(97.01824966,732.46602145)
\lineto(97.01824966,721.36602145)
\lineto(94.49824966,721.36602145)
\lineto(94.49824966,737.44602145)
}
}
{
\newrgbcolor{curcolor}{0 0 0}
\pscustom[linestyle=none,fillstyle=solid,fillcolor=curcolor]
{
\newpath
\moveto(126.52793716,721.36602145)
\lineto(124.12793716,721.36602145)
\lineto(124.12793716,723.25602145)
\lineto(124.06793716,723.25602145)
\curveto(123.72792772,722.5160203)(123.18792826,721.94602087)(122.44793716,721.54602145)
\curveto(121.70792974,721.14602167)(120.9479305,720.94602187)(120.16793716,720.94602145)
\curveto(119.10793234,720.94602187)(118.28793316,721.1160217)(117.70793716,721.45602145)
\curveto(117.1479343,721.816021)(116.72793472,722.25602056)(116.44793716,722.77602145)
\curveto(116.18793526,723.29601952)(116.03793541,723.84601897)(115.99793716,724.42602145)
\curveto(115.95793549,725.02601779)(115.93793551,725.56601725)(115.93793716,726.04602145)
\lineto(115.93793716,737.44602145)
\lineto(118.45793716,737.44602145)
\lineto(118.45793716,726.34602145)
\curveto(118.45793299,726.04601677)(118.46793298,725.70601711)(118.48793716,725.32602145)
\curveto(118.52793292,724.94601787)(118.62793282,724.58601823)(118.78793716,724.24602145)
\curveto(118.9479325,723.92601889)(119.17793227,723.65601916)(119.47793716,723.43602145)
\curveto(119.79793165,723.2160196)(120.2479312,723.10601971)(120.82793716,723.10602145)
\curveto(121.16793028,723.10601971)(121.51792993,723.16601965)(121.87793716,723.28602145)
\curveto(122.25792919,723.40601941)(122.60792884,723.59601922)(122.92793716,723.85602145)
\curveto(123.2479282,724.1160187)(123.50792794,724.44601837)(123.70793716,724.84602145)
\curveto(123.90792754,725.26601755)(124.00792744,725.76601705)(124.00793716,726.34602145)
\lineto(124.00793716,737.44602145)
\lineto(126.52793716,737.44602145)
\lineto(126.52793716,721.36602145)
}
}
{
\newrgbcolor{curcolor}{0 0 0}
\pscustom[linestyle=none,fillstyle=solid,fillcolor=curcolor]
{
\newpath
\moveto(129.82465591,737.44602145)
\lineto(132.22465591,737.44602145)
\lineto(132.22465591,735.55602145)
\lineto(132.28465591,735.55602145)
\curveto(132.62465146,736.29600652)(133.16465092,736.86600595)(133.90465591,737.26602145)
\curveto(134.64464944,737.66600515)(135.40464868,737.86600495)(136.18465591,737.86602145)
\curveto(137.24464684,737.86600495)(138.05464603,737.68600513)(138.61465591,737.32602145)
\curveto(139.19464489,736.98600583)(139.61464447,736.55600626)(139.87465591,736.03602145)
\curveto(140.15464393,735.5160073)(140.31464377,734.95600786)(140.35465591,734.35602145)
\curveto(140.39464369,733.77600904)(140.41464367,733.24600957)(140.41465591,732.76602145)
\lineto(140.41465591,721.36602145)
\lineto(137.89465591,721.36602145)
\lineto(137.89465591,732.46602145)
\curveto(137.89464619,732.76601005)(137.87464621,733.10600971)(137.83465591,733.48602145)
\curveto(137.81464627,733.86600895)(137.72464636,734.2160086)(137.56465591,734.53602145)
\curveto(137.40464668,734.87600794)(137.16464692,735.15600766)(136.84465591,735.37602145)
\curveto(136.52464756,735.59600722)(136.084648,735.70600711)(135.52465591,735.70602145)
\curveto(135.1846489,735.70600711)(134.82464926,735.64600717)(134.44465591,735.52602145)
\curveto(134.08465,735.40600741)(133.74465034,735.2160076)(133.42465591,734.95602145)
\curveto(133.10465098,734.69600812)(132.84465124,734.35600846)(132.64465591,733.93602145)
\curveto(132.44465164,733.53600928)(132.34465174,733.04600977)(132.34465591,732.46602145)
\lineto(132.34465591,721.36602145)
\lineto(129.82465591,721.36602145)
\lineto(129.82465591,737.44602145)
}
}
{
\newrgbcolor{curcolor}{0 0 0}
\pscustom[linestyle=none,fillstyle=solid,fillcolor=curcolor]
{
\newpath
\moveto(143.86137466,742.78602145)
\lineto(146.38137466,742.78602145)
\lineto(146.38137466,739.90602145)
\lineto(143.86137466,739.90602145)
\lineto(143.86137466,742.78602145)
\moveto(143.86137466,737.44602145)
\lineto(146.38137466,737.44602145)
\lineto(146.38137466,721.36602145)
\lineto(143.86137466,721.36602145)
\lineto(143.86137466,737.44602145)
}
}
{
\newrgbcolor{curcolor}{0 0 0}
\pscustom[linestyle=none,fillstyle=solid,fillcolor=curcolor]
{
\newpath
\moveto(150.55512466,742.12602145)
\lineto(153.07512466,742.12602145)
\lineto(153.07512466,737.44602145)
\lineto(155.86512466,737.44602145)
\lineto(155.86512466,735.46602145)
\lineto(153.07512466,735.46602145)
\lineto(153.07512466,725.14602145)
\curveto(153.07511974,724.50601831)(153.18511963,724.04601877)(153.40512466,723.76602145)
\curveto(153.62511919,723.48601933)(154.06511875,723.34601947)(154.72512466,723.34602145)
\curveto(155.00511781,723.34601947)(155.22511759,723.35601946)(155.38512466,723.37602145)
\curveto(155.54511727,723.39601942)(155.69511712,723.4160194)(155.83512466,723.43602145)
\lineto(155.83512466,721.36602145)
\curveto(155.67511714,721.32602149)(155.42511739,721.28602153)(155.08512466,721.24602145)
\curveto(154.74511807,721.20602161)(154.3151185,721.18602163)(153.79512466,721.18602145)
\curveto(153.13511968,721.18602163)(152.59512022,721.24602157)(152.17512466,721.36602145)
\curveto(151.75512106,721.50602131)(151.42512139,721.70602111)(151.18512466,721.96602145)
\curveto(150.94512187,722.24602057)(150.77512204,722.58602023)(150.67512466,722.98602145)
\curveto(150.59512222,723.38601943)(150.55512226,723.84601897)(150.55512466,724.36602145)
\lineto(150.55512466,735.46602145)
\lineto(148.21512466,735.46602145)
\lineto(148.21512466,737.44602145)
\lineto(150.55512466,737.44602145)
\lineto(150.55512466,742.12602145)
}
}
{
\newrgbcolor{curcolor}{0 0 0}
\pscustom[linestyle=none,fillstyle=solid,fillcolor=curcolor]
{
\newpath
\moveto(156.24809341,737.44602145)
\lineto(159.00809341,737.44602145)
\lineto(162.18809341,724.54602145)
\lineto(162.24809341,724.54602145)
\lineto(165.06809341,737.44602145)
\lineto(167.82809341,737.44602145)
\lineto(163.17809341,720.28602145)
\curveto(163.01808634,719.72602309)(162.84808651,719.19602362)(162.66809341,718.69602145)
\curveto(162.48808687,718.19602462)(162.23808712,717.75602506)(161.91809341,717.37602145)
\curveto(161.61808774,716.97602584)(161.22808813,716.66602615)(160.74809341,716.44602145)
\curveto(160.26808909,716.20602661)(159.64808971,716.08602673)(158.88809341,716.08602145)
\curveto(158.38809097,716.08602673)(158.00809135,716.09602672)(157.74809341,716.11602145)
\curveto(157.50809185,716.13602668)(157.27809208,716.15602666)(157.05809341,716.17602145)
\lineto(157.05809341,718.15602145)
\curveto(157.41809194,718.09602472)(157.90809145,718.06602475)(158.52809341,718.06602145)
\curveto(159.10809025,718.06602475)(159.53808982,718.2160246)(159.81809341,718.51602145)
\curveto(160.11808924,718.816024)(160.34808901,719.18602363)(160.50809341,719.62602145)
\lineto(160.98809341,721.03602145)
\lineto(156.24809341,737.44602145)
}
}
{
\newrgbcolor{curcolor}{0 0 0}
\pscustom[linewidth=2,linecolor=curcolor]
{
\newpath
\moveto(303.49185181,388.84050875)
\lineto(543.56634521,388.84050875)
\lineto(543.56634521,340.68987398)
\lineto(303.49185181,340.68987398)
\closepath
}
}
{
\newrgbcolor{curcolor}{0 0 0}
\pscustom[linestyle=none,fillstyle=solid,fillcolor=curcolor]
{
\newpath
\moveto(360.14567261,368.18517916)
\curveto(360.14566406,369.20516677)(359.97566423,369.98516599)(359.63567261,370.52517916)
\curveto(359.31566489,371.06516491)(358.69566551,371.33516464)(357.77567261,371.33517916)
\curveto(357.57566663,371.33516464)(357.32566688,371.30516467)(357.02567261,371.24517916)
\curveto(356.72566748,371.20516477)(356.43566777,371.10516487)(356.15567261,370.94517916)
\curveto(355.89566831,370.78516519)(355.66566854,370.53516544)(355.46567261,370.19517916)
\curveto(355.26566894,369.8751661)(355.16566904,369.43516654)(355.16567261,368.87517916)
\curveto(355.16566904,368.39516758)(355.27566893,368.00516797)(355.49567261,367.70517916)
\curveto(355.73566847,367.40516857)(356.03566817,367.14516883)(356.39567261,366.92517916)
\curveto(356.77566743,366.72516925)(357.19566701,366.55516942)(357.65567261,366.41517916)
\curveto(358.13566607,366.2751697)(358.62566558,366.12516985)(359.12567261,365.96517916)
\curveto(359.6056646,365.80517017)(360.07566413,365.62517035)(360.53567261,365.42517916)
\curveto(361.01566319,365.24517073)(361.43566277,364.98517099)(361.79567261,364.64517916)
\curveto(362.17566203,364.32517165)(362.47566173,363.90517207)(362.69567261,363.38517916)
\curveto(362.93566127,362.88517309)(363.05566115,362.23517374)(363.05567261,361.43517916)
\curveto(363.05566115,360.59517538)(362.92566128,359.85517612)(362.66567261,359.21517916)
\curveto(362.4056618,358.59517738)(362.04566216,358.0751779)(361.58567261,357.65517916)
\curveto(361.12566308,357.23517874)(360.57566363,356.92517905)(359.93567261,356.72517916)
\curveto(359.29566491,356.50517947)(358.6056656,356.39517958)(357.86567261,356.39517916)
\curveto(356.5056677,356.39517958)(355.44566876,356.61517936)(354.68567261,357.05517916)
\curveto(353.94567026,357.49517848)(353.39567081,358.01517796)(353.03567261,358.61517916)
\curveto(352.69567151,359.23517674)(352.49567171,359.86517611)(352.43567261,360.50517916)
\curveto(352.37567183,361.14517483)(352.34567186,361.6751743)(352.34567261,362.09517916)
\lineto(354.86567261,362.09517916)
\curveto(354.86566934,361.61517436)(354.9056693,361.14517483)(354.98567261,360.68517916)
\curveto(355.06566914,360.24517573)(355.21566899,359.84517613)(355.43567261,359.48517916)
\curveto(355.65566855,359.14517683)(355.95566825,358.8751771)(356.33567261,358.67517916)
\curveto(356.73566747,358.4751775)(357.24566696,358.3751776)(357.86567261,358.37517916)
\curveto(358.06566614,358.3751776)(358.31566589,358.40517757)(358.61567261,358.46517916)
\curveto(358.91566529,358.52517745)(359.205665,358.65517732)(359.48567261,358.85517916)
\curveto(359.78566442,359.05517692)(360.03566417,359.32517665)(360.23567261,359.66517916)
\curveto(360.43566377,360.00517597)(360.53566367,360.46517551)(360.53567261,361.04517916)
\curveto(360.53566367,361.58517439)(360.41566379,362.02517395)(360.17567261,362.36517916)
\curveto(359.95566425,362.72517325)(359.65566455,363.01517296)(359.27567261,363.23517916)
\curveto(358.91566529,363.4751725)(358.49566571,363.6751723)(358.01567261,363.83517916)
\curveto(357.55566665,363.99517198)(357.08566712,364.15517182)(356.60567261,364.31517916)
\curveto(356.12566808,364.4751715)(355.64566856,364.64517133)(355.16567261,364.82517916)
\curveto(354.68566952,365.02517095)(354.25566995,365.28517069)(353.87567261,365.60517916)
\curveto(353.51567069,365.92517005)(353.21567099,366.34516963)(352.97567261,366.86517916)
\curveto(352.75567145,367.38516859)(352.64567156,368.05516792)(352.64567261,368.87517916)
\curveto(352.64567156,369.61516636)(352.77567143,370.26516571)(353.03567261,370.82517916)
\curveto(353.31567089,371.38516459)(353.68567052,371.84516413)(354.14567261,372.20517916)
\curveto(354.62566958,372.58516339)(355.17566903,372.86516311)(355.79567261,373.04517916)
\curveto(356.41566779,373.22516275)(357.07566713,373.31516266)(357.77567261,373.31517916)
\curveto(358.93566527,373.31516266)(359.84566436,373.13516284)(360.50567261,372.77517916)
\curveto(361.16566304,372.41516356)(361.65566255,371.975164)(361.97567261,371.45517916)
\curveto(362.29566191,370.93516504)(362.48566172,370.3751656)(362.54567261,369.77517916)
\curveto(362.62566158,369.19516678)(362.66566154,368.66516731)(362.66567261,368.18517916)
\lineto(360.14567261,368.18517916)
}
}
{
\newrgbcolor{curcolor}{0 0 0}
\pscustom[linestyle=none,fillstyle=solid,fillcolor=curcolor]
{
\newpath
\moveto(376.02317261,356.81517916)
\lineto(373.62317261,356.81517916)
\lineto(373.62317261,358.70517916)
\lineto(373.56317261,358.70517916)
\curveto(373.22316317,357.96517801)(372.68316371,357.39517858)(371.94317261,356.99517916)
\curveto(371.20316519,356.59517938)(370.44316595,356.39517958)(369.66317261,356.39517916)
\curveto(368.60316779,356.39517958)(367.78316861,356.56517941)(367.20317261,356.90517916)
\curveto(366.64316975,357.26517871)(366.22317017,357.70517827)(365.94317261,358.22517916)
\curveto(365.68317071,358.74517723)(365.53317086,359.29517668)(365.49317261,359.87517916)
\curveto(365.45317094,360.4751755)(365.43317096,361.01517496)(365.43317261,361.49517916)
\lineto(365.43317261,372.89517916)
\lineto(367.95317261,372.89517916)
\lineto(367.95317261,361.79517916)
\curveto(367.95316844,361.49517448)(367.96316843,361.15517482)(367.98317261,360.77517916)
\curveto(368.02316837,360.39517558)(368.12316827,360.03517594)(368.28317261,359.69517916)
\curveto(368.44316795,359.3751766)(368.67316772,359.10517687)(368.97317261,358.88517916)
\curveto(369.2931671,358.66517731)(369.74316665,358.55517742)(370.32317261,358.55517916)
\curveto(370.66316573,358.55517742)(371.01316538,358.61517736)(371.37317261,358.73517916)
\curveto(371.75316464,358.85517712)(372.10316429,359.04517693)(372.42317261,359.30517916)
\curveto(372.74316365,359.56517641)(373.00316339,359.89517608)(373.20317261,360.29517916)
\curveto(373.40316299,360.71517526)(373.50316289,361.21517476)(373.50317261,361.79517916)
\lineto(373.50317261,372.89517916)
\lineto(376.02317261,372.89517916)
\lineto(376.02317261,356.81517916)
}
}
{
\newrgbcolor{curcolor}{0 0 0}
\pscustom[linestyle=none,fillstyle=solid,fillcolor=curcolor]
{
\newpath
\moveto(379.31989136,378.23517916)
\lineto(381.83989136,378.23517916)
\lineto(381.83989136,371.06517916)
\lineto(381.89989136,371.06517916)
\curveto(382.17988685,371.76516421)(382.65988637,372.31516366)(383.33989136,372.71517916)
\curveto(384.01988501,373.11516286)(384.77988425,373.31516266)(385.61989136,373.31517916)
\curveto(386.69988233,373.31516266)(387.55988147,373.02516295)(388.19989136,372.44517916)
\curveto(388.85988017,371.88516409)(389.35987967,371.18516479)(389.69989136,370.34517916)
\curveto(390.03987899,369.50516647)(390.25987877,368.58516739)(390.35989136,367.58517916)
\curveto(390.45987857,366.60516937)(390.50987852,365.69517028)(390.50989136,364.85517916)
\curveto(390.50987852,363.71517226)(390.40987862,362.63517334)(390.20989136,361.61517916)
\curveto(390.00987902,360.59517538)(389.69987933,359.69517628)(389.27989136,358.91517916)
\curveto(388.85988017,358.15517782)(388.30988072,357.54517843)(387.62989136,357.08517916)
\curveto(386.96988206,356.62517935)(386.17988285,356.39517958)(385.25989136,356.39517916)
\curveto(384.81988421,356.39517958)(384.40988462,356.46517951)(384.02989136,356.60517916)
\curveto(383.64988538,356.74517923)(383.29988573,356.92517905)(382.97989136,357.14517916)
\curveto(382.67988635,357.36517861)(382.41988661,357.61517836)(382.19989136,357.89517916)
\curveto(381.99988703,358.19517778)(381.85988717,358.49517748)(381.77989136,358.79517916)
\lineto(381.71989136,358.79517916)
\lineto(381.71989136,356.81517916)
\lineto(379.31989136,356.81517916)
\lineto(379.31989136,378.23517916)
\moveto(381.68989136,364.85517916)
\curveto(381.68988734,364.03517194)(381.7298873,363.24517273)(381.80989136,362.48517916)
\curveto(381.88988714,361.72517425)(382.04988698,361.05517492)(382.28989136,360.47517916)
\curveto(382.5298865,359.89517608)(382.85988617,359.42517655)(383.27989136,359.06517916)
\curveto(383.69988533,358.72517725)(384.24988478,358.55517742)(384.92989136,358.55517916)
\curveto(385.50988352,358.55517742)(385.98988304,358.70517727)(386.36989136,359.00517916)
\curveto(386.74988228,359.32517665)(387.04988198,359.76517621)(387.26989136,360.32517916)
\curveto(387.48988154,360.88517509)(387.63988139,361.54517443)(387.71989136,362.30517916)
\curveto(387.81988121,363.08517289)(387.86988116,363.93517204)(387.86989136,364.85517916)
\curveto(387.86988116,365.81517016)(387.81988121,366.68516929)(387.71989136,367.46517916)
\curveto(387.63988139,368.24516773)(387.48988154,368.90516707)(387.26989136,369.44517916)
\curveto(387.04988198,369.98516599)(386.74988228,370.40516557)(386.36989136,370.70517916)
\curveto(385.98988304,371.00516497)(385.50988352,371.15516482)(384.92989136,371.15517916)
\curveto(384.24988478,371.15516482)(383.69988533,370.975165)(383.27989136,370.61517916)
\curveto(382.85988617,370.25516572)(382.5298865,369.7751662)(382.28989136,369.17517916)
\curveto(382.04988698,368.5751674)(381.88988714,367.89516808)(381.80989136,367.13517916)
\curveto(381.7298873,366.39516958)(381.68988734,365.63517034)(381.68989136,364.85517916)
}
}
{
\newrgbcolor{curcolor}{0 0 0}
\pscustom[linestyle=none,fillstyle=solid,fillcolor=curcolor]
{
\newpath
\moveto(395.87661011,375.35517916)
\lineto(393.35661011,375.35517916)
\lineto(393.35661011,378.23517916)
\lineto(395.87661011,378.23517916)
\lineto(395.87661011,375.35517916)
\moveto(395.87661011,355.25517916)
\curveto(395.87660579,353.93518204)(395.57660609,352.94518303)(394.97661011,352.28517916)
\curveto(394.39660727,351.62518435)(393.41660825,351.29518468)(392.03661011,351.29517916)
\curveto(391.83660983,351.29518468)(391.64661002,351.30518467)(391.46661011,351.32517916)
\lineto(390.86661011,351.38517916)
\lineto(390.86661011,353.54517916)
\lineto(391.34661011,353.48517916)
\curveto(391.48661018,353.46518251)(391.63661003,353.45518252)(391.79661011,353.45517916)
\curveto(392.17660949,353.45518252)(392.47660919,353.52518245)(392.69661011,353.66517916)
\curveto(392.91660875,353.80518217)(393.0666086,353.99518198)(393.14661011,354.23517916)
\curveto(393.24660842,354.4751815)(393.30660836,354.76518121)(393.32661011,355.10517916)
\curveto(393.34660832,355.42518055)(393.35660831,355.7751802)(393.35661011,356.15517916)
\lineto(393.35661011,372.89517916)
\lineto(395.87661011,372.89517916)
\lineto(395.87661011,355.25517916)
}
}
{
\newrgbcolor{curcolor}{0 0 0}
\pscustom[linestyle=none,fillstyle=solid,fillcolor=curcolor]
{
\newpath
\moveto(401.40036011,364.55517916)
\curveto(401.40035636,363.93517204)(401.41035635,363.26517271)(401.43036011,362.54517916)
\curveto(401.47035629,361.82517415)(401.59035617,361.15517482)(401.79036011,360.53517916)
\curveto(401.99035577,359.91517606)(402.29035547,359.39517658)(402.69036011,358.97517916)
\curveto(403.11035465,358.5751774)(403.71035405,358.3751776)(404.49036011,358.37517916)
\curveto(405.09035267,358.3751776)(405.57035219,358.50517747)(405.93036011,358.76517916)
\curveto(406.29035147,359.04517693)(406.5603512,359.3751766)(406.74036011,359.75517916)
\curveto(406.94035082,360.15517582)(407.07035069,360.56517541)(407.13036011,360.98517916)
\curveto(407.19035057,361.42517455)(407.22035054,361.79517418)(407.22036011,362.09517916)
\lineto(409.74036011,362.09517916)
\curveto(409.74034802,361.6751743)(409.6603481,361.13517484)(409.50036011,360.47517916)
\curveto(409.3603484,359.83517614)(409.09034867,359.20517677)(408.69036011,358.58517916)
\curveto(408.29034947,357.98517799)(407.74035002,357.46517851)(407.04036011,357.02517916)
\curveto(406.34035142,356.60517937)(405.44035232,356.39517958)(404.34036011,356.39517916)
\curveto(402.3603554,356.39517958)(400.93035683,357.0751789)(400.05036011,358.43517916)
\curveto(399.19035857,359.79517618)(398.760359,361.86517411)(398.76036011,364.64517916)
\curveto(398.760359,365.64517033)(398.82035894,366.65516932)(398.94036011,367.67517916)
\curveto(399.08035868,368.71516726)(399.35035841,369.64516633)(399.75036011,370.46517916)
\curveto(400.17035759,371.30516467)(400.75035701,371.98516399)(401.49036011,372.50517916)
\curveto(402.25035551,373.04516293)(403.25035451,373.31516266)(404.49036011,373.31517916)
\curveto(405.71035205,373.31516266)(406.68035108,373.0751629)(407.40036011,372.59517916)
\curveto(408.12034964,372.11516386)(408.6603491,371.49516448)(409.02036011,370.73517916)
\curveto(409.38034838,369.99516598)(409.61034815,369.16516681)(409.71036011,368.24517916)
\curveto(409.81034795,367.32516865)(409.8603479,366.43516954)(409.86036011,365.57517916)
\lineto(409.86036011,364.55517916)
\lineto(401.40036011,364.55517916)
\moveto(407.22036011,366.53517916)
\lineto(407.22036011,367.40517916)
\curveto(407.22035054,367.84516813)(407.18035058,368.29516768)(407.10036011,368.75517916)
\curveto(407.02035074,369.23516674)(406.87035089,369.66516631)(406.65036011,370.04517916)
\curveto(406.45035131,370.42516555)(406.17035159,370.73516524)(405.81036011,370.97517916)
\curveto(405.45035231,371.21516476)(404.99035277,371.33516464)(404.43036011,371.33517916)
\curveto(403.77035399,371.33516464)(403.24035452,371.15516482)(402.84036011,370.79517916)
\curveto(402.4603553,370.45516552)(402.17035559,370.05516592)(401.97036011,369.59517916)
\curveto(401.77035599,369.13516684)(401.64035612,368.66516731)(401.58036011,368.18517916)
\curveto(401.52035624,367.72516825)(401.49035627,367.3751686)(401.49036011,367.13517916)
\lineto(401.49036011,366.53517916)
\lineto(407.22036011,366.53517916)
}
}
{
\newrgbcolor{curcolor}{0 0 0}
\pscustom[linestyle=none,fillstyle=solid,fillcolor=curcolor]
{
\newpath
\moveto(420.25114136,368.00517916)
\curveto(420.25113206,368.38516759)(420.20113211,368.7751672)(420.10114136,369.17517916)
\curveto(420.02113229,369.5751664)(419.88113243,369.93516604)(419.68114136,370.25517916)
\curveto(419.50113281,370.5751654)(419.24113307,370.83516514)(418.90114136,371.03517916)
\curveto(418.58113373,371.23516474)(418.18113413,371.33516464)(417.70114136,371.33517916)
\curveto(417.30113501,371.33516464)(416.9111354,371.26516471)(416.53114136,371.12517916)
\curveto(416.17113614,370.98516499)(415.84113647,370.6751653)(415.54114136,370.19517916)
\curveto(415.24113707,369.73516624)(415.00113731,369.06516691)(414.82114136,368.18517916)
\curveto(414.64113767,367.30516867)(414.55113776,366.13516984)(414.55114136,364.67517916)
\curveto(414.55113776,364.15517182)(414.56113775,363.53517244)(414.58114136,362.81517916)
\curveto(414.62113769,362.09517388)(414.74113757,361.40517457)(414.94114136,360.74517916)
\curveto(415.14113717,360.08517589)(415.44113687,359.52517645)(415.84114136,359.06517916)
\curveto(416.26113605,358.60517737)(416.85113546,358.3751776)(417.61114136,358.37517916)
\curveto(418.15113416,358.3751776)(418.59113372,358.50517747)(418.93114136,358.76517916)
\curveto(419.27113304,359.02517695)(419.54113277,359.34517663)(419.74114136,359.72517916)
\curveto(419.94113237,360.12517585)(420.07113224,360.55517542)(420.13114136,361.01517916)
\curveto(420.2111321,361.49517448)(420.25113206,361.95517402)(420.25114136,362.39517916)
\lineto(422.77114136,362.39517916)
\curveto(422.77112954,361.75517422)(422.68112963,361.08517489)(422.50114136,360.38517916)
\curveto(422.34112997,359.68517629)(422.05113026,359.03517694)(421.63114136,358.43517916)
\curveto(421.23113108,357.85517812)(420.69113162,357.36517861)(420.01114136,356.96517916)
\curveto(419.33113298,356.58517939)(418.49113382,356.39517958)(417.49114136,356.39517916)
\curveto(415.5111368,356.39517958)(414.08113823,357.0751789)(413.20114136,358.43517916)
\curveto(412.34113997,359.79517618)(411.9111404,361.86517411)(411.91114136,364.64517916)
\curveto(411.9111404,365.64517033)(411.97114034,366.65516932)(412.09114136,367.67517916)
\curveto(412.23114008,368.71516726)(412.50113981,369.64516633)(412.90114136,370.46517916)
\curveto(413.32113899,371.30516467)(413.90113841,371.98516399)(414.64114136,372.50517916)
\curveto(415.40113691,373.04516293)(416.40113591,373.31516266)(417.64114136,373.31517916)
\curveto(418.74113357,373.31516266)(419.62113269,373.12516285)(420.28114136,372.74517916)
\curveto(420.96113135,372.36516361)(421.48113083,371.89516408)(421.84114136,371.33517916)
\curveto(422.22113009,370.79516518)(422.47112984,370.21516576)(422.59114136,369.59517916)
\curveto(422.7111296,368.99516698)(422.77112954,368.46516751)(422.77114136,368.00517916)
\lineto(420.25114136,368.00517916)
}
}
{
\newrgbcolor{curcolor}{0 0 0}
\pscustom[linestyle=none,fillstyle=solid,fillcolor=curcolor]
{
\newpath
\moveto(426.12457886,377.57517916)
\lineto(428.64457886,377.57517916)
\lineto(428.64457886,372.89517916)
\lineto(431.43457886,372.89517916)
\lineto(431.43457886,370.91517916)
\lineto(428.64457886,370.91517916)
\lineto(428.64457886,360.59517916)
\curveto(428.64457394,359.95517602)(428.75457383,359.49517648)(428.97457886,359.21517916)
\curveto(429.19457339,358.93517704)(429.63457295,358.79517718)(430.29457886,358.79517916)
\curveto(430.57457201,358.79517718)(430.79457179,358.80517717)(430.95457886,358.82517916)
\curveto(431.11457147,358.84517713)(431.26457132,358.86517711)(431.40457886,358.88517916)
\lineto(431.40457886,356.81517916)
\curveto(431.24457134,356.7751792)(430.99457159,356.73517924)(430.65457886,356.69517916)
\curveto(430.31457227,356.65517932)(429.8845727,356.63517934)(429.36457886,356.63517916)
\curveto(428.70457388,356.63517934)(428.16457442,356.69517928)(427.74457886,356.81517916)
\curveto(427.32457526,356.95517902)(426.99457559,357.15517882)(426.75457886,357.41517916)
\curveto(426.51457607,357.69517828)(426.34457624,358.03517794)(426.24457886,358.43517916)
\curveto(426.16457642,358.83517714)(426.12457646,359.29517668)(426.12457886,359.81517916)
\lineto(426.12457886,370.91517916)
\lineto(423.78457886,370.91517916)
\lineto(423.78457886,372.89517916)
\lineto(426.12457886,372.89517916)
\lineto(426.12457886,377.57517916)
}
}
{
\newrgbcolor{curcolor}{0 0 0}
\pscustom[linestyle=none,fillstyle=solid,fillcolor=curcolor]
{
\newpath
\moveto(431.51754761,354.56517916)
\lineto(446.51754761,354.56517916)
\lineto(446.51754761,353.06517916)
\lineto(431.51754761,353.06517916)
\lineto(431.51754761,354.56517916)
}
}
{
\newrgbcolor{curcolor}{0 0 0}
\pscustom[linestyle=none,fillstyle=solid,fillcolor=curcolor]
{
\newpath
\moveto(458.75754761,356.81517916)
\lineto(456.35754761,356.81517916)
\lineto(456.35754761,358.70517916)
\lineto(456.29754761,358.70517916)
\curveto(455.95753817,357.96517801)(455.41753871,357.39517858)(454.67754761,356.99517916)
\curveto(453.93754019,356.59517938)(453.17754095,356.39517958)(452.39754761,356.39517916)
\curveto(451.33754279,356.39517958)(450.51754361,356.56517941)(449.93754761,356.90517916)
\curveto(449.37754475,357.26517871)(448.95754517,357.70517827)(448.67754761,358.22517916)
\curveto(448.41754571,358.74517723)(448.26754586,359.29517668)(448.22754761,359.87517916)
\curveto(448.18754594,360.4751755)(448.16754596,361.01517496)(448.16754761,361.49517916)
\lineto(448.16754761,372.89517916)
\lineto(450.68754761,372.89517916)
\lineto(450.68754761,361.79517916)
\curveto(450.68754344,361.49517448)(450.69754343,361.15517482)(450.71754761,360.77517916)
\curveto(450.75754337,360.39517558)(450.85754327,360.03517594)(451.01754761,359.69517916)
\curveto(451.17754295,359.3751766)(451.40754272,359.10517687)(451.70754761,358.88517916)
\curveto(452.0275421,358.66517731)(452.47754165,358.55517742)(453.05754761,358.55517916)
\curveto(453.39754073,358.55517742)(453.74754038,358.61517736)(454.10754761,358.73517916)
\curveto(454.48753964,358.85517712)(454.83753929,359.04517693)(455.15754761,359.30517916)
\curveto(455.47753865,359.56517641)(455.73753839,359.89517608)(455.93754761,360.29517916)
\curveto(456.13753799,360.71517526)(456.23753789,361.21517476)(456.23754761,361.79517916)
\lineto(456.23754761,372.89517916)
\lineto(458.75754761,372.89517916)
\lineto(458.75754761,356.81517916)
}
}
{
\newrgbcolor{curcolor}{0 0 0}
\pscustom[linestyle=none,fillstyle=solid,fillcolor=curcolor]
{
\newpath
\moveto(468.95426636,368.18517916)
\curveto(468.95425781,369.20516677)(468.78425798,369.98516599)(468.44426636,370.52517916)
\curveto(468.12425864,371.06516491)(467.50425926,371.33516464)(466.58426636,371.33517916)
\curveto(466.38426038,371.33516464)(466.13426063,371.30516467)(465.83426636,371.24517916)
\curveto(465.53426123,371.20516477)(465.24426152,371.10516487)(464.96426636,370.94517916)
\curveto(464.70426206,370.78516519)(464.47426229,370.53516544)(464.27426636,370.19517916)
\curveto(464.07426269,369.8751661)(463.97426279,369.43516654)(463.97426636,368.87517916)
\curveto(463.97426279,368.39516758)(464.08426268,368.00516797)(464.30426636,367.70517916)
\curveto(464.54426222,367.40516857)(464.84426192,367.14516883)(465.20426636,366.92517916)
\curveto(465.58426118,366.72516925)(466.00426076,366.55516942)(466.46426636,366.41517916)
\curveto(466.94425982,366.2751697)(467.43425933,366.12516985)(467.93426636,365.96517916)
\curveto(468.41425835,365.80517017)(468.88425788,365.62517035)(469.34426636,365.42517916)
\curveto(469.82425694,365.24517073)(470.24425652,364.98517099)(470.60426636,364.64517916)
\curveto(470.98425578,364.32517165)(471.28425548,363.90517207)(471.50426636,363.38517916)
\curveto(471.74425502,362.88517309)(471.8642549,362.23517374)(471.86426636,361.43517916)
\curveto(471.8642549,360.59517538)(471.73425503,359.85517612)(471.47426636,359.21517916)
\curveto(471.21425555,358.59517738)(470.85425591,358.0751779)(470.39426636,357.65517916)
\curveto(469.93425683,357.23517874)(469.38425738,356.92517905)(468.74426636,356.72517916)
\curveto(468.10425866,356.50517947)(467.41425935,356.39517958)(466.67426636,356.39517916)
\curveto(465.31426145,356.39517958)(464.25426251,356.61517936)(463.49426636,357.05517916)
\curveto(462.75426401,357.49517848)(462.20426456,358.01517796)(461.84426636,358.61517916)
\curveto(461.50426526,359.23517674)(461.30426546,359.86517611)(461.24426636,360.50517916)
\curveto(461.18426558,361.14517483)(461.15426561,361.6751743)(461.15426636,362.09517916)
\lineto(463.67426636,362.09517916)
\curveto(463.67426309,361.61517436)(463.71426305,361.14517483)(463.79426636,360.68517916)
\curveto(463.87426289,360.24517573)(464.02426274,359.84517613)(464.24426636,359.48517916)
\curveto(464.4642623,359.14517683)(464.764262,358.8751771)(465.14426636,358.67517916)
\curveto(465.54426122,358.4751775)(466.05426071,358.3751776)(466.67426636,358.37517916)
\curveto(466.87425989,358.3751776)(467.12425964,358.40517757)(467.42426636,358.46517916)
\curveto(467.72425904,358.52517745)(468.01425875,358.65517732)(468.29426636,358.85517916)
\curveto(468.59425817,359.05517692)(468.84425792,359.32517665)(469.04426636,359.66517916)
\curveto(469.24425752,360.00517597)(469.34425742,360.46517551)(469.34426636,361.04517916)
\curveto(469.34425742,361.58517439)(469.22425754,362.02517395)(468.98426636,362.36517916)
\curveto(468.764258,362.72517325)(468.4642583,363.01517296)(468.08426636,363.23517916)
\curveto(467.72425904,363.4751725)(467.30425946,363.6751723)(466.82426636,363.83517916)
\curveto(466.3642604,363.99517198)(465.89426087,364.15517182)(465.41426636,364.31517916)
\curveto(464.93426183,364.4751715)(464.45426231,364.64517133)(463.97426636,364.82517916)
\curveto(463.49426327,365.02517095)(463.0642637,365.28517069)(462.68426636,365.60517916)
\curveto(462.32426444,365.92517005)(462.02426474,366.34516963)(461.78426636,366.86517916)
\curveto(461.5642652,367.38516859)(461.45426531,368.05516792)(461.45426636,368.87517916)
\curveto(461.45426531,369.61516636)(461.58426518,370.26516571)(461.84426636,370.82517916)
\curveto(462.12426464,371.38516459)(462.49426427,371.84516413)(462.95426636,372.20517916)
\curveto(463.43426333,372.58516339)(463.98426278,372.86516311)(464.60426636,373.04517916)
\curveto(465.22426154,373.22516275)(465.88426088,373.31516266)(466.58426636,373.31517916)
\curveto(467.74425902,373.31516266)(468.65425811,373.13516284)(469.31426636,372.77517916)
\curveto(469.97425679,372.41516356)(470.4642563,371.975164)(470.78426636,371.45517916)
\curveto(471.10425566,370.93516504)(471.29425547,370.3751656)(471.35426636,369.77517916)
\curveto(471.43425533,369.19516678)(471.47425529,368.66516731)(471.47426636,368.18517916)
\lineto(468.95426636,368.18517916)
}
}
{
\newrgbcolor{curcolor}{0 0 0}
\pscustom[linestyle=none,fillstyle=solid,fillcolor=curcolor]
{
\newpath
\moveto(476.34176636,364.55517916)
\curveto(476.34176261,363.93517204)(476.3517626,363.26517271)(476.37176636,362.54517916)
\curveto(476.41176254,361.82517415)(476.53176242,361.15517482)(476.73176636,360.53517916)
\curveto(476.93176202,359.91517606)(477.23176172,359.39517658)(477.63176636,358.97517916)
\curveto(478.0517609,358.5751774)(478.6517603,358.3751776)(479.43176636,358.37517916)
\curveto(480.03175892,358.3751776)(480.51175844,358.50517747)(480.87176636,358.76517916)
\curveto(481.23175772,359.04517693)(481.50175745,359.3751766)(481.68176636,359.75517916)
\curveto(481.88175707,360.15517582)(482.01175694,360.56517541)(482.07176636,360.98517916)
\curveto(482.13175682,361.42517455)(482.16175679,361.79517418)(482.16176636,362.09517916)
\lineto(484.68176636,362.09517916)
\curveto(484.68175427,361.6751743)(484.60175435,361.13517484)(484.44176636,360.47517916)
\curveto(484.30175465,359.83517614)(484.03175492,359.20517677)(483.63176636,358.58517916)
\curveto(483.23175572,357.98517799)(482.68175627,357.46517851)(481.98176636,357.02517916)
\curveto(481.28175767,356.60517937)(480.38175857,356.39517958)(479.28176636,356.39517916)
\curveto(477.30176165,356.39517958)(475.87176308,357.0751789)(474.99176636,358.43517916)
\curveto(474.13176482,359.79517618)(473.70176525,361.86517411)(473.70176636,364.64517916)
\curveto(473.70176525,365.64517033)(473.76176519,366.65516932)(473.88176636,367.67517916)
\curveto(474.02176493,368.71516726)(474.29176466,369.64516633)(474.69176636,370.46517916)
\curveto(475.11176384,371.30516467)(475.69176326,371.98516399)(476.43176636,372.50517916)
\curveto(477.19176176,373.04516293)(478.19176076,373.31516266)(479.43176636,373.31517916)
\curveto(480.6517583,373.31516266)(481.62175733,373.0751629)(482.34176636,372.59517916)
\curveto(483.06175589,372.11516386)(483.60175535,371.49516448)(483.96176636,370.73517916)
\curveto(484.32175463,369.99516598)(484.5517544,369.16516681)(484.65176636,368.24517916)
\curveto(484.7517542,367.32516865)(484.80175415,366.43516954)(484.80176636,365.57517916)
\lineto(484.80176636,364.55517916)
\lineto(476.34176636,364.55517916)
\moveto(482.16176636,366.53517916)
\lineto(482.16176636,367.40517916)
\curveto(482.16175679,367.84516813)(482.12175683,368.29516768)(482.04176636,368.75517916)
\curveto(481.96175699,369.23516674)(481.81175714,369.66516631)(481.59176636,370.04517916)
\curveto(481.39175756,370.42516555)(481.11175784,370.73516524)(480.75176636,370.97517916)
\curveto(480.39175856,371.21516476)(479.93175902,371.33516464)(479.37176636,371.33517916)
\curveto(478.71176024,371.33516464)(478.18176077,371.15516482)(477.78176636,370.79517916)
\curveto(477.40176155,370.45516552)(477.11176184,370.05516592)(476.91176636,369.59517916)
\curveto(476.71176224,369.13516684)(476.58176237,368.66516731)(476.52176636,368.18517916)
\curveto(476.46176249,367.72516825)(476.43176252,367.3751686)(476.43176636,367.13517916)
\lineto(476.43176636,366.53517916)
\lineto(482.16176636,366.53517916)
}
}
{
\newrgbcolor{curcolor}{0 0 0}
\pscustom[linestyle=none,fillstyle=solid,fillcolor=curcolor]
{
\newpath
\moveto(487.54254761,372.89517916)
\lineto(490.06254761,372.89517916)
\lineto(490.06254761,370.49517916)
\lineto(490.12254761,370.49517916)
\curveto(490.3025432,370.8751651)(490.49254301,371.23516474)(490.69254761,371.57517916)
\curveto(490.91254259,371.91516406)(491.16254234,372.21516376)(491.44254761,372.47517916)
\curveto(491.72254178,372.73516324)(492.03254147,372.93516304)(492.37254761,373.07517916)
\curveto(492.73254077,373.23516274)(493.14254036,373.31516266)(493.60254761,373.31517916)
\curveto(494.1025394,373.31516266)(494.47253903,373.25516272)(494.71254761,373.13517916)
\lineto(494.71254761,370.67517916)
\curveto(494.59253891,370.69516528)(494.43253907,370.71516526)(494.23254761,370.73517916)
\curveto(494.05253945,370.7751652)(493.76253974,370.79516518)(493.36254761,370.79517916)
\curveto(493.04254046,370.79516518)(492.69254081,370.71516526)(492.31254761,370.55517916)
\curveto(491.93254157,370.41516556)(491.57254193,370.18516579)(491.23254761,369.86517916)
\curveto(490.91254259,369.56516641)(490.63254287,369.1751668)(490.39254761,368.69517916)
\curveto(490.17254333,368.21516776)(490.06254344,367.64516833)(490.06254761,366.98517916)
\lineto(490.06254761,356.81517916)
\lineto(487.54254761,356.81517916)
\lineto(487.54254761,372.89517916)
}
}
{
\newrgbcolor{curcolor}{0 0 0}
\pscustom[linewidth=2,linecolor=curcolor]
{
\newpath
\moveto(421.2755127,298.03887301)
\lineto(661.3500061,298.03887301)
\lineto(661.3500061,249.88823824)
\lineto(421.2755127,249.88823824)
\closepath
}
}
{
\newrgbcolor{curcolor}{0 0 0}
\pscustom[linestyle=none,fillstyle=solid,fillcolor=curcolor]
{
\newpath
\moveto(489.49665283,269.49356539)
\curveto(489.49664059,268.29356557)(489.35664073,267.28356658)(489.07665283,266.46356539)
\curveto(488.81664127,265.62356824)(488.44664164,264.94356892)(487.96665283,264.42356539)
\curveto(487.4866426,263.90356996)(486.90664318,263.53357033)(486.22665283,263.31356539)
\curveto(485.56664452,263.07357079)(484.83664525,262.95357091)(484.03665283,262.95356539)
\curveto(483.79664629,262.95357091)(483.39664669,262.97357089)(482.83665283,263.01356539)
\curveto(482.29664779,263.05357081)(481.73664835,263.20357066)(481.15665283,263.46356539)
\curveto(480.59664949,263.70357016)(480.07665001,264.10356976)(479.59665283,264.66356539)
\curveto(479.13665095,265.20356866)(478.85665123,265.97356789)(478.75665283,266.97356539)
\lineto(481.27665283,266.97356539)
\curveto(481.33664875,266.23356763)(481.61664847,265.71356815)(482.11665283,265.41356539)
\curveto(482.61664747,265.09356877)(483.19664689,264.93356893)(483.85665283,264.93356539)
\curveto(484.65664543,264.93356893)(485.26664482,265.07356879)(485.68665283,265.35356539)
\curveto(486.12664396,265.61356825)(486.43664365,265.93356793)(486.61665283,266.31356539)
\curveto(486.81664327,266.69356717)(486.92664316,267.09356677)(486.94665283,267.51356539)
\curveto(486.96664312,267.91356595)(486.97664311,268.25356561)(486.97665283,268.53356539)
\lineto(486.97665283,270.63356539)
\lineto(486.91665283,270.63356539)
\curveto(486.63664345,269.95356391)(486.15664393,269.42356444)(485.47665283,269.04356539)
\curveto(484.81664527,268.6635652)(484.086646,268.47356539)(483.28665283,268.47356539)
\curveto(482.50664758,268.47356539)(481.83664825,268.62356524)(481.27665283,268.92356539)
\curveto(480.73664935,269.22356464)(480.27664981,269.61356425)(479.89665283,270.09356539)
\curveto(479.53665055,270.57356329)(479.24665084,271.10356276)(479.02665283,271.68356539)
\curveto(478.82665126,272.28356158)(478.66665142,272.88356098)(478.54665283,273.48356539)
\curveto(478.44665164,274.08355978)(478.37665171,274.64355922)(478.33665283,275.16356539)
\curveto(478.31665177,275.70355816)(478.30665178,276.15355771)(478.30665283,276.51356539)
\curveto(478.30665178,277.59355627)(478.3866517,278.64355522)(478.54665283,279.66356539)
\curveto(478.70665138,280.68355318)(478.9866511,281.58355228)(479.38665283,282.36356539)
\curveto(479.80665028,283.1635507)(480.34664974,283.79355007)(481.00665283,284.25356539)
\curveto(481.6866484,284.73354913)(482.53664755,284.97354889)(483.55665283,284.97356539)
\curveto(483.99664609,284.97354889)(484.40664568,284.90354896)(484.78665283,284.76356539)
\curveto(485.1866449,284.62354924)(485.53664455,284.44354942)(485.83665283,284.22356539)
\curveto(486.13664395,284.00354986)(486.3866437,283.74355012)(486.58665283,283.44356539)
\curveto(486.80664328,283.14355072)(486.95664313,282.83355103)(487.03665283,282.51356539)
\lineto(487.09665283,282.51356539)
\lineto(487.09665283,284.55356539)
\lineto(489.49665283,284.55356539)
\lineto(489.49665283,269.49356539)
\moveto(483.88665283,282.81356539)
\curveto(483.30664678,282.81355105)(482.82664726,282.6635512)(482.44665283,282.36356539)
\curveto(482.06664802,282.0635518)(481.76664832,281.64355222)(481.54665283,281.10356539)
\curveto(481.32664876,280.5635533)(481.16664892,279.90355396)(481.06665283,279.12356539)
\curveto(480.9866491,278.34355552)(480.94664914,277.47355639)(480.94665283,276.51356539)
\curveto(480.94664914,275.85355801)(480.97664911,275.17355869)(481.03665283,274.47356539)
\curveto(481.11664897,273.79356007)(481.25664883,273.1635607)(481.45665283,272.58356539)
\curveto(481.65664843,272.02356184)(481.94664814,271.55356231)(482.32665283,271.17356539)
\curveto(482.70664738,270.81356305)(483.21664687,270.63356323)(483.85665283,270.63356539)
\curveto(484.53664555,270.63356323)(485.086645,270.78356308)(485.50665283,271.08356539)
\curveto(485.94664414,271.38356248)(486.2866438,271.79356207)(486.52665283,272.31356539)
\curveto(486.76664332,272.85356101)(486.92664316,273.48356038)(487.00665283,274.20356539)
\curveto(487.086643,274.92355894)(487.12664296,275.69355817)(487.12665283,276.51356539)
\curveto(487.12664296,277.29355657)(487.086643,278.05355581)(487.00665283,278.79356539)
\curveto(486.92664316,279.55355431)(486.76664332,280.23355363)(486.52665283,280.83356539)
\curveto(486.2866438,281.43355243)(485.95664413,281.91355195)(485.53665283,282.27356539)
\curveto(485.11664497,282.63355123)(484.56664552,282.81355105)(483.88665283,282.81356539)
}
}
{
\newrgbcolor{curcolor}{0 0 0}
\pscustom[linestyle=none,fillstyle=solid,fillcolor=curcolor]
{
\newpath
\moveto(492.79337158,284.55356539)
\lineto(495.31337158,284.55356539)
\lineto(495.31337158,282.15356539)
\lineto(495.37337158,282.15356539)
\curveto(495.55336717,282.53355133)(495.74336698,282.89355097)(495.94337158,283.23356539)
\curveto(496.16336656,283.57355029)(496.41336631,283.87354999)(496.69337158,284.13356539)
\curveto(496.97336575,284.39354947)(497.28336544,284.59354927)(497.62337158,284.73356539)
\curveto(497.98336474,284.89354897)(498.39336433,284.97354889)(498.85337158,284.97356539)
\curveto(499.35336337,284.97354889)(499.723363,284.91354895)(499.96337158,284.79356539)
\lineto(499.96337158,282.33356539)
\curveto(499.84336288,282.35355151)(499.68336304,282.37355149)(499.48337158,282.39356539)
\curveto(499.30336342,282.43355143)(499.01336371,282.45355141)(498.61337158,282.45356539)
\curveto(498.29336443,282.45355141)(497.94336478,282.37355149)(497.56337158,282.21356539)
\curveto(497.18336554,282.07355179)(496.8233659,281.84355202)(496.48337158,281.52356539)
\curveto(496.16336656,281.22355264)(495.88336684,280.83355303)(495.64337158,280.35356539)
\curveto(495.4233673,279.87355399)(495.31336741,279.30355456)(495.31337158,278.64356539)
\lineto(495.31337158,268.47356539)
\lineto(492.79337158,268.47356539)
\lineto(492.79337158,284.55356539)
}
}
{
\newrgbcolor{curcolor}{0 0 0}
\pscustom[linestyle=none,fillstyle=solid,fillcolor=curcolor]
{
\newpath
\moveto(500.91962158,276.51356539)
\curveto(500.91962071,277.65355621)(500.99962063,278.73355513)(501.15962158,279.75356539)
\curveto(501.33962029,280.77355309)(501.63961999,281.6635522)(502.05962158,282.42356539)
\curveto(502.49961913,283.20355066)(503.08961854,283.82355004)(503.82962158,284.28356539)
\curveto(504.58961704,284.74354912)(505.54961608,284.97354889)(506.70962158,284.97356539)
\curveto(507.86961376,284.97354889)(508.81961281,284.74354912)(509.55962158,284.28356539)
\curveto(510.31961131,283.82355004)(510.90961072,283.20355066)(511.32962158,282.42356539)
\curveto(511.76960986,281.6635522)(512.06960956,280.77355309)(512.22962158,279.75356539)
\curveto(512.40960922,278.73355513)(512.49960913,277.65355621)(512.49962158,276.51356539)
\curveto(512.49960913,275.37355849)(512.40960922,274.29355957)(512.22962158,273.27356539)
\curveto(512.06960956,272.25356161)(511.76960986,271.35356251)(511.32962158,270.57356539)
\curveto(510.88961074,269.81356405)(510.29961133,269.20356466)(509.55962158,268.74356539)
\curveto(508.81961281,268.28356558)(507.86961376,268.05356581)(506.70962158,268.05356539)
\curveto(505.54961608,268.05356581)(504.58961704,268.28356558)(503.82962158,268.74356539)
\curveto(503.08961854,269.20356466)(502.49961913,269.81356405)(502.05962158,270.57356539)
\curveto(501.63961999,271.35356251)(501.33962029,272.25356161)(501.15962158,273.27356539)
\curveto(500.99962063,274.29355957)(500.91962071,275.37355849)(500.91962158,276.51356539)
\moveto(506.64962158,270.03356539)
\curveto(507.30961432,270.03356383)(507.84961378,270.20356366)(508.26962158,270.54356539)
\curveto(508.68961294,270.90356296)(509.01961261,271.37356249)(509.25962158,271.95356539)
\curveto(509.49961213,272.55356131)(509.65961197,273.24356062)(509.73962158,274.02356539)
\curveto(509.81961181,274.80355906)(509.85961177,275.63355823)(509.85962158,276.51356539)
\curveto(509.85961177,277.37355649)(509.81961181,278.19355567)(509.73962158,278.97356539)
\curveto(509.65961197,279.77355409)(509.49961213,280.4635534)(509.25962158,281.04356539)
\curveto(509.03961259,281.64355222)(508.71961291,282.11355175)(508.29962158,282.45356539)
\curveto(507.87961375,282.81355105)(507.3296143,282.99355087)(506.64962158,282.99356539)
\curveto(506.00961562,282.99355087)(505.48961614,282.81355105)(505.08962158,282.45356539)
\curveto(504.68961694,282.11355175)(504.36961726,281.64355222)(504.12962158,281.04356539)
\curveto(503.90961772,280.4635534)(503.75961787,279.77355409)(503.67962158,278.97356539)
\curveto(503.59961803,278.19355567)(503.55961807,277.37355649)(503.55962158,276.51356539)
\curveto(503.55961807,275.63355823)(503.59961803,274.80355906)(503.67962158,274.02356539)
\curveto(503.75961787,273.24356062)(503.90961772,272.55356131)(504.12962158,271.95356539)
\curveto(504.34961728,271.37356249)(504.65961697,270.90356296)(505.05962158,270.54356539)
\curveto(505.47961615,270.20356366)(506.00961562,270.03356383)(506.64962158,270.03356539)
}
}
{
\newrgbcolor{curcolor}{0 0 0}
\pscustom[linestyle=none,fillstyle=solid,fillcolor=curcolor]
{
\newpath
\moveto(525.59040283,268.47356539)
\lineto(523.19040283,268.47356539)
\lineto(523.19040283,270.36356539)
\lineto(523.13040283,270.36356539)
\curveto(522.79039339,269.62356424)(522.25039393,269.05356481)(521.51040283,268.65356539)
\curveto(520.77039541,268.25356561)(520.01039617,268.05356581)(519.23040283,268.05356539)
\curveto(518.17039801,268.05356581)(517.35039883,268.22356564)(516.77040283,268.56356539)
\curveto(516.21039997,268.92356494)(515.79040039,269.3635645)(515.51040283,269.88356539)
\curveto(515.25040093,270.40356346)(515.10040108,270.95356291)(515.06040283,271.53356539)
\curveto(515.02040116,272.13356173)(515.00040118,272.67356119)(515.00040283,273.15356539)
\lineto(515.00040283,284.55356539)
\lineto(517.52040283,284.55356539)
\lineto(517.52040283,273.45356539)
\curveto(517.52039866,273.15356071)(517.53039865,272.81356105)(517.55040283,272.43356539)
\curveto(517.59039859,272.05356181)(517.69039849,271.69356217)(517.85040283,271.35356539)
\curveto(518.01039817,271.03356283)(518.24039794,270.7635631)(518.54040283,270.54356539)
\curveto(518.86039732,270.32356354)(519.31039687,270.21356365)(519.89040283,270.21356539)
\curveto(520.23039595,270.21356365)(520.5803956,270.27356359)(520.94040283,270.39356539)
\curveto(521.32039486,270.51356335)(521.67039451,270.70356316)(521.99040283,270.96356539)
\curveto(522.31039387,271.22356264)(522.57039361,271.55356231)(522.77040283,271.95356539)
\curveto(522.97039321,272.37356149)(523.07039311,272.87356099)(523.07040283,273.45356539)
\lineto(523.07040283,284.55356539)
\lineto(525.59040283,284.55356539)
\lineto(525.59040283,268.47356539)
}
}
{
\newrgbcolor{curcolor}{0 0 0}
\pscustom[linestyle=none,fillstyle=solid,fillcolor=curcolor]
{
\newpath
\moveto(528.88712158,284.55356539)
\lineto(531.28712158,284.55356539)
\lineto(531.28712158,282.57356539)
\lineto(531.34712158,282.57356539)
\curveto(531.42711739,282.87355099)(531.56711725,283.1635507)(531.76712158,283.44356539)
\curveto(531.98711683,283.74355012)(532.24711657,284.00354986)(532.54712158,284.22356539)
\curveto(532.86711595,284.44354942)(533.2171156,284.62354924)(533.59712158,284.76356539)
\curveto(533.97711484,284.90354896)(534.38711443,284.97354889)(534.82712158,284.97356539)
\curveto(535.78711303,284.97354889)(536.59711222,284.77354909)(537.25712158,284.37356539)
\curveto(537.9171109,283.97354989)(538.45711036,283.41355045)(538.87712158,282.69356539)
\curveto(539.29710952,281.97355189)(539.59710922,281.11355275)(539.77712158,280.11356539)
\curveto(539.97710884,279.11355475)(540.07710874,278.01355585)(540.07712158,276.81356539)
\curveto(540.07710874,275.89355797)(539.99710882,274.92355894)(539.83712158,273.90356539)
\curveto(539.67710914,272.88356098)(539.39710942,271.93356193)(538.99712158,271.05356539)
\curveto(538.6171102,270.19356367)(538.08711073,269.47356439)(537.40712158,268.89356539)
\curveto(536.72711209,268.33356553)(535.86711295,268.05356581)(534.82712158,268.05356539)
\curveto(534.10711471,268.05356581)(533.43711538,268.25356561)(532.81712158,268.65356539)
\curveto(532.19711662,269.05356481)(531.74711707,269.60356426)(531.46712158,270.30356539)
\lineto(531.40712158,270.30356539)
\lineto(531.40712158,263.19356539)
\lineto(528.88712158,263.19356539)
\lineto(528.88712158,284.55356539)
\moveto(531.25712158,276.51356539)
\curveto(531.25711756,275.73355813)(531.29711752,274.9635589)(531.37712158,274.20356539)
\curveto(531.45711736,273.4635604)(531.6171172,272.79356107)(531.85712158,272.19356539)
\curveto(532.09711672,271.59356227)(532.42711639,271.11356275)(532.84712158,270.75356539)
\curveto(533.26711555,270.39356347)(533.817115,270.21356365)(534.49712158,270.21356539)
\curveto(535.07711374,270.21356365)(535.55711326,270.3635635)(535.93712158,270.66356539)
\curveto(536.3171125,270.9635629)(536.6171122,271.39356247)(536.83712158,271.95356539)
\curveto(537.05711176,272.51356135)(537.20711161,273.20356066)(537.28712158,274.02356539)
\curveto(537.38711143,274.84355902)(537.43711138,275.77355809)(537.43712158,276.81356539)
\curveto(537.43711138,277.69355617)(537.38711143,278.50355536)(537.28712158,279.24356539)
\curveto(537.20711161,279.98355388)(537.05711176,280.61355325)(536.83712158,281.13356539)
\curveto(536.6171122,281.67355219)(536.3171125,282.08355178)(535.93712158,282.36356539)
\curveto(535.55711326,282.6635512)(535.07711374,282.81355105)(534.49712158,282.81356539)
\curveto(533.79711502,282.81355105)(533.23711558,282.65355121)(532.81712158,282.33356539)
\curveto(532.39711642,282.03355183)(532.06711675,281.60355226)(531.82712158,281.04356539)
\curveto(531.60711721,280.48355338)(531.45711736,279.81355405)(531.37712158,279.03356539)
\curveto(531.29711752,278.27355559)(531.25711756,277.43355643)(531.25712158,276.51356539)
}
}
{
\newrgbcolor{curcolor}{0 0 0}
\pscustom[linestyle=none,fillstyle=solid,fillcolor=curcolor]
{
\newpath
\moveto(541.12384033,266.22356539)
\lineto(556.12384033,266.22356539)
\lineto(556.12384033,264.72356539)
\lineto(541.12384033,264.72356539)
\lineto(541.12384033,266.22356539)
}
}
{
\newrgbcolor{curcolor}{0 0 0}
\pscustom[linestyle=none,fillstyle=solid,fillcolor=curcolor]
{
\newpath
\moveto(568.36384033,268.47356539)
\lineto(565.96384033,268.47356539)
\lineto(565.96384033,270.36356539)
\lineto(565.90384033,270.36356539)
\curveto(565.56383089,269.62356424)(565.02383143,269.05356481)(564.28384033,268.65356539)
\curveto(563.54383291,268.25356561)(562.78383367,268.05356581)(562.00384033,268.05356539)
\curveto(560.94383551,268.05356581)(560.12383633,268.22356564)(559.54384033,268.56356539)
\curveto(558.98383747,268.92356494)(558.56383789,269.3635645)(558.28384033,269.88356539)
\curveto(558.02383843,270.40356346)(557.87383858,270.95356291)(557.83384033,271.53356539)
\curveto(557.79383866,272.13356173)(557.77383868,272.67356119)(557.77384033,273.15356539)
\lineto(557.77384033,284.55356539)
\lineto(560.29384033,284.55356539)
\lineto(560.29384033,273.45356539)
\curveto(560.29383616,273.15356071)(560.30383615,272.81356105)(560.32384033,272.43356539)
\curveto(560.36383609,272.05356181)(560.46383599,271.69356217)(560.62384033,271.35356539)
\curveto(560.78383567,271.03356283)(561.01383544,270.7635631)(561.31384033,270.54356539)
\curveto(561.63383482,270.32356354)(562.08383437,270.21356365)(562.66384033,270.21356539)
\curveto(563.00383345,270.21356365)(563.3538331,270.27356359)(563.71384033,270.39356539)
\curveto(564.09383236,270.51356335)(564.44383201,270.70356316)(564.76384033,270.96356539)
\curveto(565.08383137,271.22356264)(565.34383111,271.55356231)(565.54384033,271.95356539)
\curveto(565.74383071,272.37356149)(565.84383061,272.87356099)(565.84384033,273.45356539)
\lineto(565.84384033,284.55356539)
\lineto(568.36384033,284.55356539)
\lineto(568.36384033,268.47356539)
}
}
{
\newrgbcolor{curcolor}{0 0 0}
\pscustom[linestyle=none,fillstyle=solid,fillcolor=curcolor]
{
\newpath
\moveto(578.56055908,279.84356539)
\curveto(578.56055053,280.863553)(578.3905507,281.64355222)(578.05055908,282.18356539)
\curveto(577.73055136,282.72355114)(577.11055198,282.99355087)(576.19055908,282.99356539)
\curveto(575.9905531,282.99355087)(575.74055335,282.9635509)(575.44055908,282.90356539)
\curveto(575.14055395,282.863551)(574.85055424,282.7635511)(574.57055908,282.60356539)
\curveto(574.31055478,282.44355142)(574.08055501,282.19355167)(573.88055908,281.85356539)
\curveto(573.68055541,281.53355233)(573.58055551,281.09355277)(573.58055908,280.53356539)
\curveto(573.58055551,280.05355381)(573.6905554,279.6635542)(573.91055908,279.36356539)
\curveto(574.15055494,279.0635548)(574.45055464,278.80355506)(574.81055908,278.58356539)
\curveto(575.1905539,278.38355548)(575.61055348,278.21355565)(576.07055908,278.07356539)
\curveto(576.55055254,277.93355593)(577.04055205,277.78355608)(577.54055908,277.62356539)
\curveto(578.02055107,277.4635564)(578.4905506,277.28355658)(578.95055908,277.08356539)
\curveto(579.43054966,276.90355696)(579.85054924,276.64355722)(580.21055908,276.30356539)
\curveto(580.5905485,275.98355788)(580.8905482,275.5635583)(581.11055908,275.04356539)
\curveto(581.35054774,274.54355932)(581.47054762,273.89355997)(581.47055908,273.09356539)
\curveto(581.47054762,272.25356161)(581.34054775,271.51356235)(581.08055908,270.87356539)
\curveto(580.82054827,270.25356361)(580.46054863,269.73356413)(580.00055908,269.31356539)
\curveto(579.54054955,268.89356497)(578.9905501,268.58356528)(578.35055908,268.38356539)
\curveto(577.71055138,268.1635657)(577.02055207,268.05356581)(576.28055908,268.05356539)
\curveto(574.92055417,268.05356581)(573.86055523,268.27356559)(573.10055908,268.71356539)
\curveto(572.36055673,269.15356471)(571.81055728,269.67356419)(571.45055908,270.27356539)
\curveto(571.11055798,270.89356297)(570.91055818,271.52356234)(570.85055908,272.16356539)
\curveto(570.7905583,272.80356106)(570.76055833,273.33356053)(570.76055908,273.75356539)
\lineto(573.28055908,273.75356539)
\curveto(573.28055581,273.27356059)(573.32055577,272.80356106)(573.40055908,272.34356539)
\curveto(573.48055561,271.90356196)(573.63055546,271.50356236)(573.85055908,271.14356539)
\curveto(574.07055502,270.80356306)(574.37055472,270.53356333)(574.75055908,270.33356539)
\curveto(575.15055394,270.13356373)(575.66055343,270.03356383)(576.28055908,270.03356539)
\curveto(576.48055261,270.03356383)(576.73055236,270.0635638)(577.03055908,270.12356539)
\curveto(577.33055176,270.18356368)(577.62055147,270.31356355)(577.90055908,270.51356539)
\curveto(578.20055089,270.71356315)(578.45055064,270.98356288)(578.65055908,271.32356539)
\curveto(578.85055024,271.6635622)(578.95055014,272.12356174)(578.95055908,272.70356539)
\curveto(578.95055014,273.24356062)(578.83055026,273.68356018)(578.59055908,274.02356539)
\curveto(578.37055072,274.38355948)(578.07055102,274.67355919)(577.69055908,274.89356539)
\curveto(577.33055176,275.13355873)(576.91055218,275.33355853)(576.43055908,275.49356539)
\curveto(575.97055312,275.65355821)(575.50055359,275.81355805)(575.02055908,275.97356539)
\curveto(574.54055455,276.13355773)(574.06055503,276.30355756)(573.58055908,276.48356539)
\curveto(573.10055599,276.68355718)(572.67055642,276.94355692)(572.29055908,277.26356539)
\curveto(571.93055716,277.58355628)(571.63055746,278.00355586)(571.39055908,278.52356539)
\curveto(571.17055792,279.04355482)(571.06055803,279.71355415)(571.06055908,280.53356539)
\curveto(571.06055803,281.27355259)(571.1905579,281.92355194)(571.45055908,282.48356539)
\curveto(571.73055736,283.04355082)(572.10055699,283.50355036)(572.56055908,283.86356539)
\curveto(573.04055605,284.24354962)(573.5905555,284.52354934)(574.21055908,284.70356539)
\curveto(574.83055426,284.88354898)(575.4905536,284.97354889)(576.19055908,284.97356539)
\curveto(577.35055174,284.97354889)(578.26055083,284.79354907)(578.92055908,284.43356539)
\curveto(579.58054951,284.07354979)(580.07054902,283.63355023)(580.39055908,283.11356539)
\curveto(580.71054838,282.59355127)(580.90054819,282.03355183)(580.96055908,281.43356539)
\curveto(581.04054805,280.85355301)(581.08054801,280.32355354)(581.08055908,279.84356539)
\lineto(578.56055908,279.84356539)
}
}
{
\newrgbcolor{curcolor}{0 0 0}
\pscustom[linestyle=none,fillstyle=solid,fillcolor=curcolor]
{
\newpath
\moveto(585.94805908,276.21356539)
\curveto(585.94805533,275.59355827)(585.95805532,274.92355894)(585.97805908,274.20356539)
\curveto(586.01805526,273.48356038)(586.13805514,272.81356105)(586.33805908,272.19356539)
\curveto(586.53805474,271.57356229)(586.83805444,271.05356281)(587.23805908,270.63356539)
\curveto(587.65805362,270.23356363)(588.25805302,270.03356383)(589.03805908,270.03356539)
\curveto(589.63805164,270.03356383)(590.11805116,270.1635637)(590.47805908,270.42356539)
\curveto(590.83805044,270.70356316)(591.10805017,271.03356283)(591.28805908,271.41356539)
\curveto(591.48804979,271.81356205)(591.61804966,272.22356164)(591.67805908,272.64356539)
\curveto(591.73804954,273.08356078)(591.76804951,273.45356041)(591.76805908,273.75356539)
\lineto(594.28805908,273.75356539)
\curveto(594.28804699,273.33356053)(594.20804707,272.79356107)(594.04805908,272.13356539)
\curveto(593.90804737,271.49356237)(593.63804764,270.863563)(593.23805908,270.24356539)
\curveto(592.83804844,269.64356422)(592.28804899,269.12356474)(591.58805908,268.68356539)
\curveto(590.88805039,268.2635656)(589.98805129,268.05356581)(588.88805908,268.05356539)
\curveto(586.90805437,268.05356581)(585.4780558,268.73356513)(584.59805908,270.09356539)
\curveto(583.73805754,271.45356241)(583.30805797,273.52356034)(583.30805908,276.30356539)
\curveto(583.30805797,277.30355656)(583.36805791,278.31355555)(583.48805908,279.33356539)
\curveto(583.62805765,280.37355349)(583.89805738,281.30355256)(584.29805908,282.12356539)
\curveto(584.71805656,282.9635509)(585.29805598,283.64355022)(586.03805908,284.16356539)
\curveto(586.79805448,284.70354916)(587.79805348,284.97354889)(589.03805908,284.97356539)
\curveto(590.25805102,284.97354889)(591.22805005,284.73354913)(591.94805908,284.25356539)
\curveto(592.66804861,283.77355009)(593.20804807,283.15355071)(593.56805908,282.39356539)
\curveto(593.92804735,281.65355221)(594.15804712,280.82355304)(594.25805908,279.90356539)
\curveto(594.35804692,278.98355488)(594.40804687,278.09355577)(594.40805908,277.23356539)
\lineto(594.40805908,276.21356539)
\lineto(585.94805908,276.21356539)
\moveto(591.76805908,278.19356539)
\lineto(591.76805908,279.06356539)
\curveto(591.76804951,279.50355436)(591.72804955,279.95355391)(591.64805908,280.41356539)
\curveto(591.56804971,280.89355297)(591.41804986,281.32355254)(591.19805908,281.70356539)
\curveto(590.99805028,282.08355178)(590.71805056,282.39355147)(590.35805908,282.63356539)
\curveto(589.99805128,282.87355099)(589.53805174,282.99355087)(588.97805908,282.99356539)
\curveto(588.31805296,282.99355087)(587.78805349,282.81355105)(587.38805908,282.45356539)
\curveto(587.00805427,282.11355175)(586.71805456,281.71355215)(586.51805908,281.25356539)
\curveto(586.31805496,280.79355307)(586.18805509,280.32355354)(586.12805908,279.84356539)
\curveto(586.06805521,279.38355448)(586.03805524,279.03355483)(586.03805908,278.79356539)
\lineto(586.03805908,278.19356539)
\lineto(591.76805908,278.19356539)
}
}
{
\newrgbcolor{curcolor}{0 0 0}
\pscustom[linestyle=none,fillstyle=solid,fillcolor=curcolor]
{
\newpath
\moveto(597.14884033,284.55356539)
\lineto(599.66884033,284.55356539)
\lineto(599.66884033,282.15356539)
\lineto(599.72884033,282.15356539)
\curveto(599.90883592,282.53355133)(600.09883573,282.89355097)(600.29884033,283.23356539)
\curveto(600.51883531,283.57355029)(600.76883506,283.87354999)(601.04884033,284.13356539)
\curveto(601.3288345,284.39354947)(601.63883419,284.59354927)(601.97884033,284.73356539)
\curveto(602.33883349,284.89354897)(602.74883308,284.97354889)(603.20884033,284.97356539)
\curveto(603.70883212,284.97354889)(604.07883175,284.91354895)(604.31884033,284.79356539)
\lineto(604.31884033,282.33356539)
\curveto(604.19883163,282.35355151)(604.03883179,282.37355149)(603.83884033,282.39356539)
\curveto(603.65883217,282.43355143)(603.36883246,282.45355141)(602.96884033,282.45356539)
\curveto(602.64883318,282.45355141)(602.29883353,282.37355149)(601.91884033,282.21356539)
\curveto(601.53883429,282.07355179)(601.17883465,281.84355202)(600.83884033,281.52356539)
\curveto(600.51883531,281.22355264)(600.23883559,280.83355303)(599.99884033,280.35356539)
\curveto(599.77883605,279.87355399)(599.66883616,279.30355456)(599.66884033,278.64356539)
\lineto(599.66884033,268.47356539)
\lineto(597.14884033,268.47356539)
\lineto(597.14884033,284.55356539)
}
}
{
\newrgbcolor{curcolor}{0 0 0}
\pscustom[linewidth=2,linecolor=curcolor]
{
\newpath
\moveto(539.0592041,207.2371152)
\lineto(779.13369751,207.2371152)
\lineto(779.13369751,159.08648043)
\lineto(539.0592041,159.08648043)
\closepath
}
}
{
\newrgbcolor{curcolor}{0 0 0}
\pscustom[linestyle=none,fillstyle=solid,fillcolor=curcolor]
{
\newpath
\moveto(601.83182373,195.41679293)
\lineto(604.35182373,195.41679293)
\lineto(604.35182373,190.73679293)
\lineto(607.14182373,190.73679293)
\lineto(607.14182373,188.75679293)
\lineto(604.35182373,188.75679293)
\lineto(604.35182373,178.43679293)
\curveto(604.35181881,177.79678979)(604.4618187,177.33679025)(604.68182373,177.05679293)
\curveto(604.90181826,176.77679081)(605.34181782,176.63679095)(606.00182373,176.63679293)
\curveto(606.28181688,176.63679095)(606.50181666,176.64679094)(606.66182373,176.66679293)
\curveto(606.82181634,176.6867909)(606.97181619,176.70679088)(607.11182373,176.72679293)
\lineto(607.11182373,174.65679293)
\curveto(606.95181621,174.61679297)(606.70181646,174.57679301)(606.36182373,174.53679293)
\curveto(606.02181714,174.49679309)(605.59181757,174.47679311)(605.07182373,174.47679293)
\curveto(604.41181875,174.47679311)(603.87181929,174.53679305)(603.45182373,174.65679293)
\curveto(603.03182013,174.79679279)(602.70182046,174.99679259)(602.46182373,175.25679293)
\curveto(602.22182094,175.53679205)(602.05182111,175.87679171)(601.95182373,176.27679293)
\curveto(601.87182129,176.67679091)(601.83182133,177.13679045)(601.83182373,177.65679293)
\lineto(601.83182373,188.75679293)
\lineto(599.49182373,188.75679293)
\lineto(599.49182373,190.73679293)
\lineto(601.83182373,190.73679293)
\lineto(601.83182373,195.41679293)
}
}
{
\newrgbcolor{curcolor}{0 0 0}
\pscustom[linestyle=none,fillstyle=solid,fillcolor=curcolor]
{
\newpath
\moveto(610.97479248,182.39679293)
\curveto(610.97478873,181.77678581)(610.98478872,181.10678648)(611.00479248,180.38679293)
\curveto(611.04478866,179.66678792)(611.16478854,178.99678859)(611.36479248,178.37679293)
\curveto(611.56478814,177.75678983)(611.86478784,177.23679035)(612.26479248,176.81679293)
\curveto(612.68478702,176.41679117)(613.28478642,176.21679137)(614.06479248,176.21679293)
\curveto(614.66478504,176.21679137)(615.14478456,176.34679124)(615.50479248,176.60679293)
\curveto(615.86478384,176.8867907)(616.13478357,177.21679037)(616.31479248,177.59679293)
\curveto(616.51478319,177.99678959)(616.64478306,178.40678918)(616.70479248,178.82679293)
\curveto(616.76478294,179.26678832)(616.79478291,179.63678795)(616.79479248,179.93679293)
\lineto(619.31479248,179.93679293)
\curveto(619.31478039,179.51678807)(619.23478047,178.97678861)(619.07479248,178.31679293)
\curveto(618.93478077,177.67678991)(618.66478104,177.04679054)(618.26479248,176.42679293)
\curveto(617.86478184,175.82679176)(617.31478239,175.30679228)(616.61479248,174.86679293)
\curveto(615.91478379,174.44679314)(615.01478469,174.23679335)(613.91479248,174.23679293)
\curveto(611.93478777,174.23679335)(610.5047892,174.91679267)(609.62479248,176.27679293)
\curveto(608.76479094,177.63678995)(608.33479137,179.70678788)(608.33479248,182.48679293)
\curveto(608.33479137,183.4867841)(608.39479131,184.49678309)(608.51479248,185.51679293)
\curveto(608.65479105,186.55678103)(608.92479078,187.4867801)(609.32479248,188.30679293)
\curveto(609.74478996,189.14677844)(610.32478938,189.82677776)(611.06479248,190.34679293)
\curveto(611.82478788,190.8867767)(612.82478688,191.15677643)(614.06479248,191.15679293)
\curveto(615.28478442,191.15677643)(616.25478345,190.91677667)(616.97479248,190.43679293)
\curveto(617.69478201,189.95677763)(618.23478147,189.33677825)(618.59479248,188.57679293)
\curveto(618.95478075,187.83677975)(619.18478052,187.00678058)(619.28479248,186.08679293)
\curveto(619.38478032,185.16678242)(619.43478027,184.27678331)(619.43479248,183.41679293)
\lineto(619.43479248,182.39679293)
\lineto(610.97479248,182.39679293)
\moveto(616.79479248,184.37679293)
\lineto(616.79479248,185.24679293)
\curveto(616.79478291,185.6867819)(616.75478295,186.13678145)(616.67479248,186.59679293)
\curveto(616.59478311,187.07678051)(616.44478326,187.50678008)(616.22479248,187.88679293)
\curveto(616.02478368,188.26677932)(615.74478396,188.57677901)(615.38479248,188.81679293)
\curveto(615.02478468,189.05677853)(614.56478514,189.17677841)(614.00479248,189.17679293)
\curveto(613.34478636,189.17677841)(612.81478689,188.99677859)(612.41479248,188.63679293)
\curveto(612.03478767,188.29677929)(611.74478796,187.89677969)(611.54479248,187.43679293)
\curveto(611.34478836,186.97678061)(611.21478849,186.50678108)(611.15479248,186.02679293)
\curveto(611.09478861,185.56678202)(611.06478864,185.21678237)(611.06479248,184.97679293)
\lineto(611.06479248,184.37679293)
\lineto(616.79479248,184.37679293)
}
}
{
\newrgbcolor{curcolor}{0 0 0}
\pscustom[linestyle=none,fillstyle=solid,fillcolor=curcolor]
{
\newpath
\moveto(629.55557373,183.41679293)
\curveto(629.31556494,183.17678441)(629.00556525,182.97678461)(628.62557373,182.81679293)
\curveto(628.26556599,182.65678493)(627.87556638,182.50678508)(627.45557373,182.36679293)
\curveto(627.0555672,182.24678534)(626.6555676,182.11678547)(626.25557373,181.97679293)
\curveto(625.87556838,181.85678573)(625.54556871,181.70678588)(625.26557373,181.52679293)
\curveto(624.86556939,181.26678632)(624.5555697,180.94678664)(624.33557373,180.56679293)
\curveto(624.13557012,180.20678738)(624.03557022,179.67678791)(624.03557373,178.97679293)
\curveto(624.03557022,178.13678945)(624.19557006,177.46679012)(624.51557373,176.96679293)
\curveto(624.8555694,176.46679112)(625.4555688,176.21679137)(626.31557373,176.21679293)
\curveto(626.73556752,176.21679137)(627.13556712,176.29679129)(627.51557373,176.45679293)
\curveto(627.91556634,176.61679097)(628.26556599,176.83679075)(628.56557373,177.11679293)
\curveto(628.86556539,177.39679019)(629.10556515,177.71678987)(629.28557373,178.07679293)
\curveto(629.46556479,178.45678913)(629.5555647,178.85678873)(629.55557373,179.27679293)
\lineto(629.55557373,183.41679293)
\moveto(621.78557373,185.87679293)
\curveto(621.78557247,187.69677989)(622.20557205,189.02677856)(623.04557373,189.86679293)
\curveto(623.88557037,190.72677686)(625.26556899,191.15677643)(627.18557373,191.15679293)
\curveto(628.40556585,191.15677643)(629.34556491,190.99677659)(630.00557373,190.67679293)
\curveto(630.66556359,190.35677723)(631.14556311,189.95677763)(631.44557373,189.47679293)
\curveto(631.76556249,188.99677859)(631.94556231,188.4867791)(631.98557373,187.94679293)
\curveto(632.04556221,187.42678016)(632.07556218,186.95678063)(632.07557373,186.53679293)
\lineto(632.07557373,177.56679293)
\curveto(632.07556218,177.22679036)(632.10556215,176.92679066)(632.16557373,176.66679293)
\curveto(632.22556203,176.40679118)(632.4555618,176.27679131)(632.85557373,176.27679293)
\curveto(633.01556124,176.27679131)(633.13556112,176.2867913)(633.21557373,176.30679293)
\curveto(633.31556094,176.34679124)(633.39556086,176.3867912)(633.45557373,176.42679293)
\lineto(633.45557373,174.62679293)
\curveto(633.3555609,174.60679298)(633.16556109,174.57679301)(632.88557373,174.53679293)
\curveto(632.60556165,174.49679309)(632.30556195,174.47679311)(631.98557373,174.47679293)
\curveto(631.74556251,174.47679311)(631.49556276,174.4867931)(631.23557373,174.50679293)
\curveto(630.99556326,174.52679306)(630.7555635,174.59679299)(630.51557373,174.71679293)
\curveto(630.29556396,174.85679273)(630.10556415,175.06679252)(629.94557373,175.34679293)
\curveto(629.80556445,175.62679196)(629.72556453,176.02679156)(629.70557373,176.54679293)
\lineto(629.64557373,176.54679293)
\curveto(629.24556501,175.82679176)(628.68556557,175.25679233)(627.96557373,174.83679293)
\curveto(627.26556699,174.43679315)(626.53556772,174.23679335)(625.77557373,174.23679293)
\curveto(624.27556998,174.23679335)(623.16557109,174.65679293)(622.44557373,175.49679293)
\curveto(621.74557251,176.33679125)(621.39557286,177.47679011)(621.39557373,178.91679293)
\curveto(621.39557286,180.05678753)(621.63557262,180.99678659)(622.11557373,181.73679293)
\curveto(622.61557164,182.49678509)(623.38557087,183.03678455)(624.42557373,183.35679293)
\lineto(627.81557373,184.37679293)
\curveto(628.27556598,184.51678307)(628.62556563,184.65678293)(628.86557373,184.79679293)
\curveto(629.12556513,184.95678263)(629.30556495,185.12678246)(629.40557373,185.30679293)
\curveto(629.52556473,185.4867821)(629.59556466,185.69678189)(629.61557373,185.93679293)
\curveto(629.63556462,186.17678141)(629.64556461,186.46678112)(629.64557373,186.80679293)
\curveto(629.64556461,188.3867792)(628.78556547,189.17677841)(627.06557373,189.17679293)
\curveto(626.36556789,189.17677841)(625.82556843,189.03677855)(625.44557373,188.75679293)
\curveto(625.08556917,188.49677909)(624.81556944,188.1867794)(624.63557373,187.82679293)
\curveto(624.47556978,187.46678012)(624.37556988,187.10678048)(624.33557373,186.74679293)
\curveto(624.31556994,186.40678118)(624.30556995,186.16678142)(624.30557373,186.02679293)
\lineto(624.30557373,185.87679293)
\lineto(621.78557373,185.87679293)
}
}
{
\newrgbcolor{curcolor}{0 0 0}
\pscustom[linestyle=none,fillstyle=solid,fillcolor=curcolor]
{
\newpath
\moveto(635.71635498,190.73679293)
\lineto(638.11635498,190.73679293)
\lineto(638.11635498,188.84679293)
\lineto(638.17635498,188.84679293)
\curveto(638.51635029,189.586778)(639.05634975,190.15677743)(639.79635498,190.55679293)
\curveto(640.53634827,190.95677663)(641.29634751,191.15677643)(642.07635498,191.15679293)
\curveto(643.03634577,191.15677643)(643.78634502,190.95677663)(644.32635498,190.55679293)
\curveto(644.88634392,190.17677741)(645.29634351,189.52677806)(645.55635498,188.60679293)
\curveto(645.91634289,189.30677828)(646.42634238,189.90677768)(647.08635498,190.40679293)
\curveto(647.76634104,190.90677668)(648.52634028,191.15677643)(649.36635498,191.15679293)
\curveto(650.42633838,191.15677643)(651.23633757,190.97677661)(651.79635498,190.61679293)
\curveto(652.37633643,190.27677731)(652.79633601,189.84677774)(653.05635498,189.32679293)
\curveto(653.33633547,188.80677878)(653.49633531,188.24677934)(653.53635498,187.64679293)
\curveto(653.57633523,187.06678052)(653.59633521,186.53678105)(653.59635498,186.05679293)
\lineto(653.59635498,174.65679293)
\lineto(651.07635498,174.65679293)
\lineto(651.07635498,185.75679293)
\curveto(651.07633773,186.05678153)(651.05633775,186.39678119)(651.01635498,186.77679293)
\curveto(650.99633781,187.15678043)(650.91633789,187.50678008)(650.77635498,187.82679293)
\curveto(650.63633817,188.16677942)(650.41633839,188.44677914)(650.11635498,188.66679293)
\curveto(649.83633897,188.8867787)(649.43633937,188.99677859)(648.91635498,188.99679293)
\curveto(648.61634019,188.99677859)(648.28634052,188.94677864)(647.92635498,188.84679293)
\curveto(647.58634122,188.74677884)(647.26634154,188.56677902)(646.96635498,188.30679293)
\curveto(646.66634214,188.06677952)(646.41634239,187.73677985)(646.21635498,187.31679293)
\curveto(646.01634279,186.89678069)(645.91634289,186.37678121)(645.91635498,185.75679293)
\lineto(645.91635498,174.65679293)
\lineto(643.39635498,174.65679293)
\lineto(643.39635498,185.75679293)
\curveto(643.39634541,186.05678153)(643.37634543,186.39678119)(643.33635498,186.77679293)
\curveto(643.31634549,187.15678043)(643.23634557,187.50678008)(643.09635498,187.82679293)
\curveto(642.95634585,188.16677942)(642.73634607,188.44677914)(642.43635498,188.66679293)
\curveto(642.15634665,188.8867787)(641.75634705,188.99677859)(641.23635498,188.99679293)
\curveto(640.93634787,188.99677859)(640.6063482,188.94677864)(640.24635498,188.84679293)
\curveto(639.9063489,188.74677884)(639.58634922,188.56677902)(639.28635498,188.30679293)
\curveto(638.98634982,188.06677952)(638.73635007,187.73677985)(638.53635498,187.31679293)
\curveto(638.33635047,186.89678069)(638.23635057,186.37678121)(638.23635498,185.75679293)
\lineto(638.23635498,174.65679293)
\lineto(635.71635498,174.65679293)
\lineto(635.71635498,190.73679293)
}
}
{
\newrgbcolor{curcolor}{0 0 0}
\pscustom[linestyle=none,fillstyle=solid,fillcolor=curcolor]
{
\newpath
\moveto(655.50604248,172.40679293)
\lineto(670.50604248,172.40679293)
\lineto(670.50604248,170.90679293)
\lineto(655.50604248,170.90679293)
\lineto(655.50604248,172.40679293)
}
}
{
\newrgbcolor{curcolor}{0 0 0}
\pscustom[linestyle=none,fillstyle=solid,fillcolor=curcolor]
{
\newpath
\moveto(682.74604248,174.65679293)
\lineto(680.34604248,174.65679293)
\lineto(680.34604248,176.54679293)
\lineto(680.28604248,176.54679293)
\curveto(679.94603304,175.80679178)(679.40603358,175.23679235)(678.66604248,174.83679293)
\curveto(677.92603506,174.43679315)(677.16603582,174.23679335)(676.38604248,174.23679293)
\curveto(675.32603766,174.23679335)(674.50603848,174.40679318)(673.92604248,174.74679293)
\curveto(673.36603962,175.10679248)(672.94604004,175.54679204)(672.66604248,176.06679293)
\curveto(672.40604058,176.586791)(672.25604073,177.13679045)(672.21604248,177.71679293)
\curveto(672.17604081,178.31678927)(672.15604083,178.85678873)(672.15604248,179.33679293)
\lineto(672.15604248,190.73679293)
\lineto(674.67604248,190.73679293)
\lineto(674.67604248,179.63679293)
\curveto(674.67603831,179.33678825)(674.6860383,178.99678859)(674.70604248,178.61679293)
\curveto(674.74603824,178.23678935)(674.84603814,177.87678971)(675.00604248,177.53679293)
\curveto(675.16603782,177.21679037)(675.39603759,176.94679064)(675.69604248,176.72679293)
\curveto(676.01603697,176.50679108)(676.46603652,176.39679119)(677.04604248,176.39679293)
\curveto(677.3860356,176.39679119)(677.73603525,176.45679113)(678.09604248,176.57679293)
\curveto(678.47603451,176.69679089)(678.82603416,176.8867907)(679.14604248,177.14679293)
\curveto(679.46603352,177.40679018)(679.72603326,177.73678985)(679.92604248,178.13679293)
\curveto(680.12603286,178.55678903)(680.22603276,179.05678853)(680.22604248,179.63679293)
\lineto(680.22604248,190.73679293)
\lineto(682.74604248,190.73679293)
\lineto(682.74604248,174.65679293)
}
}
{
\newrgbcolor{curcolor}{0 0 0}
\pscustom[linestyle=none,fillstyle=solid,fillcolor=curcolor]
{
\newpath
\moveto(692.94276123,186.02679293)
\curveto(692.94275268,187.04678054)(692.77275285,187.82677976)(692.43276123,188.36679293)
\curveto(692.11275351,188.90677868)(691.49275413,189.17677841)(690.57276123,189.17679293)
\curveto(690.37275525,189.17677841)(690.1227555,189.14677844)(689.82276123,189.08679293)
\curveto(689.5227561,189.04677854)(689.23275639,188.94677864)(688.95276123,188.78679293)
\curveto(688.69275693,188.62677896)(688.46275716,188.37677921)(688.26276123,188.03679293)
\curveto(688.06275756,187.71677987)(687.96275766,187.27678031)(687.96276123,186.71679293)
\curveto(687.96275766,186.23678135)(688.07275755,185.84678174)(688.29276123,185.54679293)
\curveto(688.53275709,185.24678234)(688.83275679,184.9867826)(689.19276123,184.76679293)
\curveto(689.57275605,184.56678302)(689.99275563,184.39678319)(690.45276123,184.25679293)
\curveto(690.93275469,184.11678347)(691.4227542,183.96678362)(691.92276123,183.80679293)
\curveto(692.40275322,183.64678394)(692.87275275,183.46678412)(693.33276123,183.26679293)
\curveto(693.81275181,183.0867845)(694.23275139,182.82678476)(694.59276123,182.48679293)
\curveto(694.97275065,182.16678542)(695.27275035,181.74678584)(695.49276123,181.22679293)
\curveto(695.73274989,180.72678686)(695.85274977,180.07678751)(695.85276123,179.27679293)
\curveto(695.85274977,178.43678915)(695.7227499,177.69678989)(695.46276123,177.05679293)
\curveto(695.20275042,176.43679115)(694.84275078,175.91679167)(694.38276123,175.49679293)
\curveto(693.9227517,175.07679251)(693.37275225,174.76679282)(692.73276123,174.56679293)
\curveto(692.09275353,174.34679324)(691.40275422,174.23679335)(690.66276123,174.23679293)
\curveto(689.30275632,174.23679335)(688.24275738,174.45679313)(687.48276123,174.89679293)
\curveto(686.74275888,175.33679225)(686.19275943,175.85679173)(685.83276123,176.45679293)
\curveto(685.49276013,177.07679051)(685.29276033,177.70678988)(685.23276123,178.34679293)
\curveto(685.17276045,178.9867886)(685.14276048,179.51678807)(685.14276123,179.93679293)
\lineto(687.66276123,179.93679293)
\curveto(687.66275796,179.45678813)(687.70275792,178.9867886)(687.78276123,178.52679293)
\curveto(687.86275776,178.0867895)(688.01275761,177.6867899)(688.23276123,177.32679293)
\curveto(688.45275717,176.9867906)(688.75275687,176.71679087)(689.13276123,176.51679293)
\curveto(689.53275609,176.31679127)(690.04275558,176.21679137)(690.66276123,176.21679293)
\curveto(690.86275476,176.21679137)(691.11275451,176.24679134)(691.41276123,176.30679293)
\curveto(691.71275391,176.36679122)(692.00275362,176.49679109)(692.28276123,176.69679293)
\curveto(692.58275304,176.89679069)(692.83275279,177.16679042)(693.03276123,177.50679293)
\curveto(693.23275239,177.84678974)(693.33275229,178.30678928)(693.33276123,178.88679293)
\curveto(693.33275229,179.42678816)(693.21275241,179.86678772)(692.97276123,180.20679293)
\curveto(692.75275287,180.56678702)(692.45275317,180.85678673)(692.07276123,181.07679293)
\curveto(691.71275391,181.31678627)(691.29275433,181.51678607)(690.81276123,181.67679293)
\curveto(690.35275527,181.83678575)(689.88275574,181.99678559)(689.40276123,182.15679293)
\curveto(688.9227567,182.31678527)(688.44275718,182.4867851)(687.96276123,182.66679293)
\curveto(687.48275814,182.86678472)(687.05275857,183.12678446)(686.67276123,183.44679293)
\curveto(686.31275931,183.76678382)(686.01275961,184.1867834)(685.77276123,184.70679293)
\curveto(685.55276007,185.22678236)(685.44276018,185.89678169)(685.44276123,186.71679293)
\curveto(685.44276018,187.45678013)(685.57276005,188.10677948)(685.83276123,188.66679293)
\curveto(686.11275951,189.22677836)(686.48275914,189.6867779)(686.94276123,190.04679293)
\curveto(687.4227582,190.42677716)(687.97275765,190.70677688)(688.59276123,190.88679293)
\curveto(689.21275641,191.06677652)(689.87275575,191.15677643)(690.57276123,191.15679293)
\curveto(691.73275389,191.15677643)(692.64275298,190.97677661)(693.30276123,190.61679293)
\curveto(693.96275166,190.25677733)(694.45275117,189.81677777)(694.77276123,189.29679293)
\curveto(695.09275053,188.77677881)(695.28275034,188.21677937)(695.34276123,187.61679293)
\curveto(695.4227502,187.03678055)(695.46275016,186.50678108)(695.46276123,186.02679293)
\lineto(692.94276123,186.02679293)
}
}
{
\newrgbcolor{curcolor}{0 0 0}
\pscustom[linestyle=none,fillstyle=solid,fillcolor=curcolor]
{
\newpath
\moveto(700.33026123,182.39679293)
\curveto(700.33025748,181.77678581)(700.34025747,181.10678648)(700.36026123,180.38679293)
\curveto(700.40025741,179.66678792)(700.52025729,178.99678859)(700.72026123,178.37679293)
\curveto(700.92025689,177.75678983)(701.22025659,177.23679035)(701.62026123,176.81679293)
\curveto(702.04025577,176.41679117)(702.64025517,176.21679137)(703.42026123,176.21679293)
\curveto(704.02025379,176.21679137)(704.50025331,176.34679124)(704.86026123,176.60679293)
\curveto(705.22025259,176.8867907)(705.49025232,177.21679037)(705.67026123,177.59679293)
\curveto(705.87025194,177.99678959)(706.00025181,178.40678918)(706.06026123,178.82679293)
\curveto(706.12025169,179.26678832)(706.15025166,179.63678795)(706.15026123,179.93679293)
\lineto(708.67026123,179.93679293)
\curveto(708.67024914,179.51678807)(708.59024922,178.97678861)(708.43026123,178.31679293)
\curveto(708.29024952,177.67678991)(708.02024979,177.04679054)(707.62026123,176.42679293)
\curveto(707.22025059,175.82679176)(706.67025114,175.30679228)(705.97026123,174.86679293)
\curveto(705.27025254,174.44679314)(704.37025344,174.23679335)(703.27026123,174.23679293)
\curveto(701.29025652,174.23679335)(699.86025795,174.91679267)(698.98026123,176.27679293)
\curveto(698.12025969,177.63678995)(697.69026012,179.70678788)(697.69026123,182.48679293)
\curveto(697.69026012,183.4867841)(697.75026006,184.49678309)(697.87026123,185.51679293)
\curveto(698.0102598,186.55678103)(698.28025953,187.4867801)(698.68026123,188.30679293)
\curveto(699.10025871,189.14677844)(699.68025813,189.82677776)(700.42026123,190.34679293)
\curveto(701.18025663,190.8867767)(702.18025563,191.15677643)(703.42026123,191.15679293)
\curveto(704.64025317,191.15677643)(705.6102522,190.91677667)(706.33026123,190.43679293)
\curveto(707.05025076,189.95677763)(707.59025022,189.33677825)(707.95026123,188.57679293)
\curveto(708.3102495,187.83677975)(708.54024927,187.00678058)(708.64026123,186.08679293)
\curveto(708.74024907,185.16678242)(708.79024902,184.27678331)(708.79026123,183.41679293)
\lineto(708.79026123,182.39679293)
\lineto(700.33026123,182.39679293)
\moveto(706.15026123,184.37679293)
\lineto(706.15026123,185.24679293)
\curveto(706.15025166,185.6867819)(706.1102517,186.13678145)(706.03026123,186.59679293)
\curveto(705.95025186,187.07678051)(705.80025201,187.50678008)(705.58026123,187.88679293)
\curveto(705.38025243,188.26677932)(705.10025271,188.57677901)(704.74026123,188.81679293)
\curveto(704.38025343,189.05677853)(703.92025389,189.17677841)(703.36026123,189.17679293)
\curveto(702.70025511,189.17677841)(702.17025564,188.99677859)(701.77026123,188.63679293)
\curveto(701.39025642,188.29677929)(701.10025671,187.89677969)(700.90026123,187.43679293)
\curveto(700.70025711,186.97678061)(700.57025724,186.50678108)(700.51026123,186.02679293)
\curveto(700.45025736,185.56678202)(700.42025739,185.21678237)(700.42026123,184.97679293)
\lineto(700.42026123,184.37679293)
\lineto(706.15026123,184.37679293)
}
}
{
\newrgbcolor{curcolor}{0 0 0}
\pscustom[linestyle=none,fillstyle=solid,fillcolor=curcolor]
{
\newpath
\moveto(711.53104248,190.73679293)
\lineto(714.05104248,190.73679293)
\lineto(714.05104248,188.33679293)
\lineto(714.11104248,188.33679293)
\curveto(714.29103807,188.71677887)(714.48103788,189.07677851)(714.68104248,189.41679293)
\curveto(714.90103746,189.75677783)(715.15103721,190.05677753)(715.43104248,190.31679293)
\curveto(715.71103665,190.57677701)(716.02103634,190.77677681)(716.36104248,190.91679293)
\curveto(716.72103564,191.07677651)(717.13103523,191.15677643)(717.59104248,191.15679293)
\curveto(718.09103427,191.15677643)(718.4610339,191.09677649)(718.70104248,190.97679293)
\lineto(718.70104248,188.51679293)
\curveto(718.58103378,188.53677905)(718.42103394,188.55677903)(718.22104248,188.57679293)
\curveto(718.04103432,188.61677897)(717.75103461,188.63677895)(717.35104248,188.63679293)
\curveto(717.03103533,188.63677895)(716.68103568,188.55677903)(716.30104248,188.39679293)
\curveto(715.92103644,188.25677933)(715.5610368,188.02677956)(715.22104248,187.70679293)
\curveto(714.90103746,187.40678018)(714.62103774,187.01678057)(714.38104248,186.53679293)
\curveto(714.1610382,186.05678153)(714.05103831,185.4867821)(714.05104248,184.82679293)
\lineto(714.05104248,174.65679293)
\lineto(711.53104248,174.65679293)
\lineto(711.53104248,190.73679293)
}
}
{
\newrgbcolor{curcolor}{0 0 0}
\pscustom[linewidth=2,linecolor=curcolor]
{
\newpath
\moveto(19.35305214,570.44393282)
\lineto(259.42754555,570.44393282)
\lineto(259.42754555,522.29329806)
\lineto(19.35305214,522.29329806)
\closepath
}
}
{
\newrgbcolor{curcolor}{0 0 0}
\pscustom[linestyle=none,fillstyle=solid,fillcolor=curcolor]
{
\newpath
\moveto(54.97217255,549.48863863)
\curveto(54.97216325,549.86862706)(54.9221633,550.25862667)(54.82217255,550.65863863)
\curveto(54.74216348,551.05862587)(54.60216362,551.41862551)(54.40217255,551.73863863)
\curveto(54.222164,552.05862487)(53.96216426,552.31862461)(53.62217255,552.51863863)
\curveto(53.30216492,552.71862421)(52.90216532,552.81862411)(52.42217255,552.81863863)
\curveto(52.0221662,552.81862411)(51.63216659,552.74862418)(51.25217255,552.60863863)
\curveto(50.89216733,552.46862446)(50.56216766,552.15862477)(50.26217255,551.67863863)
\curveto(49.96216826,551.21862571)(49.7221685,550.54862638)(49.54217255,549.66863863)
\curveto(49.36216886,548.78862814)(49.27216895,547.61862931)(49.27217255,546.15863863)
\curveto(49.27216895,545.63863129)(49.28216894,545.01863191)(49.30217255,544.29863863)
\curveto(49.34216888,543.57863335)(49.46216876,542.88863404)(49.66217255,542.22863863)
\curveto(49.86216836,541.56863536)(50.16216806,541.00863592)(50.56217255,540.54863863)
\curveto(50.98216724,540.08863684)(51.57216665,539.85863707)(52.33217255,539.85863863)
\curveto(52.87216535,539.85863707)(53.31216491,539.98863694)(53.65217255,540.24863863)
\curveto(53.99216423,540.50863642)(54.26216396,540.8286361)(54.46217255,541.20863863)
\curveto(54.66216356,541.60863532)(54.79216343,542.03863489)(54.85217255,542.49863863)
\curveto(54.93216329,542.97863395)(54.97216325,543.43863349)(54.97217255,543.87863863)
\lineto(57.49217255,543.87863863)
\curveto(57.49216073,543.23863369)(57.40216082,542.56863436)(57.22217255,541.86863863)
\curveto(57.06216116,541.16863576)(56.77216145,540.51863641)(56.35217255,539.91863863)
\curveto(55.95216227,539.33863759)(55.41216281,538.84863808)(54.73217255,538.44863863)
\curveto(54.05216417,538.06863886)(53.21216501,537.87863905)(52.21217255,537.87863863)
\curveto(50.23216799,537.87863905)(48.80216942,538.55863837)(47.92217255,539.91863863)
\curveto(47.06217116,541.27863565)(46.63217159,543.34863358)(46.63217255,546.12863863)
\curveto(46.63217159,547.1286298)(46.69217153,548.13862879)(46.81217255,549.15863863)
\curveto(46.95217127,550.19862673)(47.222171,551.1286258)(47.62217255,551.94863863)
\curveto(48.04217018,552.78862414)(48.6221696,553.46862346)(49.36217255,553.98863863)
\curveto(50.1221681,554.5286224)(51.1221671,554.79862213)(52.36217255,554.79863863)
\curveto(53.46216476,554.79862213)(54.34216388,554.60862232)(55.00217255,554.22863863)
\curveto(55.68216254,553.84862308)(56.20216202,553.37862355)(56.56217255,552.81863863)
\curveto(56.94216128,552.27862465)(57.19216103,551.69862523)(57.31217255,551.07863863)
\curveto(57.43216079,550.47862645)(57.49216073,549.94862698)(57.49217255,549.48863863)
\lineto(54.97217255,549.48863863)
}
}
{
\newrgbcolor{curcolor}{0 0 0}
\pscustom[linestyle=none,fillstyle=solid,fillcolor=curcolor]
{
\newpath
\moveto(59.31561005,546.33863863)
\curveto(59.31560918,547.47862945)(59.3956091,548.55862837)(59.55561005,549.57863863)
\curveto(59.73560876,550.59862633)(60.03560846,551.48862544)(60.45561005,552.24863863)
\curveto(60.8956076,553.0286239)(61.48560701,553.64862328)(62.22561005,554.10863863)
\curveto(62.98560551,554.56862236)(63.94560455,554.79862213)(65.10561005,554.79863863)
\curveto(66.26560223,554.79862213)(67.21560128,554.56862236)(67.95561005,554.10863863)
\curveto(68.71559978,553.64862328)(69.30559919,553.0286239)(69.72561005,552.24863863)
\curveto(70.16559833,551.48862544)(70.46559803,550.59862633)(70.62561005,549.57863863)
\curveto(70.80559769,548.55862837)(70.8955976,547.47862945)(70.89561005,546.33863863)
\curveto(70.8955976,545.19863173)(70.80559769,544.11863281)(70.62561005,543.09863863)
\curveto(70.46559803,542.07863485)(70.16559833,541.17863575)(69.72561005,540.39863863)
\curveto(69.28559921,539.63863729)(68.6955998,539.0286379)(67.95561005,538.56863863)
\curveto(67.21560128,538.10863882)(66.26560223,537.87863905)(65.10561005,537.87863863)
\curveto(63.94560455,537.87863905)(62.98560551,538.10863882)(62.22561005,538.56863863)
\curveto(61.48560701,539.0286379)(60.8956076,539.63863729)(60.45561005,540.39863863)
\curveto(60.03560846,541.17863575)(59.73560876,542.07863485)(59.55561005,543.09863863)
\curveto(59.3956091,544.11863281)(59.31560918,545.19863173)(59.31561005,546.33863863)
\moveto(65.04561005,539.85863863)
\curveto(65.70560279,539.85863707)(66.24560225,540.0286369)(66.66561005,540.36863863)
\curveto(67.08560141,540.7286362)(67.41560108,541.19863573)(67.65561005,541.77863863)
\curveto(67.8956006,542.37863455)(68.05560044,543.06863386)(68.13561005,543.84863863)
\curveto(68.21560028,544.6286323)(68.25560024,545.45863147)(68.25561005,546.33863863)
\curveto(68.25560024,547.19862973)(68.21560028,548.01862891)(68.13561005,548.79863863)
\curveto(68.05560044,549.59862733)(67.8956006,550.28862664)(67.65561005,550.86863863)
\curveto(67.43560106,551.46862546)(67.11560138,551.93862499)(66.69561005,552.27863863)
\curveto(66.27560222,552.63862429)(65.72560277,552.81862411)(65.04561005,552.81863863)
\curveto(64.40560409,552.81862411)(63.88560461,552.63862429)(63.48561005,552.27863863)
\curveto(63.08560541,551.93862499)(62.76560573,551.46862546)(62.52561005,550.86863863)
\curveto(62.30560619,550.28862664)(62.15560634,549.59862733)(62.07561005,548.79863863)
\curveto(61.9956065,548.01862891)(61.95560654,547.19862973)(61.95561005,546.33863863)
\curveto(61.95560654,545.45863147)(61.9956065,544.6286323)(62.07561005,543.84863863)
\curveto(62.15560634,543.06863386)(62.30560619,542.37863455)(62.52561005,541.77863863)
\curveto(62.74560575,541.19863573)(63.05560544,540.7286362)(63.45561005,540.36863863)
\curveto(63.87560462,540.0286369)(64.40560409,539.85863707)(65.04561005,539.85863863)
}
}
{
\newrgbcolor{curcolor}{0 0 0}
\pscustom[linestyle=none,fillstyle=solid,fillcolor=curcolor]
{
\newpath
\moveto(73.6363913,554.37863863)
\lineto(76.0363913,554.37863863)
\lineto(76.0363913,552.48863863)
\lineto(76.0963913,552.48863863)
\curveto(76.43638661,553.2286237)(76.97638607,553.79862313)(77.7163913,554.19863863)
\curveto(78.45638459,554.59862233)(79.21638383,554.79862213)(79.9963913,554.79863863)
\curveto(80.95638209,554.79862213)(81.70638134,554.59862233)(82.2463913,554.19863863)
\curveto(82.80638024,553.81862311)(83.21637983,553.16862376)(83.4763913,552.24863863)
\curveto(83.83637921,552.94862398)(84.3463787,553.54862338)(85.0063913,554.04863863)
\curveto(85.68637736,554.54862238)(86.4463766,554.79862213)(87.2863913,554.79863863)
\curveto(88.3463747,554.79862213)(89.15637389,554.61862231)(89.7163913,554.25863863)
\curveto(90.29637275,553.91862301)(90.71637233,553.48862344)(90.9763913,552.96863863)
\curveto(91.25637179,552.44862448)(91.41637163,551.88862504)(91.4563913,551.28863863)
\curveto(91.49637155,550.70862622)(91.51637153,550.17862675)(91.5163913,549.69863863)
\lineto(91.5163913,538.29863863)
\lineto(88.9963913,538.29863863)
\lineto(88.9963913,549.39863863)
\curveto(88.99637405,549.69862723)(88.97637407,550.03862689)(88.9363913,550.41863863)
\curveto(88.91637413,550.79862613)(88.83637421,551.14862578)(88.6963913,551.46863863)
\curveto(88.55637449,551.80862512)(88.33637471,552.08862484)(88.0363913,552.30863863)
\curveto(87.75637529,552.5286244)(87.35637569,552.63862429)(86.8363913,552.63863863)
\curveto(86.53637651,552.63862429)(86.20637684,552.58862434)(85.8463913,552.48863863)
\curveto(85.50637754,552.38862454)(85.18637786,552.20862472)(84.8863913,551.94863863)
\curveto(84.58637846,551.70862522)(84.33637871,551.37862555)(84.1363913,550.95863863)
\curveto(83.93637911,550.53862639)(83.83637921,550.01862691)(83.8363913,549.39863863)
\lineto(83.8363913,538.29863863)
\lineto(81.3163913,538.29863863)
\lineto(81.3163913,549.39863863)
\curveto(81.31638173,549.69862723)(81.29638175,550.03862689)(81.2563913,550.41863863)
\curveto(81.23638181,550.79862613)(81.15638189,551.14862578)(81.0163913,551.46863863)
\curveto(80.87638217,551.80862512)(80.65638239,552.08862484)(80.3563913,552.30863863)
\curveto(80.07638297,552.5286244)(79.67638337,552.63862429)(79.1563913,552.63863863)
\curveto(78.85638419,552.63862429)(78.52638452,552.58862434)(78.1663913,552.48863863)
\curveto(77.82638522,552.38862454)(77.50638554,552.20862472)(77.2063913,551.94863863)
\curveto(76.90638614,551.70862522)(76.65638639,551.37862555)(76.4563913,550.95863863)
\curveto(76.25638679,550.53862639)(76.15638689,550.01862691)(76.1563913,549.39863863)
\lineto(76.1563913,538.29863863)
\lineto(73.6363913,538.29863863)
\lineto(73.6363913,554.37863863)
}
}
{
\newrgbcolor{curcolor}{0 0 0}
\pscustom[linestyle=none,fillstyle=solid,fillcolor=curcolor]
{
\newpath
\moveto(95.3160788,554.37863863)
\lineto(97.7160788,554.37863863)
\lineto(97.7160788,552.48863863)
\lineto(97.7760788,552.48863863)
\curveto(98.11607411,553.2286237)(98.65607357,553.79862313)(99.3960788,554.19863863)
\curveto(100.13607209,554.59862233)(100.89607133,554.79862213)(101.6760788,554.79863863)
\curveto(102.63606959,554.79862213)(103.38606884,554.59862233)(103.9260788,554.19863863)
\curveto(104.48606774,553.81862311)(104.89606733,553.16862376)(105.1560788,552.24863863)
\curveto(105.51606671,552.94862398)(106.0260662,553.54862338)(106.6860788,554.04863863)
\curveto(107.36606486,554.54862238)(108.1260641,554.79862213)(108.9660788,554.79863863)
\curveto(110.0260622,554.79862213)(110.83606139,554.61862231)(111.3960788,554.25863863)
\curveto(111.97606025,553.91862301)(112.39605983,553.48862344)(112.6560788,552.96863863)
\curveto(112.93605929,552.44862448)(113.09605913,551.88862504)(113.1360788,551.28863863)
\curveto(113.17605905,550.70862622)(113.19605903,550.17862675)(113.1960788,549.69863863)
\lineto(113.1960788,538.29863863)
\lineto(110.6760788,538.29863863)
\lineto(110.6760788,549.39863863)
\curveto(110.67606155,549.69862723)(110.65606157,550.03862689)(110.6160788,550.41863863)
\curveto(110.59606163,550.79862613)(110.51606171,551.14862578)(110.3760788,551.46863863)
\curveto(110.23606199,551.80862512)(110.01606221,552.08862484)(109.7160788,552.30863863)
\curveto(109.43606279,552.5286244)(109.03606319,552.63862429)(108.5160788,552.63863863)
\curveto(108.21606401,552.63862429)(107.88606434,552.58862434)(107.5260788,552.48863863)
\curveto(107.18606504,552.38862454)(106.86606536,552.20862472)(106.5660788,551.94863863)
\curveto(106.26606596,551.70862522)(106.01606621,551.37862555)(105.8160788,550.95863863)
\curveto(105.61606661,550.53862639)(105.51606671,550.01862691)(105.5160788,549.39863863)
\lineto(105.5160788,538.29863863)
\lineto(102.9960788,538.29863863)
\lineto(102.9960788,549.39863863)
\curveto(102.99606923,549.69862723)(102.97606925,550.03862689)(102.9360788,550.41863863)
\curveto(102.91606931,550.79862613)(102.83606939,551.14862578)(102.6960788,551.46863863)
\curveto(102.55606967,551.80862512)(102.33606989,552.08862484)(102.0360788,552.30863863)
\curveto(101.75607047,552.5286244)(101.35607087,552.63862429)(100.8360788,552.63863863)
\curveto(100.53607169,552.63862429)(100.20607202,552.58862434)(99.8460788,552.48863863)
\curveto(99.50607272,552.38862454)(99.18607304,552.20862472)(98.8860788,551.94863863)
\curveto(98.58607364,551.70862522)(98.33607389,551.37862555)(98.1360788,550.95863863)
\curveto(97.93607429,550.53862639)(97.83607439,550.01862691)(97.8360788,549.39863863)
\lineto(97.8360788,538.29863863)
\lineto(95.3160788,538.29863863)
\lineto(95.3160788,554.37863863)
}
}
{
\newrgbcolor{curcolor}{0 0 0}
\pscustom[linestyle=none,fillstyle=solid,fillcolor=curcolor]
{
\newpath
\moveto(127.3457663,538.29863863)
\lineto(124.9457663,538.29863863)
\lineto(124.9457663,540.18863863)
\lineto(124.8857663,540.18863863)
\curveto(124.54575686,539.44863748)(124.0057574,538.87863805)(123.2657663,538.47863863)
\curveto(122.52575888,538.07863885)(121.76575964,537.87863905)(120.9857663,537.87863863)
\curveto(119.92576148,537.87863905)(119.1057623,538.04863888)(118.5257663,538.38863863)
\curveto(117.96576344,538.74863818)(117.54576386,539.18863774)(117.2657663,539.70863863)
\curveto(117.0057644,540.2286367)(116.85576455,540.77863615)(116.8157663,541.35863863)
\curveto(116.77576463,541.95863497)(116.75576465,542.49863443)(116.7557663,542.97863863)
\lineto(116.7557663,554.37863863)
\lineto(119.2757663,554.37863863)
\lineto(119.2757663,543.27863863)
\curveto(119.27576213,542.97863395)(119.28576212,542.63863429)(119.3057663,542.25863863)
\curveto(119.34576206,541.87863505)(119.44576196,541.51863541)(119.6057663,541.17863863)
\curveto(119.76576164,540.85863607)(119.99576141,540.58863634)(120.2957663,540.36863863)
\curveto(120.61576079,540.14863678)(121.06576034,540.03863689)(121.6457663,540.03863863)
\curveto(121.98575942,540.03863689)(122.33575907,540.09863683)(122.6957663,540.21863863)
\curveto(123.07575833,540.33863659)(123.42575798,540.5286364)(123.7457663,540.78863863)
\curveto(124.06575734,541.04863588)(124.32575708,541.37863555)(124.5257663,541.77863863)
\curveto(124.72575668,542.19863473)(124.82575658,542.69863423)(124.8257663,543.27863863)
\lineto(124.8257663,554.37863863)
\lineto(127.3457663,554.37863863)
\lineto(127.3457663,538.29863863)
}
}
{
\newrgbcolor{curcolor}{0 0 0}
\pscustom[linestyle=none,fillstyle=solid,fillcolor=curcolor]
{
\newpath
\moveto(130.64248505,554.37863863)
\lineto(133.04248505,554.37863863)
\lineto(133.04248505,552.48863863)
\lineto(133.10248505,552.48863863)
\curveto(133.4424806,553.2286237)(133.98248006,553.79862313)(134.72248505,554.19863863)
\curveto(135.46247858,554.59862233)(136.22247782,554.79862213)(137.00248505,554.79863863)
\curveto(138.06247598,554.79862213)(138.87247517,554.61862231)(139.43248505,554.25863863)
\curveto(140.01247403,553.91862301)(140.43247361,553.48862344)(140.69248505,552.96863863)
\curveto(140.97247307,552.44862448)(141.13247291,551.88862504)(141.17248505,551.28863863)
\curveto(141.21247283,550.70862622)(141.23247281,550.17862675)(141.23248505,549.69863863)
\lineto(141.23248505,538.29863863)
\lineto(138.71248505,538.29863863)
\lineto(138.71248505,549.39863863)
\curveto(138.71247533,549.69862723)(138.69247535,550.03862689)(138.65248505,550.41863863)
\curveto(138.63247541,550.79862613)(138.5424755,551.14862578)(138.38248505,551.46863863)
\curveto(138.22247582,551.80862512)(137.98247606,552.08862484)(137.66248505,552.30863863)
\curveto(137.3424767,552.5286244)(136.90247714,552.63862429)(136.34248505,552.63863863)
\curveto(136.00247804,552.63862429)(135.6424784,552.57862435)(135.26248505,552.45863863)
\curveto(134.90247914,552.33862459)(134.56247948,552.14862478)(134.24248505,551.88863863)
\curveto(133.92248012,551.6286253)(133.66248038,551.28862564)(133.46248505,550.86863863)
\curveto(133.26248078,550.46862646)(133.16248088,549.97862695)(133.16248505,549.39863863)
\lineto(133.16248505,538.29863863)
\lineto(130.64248505,538.29863863)
\lineto(130.64248505,554.37863863)
}
}
{
\newrgbcolor{curcolor}{0 0 0}
\pscustom[linestyle=none,fillstyle=solid,fillcolor=curcolor]
{
\newpath
\moveto(144.6792038,559.71863863)
\lineto(147.1992038,559.71863863)
\lineto(147.1992038,556.83863863)
\lineto(144.6792038,556.83863863)
\lineto(144.6792038,559.71863863)
\moveto(144.6792038,554.37863863)
\lineto(147.1992038,554.37863863)
\lineto(147.1992038,538.29863863)
\lineto(144.6792038,538.29863863)
\lineto(144.6792038,554.37863863)
}
}
{
\newrgbcolor{curcolor}{0 0 0}
\pscustom[linestyle=none,fillstyle=solid,fillcolor=curcolor]
{
\newpath
\moveto(151.3729538,559.05863863)
\lineto(153.8929538,559.05863863)
\lineto(153.8929538,554.37863863)
\lineto(156.6829538,554.37863863)
\lineto(156.6829538,552.39863863)
\lineto(153.8929538,552.39863863)
\lineto(153.8929538,542.07863863)
\curveto(153.89294888,541.43863549)(154.00294877,540.97863595)(154.2229538,540.69863863)
\curveto(154.44294833,540.41863651)(154.88294789,540.27863665)(155.5429538,540.27863863)
\curveto(155.82294695,540.27863665)(156.04294673,540.28863664)(156.2029538,540.30863863)
\curveto(156.36294641,540.3286366)(156.51294626,540.34863658)(156.6529538,540.36863863)
\lineto(156.6529538,538.29863863)
\curveto(156.49294628,538.25863867)(156.24294653,538.21863871)(155.9029538,538.17863863)
\curveto(155.56294721,538.13863879)(155.13294764,538.11863881)(154.6129538,538.11863863)
\curveto(153.95294882,538.11863881)(153.41294936,538.17863875)(152.9929538,538.29863863)
\curveto(152.5729502,538.43863849)(152.24295053,538.63863829)(152.0029538,538.89863863)
\curveto(151.76295101,539.17863775)(151.59295118,539.51863741)(151.4929538,539.91863863)
\curveto(151.41295136,540.31863661)(151.3729514,540.77863615)(151.3729538,541.29863863)
\lineto(151.3729538,552.39863863)
\lineto(149.0329538,552.39863863)
\lineto(149.0329538,554.37863863)
\lineto(151.3729538,554.37863863)
\lineto(151.3729538,559.05863863)
}
}
{
\newrgbcolor{curcolor}{0 0 0}
\pscustom[linestyle=none,fillstyle=solid,fillcolor=curcolor]
{
\newpath
\moveto(157.06592255,554.37863863)
\lineto(159.82592255,554.37863863)
\lineto(163.00592255,541.47863863)
\lineto(163.06592255,541.47863863)
\lineto(165.88592255,554.37863863)
\lineto(168.64592255,554.37863863)
\lineto(163.99592255,537.21863863)
\curveto(163.83591548,536.65864027)(163.66591565,536.1286408)(163.48592255,535.62863863)
\curveto(163.30591601,535.1286418)(163.05591626,534.68864224)(162.73592255,534.30863863)
\curveto(162.43591688,533.90864302)(162.04591727,533.59864333)(161.56592255,533.37863863)
\curveto(161.08591823,533.13864379)(160.46591885,533.01864391)(159.70592255,533.01863863)
\curveto(159.20592011,533.01864391)(158.82592049,533.0286439)(158.56592255,533.04863863)
\curveto(158.32592099,533.06864386)(158.09592122,533.08864384)(157.87592255,533.10863863)
\lineto(157.87592255,535.08863863)
\curveto(158.23592108,535.0286419)(158.72592059,534.99864193)(159.34592255,534.99863863)
\curveto(159.92591939,534.99864193)(160.35591896,535.14864178)(160.63592255,535.44863863)
\curveto(160.93591838,535.74864118)(161.16591815,536.11864081)(161.32592255,536.55863863)
\lineto(161.80592255,537.96863863)
\lineto(157.06592255,554.37863863)
}
}
{
\newrgbcolor{curcolor}{0 0 0}
\pscustom[linestyle=none,fillstyle=solid,fillcolor=curcolor]
{
\newpath
\moveto(168.95342255,536.04863863)
\lineto(183.95342255,536.04863863)
\lineto(183.95342255,534.54863863)
\lineto(168.95342255,534.54863863)
\lineto(168.95342255,536.04863863)
}
}
{
\newrgbcolor{curcolor}{0 0 0}
\pscustom[linestyle=none,fillstyle=solid,fillcolor=curcolor]
{
\newpath
\moveto(196.19342255,538.29863863)
\lineto(193.79342255,538.29863863)
\lineto(193.79342255,540.18863863)
\lineto(193.73342255,540.18863863)
\curveto(193.39341311,539.44863748)(192.85341365,538.87863805)(192.11342255,538.47863863)
\curveto(191.37341513,538.07863885)(190.61341589,537.87863905)(189.83342255,537.87863863)
\curveto(188.77341773,537.87863905)(187.95341855,538.04863888)(187.37342255,538.38863863)
\curveto(186.81341969,538.74863818)(186.39342011,539.18863774)(186.11342255,539.70863863)
\curveto(185.85342065,540.2286367)(185.7034208,540.77863615)(185.66342255,541.35863863)
\curveto(185.62342088,541.95863497)(185.6034209,542.49863443)(185.60342255,542.97863863)
\lineto(185.60342255,554.37863863)
\lineto(188.12342255,554.37863863)
\lineto(188.12342255,543.27863863)
\curveto(188.12341838,542.97863395)(188.13341837,542.63863429)(188.15342255,542.25863863)
\curveto(188.19341831,541.87863505)(188.29341821,541.51863541)(188.45342255,541.17863863)
\curveto(188.61341789,540.85863607)(188.84341766,540.58863634)(189.14342255,540.36863863)
\curveto(189.46341704,540.14863678)(189.91341659,540.03863689)(190.49342255,540.03863863)
\curveto(190.83341567,540.03863689)(191.18341532,540.09863683)(191.54342255,540.21863863)
\curveto(191.92341458,540.33863659)(192.27341423,540.5286364)(192.59342255,540.78863863)
\curveto(192.91341359,541.04863588)(193.17341333,541.37863555)(193.37342255,541.77863863)
\curveto(193.57341293,542.19863473)(193.67341283,542.69863423)(193.67342255,543.27863863)
\lineto(193.67342255,554.37863863)
\lineto(196.19342255,554.37863863)
\lineto(196.19342255,538.29863863)
}
}
{
\newrgbcolor{curcolor}{0 0 0}
\pscustom[linestyle=none,fillstyle=solid,fillcolor=curcolor]
{
\newpath
\moveto(206.3901413,549.66863863)
\curveto(206.39013275,550.68862624)(206.22013292,551.46862546)(205.8801413,552.00863863)
\curveto(205.56013358,552.54862438)(204.9401342,552.81862411)(204.0201413,552.81863863)
\curveto(203.82013532,552.81862411)(203.57013557,552.78862414)(203.2701413,552.72863863)
\curveto(202.97013617,552.68862424)(202.68013646,552.58862434)(202.4001413,552.42863863)
\curveto(202.140137,552.26862466)(201.91013723,552.01862491)(201.7101413,551.67863863)
\curveto(201.51013763,551.35862557)(201.41013773,550.91862601)(201.4101413,550.35863863)
\curveto(201.41013773,549.87862705)(201.52013762,549.48862744)(201.7401413,549.18863863)
\curveto(201.98013716,548.88862804)(202.28013686,548.6286283)(202.6401413,548.40863863)
\curveto(203.02013612,548.20862872)(203.4401357,548.03862889)(203.9001413,547.89863863)
\curveto(204.38013476,547.75862917)(204.87013427,547.60862932)(205.3701413,547.44863863)
\curveto(205.85013329,547.28862964)(206.32013282,547.10862982)(206.7801413,546.90863863)
\curveto(207.26013188,546.7286302)(207.68013146,546.46863046)(208.0401413,546.12863863)
\curveto(208.42013072,545.80863112)(208.72013042,545.38863154)(208.9401413,544.86863863)
\curveto(209.18012996,544.36863256)(209.30012984,543.71863321)(209.3001413,542.91863863)
\curveto(209.30012984,542.07863485)(209.17012997,541.33863559)(208.9101413,540.69863863)
\curveto(208.65013049,540.07863685)(208.29013085,539.55863737)(207.8301413,539.13863863)
\curveto(207.37013177,538.71863821)(206.82013232,538.40863852)(206.1801413,538.20863863)
\curveto(205.5401336,537.98863894)(204.85013429,537.87863905)(204.1101413,537.87863863)
\curveto(202.75013639,537.87863905)(201.69013745,538.09863883)(200.9301413,538.53863863)
\curveto(200.19013895,538.97863795)(199.6401395,539.49863743)(199.2801413,540.09863863)
\curveto(198.9401402,540.71863621)(198.7401404,541.34863558)(198.6801413,541.98863863)
\curveto(198.62014052,542.6286343)(198.59014055,543.15863377)(198.5901413,543.57863863)
\lineto(201.1101413,543.57863863)
\curveto(201.11013803,543.09863383)(201.15013799,542.6286343)(201.2301413,542.16863863)
\curveto(201.31013783,541.7286352)(201.46013768,541.3286356)(201.6801413,540.96863863)
\curveto(201.90013724,540.6286363)(202.20013694,540.35863657)(202.5801413,540.15863863)
\curveto(202.98013616,539.95863697)(203.49013565,539.85863707)(204.1101413,539.85863863)
\curveto(204.31013483,539.85863707)(204.56013458,539.88863704)(204.8601413,539.94863863)
\curveto(205.16013398,540.00863692)(205.45013369,540.13863679)(205.7301413,540.33863863)
\curveto(206.03013311,540.53863639)(206.28013286,540.80863612)(206.4801413,541.14863863)
\curveto(206.68013246,541.48863544)(206.78013236,541.94863498)(206.7801413,542.52863863)
\curveto(206.78013236,543.06863386)(206.66013248,543.50863342)(206.4201413,543.84863863)
\curveto(206.20013294,544.20863272)(205.90013324,544.49863243)(205.5201413,544.71863863)
\curveto(205.16013398,544.95863197)(204.7401344,545.15863177)(204.2601413,545.31863863)
\curveto(203.80013534,545.47863145)(203.33013581,545.63863129)(202.8501413,545.79863863)
\curveto(202.37013677,545.95863097)(201.89013725,546.1286308)(201.4101413,546.30863863)
\curveto(200.93013821,546.50863042)(200.50013864,546.76863016)(200.1201413,547.08863863)
\curveto(199.76013938,547.40862952)(199.46013968,547.8286291)(199.2201413,548.34863863)
\curveto(199.00014014,548.86862806)(198.89014025,549.53862739)(198.8901413,550.35863863)
\curveto(198.89014025,551.09862583)(199.02014012,551.74862518)(199.2801413,552.30863863)
\curveto(199.56013958,552.86862406)(199.93013921,553.3286236)(200.3901413,553.68863863)
\curveto(200.87013827,554.06862286)(201.42013772,554.34862258)(202.0401413,554.52863863)
\curveto(202.66013648,554.70862222)(203.32013582,554.79862213)(204.0201413,554.79863863)
\curveto(205.18013396,554.79862213)(206.09013305,554.61862231)(206.7501413,554.25863863)
\curveto(207.41013173,553.89862303)(207.90013124,553.45862347)(208.2201413,552.93863863)
\curveto(208.5401306,552.41862451)(208.73013041,551.85862507)(208.7901413,551.25863863)
\curveto(208.87013027,550.67862625)(208.91013023,550.14862678)(208.9101413,549.66863863)
\lineto(206.3901413,549.66863863)
}
}
{
\newrgbcolor{curcolor}{0 0 0}
\pscustom[linestyle=none,fillstyle=solid,fillcolor=curcolor]
{
\newpath
\moveto(213.7776413,546.03863863)
\curveto(213.77763755,545.41863151)(213.78763754,544.74863218)(213.8076413,544.02863863)
\curveto(213.84763748,543.30863362)(213.96763736,542.63863429)(214.1676413,542.01863863)
\curveto(214.36763696,541.39863553)(214.66763666,540.87863605)(215.0676413,540.45863863)
\curveto(215.48763584,540.05863687)(216.08763524,539.85863707)(216.8676413,539.85863863)
\curveto(217.46763386,539.85863707)(217.94763338,539.98863694)(218.3076413,540.24863863)
\curveto(218.66763266,540.5286364)(218.93763239,540.85863607)(219.1176413,541.23863863)
\curveto(219.31763201,541.63863529)(219.44763188,542.04863488)(219.5076413,542.46863863)
\curveto(219.56763176,542.90863402)(219.59763173,543.27863365)(219.5976413,543.57863863)
\lineto(222.1176413,543.57863863)
\curveto(222.11762921,543.15863377)(222.03762929,542.61863431)(221.8776413,541.95863863)
\curveto(221.73762959,541.31863561)(221.46762986,540.68863624)(221.0676413,540.06863863)
\curveto(220.66763066,539.46863746)(220.11763121,538.94863798)(219.4176413,538.50863863)
\curveto(218.71763261,538.08863884)(217.81763351,537.87863905)(216.7176413,537.87863863)
\curveto(214.73763659,537.87863905)(213.30763802,538.55863837)(212.4276413,539.91863863)
\curveto(211.56763976,541.27863565)(211.13764019,543.34863358)(211.1376413,546.12863863)
\curveto(211.13764019,547.1286298)(211.19764013,548.13862879)(211.3176413,549.15863863)
\curveto(211.45763987,550.19862673)(211.7276396,551.1286258)(212.1276413,551.94863863)
\curveto(212.54763878,552.78862414)(213.1276382,553.46862346)(213.8676413,553.98863863)
\curveto(214.6276367,554.5286224)(215.6276357,554.79862213)(216.8676413,554.79863863)
\curveto(218.08763324,554.79862213)(219.05763227,554.55862237)(219.7776413,554.07863863)
\curveto(220.49763083,553.59862333)(221.03763029,552.97862395)(221.3976413,552.21863863)
\curveto(221.75762957,551.47862545)(221.98762934,550.64862628)(222.0876413,549.72863863)
\curveto(222.18762914,548.80862812)(222.23762909,547.91862901)(222.2376413,547.05863863)
\lineto(222.2376413,546.03863863)
\lineto(213.7776413,546.03863863)
\moveto(219.5976413,548.01863863)
\lineto(219.5976413,548.88863863)
\curveto(219.59763173,549.3286276)(219.55763177,549.77862715)(219.4776413,550.23863863)
\curveto(219.39763193,550.71862621)(219.24763208,551.14862578)(219.0276413,551.52863863)
\curveto(218.8276325,551.90862502)(218.54763278,552.21862471)(218.1876413,552.45863863)
\curveto(217.8276335,552.69862423)(217.36763396,552.81862411)(216.8076413,552.81863863)
\curveto(216.14763518,552.81862411)(215.61763571,552.63862429)(215.2176413,552.27863863)
\curveto(214.83763649,551.93862499)(214.54763678,551.53862539)(214.3476413,551.07863863)
\curveto(214.14763718,550.61862631)(214.01763731,550.14862678)(213.9576413,549.66863863)
\curveto(213.89763743,549.20862772)(213.86763746,548.85862807)(213.8676413,548.61863863)
\lineto(213.8676413,548.01863863)
\lineto(219.5976413,548.01863863)
}
}
{
\newrgbcolor{curcolor}{0 0 0}
\pscustom[linestyle=none,fillstyle=solid,fillcolor=curcolor]
{
\newpath
\moveto(224.97842255,554.37863863)
\lineto(227.49842255,554.37863863)
\lineto(227.49842255,551.97863863)
\lineto(227.55842255,551.97863863)
\curveto(227.73841814,552.35862457)(227.92841795,552.71862421)(228.12842255,553.05863863)
\curveto(228.34841753,553.39862353)(228.59841728,553.69862323)(228.87842255,553.95863863)
\curveto(229.15841672,554.21862271)(229.46841641,554.41862251)(229.80842255,554.55863863)
\curveto(230.16841571,554.71862221)(230.5784153,554.79862213)(231.03842255,554.79863863)
\curveto(231.53841434,554.79862213)(231.90841397,554.73862219)(232.14842255,554.61863863)
\lineto(232.14842255,552.15863863)
\curveto(232.02841385,552.17862475)(231.86841401,552.19862473)(231.66842255,552.21863863)
\curveto(231.48841439,552.25862467)(231.19841468,552.27862465)(230.79842255,552.27863863)
\curveto(230.4784154,552.27862465)(230.12841575,552.19862473)(229.74842255,552.03863863)
\curveto(229.36841651,551.89862503)(229.00841687,551.66862526)(228.66842255,551.34863863)
\curveto(228.34841753,551.04862588)(228.06841781,550.65862627)(227.82842255,550.17863863)
\curveto(227.60841827,549.69862723)(227.49841838,549.1286278)(227.49842255,548.46863863)
\lineto(227.49842255,538.29863863)
\lineto(224.97842255,538.29863863)
\lineto(224.97842255,554.37863863)
}
}
{
\newrgbcolor{curcolor}{0 0 0}
\pscustom[linewidth=2,linecolor=curcolor]
{
\newpath
\moveto(76.52758026,479.64220553)
\lineto(365.17351532,479.64220553)
\lineto(365.17351532,431.49157076)
\lineto(76.52758026,431.49157076)
\closepath
}
}
{
\newrgbcolor{curcolor}{0 0 0}
\pscustom[linestyle=none,fillstyle=solid,fillcolor=curcolor]
{
\newpath
\moveto(120.27905731,458.68688082)
\curveto(120.27904801,459.06686925)(120.22904806,459.45686886)(120.12905731,459.85688082)
\curveto(120.04904824,460.25686806)(119.90904838,460.6168677)(119.70905731,460.93688082)
\curveto(119.52904876,461.25686706)(119.26904902,461.5168668)(118.92905731,461.71688082)
\curveto(118.60904968,461.9168664)(118.20905008,462.0168663)(117.72905731,462.01688082)
\curveto(117.32905096,462.0168663)(116.93905135,461.94686637)(116.55905731,461.80688082)
\curveto(116.19905209,461.66686665)(115.86905242,461.35686696)(115.56905731,460.87688082)
\curveto(115.26905302,460.4168679)(115.02905326,459.74686857)(114.84905731,458.86688082)
\curveto(114.66905362,457.98687033)(114.57905371,456.8168715)(114.57905731,455.35688082)
\curveto(114.57905371,454.83687348)(114.5890537,454.2168741)(114.60905731,453.49688082)
\curveto(114.64905364,452.77687554)(114.76905352,452.08687623)(114.96905731,451.42688082)
\curveto(115.16905312,450.76687755)(115.46905282,450.20687811)(115.86905731,449.74688082)
\curveto(116.289052,449.28687903)(116.87905141,449.05687926)(117.63905731,449.05688082)
\curveto(118.17905011,449.05687926)(118.61904967,449.18687913)(118.95905731,449.44688082)
\curveto(119.29904899,449.70687861)(119.56904872,450.02687829)(119.76905731,450.40688082)
\curveto(119.96904832,450.80687751)(120.09904819,451.23687708)(120.15905731,451.69688082)
\curveto(120.23904805,452.17687614)(120.27904801,452.63687568)(120.27905731,453.07688082)
\lineto(122.79905731,453.07688082)
\curveto(122.79904549,452.43687588)(122.70904558,451.76687655)(122.52905731,451.06688082)
\curveto(122.36904592,450.36687795)(122.07904621,449.7168786)(121.65905731,449.11688082)
\curveto(121.25904703,448.53687978)(120.71904757,448.04688027)(120.03905731,447.64688082)
\curveto(119.35904893,447.26688105)(118.51904977,447.07688124)(117.51905731,447.07688082)
\curveto(115.53905275,447.07688124)(114.10905418,447.75688056)(113.22905731,449.11688082)
\curveto(112.36905592,450.47687784)(111.93905635,452.54687577)(111.93905731,455.32688082)
\curveto(111.93905635,456.32687199)(111.99905629,457.33687098)(112.11905731,458.35688082)
\curveto(112.25905603,459.39686892)(112.52905576,460.32686799)(112.92905731,461.14688082)
\curveto(113.34905494,461.98686633)(113.92905436,462.66686565)(114.66905731,463.18688082)
\curveto(115.42905286,463.72686459)(116.42905186,463.99686432)(117.66905731,463.99688082)
\curveto(118.76904952,463.99686432)(119.64904864,463.80686451)(120.30905731,463.42688082)
\curveto(120.9890473,463.04686527)(121.50904678,462.57686574)(121.86905731,462.01688082)
\curveto(122.24904604,461.47686684)(122.49904579,460.89686742)(122.61905731,460.27688082)
\curveto(122.73904555,459.67686864)(122.79904549,459.14686917)(122.79905731,458.68688082)
\lineto(120.27905731,458.68688082)
}
}
{
\newrgbcolor{curcolor}{0 0 0}
\pscustom[linestyle=none,fillstyle=solid,fillcolor=curcolor]
{
\newpath
\moveto(124.62249481,455.53688082)
\curveto(124.62249394,456.67687164)(124.70249386,457.75687056)(124.86249481,458.77688082)
\curveto(125.04249352,459.79686852)(125.34249322,460.68686763)(125.76249481,461.44688082)
\curveto(126.20249236,462.22686609)(126.79249177,462.84686547)(127.53249481,463.30688082)
\curveto(128.29249027,463.76686455)(129.25248931,463.99686432)(130.41249481,463.99688082)
\curveto(131.57248699,463.99686432)(132.52248604,463.76686455)(133.26249481,463.30688082)
\curveto(134.02248454,462.84686547)(134.61248395,462.22686609)(135.03249481,461.44688082)
\curveto(135.47248309,460.68686763)(135.77248279,459.79686852)(135.93249481,458.77688082)
\curveto(136.11248245,457.75687056)(136.20248236,456.67687164)(136.20249481,455.53688082)
\curveto(136.20248236,454.39687392)(136.11248245,453.316875)(135.93249481,452.29688082)
\curveto(135.77248279,451.27687704)(135.47248309,450.37687794)(135.03249481,449.59688082)
\curveto(134.59248397,448.83687948)(134.00248456,448.22688009)(133.26249481,447.76688082)
\curveto(132.52248604,447.30688101)(131.57248699,447.07688124)(130.41249481,447.07688082)
\curveto(129.25248931,447.07688124)(128.29249027,447.30688101)(127.53249481,447.76688082)
\curveto(126.79249177,448.22688009)(126.20249236,448.83687948)(125.76249481,449.59688082)
\curveto(125.34249322,450.37687794)(125.04249352,451.27687704)(124.86249481,452.29688082)
\curveto(124.70249386,453.316875)(124.62249394,454.39687392)(124.62249481,455.53688082)
\moveto(130.35249481,449.05688082)
\curveto(131.01248755,449.05687926)(131.55248701,449.22687909)(131.97249481,449.56688082)
\curveto(132.39248617,449.92687839)(132.72248584,450.39687792)(132.96249481,450.97688082)
\curveto(133.20248536,451.57687674)(133.3624852,452.26687605)(133.44249481,453.04688082)
\curveto(133.52248504,453.82687449)(133.562485,454.65687366)(133.56249481,455.53688082)
\curveto(133.562485,456.39687192)(133.52248504,457.2168711)(133.44249481,457.99688082)
\curveto(133.3624852,458.79686952)(133.20248536,459.48686883)(132.96249481,460.06688082)
\curveto(132.74248582,460.66686765)(132.42248614,461.13686718)(132.00249481,461.47688082)
\curveto(131.58248698,461.83686648)(131.03248753,462.0168663)(130.35249481,462.01688082)
\curveto(129.71248885,462.0168663)(129.19248937,461.83686648)(128.79249481,461.47688082)
\curveto(128.39249017,461.13686718)(128.07249049,460.66686765)(127.83249481,460.06688082)
\curveto(127.61249095,459.48686883)(127.4624911,458.79686952)(127.38249481,457.99688082)
\curveto(127.30249126,457.2168711)(127.2624913,456.39687192)(127.26249481,455.53688082)
\curveto(127.2624913,454.65687366)(127.30249126,453.82687449)(127.38249481,453.04688082)
\curveto(127.4624911,452.26687605)(127.61249095,451.57687674)(127.83249481,450.97688082)
\curveto(128.05249051,450.39687792)(128.3624902,449.92687839)(128.76249481,449.56688082)
\curveto(129.18248938,449.22687909)(129.71248885,449.05687926)(130.35249481,449.05688082)
}
}
{
\newrgbcolor{curcolor}{0 0 0}
\pscustom[linestyle=none,fillstyle=solid,fillcolor=curcolor]
{
\newpath
\moveto(138.94327606,463.57688082)
\lineto(141.34327606,463.57688082)
\lineto(141.34327606,461.68688082)
\lineto(141.40327606,461.68688082)
\curveto(141.74327137,462.42686589)(142.28327083,462.99686532)(143.02327606,463.39688082)
\curveto(143.76326935,463.79686452)(144.52326859,463.99686432)(145.30327606,463.99688082)
\curveto(146.26326685,463.99686432)(147.0132661,463.79686452)(147.55327606,463.39688082)
\curveto(148.113265,463.0168653)(148.52326459,462.36686595)(148.78327606,461.44688082)
\curveto(149.14326397,462.14686617)(149.65326346,462.74686557)(150.31327606,463.24688082)
\curveto(150.99326212,463.74686457)(151.75326136,463.99686432)(152.59327606,463.99688082)
\curveto(153.65325946,463.99686432)(154.46325865,463.8168645)(155.02327606,463.45688082)
\curveto(155.60325751,463.1168652)(156.02325709,462.68686563)(156.28327606,462.16688082)
\curveto(156.56325655,461.64686667)(156.72325639,461.08686723)(156.76327606,460.48688082)
\curveto(156.80325631,459.90686841)(156.82325629,459.37686894)(156.82327606,458.89688082)
\lineto(156.82327606,447.49688082)
\lineto(154.30327606,447.49688082)
\lineto(154.30327606,458.59688082)
\curveto(154.30325881,458.89686942)(154.28325883,459.23686908)(154.24327606,459.61688082)
\curveto(154.22325889,459.99686832)(154.14325897,460.34686797)(154.00327606,460.66688082)
\curveto(153.86325925,461.00686731)(153.64325947,461.28686703)(153.34327606,461.50688082)
\curveto(153.06326005,461.72686659)(152.66326045,461.83686648)(152.14327606,461.83688082)
\curveto(151.84326127,461.83686648)(151.5132616,461.78686653)(151.15327606,461.68688082)
\curveto(150.8132623,461.58686673)(150.49326262,461.40686691)(150.19327606,461.14688082)
\curveto(149.89326322,460.90686741)(149.64326347,460.57686774)(149.44327606,460.15688082)
\curveto(149.24326387,459.73686858)(149.14326397,459.2168691)(149.14327606,458.59688082)
\lineto(149.14327606,447.49688082)
\lineto(146.62327606,447.49688082)
\lineto(146.62327606,458.59688082)
\curveto(146.62326649,458.89686942)(146.60326651,459.23686908)(146.56327606,459.61688082)
\curveto(146.54326657,459.99686832)(146.46326665,460.34686797)(146.32327606,460.66688082)
\curveto(146.18326693,461.00686731)(145.96326715,461.28686703)(145.66327606,461.50688082)
\curveto(145.38326773,461.72686659)(144.98326813,461.83686648)(144.46327606,461.83688082)
\curveto(144.16326895,461.83686648)(143.83326928,461.78686653)(143.47327606,461.68688082)
\curveto(143.13326998,461.58686673)(142.8132703,461.40686691)(142.51327606,461.14688082)
\curveto(142.2132709,460.90686741)(141.96327115,460.57686774)(141.76327606,460.15688082)
\curveto(141.56327155,459.73686858)(141.46327165,459.2168691)(141.46327606,458.59688082)
\lineto(141.46327606,447.49688082)
\lineto(138.94327606,447.49688082)
\lineto(138.94327606,463.57688082)
}
}
{
\newrgbcolor{curcolor}{0 0 0}
\pscustom[linestyle=none,fillstyle=solid,fillcolor=curcolor]
{
\newpath
\moveto(160.62296356,463.57688082)
\lineto(163.02296356,463.57688082)
\lineto(163.02296356,461.68688082)
\lineto(163.08296356,461.68688082)
\curveto(163.42295887,462.42686589)(163.96295833,462.99686532)(164.70296356,463.39688082)
\curveto(165.44295685,463.79686452)(166.20295609,463.99686432)(166.98296356,463.99688082)
\curveto(167.94295435,463.99686432)(168.6929536,463.79686452)(169.23296356,463.39688082)
\curveto(169.7929525,463.0168653)(170.20295209,462.36686595)(170.46296356,461.44688082)
\curveto(170.82295147,462.14686617)(171.33295096,462.74686557)(171.99296356,463.24688082)
\curveto(172.67294962,463.74686457)(173.43294886,463.99686432)(174.27296356,463.99688082)
\curveto(175.33294696,463.99686432)(176.14294615,463.8168645)(176.70296356,463.45688082)
\curveto(177.28294501,463.1168652)(177.70294459,462.68686563)(177.96296356,462.16688082)
\curveto(178.24294405,461.64686667)(178.40294389,461.08686723)(178.44296356,460.48688082)
\curveto(178.48294381,459.90686841)(178.50294379,459.37686894)(178.50296356,458.89688082)
\lineto(178.50296356,447.49688082)
\lineto(175.98296356,447.49688082)
\lineto(175.98296356,458.59688082)
\curveto(175.98294631,458.89686942)(175.96294633,459.23686908)(175.92296356,459.61688082)
\curveto(175.90294639,459.99686832)(175.82294647,460.34686797)(175.68296356,460.66688082)
\curveto(175.54294675,461.00686731)(175.32294697,461.28686703)(175.02296356,461.50688082)
\curveto(174.74294755,461.72686659)(174.34294795,461.83686648)(173.82296356,461.83688082)
\curveto(173.52294877,461.83686648)(173.1929491,461.78686653)(172.83296356,461.68688082)
\curveto(172.4929498,461.58686673)(172.17295012,461.40686691)(171.87296356,461.14688082)
\curveto(171.57295072,460.90686741)(171.32295097,460.57686774)(171.12296356,460.15688082)
\curveto(170.92295137,459.73686858)(170.82295147,459.2168691)(170.82296356,458.59688082)
\lineto(170.82296356,447.49688082)
\lineto(168.30296356,447.49688082)
\lineto(168.30296356,458.59688082)
\curveto(168.30295399,458.89686942)(168.28295401,459.23686908)(168.24296356,459.61688082)
\curveto(168.22295407,459.99686832)(168.14295415,460.34686797)(168.00296356,460.66688082)
\curveto(167.86295443,461.00686731)(167.64295465,461.28686703)(167.34296356,461.50688082)
\curveto(167.06295523,461.72686659)(166.66295563,461.83686648)(166.14296356,461.83688082)
\curveto(165.84295645,461.83686648)(165.51295678,461.78686653)(165.15296356,461.68688082)
\curveto(164.81295748,461.58686673)(164.4929578,461.40686691)(164.19296356,461.14688082)
\curveto(163.8929584,460.90686741)(163.64295865,460.57686774)(163.44296356,460.15688082)
\curveto(163.24295905,459.73686858)(163.14295915,459.2168691)(163.14296356,458.59688082)
\lineto(163.14296356,447.49688082)
\lineto(160.62296356,447.49688082)
\lineto(160.62296356,463.57688082)
}
}
{
\newrgbcolor{curcolor}{0 0 0}
\pscustom[linestyle=none,fillstyle=solid,fillcolor=curcolor]
{
\newpath
\moveto(192.65265106,447.49688082)
\lineto(190.25265106,447.49688082)
\lineto(190.25265106,449.38688082)
\lineto(190.19265106,449.38688082)
\curveto(189.85264162,448.64687967)(189.31264216,448.07688024)(188.57265106,447.67688082)
\curveto(187.83264364,447.27688104)(187.0726444,447.07688124)(186.29265106,447.07688082)
\curveto(185.23264624,447.07688124)(184.41264706,447.24688107)(183.83265106,447.58688082)
\curveto(183.2726482,447.94688037)(182.85264862,448.38687993)(182.57265106,448.90688082)
\curveto(182.31264916,449.42687889)(182.16264931,449.97687834)(182.12265106,450.55688082)
\curveto(182.08264939,451.15687716)(182.06264941,451.69687662)(182.06265106,452.17688082)
\lineto(182.06265106,463.57688082)
\lineto(184.58265106,463.57688082)
\lineto(184.58265106,452.47688082)
\curveto(184.58264689,452.17687614)(184.59264688,451.83687648)(184.61265106,451.45688082)
\curveto(184.65264682,451.07687724)(184.75264672,450.7168776)(184.91265106,450.37688082)
\curveto(185.0726464,450.05687826)(185.30264617,449.78687853)(185.60265106,449.56688082)
\curveto(185.92264555,449.34687897)(186.3726451,449.23687908)(186.95265106,449.23688082)
\curveto(187.29264418,449.23687908)(187.64264383,449.29687902)(188.00265106,449.41688082)
\curveto(188.38264309,449.53687878)(188.73264274,449.72687859)(189.05265106,449.98688082)
\curveto(189.3726421,450.24687807)(189.63264184,450.57687774)(189.83265106,450.97688082)
\curveto(190.03264144,451.39687692)(190.13264134,451.89687642)(190.13265106,452.47688082)
\lineto(190.13265106,463.57688082)
\lineto(192.65265106,463.57688082)
\lineto(192.65265106,447.49688082)
}
}
{
\newrgbcolor{curcolor}{0 0 0}
\pscustom[linestyle=none,fillstyle=solid,fillcolor=curcolor]
{
\newpath
\moveto(195.94936981,463.57688082)
\lineto(198.34936981,463.57688082)
\lineto(198.34936981,461.68688082)
\lineto(198.40936981,461.68688082)
\curveto(198.74936536,462.42686589)(199.28936482,462.99686532)(200.02936981,463.39688082)
\curveto(200.76936334,463.79686452)(201.52936258,463.99686432)(202.30936981,463.99688082)
\curveto(203.36936074,463.99686432)(204.17935993,463.8168645)(204.73936981,463.45688082)
\curveto(205.31935879,463.1168652)(205.73935837,462.68686563)(205.99936981,462.16688082)
\curveto(206.27935783,461.64686667)(206.43935767,461.08686723)(206.47936981,460.48688082)
\curveto(206.51935759,459.90686841)(206.53935757,459.37686894)(206.53936981,458.89688082)
\lineto(206.53936981,447.49688082)
\lineto(204.01936981,447.49688082)
\lineto(204.01936981,458.59688082)
\curveto(204.01936009,458.89686942)(203.99936011,459.23686908)(203.95936981,459.61688082)
\curveto(203.93936017,459.99686832)(203.84936026,460.34686797)(203.68936981,460.66688082)
\curveto(203.52936058,461.00686731)(203.28936082,461.28686703)(202.96936981,461.50688082)
\curveto(202.64936146,461.72686659)(202.2093619,461.83686648)(201.64936981,461.83688082)
\curveto(201.3093628,461.83686648)(200.94936316,461.77686654)(200.56936981,461.65688082)
\curveto(200.2093639,461.53686678)(199.86936424,461.34686697)(199.54936981,461.08688082)
\curveto(199.22936488,460.82686749)(198.96936514,460.48686783)(198.76936981,460.06688082)
\curveto(198.56936554,459.66686865)(198.46936564,459.17686914)(198.46936981,458.59688082)
\lineto(198.46936981,447.49688082)
\lineto(195.94936981,447.49688082)
\lineto(195.94936981,463.57688082)
}
}
{
\newrgbcolor{curcolor}{0 0 0}
\pscustom[linestyle=none,fillstyle=solid,fillcolor=curcolor]
{
\newpath
\moveto(209.98608856,468.91688082)
\lineto(212.50608856,468.91688082)
\lineto(212.50608856,466.03688082)
\lineto(209.98608856,466.03688082)
\lineto(209.98608856,468.91688082)
\moveto(209.98608856,463.57688082)
\lineto(212.50608856,463.57688082)
\lineto(212.50608856,447.49688082)
\lineto(209.98608856,447.49688082)
\lineto(209.98608856,463.57688082)
}
}
{
\newrgbcolor{curcolor}{0 0 0}
\pscustom[linestyle=none,fillstyle=solid,fillcolor=curcolor]
{
\newpath
\moveto(216.67983856,468.25688082)
\lineto(219.19983856,468.25688082)
\lineto(219.19983856,463.57688082)
\lineto(221.98983856,463.57688082)
\lineto(221.98983856,461.59688082)
\lineto(219.19983856,461.59688082)
\lineto(219.19983856,451.27688082)
\curveto(219.19983364,450.63687768)(219.30983353,450.17687814)(219.52983856,449.89688082)
\curveto(219.74983309,449.6168787)(220.18983265,449.47687884)(220.84983856,449.47688082)
\curveto(221.12983171,449.47687884)(221.34983149,449.48687883)(221.50983856,449.50688082)
\curveto(221.66983117,449.52687879)(221.81983102,449.54687877)(221.95983856,449.56688082)
\lineto(221.95983856,447.49688082)
\curveto(221.79983104,447.45688086)(221.54983129,447.4168809)(221.20983856,447.37688082)
\curveto(220.86983197,447.33688098)(220.4398324,447.316881)(219.91983856,447.31688082)
\curveto(219.25983358,447.316881)(218.71983412,447.37688094)(218.29983856,447.49688082)
\curveto(217.87983496,447.63688068)(217.54983529,447.83688048)(217.30983856,448.09688082)
\curveto(217.06983577,448.37687994)(216.89983594,448.7168796)(216.79983856,449.11688082)
\curveto(216.71983612,449.5168788)(216.67983616,449.97687834)(216.67983856,450.49688082)
\lineto(216.67983856,461.59688082)
\lineto(214.33983856,461.59688082)
\lineto(214.33983856,463.57688082)
\lineto(216.67983856,463.57688082)
\lineto(216.67983856,468.25688082)
}
}
{
\newrgbcolor{curcolor}{0 0 0}
\pscustom[linestyle=none,fillstyle=solid,fillcolor=curcolor]
{
\newpath
\moveto(222.37280731,463.57688082)
\lineto(225.13280731,463.57688082)
\lineto(228.31280731,450.67688082)
\lineto(228.37280731,450.67688082)
\lineto(231.19280731,463.57688082)
\lineto(233.95280731,463.57688082)
\lineto(229.30280731,446.41688082)
\curveto(229.14280024,445.85688246)(228.97280041,445.32688299)(228.79280731,444.82688082)
\curveto(228.61280077,444.32688399)(228.36280102,443.88688443)(228.04280731,443.50688082)
\curveto(227.74280164,443.10688521)(227.35280203,442.79688552)(226.87280731,442.57688082)
\curveto(226.39280299,442.33688598)(225.77280361,442.2168861)(225.01280731,442.21688082)
\curveto(224.51280487,442.2168861)(224.13280525,442.22688609)(223.87280731,442.24688082)
\curveto(223.63280575,442.26688605)(223.40280598,442.28688603)(223.18280731,442.30688082)
\lineto(223.18280731,444.28688082)
\curveto(223.54280584,444.22688409)(224.03280535,444.19688412)(224.65280731,444.19688082)
\curveto(225.23280415,444.19688412)(225.66280372,444.34688397)(225.94280731,444.64688082)
\curveto(226.24280314,444.94688337)(226.47280291,445.316883)(226.63280731,445.75688082)
\lineto(227.11280731,447.16688082)
\lineto(222.37280731,463.57688082)
}
}
{
\newrgbcolor{curcolor}{0 0 0}
\pscustom[linestyle=none,fillstyle=solid,fillcolor=curcolor]
{
\newpath
\moveto(234.26030731,445.24688082)
\lineto(249.26030731,445.24688082)
\lineto(249.26030731,443.74688082)
\lineto(234.26030731,443.74688082)
\lineto(234.26030731,445.24688082)
}
}
{
\newrgbcolor{curcolor}{0 0 0}
\pscustom[linestyle=none,fillstyle=solid,fillcolor=curcolor]
{
\newpath
\moveto(250.91030731,463.57688082)
\lineto(253.31030731,463.57688082)
\lineto(253.31030731,461.59688082)
\lineto(253.37030731,461.59688082)
\curveto(253.45030312,461.89686642)(253.59030298,462.18686613)(253.79030731,462.46688082)
\curveto(254.01030256,462.76686555)(254.2703023,463.02686529)(254.57030731,463.24688082)
\curveto(254.89030168,463.46686485)(255.24030133,463.64686467)(255.62030731,463.78688082)
\curveto(256.00030057,463.92686439)(256.41030016,463.99686432)(256.85030731,463.99688082)
\curveto(257.81029876,463.99686432)(258.62029795,463.79686452)(259.28030731,463.39688082)
\curveto(259.94029663,462.99686532)(260.48029609,462.43686588)(260.90030731,461.71688082)
\curveto(261.32029525,460.99686732)(261.62029495,460.13686818)(261.80030731,459.13688082)
\curveto(262.00029457,458.13687018)(262.10029447,457.03687128)(262.10030731,455.83688082)
\curveto(262.10029447,454.9168734)(262.02029455,453.94687437)(261.86030731,452.92688082)
\curveto(261.70029487,451.90687641)(261.42029515,450.95687736)(261.02030731,450.07688082)
\curveto(260.64029593,449.2168791)(260.11029646,448.49687982)(259.43030731,447.91688082)
\curveto(258.75029782,447.35688096)(257.89029868,447.07688124)(256.85030731,447.07688082)
\curveto(256.13030044,447.07688124)(255.46030111,447.27688104)(254.84030731,447.67688082)
\curveto(254.22030235,448.07688024)(253.7703028,448.62687969)(253.49030731,449.32688082)
\lineto(253.43030731,449.32688082)
\lineto(253.43030731,442.21688082)
\lineto(250.91030731,442.21688082)
\lineto(250.91030731,463.57688082)
\moveto(253.28030731,455.53688082)
\curveto(253.28030329,454.75687356)(253.32030325,453.98687433)(253.40030731,453.22688082)
\curveto(253.48030309,452.48687583)(253.64030293,451.8168765)(253.88030731,451.21688082)
\curveto(254.12030245,450.6168777)(254.45030212,450.13687818)(254.87030731,449.77688082)
\curveto(255.29030128,449.4168789)(255.84030073,449.23687908)(256.52030731,449.23688082)
\curveto(257.10029947,449.23687908)(257.58029899,449.38687893)(257.96030731,449.68688082)
\curveto(258.34029823,449.98687833)(258.64029793,450.4168779)(258.86030731,450.97688082)
\curveto(259.08029749,451.53687678)(259.23029734,452.22687609)(259.31030731,453.04688082)
\curveto(259.41029716,453.86687445)(259.46029711,454.79687352)(259.46030731,455.83688082)
\curveto(259.46029711,456.7168716)(259.41029716,457.52687079)(259.31030731,458.26688082)
\curveto(259.23029734,459.00686931)(259.08029749,459.63686868)(258.86030731,460.15688082)
\curveto(258.64029793,460.69686762)(258.34029823,461.10686721)(257.96030731,461.38688082)
\curveto(257.58029899,461.68686663)(257.10029947,461.83686648)(256.52030731,461.83688082)
\curveto(255.82030075,461.83686648)(255.26030131,461.67686664)(254.84030731,461.35688082)
\curveto(254.42030215,461.05686726)(254.09030248,460.62686769)(253.85030731,460.06688082)
\curveto(253.63030294,459.50686881)(253.48030309,458.83686948)(253.40030731,458.05688082)
\curveto(253.32030325,457.29687102)(253.28030329,456.45687186)(253.28030731,455.53688082)
}
}
{
\newrgbcolor{curcolor}{0 0 0}
\pscustom[linestyle=none,fillstyle=solid,fillcolor=curcolor]
{
\newpath
\moveto(266.89702606,455.23688082)
\curveto(266.89702231,454.6168737)(266.9070223,453.94687437)(266.92702606,453.22688082)
\curveto(266.96702224,452.50687581)(267.08702212,451.83687648)(267.28702606,451.21688082)
\curveto(267.48702172,450.59687772)(267.78702142,450.07687824)(268.18702606,449.65688082)
\curveto(268.6070206,449.25687906)(269.20702,449.05687926)(269.98702606,449.05688082)
\curveto(270.58701862,449.05687926)(271.06701814,449.18687913)(271.42702606,449.44688082)
\curveto(271.78701742,449.72687859)(272.05701715,450.05687826)(272.23702606,450.43688082)
\curveto(272.43701677,450.83687748)(272.56701664,451.24687707)(272.62702606,451.66688082)
\curveto(272.68701652,452.10687621)(272.71701649,452.47687584)(272.71702606,452.77688082)
\lineto(275.23702606,452.77688082)
\curveto(275.23701397,452.35687596)(275.15701405,451.8168765)(274.99702606,451.15688082)
\curveto(274.85701435,450.5168778)(274.58701462,449.88687843)(274.18702606,449.26688082)
\curveto(273.78701542,448.66687965)(273.23701597,448.14688017)(272.53702606,447.70688082)
\curveto(271.83701737,447.28688103)(270.93701827,447.07688124)(269.83702606,447.07688082)
\curveto(267.85702135,447.07688124)(266.42702278,447.75688056)(265.54702606,449.11688082)
\curveto(264.68702452,450.47687784)(264.25702495,452.54687577)(264.25702606,455.32688082)
\curveto(264.25702495,456.32687199)(264.31702489,457.33687098)(264.43702606,458.35688082)
\curveto(264.57702463,459.39686892)(264.84702436,460.32686799)(265.24702606,461.14688082)
\curveto(265.66702354,461.98686633)(266.24702296,462.66686565)(266.98702606,463.18688082)
\curveto(267.74702146,463.72686459)(268.74702046,463.99686432)(269.98702606,463.99688082)
\curveto(271.207018,463.99686432)(272.17701703,463.75686456)(272.89702606,463.27688082)
\curveto(273.61701559,462.79686552)(274.15701505,462.17686614)(274.51702606,461.41688082)
\curveto(274.87701433,460.67686764)(275.1070141,459.84686847)(275.20702606,458.92688082)
\curveto(275.3070139,458.00687031)(275.35701385,457.1168712)(275.35702606,456.25688082)
\lineto(275.35702606,455.23688082)
\lineto(266.89702606,455.23688082)
\moveto(272.71702606,457.21688082)
\lineto(272.71702606,458.08688082)
\curveto(272.71701649,458.52686979)(272.67701653,458.97686934)(272.59702606,459.43688082)
\curveto(272.51701669,459.9168684)(272.36701684,460.34686797)(272.14702606,460.72688082)
\curveto(271.94701726,461.10686721)(271.66701754,461.4168669)(271.30702606,461.65688082)
\curveto(270.94701826,461.89686642)(270.48701872,462.0168663)(269.92702606,462.01688082)
\curveto(269.26701994,462.0168663)(268.73702047,461.83686648)(268.33702606,461.47688082)
\curveto(267.95702125,461.13686718)(267.66702154,460.73686758)(267.46702606,460.27688082)
\curveto(267.26702194,459.8168685)(267.13702207,459.34686897)(267.07702606,458.86688082)
\curveto(267.01702219,458.40686991)(266.98702222,458.05687026)(266.98702606,457.81688082)
\lineto(266.98702606,457.21688082)
\lineto(272.71702606,457.21688082)
}
}
{
\newrgbcolor{curcolor}{0 0 0}
\pscustom[linestyle=none,fillstyle=solid,fillcolor=curcolor]
{
\newpath
\moveto(278.84780731,468.25688082)
\lineto(281.36780731,468.25688082)
\lineto(281.36780731,463.57688082)
\lineto(284.15780731,463.57688082)
\lineto(284.15780731,461.59688082)
\lineto(281.36780731,461.59688082)
\lineto(281.36780731,451.27688082)
\curveto(281.36780239,450.63687768)(281.47780228,450.17687814)(281.69780731,449.89688082)
\curveto(281.91780184,449.6168787)(282.3578014,449.47687884)(283.01780731,449.47688082)
\curveto(283.29780046,449.47687884)(283.51780024,449.48687883)(283.67780731,449.50688082)
\curveto(283.83779992,449.52687879)(283.98779977,449.54687877)(284.12780731,449.56688082)
\lineto(284.12780731,447.49688082)
\curveto(283.96779979,447.45688086)(283.71780004,447.4168809)(283.37780731,447.37688082)
\curveto(283.03780072,447.33688098)(282.60780115,447.316881)(282.08780731,447.31688082)
\curveto(281.42780233,447.316881)(280.88780287,447.37688094)(280.46780731,447.49688082)
\curveto(280.04780371,447.63688068)(279.71780404,447.83688048)(279.47780731,448.09688082)
\curveto(279.23780452,448.37687994)(279.06780469,448.7168796)(278.96780731,449.11688082)
\curveto(278.88780487,449.5168788)(278.84780491,449.97687834)(278.84780731,450.49688082)
\lineto(278.84780731,461.59688082)
\lineto(276.50780731,461.59688082)
\lineto(276.50780731,463.57688082)
\lineto(278.84780731,463.57688082)
\lineto(278.84780731,468.25688082)
}
}
{
\newrgbcolor{curcolor}{0 0 0}
\pscustom[linestyle=none,fillstyle=solid,fillcolor=curcolor]
{
\newpath
\moveto(286.04077606,468.91688082)
\lineto(288.56077606,468.91688082)
\lineto(288.56077606,466.03688082)
\lineto(286.04077606,466.03688082)
\lineto(286.04077606,468.91688082)
\moveto(286.04077606,463.57688082)
\lineto(288.56077606,463.57688082)
\lineto(288.56077606,447.49688082)
\lineto(286.04077606,447.49688082)
\lineto(286.04077606,463.57688082)
}
}
{
\newrgbcolor{curcolor}{0 0 0}
\pscustom[linestyle=none,fillstyle=solid,fillcolor=curcolor]
{
\newpath
\moveto(292.73452606,468.25688082)
\lineto(295.25452606,468.25688082)
\lineto(295.25452606,463.57688082)
\lineto(298.04452606,463.57688082)
\lineto(298.04452606,461.59688082)
\lineto(295.25452606,461.59688082)
\lineto(295.25452606,451.27688082)
\curveto(295.25452114,450.63687768)(295.36452103,450.17687814)(295.58452606,449.89688082)
\curveto(295.80452059,449.6168787)(296.24452015,449.47687884)(296.90452606,449.47688082)
\curveto(297.18451921,449.47687884)(297.40451899,449.48687883)(297.56452606,449.50688082)
\curveto(297.72451867,449.52687879)(297.87451852,449.54687877)(298.01452606,449.56688082)
\lineto(298.01452606,447.49688082)
\curveto(297.85451854,447.45688086)(297.60451879,447.4168809)(297.26452606,447.37688082)
\curveto(296.92451947,447.33688098)(296.4945199,447.316881)(295.97452606,447.31688082)
\curveto(295.31452108,447.316881)(294.77452162,447.37688094)(294.35452606,447.49688082)
\curveto(293.93452246,447.63688068)(293.60452279,447.83688048)(293.36452606,448.09688082)
\curveto(293.12452327,448.37687994)(292.95452344,448.7168796)(292.85452606,449.11688082)
\curveto(292.77452362,449.5168788)(292.73452366,449.97687834)(292.73452606,450.49688082)
\lineto(292.73452606,461.59688082)
\lineto(290.39452606,461.59688082)
\lineto(290.39452606,463.57688082)
\lineto(292.73452606,463.57688082)
\lineto(292.73452606,468.25688082)
}
}
{
\newrgbcolor{curcolor}{0 0 0}
\pscustom[linestyle=none,fillstyle=solid,fillcolor=curcolor]
{
\newpath
\moveto(299.92749481,468.91688082)
\lineto(302.44749481,468.91688082)
\lineto(302.44749481,466.03688082)
\lineto(299.92749481,466.03688082)
\lineto(299.92749481,468.91688082)
\moveto(299.92749481,463.57688082)
\lineto(302.44749481,463.57688082)
\lineto(302.44749481,447.49688082)
\lineto(299.92749481,447.49688082)
\lineto(299.92749481,463.57688082)
}
}
{
\newrgbcolor{curcolor}{0 0 0}
\pscustom[linestyle=none,fillstyle=solid,fillcolor=curcolor]
{
\newpath
\moveto(305.09124481,455.53688082)
\curveto(305.09124394,456.67687164)(305.17124386,457.75687056)(305.33124481,458.77688082)
\curveto(305.51124352,459.79686852)(305.81124322,460.68686763)(306.23124481,461.44688082)
\curveto(306.67124236,462.22686609)(307.26124177,462.84686547)(308.00124481,463.30688082)
\curveto(308.76124027,463.76686455)(309.72123931,463.99686432)(310.88124481,463.99688082)
\curveto(312.04123699,463.99686432)(312.99123604,463.76686455)(313.73124481,463.30688082)
\curveto(314.49123454,462.84686547)(315.08123395,462.22686609)(315.50124481,461.44688082)
\curveto(315.94123309,460.68686763)(316.24123279,459.79686852)(316.40124481,458.77688082)
\curveto(316.58123245,457.75687056)(316.67123236,456.67687164)(316.67124481,455.53688082)
\curveto(316.67123236,454.39687392)(316.58123245,453.316875)(316.40124481,452.29688082)
\curveto(316.24123279,451.27687704)(315.94123309,450.37687794)(315.50124481,449.59688082)
\curveto(315.06123397,448.83687948)(314.47123456,448.22688009)(313.73124481,447.76688082)
\curveto(312.99123604,447.30688101)(312.04123699,447.07688124)(310.88124481,447.07688082)
\curveto(309.72123931,447.07688124)(308.76124027,447.30688101)(308.00124481,447.76688082)
\curveto(307.26124177,448.22688009)(306.67124236,448.83687948)(306.23124481,449.59688082)
\curveto(305.81124322,450.37687794)(305.51124352,451.27687704)(305.33124481,452.29688082)
\curveto(305.17124386,453.316875)(305.09124394,454.39687392)(305.09124481,455.53688082)
\moveto(310.82124481,449.05688082)
\curveto(311.48123755,449.05687926)(312.02123701,449.22687909)(312.44124481,449.56688082)
\curveto(312.86123617,449.92687839)(313.19123584,450.39687792)(313.43124481,450.97688082)
\curveto(313.67123536,451.57687674)(313.8312352,452.26687605)(313.91124481,453.04688082)
\curveto(313.99123504,453.82687449)(314.031235,454.65687366)(314.03124481,455.53688082)
\curveto(314.031235,456.39687192)(313.99123504,457.2168711)(313.91124481,457.99688082)
\curveto(313.8312352,458.79686952)(313.67123536,459.48686883)(313.43124481,460.06688082)
\curveto(313.21123582,460.66686765)(312.89123614,461.13686718)(312.47124481,461.47688082)
\curveto(312.05123698,461.83686648)(311.50123753,462.0168663)(310.82124481,462.01688082)
\curveto(310.18123885,462.0168663)(309.66123937,461.83686648)(309.26124481,461.47688082)
\curveto(308.86124017,461.13686718)(308.54124049,460.66686765)(308.30124481,460.06688082)
\curveto(308.08124095,459.48686883)(307.9312411,458.79686952)(307.85124481,457.99688082)
\curveto(307.77124126,457.2168711)(307.7312413,456.39687192)(307.73124481,455.53688082)
\curveto(307.7312413,454.65687366)(307.77124126,453.82687449)(307.85124481,453.04688082)
\curveto(307.9312411,452.26687605)(308.08124095,451.57687674)(308.30124481,450.97688082)
\curveto(308.52124051,450.39687792)(308.8312402,449.92687839)(309.23124481,449.56688082)
\curveto(309.65123938,449.22687909)(310.18123885,449.05687926)(310.82124481,449.05688082)
}
}
{
\newrgbcolor{curcolor}{0 0 0}
\pscustom[linestyle=none,fillstyle=solid,fillcolor=curcolor]
{
\newpath
\moveto(319.17202606,463.57688082)
\lineto(321.57202606,463.57688082)
\lineto(321.57202606,461.68688082)
\lineto(321.63202606,461.68688082)
\curveto(321.97202161,462.42686589)(322.51202107,462.99686532)(323.25202606,463.39688082)
\curveto(323.99201959,463.79686452)(324.75201883,463.99686432)(325.53202606,463.99688082)
\curveto(326.59201699,463.99686432)(327.40201618,463.8168645)(327.96202606,463.45688082)
\curveto(328.54201504,463.1168652)(328.96201462,462.68686563)(329.22202606,462.16688082)
\curveto(329.50201408,461.64686667)(329.66201392,461.08686723)(329.70202606,460.48688082)
\curveto(329.74201384,459.90686841)(329.76201382,459.37686894)(329.76202606,458.89688082)
\lineto(329.76202606,447.49688082)
\lineto(327.24202606,447.49688082)
\lineto(327.24202606,458.59688082)
\curveto(327.24201634,458.89686942)(327.22201636,459.23686908)(327.18202606,459.61688082)
\curveto(327.16201642,459.99686832)(327.07201651,460.34686797)(326.91202606,460.66688082)
\curveto(326.75201683,461.00686731)(326.51201707,461.28686703)(326.19202606,461.50688082)
\curveto(325.87201771,461.72686659)(325.43201815,461.83686648)(324.87202606,461.83688082)
\curveto(324.53201905,461.83686648)(324.17201941,461.77686654)(323.79202606,461.65688082)
\curveto(323.43202015,461.53686678)(323.09202049,461.34686697)(322.77202606,461.08688082)
\curveto(322.45202113,460.82686749)(322.19202139,460.48686783)(321.99202606,460.06688082)
\curveto(321.79202179,459.66686865)(321.69202189,459.17686914)(321.69202606,458.59688082)
\lineto(321.69202606,447.49688082)
\lineto(319.17202606,447.49688082)
\lineto(319.17202606,463.57688082)
}
}
{
\newrgbcolor{curcolor}{0 0 0}
\pscustom[linewidth=2,linecolor=curcolor]
{
\newpath
\moveto(630.98297119,77.3902524)
\lineto(779.13360596,77.3902524)
\lineto(779.13360596,29.23962908)
\lineto(630.98297119,29.23962908)
\closepath
}
}
{
\newrgbcolor{curcolor}{0 0 0}
\pscustom[linestyle=none,fillstyle=solid,fillcolor=curcolor]
{
\newpath
\moveto(692.37580322,45.27495699)
\lineto(689.97580322,45.27495699)
\lineto(689.97580322,47.16495699)
\lineto(689.91580322,47.16495699)
\curveto(689.57579378,46.42495584)(689.03579432,45.85495641)(688.29580322,45.45495699)
\curveto(687.5557958,45.05495721)(686.79579656,44.85495741)(686.01580322,44.85495699)
\curveto(684.9557984,44.85495741)(684.13579922,45.02495724)(683.55580322,45.36495699)
\curveto(682.99580036,45.72495654)(682.57580078,46.1649561)(682.29580322,46.68495699)
\curveto(682.03580132,47.20495506)(681.88580147,47.75495451)(681.84580322,48.33495699)
\curveto(681.80580155,48.93495333)(681.78580157,49.47495279)(681.78580322,49.95495699)
\lineto(681.78580322,61.35495699)
\lineto(684.30580322,61.35495699)
\lineto(684.30580322,50.25495699)
\curveto(684.30579905,49.95495231)(684.31579904,49.61495265)(684.33580322,49.23495699)
\curveto(684.37579898,48.85495341)(684.47579888,48.49495377)(684.63580322,48.15495699)
\curveto(684.79579856,47.83495443)(685.02579833,47.5649547)(685.32580322,47.34495699)
\curveto(685.64579771,47.12495514)(686.09579726,47.01495525)(686.67580322,47.01495699)
\curveto(687.01579634,47.01495525)(687.36579599,47.07495519)(687.72580322,47.19495699)
\curveto(688.10579525,47.31495495)(688.4557949,47.50495476)(688.77580322,47.76495699)
\curveto(689.09579426,48.02495424)(689.355794,48.35495391)(689.55580322,48.75495699)
\curveto(689.7557936,49.17495309)(689.8557935,49.67495259)(689.85580322,50.25495699)
\lineto(689.85580322,61.35495699)
\lineto(692.37580322,61.35495699)
\lineto(692.37580322,45.27495699)
}
}
{
\newrgbcolor{curcolor}{0 0 0}
\pscustom[linestyle=none,fillstyle=solid,fillcolor=curcolor]
{
\newpath
\moveto(702.57252197,56.64495699)
\curveto(702.57251342,57.6649446)(702.40251359,58.44494382)(702.06252197,58.98495699)
\curveto(701.74251425,59.52494274)(701.12251487,59.79494247)(700.20252197,59.79495699)
\curveto(700.00251599,59.79494247)(699.75251624,59.7649425)(699.45252197,59.70495699)
\curveto(699.15251684,59.6649426)(698.86251713,59.5649427)(698.58252197,59.40495699)
\curveto(698.32251767,59.24494302)(698.0925179,58.99494327)(697.89252197,58.65495699)
\curveto(697.6925183,58.33494393)(697.5925184,57.89494437)(697.59252197,57.33495699)
\curveto(697.5925184,56.85494541)(697.70251829,56.4649458)(697.92252197,56.16495699)
\curveto(698.16251783,55.8649464)(698.46251753,55.60494666)(698.82252197,55.38495699)
\curveto(699.20251679,55.18494708)(699.62251637,55.01494725)(700.08252197,54.87495699)
\curveto(700.56251543,54.73494753)(701.05251494,54.58494768)(701.55252197,54.42495699)
\curveto(702.03251396,54.264948)(702.50251349,54.08494818)(702.96252197,53.88495699)
\curveto(703.44251255,53.70494856)(703.86251213,53.44494882)(704.22252197,53.10495699)
\curveto(704.60251139,52.78494948)(704.90251109,52.3649499)(705.12252197,51.84495699)
\curveto(705.36251063,51.34495092)(705.48251051,50.69495157)(705.48252197,49.89495699)
\curveto(705.48251051,49.05495321)(705.35251064,48.31495395)(705.09252197,47.67495699)
\curveto(704.83251116,47.05495521)(704.47251152,46.53495573)(704.01252197,46.11495699)
\curveto(703.55251244,45.69495657)(703.00251299,45.38495688)(702.36252197,45.18495699)
\curveto(701.72251427,44.9649573)(701.03251496,44.85495741)(700.29252197,44.85495699)
\curveto(698.93251706,44.85495741)(697.87251812,45.07495719)(697.11252197,45.51495699)
\curveto(696.37251962,45.95495631)(695.82252017,46.47495579)(695.46252197,47.07495699)
\curveto(695.12252087,47.69495457)(694.92252107,48.32495394)(694.86252197,48.96495699)
\curveto(694.80252119,49.60495266)(694.77252122,50.13495213)(694.77252197,50.55495699)
\lineto(697.29252197,50.55495699)
\curveto(697.2925187,50.07495219)(697.33251866,49.60495266)(697.41252197,49.14495699)
\curveto(697.4925185,48.70495356)(697.64251835,48.30495396)(697.86252197,47.94495699)
\curveto(698.08251791,47.60495466)(698.38251761,47.33495493)(698.76252197,47.13495699)
\curveto(699.16251683,46.93495533)(699.67251632,46.83495543)(700.29252197,46.83495699)
\curveto(700.4925155,46.83495543)(700.74251525,46.8649554)(701.04252197,46.92495699)
\curveto(701.34251465,46.98495528)(701.63251436,47.11495515)(701.91252197,47.31495699)
\curveto(702.21251378,47.51495475)(702.46251353,47.78495448)(702.66252197,48.12495699)
\curveto(702.86251313,48.4649538)(702.96251303,48.92495334)(702.96252197,49.50495699)
\curveto(702.96251303,50.04495222)(702.84251315,50.48495178)(702.60252197,50.82495699)
\curveto(702.38251361,51.18495108)(702.08251391,51.47495079)(701.70252197,51.69495699)
\curveto(701.34251465,51.93495033)(700.92251507,52.13495013)(700.44252197,52.29495699)
\curveto(699.98251601,52.45494981)(699.51251648,52.61494965)(699.03252197,52.77495699)
\curveto(698.55251744,52.93494933)(698.07251792,53.10494916)(697.59252197,53.28495699)
\curveto(697.11251888,53.48494878)(696.68251931,53.74494852)(696.30252197,54.06495699)
\curveto(695.94252005,54.38494788)(695.64252035,54.80494746)(695.40252197,55.32495699)
\curveto(695.18252081,55.84494642)(695.07252092,56.51494575)(695.07252197,57.33495699)
\curveto(695.07252092,58.07494419)(695.20252079,58.72494354)(695.46252197,59.28495699)
\curveto(695.74252025,59.84494242)(696.11251988,60.30494196)(696.57252197,60.66495699)
\curveto(697.05251894,61.04494122)(697.60251839,61.32494094)(698.22252197,61.50495699)
\curveto(698.84251715,61.68494058)(699.50251649,61.77494049)(700.20252197,61.77495699)
\curveto(701.36251463,61.77494049)(702.27251372,61.59494067)(702.93252197,61.23495699)
\curveto(703.5925124,60.87494139)(704.08251191,60.43494183)(704.40252197,59.91495699)
\curveto(704.72251127,59.39494287)(704.91251108,58.83494343)(704.97252197,58.23495699)
\curveto(705.05251094,57.65494461)(705.0925109,57.12494514)(705.09252197,56.64495699)
\lineto(702.57252197,56.64495699)
}
}
{
\newrgbcolor{curcolor}{0 0 0}
\pscustom[linestyle=none,fillstyle=solid,fillcolor=curcolor]
{
\newpath
\moveto(709.96002197,53.01495699)
\curveto(709.96001822,52.39494987)(709.97001821,51.72495054)(709.99002197,51.00495699)
\curveto(710.03001815,50.28495198)(710.15001803,49.61495265)(710.35002197,48.99495699)
\curveto(710.55001763,48.37495389)(710.85001733,47.85495441)(711.25002197,47.43495699)
\curveto(711.67001651,47.03495523)(712.27001591,46.83495543)(713.05002197,46.83495699)
\curveto(713.65001453,46.83495543)(714.13001405,46.9649553)(714.49002197,47.22495699)
\curveto(714.85001333,47.50495476)(715.12001306,47.83495443)(715.30002197,48.21495699)
\curveto(715.50001268,48.61495365)(715.63001255,49.02495324)(715.69002197,49.44495699)
\curveto(715.75001243,49.88495238)(715.7800124,50.25495201)(715.78002197,50.55495699)
\lineto(718.30002197,50.55495699)
\curveto(718.30000988,50.13495213)(718.22000996,49.59495267)(718.06002197,48.93495699)
\curveto(717.92001026,48.29495397)(717.65001053,47.6649546)(717.25002197,47.04495699)
\curveto(716.85001133,46.44495582)(716.30001188,45.92495634)(715.60002197,45.48495699)
\curveto(714.90001328,45.0649572)(714.00001418,44.85495741)(712.90002197,44.85495699)
\curveto(710.92001726,44.85495741)(709.49001869,45.53495673)(708.61002197,46.89495699)
\curveto(707.75002043,48.25495401)(707.32002086,50.32495194)(707.32002197,53.10495699)
\curveto(707.32002086,54.10494816)(707.3800208,55.11494715)(707.50002197,56.13495699)
\curveto(707.64002054,57.17494509)(707.91002027,58.10494416)(708.31002197,58.92495699)
\curveto(708.73001945,59.7649425)(709.31001887,60.44494182)(710.05002197,60.96495699)
\curveto(710.81001737,61.50494076)(711.81001637,61.77494049)(713.05002197,61.77495699)
\curveto(714.27001391,61.77494049)(715.24001294,61.53494073)(715.96002197,61.05495699)
\curveto(716.6800115,60.57494169)(717.22001096,59.95494231)(717.58002197,59.19495699)
\curveto(717.94001024,58.45494381)(718.17001001,57.62494464)(718.27002197,56.70495699)
\curveto(718.37000981,55.78494648)(718.42000976,54.89494737)(718.42002197,54.03495699)
\lineto(718.42002197,53.01495699)
\lineto(709.96002197,53.01495699)
\moveto(715.78002197,54.99495699)
\lineto(715.78002197,55.86495699)
\curveto(715.7800124,56.30494596)(715.74001244,56.75494551)(715.66002197,57.21495699)
\curveto(715.5800126,57.69494457)(715.43001275,58.12494414)(715.21002197,58.50495699)
\curveto(715.01001317,58.88494338)(714.73001345,59.19494307)(714.37002197,59.43495699)
\curveto(714.01001417,59.67494259)(713.55001463,59.79494247)(712.99002197,59.79495699)
\curveto(712.33001585,59.79494247)(711.80001638,59.61494265)(711.40002197,59.25495699)
\curveto(711.02001716,58.91494335)(710.73001745,58.51494375)(710.53002197,58.05495699)
\curveto(710.33001785,57.59494467)(710.20001798,57.12494514)(710.14002197,56.64495699)
\curveto(710.0800181,56.18494608)(710.05001813,55.83494643)(710.05002197,55.59495699)
\lineto(710.05002197,54.99495699)
\lineto(715.78002197,54.99495699)
}
}
{
\newrgbcolor{curcolor}{0 0 0}
\pscustom[linestyle=none,fillstyle=solid,fillcolor=curcolor]
{
\newpath
\moveto(721.16080322,61.35495699)
\lineto(723.68080322,61.35495699)
\lineto(723.68080322,58.95495699)
\lineto(723.74080322,58.95495699)
\curveto(723.92079881,59.33494293)(724.11079862,59.69494257)(724.31080322,60.03495699)
\curveto(724.5307982,60.37494189)(724.78079795,60.67494159)(725.06080322,60.93495699)
\curveto(725.34079739,61.19494107)(725.65079708,61.39494087)(725.99080322,61.53495699)
\curveto(726.35079638,61.69494057)(726.76079597,61.77494049)(727.22080322,61.77495699)
\curveto(727.72079501,61.77494049)(728.09079464,61.71494055)(728.33080322,61.59495699)
\lineto(728.33080322,59.13495699)
\curveto(728.21079452,59.15494311)(728.05079468,59.17494309)(727.85080322,59.19495699)
\curveto(727.67079506,59.23494303)(727.38079535,59.25494301)(726.98080322,59.25495699)
\curveto(726.66079607,59.25494301)(726.31079642,59.17494309)(725.93080322,59.01495699)
\curveto(725.55079718,58.87494339)(725.19079754,58.64494362)(724.85080322,58.32495699)
\curveto(724.5307982,58.02494424)(724.25079848,57.63494463)(724.01080322,57.15495699)
\curveto(723.79079894,56.67494559)(723.68079905,56.10494616)(723.68080322,55.44495699)
\lineto(723.68080322,45.27495699)
\lineto(721.16080322,45.27495699)
\lineto(721.16080322,61.35495699)
}
}
{
\newrgbcolor{curcolor}{0 0 0}
\pscustom[linewidth=2,linecolor=curcolor]
{
\newpath
\moveto(371.20087,1035.714278)
\lineto(371.20087,963.127212)
}
}
{
\newrgbcolor{curcolor}{0 0 0}
\pscustom[linewidth=2,linecolor=curcolor]
{
\newpath
\moveto(371.20087,913.80149)
\lineto(371.20087,830.60782)
}
}
{
\newrgbcolor{curcolor}{0 0 0}
\pscustom[linewidth=2,linecolor=curcolor]
{
\newpath
\moveto(658.61944,914.32278)
\lineto(658.973,818.42062)
\lineto(453.47181,818.42062)
}
}
{
\newrgbcolor{curcolor}{0 0 0}
\pscustom[linewidth=2,linecolor=curcolor]
{
\newpath
\moveto(658.61944,1035.751358)
\lineto(658.973,963.023232)
}
}
{
\newrgbcolor{curcolor}{0 0 0}
\pscustom[linewidth=2,linecolor=curcolor]
{
\newpath
\moveto(520.30781,741.26898)
\lineto(520.30781,791.98345)
\lineto(453.87924,791.98345)
}
}
{
\newrgbcolor{curcolor}{0 0 0}
\pscustom[linewidth=2,linecolor=curcolor]
{
\newpath
\moveto(416.7016,781.25896)
\lineto(416.7016,389.40147)
}
}
{
\newrgbcolor{curcolor}{0 0 0}
\pscustom[linewidth=2,linecolor=curcolor]
{
\newpath
\moveto(584.9142,692.62597)
\lineto(584.9142,299.25737)
}
}
{
\newrgbcolor{curcolor}{0 0 0}
\pscustom[linewidth=2,linecolor=curcolor]
{
\newpath
\moveto(701.01162,603.57137)
\lineto(701.01162,207.55107)
}
}
{
\newrgbcolor{curcolor}{0 0 0}
\pscustom[linewidth=2,linecolor=curcolor]
{
\newpath
\moveto(124.48769,570.65877)
\lineto(124.48769,704.92737)
}
}
{
\newrgbcolor{curcolor}{0 0 0}
\pscustom[linewidth=2,linecolor=curcolor]
{
\newpath
\moveto(298.99493,480.57287)
\lineto(298.99493,731.20807)
\lineto(194.7723,731.20807)
}
}
{
\newrgbcolor{curcolor}{0 0 0}
\pscustom[linewidth=2,linecolor=curcolor]
{
\newpath
\moveto(41.200886,521.60717)
\lineto(41.200886,52.45348)
\lineto(630.39729,52.45348)
}
}
{
\newrgbcolor{curcolor}{0 0 0}
\pscustom[linewidth=2,linecolor=curcolor]
{
\newpath
\moveto(129.29951,431.27967)
\lineto(129.29951,53.35628)
}
}
{
\newrgbcolor{curcolor}{0 0 0}
\pscustom[linewidth=2,linecolor=curcolor]
{
\newpath
\moveto(344.462,340.23967)
\lineto(344.462,53.48258)
}
}
{
\newrgbcolor{curcolor}{0 0 0}
\pscustom[linewidth=2,linecolor=curcolor]
{
\newpath
\moveto(459.61939,249.70477)
\lineto(459.61939,53.22998)
}
}
{
\newrgbcolor{curcolor}{0 0 0}
\pscustom[linewidth=2,linecolor=curcolor]
{
\newpath
\moveto(567.70571,158.41217)
\lineto(567.70571,52.59868)
}
}
{
\newrgbcolor{curcolor}{0 0 0}
\pscustom[linewidth=2,linecolor=curcolor]
{
\newpath
\moveto(684.23637,653.76897)
\lineto(684.23637,713.05487)
\lineto(617.8078,713.05487)
}
}
{
\newrgbcolor{curcolor}{1 1 1}
\pscustom[linestyle=none,fillstyle=solid,fillcolor=curcolor]
{
\newpath
\moveto(365.89439,971.1531)
\lineto(371.07142,979.67626)
\lineto(376.24845,971.1531)
\closepath
}
}
{
\newrgbcolor{curcolor}{0 0 0}
\pscustom[linewidth=2,linecolor=curcolor]
{
\newpath
\moveto(365.89439,971.1531)
\lineto(371.07142,979.67626)
\lineto(376.24845,971.1531)
\closepath
}
}
{
\newrgbcolor{curcolor}{1 1 1}
\pscustom[linestyle=none,fillstyle=solid,fillcolor=curcolor]
{
\newpath
\moveto(374.00668758,982.06664419)
\curveto(374.00668758,980.39006543)(372.64755196,979.0309298)(370.9709732,979.0309298)
\curveto(369.29439443,979.0309298)(367.93525881,980.39006543)(367.93525881,982.06664419)
\curveto(367.93525881,983.74322295)(369.29439443,985.10235857)(370.9709732,985.10235857)
\curveto(372.64755196,985.10235857)(374.00668758,983.74322295)(374.00668758,982.06664419)
\closepath
}
}
{
\newrgbcolor{curcolor}{0 0 0}
\pscustom[linewidth=2,linecolor=curcolor]
{
\newpath
\moveto(374.00668758,982.06664419)
\curveto(374.00668758,980.39006543)(372.64755196,979.0309298)(370.9709732,979.0309298)
\curveto(369.29439443,979.0309298)(367.93525881,980.39006543)(367.93525881,982.06664419)
\curveto(367.93525881,983.74322295)(369.29439443,985.10235857)(370.9709732,985.10235857)
\curveto(372.64755196,985.10235857)(374.00668758,983.74322295)(374.00668758,982.06664419)
\closepath
}
}
{
\newrgbcolor{curcolor}{0 0 0}
\pscustom[linewidth=2,linecolor=curcolor]
{
\newpath
\moveto(374.45804,971.0845)
\lineto(374.45804,963.76308)
}
}
{
\newrgbcolor{curcolor}{0 0 0}
\pscustom[linewidth=2,linecolor=curcolor]
{
\newpath
\moveto(367.66251,971.05835)
\lineto(367.66251,963.46907)
}
}
{
\newrgbcolor{curcolor}{1 1 1}
\pscustom[linestyle=none,fillstyle=solid,fillcolor=curcolor]
{
\newpath
\moveto(376.24845,905.81122)
\lineto(371.07142,897.28806)
\lineto(365.89439,905.81122)
\closepath
}
}
{
\newrgbcolor{curcolor}{0 0 0}
\pscustom[linewidth=2,linecolor=curcolor]
{
\newpath
\moveto(376.24845,905.81122)
\lineto(371.07142,897.28806)
\lineto(365.89439,905.81122)
\closepath
}
}
{
\newrgbcolor{curcolor}{1 1 1}
\pscustom[linestyle=none,fillstyle=solid,fillcolor=curcolor]
{
\newpath
\moveto(368.13615242,894.89767581)
\curveto(368.13615242,896.57425457)(369.49528804,897.9333902)(371.1718668,897.9333902)
\curveto(372.84844557,897.9333902)(374.20758119,896.57425457)(374.20758119,894.89767581)
\curveto(374.20758119,893.22109705)(372.84844557,891.86196143)(371.1718668,891.86196143)
\curveto(369.49528804,891.86196143)(368.13615242,893.22109705)(368.13615242,894.89767581)
\closepath
}
}
{
\newrgbcolor{curcolor}{0 0 0}
\pscustom[linewidth=2,linecolor=curcolor]
{
\newpath
\moveto(368.13615242,894.89767581)
\curveto(368.13615242,896.57425457)(369.49528804,897.9333902)(371.1718668,897.9333902)
\curveto(372.84844557,897.9333902)(374.20758119,896.57425457)(374.20758119,894.89767581)
\curveto(374.20758119,893.22109705)(372.84844557,891.86196143)(371.1718668,891.86196143)
\curveto(369.49528804,891.86196143)(368.13615242,893.22109705)(368.13615242,894.89767581)
\closepath
}
}
{
\newrgbcolor{curcolor}{0 0 0}
\pscustom[linewidth=2,linecolor=curcolor]
{
\newpath
\moveto(367.6848,905.87982)
\lineto(367.6848,913.20124)
}
}
{
\newrgbcolor{curcolor}{0 0 0}
\pscustom[linewidth=2,linecolor=curcolor]
{
\newpath
\moveto(374.48033,905.90597)
\lineto(374.48033,913.49525)
}
}
{
\newrgbcolor{curcolor}{1 1 1}
\pscustom[linestyle=none,fillstyle=solid,fillcolor=curcolor]
{
\newpath
\moveto(653.57296,971.1531)
\lineto(658.74999,979.67626)
\lineto(663.92702,971.1531)
\closepath
}
}
{
\newrgbcolor{curcolor}{0 0 0}
\pscustom[linewidth=2,linecolor=curcolor]
{
\newpath
\moveto(653.57296,971.1531)
\lineto(658.74999,979.67626)
\lineto(663.92702,971.1531)
\closepath
}
}
{
\newrgbcolor{curcolor}{1 1 1}
\pscustom[linestyle=none,fillstyle=solid,fillcolor=curcolor]
{
\newpath
\moveto(661.68525758,982.06664419)
\curveto(661.68525758,980.39006543)(660.32612196,979.0309298)(658.6495432,979.0309298)
\curveto(656.97296443,979.0309298)(655.61382881,980.39006543)(655.61382881,982.06664419)
\curveto(655.61382881,983.74322295)(656.97296443,985.10235857)(658.6495432,985.10235857)
\curveto(660.32612196,985.10235857)(661.68525758,983.74322295)(661.68525758,982.06664419)
\closepath
}
}
{
\newrgbcolor{curcolor}{0 0 0}
\pscustom[linewidth=2,linecolor=curcolor]
{
\newpath
\moveto(661.68525758,982.06664419)
\curveto(661.68525758,980.39006543)(660.32612196,979.0309298)(658.6495432,979.0309298)
\curveto(656.97296443,979.0309298)(655.61382881,980.39006543)(655.61382881,982.06664419)
\curveto(655.61382881,983.74322295)(656.97296443,985.10235857)(658.6495432,985.10235857)
\curveto(660.32612196,985.10235857)(661.68525758,983.74322295)(661.68525758,982.06664419)
\closepath
}
}
{
\newrgbcolor{curcolor}{0 0 0}
\pscustom[linewidth=2,linecolor=curcolor]
{
\newpath
\moveto(662.13661,971.0845)
\lineto(662.13661,963.76308)
}
}
{
\newrgbcolor{curcolor}{0 0 0}
\pscustom[linewidth=2,linecolor=curcolor]
{
\newpath
\moveto(655.34108,971.05835)
\lineto(655.34108,963.46907)
}
}
{
\newrgbcolor{curcolor}{1 1 1}
\pscustom[linestyle=none,fillstyle=solid,fillcolor=curcolor]
{
\newpath
\moveto(663.92702,905.81122)
\lineto(658.74999,897.28806)
\lineto(653.57296,905.81122)
\closepath
}
}
{
\newrgbcolor{curcolor}{0 0 0}
\pscustom[linewidth=2,linecolor=curcolor]
{
\newpath
\moveto(663.92702,905.81122)
\lineto(658.74999,897.28806)
\lineto(653.57296,905.81122)
\closepath
}
}
{
\newrgbcolor{curcolor}{1 1 1}
\pscustom[linestyle=none,fillstyle=solid,fillcolor=curcolor]
{
\newpath
\moveto(655.81472242,894.89767581)
\curveto(655.81472242,896.57425457)(657.17385804,897.9333902)(658.8504368,897.9333902)
\curveto(660.52701557,897.9333902)(661.88615119,896.57425457)(661.88615119,894.89767581)
\curveto(661.88615119,893.22109705)(660.52701557,891.86196143)(658.8504368,891.86196143)
\curveto(657.17385804,891.86196143)(655.81472242,893.22109705)(655.81472242,894.89767581)
\closepath
}
}
{
\newrgbcolor{curcolor}{0 0 0}
\pscustom[linewidth=2,linecolor=curcolor]
{
\newpath
\moveto(655.81472242,894.89767581)
\curveto(655.81472242,896.57425457)(657.17385804,897.9333902)(658.8504368,897.9333902)
\curveto(660.52701557,897.9333902)(661.88615119,896.57425457)(661.88615119,894.89767581)
\curveto(661.88615119,893.22109705)(660.52701557,891.86196143)(658.8504368,891.86196143)
\curveto(657.17385804,891.86196143)(655.81472242,893.22109705)(655.81472242,894.89767581)
\closepath
}
}
{
\newrgbcolor{curcolor}{0 0 0}
\pscustom[linewidth=2,linecolor=curcolor]
{
\newpath
\moveto(655.36337,905.87982)
\lineto(655.36337,913.20124)
}
}
{
\newrgbcolor{curcolor}{0 0 0}
\pscustom[linewidth=2,linecolor=curcolor]
{
\newpath
\moveto(662.1589,905.90597)
\lineto(662.1589,913.49525)
}
}
{
\newrgbcolor{curcolor}{1 1 1}
\pscustom[linestyle=none,fillstyle=solid,fillcolor=curcolor]
{
\newpath
\moveto(515.1778,749.30475)
\lineto(520.35483,757.82791)
\lineto(525.53186,749.30475)
\closepath
}
}
{
\newrgbcolor{curcolor}{0 0 0}
\pscustom[linewidth=2,linecolor=curcolor]
{
\newpath
\moveto(515.1778,749.30475)
\lineto(520.35483,757.82791)
\lineto(525.53186,749.30475)
\closepath
}
}
{
\newrgbcolor{curcolor}{1 1 1}
\pscustom[linestyle=none,fillstyle=solid,fillcolor=curcolor]
{
\newpath
\moveto(523.29009758,760.21829419)
\curveto(523.29009758,758.54171543)(521.93096196,757.1825798)(520.2543832,757.1825798)
\curveto(518.57780443,757.1825798)(517.21866881,758.54171543)(517.21866881,760.21829419)
\curveto(517.21866881,761.89487295)(518.57780443,763.25400857)(520.2543832,763.25400857)
\curveto(521.93096196,763.25400857)(523.29009758,761.89487295)(523.29009758,760.21829419)
\closepath
}
}
{
\newrgbcolor{curcolor}{0 0 0}
\pscustom[linewidth=2,linecolor=curcolor]
{
\newpath
\moveto(523.29009758,760.21829419)
\curveto(523.29009758,758.54171543)(521.93096196,757.1825798)(520.2543832,757.1825798)
\curveto(518.57780443,757.1825798)(517.21866881,758.54171543)(517.21866881,760.21829419)
\curveto(517.21866881,761.89487295)(518.57780443,763.25400857)(520.2543832,763.25400857)
\curveto(521.93096196,763.25400857)(523.29009758,761.89487295)(523.29009758,760.21829419)
\closepath
}
}
{
\newrgbcolor{curcolor}{0 0 0}
\pscustom[linewidth=2,linecolor=curcolor]
{
\newpath
\moveto(523.74145,749.23615)
\lineto(523.74145,741.91473)
}
}
{
\newrgbcolor{curcolor}{0 0 0}
\pscustom[linewidth=2,linecolor=curcolor]
{
\newpath
\moveto(516.94592,749.21)
\lineto(516.94592,741.62072)
}
}
{
\newrgbcolor{curcolor}{1 1 1}
\pscustom[linestyle=none,fillstyle=solid,fillcolor=curcolor]
{
\newpath
\moveto(515.1778,749.30475)
\lineto(520.35483,757.82791)
\lineto(525.53186,749.30475)
\closepath
}
}
{
\newrgbcolor{curcolor}{0 0 0}
\pscustom[linewidth=2,linecolor=curcolor]
{
\newpath
\moveto(515.1778,749.30475)
\lineto(520.35483,757.82791)
\lineto(525.53186,749.30475)
\closepath
}
}
{
\newrgbcolor{curcolor}{1 1 1}
\pscustom[linestyle=none,fillstyle=solid,fillcolor=curcolor]
{
\newpath
\moveto(523.29009758,760.21829419)
\curveto(523.29009758,758.54171543)(521.93096196,757.1825798)(520.2543832,757.1825798)
\curveto(518.57780443,757.1825798)(517.21866881,758.54171543)(517.21866881,760.21829419)
\curveto(517.21866881,761.89487295)(518.57780443,763.25400857)(520.2543832,763.25400857)
\curveto(521.93096196,763.25400857)(523.29009758,761.89487295)(523.29009758,760.21829419)
\closepath
}
}
{
\newrgbcolor{curcolor}{0 0 0}
\pscustom[linewidth=2,linecolor=curcolor]
{
\newpath
\moveto(523.29009758,760.21829419)
\curveto(523.29009758,758.54171543)(521.93096196,757.1825798)(520.2543832,757.1825798)
\curveto(518.57780443,757.1825798)(517.21866881,758.54171543)(517.21866881,760.21829419)
\curveto(517.21866881,761.89487295)(518.57780443,763.25400857)(520.2543832,763.25400857)
\curveto(521.93096196,763.25400857)(523.29009758,761.89487295)(523.29009758,760.21829419)
\closepath
}
}
{
\newrgbcolor{curcolor}{0 0 0}
\pscustom[linewidth=2,linecolor=curcolor]
{
\newpath
\moveto(523.74145,749.23615)
\lineto(523.74145,741.91473)
}
}
{
\newrgbcolor{curcolor}{0 0 0}
\pscustom[linewidth=2,linecolor=curcolor]
{
\newpath
\moveto(516.94592,749.21)
\lineto(516.94592,741.62072)
}
}
{
\newrgbcolor{curcolor}{1 1 1}
\pscustom[linestyle=none,fillstyle=solid,fillcolor=curcolor]
{
\newpath
\moveto(679.07511,660.28507)
\lineto(684.25214,668.80823)
\lineto(689.42917,660.28507)
\closepath
}
}
{
\newrgbcolor{curcolor}{0 0 0}
\pscustom[linewidth=2,linecolor=curcolor]
{
\newpath
\moveto(679.07511,660.28507)
\lineto(684.25214,668.80823)
\lineto(689.42917,660.28507)
\closepath
}
}
{
\newrgbcolor{curcolor}{1 1 1}
\pscustom[linestyle=none,fillstyle=solid,fillcolor=curcolor]
{
\newpath
\moveto(687.18740758,671.19861419)
\curveto(687.18740758,669.52203543)(685.82827196,668.1628998)(684.1516932,668.1628998)
\curveto(682.47511443,668.1628998)(681.11597881,669.52203543)(681.11597881,671.19861419)
\curveto(681.11597881,672.87519295)(682.47511443,674.23432857)(684.1516932,674.23432857)
\curveto(685.82827196,674.23432857)(687.18740758,672.87519295)(687.18740758,671.19861419)
\closepath
}
}
{
\newrgbcolor{curcolor}{0 0 0}
\pscustom[linewidth=2,linecolor=curcolor]
{
\newpath
\moveto(687.18740758,671.19861419)
\curveto(687.18740758,669.52203543)(685.82827196,668.1628998)(684.1516932,668.1628998)
\curveto(682.47511443,668.1628998)(681.11597881,669.52203543)(681.11597881,671.19861419)
\curveto(681.11597881,672.87519295)(682.47511443,674.23432857)(684.1516932,674.23432857)
\curveto(685.82827196,674.23432857)(687.18740758,672.87519295)(687.18740758,671.19861419)
\closepath
}
}
{
\newrgbcolor{curcolor}{0 0 0}
\pscustom[linewidth=2,linecolor=curcolor]
{
\newpath
\moveto(687.63876,660.21647)
\lineto(687.63876,652.89505)
}
}
{
\newrgbcolor{curcolor}{0 0 0}
\pscustom[linewidth=2,linecolor=curcolor]
{
\newpath
\moveto(680.84323,660.19032)
\lineto(680.84323,652.60104)
}
}
{
\newrgbcolor{curcolor}{1 1 1}
\pscustom[linestyle=none,fillstyle=solid,fillcolor=curcolor]
{
\newpath
\moveto(119.27828,579.30961)
\lineto(124.45531,587.83277)
\lineto(129.63234,579.30961)
\closepath
}
}
{
\newrgbcolor{curcolor}{0 0 0}
\pscustom[linewidth=2,linecolor=curcolor]
{
\newpath
\moveto(119.27828,579.30961)
\lineto(124.45531,587.83277)
\lineto(129.63234,579.30961)
\closepath
}
}
{
\newrgbcolor{curcolor}{1 1 1}
\pscustom[linestyle=none,fillstyle=solid,fillcolor=curcolor]
{
\newpath
\moveto(127.39057758,590.22315419)
\curveto(127.39057758,588.54657543)(126.03144196,587.1874398)(124.3548632,587.1874398)
\curveto(122.67828443,587.1874398)(121.31914881,588.54657543)(121.31914881,590.22315419)
\curveto(121.31914881,591.89973295)(122.67828443,593.25886857)(124.3548632,593.25886857)
\curveto(126.03144196,593.25886857)(127.39057758,591.89973295)(127.39057758,590.22315419)
\closepath
}
}
{
\newrgbcolor{curcolor}{0 0 0}
\pscustom[linewidth=2,linecolor=curcolor]
{
\newpath
\moveto(127.39057758,590.22315419)
\curveto(127.39057758,588.54657543)(126.03144196,587.1874398)(124.3548632,587.1874398)
\curveto(122.67828443,587.1874398)(121.31914881,588.54657543)(121.31914881,590.22315419)
\curveto(121.31914881,591.89973295)(122.67828443,593.25886857)(124.3548632,593.25886857)
\curveto(126.03144196,593.25886857)(127.39057758,591.89973295)(127.39057758,590.22315419)
\closepath
}
}
{
\newrgbcolor{curcolor}{0 0 0}
\pscustom[linewidth=2,linecolor=curcolor]
{
\newpath
\moveto(127.84193,579.24101)
\lineto(127.84193,571.91959)
}
}
{
\newrgbcolor{curcolor}{0 0 0}
\pscustom[linewidth=2,linecolor=curcolor]
{
\newpath
\moveto(121.0464,579.21486)
\lineto(121.0464,571.62558)
}
}
{
\newrgbcolor{curcolor}{1 1 1}
\pscustom[linestyle=none,fillstyle=solid,fillcolor=curcolor]
{
\newpath
\moveto(293.82812,487.92777)
\lineto(299.00515,496.45093)
\lineto(304.18218,487.92777)
\closepath
}
}
{
\newrgbcolor{curcolor}{0 0 0}
\pscustom[linewidth=2,linecolor=curcolor]
{
\newpath
\moveto(293.82812,487.92777)
\lineto(299.00515,496.45093)
\lineto(304.18218,487.92777)
\closepath
}
}
{
\newrgbcolor{curcolor}{1 1 1}
\pscustom[linestyle=none,fillstyle=solid,fillcolor=curcolor]
{
\newpath
\moveto(301.94041758,498.84131419)
\curveto(301.94041758,497.16473543)(300.58128196,495.8055998)(298.9047032,495.8055998)
\curveto(297.22812443,495.8055998)(295.86898881,497.16473543)(295.86898881,498.84131419)
\curveto(295.86898881,500.51789295)(297.22812443,501.87702857)(298.9047032,501.87702857)
\curveto(300.58128196,501.87702857)(301.94041758,500.51789295)(301.94041758,498.84131419)
\closepath
}
}
{
\newrgbcolor{curcolor}{0 0 0}
\pscustom[linewidth=2,linecolor=curcolor]
{
\newpath
\moveto(301.94041758,498.84131419)
\curveto(301.94041758,497.16473543)(300.58128196,495.8055998)(298.9047032,495.8055998)
\curveto(297.22812443,495.8055998)(295.86898881,497.16473543)(295.86898881,498.84131419)
\curveto(295.86898881,500.51789295)(297.22812443,501.87702857)(298.9047032,501.87702857)
\curveto(300.58128196,501.87702857)(301.94041758,500.51789295)(301.94041758,498.84131419)
\closepath
}
}
{
\newrgbcolor{curcolor}{0 0 0}
\pscustom[linewidth=2,linecolor=curcolor]
{
\newpath
\moveto(302.39177,487.85917)
\lineto(302.39177,480.53775)
}
}
{
\newrgbcolor{curcolor}{0 0 0}
\pscustom[linewidth=2,linecolor=curcolor]
{
\newpath
\moveto(295.59624,487.83302)
\lineto(295.59624,480.24374)
}
}
{
\newrgbcolor{curcolor}{1 1 1}
\pscustom[linestyle=none,fillstyle=solid,fillcolor=curcolor]
{
\newpath
\moveto(411.51938,397.04596)
\lineto(416.69641,405.56912)
\lineto(421.87344,397.04596)
\closepath
}
}
{
\newrgbcolor{curcolor}{0 0 0}
\pscustom[linewidth=2,linecolor=curcolor]
{
\newpath
\moveto(411.51938,397.04596)
\lineto(416.69641,405.56912)
\lineto(421.87344,397.04596)
\closepath
}
}
{
\newrgbcolor{curcolor}{1 1 1}
\pscustom[linestyle=none,fillstyle=solid,fillcolor=curcolor]
{
\newpath
\moveto(419.63167758,407.95950419)
\curveto(419.63167758,406.28292543)(418.27254196,404.9237898)(416.5959632,404.9237898)
\curveto(414.91938443,404.9237898)(413.56024881,406.28292543)(413.56024881,407.95950419)
\curveto(413.56024881,409.63608295)(414.91938443,410.99521857)(416.5959632,410.99521857)
\curveto(418.27254196,410.99521857)(419.63167758,409.63608295)(419.63167758,407.95950419)
\closepath
}
}
{
\newrgbcolor{curcolor}{0 0 0}
\pscustom[linewidth=2,linecolor=curcolor]
{
\newpath
\moveto(419.63167758,407.95950419)
\curveto(419.63167758,406.28292543)(418.27254196,404.9237898)(416.5959632,404.9237898)
\curveto(414.91938443,404.9237898)(413.56024881,406.28292543)(413.56024881,407.95950419)
\curveto(413.56024881,409.63608295)(414.91938443,410.99521857)(416.5959632,410.99521857)
\curveto(418.27254196,410.99521857)(419.63167758,409.63608295)(419.63167758,407.95950419)
\closepath
}
}
{
\newrgbcolor{curcolor}{0 0 0}
\pscustom[linewidth=2,linecolor=curcolor]
{
\newpath
\moveto(420.08303,396.97736)
\lineto(420.08303,389.65594)
}
}
{
\newrgbcolor{curcolor}{0 0 0}
\pscustom[linewidth=2,linecolor=curcolor]
{
\newpath
\moveto(413.2875,396.95121)
\lineto(413.2875,389.36193)
}
}
{
\newrgbcolor{curcolor}{1 1 1}
\pscustom[linestyle=none,fillstyle=solid,fillcolor=curcolor]
{
\newpath
\moveto(579.73368,306.15311)
\lineto(584.91071,314.67627)
\lineto(590.08774,306.15311)
\closepath
}
}
{
\newrgbcolor{curcolor}{0 0 0}
\pscustom[linewidth=2,linecolor=curcolor]
{
\newpath
\moveto(579.73368,306.15311)
\lineto(584.91071,314.67627)
\lineto(590.08774,306.15311)
\closepath
}
}
{
\newrgbcolor{curcolor}{1 1 1}
\pscustom[linestyle=none,fillstyle=solid,fillcolor=curcolor]
{
\newpath
\moveto(587.84597758,317.06665419)
\curveto(587.84597758,315.39007543)(586.48684196,314.0309398)(584.8102632,314.0309398)
\curveto(583.13368443,314.0309398)(581.77454881,315.39007543)(581.77454881,317.06665419)
\curveto(581.77454881,318.74323295)(583.13368443,320.10236857)(584.8102632,320.10236857)
\curveto(586.48684196,320.10236857)(587.84597758,318.74323295)(587.84597758,317.06665419)
\closepath
}
}
{
\newrgbcolor{curcolor}{0 0 0}
\pscustom[linewidth=2,linecolor=curcolor]
{
\newpath
\moveto(587.84597758,317.06665419)
\curveto(587.84597758,315.39007543)(586.48684196,314.0309398)(584.8102632,314.0309398)
\curveto(583.13368443,314.0309398)(581.77454881,315.39007543)(581.77454881,317.06665419)
\curveto(581.77454881,318.74323295)(583.13368443,320.10236857)(584.8102632,320.10236857)
\curveto(586.48684196,320.10236857)(587.84597758,318.74323295)(587.84597758,317.06665419)
\closepath
}
}
{
\newrgbcolor{curcolor}{0 0 0}
\pscustom[linewidth=2,linecolor=curcolor]
{
\newpath
\moveto(588.29733,306.08451)
\lineto(588.29733,298.76309)
}
}
{
\newrgbcolor{curcolor}{0 0 0}
\pscustom[linewidth=2,linecolor=curcolor]
{
\newpath
\moveto(581.5018,306.05836)
\lineto(581.5018,298.46908)
}
}
{
\newrgbcolor{curcolor}{1 1 1}
\pscustom[linestyle=none,fillstyle=solid,fillcolor=curcolor]
{
\newpath
\moveto(695.86881,215.43915)
\lineto(701.04584,223.96231)
\lineto(706.22287,215.43915)
\closepath
}
}
{
\newrgbcolor{curcolor}{0 0 0}
\pscustom[linewidth=2,linecolor=curcolor]
{
\newpath
\moveto(695.86881,215.43915)
\lineto(701.04584,223.96231)
\lineto(706.22287,215.43915)
\closepath
}
}
{
\newrgbcolor{curcolor}{1 1 1}
\pscustom[linestyle=none,fillstyle=solid,fillcolor=curcolor]
{
\newpath
\moveto(703.98110758,226.35269419)
\curveto(703.98110758,224.67611543)(702.62197196,223.3169798)(700.9453932,223.3169798)
\curveto(699.26881443,223.3169798)(697.90967881,224.67611543)(697.90967881,226.35269419)
\curveto(697.90967881,228.02927295)(699.26881443,229.38840857)(700.9453932,229.38840857)
\curveto(702.62197196,229.38840857)(703.98110758,228.02927295)(703.98110758,226.35269419)
\closepath
}
}
{
\newrgbcolor{curcolor}{0 0 0}
\pscustom[linewidth=2,linecolor=curcolor]
{
\newpath
\moveto(703.98110758,226.35269419)
\curveto(703.98110758,224.67611543)(702.62197196,223.3169798)(700.9453932,223.3169798)
\curveto(699.26881443,223.3169798)(697.90967881,224.67611543)(697.90967881,226.35269419)
\curveto(697.90967881,228.02927295)(699.26881443,229.38840857)(700.9453932,229.38840857)
\curveto(702.62197196,229.38840857)(703.98110758,228.02927295)(703.98110758,226.35269419)
\closepath
}
}
{
\newrgbcolor{curcolor}{0 0 0}
\pscustom[linewidth=2,linecolor=curcolor]
{
\newpath
\moveto(704.43246,215.37055)
\lineto(704.43246,208.04913)
}
}
{
\newrgbcolor{curcolor}{0 0 0}
\pscustom[linewidth=2,linecolor=curcolor]
{
\newpath
\moveto(697.63693,215.3444)
\lineto(697.63693,207.75512)
}
}
{
\newrgbcolor{curcolor}{1 1 1}
\pscustom[linestyle=none,fillstyle=solid,fillcolor=curcolor]
{
\newpath
\moveto(46.33775,514.20402)
\lineto(41.16072,505.68086)
\lineto(35.98369,514.20402)
\closepath
}
}
{
\newrgbcolor{curcolor}{0 0 0}
\pscustom[linewidth=2,linecolor=curcolor]
{
\newpath
\moveto(46.33775,514.20402)
\lineto(41.16072,505.68086)
\lineto(35.98369,514.20402)
\closepath
}
}
{
\newrgbcolor{curcolor}{1 1 1}
\pscustom[linestyle=none,fillstyle=solid,fillcolor=curcolor]
{
\newpath
\moveto(38.22545242,503.29047581)
\curveto(38.22545242,504.96705457)(39.58458804,506.3261902)(41.2611668,506.3261902)
\curveto(42.93774557,506.3261902)(44.29688119,504.96705457)(44.29688119,503.29047581)
\curveto(44.29688119,501.61389705)(42.93774557,500.25476143)(41.2611668,500.25476143)
\curveto(39.58458804,500.25476143)(38.22545242,501.61389705)(38.22545242,503.29047581)
\closepath
}
}
{
\newrgbcolor{curcolor}{0 0 0}
\pscustom[linewidth=2,linecolor=curcolor]
{
\newpath
\moveto(38.22545242,503.29047581)
\curveto(38.22545242,504.96705457)(39.58458804,506.3261902)(41.2611668,506.3261902)
\curveto(42.93774557,506.3261902)(44.29688119,504.96705457)(44.29688119,503.29047581)
\curveto(44.29688119,501.61389705)(42.93774557,500.25476143)(41.2611668,500.25476143)
\curveto(39.58458804,500.25476143)(38.22545242,501.61389705)(38.22545242,503.29047581)
\closepath
}
}
{
\newrgbcolor{curcolor}{0 0 0}
\pscustom[linewidth=2,linecolor=curcolor]
{
\newpath
\moveto(37.7741,514.27262)
\lineto(37.7741,521.59404)
}
}
{
\newrgbcolor{curcolor}{0 0 0}
\pscustom[linewidth=2,linecolor=curcolor]
{
\newpath
\moveto(44.56963,514.29877)
\lineto(44.56963,521.88805)
}
}
{
\newrgbcolor{curcolor}{1 1 1}
\pscustom[linestyle=none,fillstyle=solid,fillcolor=curcolor]
{
\newpath
\moveto(134.37346,423.57912)
\lineto(129.19643,415.05596)
\lineto(124.0194,423.57912)
\closepath
}
}
{
\newrgbcolor{curcolor}{0 0 0}
\pscustom[linewidth=2,linecolor=curcolor]
{
\newpath
\moveto(134.37346,423.57912)
\lineto(129.19643,415.05596)
\lineto(124.0194,423.57912)
\closepath
}
}
{
\newrgbcolor{curcolor}{1 1 1}
\pscustom[linestyle=none,fillstyle=solid,fillcolor=curcolor]
{
\newpath
\moveto(126.26116242,412.66557581)
\curveto(126.26116242,414.34215457)(127.62029804,415.7012902)(129.2968768,415.7012902)
\curveto(130.97345557,415.7012902)(132.33259119,414.34215457)(132.33259119,412.66557581)
\curveto(132.33259119,410.98899705)(130.97345557,409.62986143)(129.2968768,409.62986143)
\curveto(127.62029804,409.62986143)(126.26116242,410.98899705)(126.26116242,412.66557581)
\closepath
}
}
{
\newrgbcolor{curcolor}{0 0 0}
\pscustom[linewidth=2,linecolor=curcolor]
{
\newpath
\moveto(126.26116242,412.66557581)
\curveto(126.26116242,414.34215457)(127.62029804,415.7012902)(129.2968768,415.7012902)
\curveto(130.97345557,415.7012902)(132.33259119,414.34215457)(132.33259119,412.66557581)
\curveto(132.33259119,410.98899705)(130.97345557,409.62986143)(129.2968768,409.62986143)
\curveto(127.62029804,409.62986143)(126.26116242,410.98899705)(126.26116242,412.66557581)
\closepath
}
}
{
\newrgbcolor{curcolor}{0 0 0}
\pscustom[linewidth=2,linecolor=curcolor]
{
\newpath
\moveto(125.80981,423.64772)
\lineto(125.80981,430.96914)
}
}
{
\newrgbcolor{curcolor}{0 0 0}
\pscustom[linewidth=2,linecolor=curcolor]
{
\newpath
\moveto(132.60534,423.67387)
\lineto(132.60534,431.26315)
}
}
{
\newrgbcolor{curcolor}{1 1 1}
\pscustom[linestyle=none,fillstyle=solid,fillcolor=curcolor]
{
\newpath
\moveto(349.70218,332.50552)
\lineto(344.52515,323.98236)
\lineto(339.34812,332.50552)
\closepath
}
}
{
\newrgbcolor{curcolor}{0 0 0}
\pscustom[linewidth=2,linecolor=curcolor]
{
\newpath
\moveto(349.70218,332.50552)
\lineto(344.52515,323.98236)
\lineto(339.34812,332.50552)
\closepath
}
}
{
\newrgbcolor{curcolor}{1 1 1}
\pscustom[linestyle=none,fillstyle=solid,fillcolor=curcolor]
{
\newpath
\moveto(341.58988242,321.59197581)
\curveto(341.58988242,323.26855457)(342.94901804,324.6276902)(344.6255968,324.6276902)
\curveto(346.30217557,324.6276902)(347.66131119,323.26855457)(347.66131119,321.59197581)
\curveto(347.66131119,319.91539705)(346.30217557,318.55626143)(344.6255968,318.55626143)
\curveto(342.94901804,318.55626143)(341.58988242,319.91539705)(341.58988242,321.59197581)
\closepath
}
}
{
\newrgbcolor{curcolor}{0 0 0}
\pscustom[linewidth=2,linecolor=curcolor]
{
\newpath
\moveto(341.58988242,321.59197581)
\curveto(341.58988242,323.26855457)(342.94901804,324.6276902)(344.6255968,324.6276902)
\curveto(346.30217557,324.6276902)(347.66131119,323.26855457)(347.66131119,321.59197581)
\curveto(347.66131119,319.91539705)(346.30217557,318.55626143)(344.6255968,318.55626143)
\curveto(342.94901804,318.55626143)(341.58988242,319.91539705)(341.58988242,321.59197581)
\closepath
}
}
{
\newrgbcolor{curcolor}{0 0 0}
\pscustom[linewidth=2,linecolor=curcolor]
{
\newpath
\moveto(341.13853,332.57412)
\lineto(341.13853,339.89554)
}
}
{
\newrgbcolor{curcolor}{0 0 0}
\pscustom[linewidth=2,linecolor=curcolor]
{
\newpath
\moveto(347.93406,332.60027)
\lineto(347.93406,340.18955)
}
}
{
\newrgbcolor{curcolor}{1 1 1}
\pscustom[linestyle=none,fillstyle=solid,fillcolor=curcolor]
{
\newpath
\moveto(464.73059,242.23972)
\lineto(459.55356,233.71656)
\lineto(454.37653,242.23972)
\closepath
}
}
{
\newrgbcolor{curcolor}{0 0 0}
\pscustom[linewidth=2,linecolor=curcolor]
{
\newpath
\moveto(464.73059,242.23972)
\lineto(459.55356,233.71656)
\lineto(454.37653,242.23972)
\closepath
}
}
{
\newrgbcolor{curcolor}{1 1 1}
\pscustom[linestyle=none,fillstyle=solid,fillcolor=curcolor]
{
\newpath
\moveto(456.61829242,231.32617581)
\curveto(456.61829242,233.00275457)(457.97742804,234.3618902)(459.6540068,234.3618902)
\curveto(461.33058557,234.3618902)(462.68972119,233.00275457)(462.68972119,231.32617581)
\curveto(462.68972119,229.64959705)(461.33058557,228.29046143)(459.6540068,228.29046143)
\curveto(457.97742804,228.29046143)(456.61829242,229.64959705)(456.61829242,231.32617581)
\closepath
}
}
{
\newrgbcolor{curcolor}{0 0 0}
\pscustom[linewidth=2,linecolor=curcolor]
{
\newpath
\moveto(456.61829242,231.32617581)
\curveto(456.61829242,233.00275457)(457.97742804,234.3618902)(459.6540068,234.3618902)
\curveto(461.33058557,234.3618902)(462.68972119,233.00275457)(462.68972119,231.32617581)
\curveto(462.68972119,229.64959705)(461.33058557,228.29046143)(459.6540068,228.29046143)
\curveto(457.97742804,228.29046143)(456.61829242,229.64959705)(456.61829242,231.32617581)
\closepath
}
}
{
\newrgbcolor{curcolor}{0 0 0}
\pscustom[linewidth=2,linecolor=curcolor]
{
\newpath
\moveto(456.16694,242.30832)
\lineto(456.16694,249.62974)
}
}
{
\newrgbcolor{curcolor}{0 0 0}
\pscustom[linewidth=2,linecolor=curcolor]
{
\newpath
\moveto(462.96247,242.33447)
\lineto(462.96247,249.92375)
}
}
{
\newrgbcolor{curcolor}{1 1 1}
\pscustom[linestyle=none,fillstyle=solid,fillcolor=curcolor]
{
\newpath
\moveto(572.76632,151.25762)
\lineto(567.58929,142.73446)
\lineto(562.41226,151.25762)
\closepath
}
}
{
\newrgbcolor{curcolor}{0 0 0}
\pscustom[linewidth=2,linecolor=curcolor]
{
\newpath
\moveto(572.76632,151.25762)
\lineto(567.58929,142.73446)
\lineto(562.41226,151.25762)
\closepath
}
}
{
\newrgbcolor{curcolor}{1 1 1}
\pscustom[linestyle=none,fillstyle=solid,fillcolor=curcolor]
{
\newpath
\moveto(564.65402242,140.34403581)
\curveto(564.65402242,142.02061457)(566.01315804,143.3797502)(567.6897368,143.3797502)
\curveto(569.36631557,143.3797502)(570.72545119,142.02061457)(570.72545119,140.34403581)
\curveto(570.72545119,138.66745705)(569.36631557,137.30832143)(567.6897368,137.30832143)
\curveto(566.01315804,137.30832143)(564.65402242,138.66745705)(564.65402242,140.34403581)
\closepath
}
}
{
\newrgbcolor{curcolor}{0 0 0}
\pscustom[linewidth=2,linecolor=curcolor]
{
\newpath
\moveto(564.65402242,140.34403581)
\curveto(564.65402242,142.02061457)(566.01315804,143.3797502)(567.6897368,143.3797502)
\curveto(569.36631557,143.3797502)(570.72545119,142.02061457)(570.72545119,140.34403581)
\curveto(570.72545119,138.66745705)(569.36631557,137.30832143)(567.6897368,137.30832143)
\curveto(566.01315804,137.30832143)(564.65402242,138.66745705)(564.65402242,140.34403581)
\closepath
}
}
{
\newrgbcolor{curcolor}{0 0 0}
\pscustom[linewidth=2,linecolor=curcolor]
{
\newpath
\moveto(564.20267,151.32622)
\lineto(564.20267,158.64764)
}
}
{
\newrgbcolor{curcolor}{0 0 0}
\pscustom[linewidth=2,linecolor=curcolor]
{
\newpath
\moveto(570.9982,151.35237)
\lineto(570.9982,158.94165)
}
}
{
\newrgbcolor{curcolor}{0 0 0}
\pscustom[linewidth=2,linecolor=curcolor]
{
\newpath
\moveto(623.89547,47.34934)
\lineto(623.89547,57.70652)
}
}
{
\newrgbcolor{curcolor}{0 0 0}
\pscustom[linewidth=2,linecolor=curcolor]
{
\newpath
\moveto(203.39286,726.07142)
\lineto(203.39286,736.42857)
}
}
{
\newrgbcolor{curcolor}{0 0 0}
\pscustom[linewidth=2,linecolor=curcolor]
{
\newpath
\moveto(119.10715,694.82144)
\lineto(129.4643,694.82144)
}
}
{
\newrgbcolor{curcolor}{0 0 0}
\pscustom[linewidth=2,linecolor=curcolor]
{
\newpath
\moveto(411.51787,773.21429)
\lineto(421.87502,773.21429)
}
}
{
\newrgbcolor{curcolor}{0 0 0}
\pscustom[linewidth=2,linecolor=curcolor]
{
\newpath
\moveto(579.67155,684.98477)
\lineto(590.0287,684.98477)
}
}
{
\newrgbcolor{curcolor}{0 0 0}
\pscustom[linewidth=2,linecolor=curcolor]
{
\newpath
\moveto(695.71283,595.33365)
\lineto(706.06998,595.33365)
}
}
{
\newrgbcolor{curcolor}{0 0 0}
\pscustom[linewidth=2,linecolor=curcolor]
{
\newpath
\moveto(462.27106,813.29752)
\lineto(462.27106,823.65467)
}
}
{
\newrgbcolor{curcolor}{0 0 0}
\pscustom[linewidth=2,linecolor=curcolor]
{
\newpath
\moveto(144.51341,1027.325133)
\lineto(154.87056,1027.325133)
}
}
{
\newrgbcolor{curcolor}{0 0 0}
\pscustom[linewidth=2,linecolor=curcolor]
{
\newpath
\moveto(365.942,1027.325133)
\lineto(376.29915,1027.325133)
}
}
{
\newrgbcolor{curcolor}{0 0 0}
\pscustom[linewidth=2,linecolor=curcolor]
{
\newpath
\moveto(653.08486,1027.325133)
\lineto(663.44201,1027.325133)
}
}
{
\newrgbcolor{curcolor}{0 0 0}
\pscustom[linewidth=2,linecolor=curcolor]
{
\newpath
\moveto(365.92622,838.55287)
\lineto(376.28337,838.55287)
}
}
{
\newrgbcolor{curcolor}{0 0 0}
\pscustom[linewidth=2,linecolor=curcolor]
{
\newpath
\moveto(462.27106,786.86895)
\lineto(462.27106,797.2261)
}
}
{
\newrgbcolor{curcolor}{0 0 0}
\pscustom[linewidth=2,linecolor=curcolor]
{
\newpath
\moveto(627.23211,707.7679)
\lineto(627.23211,718.12505)
}
}
{
\newrgbcolor{curcolor}{1 1 1}
\pscustom[linestyle=none,fillstyle=solid,fillcolor=curcolor]
{
\newpath
\moveto(295.95058,802.06112)
\lineto(287.42742,807.23815)
\lineto(295.95058,812.41518)
\closepath
}
}
{
\newrgbcolor{curcolor}{0 0 0}
\pscustom[linewidth=2,linecolor=curcolor]
{
\newpath
\moveto(295.95058,802.06112)
\lineto(287.42742,807.23815)
\lineto(295.95058,812.41518)
\closepath
}
}
{
\newrgbcolor{curcolor}{1 1 1}
\pscustom[linestyle=none,fillstyle=solid,fillcolor=curcolor]
{
\newpath
\moveto(285.03703581,810.17341758)
\curveto(286.71361457,810.17341758)(288.0727502,808.81428196)(288.0727502,807.1377032)
\curveto(288.0727502,805.46112443)(286.71361457,804.10198881)(285.03703581,804.10198881)
\curveto(283.36045705,804.10198881)(282.00132143,805.46112443)(282.00132143,807.1377032)
\curveto(282.00132143,808.81428196)(283.36045705,810.17341758)(285.03703581,810.17341758)
\closepath
}
}
{
\newrgbcolor{curcolor}{0 0 0}
\pscustom[linewidth=2,linecolor=curcolor]
{
\newpath
\moveto(285.03703581,810.17341758)
\curveto(286.71361457,810.17341758)(288.0727502,808.81428196)(288.0727502,807.1377032)
\curveto(288.0727502,805.46112443)(286.71361457,804.10198881)(285.03703581,804.10198881)
\curveto(283.36045705,804.10198881)(282.00132143,805.46112443)(282.00132143,807.1377032)
\curveto(282.00132143,808.81428196)(283.36045705,810.17341758)(285.03703581,810.17341758)
\closepath
}
}
{
\newrgbcolor{curcolor}{0 0 0}
\pscustom[linewidth=2,linecolor=curcolor]
{
\newpath
\moveto(296.01918,810.62477)
\lineto(303.3406,810.62477)
}
}
{
\newrgbcolor{curcolor}{0 0 0}
\pscustom[linewidth=2,linecolor=curcolor]
{
\newpath
\moveto(296.04533,803.82924)
\lineto(303.63461,803.82924)
}
}
\end{pspicture}

\caption{Modelo Entidad-Relación de las entidades que representan espacios
virtuales.}
\label{modelo2}
\end{figure}

En la figura (\ref{modelo2}), se presenta la parte del modelo entidad-relación
del sistema, que comprende a las entidades que representan los espacios
virtuales, además de su relación necesaria con la entidad \emph{usuario}.

\subsection{\emph{spaces}: La abstracción de todos los espacios}
Para facilitar el manejo de múltiples tipos de espacios, se ha optado por
abstraer muchas de las funcionalidades comunes (ya sean de presentación, o de
funcionalidad), es así como se ha creado este paquete que acumula la mayor parte
de la funcionalidad para los demás espacios.

Las principales funciones de este paquete son:

\begin{itemize}
\item Concentrar las funciones sobre los recursos del sistema.
\item Generación de una capa de abstracción para la creación de espacios con
condiciones de uso diferente.
\item Administración de los espacios (creación, modificación, presentación,
eliminación).
\item Control de permisos sobre los espacios.
\end{itemize}

\subsection{\emph{gestions}, \emph{careers}, \emph{areas}: Los espacios
formales}
Al comienzo del diseño se observo que una gran parte de los espacios vitales que
eran acordes al modelo al uso en la Universidad, eran dependientes de la
gestión, es decir, que una materia/grupo acumula recursos únicamente validos por
un determinado periodo de tiempo, es así como se crearon estos espacios.

\begin{description}
\item [Gestión] Es el espacio que determina el inicio y final de la validez de
un recurso temporal, esta muy asociado a la idea que se tiene de gestión
académica en la Universidad.
\item [Área] Un área es un tipo de espacio temporal, se creo para agrupar otros
tipos de espacio de forma transversal, y de esta forma flexibilizar las
jerarquías que poseía el modelo tradicional; un ejemplo de esto se ve en las
diferentes relaciones que existen entre diferentes materias o grupos en el
modelo.
\item [Carrera] Se creo este espacio para agrupar los espacios que tienen como
característica principal estar dentro de una malla curricular especifica.
\end{description}

Las principales funciones de estos paquetes son:

\begin{itemize}
\item Conformación de los espacios organizados acorde a la estructura real
observada en el contexto del sistema.
\item Agrupación de otros espacios que comparten algún grado de relacionamiento
entre si.
\end{itemize}

\subsection{\emph{subjects}, \emph{groups}, \emph{teams}: Los espacios
jerárquicos}
Estos espacios representan el eje central sobre el que se ha construido el
sistema, también están modelados acorde a la estructura observada en el contexto
del sistema, estas poseen una estructura de tipo jerárquico (es decir, cada una
esta contenida dentro de otra).

\begin{description}
\item [Materia] Representa el espacio de materia y a su vez puede poseer varios
grupos en su propio contexto.
\item [Grupo] Representa el espacio de trabajo de un grupo especifico de una
materia, esta a su vez puede contener varios equipos, según la organización y
métodos que utilice el instructor del grupo.
\item [Equipo] Representa el mínimo espacio organizacional en un grupo, se
construyo para facilitar la organización de estudiantes dentro de un grupo.
\end{description}

Las principales funciones de estos paquetes son:

\begin{itemize}
\item Creación de la estructura jerárquica que es también un reflejo de la
estructura real observada en el contexto del sistema.
\item Asignación de usuarios (con algún rol especifico), a los sub-espacios
establecidos.
\end{itemize}

\subsection{\emph{communities}: El espacio informal}
Como una forma de crear transversales al modelo jerárquico presentado
anteriormente, se construyo el espacio de comunidad, que agrupa usuarios a
partir de criterios no relacionados a una malla curricular, es decir, como una
forma alternativa de crear relaciones entre usuarios, a partir de un interés
particular. Este es un espacio atemporal, es decir, esta no depende de la
gestión.

Las principales funciones de estos paquetes son:

\begin{itemize}
\item Creación de un espacio común, transversal a las estructuras del contexto
del sistema.
\item Manejo de permisos para el espacio (una comunidad puede ser abierta a
cualquier usuario; o cerrada, de forma que un usuario pueda ser aceptado o
rechazado en la comunidad).
\end{itemize}

\section{B-learning}
Para facilitar el manejo por parte del docente asignado a un grupo, se crearon
dos paquetes que administran las formas de calificación, y la presentación de
calificaciones.

\begin{figure}
\centering
%LaTeX with PSTricks extensions
%%Creator: inkscape 0.48.5
%%Please note this file requires PSTricks extensions
\psset{xunit=.5pt,yunit=.5pt,runit=.5pt}
\begin{pspicture}(800,500)
{
\newrgbcolor{curcolor}{0 0 0}
\pscustom[linewidth=2,linecolor=curcolor]
{
\newpath
\moveto(26.04641171,473.82646109)
\lineto(192.37978573,473.82646109)
\lineto(192.37978573,425.67582632)
\lineto(26.04641171,425.67582632)
\closepath
}
}
{
\newrgbcolor{curcolor}{0 0 0}
\pscustom[linestyle=none,fillstyle=solid,fillcolor=curcolor]
{
\newpath
\moveto(42.29083676,460.46112173)
\lineto(53.36083676,460.46112173)
\lineto(53.36083676,458.12112173)
\lineto(44.93083676,458.12112173)
\lineto(44.93083676,451.40112173)
\lineto(52.88083676,451.40112173)
\lineto(52.88083676,449.06112173)
\lineto(44.93083676,449.06112173)
\lineto(44.93083676,441.38112173)
\lineto(53.72083676,441.38112173)
\lineto(53.72083676,439.04112173)
\lineto(42.29083676,439.04112173)
\lineto(42.29083676,460.46112173)
}
}
{
\newrgbcolor{curcolor}{0 0 0}
\pscustom[linestyle=none,fillstyle=solid,fillcolor=curcolor]
{
\newpath
\moveto(54.37755551,460.46112173)
\lineto(57.13755551,460.46112173)
\lineto(61.33755551,442.07112173)
\lineto(61.39755551,442.07112173)
\lineto(65.59755551,460.46112173)
\lineto(68.35755551,460.46112173)
\lineto(62.95755551,439.04112173)
\lineto(59.59755551,439.04112173)
\lineto(54.37755551,460.46112173)
}
}
{
\newrgbcolor{curcolor}{0 0 0}
\pscustom[linestyle=none,fillstyle=solid,fillcolor=curcolor]
{
\newpath
\moveto(71.89028988,447.08112173)
\lineto(77.65028988,447.08112173)
\lineto(74.92028988,457.43112173)
\lineto(74.86028988,457.43112173)
\lineto(71.89028988,447.08112173)
\moveto(73.18028988,460.46112173)
\lineto(76.72028988,460.46112173)
\lineto(82.48028988,439.04112173)
\lineto(79.72028988,439.04112173)
\lineto(78.19028988,444.92112173)
\lineto(71.35028988,444.92112173)
\lineto(69.76028988,439.04112173)
\lineto(67.00028988,439.04112173)
\lineto(73.18028988,460.46112173)
}
}
{
\newrgbcolor{curcolor}{0 0 0}
\pscustom[linestyle=none,fillstyle=solid,fillcolor=curcolor]
{
\newpath
\moveto(84.00958676,460.46112173)
\lineto(86.64958676,460.46112173)
\lineto(86.64958676,441.38112173)
\lineto(95.34958676,441.38112173)
\lineto(95.34958676,439.04112173)
\lineto(84.00958676,439.04112173)
\lineto(84.00958676,460.46112173)
}
}
{
\newrgbcolor{curcolor}{0 0 0}
\pscustom[linestyle=none,fillstyle=solid,fillcolor=curcolor]
{
\newpath
\moveto(97.07036801,460.46112173)
\lineto(99.71036801,460.46112173)
\lineto(99.71036801,445.40112173)
\curveto(99.71036387,443.82111695)(99.9803636,442.65111812)(100.52036801,441.89112173)
\curveto(101.0803625,441.15111962)(102.02036156,440.78111999)(103.34036801,440.78112173)
\curveto(104.74035884,440.78111999)(105.70035788,441.1711196)(106.22036801,441.95112173)
\curveto(106.74035684,442.75111802)(107.00035658,443.90111687)(107.00036801,445.40112173)
\lineto(107.00036801,460.46112173)
\lineto(109.64036801,460.46112173)
\lineto(109.64036801,445.40112173)
\curveto(109.64035394,443.34111743)(109.11035447,441.69111908)(108.05036801,440.45112173)
\curveto(107.01035657,439.23112154)(105.44035814,438.62112215)(103.34036801,438.62112173)
\curveto(101.16036242,438.62112215)(99.57036401,439.19112158)(98.57036801,440.33112173)
\curveto(97.57036601,441.4711193)(97.07036651,443.16111761)(97.07036801,445.40112173)
\lineto(97.07036801,460.46112173)
}
}
{
\newrgbcolor{curcolor}{0 0 0}
\pscustom[linestyle=none,fillstyle=solid,fillcolor=curcolor]
{
\newpath
\moveto(115.80630551,447.08112173)
\lineto(121.56630551,447.08112173)
\lineto(118.83630551,457.43112173)
\lineto(118.77630551,457.43112173)
\lineto(115.80630551,447.08112173)
\moveto(117.09630551,460.46112173)
\lineto(120.63630551,460.46112173)
\lineto(126.39630551,439.04112173)
\lineto(123.63630551,439.04112173)
\lineto(122.10630551,444.92112173)
\lineto(115.26630551,444.92112173)
\lineto(113.67630551,439.04112173)
\lineto(110.91630551,439.04112173)
\lineto(117.09630551,460.46112173)
}
}
{
\newrgbcolor{curcolor}{0 0 0}
\pscustom[linestyle=none,fillstyle=solid,fillcolor=curcolor]
{
\newpath
\moveto(132.76568051,439.04112173)
\lineto(130.12568051,439.04112173)
\lineto(130.12568051,458.12112173)
\lineto(124.75568051,458.12112173)
\lineto(124.75568051,460.46112173)
\lineto(138.16568051,460.46112173)
\lineto(138.16568051,458.12112173)
\lineto(132.76568051,458.12112173)
\lineto(132.76568051,439.04112173)
}
}
{
\newrgbcolor{curcolor}{0 0 0}
\pscustom[linestyle=none,fillstyle=solid,fillcolor=curcolor]
{
\newpath
\moveto(140.14239926,460.46112173)
\lineto(142.78239926,460.46112173)
\lineto(142.78239926,439.04112173)
\lineto(140.14239926,439.04112173)
\lineto(140.14239926,460.46112173)
}
}
{
\newrgbcolor{curcolor}{0 0 0}
\pscustom[linestyle=none,fillstyle=solid,fillcolor=curcolor]
{
\newpath
\moveto(152.83614926,458.72112173)
\curveto(151.91614184,458.72110205)(151.18614257,458.48110229)(150.64614926,458.00112173)
\curveto(150.10614365,457.54110323)(149.68614407,456.90110387)(149.38614926,456.08112173)
\curveto(149.10614465,455.28110549)(148.91614484,454.33110644)(148.81614926,453.23112173)
\curveto(148.73614502,452.15110862)(148.69614506,450.99110978)(148.69614926,449.75112173)
\curveto(148.69614506,448.51111226)(148.73614502,447.34111343)(148.81614926,446.24112173)
\curveto(148.91614484,445.16111561)(149.10614465,444.21111656)(149.38614926,443.39112173)
\curveto(149.68614407,442.59111818)(150.10614365,441.95111882)(150.64614926,441.47112173)
\curveto(151.18614257,441.01111976)(151.91614184,440.78111999)(152.83614926,440.78112173)
\curveto(153.75614,440.78111999)(154.48613927,441.01111976)(155.02614926,441.47112173)
\curveto(155.56613819,441.95111882)(155.97613778,442.59111818)(156.25614926,443.39112173)
\curveto(156.5561372,444.21111656)(156.74613701,445.16111561)(156.82614926,446.24112173)
\curveto(156.92613683,447.34111343)(156.97613678,448.51111226)(156.97614926,449.75112173)
\curveto(156.97613678,450.99110978)(156.92613683,452.15110862)(156.82614926,453.23112173)
\curveto(156.74613701,454.33110644)(156.5561372,455.28110549)(156.25614926,456.08112173)
\curveto(155.97613778,456.90110387)(155.56613819,457.54110323)(155.02614926,458.00112173)
\curveto(154.48613927,458.48110229)(153.75614,458.72110205)(152.83614926,458.72112173)
\moveto(152.83614926,460.88112173)
\curveto(154.31613944,460.88109989)(155.50613825,460.55110022)(156.40614926,459.89112173)
\curveto(157.30613645,459.25110152)(158.00613575,458.40110237)(158.50614926,457.34112173)
\curveto(159.00613475,456.28110449)(159.33613442,455.08110569)(159.49614926,453.74112173)
\curveto(159.6561341,452.42110835)(159.73613402,451.09110968)(159.73614926,449.75112173)
\curveto(159.73613402,448.39111238)(159.6561341,447.05111372)(159.49614926,445.73112173)
\curveto(159.33613442,444.41111636)(159.00613475,443.22111755)(158.50614926,442.16112173)
\curveto(158.00613575,441.10111967)(157.30613645,440.24112053)(156.40614926,439.58112173)
\curveto(155.50613825,438.94112183)(154.31613944,438.62112215)(152.83614926,438.62112173)
\curveto(151.3561424,438.62112215)(150.16614359,438.94112183)(149.26614926,439.58112173)
\curveto(148.36614539,440.24112053)(147.66614609,441.10111967)(147.16614926,442.16112173)
\curveto(146.66614709,443.22111755)(146.33614742,444.41111636)(146.17614926,445.73112173)
\curveto(146.01614774,447.05111372)(145.93614782,448.39111238)(145.93614926,449.75112173)
\curveto(145.93614782,451.09110968)(146.01614774,452.42110835)(146.17614926,453.74112173)
\curveto(146.33614742,455.08110569)(146.66614709,456.28110449)(147.16614926,457.34112173)
\curveto(147.66614609,458.40110237)(148.36614539,459.25110152)(149.26614926,459.89112173)
\curveto(150.16614359,460.55110022)(151.3561424,460.88109989)(152.83614926,460.88112173)
}
}
{
\newrgbcolor{curcolor}{0 0 0}
\pscustom[linestyle=none,fillstyle=solid,fillcolor=curcolor]
{
\newpath
\moveto(162.93536801,460.46112173)
\lineto(165.54536801,460.46112173)
\lineto(166.35536801,460.46112173)
\lineto(173.43536801,442.58112173)
\lineto(173.49536801,442.58112173)
\lineto(173.49536801,460.46112173)
\lineto(176.13536801,460.46112173)
\lineto(176.13536801,439.04112173)
\lineto(173.52536801,439.04112173)
\lineto(172.50536801,439.04112173)
\lineto(165.63536801,456.38112173)
\lineto(165.57536801,456.38112173)
\lineto(165.57536801,439.04112173)
\lineto(162.93536801,439.04112173)
\lineto(162.93536801,460.46112173)
}
}
{
\newrgbcolor{curcolor}{0 0 0}
\pscustom[linewidth=2,linecolor=curcolor]
{
\newpath
\moveto(342.2242075,371.54851241)
\lineto(547.95348606,371.54851241)
\lineto(547.95348606,323.39787764)
\lineto(342.2242075,323.39787764)
\closepath
}
}
{
\newrgbcolor{curcolor}{0 0 0}
\pscustom[linestyle=none,fillstyle=solid,fillcolor=curcolor]
{
\newpath
\moveto(384.2488596,344.11317305)
\curveto(384.18884481,343.09316672)(384.02884497,342.1131677)(383.7688596,341.17317305)
\curveto(383.52884547,340.25316956)(383.15884584,339.43317038)(382.6588596,338.71317305)
\curveto(382.15884684,337.99317182)(381.4988475,337.4131724)(380.6788596,336.97317305)
\curveto(379.87884912,336.55317326)(378.8988501,336.34317347)(377.7388596,336.34317305)
\curveto(376.21885278,336.34317347)(374.998854,336.66317315)(374.0788596,337.30317305)
\curveto(373.17885582,337.96317185)(372.48885651,338.82317099)(372.0088596,339.88317305)
\curveto(371.52885747,340.94316887)(371.20885779,342.13316768)(371.0488596,343.45317305)
\curveto(370.90885809,344.77316504)(370.83885816,346.1131637)(370.8388596,347.47317305)
\curveto(370.83885816,348.813161)(370.91885808,350.14315967)(371.0788596,351.46317305)
\curveto(371.23885776,352.80315701)(371.56885743,354.00315581)(372.0688596,355.06317305)
\curveto(372.56885643,356.12315369)(373.26885573,356.97315284)(374.1688596,357.61317305)
\curveto(375.06885393,358.27315154)(376.25885274,358.60315121)(377.7388596,358.60317305)
\curveto(379.91884908,358.60315121)(381.4988475,358.02315179)(382.4788596,356.86317305)
\curveto(383.47884552,355.70315411)(384.00884499,354.06315575)(384.0688596,351.94317305)
\lineto(381.3088596,351.94317305)
\curveto(381.28884771,352.54315727)(381.21884778,353.1131567)(381.0988596,353.65317305)
\curveto(380.97884802,354.2131556)(380.77884822,354.69315512)(380.4988596,355.09317305)
\curveto(380.23884876,355.5131543)(379.87884912,355.84315397)(379.4188596,356.08317305)
\curveto(378.97885002,356.32315349)(378.41885058,356.44315337)(377.7388596,356.44317305)
\curveto(376.81885218,356.44315337)(376.08885291,356.20315361)(375.5488596,355.72317305)
\curveto(375.00885399,355.26315455)(374.58885441,354.62315519)(374.2888596,353.80317305)
\curveto(374.00885499,353.00315681)(373.81885518,352.05315776)(373.7188596,350.95317305)
\curveto(373.63885536,349.87315994)(373.5988554,348.7131611)(373.5988596,347.47317305)
\curveto(373.5988554,346.23316358)(373.63885536,345.06316475)(373.7188596,343.96317305)
\curveto(373.81885518,342.88316693)(374.00885499,341.93316788)(374.2888596,341.11317305)
\curveto(374.58885441,340.3131695)(375.00885399,339.67317014)(375.5488596,339.19317305)
\curveto(376.08885291,338.73317108)(376.81885218,338.50317131)(377.7388596,338.50317305)
\curveto(378.53885046,338.50317131)(379.17884982,338.67317114)(379.6588596,339.01317305)
\curveto(380.13884886,339.35317046)(380.50884849,339.79317002)(380.7688596,340.33317305)
\curveto(381.04884795,340.87316894)(381.22884777,341.47316834)(381.3088596,342.13317305)
\curveto(381.40884759,342.79316702)(381.46884753,343.45316636)(381.4888596,344.11317305)
\lineto(384.2488596,344.11317305)
}
}
{
\newrgbcolor{curcolor}{0 0 0}
\pscustom[linestyle=none,fillstyle=solid,fillcolor=curcolor]
{
\newpath
\moveto(389.6347971,344.80317305)
\lineto(395.3947971,344.80317305)
\lineto(392.6647971,355.15317305)
\lineto(392.6047971,355.15317305)
\lineto(389.6347971,344.80317305)
\moveto(390.9247971,358.18317305)
\lineto(394.4647971,358.18317305)
\lineto(400.2247971,336.76317305)
\lineto(397.4647971,336.76317305)
\lineto(395.9347971,342.64317305)
\lineto(389.0947971,342.64317305)
\lineto(387.5047971,336.76317305)
\lineto(384.7447971,336.76317305)
\lineto(390.9247971,358.18317305)
}
}
{
\newrgbcolor{curcolor}{0 0 0}
\pscustom[linestyle=none,fillstyle=solid,fillcolor=curcolor]
{
\newpath
\moveto(401.7247971,358.18317305)
\lineto(404.3647971,358.18317305)
\lineto(404.3647971,339.10317305)
\lineto(413.0647971,339.10317305)
\lineto(413.0647971,336.76317305)
\lineto(401.7247971,336.76317305)
\lineto(401.7247971,358.18317305)
}
}
{
\newrgbcolor{curcolor}{0 0 0}
\pscustom[linestyle=none,fillstyle=solid,fillcolor=curcolor]
{
\newpath
\moveto(415.02557835,358.18317305)
\lineto(417.66557835,358.18317305)
\lineto(417.66557835,336.76317305)
\lineto(415.02557835,336.76317305)
\lineto(415.02557835,358.18317305)
}
}
{
\newrgbcolor{curcolor}{0 0 0}
\pscustom[linestyle=none,fillstyle=solid,fillcolor=curcolor]
{
\newpath
\moveto(421.11932835,358.18317305)
\lineto(432.18932835,358.18317305)
\lineto(432.18932835,355.84317305)
\lineto(423.75932835,355.84317305)
\lineto(423.75932835,349.12317305)
\lineto(431.70932835,349.12317305)
\lineto(431.70932835,346.78317305)
\lineto(423.75932835,346.78317305)
\lineto(423.75932835,336.76317305)
\lineto(421.11932835,336.76317305)
\lineto(421.11932835,358.18317305)
}
}
{
\newrgbcolor{curcolor}{0 0 0}
\pscustom[linestyle=none,fillstyle=solid,fillcolor=curcolor]
{
\newpath
\moveto(434.4201096,358.18317305)
\lineto(437.0601096,358.18317305)
\lineto(437.0601096,336.76317305)
\lineto(434.4201096,336.76317305)
\lineto(434.4201096,358.18317305)
}
}
{
\newrgbcolor{curcolor}{0 0 0}
\pscustom[linestyle=none,fillstyle=solid,fillcolor=curcolor]
{
\newpath
\moveto(453.6238596,344.11317305)
\curveto(453.56384481,343.09316672)(453.40384497,342.1131677)(453.1438596,341.17317305)
\curveto(452.90384547,340.25316956)(452.53384584,339.43317038)(452.0338596,338.71317305)
\curveto(451.53384684,337.99317182)(450.8738475,337.4131724)(450.0538596,336.97317305)
\curveto(449.25384912,336.55317326)(448.2738501,336.34317347)(447.1138596,336.34317305)
\curveto(445.59385278,336.34317347)(444.373854,336.66317315)(443.4538596,337.30317305)
\curveto(442.55385582,337.96317185)(441.86385651,338.82317099)(441.3838596,339.88317305)
\curveto(440.90385747,340.94316887)(440.58385779,342.13316768)(440.4238596,343.45317305)
\curveto(440.28385809,344.77316504)(440.21385816,346.1131637)(440.2138596,347.47317305)
\curveto(440.21385816,348.813161)(440.29385808,350.14315967)(440.4538596,351.46317305)
\curveto(440.61385776,352.80315701)(440.94385743,354.00315581)(441.4438596,355.06317305)
\curveto(441.94385643,356.12315369)(442.64385573,356.97315284)(443.5438596,357.61317305)
\curveto(444.44385393,358.27315154)(445.63385274,358.60315121)(447.1138596,358.60317305)
\curveto(449.29384908,358.60315121)(450.8738475,358.02315179)(451.8538596,356.86317305)
\curveto(452.85384552,355.70315411)(453.38384499,354.06315575)(453.4438596,351.94317305)
\lineto(450.6838596,351.94317305)
\curveto(450.66384771,352.54315727)(450.59384778,353.1131567)(450.4738596,353.65317305)
\curveto(450.35384802,354.2131556)(450.15384822,354.69315512)(449.8738596,355.09317305)
\curveto(449.61384876,355.5131543)(449.25384912,355.84315397)(448.7938596,356.08317305)
\curveto(448.35385002,356.32315349)(447.79385058,356.44315337)(447.1138596,356.44317305)
\curveto(446.19385218,356.44315337)(445.46385291,356.20315361)(444.9238596,355.72317305)
\curveto(444.38385399,355.26315455)(443.96385441,354.62315519)(443.6638596,353.80317305)
\curveto(443.38385499,353.00315681)(443.19385518,352.05315776)(443.0938596,350.95317305)
\curveto(443.01385536,349.87315994)(442.9738554,348.7131611)(442.9738596,347.47317305)
\curveto(442.9738554,346.23316358)(443.01385536,345.06316475)(443.0938596,343.96317305)
\curveto(443.19385518,342.88316693)(443.38385499,341.93316788)(443.6638596,341.11317305)
\curveto(443.96385441,340.3131695)(444.38385399,339.67317014)(444.9238596,339.19317305)
\curveto(445.46385291,338.73317108)(446.19385218,338.50317131)(447.1138596,338.50317305)
\curveto(447.91385046,338.50317131)(448.55384982,338.67317114)(449.0338596,339.01317305)
\curveto(449.51384886,339.35317046)(449.88384849,339.79317002)(450.1438596,340.33317305)
\curveto(450.42384795,340.87316894)(450.60384777,341.47316834)(450.6838596,342.13317305)
\curveto(450.78384759,342.79316702)(450.84384753,343.45316636)(450.8638596,344.11317305)
\lineto(453.6238596,344.11317305)
}
}
{
\newrgbcolor{curcolor}{0 0 0}
\pscustom[linestyle=none,fillstyle=solid,fillcolor=curcolor]
{
\newpath
\moveto(459.0097971,344.80317305)
\lineto(464.7697971,344.80317305)
\lineto(462.0397971,355.15317305)
\lineto(461.9797971,355.15317305)
\lineto(459.0097971,344.80317305)
\moveto(460.2997971,358.18317305)
\lineto(463.8397971,358.18317305)
\lineto(469.5997971,336.76317305)
\lineto(466.8397971,336.76317305)
\lineto(465.3097971,342.64317305)
\lineto(458.4697971,342.64317305)
\lineto(456.8797971,336.76317305)
\lineto(454.1197971,336.76317305)
\lineto(460.2997971,358.18317305)
}
}
{
\newrgbcolor{curcolor}{0 0 0}
\pscustom[linestyle=none,fillstyle=solid,fillcolor=curcolor]
{
\newpath
\moveto(475.9691721,336.76317305)
\lineto(473.3291721,336.76317305)
\lineto(473.3291721,355.84317305)
\lineto(467.9591721,355.84317305)
\lineto(467.9591721,358.18317305)
\lineto(481.3691721,358.18317305)
\lineto(481.3691721,355.84317305)
\lineto(475.9691721,355.84317305)
\lineto(475.9691721,336.76317305)
}
}
{
\newrgbcolor{curcolor}{0 0 0}
\pscustom[linestyle=none,fillstyle=solid,fillcolor=curcolor]
{
\newpath
\moveto(483.34589085,358.18317305)
\lineto(485.98589085,358.18317305)
\lineto(485.98589085,336.76317305)
\lineto(483.34589085,336.76317305)
\lineto(483.34589085,358.18317305)
}
}
{
\newrgbcolor{curcolor}{0 0 0}
\pscustom[linestyle=none,fillstyle=solid,fillcolor=curcolor]
{
\newpath
\moveto(496.03964085,356.44317305)
\curveto(495.11963343,356.44315337)(494.38963416,356.20315361)(493.84964085,355.72317305)
\curveto(493.30963524,355.26315455)(492.88963566,354.62315519)(492.58964085,353.80317305)
\curveto(492.30963624,353.00315681)(492.11963643,352.05315776)(492.01964085,350.95317305)
\curveto(491.93963661,349.87315994)(491.89963665,348.7131611)(491.89964085,347.47317305)
\curveto(491.89963665,346.23316358)(491.93963661,345.06316475)(492.01964085,343.96317305)
\curveto(492.11963643,342.88316693)(492.30963624,341.93316788)(492.58964085,341.11317305)
\curveto(492.88963566,340.3131695)(493.30963524,339.67317014)(493.84964085,339.19317305)
\curveto(494.38963416,338.73317108)(495.11963343,338.50317131)(496.03964085,338.50317305)
\curveto(496.95963159,338.50317131)(497.68963086,338.73317108)(498.22964085,339.19317305)
\curveto(498.76962978,339.67317014)(499.17962937,340.3131695)(499.45964085,341.11317305)
\curveto(499.75962879,341.93316788)(499.9496286,342.88316693)(500.02964085,343.96317305)
\curveto(500.12962842,345.06316475)(500.17962837,346.23316358)(500.17964085,347.47317305)
\curveto(500.17962837,348.7131611)(500.12962842,349.87315994)(500.02964085,350.95317305)
\curveto(499.9496286,352.05315776)(499.75962879,353.00315681)(499.45964085,353.80317305)
\curveto(499.17962937,354.62315519)(498.76962978,355.26315455)(498.22964085,355.72317305)
\curveto(497.68963086,356.20315361)(496.95963159,356.44315337)(496.03964085,356.44317305)
\moveto(496.03964085,358.60317305)
\curveto(497.51963103,358.60315121)(498.70962984,358.27315154)(499.60964085,357.61317305)
\curveto(500.50962804,356.97315284)(501.20962734,356.12315369)(501.70964085,355.06317305)
\curveto(502.20962634,354.00315581)(502.53962601,352.80315701)(502.69964085,351.46317305)
\curveto(502.85962569,350.14315967)(502.93962561,348.813161)(502.93964085,347.47317305)
\curveto(502.93962561,346.1131637)(502.85962569,344.77316504)(502.69964085,343.45317305)
\curveto(502.53962601,342.13316768)(502.20962634,340.94316887)(501.70964085,339.88317305)
\curveto(501.20962734,338.82317099)(500.50962804,337.96317185)(499.60964085,337.30317305)
\curveto(498.70962984,336.66317315)(497.51963103,336.34317347)(496.03964085,336.34317305)
\curveto(494.55963399,336.34317347)(493.36963518,336.66317315)(492.46964085,337.30317305)
\curveto(491.56963698,337.96317185)(490.86963768,338.82317099)(490.36964085,339.88317305)
\curveto(489.86963868,340.94316887)(489.53963901,342.13316768)(489.37964085,343.45317305)
\curveto(489.21963933,344.77316504)(489.13963941,346.1131637)(489.13964085,347.47317305)
\curveto(489.13963941,348.813161)(489.21963933,350.14315967)(489.37964085,351.46317305)
\curveto(489.53963901,352.80315701)(489.86963868,354.00315581)(490.36964085,355.06317305)
\curveto(490.86963768,356.12315369)(491.56963698,356.97315284)(492.46964085,357.61317305)
\curveto(493.36963518,358.27315154)(494.55963399,358.60315121)(496.03964085,358.60317305)
}
}
{
\newrgbcolor{curcolor}{0 0 0}
\pscustom[linestyle=none,fillstyle=solid,fillcolor=curcolor]
{
\newpath
\moveto(506.1388596,358.18317305)
\lineto(508.7488596,358.18317305)
\lineto(509.5588596,358.18317305)
\lineto(516.6388596,340.30317305)
\lineto(516.6988596,340.30317305)
\lineto(516.6988596,358.18317305)
\lineto(519.3388596,358.18317305)
\lineto(519.3388596,336.76317305)
\lineto(516.7288596,336.76317305)
\lineto(515.7088596,336.76317305)
\lineto(508.8388596,354.10317305)
\lineto(508.7788596,354.10317305)
\lineto(508.7788596,336.76317305)
\lineto(506.1388596,336.76317305)
\lineto(506.1388596,358.18317305)
}
}
{
\newrgbcolor{curcolor}{0 0 0}
\pscustom[linewidth=2,linecolor=curcolor]
{
\newpath
\moveto(102.81800171,256.13856209)
\lineto(340.87223145,256.13856209)
\lineto(340.87223145,207.98792732)
\lineto(102.81800171,207.98792732)
\closepath
}
}
{
\newrgbcolor{curcolor}{0 0 0}
\pscustom[linestyle=none,fillstyle=solid,fillcolor=curcolor]
{
\newpath
\moveto(118.33153076,244.4382618)
\lineto(129.40153076,244.4382618)
\lineto(129.40153076,242.0982618)
\lineto(120.97153076,242.0982618)
\lineto(120.97153076,235.3782618)
\lineto(128.92153076,235.3782618)
\lineto(128.92153076,233.0382618)
\lineto(120.97153076,233.0382618)
\lineto(120.97153076,225.3582618)
\lineto(129.76153076,225.3582618)
\lineto(129.76153076,223.0182618)
\lineto(118.33153076,223.0182618)
\lineto(118.33153076,244.4382618)
}
}
{
\newrgbcolor{curcolor}{0 0 0}
\pscustom[linestyle=none,fillstyle=solid,fillcolor=curcolor]
{
\newpath
\moveto(130.41824951,244.4382618)
\lineto(133.17824951,244.4382618)
\lineto(137.37824951,226.0482618)
\lineto(137.43824951,226.0482618)
\lineto(141.63824951,244.4382618)
\lineto(144.39824951,244.4382618)
\lineto(138.99824951,223.0182618)
\lineto(135.63824951,223.0182618)
\lineto(130.41824951,244.4382618)
}
}
{
\newrgbcolor{curcolor}{0 0 0}
\pscustom[linestyle=none,fillstyle=solid,fillcolor=curcolor]
{
\newpath
\moveto(147.93098389,231.0582618)
\lineto(153.69098389,231.0582618)
\lineto(150.96098389,241.4082618)
\lineto(150.90098389,241.4082618)
\lineto(147.93098389,231.0582618)
\moveto(149.22098389,244.4382618)
\lineto(152.76098389,244.4382618)
\lineto(158.52098389,223.0182618)
\lineto(155.76098389,223.0182618)
\lineto(154.23098389,228.8982618)
\lineto(147.39098389,228.8982618)
\lineto(145.80098389,223.0182618)
\lineto(143.04098389,223.0182618)
\lineto(149.22098389,244.4382618)
}
}
{
\newrgbcolor{curcolor}{0 0 0}
\pscustom[linestyle=none,fillstyle=solid,fillcolor=curcolor]
{
\newpath
\moveto(160.05028076,244.4382618)
\lineto(162.69028076,244.4382618)
\lineto(162.69028076,225.3582618)
\lineto(171.39028076,225.3582618)
\lineto(171.39028076,223.0182618)
\lineto(160.05028076,223.0182618)
\lineto(160.05028076,244.4382618)
}
}
{
\newrgbcolor{curcolor}{0 0 0}
\pscustom[linestyle=none,fillstyle=solid,fillcolor=curcolor]
{
\newpath
\moveto(173.11106201,244.4382618)
\lineto(175.75106201,244.4382618)
\lineto(175.75106201,229.3782618)
\curveto(175.75105787,227.79825702)(176.0210576,226.62825819)(176.56106201,225.8682618)
\curveto(177.1210565,225.12825969)(178.06105556,224.75826006)(179.38106201,224.7582618)
\curveto(180.78105284,224.75826006)(181.74105188,225.14825967)(182.26106201,225.9282618)
\curveto(182.78105084,226.72825809)(183.04105058,227.87825694)(183.04106201,229.3782618)
\lineto(183.04106201,244.4382618)
\lineto(185.68106201,244.4382618)
\lineto(185.68106201,229.3782618)
\curveto(185.68104794,227.3182575)(185.15104847,225.66825915)(184.09106201,224.4282618)
\curveto(183.05105057,223.20826161)(181.48105214,222.59826222)(179.38106201,222.5982618)
\curveto(177.20105642,222.59826222)(175.61105801,223.16826165)(174.61106201,224.3082618)
\curveto(173.61106001,225.44825937)(173.11106051,227.13825768)(173.11106201,229.3782618)
\lineto(173.11106201,244.4382618)
}
}
{
\newrgbcolor{curcolor}{0 0 0}
\pscustom[linestyle=none,fillstyle=solid,fillcolor=curcolor]
{
\newpath
\moveto(191.84699951,231.0582618)
\lineto(197.60699951,231.0582618)
\lineto(194.87699951,241.4082618)
\lineto(194.81699951,241.4082618)
\lineto(191.84699951,231.0582618)
\moveto(193.13699951,244.4382618)
\lineto(196.67699951,244.4382618)
\lineto(202.43699951,223.0182618)
\lineto(199.67699951,223.0182618)
\lineto(198.14699951,228.8982618)
\lineto(191.30699951,228.8982618)
\lineto(189.71699951,223.0182618)
\lineto(186.95699951,223.0182618)
\lineto(193.13699951,244.4382618)
}
}
{
\newrgbcolor{curcolor}{0 0 0}
\pscustom[linestyle=none,fillstyle=solid,fillcolor=curcolor]
{
\newpath
\moveto(208.80637451,223.0182618)
\lineto(206.16637451,223.0182618)
\lineto(206.16637451,242.0982618)
\lineto(200.79637451,242.0982618)
\lineto(200.79637451,244.4382618)
\lineto(214.20637451,244.4382618)
\lineto(214.20637451,242.0982618)
\lineto(208.80637451,242.0982618)
\lineto(208.80637451,223.0182618)
}
}
{
\newrgbcolor{curcolor}{0 0 0}
\pscustom[linestyle=none,fillstyle=solid,fillcolor=curcolor]
{
\newpath
\moveto(216.18309326,244.4382618)
\lineto(218.82309326,244.4382618)
\lineto(218.82309326,223.0182618)
\lineto(216.18309326,223.0182618)
\lineto(216.18309326,244.4382618)
}
}
{
\newrgbcolor{curcolor}{0 0 0}
\pscustom[linestyle=none,fillstyle=solid,fillcolor=curcolor]
{
\newpath
\moveto(228.87684326,242.6982618)
\curveto(227.95683584,242.69824212)(227.22683657,242.45824236)(226.68684326,241.9782618)
\curveto(226.14683765,241.5182433)(225.72683807,240.87824394)(225.42684326,240.0582618)
\curveto(225.14683865,239.25824556)(224.95683884,238.30824651)(224.85684326,237.2082618)
\curveto(224.77683902,236.12824869)(224.73683906,234.96824985)(224.73684326,233.7282618)
\curveto(224.73683906,232.48825233)(224.77683902,231.3182535)(224.85684326,230.2182618)
\curveto(224.95683884,229.13825568)(225.14683865,228.18825663)(225.42684326,227.3682618)
\curveto(225.72683807,226.56825825)(226.14683765,225.92825889)(226.68684326,225.4482618)
\curveto(227.22683657,224.98825983)(227.95683584,224.75826006)(228.87684326,224.7582618)
\curveto(229.796834,224.75826006)(230.52683327,224.98825983)(231.06684326,225.4482618)
\curveto(231.60683219,225.92825889)(232.01683178,226.56825825)(232.29684326,227.3682618)
\curveto(232.5968312,228.18825663)(232.78683101,229.13825568)(232.86684326,230.2182618)
\curveto(232.96683083,231.3182535)(233.01683078,232.48825233)(233.01684326,233.7282618)
\curveto(233.01683078,234.96824985)(232.96683083,236.12824869)(232.86684326,237.2082618)
\curveto(232.78683101,238.30824651)(232.5968312,239.25824556)(232.29684326,240.0582618)
\curveto(232.01683178,240.87824394)(231.60683219,241.5182433)(231.06684326,241.9782618)
\curveto(230.52683327,242.45824236)(229.796834,242.69824212)(228.87684326,242.6982618)
\moveto(228.87684326,244.8582618)
\curveto(230.35683344,244.85823996)(231.54683225,244.52824029)(232.44684326,243.8682618)
\curveto(233.34683045,243.22824159)(234.04682975,242.37824244)(234.54684326,241.3182618)
\curveto(235.04682875,240.25824456)(235.37682842,239.05824576)(235.53684326,237.7182618)
\curveto(235.6968281,236.39824842)(235.77682802,235.06824975)(235.77684326,233.7282618)
\curveto(235.77682802,232.36825245)(235.6968281,231.02825379)(235.53684326,229.7082618)
\curveto(235.37682842,228.38825643)(235.04682875,227.19825762)(234.54684326,226.1382618)
\curveto(234.04682975,225.07825974)(233.34683045,224.2182606)(232.44684326,223.5582618)
\curveto(231.54683225,222.9182619)(230.35683344,222.59826222)(228.87684326,222.5982618)
\curveto(227.3968364,222.59826222)(226.20683759,222.9182619)(225.30684326,223.5582618)
\curveto(224.40683939,224.2182606)(223.70684009,225.07825974)(223.20684326,226.1382618)
\curveto(222.70684109,227.19825762)(222.37684142,228.38825643)(222.21684326,229.7082618)
\curveto(222.05684174,231.02825379)(221.97684182,232.36825245)(221.97684326,233.7282618)
\curveto(221.97684182,235.06824975)(222.05684174,236.39824842)(222.21684326,237.7182618)
\curveto(222.37684142,239.05824576)(222.70684109,240.25824456)(223.20684326,241.3182618)
\curveto(223.70684009,242.37824244)(224.40683939,243.22824159)(225.30684326,243.8682618)
\curveto(226.20683759,244.52824029)(227.3968364,244.85823996)(228.87684326,244.8582618)
}
}
{
\newrgbcolor{curcolor}{0 0 0}
\pscustom[linestyle=none,fillstyle=solid,fillcolor=curcolor]
{
\newpath
\moveto(238.97606201,244.4382618)
\lineto(241.58606201,244.4382618)
\lineto(242.39606201,244.4382618)
\lineto(249.47606201,226.5582618)
\lineto(249.53606201,226.5582618)
\lineto(249.53606201,244.4382618)
\lineto(252.17606201,244.4382618)
\lineto(252.17606201,223.0182618)
\lineto(249.56606201,223.0182618)
\lineto(248.54606201,223.0182618)
\lineto(241.67606201,240.3582618)
\lineto(241.61606201,240.3582618)
\lineto(241.61606201,223.0182618)
\lineto(238.97606201,223.0182618)
\lineto(238.97606201,244.4382618)
}
}
{
\newrgbcolor{curcolor}{0 0 0}
\pscustom[linestyle=none,fillstyle=solid,fillcolor=curcolor]
{
\newpath
\moveto(253.93528076,220.7682618)
\lineto(268.93528076,220.7682618)
\lineto(268.93528076,219.2682618)
\lineto(253.93528076,219.2682618)
\lineto(253.93528076,220.7682618)
}
}
{
\newrgbcolor{curcolor}{0 0 0}
\pscustom[linestyle=none,fillstyle=solid,fillcolor=curcolor]
{
\newpath
\moveto(277.18528076,223.0182618)
\lineto(274.54528076,223.0182618)
\lineto(274.54528076,242.0982618)
\lineto(269.17528076,242.0982618)
\lineto(269.17528076,244.4382618)
\lineto(282.58528076,244.4382618)
\lineto(282.58528076,242.0982618)
\lineto(277.18528076,242.0982618)
\lineto(277.18528076,223.0182618)
}
}
{
\newrgbcolor{curcolor}{0 0 0}
\pscustom[linestyle=none,fillstyle=solid,fillcolor=curcolor]
{
\newpath
\moveto(284.56199951,244.4382618)
\lineto(295.63199951,244.4382618)
\lineto(295.63199951,242.0982618)
\lineto(287.20199951,242.0982618)
\lineto(287.20199951,235.3782618)
\lineto(295.15199951,235.3782618)
\lineto(295.15199951,233.0382618)
\lineto(287.20199951,233.0382618)
\lineto(287.20199951,225.3582618)
\lineto(295.99199951,225.3582618)
\lineto(295.99199951,223.0182618)
\lineto(284.56199951,223.0182618)
\lineto(284.56199951,244.4382618)
}
}
{
\newrgbcolor{curcolor}{0 0 0}
\pscustom[linestyle=none,fillstyle=solid,fillcolor=curcolor]
{
\newpath
\moveto(307.53871826,238.8882618)
\curveto(307.53870743,239.44824537)(307.47870749,239.95824486)(307.35871826,240.4182618)
\curveto(307.25870771,240.89824392)(307.07870789,241.29824352)(306.81871826,241.6182618)
\curveto(306.55870841,241.95824286)(306.20870876,242.2182426)(305.76871826,242.3982618)
\curveto(305.34870962,242.59824222)(304.82871014,242.69824212)(304.20871826,242.6982618)
\curveto(303.0687119,242.69824212)(302.18871278,242.39824242)(301.56871826,241.7982618)
\curveto(300.968714,241.2182436)(300.6687143,240.35824446)(300.66871826,239.2182618)
\curveto(300.6687143,238.2182466)(300.91871405,237.44824737)(301.41871826,236.9082618)
\curveto(301.91871305,236.36824845)(302.53871243,235.93824888)(303.27871826,235.6182618)
\curveto(304.03871093,235.29824952)(304.84871012,235.00824981)(305.70871826,234.7482618)
\curveto(306.58870838,234.50825031)(307.39870757,234.17825064)(308.13871826,233.7582618)
\curveto(308.89870607,233.33825148)(309.52870544,232.75825206)(310.02871826,232.0182618)
\curveto(310.52870444,231.27825354)(310.77870419,230.25825456)(310.77871826,228.9582618)
\curveto(310.77870419,227.7182571)(310.5687044,226.68825813)(310.14871826,225.8682618)
\curveto(309.74870522,225.04825977)(309.22870574,224.39826042)(308.58871826,223.9182618)
\curveto(307.94870702,223.43826138)(307.22870774,223.09826172)(306.42871826,222.8982618)
\curveto(305.64870932,222.69826212)(304.87871009,222.59826222)(304.11871826,222.5982618)
\curveto(302.85871211,222.59826222)(301.80871316,222.75826206)(300.96871826,223.0782618)
\curveto(300.14871482,223.39826142)(299.48871548,223.85826096)(298.98871826,224.4582618)
\curveto(298.50871646,225.05825976)(298.15871681,225.79825902)(297.93871826,226.6782618)
\curveto(297.73871723,227.57825724)(297.63871733,228.59825622)(297.63871826,229.7382618)
\lineto(300.27871826,229.7382618)
\curveto(300.27871469,229.13825568)(300.30871466,228.53825628)(300.36871826,227.9382618)
\curveto(300.42871454,227.35825746)(300.58871438,226.82825799)(300.84871826,226.3482618)
\curveto(301.10871386,225.86825895)(301.50871346,225.47825934)(302.04871826,225.1782618)
\curveto(302.58871238,224.89825992)(303.33871163,224.75826006)(304.29871826,224.7582618)
\curveto(304.81871015,224.75826006)(305.30870966,224.84825997)(305.76871826,225.0282618)
\curveto(306.22870874,225.20825961)(306.61870835,225.45825936)(306.93871826,225.7782618)
\curveto(307.27870769,226.1182587)(307.53870743,226.5182583)(307.71871826,226.9782618)
\curveto(307.91870705,227.43825738)(308.01870695,227.95825686)(308.01871826,228.5382618)
\curveto(308.01870695,229.29825552)(307.8687071,229.9182549)(307.56871826,230.3982618)
\curveto(307.28870768,230.87825394)(306.90870806,231.27825354)(306.42871826,231.5982618)
\curveto(305.968709,231.93825288)(305.42870954,232.20825261)(304.80871826,232.4082618)
\curveto(304.20871076,232.62825219)(303.58871138,232.84825197)(302.94871826,233.0682618)
\curveto(302.32871264,233.28825153)(301.70871326,233.52825129)(301.08871826,233.7882618)
\curveto(300.48871448,234.06825075)(299.94871502,234.4182504)(299.46871826,234.8382618)
\curveto(299.00871596,235.27824954)(298.62871634,235.82824899)(298.32871826,236.4882618)
\curveto(298.04871692,237.14824767)(297.90871706,237.96824685)(297.90871826,238.9482618)
\curveto(297.90871706,239.44824537)(297.97871699,240.02824479)(298.11871826,240.6882618)
\curveto(298.25871671,241.36824345)(298.54871642,242.0182428)(298.98871826,242.6382618)
\curveto(299.44871552,243.25824156)(300.08871488,243.77824104)(300.90871826,244.1982618)
\curveto(301.72871324,244.63824018)(302.81871215,244.85823996)(304.17871826,244.8582618)
\curveto(306.23870873,244.85823996)(307.73870723,244.35824046)(308.67871826,243.3582618)
\curveto(309.63870533,242.37824244)(310.13870483,240.88824393)(310.17871826,238.8882618)
\lineto(307.53871826,238.8882618)
}
}
{
\newrgbcolor{curcolor}{0 0 0}
\pscustom[linestyle=none,fillstyle=solid,fillcolor=curcolor]
{
\newpath
\moveto(319.95871826,223.0182618)
\lineto(317.31871826,223.0182618)
\lineto(317.31871826,242.0982618)
\lineto(311.94871826,242.0982618)
\lineto(311.94871826,244.4382618)
\lineto(325.35871826,244.4382618)
\lineto(325.35871826,242.0982618)
\lineto(319.95871826,242.0982618)
\lineto(319.95871826,223.0182618)
}
}
{
\newrgbcolor{curcolor}{0 0 0}
\pscustom[linewidth=2,linecolor=curcolor]
{
\newpath
\moveto(219.49062171,152.59797209)
\lineto(550.47890296,152.59797209)
\lineto(550.47890296,104.44733732)
\lineto(219.49062171,104.44733732)
\closepath
}
}
{
\newrgbcolor{curcolor}{0 0 0}
\pscustom[linestyle=none,fillstyle=solid,fillcolor=curcolor]
{
\newpath
\moveto(238.90844764,140.8976718)
\lineto(249.97844764,140.8976718)
\lineto(249.97844764,138.5576718)
\lineto(241.54844764,138.5576718)
\lineto(241.54844764,131.8376718)
\lineto(249.49844764,131.8376718)
\lineto(249.49844764,129.4976718)
\lineto(241.54844764,129.4976718)
\lineto(241.54844764,121.8176718)
\lineto(250.33844764,121.8176718)
\lineto(250.33844764,119.4776718)
\lineto(238.90844764,119.4776718)
\lineto(238.90844764,140.8976718)
}
}
{
\newrgbcolor{curcolor}{0 0 0}
\pscustom[linestyle=none,fillstyle=solid,fillcolor=curcolor]
{
\newpath
\moveto(250.99516639,140.8976718)
\lineto(253.75516639,140.8976718)
\lineto(257.95516639,122.5076718)
\lineto(258.01516639,122.5076718)
\lineto(262.21516639,140.8976718)
\lineto(264.97516639,140.8976718)
\lineto(259.57516639,119.4776718)
\lineto(256.21516639,119.4776718)
\lineto(250.99516639,140.8976718)
}
}
{
\newrgbcolor{curcolor}{0 0 0}
\pscustom[linestyle=none,fillstyle=solid,fillcolor=curcolor]
{
\newpath
\moveto(268.50790076,127.5176718)
\lineto(274.26790076,127.5176718)
\lineto(271.53790076,137.8676718)
\lineto(271.47790076,137.8676718)
\lineto(268.50790076,127.5176718)
\moveto(269.79790076,140.8976718)
\lineto(273.33790076,140.8976718)
\lineto(279.09790076,119.4776718)
\lineto(276.33790076,119.4776718)
\lineto(274.80790076,125.3576718)
\lineto(267.96790076,125.3576718)
\lineto(266.37790076,119.4776718)
\lineto(263.61790076,119.4776718)
\lineto(269.79790076,140.8976718)
}
}
{
\newrgbcolor{curcolor}{0 0 0}
\pscustom[linestyle=none,fillstyle=solid,fillcolor=curcolor]
{
\newpath
\moveto(280.62719764,140.8976718)
\lineto(283.26719764,140.8976718)
\lineto(283.26719764,121.8176718)
\lineto(291.96719764,121.8176718)
\lineto(291.96719764,119.4776718)
\lineto(280.62719764,119.4776718)
\lineto(280.62719764,140.8976718)
}
}
{
\newrgbcolor{curcolor}{0 0 0}
\pscustom[linestyle=none,fillstyle=solid,fillcolor=curcolor]
{
\newpath
\moveto(293.68797889,140.8976718)
\lineto(296.32797889,140.8976718)
\lineto(296.32797889,125.8376718)
\curveto(296.32797475,124.25766702)(296.59797448,123.08766819)(297.13797889,122.3276718)
\curveto(297.69797338,121.58766969)(298.63797244,121.21767006)(299.95797889,121.2176718)
\curveto(301.35796972,121.21767006)(302.31796876,121.60766967)(302.83797889,122.3876718)
\curveto(303.35796772,123.18766809)(303.61796746,124.33766694)(303.61797889,125.8376718)
\lineto(303.61797889,140.8976718)
\lineto(306.25797889,140.8976718)
\lineto(306.25797889,125.8376718)
\curveto(306.25796482,123.7776675)(305.72796535,122.12766915)(304.66797889,120.8876718)
\curveto(303.62796745,119.66767161)(302.05796902,119.05767222)(299.95797889,119.0576718)
\curveto(297.7779733,119.05767222)(296.18797489,119.62767165)(295.18797889,120.7676718)
\curveto(294.18797689,121.90766937)(293.68797739,123.59766768)(293.68797889,125.8376718)
\lineto(293.68797889,140.8976718)
}
}
{
\newrgbcolor{curcolor}{0 0 0}
\pscustom[linestyle=none,fillstyle=solid,fillcolor=curcolor]
{
\newpath
\moveto(312.42391639,127.5176718)
\lineto(318.18391639,127.5176718)
\lineto(315.45391639,137.8676718)
\lineto(315.39391639,137.8676718)
\lineto(312.42391639,127.5176718)
\moveto(313.71391639,140.8976718)
\lineto(317.25391639,140.8976718)
\lineto(323.01391639,119.4776718)
\lineto(320.25391639,119.4776718)
\lineto(318.72391639,125.3576718)
\lineto(311.88391639,125.3576718)
\lineto(310.29391639,119.4776718)
\lineto(307.53391639,119.4776718)
\lineto(313.71391639,140.8976718)
}
}
{
\newrgbcolor{curcolor}{0 0 0}
\pscustom[linestyle=none,fillstyle=solid,fillcolor=curcolor]
{
\newpath
\moveto(329.38329139,119.4776718)
\lineto(326.74329139,119.4776718)
\lineto(326.74329139,138.5576718)
\lineto(321.37329139,138.5576718)
\lineto(321.37329139,140.8976718)
\lineto(334.78329139,140.8976718)
\lineto(334.78329139,138.5576718)
\lineto(329.38329139,138.5576718)
\lineto(329.38329139,119.4776718)
}
}
{
\newrgbcolor{curcolor}{0 0 0}
\pscustom[linestyle=none,fillstyle=solid,fillcolor=curcolor]
{
\newpath
\moveto(336.76001014,140.8976718)
\lineto(339.40001014,140.8976718)
\lineto(339.40001014,119.4776718)
\lineto(336.76001014,119.4776718)
\lineto(336.76001014,140.8976718)
}
}
{
\newrgbcolor{curcolor}{0 0 0}
\pscustom[linestyle=none,fillstyle=solid,fillcolor=curcolor]
{
\newpath
\moveto(349.45376014,139.1576718)
\curveto(348.53375272,139.15765212)(347.80375345,138.91765236)(347.26376014,138.4376718)
\curveto(346.72375453,137.9776533)(346.30375495,137.33765394)(346.00376014,136.5176718)
\curveto(345.72375553,135.71765556)(345.53375572,134.76765651)(345.43376014,133.6676718)
\curveto(345.3537559,132.58765869)(345.31375594,131.42765985)(345.31376014,130.1876718)
\curveto(345.31375594,128.94766233)(345.3537559,127.7776635)(345.43376014,126.6776718)
\curveto(345.53375572,125.59766568)(345.72375553,124.64766663)(346.00376014,123.8276718)
\curveto(346.30375495,123.02766825)(346.72375453,122.38766889)(347.26376014,121.9076718)
\curveto(347.80375345,121.44766983)(348.53375272,121.21767006)(349.45376014,121.2176718)
\curveto(350.37375088,121.21767006)(351.10375015,121.44766983)(351.64376014,121.9076718)
\curveto(352.18374907,122.38766889)(352.59374866,123.02766825)(352.87376014,123.8276718)
\curveto(353.17374808,124.64766663)(353.36374789,125.59766568)(353.44376014,126.6776718)
\curveto(353.54374771,127.7776635)(353.59374766,128.94766233)(353.59376014,130.1876718)
\curveto(353.59374766,131.42765985)(353.54374771,132.58765869)(353.44376014,133.6676718)
\curveto(353.36374789,134.76765651)(353.17374808,135.71765556)(352.87376014,136.5176718)
\curveto(352.59374866,137.33765394)(352.18374907,137.9776533)(351.64376014,138.4376718)
\curveto(351.10375015,138.91765236)(350.37375088,139.15765212)(349.45376014,139.1576718)
\moveto(349.45376014,141.3176718)
\curveto(350.93375032,141.31764996)(352.12374913,140.98765029)(353.02376014,140.3276718)
\curveto(353.92374733,139.68765159)(354.62374663,138.83765244)(355.12376014,137.7776718)
\curveto(355.62374563,136.71765456)(355.9537453,135.51765576)(356.11376014,134.1776718)
\curveto(356.27374498,132.85765842)(356.3537449,131.52765975)(356.35376014,130.1876718)
\curveto(356.3537449,128.82766245)(356.27374498,127.48766379)(356.11376014,126.1676718)
\curveto(355.9537453,124.84766643)(355.62374563,123.65766762)(355.12376014,122.5976718)
\curveto(354.62374663,121.53766974)(353.92374733,120.6776706)(353.02376014,120.0176718)
\curveto(352.12374913,119.3776719)(350.93375032,119.05767222)(349.45376014,119.0576718)
\curveto(347.97375328,119.05767222)(346.78375447,119.3776719)(345.88376014,120.0176718)
\curveto(344.98375627,120.6776706)(344.28375697,121.53766974)(343.78376014,122.5976718)
\curveto(343.28375797,123.65766762)(342.9537583,124.84766643)(342.79376014,126.1676718)
\curveto(342.63375862,127.48766379)(342.5537587,128.82766245)(342.55376014,130.1876718)
\curveto(342.5537587,131.52765975)(342.63375862,132.85765842)(342.79376014,134.1776718)
\curveto(342.9537583,135.51765576)(343.28375797,136.71765456)(343.78376014,137.7776718)
\curveto(344.28375697,138.83765244)(344.98375627,139.68765159)(345.88376014,140.3276718)
\curveto(346.78375447,140.98765029)(347.97375328,141.31764996)(349.45376014,141.3176718)
}
}
{
\newrgbcolor{curcolor}{0 0 0}
\pscustom[linestyle=none,fillstyle=solid,fillcolor=curcolor]
{
\newpath
\moveto(359.55297889,140.8976718)
\lineto(362.16297889,140.8976718)
\lineto(362.97297889,140.8976718)
\lineto(370.05297889,123.0176718)
\lineto(370.11297889,123.0176718)
\lineto(370.11297889,140.8976718)
\lineto(372.75297889,140.8976718)
\lineto(372.75297889,119.4776718)
\lineto(370.14297889,119.4776718)
\lineto(369.12297889,119.4776718)
\lineto(362.25297889,136.8176718)
\lineto(362.19297889,136.8176718)
\lineto(362.19297889,119.4776718)
\lineto(359.55297889,119.4776718)
\lineto(359.55297889,140.8976718)
}
}
{
\newrgbcolor{curcolor}{0 0 0}
\pscustom[linestyle=none,fillstyle=solid,fillcolor=curcolor]
{
\newpath
\moveto(374.51219764,117.2276718)
\lineto(389.51219764,117.2276718)
\lineto(389.51219764,115.7276718)
\lineto(374.51219764,115.7276718)
\lineto(374.51219764,117.2276718)
}
}
{
\newrgbcolor{curcolor}{0 0 0}
\pscustom[linestyle=none,fillstyle=solid,fillcolor=curcolor]
{
\newpath
\moveto(397.76219764,119.4776718)
\lineto(395.12219764,119.4776718)
\lineto(395.12219764,138.5576718)
\lineto(389.75219764,138.5576718)
\lineto(389.75219764,140.8976718)
\lineto(403.16219764,140.8976718)
\lineto(403.16219764,138.5576718)
\lineto(397.76219764,138.5576718)
\lineto(397.76219764,119.4776718)
}
}
{
\newrgbcolor{curcolor}{0 0 0}
\pscustom[linestyle=none,fillstyle=solid,fillcolor=curcolor]
{
\newpath
\moveto(405.13891639,140.8976718)
\lineto(416.20891639,140.8976718)
\lineto(416.20891639,138.5576718)
\lineto(407.77891639,138.5576718)
\lineto(407.77891639,131.8376718)
\lineto(415.72891639,131.8376718)
\lineto(415.72891639,129.4976718)
\lineto(407.77891639,129.4976718)
\lineto(407.77891639,121.8176718)
\lineto(416.56891639,121.8176718)
\lineto(416.56891639,119.4776718)
\lineto(405.13891639,119.4776718)
\lineto(405.13891639,140.8976718)
}
}
{
\newrgbcolor{curcolor}{0 0 0}
\pscustom[linestyle=none,fillstyle=solid,fillcolor=curcolor]
{
\newpath
\moveto(428.11563514,135.3476718)
\curveto(428.11562431,135.90765537)(428.05562437,136.41765486)(427.93563514,136.8776718)
\curveto(427.83562459,137.35765392)(427.65562477,137.75765352)(427.39563514,138.0776718)
\curveto(427.13562529,138.41765286)(426.78562564,138.6776526)(426.34563514,138.8576718)
\curveto(425.9256265,139.05765222)(425.40562702,139.15765212)(424.78563514,139.1576718)
\curveto(423.64562878,139.15765212)(422.76562966,138.85765242)(422.14563514,138.2576718)
\curveto(421.54563088,137.6776536)(421.24563118,136.81765446)(421.24563514,135.6776718)
\curveto(421.24563118,134.6776566)(421.49563093,133.90765737)(421.99563514,133.3676718)
\curveto(422.49562993,132.82765845)(423.11562931,132.39765888)(423.85563514,132.0776718)
\curveto(424.61562781,131.75765952)(425.425627,131.46765981)(426.28563514,131.2076718)
\curveto(427.16562526,130.96766031)(427.97562445,130.63766064)(428.71563514,130.2176718)
\curveto(429.47562295,129.79766148)(430.10562232,129.21766206)(430.60563514,128.4776718)
\curveto(431.10562132,127.73766354)(431.35562107,126.71766456)(431.35563514,125.4176718)
\curveto(431.35562107,124.1776671)(431.14562128,123.14766813)(430.72563514,122.3276718)
\curveto(430.3256221,121.50766977)(429.80562262,120.85767042)(429.16563514,120.3776718)
\curveto(428.5256239,119.89767138)(427.80562462,119.55767172)(427.00563514,119.3576718)
\curveto(426.2256262,119.15767212)(425.45562697,119.05767222)(424.69563514,119.0576718)
\curveto(423.43562899,119.05767222)(422.38563004,119.21767206)(421.54563514,119.5376718)
\curveto(420.7256317,119.85767142)(420.06563236,120.31767096)(419.56563514,120.9176718)
\curveto(419.08563334,121.51766976)(418.73563369,122.25766902)(418.51563514,123.1376718)
\curveto(418.31563411,124.03766724)(418.21563421,125.05766622)(418.21563514,126.1976718)
\lineto(420.85563514,126.1976718)
\curveto(420.85563157,125.59766568)(420.88563154,124.99766628)(420.94563514,124.3976718)
\curveto(421.00563142,123.81766746)(421.16563126,123.28766799)(421.42563514,122.8076718)
\curveto(421.68563074,122.32766895)(422.08563034,121.93766934)(422.62563514,121.6376718)
\curveto(423.16562926,121.35766992)(423.91562851,121.21767006)(424.87563514,121.2176718)
\curveto(425.39562703,121.21767006)(425.88562654,121.30766997)(426.34563514,121.4876718)
\curveto(426.80562562,121.66766961)(427.19562523,121.91766936)(427.51563514,122.2376718)
\curveto(427.85562457,122.5776687)(428.11562431,122.9776683)(428.29563514,123.4376718)
\curveto(428.49562393,123.89766738)(428.59562383,124.41766686)(428.59563514,124.9976718)
\curveto(428.59562383,125.75766552)(428.44562398,126.3776649)(428.14563514,126.8576718)
\curveto(427.86562456,127.33766394)(427.48562494,127.73766354)(427.00563514,128.0576718)
\curveto(426.54562588,128.39766288)(426.00562642,128.66766261)(425.38563514,128.8676718)
\curveto(424.78562764,129.08766219)(424.16562826,129.30766197)(423.52563514,129.5276718)
\curveto(422.90562952,129.74766153)(422.28563014,129.98766129)(421.66563514,130.2476718)
\curveto(421.06563136,130.52766075)(420.5256319,130.8776604)(420.04563514,131.2976718)
\curveto(419.58563284,131.73765954)(419.20563322,132.28765899)(418.90563514,132.9476718)
\curveto(418.6256338,133.60765767)(418.48563394,134.42765685)(418.48563514,135.4076718)
\curveto(418.48563394,135.90765537)(418.55563387,136.48765479)(418.69563514,137.1476718)
\curveto(418.83563359,137.82765345)(419.1256333,138.4776528)(419.56563514,139.0976718)
\curveto(420.0256324,139.71765156)(420.66563176,140.23765104)(421.48563514,140.6576718)
\curveto(422.30563012,141.09765018)(423.39562903,141.31764996)(424.75563514,141.3176718)
\curveto(426.81562561,141.31764996)(428.31562411,140.81765046)(429.25563514,139.8176718)
\curveto(430.21562221,138.83765244)(430.71562171,137.34765393)(430.75563514,135.3476718)
\lineto(428.11563514,135.3476718)
}
}
{
\newrgbcolor{curcolor}{0 0 0}
\pscustom[linestyle=none,fillstyle=solid,fillcolor=curcolor]
{
\newpath
\moveto(440.53563514,119.4776718)
\lineto(437.89563514,119.4776718)
\lineto(437.89563514,138.5576718)
\lineto(432.52563514,138.5576718)
\lineto(432.52563514,140.8976718)
\lineto(445.93563514,140.8976718)
\lineto(445.93563514,138.5576718)
\lineto(440.53563514,138.5576718)
\lineto(440.53563514,119.4776718)
}
}
{
\newrgbcolor{curcolor}{0 0 0}
\pscustom[linestyle=none,fillstyle=solid,fillcolor=curcolor]
{
\newpath
\moveto(446.17235389,117.2276718)
\lineto(461.17235389,117.2276718)
\lineto(461.17235389,115.7276718)
\lineto(446.17235389,115.7276718)
\lineto(446.17235389,117.2276718)
}
}
{
\newrgbcolor{curcolor}{0 0 0}
\pscustom[linestyle=none,fillstyle=solid,fillcolor=curcolor]
{
\newpath
\moveto(461.11235389,140.8976718)
\lineto(463.87235389,140.8976718)
\lineto(468.07235389,122.5076718)
\lineto(468.13235389,122.5076718)
\lineto(472.33235389,140.8976718)
\lineto(475.09235389,140.8976718)
\lineto(469.69235389,119.4776718)
\lineto(466.33235389,119.4776718)
\lineto(461.11235389,140.8976718)
}
}
{
\newrgbcolor{curcolor}{0 0 0}
\pscustom[linestyle=none,fillstyle=solid,fillcolor=curcolor]
{
\newpath
\moveto(478.62508826,127.5176718)
\lineto(484.38508826,127.5176718)
\lineto(481.65508826,137.8676718)
\lineto(481.59508826,137.8676718)
\lineto(478.62508826,127.5176718)
\moveto(479.91508826,140.8976718)
\lineto(483.45508826,140.8976718)
\lineto(489.21508826,119.4776718)
\lineto(486.45508826,119.4776718)
\lineto(484.92508826,125.3576718)
\lineto(478.08508826,125.3576718)
\lineto(476.49508826,119.4776718)
\lineto(473.73508826,119.4776718)
\lineto(479.91508826,140.8976718)
}
}
{
\newrgbcolor{curcolor}{0 0 0}
\pscustom[linestyle=none,fillstyle=solid,fillcolor=curcolor]
{
\newpath
\moveto(490.74438514,140.8976718)
\lineto(493.38438514,140.8976718)
\lineto(493.38438514,121.8176718)
\lineto(502.08438514,121.8176718)
\lineto(502.08438514,119.4776718)
\lineto(490.74438514,119.4776718)
\lineto(490.74438514,140.8976718)
}
}
{
\newrgbcolor{curcolor}{0 0 0}
\pscustom[linestyle=none,fillstyle=solid,fillcolor=curcolor]
{
\newpath
\moveto(503.80516639,140.8976718)
\lineto(506.44516639,140.8976718)
\lineto(506.44516639,125.8376718)
\curveto(506.44516225,124.25766702)(506.71516198,123.08766819)(507.25516639,122.3276718)
\curveto(507.81516088,121.58766969)(508.75515994,121.21767006)(510.07516639,121.2176718)
\curveto(511.47515722,121.21767006)(512.43515626,121.60766967)(512.95516639,122.3876718)
\curveto(513.47515522,123.18766809)(513.73515496,124.33766694)(513.73516639,125.8376718)
\lineto(513.73516639,140.8976718)
\lineto(516.37516639,140.8976718)
\lineto(516.37516639,125.8376718)
\curveto(516.37515232,123.7776675)(515.84515285,122.12766915)(514.78516639,120.8876718)
\curveto(513.74515495,119.66767161)(512.17515652,119.05767222)(510.07516639,119.0576718)
\curveto(507.8951608,119.05767222)(506.30516239,119.62767165)(505.30516639,120.7676718)
\curveto(504.30516439,121.90766937)(503.80516489,123.59766768)(503.80516639,125.8376718)
\lineto(503.80516639,140.8976718)
}
}
{
\newrgbcolor{curcolor}{0 0 0}
\pscustom[linestyle=none,fillstyle=solid,fillcolor=curcolor]
{
\newpath
\moveto(519.63110389,140.8976718)
\lineto(530.70110389,140.8976718)
\lineto(530.70110389,138.5576718)
\lineto(522.27110389,138.5576718)
\lineto(522.27110389,131.8376718)
\lineto(530.22110389,131.8376718)
\lineto(530.22110389,129.4976718)
\lineto(522.27110389,129.4976718)
\lineto(522.27110389,121.8176718)
\lineto(531.06110389,121.8176718)
\lineto(531.06110389,119.4776718)
\lineto(519.63110389,119.4776718)
\lineto(519.63110389,140.8976718)
}
}
{
\newrgbcolor{curcolor}{0 0 0}
\pscustom[linewidth=2,linecolor=curcolor]
{
\newpath
\moveto(549.30539771,256.13856209)
\lineto(744.93322058,256.13856209)
\lineto(744.93322058,207.98792732)
\lineto(549.30539771,207.98792732)
\closepath
}
}
{
\newrgbcolor{curcolor}{0 0 0}
\pscustom[linestyle=none,fillstyle=solid,fillcolor=curcolor]
{
\newpath
\moveto(581.76877295,238.4982618)
\curveto(581.72876108,239.05824576)(581.63876117,239.58824523)(581.49877295,240.0882618)
\curveto(581.37876143,240.60824421)(581.17876163,241.05824376)(580.89877295,241.4382618)
\curveto(580.63876217,241.818243)(580.28876252,242.1182427)(579.84877295,242.3382618)
\curveto(579.4087634,242.57824224)(578.85876395,242.69824212)(578.19877295,242.6982618)
\curveto(577.27876553,242.69824212)(576.54876626,242.45824236)(576.00877295,241.9782618)
\curveto(575.46876734,241.5182433)(575.04876776,240.87824394)(574.74877295,240.0582618)
\curveto(574.46876834,239.25824556)(574.27876853,238.30824651)(574.17877295,237.2082618)
\curveto(574.09876871,236.12824869)(574.05876875,234.96824985)(574.05877295,233.7282618)
\curveto(574.05876875,232.48825233)(574.09876871,231.3182535)(574.17877295,230.2182618)
\curveto(574.27876853,229.13825568)(574.46876834,228.18825663)(574.74877295,227.3682618)
\curveto(575.04876776,226.56825825)(575.46876734,225.92825889)(576.00877295,225.4482618)
\curveto(576.54876626,224.98825983)(577.27876553,224.75826006)(578.19877295,224.7582618)
\curveto(579.11876369,224.75826006)(579.83876297,224.99825982)(580.35877295,225.4782618)
\curveto(580.87876193,225.97825884)(581.26876154,226.58825823)(581.52877295,227.3082618)
\curveto(581.78876102,228.02825679)(581.94876086,228.80825601)(582.00877295,229.6482618)
\curveto(582.08876072,230.48825433)(582.12876068,231.26825355)(582.12877295,231.9882618)
\lineto(577.89877295,231.9882618)
\lineto(577.89877295,234.1482618)
\lineto(584.52877295,234.1482618)
\lineto(584.52877295,223.0182618)
\lineto(582.54877295,223.0182618)
\lineto(582.54877295,225.9282618)
\lineto(582.48877295,225.9282618)
\curveto(582.2087606,225.00825981)(581.66876114,224.2182606)(580.86877295,223.5582618)
\curveto(580.06876274,222.9182619)(579.06876374,222.59826222)(577.86877295,222.5982618)
\curveto(576.46876634,222.59826222)(575.33876747,222.90826191)(574.47877295,223.5282618)
\curveto(573.61876919,224.14826067)(572.94876986,224.96825985)(572.46877295,225.9882618)
\curveto(572.0087708,227.02825779)(571.69877111,228.2182566)(571.53877295,229.5582618)
\curveto(571.37877143,230.89825392)(571.29877151,232.28825253)(571.29877295,233.7282618)
\curveto(571.29877151,235.06824975)(571.37877143,236.39824842)(571.53877295,237.7182618)
\curveto(571.69877111,239.05824576)(572.02877078,240.25824456)(572.52877295,241.3182618)
\curveto(573.02876978,242.37824244)(573.72876908,243.22824159)(574.62877295,243.8682618)
\curveto(575.52876728,244.52824029)(576.71876609,244.85823996)(578.19877295,244.8582618)
\curveto(579.21876359,244.85823996)(580.07876273,244.72824009)(580.77877295,244.4682618)
\curveto(581.49876131,244.20824061)(582.08876072,243.86824095)(582.54877295,243.4482618)
\curveto(583.02875978,243.04824177)(583.39875941,242.59824222)(583.65877295,242.0982618)
\curveto(583.91875889,241.59824322)(584.1087587,241.10824371)(584.22877295,240.6282618)
\curveto(584.36875844,240.16824465)(584.44875836,239.73824508)(584.46877295,239.3382618)
\curveto(584.5087583,238.95824586)(584.52875828,238.67824614)(584.52877295,238.4982618)
\lineto(581.76877295,238.4982618)
}
}
{
\newrgbcolor{curcolor}{0 0 0}
\pscustom[linestyle=none,fillstyle=solid,fillcolor=curcolor]
{
\newpath
\moveto(590.3520542,234.7182618)
\lineto(592.9620542,234.7182618)
\curveto(593.34204683,234.7182501)(593.78204639,234.73825008)(594.2820542,234.7782618)
\curveto(594.80204537,234.81825)(595.29204488,234.96824985)(595.7520542,235.2282618)
\curveto(596.21204396,235.48824933)(596.59204358,235.89824892)(596.8920542,236.4582618)
\curveto(597.21204296,237.0182478)(597.3720428,237.818247)(597.3720542,238.8582618)
\curveto(597.3720428,239.93824488)(597.04204313,240.77824404)(596.3820542,241.3782618)
\curveto(595.72204445,241.97824284)(594.76204541,242.27824254)(593.5020542,242.2782618)
\lineto(590.3520542,242.2782618)
\lineto(590.3520542,234.7182618)
\moveto(587.7120542,244.4382618)
\lineto(594.6420542,244.4382618)
\curveto(596.34204383,244.43824038)(597.68204249,243.96824085)(598.6620542,243.0282618)
\curveto(599.64204053,242.08824273)(600.13204004,240.76824405)(600.1320542,239.0682618)
\curveto(600.13204004,238.48824633)(600.0720401,237.90824691)(599.9520542,237.3282618)
\curveto(599.85204032,236.74824807)(599.6720405,236.20824861)(599.4120542,235.7082618)
\curveto(599.15204102,235.22824959)(598.81204136,234.79825002)(598.3920542,234.4182618)
\curveto(597.9720422,234.05825076)(597.45204272,233.80825101)(596.8320542,233.6682618)
\lineto(596.8320542,233.6082618)
\curveto(597.7720424,233.50825131)(598.49204168,233.1182517)(598.9920542,232.4382618)
\curveto(599.51204066,231.75825306)(599.80204037,230.95825386)(599.8620542,230.0382618)
\lineto(600.0420542,226.3782618)
\curveto(600.06204011,225.77825904)(600.10204007,225.28825953)(600.1620542,224.9082618)
\curveto(600.22203995,224.52826029)(600.30203987,224.20826061)(600.4020542,223.9482618)
\curveto(600.50203967,223.70826111)(600.61203956,223.5182613)(600.7320542,223.3782618)
\curveto(600.8720393,223.23826158)(601.02203915,223.1182617)(601.1820542,223.0182618)
\lineto(598.0020542,223.0182618)
\curveto(597.88204229,223.13826168)(597.78204239,223.3182615)(597.7020542,223.5582618)
\curveto(597.62204255,223.79826102)(597.55204262,224.05826076)(597.4920542,224.3382618)
\curveto(597.43204274,224.63826018)(597.38204279,224.93825988)(597.3420542,225.2382618)
\curveto(597.32204285,225.55825926)(597.30204287,225.84825897)(597.2820542,226.1082618)
\lineto(597.1020542,229.4382618)
\curveto(597.04204313,230.17825464)(596.90204327,230.74825407)(596.6820542,231.1482618)
\curveto(596.48204369,231.56825325)(596.23204394,231.87825294)(595.9320542,232.0782618)
\curveto(595.63204454,232.29825252)(595.30204487,232.42825239)(594.9420542,232.4682618)
\curveto(594.60204557,232.52825229)(594.26204591,232.55825226)(593.9220542,232.5582618)
\lineto(590.3520542,232.5582618)
\lineto(590.3520542,223.0182618)
\lineto(587.7120542,223.0182618)
\lineto(587.7120542,244.4382618)
}
}
{
\newrgbcolor{curcolor}{0 0 0}
\pscustom[linestyle=none,fillstyle=solid,fillcolor=curcolor]
{
\newpath
\moveto(609.8979917,242.6982618)
\curveto(608.97798428,242.69824212)(608.24798501,242.45824236)(607.7079917,241.9782618)
\curveto(607.16798609,241.5182433)(606.74798651,240.87824394)(606.4479917,240.0582618)
\curveto(606.16798709,239.25824556)(605.97798728,238.30824651)(605.8779917,237.2082618)
\curveto(605.79798746,236.12824869)(605.7579875,234.96824985)(605.7579917,233.7282618)
\curveto(605.7579875,232.48825233)(605.79798746,231.3182535)(605.8779917,230.2182618)
\curveto(605.97798728,229.13825568)(606.16798709,228.18825663)(606.4479917,227.3682618)
\curveto(606.74798651,226.56825825)(607.16798609,225.92825889)(607.7079917,225.4482618)
\curveto(608.24798501,224.98825983)(608.97798428,224.75826006)(609.8979917,224.7582618)
\curveto(610.81798244,224.75826006)(611.54798171,224.98825983)(612.0879917,225.4482618)
\curveto(612.62798063,225.92825889)(613.03798022,226.56825825)(613.3179917,227.3682618)
\curveto(613.61797964,228.18825663)(613.80797945,229.13825568)(613.8879917,230.2182618)
\curveto(613.98797927,231.3182535)(614.03797922,232.48825233)(614.0379917,233.7282618)
\curveto(614.03797922,234.96824985)(613.98797927,236.12824869)(613.8879917,237.2082618)
\curveto(613.80797945,238.30824651)(613.61797964,239.25824556)(613.3179917,240.0582618)
\curveto(613.03798022,240.87824394)(612.62798063,241.5182433)(612.0879917,241.9782618)
\curveto(611.54798171,242.45824236)(610.81798244,242.69824212)(609.8979917,242.6982618)
\moveto(609.8979917,244.8582618)
\curveto(611.37798188,244.85823996)(612.56798069,244.52824029)(613.4679917,243.8682618)
\curveto(614.36797889,243.22824159)(615.06797819,242.37824244)(615.5679917,241.3182618)
\curveto(616.06797719,240.25824456)(616.39797686,239.05824576)(616.5579917,237.7182618)
\curveto(616.71797654,236.39824842)(616.79797646,235.06824975)(616.7979917,233.7282618)
\curveto(616.79797646,232.36825245)(616.71797654,231.02825379)(616.5579917,229.7082618)
\curveto(616.39797686,228.38825643)(616.06797719,227.19825762)(615.5679917,226.1382618)
\curveto(615.06797819,225.07825974)(614.36797889,224.2182606)(613.4679917,223.5582618)
\curveto(612.56798069,222.9182619)(611.37798188,222.59826222)(609.8979917,222.5982618)
\curveto(608.41798484,222.59826222)(607.22798603,222.9182619)(606.3279917,223.5582618)
\curveto(605.42798783,224.2182606)(604.72798853,225.07825974)(604.2279917,226.1382618)
\curveto(603.72798953,227.19825762)(603.39798986,228.38825643)(603.2379917,229.7082618)
\curveto(603.07799018,231.02825379)(602.99799026,232.36825245)(602.9979917,233.7282618)
\curveto(602.99799026,235.06824975)(603.07799018,236.39824842)(603.2379917,237.7182618)
\curveto(603.39798986,239.05824576)(603.72798953,240.25824456)(604.2279917,241.3182618)
\curveto(604.72798853,242.37824244)(605.42798783,243.22824159)(606.3279917,243.8682618)
\curveto(607.22798603,244.52824029)(608.41798484,244.85823996)(609.8979917,244.8582618)
}
}
{
\newrgbcolor{curcolor}{0 0 0}
\pscustom[linestyle=none,fillstyle=solid,fillcolor=curcolor]
{
\newpath
\moveto(619.75721045,244.4382618)
\lineto(622.39721045,244.4382618)
\lineto(622.39721045,229.3782618)
\curveto(622.39720631,227.79825702)(622.66720604,226.62825819)(623.20721045,225.8682618)
\curveto(623.76720494,225.12825969)(624.707204,224.75826006)(626.02721045,224.7582618)
\curveto(627.42720128,224.75826006)(628.38720032,225.14825967)(628.90721045,225.9282618)
\curveto(629.42719928,226.72825809)(629.68719902,227.87825694)(629.68721045,229.3782618)
\lineto(629.68721045,244.4382618)
\lineto(632.32721045,244.4382618)
\lineto(632.32721045,229.3782618)
\curveto(632.32719638,227.3182575)(631.79719691,225.66825915)(630.73721045,224.4282618)
\curveto(629.69719901,223.20826161)(628.12720058,222.59826222)(626.02721045,222.5982618)
\curveto(623.84720486,222.59826222)(622.25720645,223.16826165)(621.25721045,224.3082618)
\curveto(620.25720845,225.44825937)(619.75720895,227.13825768)(619.75721045,229.3782618)
\lineto(619.75721045,244.4382618)
}
}
{
\newrgbcolor{curcolor}{0 0 0}
\pscustom[linestyle=none,fillstyle=solid,fillcolor=curcolor]
{
\newpath
\moveto(638.22314795,234.2382618)
\lineto(641.37314795,234.2382618)
\curveto(642.33313946,234.23825058)(643.16313863,234.57825024)(643.86314795,235.2582618)
\curveto(644.56313723,235.93824888)(644.91313688,236.98824783)(644.91314795,238.4082618)
\curveto(644.91313688,239.60824521)(644.62313717,240.54824427)(644.04314795,241.2282618)
\curveto(643.46313833,241.92824289)(642.51313928,242.27824254)(641.19314795,242.2782618)
\lineto(638.22314795,242.2782618)
\lineto(638.22314795,234.2382618)
\moveto(635.58314795,244.4382618)
\lineto(641.04314795,244.4382618)
\curveto(641.34314045,244.43824038)(641.71314008,244.42824039)(642.15314795,244.4082618)
\curveto(642.61313918,244.38824043)(643.08313871,244.3182405)(643.56314795,244.1982618)
\curveto(644.06313773,244.09824072)(644.55313724,243.9182409)(645.03314795,243.6582618)
\curveto(645.53313626,243.4182414)(645.97313582,243.06824175)(646.35314795,242.6082618)
\curveto(646.75313504,242.14824267)(647.07313472,241.56824325)(647.31314795,240.8682618)
\curveto(647.55313424,240.16824465)(647.67313412,239.30824551)(647.67314795,238.2882618)
\curveto(647.67313412,237.28824753)(647.52313427,236.39824842)(647.22314795,235.6182618)
\curveto(646.92313487,234.85824996)(646.4931353,234.20825061)(645.93314795,233.6682618)
\curveto(645.3931364,233.14825167)(644.74313705,232.74825207)(643.98314795,232.4682618)
\curveto(643.22313857,232.20825261)(642.3931394,232.07825274)(641.49314795,232.0782618)
\lineto(638.22314795,232.0782618)
\lineto(638.22314795,223.0182618)
\lineto(635.58314795,223.0182618)
\lineto(635.58314795,244.4382618)
}
}
{
\newrgbcolor{curcolor}{0 0 0}
\pscustom[linestyle=none,fillstyle=solid,fillcolor=curcolor]
{
\newpath
\moveto(648.25721045,220.7682618)
\lineto(663.25721045,220.7682618)
\lineto(663.25721045,219.2682618)
\lineto(648.25721045,219.2682618)
\lineto(648.25721045,220.7682618)
}
}
{
\newrgbcolor{curcolor}{0 0 0}
\pscustom[linestyle=none,fillstyle=solid,fillcolor=curcolor]
{
\newpath
\moveto(664.75721045,244.4382618)
\lineto(667.39721045,244.4382618)
\lineto(667.39721045,229.3782618)
\curveto(667.39720631,227.79825702)(667.66720604,226.62825819)(668.20721045,225.8682618)
\curveto(668.76720494,225.12825969)(669.707204,224.75826006)(671.02721045,224.7582618)
\curveto(672.42720128,224.75826006)(673.38720032,225.14825967)(673.90721045,225.9282618)
\curveto(674.42719928,226.72825809)(674.68719902,227.87825694)(674.68721045,229.3782618)
\lineto(674.68721045,244.4382618)
\lineto(677.32721045,244.4382618)
\lineto(677.32721045,229.3782618)
\curveto(677.32719638,227.3182575)(676.79719691,225.66825915)(675.73721045,224.4282618)
\curveto(674.69719901,223.20826161)(673.12720058,222.59826222)(671.02721045,222.5982618)
\curveto(668.84720486,222.59826222)(667.25720645,223.16826165)(666.25721045,224.3082618)
\curveto(665.25720845,225.44825937)(664.75720895,227.13825768)(664.75721045,229.3782618)
\lineto(664.75721045,244.4382618)
}
}
{
\newrgbcolor{curcolor}{0 0 0}
\pscustom[linestyle=none,fillstyle=solid,fillcolor=curcolor]
{
\newpath
\moveto(689.67314795,238.8882618)
\curveto(689.67313712,239.44824537)(689.61313718,239.95824486)(689.49314795,240.4182618)
\curveto(689.3931374,240.89824392)(689.21313758,241.29824352)(688.95314795,241.6182618)
\curveto(688.6931381,241.95824286)(688.34313845,242.2182426)(687.90314795,242.3982618)
\curveto(687.48313931,242.59824222)(686.96313983,242.69824212)(686.34314795,242.6982618)
\curveto(685.20314159,242.69824212)(684.32314247,242.39824242)(683.70314795,241.7982618)
\curveto(683.10314369,241.2182436)(682.80314399,240.35824446)(682.80314795,239.2182618)
\curveto(682.80314399,238.2182466)(683.05314374,237.44824737)(683.55314795,236.9082618)
\curveto(684.05314274,236.36824845)(684.67314212,235.93824888)(685.41314795,235.6182618)
\curveto(686.17314062,235.29824952)(686.98313981,235.00824981)(687.84314795,234.7482618)
\curveto(688.72313807,234.50825031)(689.53313726,234.17825064)(690.27314795,233.7582618)
\curveto(691.03313576,233.33825148)(691.66313513,232.75825206)(692.16314795,232.0182618)
\curveto(692.66313413,231.27825354)(692.91313388,230.25825456)(692.91314795,228.9582618)
\curveto(692.91313388,227.7182571)(692.70313409,226.68825813)(692.28314795,225.8682618)
\curveto(691.88313491,225.04825977)(691.36313543,224.39826042)(690.72314795,223.9182618)
\curveto(690.08313671,223.43826138)(689.36313743,223.09826172)(688.56314795,222.8982618)
\curveto(687.78313901,222.69826212)(687.01313978,222.59826222)(686.25314795,222.5982618)
\curveto(684.9931418,222.59826222)(683.94314285,222.75826206)(683.10314795,223.0782618)
\curveto(682.28314451,223.39826142)(681.62314517,223.85826096)(681.12314795,224.4582618)
\curveto(680.64314615,225.05825976)(680.2931465,225.79825902)(680.07314795,226.6782618)
\curveto(679.87314692,227.57825724)(679.77314702,228.59825622)(679.77314795,229.7382618)
\lineto(682.41314795,229.7382618)
\curveto(682.41314438,229.13825568)(682.44314435,228.53825628)(682.50314795,227.9382618)
\curveto(682.56314423,227.35825746)(682.72314407,226.82825799)(682.98314795,226.3482618)
\curveto(683.24314355,225.86825895)(683.64314315,225.47825934)(684.18314795,225.1782618)
\curveto(684.72314207,224.89825992)(685.47314132,224.75826006)(686.43314795,224.7582618)
\curveto(686.95313984,224.75826006)(687.44313935,224.84825997)(687.90314795,225.0282618)
\curveto(688.36313843,225.20825961)(688.75313804,225.45825936)(689.07314795,225.7782618)
\curveto(689.41313738,226.1182587)(689.67313712,226.5182583)(689.85314795,226.9782618)
\curveto(690.05313674,227.43825738)(690.15313664,227.95825686)(690.15314795,228.5382618)
\curveto(690.15313664,229.29825552)(690.00313679,229.9182549)(689.70314795,230.3982618)
\curveto(689.42313737,230.87825394)(689.04313775,231.27825354)(688.56314795,231.5982618)
\curveto(688.10313869,231.93825288)(687.56313923,232.20825261)(686.94314795,232.4082618)
\curveto(686.34314045,232.62825219)(685.72314107,232.84825197)(685.08314795,233.0682618)
\curveto(684.46314233,233.28825153)(683.84314295,233.52825129)(683.22314795,233.7882618)
\curveto(682.62314417,234.06825075)(682.08314471,234.4182504)(681.60314795,234.8382618)
\curveto(681.14314565,235.27824954)(680.76314603,235.82824899)(680.46314795,236.4882618)
\curveto(680.18314661,237.14824767)(680.04314675,237.96824685)(680.04314795,238.9482618)
\curveto(680.04314675,239.44824537)(680.11314668,240.02824479)(680.25314795,240.6882618)
\curveto(680.3931464,241.36824345)(680.68314611,242.0182428)(681.12314795,242.6382618)
\curveto(681.58314521,243.25824156)(682.22314457,243.77824104)(683.04314795,244.1982618)
\curveto(683.86314293,244.63824018)(684.95314184,244.85823996)(686.31314795,244.8582618)
\curveto(688.37313842,244.85823996)(689.87313692,244.35824046)(690.81314795,243.3582618)
\curveto(691.77313502,242.37824244)(692.27313452,240.88824393)(692.31314795,238.8882618)
\lineto(689.67314795,238.8882618)
}
}
{
\newrgbcolor{curcolor}{0 0 0}
\pscustom[linestyle=none,fillstyle=solid,fillcolor=curcolor]
{
\newpath
\moveto(695.58314795,244.4382618)
\lineto(706.65314795,244.4382618)
\lineto(706.65314795,242.0982618)
\lineto(698.22314795,242.0982618)
\lineto(698.22314795,235.3782618)
\lineto(706.17314795,235.3782618)
\lineto(706.17314795,233.0382618)
\lineto(698.22314795,233.0382618)
\lineto(698.22314795,225.3582618)
\lineto(707.01314795,225.3582618)
\lineto(707.01314795,223.0182618)
\lineto(695.58314795,223.0182618)
\lineto(695.58314795,244.4382618)
}
}
{
\newrgbcolor{curcolor}{0 0 0}
\pscustom[linestyle=none,fillstyle=solid,fillcolor=curcolor]
{
\newpath
\moveto(712.1098667,234.7182618)
\lineto(714.7198667,234.7182618)
\curveto(715.09985933,234.7182501)(715.53985889,234.73825008)(716.0398667,234.7782618)
\curveto(716.55985787,234.81825)(717.04985738,234.96824985)(717.5098667,235.2282618)
\curveto(717.96985646,235.48824933)(718.34985608,235.89824892)(718.6498667,236.4582618)
\curveto(718.96985546,237.0182478)(719.1298553,237.818247)(719.1298667,238.8582618)
\curveto(719.1298553,239.93824488)(718.79985563,240.77824404)(718.1398667,241.3782618)
\curveto(717.47985695,241.97824284)(716.51985791,242.27824254)(715.2598667,242.2782618)
\lineto(712.1098667,242.2782618)
\lineto(712.1098667,234.7182618)
\moveto(709.4698667,244.4382618)
\lineto(716.3998667,244.4382618)
\curveto(718.09985633,244.43824038)(719.43985499,243.96824085)(720.4198667,243.0282618)
\curveto(721.39985303,242.08824273)(721.88985254,240.76824405)(721.8898667,239.0682618)
\curveto(721.88985254,238.48824633)(721.8298526,237.90824691)(721.7098667,237.3282618)
\curveto(721.60985282,236.74824807)(721.429853,236.20824861)(721.1698667,235.7082618)
\curveto(720.90985352,235.22824959)(720.56985386,234.79825002)(720.1498667,234.4182618)
\curveto(719.7298547,234.05825076)(719.20985522,233.80825101)(718.5898667,233.6682618)
\lineto(718.5898667,233.6082618)
\curveto(719.5298549,233.50825131)(720.24985418,233.1182517)(720.7498667,232.4382618)
\curveto(721.26985316,231.75825306)(721.55985287,230.95825386)(721.6198667,230.0382618)
\lineto(721.7998667,226.3782618)
\curveto(721.81985261,225.77825904)(721.85985257,225.28825953)(721.9198667,224.9082618)
\curveto(721.97985245,224.52826029)(722.05985237,224.20826061)(722.1598667,223.9482618)
\curveto(722.25985217,223.70826111)(722.36985206,223.5182613)(722.4898667,223.3782618)
\curveto(722.6298518,223.23826158)(722.77985165,223.1182617)(722.9398667,223.0182618)
\lineto(719.7598667,223.0182618)
\curveto(719.63985479,223.13826168)(719.53985489,223.3182615)(719.4598667,223.5582618)
\curveto(719.37985505,223.79826102)(719.30985512,224.05826076)(719.2498667,224.3382618)
\curveto(719.18985524,224.63826018)(719.13985529,224.93825988)(719.0998667,225.2382618)
\curveto(719.07985535,225.55825926)(719.05985537,225.84825897)(719.0398667,226.1082618)
\lineto(718.8598667,229.4382618)
\curveto(718.79985563,230.17825464)(718.65985577,230.74825407)(718.4398667,231.1482618)
\curveto(718.23985619,231.56825325)(717.98985644,231.87825294)(717.6898667,232.0782618)
\curveto(717.38985704,232.29825252)(717.05985737,232.42825239)(716.6998667,232.4682618)
\curveto(716.35985807,232.52825229)(716.01985841,232.55825226)(715.6798667,232.5582618)
\lineto(712.1098667,232.5582618)
\lineto(712.1098667,223.0182618)
\lineto(709.4698667,223.0182618)
\lineto(709.4698667,244.4382618)
}
}
{
\newrgbcolor{curcolor}{0 0 0}
\pscustom[linewidth=2,linecolor=curcolor]
{
\newpath
\moveto(26.04641171,78.35181209)
\lineto(143.89248867,78.35181209)
\lineto(143.89248867,30.20117732)
\lineto(26.04641171,30.20117732)
\closepath
}
}
{
\newrgbcolor{curcolor}{0 0 0}
\pscustom[linestyle=none,fillstyle=solid,fillcolor=curcolor]
{
\newpath
\moveto(55.87810727,64.98647273)
\lineto(58.51810727,64.98647273)
\lineto(58.51810727,49.92647273)
\curveto(58.51810313,48.34646795)(58.78810286,47.17646912)(59.32810727,46.41647273)
\curveto(59.88810176,45.67647062)(60.82810082,45.30647099)(62.14810727,45.30647273)
\curveto(63.5480981,45.30647099)(64.50809714,45.6964706)(65.02810727,46.47647273)
\curveto(65.5480961,47.27646902)(65.80809584,48.42646787)(65.80810727,49.92647273)
\lineto(65.80810727,64.98647273)
\lineto(68.44810727,64.98647273)
\lineto(68.44810727,49.92647273)
\curveto(68.4480932,47.86646843)(67.91809373,46.21647008)(66.85810727,44.97647273)
\curveto(65.81809583,43.75647254)(64.2480974,43.14647315)(62.14810727,43.14647273)
\curveto(59.96810168,43.14647315)(58.37810327,43.71647258)(57.37810727,44.85647273)
\curveto(56.37810527,45.9964703)(55.87810577,47.68646861)(55.87810727,49.92647273)
\lineto(55.87810727,64.98647273)
}
}
{
\newrgbcolor{curcolor}{0 0 0}
\pscustom[linestyle=none,fillstyle=solid,fillcolor=curcolor]
{
\newpath
\moveto(80.79404477,59.43647273)
\curveto(80.79403394,59.9964563)(80.734034,60.50645579)(80.61404477,60.96647273)
\curveto(80.51403422,61.44645485)(80.3340344,61.84645445)(80.07404477,62.16647273)
\curveto(79.81403492,62.50645379)(79.46403527,62.76645353)(79.02404477,62.94647273)
\curveto(78.60403613,63.14645315)(78.08403665,63.24645305)(77.46404477,63.24647273)
\curveto(76.32403841,63.24645305)(75.44403929,62.94645335)(74.82404477,62.34647273)
\curveto(74.22404051,61.76645453)(73.92404081,60.90645539)(73.92404477,59.76647273)
\curveto(73.92404081,58.76645753)(74.17404056,57.9964583)(74.67404477,57.45647273)
\curveto(75.17403956,56.91645938)(75.79403894,56.48645981)(76.53404477,56.16647273)
\curveto(77.29403744,55.84646045)(78.10403663,55.55646074)(78.96404477,55.29647273)
\curveto(79.84403489,55.05646124)(80.65403408,54.72646157)(81.39404477,54.30647273)
\curveto(82.15403258,53.88646241)(82.78403195,53.30646299)(83.28404477,52.56647273)
\curveto(83.78403095,51.82646447)(84.0340307,50.80646549)(84.03404477,49.50647273)
\curveto(84.0340307,48.26646803)(83.82403091,47.23646906)(83.40404477,46.41647273)
\curveto(83.00403173,45.5964707)(82.48403225,44.94647135)(81.84404477,44.46647273)
\curveto(81.20403353,43.98647231)(80.48403425,43.64647265)(79.68404477,43.44647273)
\curveto(78.90403583,43.24647305)(78.1340366,43.14647315)(77.37404477,43.14647273)
\curveto(76.11403862,43.14647315)(75.06403967,43.30647299)(74.22404477,43.62647273)
\curveto(73.40404133,43.94647235)(72.74404199,44.40647189)(72.24404477,45.00647273)
\curveto(71.76404297,45.60647069)(71.41404332,46.34646995)(71.19404477,47.22647273)
\curveto(70.99404374,48.12646817)(70.89404384,49.14646715)(70.89404477,50.28647273)
\lineto(73.53404477,50.28647273)
\curveto(73.5340412,49.68646661)(73.56404117,49.08646721)(73.62404477,48.48647273)
\curveto(73.68404105,47.90646839)(73.84404089,47.37646892)(74.10404477,46.89647273)
\curveto(74.36404037,46.41646988)(74.76403997,46.02647027)(75.30404477,45.72647273)
\curveto(75.84403889,45.44647085)(76.59403814,45.30647099)(77.55404477,45.30647273)
\curveto(78.07403666,45.30647099)(78.56403617,45.3964709)(79.02404477,45.57647273)
\curveto(79.48403525,45.75647054)(79.87403486,46.00647029)(80.19404477,46.32647273)
\curveto(80.5340342,46.66646963)(80.79403394,47.06646923)(80.97404477,47.52647273)
\curveto(81.17403356,47.98646831)(81.27403346,48.50646779)(81.27404477,49.08647273)
\curveto(81.27403346,49.84646645)(81.12403361,50.46646583)(80.82404477,50.94647273)
\curveto(80.54403419,51.42646487)(80.16403457,51.82646447)(79.68404477,52.14647273)
\curveto(79.22403551,52.48646381)(78.68403605,52.75646354)(78.06404477,52.95647273)
\curveto(77.46403727,53.17646312)(76.84403789,53.3964629)(76.20404477,53.61647273)
\curveto(75.58403915,53.83646246)(74.96403977,54.07646222)(74.34404477,54.33647273)
\curveto(73.74404099,54.61646168)(73.20404153,54.96646133)(72.72404477,55.38647273)
\curveto(72.26404247,55.82646047)(71.88404285,56.37645992)(71.58404477,57.03647273)
\curveto(71.30404343,57.6964586)(71.16404357,58.51645778)(71.16404477,59.49647273)
\curveto(71.16404357,59.9964563)(71.2340435,60.57645572)(71.37404477,61.23647273)
\curveto(71.51404322,61.91645438)(71.80404293,62.56645373)(72.24404477,63.18647273)
\curveto(72.70404203,63.80645249)(73.34404139,64.32645197)(74.16404477,64.74647273)
\curveto(74.98403975,65.18645111)(76.07403866,65.40645089)(77.43404477,65.40647273)
\curveto(79.49403524,65.40645089)(80.99403374,64.90645139)(81.93404477,63.90647273)
\curveto(82.89403184,62.92645337)(83.39403134,61.43645486)(83.43404477,59.43647273)
\lineto(80.79404477,59.43647273)
}
}
{
\newrgbcolor{curcolor}{0 0 0}
\pscustom[linestyle=none,fillstyle=solid,fillcolor=curcolor]
{
\newpath
\moveto(86.70404477,64.98647273)
\lineto(97.77404477,64.98647273)
\lineto(97.77404477,62.64647273)
\lineto(89.34404477,62.64647273)
\lineto(89.34404477,55.92647273)
\lineto(97.29404477,55.92647273)
\lineto(97.29404477,53.58647273)
\lineto(89.34404477,53.58647273)
\lineto(89.34404477,45.90647273)
\lineto(98.13404477,45.90647273)
\lineto(98.13404477,43.56647273)
\lineto(86.70404477,43.56647273)
\lineto(86.70404477,64.98647273)
}
}
{
\newrgbcolor{curcolor}{0 0 0}
\pscustom[linestyle=none,fillstyle=solid,fillcolor=curcolor]
{
\newpath
\moveto(103.23076352,55.26647273)
\lineto(105.84076352,55.26647273)
\curveto(106.22075615,55.26646103)(106.66075571,55.28646101)(107.16076352,55.32647273)
\curveto(107.68075469,55.36646093)(108.1707542,55.51646078)(108.63076352,55.77647273)
\curveto(109.09075328,56.03646026)(109.4707529,56.44645985)(109.77076352,57.00647273)
\curveto(110.09075228,57.56645873)(110.25075212,58.36645793)(110.25076352,59.40647273)
\curveto(110.25075212,60.48645581)(109.92075245,61.32645497)(109.26076352,61.92647273)
\curveto(108.60075377,62.52645377)(107.64075473,62.82645347)(106.38076352,62.82647273)
\lineto(103.23076352,62.82647273)
\lineto(103.23076352,55.26647273)
\moveto(100.59076352,64.98647273)
\lineto(107.52076352,64.98647273)
\curveto(109.22075315,64.98645131)(110.56075181,64.51645178)(111.54076352,63.57647273)
\curveto(112.52074985,62.63645366)(113.01074936,61.31645498)(113.01076352,59.61647273)
\curveto(113.01074936,59.03645726)(112.95074942,58.45645784)(112.83076352,57.87647273)
\curveto(112.73074964,57.296459)(112.55074982,56.75645954)(112.29076352,56.25647273)
\curveto(112.03075034,55.77646052)(111.69075068,55.34646095)(111.27076352,54.96647273)
\curveto(110.85075152,54.60646169)(110.33075204,54.35646194)(109.71076352,54.21647273)
\lineto(109.71076352,54.15647273)
\curveto(110.65075172,54.05646224)(111.370751,53.66646263)(111.87076352,52.98647273)
\curveto(112.39074998,52.30646399)(112.68074969,51.50646479)(112.74076352,50.58647273)
\lineto(112.92076352,46.92647273)
\curveto(112.94074943,46.32646997)(112.98074939,45.83647046)(113.04076352,45.45647273)
\curveto(113.10074927,45.07647122)(113.18074919,44.75647154)(113.28076352,44.49647273)
\curveto(113.38074899,44.25647204)(113.49074888,44.06647223)(113.61076352,43.92647273)
\curveto(113.75074862,43.78647251)(113.90074847,43.66647263)(114.06076352,43.56647273)
\lineto(110.88076352,43.56647273)
\curveto(110.76075161,43.68647261)(110.66075171,43.86647243)(110.58076352,44.10647273)
\curveto(110.50075187,44.34647195)(110.43075194,44.60647169)(110.37076352,44.88647273)
\curveto(110.31075206,45.18647111)(110.26075211,45.48647081)(110.22076352,45.78647273)
\curveto(110.20075217,46.10647019)(110.18075219,46.3964699)(110.16076352,46.65647273)
\lineto(109.98076352,49.98647273)
\curveto(109.92075245,50.72646557)(109.78075259,51.296465)(109.56076352,51.69647273)
\curveto(109.36075301,52.11646418)(109.11075326,52.42646387)(108.81076352,52.62647273)
\curveto(108.51075386,52.84646345)(108.18075419,52.97646332)(107.82076352,53.01647273)
\curveto(107.48075489,53.07646322)(107.14075523,53.10646319)(106.80076352,53.10647273)
\lineto(103.23076352,53.10647273)
\lineto(103.23076352,43.56647273)
\lineto(100.59076352,43.56647273)
\lineto(100.59076352,64.98647273)
}
}
{
\newrgbcolor{curcolor}{0 0 0}
\pscustom[linewidth=2,linecolor=curcolor]
{
\newpath
\moveto(631.12778171,473.82645209)
\lineto(744.93325504,473.82645209)
\lineto(744.93325504,425.67581732)
\lineto(631.12778171,425.67581732)
\closepath
}
}
{
\newrgbcolor{curcolor}{0 0 0}
\pscustom[linestyle=none,fillstyle=solid,fillcolor=curcolor]
{
\newpath
\moveto(660.31332492,454.52111273)
\curveto(660.27331305,455.08109669)(660.18331314,455.61109616)(660.04332492,456.11111273)
\curveto(659.9233134,456.63109514)(659.7233136,457.08109469)(659.44332492,457.46111273)
\curveto(659.18331414,457.84109393)(658.83331449,458.14109363)(658.39332492,458.36111273)
\curveto(657.95331537,458.60109317)(657.40331592,458.72109305)(656.74332492,458.72111273)
\curveto(655.8233175,458.72109305)(655.09331823,458.48109329)(654.55332492,458.00111273)
\curveto(654.01331931,457.54109423)(653.59331973,456.90109487)(653.29332492,456.08111273)
\curveto(653.01332031,455.28109649)(652.8233205,454.33109744)(652.72332492,453.23111273)
\curveto(652.64332068,452.15109962)(652.60332072,450.99110078)(652.60332492,449.75111273)
\curveto(652.60332072,448.51110326)(652.64332068,447.34110443)(652.72332492,446.24111273)
\curveto(652.8233205,445.16110661)(653.01332031,444.21110756)(653.29332492,443.39111273)
\curveto(653.59331973,442.59110918)(654.01331931,441.95110982)(654.55332492,441.47111273)
\curveto(655.09331823,441.01111076)(655.8233175,440.78111099)(656.74332492,440.78111273)
\curveto(657.66331566,440.78111099)(658.38331494,441.02111075)(658.90332492,441.50111273)
\curveto(659.4233139,442.00110977)(659.81331351,442.61110916)(660.07332492,443.33111273)
\curveto(660.33331299,444.05110772)(660.49331283,444.83110694)(660.55332492,445.67111273)
\curveto(660.63331269,446.51110526)(660.67331265,447.29110448)(660.67332492,448.01111273)
\lineto(656.44332492,448.01111273)
\lineto(656.44332492,450.17111273)
\lineto(663.07332492,450.17111273)
\lineto(663.07332492,439.04111273)
\lineto(661.09332492,439.04111273)
\lineto(661.09332492,441.95111273)
\lineto(661.03332492,441.95111273)
\curveto(660.75331257,441.03111074)(660.21331311,440.24111153)(659.41332492,439.58111273)
\curveto(658.61331471,438.94111283)(657.61331571,438.62111315)(656.41332492,438.62111273)
\curveto(655.01331831,438.62111315)(653.88331944,438.93111284)(653.02332492,439.55111273)
\curveto(652.16332116,440.1711116)(651.49332183,440.99111078)(651.01332492,442.01111273)
\curveto(650.55332277,443.05110872)(650.24332308,444.24110753)(650.08332492,445.58111273)
\curveto(649.9233234,446.92110485)(649.84332348,448.31110346)(649.84332492,449.75111273)
\curveto(649.84332348,451.09110068)(649.9233234,452.42109935)(650.08332492,453.74111273)
\curveto(650.24332308,455.08109669)(650.57332275,456.28109549)(651.07332492,457.34111273)
\curveto(651.57332175,458.40109337)(652.27332105,459.25109252)(653.17332492,459.89111273)
\curveto(654.07331925,460.55109122)(655.26331806,460.88109089)(656.74332492,460.88111273)
\curveto(657.76331556,460.88109089)(658.6233147,460.75109102)(659.32332492,460.49111273)
\curveto(660.04331328,460.23109154)(660.63331269,459.89109188)(661.09332492,459.47111273)
\curveto(661.57331175,459.0710927)(661.94331138,458.62109315)(662.20332492,458.12111273)
\curveto(662.46331086,457.62109415)(662.65331067,457.13109464)(662.77332492,456.65111273)
\curveto(662.91331041,456.19109558)(662.99331033,455.76109601)(663.01332492,455.36111273)
\curveto(663.05331027,454.98109679)(663.07331025,454.70109707)(663.07332492,454.52111273)
\lineto(660.31332492,454.52111273)
}
}
{
\newrgbcolor{curcolor}{0 0 0}
\pscustom[linestyle=none,fillstyle=solid,fillcolor=curcolor]
{
\newpath
\moveto(668.89660617,450.74111273)
\lineto(671.50660617,450.74111273)
\curveto(671.8865988,450.74110103)(672.32659836,450.76110101)(672.82660617,450.80111273)
\curveto(673.34659734,450.84110093)(673.83659685,450.99110078)(674.29660617,451.25111273)
\curveto(674.75659593,451.51110026)(675.13659555,451.92109985)(675.43660617,452.48111273)
\curveto(675.75659493,453.04109873)(675.91659477,453.84109793)(675.91660617,454.88111273)
\curveto(675.91659477,455.96109581)(675.5865951,456.80109497)(674.92660617,457.40111273)
\curveto(674.26659642,458.00109377)(673.30659738,458.30109347)(672.04660617,458.30111273)
\lineto(668.89660617,458.30111273)
\lineto(668.89660617,450.74111273)
\moveto(666.25660617,460.46111273)
\lineto(673.18660617,460.46111273)
\curveto(674.8865958,460.46109131)(676.22659446,459.99109178)(677.20660617,459.05111273)
\curveto(678.1865925,458.11109366)(678.67659201,456.79109498)(678.67660617,455.09111273)
\curveto(678.67659201,454.51109726)(678.61659207,453.93109784)(678.49660617,453.35111273)
\curveto(678.39659229,452.771099)(678.21659247,452.23109954)(677.95660617,451.73111273)
\curveto(677.69659299,451.25110052)(677.35659333,450.82110095)(676.93660617,450.44111273)
\curveto(676.51659417,450.08110169)(675.99659469,449.83110194)(675.37660617,449.69111273)
\lineto(675.37660617,449.63111273)
\curveto(676.31659437,449.53110224)(677.03659365,449.14110263)(677.53660617,448.46111273)
\curveto(678.05659263,447.78110399)(678.34659234,446.98110479)(678.40660617,446.06111273)
\lineto(678.58660617,442.40111273)
\curveto(678.60659208,441.80110997)(678.64659204,441.31111046)(678.70660617,440.93111273)
\curveto(678.76659192,440.55111122)(678.84659184,440.23111154)(678.94660617,439.97111273)
\curveto(679.04659164,439.73111204)(679.15659153,439.54111223)(679.27660617,439.40111273)
\curveto(679.41659127,439.26111251)(679.56659112,439.14111263)(679.72660617,439.04111273)
\lineto(676.54660617,439.04111273)
\curveto(676.42659426,439.16111261)(676.32659436,439.34111243)(676.24660617,439.58111273)
\curveto(676.16659452,439.82111195)(676.09659459,440.08111169)(676.03660617,440.36111273)
\curveto(675.97659471,440.66111111)(675.92659476,440.96111081)(675.88660617,441.26111273)
\curveto(675.86659482,441.58111019)(675.84659484,441.8711099)(675.82660617,442.13111273)
\lineto(675.64660617,445.46111273)
\curveto(675.5865951,446.20110557)(675.44659524,446.771105)(675.22660617,447.17111273)
\curveto(675.02659566,447.59110418)(674.77659591,447.90110387)(674.47660617,448.10111273)
\curveto(674.17659651,448.32110345)(673.84659684,448.45110332)(673.48660617,448.49111273)
\curveto(673.14659754,448.55110322)(672.80659788,448.58110319)(672.46660617,448.58111273)
\lineto(668.89660617,448.58111273)
\lineto(668.89660617,439.04111273)
\lineto(666.25660617,439.04111273)
\lineto(666.25660617,460.46111273)
}
}
{
\newrgbcolor{curcolor}{0 0 0}
\pscustom[linestyle=none,fillstyle=solid,fillcolor=curcolor]
{
\newpath
\moveto(688.44254367,458.72111273)
\curveto(687.52253625,458.72109305)(686.79253698,458.48109329)(686.25254367,458.00111273)
\curveto(685.71253806,457.54109423)(685.29253848,456.90109487)(684.99254367,456.08111273)
\curveto(684.71253906,455.28109649)(684.52253925,454.33109744)(684.42254367,453.23111273)
\curveto(684.34253943,452.15109962)(684.30253947,450.99110078)(684.30254367,449.75111273)
\curveto(684.30253947,448.51110326)(684.34253943,447.34110443)(684.42254367,446.24111273)
\curveto(684.52253925,445.16110661)(684.71253906,444.21110756)(684.99254367,443.39111273)
\curveto(685.29253848,442.59110918)(685.71253806,441.95110982)(686.25254367,441.47111273)
\curveto(686.79253698,441.01111076)(687.52253625,440.78111099)(688.44254367,440.78111273)
\curveto(689.36253441,440.78111099)(690.09253368,441.01111076)(690.63254367,441.47111273)
\curveto(691.1725326,441.95110982)(691.58253219,442.59110918)(691.86254367,443.39111273)
\curveto(692.16253161,444.21110756)(692.35253142,445.16110661)(692.43254367,446.24111273)
\curveto(692.53253124,447.34110443)(692.58253119,448.51110326)(692.58254367,449.75111273)
\curveto(692.58253119,450.99110078)(692.53253124,452.15109962)(692.43254367,453.23111273)
\curveto(692.35253142,454.33109744)(692.16253161,455.28109649)(691.86254367,456.08111273)
\curveto(691.58253219,456.90109487)(691.1725326,457.54109423)(690.63254367,458.00111273)
\curveto(690.09253368,458.48109329)(689.36253441,458.72109305)(688.44254367,458.72111273)
\moveto(688.44254367,460.88111273)
\curveto(689.92253385,460.88109089)(691.11253266,460.55109122)(692.01254367,459.89111273)
\curveto(692.91253086,459.25109252)(693.61253016,458.40109337)(694.11254367,457.34111273)
\curveto(694.61252916,456.28109549)(694.94252883,455.08109669)(695.10254367,453.74111273)
\curveto(695.26252851,452.42109935)(695.34252843,451.09110068)(695.34254367,449.75111273)
\curveto(695.34252843,448.39110338)(695.26252851,447.05110472)(695.10254367,445.73111273)
\curveto(694.94252883,444.41110736)(694.61252916,443.22110855)(694.11254367,442.16111273)
\curveto(693.61253016,441.10111067)(692.91253086,440.24111153)(692.01254367,439.58111273)
\curveto(691.11253266,438.94111283)(689.92253385,438.62111315)(688.44254367,438.62111273)
\curveto(686.96253681,438.62111315)(685.772538,438.94111283)(684.87254367,439.58111273)
\curveto(683.9725398,440.24111153)(683.2725405,441.10111067)(682.77254367,442.16111273)
\curveto(682.2725415,443.22110855)(681.94254183,444.41110736)(681.78254367,445.73111273)
\curveto(681.62254215,447.05110472)(681.54254223,448.39110338)(681.54254367,449.75111273)
\curveto(681.54254223,451.09110068)(681.62254215,452.42109935)(681.78254367,453.74111273)
\curveto(681.94254183,455.08109669)(682.2725415,456.28109549)(682.77254367,457.34111273)
\curveto(683.2725405,458.40109337)(683.9725398,459.25109252)(684.87254367,459.89111273)
\curveto(685.772538,460.55109122)(686.96253681,460.88109089)(688.44254367,460.88111273)
}
}
{
\newrgbcolor{curcolor}{0 0 0}
\pscustom[linestyle=none,fillstyle=solid,fillcolor=curcolor]
{
\newpath
\moveto(698.30176242,460.46111273)
\lineto(700.94176242,460.46111273)
\lineto(700.94176242,445.40111273)
\curveto(700.94175828,443.82110795)(701.21175801,442.65110912)(701.75176242,441.89111273)
\curveto(702.31175691,441.15111062)(703.25175597,440.78111099)(704.57176242,440.78111273)
\curveto(705.97175325,440.78111099)(706.93175229,441.1711106)(707.45176242,441.95111273)
\curveto(707.97175125,442.75110902)(708.23175099,443.90110787)(708.23176242,445.40111273)
\lineto(708.23176242,460.46111273)
\lineto(710.87176242,460.46111273)
\lineto(710.87176242,445.40111273)
\curveto(710.87174835,443.34110843)(710.34174888,441.69111008)(709.28176242,440.45111273)
\curveto(708.24175098,439.23111254)(706.67175255,438.62111315)(704.57176242,438.62111273)
\curveto(702.39175683,438.62111315)(700.80175842,439.19111258)(699.80176242,440.33111273)
\curveto(698.80176042,441.4711103)(698.30176092,443.16110861)(698.30176242,445.40111273)
\lineto(698.30176242,460.46111273)
}
}
{
\newrgbcolor{curcolor}{0 0 0}
\pscustom[linestyle=none,fillstyle=solid,fillcolor=curcolor]
{
\newpath
\moveto(716.76769992,450.26111273)
\lineto(719.91769992,450.26111273)
\curveto(720.87769143,450.26110151)(721.7076906,450.60110117)(722.40769992,451.28111273)
\curveto(723.1076892,451.96109981)(723.45768885,453.01109876)(723.45769992,454.43111273)
\curveto(723.45768885,455.63109614)(723.16768914,456.5710952)(722.58769992,457.25111273)
\curveto(722.0076903,457.95109382)(721.05769125,458.30109347)(719.73769992,458.30111273)
\lineto(716.76769992,458.30111273)
\lineto(716.76769992,450.26111273)
\moveto(714.12769992,460.46111273)
\lineto(719.58769992,460.46111273)
\curveto(719.88769242,460.46109131)(720.25769205,460.45109132)(720.69769992,460.43111273)
\curveto(721.15769115,460.41109136)(721.62769068,460.34109143)(722.10769992,460.22111273)
\curveto(722.6076897,460.12109165)(723.09768921,459.94109183)(723.57769992,459.68111273)
\curveto(724.07768823,459.44109233)(724.51768779,459.09109268)(724.89769992,458.63111273)
\curveto(725.29768701,458.1710936)(725.61768669,457.59109418)(725.85769992,456.89111273)
\curveto(726.09768621,456.19109558)(726.21768609,455.33109644)(726.21769992,454.31111273)
\curveto(726.21768609,453.31109846)(726.06768624,452.42109935)(725.76769992,451.64111273)
\curveto(725.46768684,450.88110089)(725.03768727,450.23110154)(724.47769992,449.69111273)
\curveto(723.93768837,449.1711026)(723.28768902,448.771103)(722.52769992,448.49111273)
\curveto(721.76769054,448.23110354)(720.93769137,448.10110367)(720.03769992,448.10111273)
\lineto(716.76769992,448.10111273)
\lineto(716.76769992,439.04111273)
\lineto(714.12769992,439.04111273)
\lineto(714.12769992,460.46111273)
}
}
{
\newrgbcolor{curcolor}{0 0 0}
\pscustom[linewidth=2,linecolor=curcolor]
{
\newpath
\moveto(145.46197,425.570085)
\lineto(145.46197,256.57874)
}
}
{
\newrgbcolor{curcolor}{0 0 0}
\pscustom[linewidth=2,linecolor=curcolor]
{
\newpath
\moveto(192.93913,451.033108)
\lineto(628.81995,451.033108)
}
}
{
\newrgbcolor{curcolor}{0 0 0}
\pscustom[linewidth=2,linecolor=curcolor]
{
\newpath
\moveto(341.43156,230.31478)
\lineto(410.12193,230.31478)
\lineto(410.12193,323.24881)
}
}
{
\newrgbcolor{curcolor}{0 0 0}
\pscustom[linewidth=2,linecolor=curcolor]
{
\newpath
\moveto(547.50268,230.31478)
\lineto(478.81231,230.31478)
\lineto(478.81231,323.24881)
}
}
{
\newrgbcolor{curcolor}{0 0 0}
\pscustom[linewidth=2,linecolor=curcolor]
{
\newpath
\moveto(679.88339,425.27422)
\lineto(679.88339,256.57874)
}
}
{
\newrgbcolor{curcolor}{0 0 0}
\pscustom[linewidth=2,linecolor=curcolor]
{
\newpath
\moveto(145.46197,207.56097)
\lineto(145.46197,129.8046)
\lineto(219.13778,129.8046)
}
}
{
\newrgbcolor{curcolor}{0 0 0}
\pscustom[linewidth=2,linecolor=curcolor]
{
\newpath
\moveto(61.428571,78.57143)
\lineto(61.428571,425.401783)
}
}
{
\newrgbcolor{curcolor}{0 0 0}
\pscustom[linewidth=2,linecolor=curcolor]
{
\newpath
\moveto(144.73214,52.85714)
\lineto(680.71428,52.85714)
\lineto(679.82142,207.94643)
}
}
{
\newrgbcolor{curcolor}{0 0 0}
\pscustom[linewidth=2,linecolor=curcolor]
{
\newpath
\moveto(56.249995,86.51786)
\lineto(66.607149,86.51786)
}
}
{
\newrgbcolor{curcolor}{0 0 0}
\pscustom[linewidth=2,linecolor=curcolor]
{
\newpath
\moveto(140.15712,199.00005)
\lineto(150.51427,199.00005)
}
}
{
\newrgbcolor{curcolor}{0 0 0}
\pscustom[linewidth=2,linecolor=curcolor]
{
\newpath
\moveto(140.26786,417.14285)
\lineto(150.62501,417.14285)
}
}
{
\newrgbcolor{curcolor}{0 0 0}
\pscustom[linewidth=2,linecolor=curcolor]
{
\newpath
\moveto(349.95472,235.49332)
\lineto(349.95472,225.13616)
}
}
{
\newrgbcolor{curcolor}{0 0 0}
\pscustom[linewidth=2,linecolor=curcolor]
{
\newpath
\moveto(540.669,235.49332)
\lineto(540.669,225.13616)
}
}
{
\newrgbcolor{curcolor}{0 0 0}
\pscustom[linewidth=2,linecolor=curcolor]
{
\newpath
\moveto(153.03812,58.08533)
\lineto(153.03812,47.72814)
}
}
{
\newrgbcolor{curcolor}{0 0 0}
\pscustom[linewidth=2,linecolor=curcolor]
{
\newpath
\moveto(674.65408,416.43538)
\lineto(685.01124,416.43538)
}
}
{
\newrgbcolor{curcolor}{1 1 1}
\pscustom[linestyle=none,fillstyle=solid,fillcolor=curcolor]
{
\newpath
\moveto(66.51634,417.41832)
\lineto(61.33931,408.89516)
\lineto(56.16228,417.41832)
\closepath
}
}
{
\newrgbcolor{curcolor}{0 0 0}
\pscustom[linewidth=2,linecolor=curcolor]
{
\newpath
\moveto(66.51634,417.41832)
\lineto(61.33931,408.89516)
\lineto(56.16228,417.41832)
\closepath
}
}
{
\newrgbcolor{curcolor}{1 1 1}
\pscustom[linestyle=none,fillstyle=solid,fillcolor=curcolor]
{
\newpath
\moveto(58.40404242,406.50477581)
\curveto(58.40404242,408.18135457)(59.76317804,409.5404902)(61.4397568,409.5404902)
\curveto(63.11633557,409.5404902)(64.47547119,408.18135457)(64.47547119,406.50477581)
\curveto(64.47547119,404.82819705)(63.11633557,403.46906143)(61.4397568,403.46906143)
\curveto(59.76317804,403.46906143)(58.40404242,404.82819705)(58.40404242,406.50477581)
\closepath
}
}
{
\newrgbcolor{curcolor}{0 0 0}
\pscustom[linewidth=2,linecolor=curcolor]
{
\newpath
\moveto(58.40404242,406.50477581)
\curveto(58.40404242,408.18135457)(59.76317804,409.5404902)(61.4397568,409.5404902)
\curveto(63.11633557,409.5404902)(64.47547119,408.18135457)(64.47547119,406.50477581)
\curveto(64.47547119,404.82819705)(63.11633557,403.46906143)(61.4397568,403.46906143)
\curveto(59.76317804,403.46906143)(58.40404242,404.82819705)(58.40404242,406.50477581)
\closepath
}
}
{
\newrgbcolor{curcolor}{0 0 0}
\pscustom[linewidth=2,linecolor=curcolor]
{
\newpath
\moveto(57.95269,417.48692)
\lineto(57.95269,424.80834)
}
}
{
\newrgbcolor{curcolor}{0 0 0}
\pscustom[linewidth=2,linecolor=curcolor]
{
\newpath
\moveto(64.74822,417.51307)
\lineto(64.74822,425.10235)
}
}
{
\newrgbcolor{curcolor}{1 1 1}
\pscustom[linestyle=none,fillstyle=solid,fillcolor=curcolor]
{
\newpath
\moveto(415.172696,314.89092)
\lineto(409.995666,306.36776)
\lineto(404.818636,314.89092)
\closepath
}
}
{
\newrgbcolor{curcolor}{0 0 0}
\pscustom[linewidth=2,linecolor=curcolor]
{
\newpath
\moveto(415.172696,314.89092)
\lineto(409.995666,306.36776)
\lineto(404.818636,314.89092)
\closepath
}
}
{
\newrgbcolor{curcolor}{1 1 1}
\pscustom[linestyle=none,fillstyle=solid,fillcolor=curcolor]
{
\newpath
\moveto(407.06039842,303.97737581)
\curveto(407.06039842,305.65395457)(408.41953404,307.0130902)(410.0961128,307.0130902)
\curveto(411.77269157,307.0130902)(413.13182719,305.65395457)(413.13182719,303.97737581)
\curveto(413.13182719,302.30079705)(411.77269157,300.94166143)(410.0961128,300.94166143)
\curveto(408.41953404,300.94166143)(407.06039842,302.30079705)(407.06039842,303.97737581)
\closepath
}
}
{
\newrgbcolor{curcolor}{0 0 0}
\pscustom[linewidth=2,linecolor=curcolor]
{
\newpath
\moveto(407.06039842,303.97737581)
\curveto(407.06039842,305.65395457)(408.41953404,307.0130902)(410.0961128,307.0130902)
\curveto(411.77269157,307.0130902)(413.13182719,305.65395457)(413.13182719,303.97737581)
\curveto(413.13182719,302.30079705)(411.77269157,300.94166143)(410.0961128,300.94166143)
\curveto(408.41953404,300.94166143)(407.06039842,302.30079705)(407.06039842,303.97737581)
\closepath
}
}
{
\newrgbcolor{curcolor}{0 0 0}
\pscustom[linewidth=2,linecolor=curcolor]
{
\newpath
\moveto(406.609046,314.95952)
\lineto(406.609046,322.28094)
}
}
{
\newrgbcolor{curcolor}{0 0 0}
\pscustom[linewidth=2,linecolor=curcolor]
{
\newpath
\moveto(413.404576,314.98567)
\lineto(413.404576,322.57495)
}
}
{
\newrgbcolor{curcolor}{1 1 1}
\pscustom[linestyle=none,fillstyle=solid,fillcolor=curcolor]
{
\newpath
\moveto(483.98934,314.89092)
\lineto(478.81231,306.36776)
\lineto(473.63528,314.89092)
\closepath
}
}
{
\newrgbcolor{curcolor}{0 0 0}
\pscustom[linewidth=2,linecolor=curcolor]
{
\newpath
\moveto(483.98934,314.89092)
\lineto(478.81231,306.36776)
\lineto(473.63528,314.89092)
\closepath
}
}
{
\newrgbcolor{curcolor}{1 1 1}
\pscustom[linestyle=none,fillstyle=solid,fillcolor=curcolor]
{
\newpath
\moveto(475.87704242,303.97737581)
\curveto(475.87704242,305.65395457)(477.23617804,307.0130902)(478.9127568,307.0130902)
\curveto(480.58933557,307.0130902)(481.94847119,305.65395457)(481.94847119,303.97737581)
\curveto(481.94847119,302.30079705)(480.58933557,300.94166143)(478.9127568,300.94166143)
\curveto(477.23617804,300.94166143)(475.87704242,302.30079705)(475.87704242,303.97737581)
\closepath
}
}
{
\newrgbcolor{curcolor}{0 0 0}
\pscustom[linewidth=2,linecolor=curcolor]
{
\newpath
\moveto(475.87704242,303.97737581)
\curveto(475.87704242,305.65395457)(477.23617804,307.0130902)(478.9127568,307.0130902)
\curveto(480.58933557,307.0130902)(481.94847119,305.65395457)(481.94847119,303.97737581)
\curveto(481.94847119,302.30079705)(480.58933557,300.94166143)(478.9127568,300.94166143)
\curveto(477.23617804,300.94166143)(475.87704242,302.30079705)(475.87704242,303.97737581)
\closepath
}
}
{
\newrgbcolor{curcolor}{0 0 0}
\pscustom[linewidth=2,linecolor=curcolor]
{
\newpath
\moveto(475.42569,314.95952)
\lineto(475.42569,322.28094)
}
}
{
\newrgbcolor{curcolor}{0 0 0}
\pscustom[linewidth=2,linecolor=curcolor]
{
\newpath
\moveto(482.22122,314.98567)
\lineto(482.22122,322.57495)
}
}
{
\newrgbcolor{curcolor}{1 1 1}
\pscustom[linestyle=none,fillstyle=solid,fillcolor=curcolor]
{
\newpath
\moveto(674.82297,264.18881)
\lineto(680,272.71197)
\lineto(685.17703,264.18881)
\closepath
}
}
{
\newrgbcolor{curcolor}{0 0 0}
\pscustom[linewidth=2,linecolor=curcolor]
{
\newpath
\moveto(674.82297,264.18881)
\lineto(680,272.71197)
\lineto(685.17703,264.18881)
\closepath
}
}
{
\newrgbcolor{curcolor}{1 1 1}
\pscustom[linestyle=none,fillstyle=solid,fillcolor=curcolor]
{
\newpath
\moveto(682.93526758,275.10235419)
\curveto(682.93526758,273.42577543)(681.57613196,272.0666398)(679.8995532,272.0666398)
\curveto(678.22297443,272.0666398)(676.86383881,273.42577543)(676.86383881,275.10235419)
\curveto(676.86383881,276.77893295)(678.22297443,278.13806857)(679.8995532,278.13806857)
\curveto(681.57613196,278.13806857)(682.93526758,276.77893295)(682.93526758,275.10235419)
\closepath
}
}
{
\newrgbcolor{curcolor}{0 0 0}
\pscustom[linewidth=2,linecolor=curcolor]
{
\newpath
\moveto(682.93526758,275.10235419)
\curveto(682.93526758,273.42577543)(681.57613196,272.0666398)(679.8995532,272.0666398)
\curveto(678.22297443,272.0666398)(676.86383881,273.42577543)(676.86383881,275.10235419)
\curveto(676.86383881,276.77893295)(678.22297443,278.13806857)(679.8995532,278.13806857)
\curveto(681.57613196,278.13806857)(682.93526758,276.77893295)(682.93526758,275.10235419)
\closepath
}
}
{
\newrgbcolor{curcolor}{0 0 0}
\pscustom[linewidth=2,linecolor=curcolor]
{
\newpath
\moveto(683.38662,264.12021)
\lineto(683.38662,256.79879)
}
}
{
\newrgbcolor{curcolor}{0 0 0}
\pscustom[linewidth=2,linecolor=curcolor]
{
\newpath
\moveto(676.59109,264.09406)
\lineto(676.59109,256.50478)
}
}
{
\newrgbcolor{curcolor}{1 1 1}
\pscustom[linestyle=none,fillstyle=solid,fillcolor=curcolor]
{
\newpath
\moveto(674.82297,200.27532)
\lineto(680,191.75216)
\lineto(685.17703,200.27532)
\closepath
}
}
{
\newrgbcolor{curcolor}{0 0 0}
\pscustom[linewidth=2,linecolor=curcolor]
{
\newpath
\moveto(674.82297,200.27532)
\lineto(680,191.75216)
\lineto(685.17703,200.27532)
\closepath
}
}
{
\newrgbcolor{curcolor}{1 1 1}
\pscustom[linestyle=none,fillstyle=solid,fillcolor=curcolor]
{
\newpath
\moveto(682.93526758,189.36177581)
\curveto(682.93526758,191.03835457)(681.57613196,192.3974902)(679.8995532,192.3974902)
\curveto(678.22297443,192.3974902)(676.86383881,191.03835457)(676.86383881,189.36177581)
\curveto(676.86383881,187.68519705)(678.22297443,186.32606143)(679.8995532,186.32606143)
\curveto(681.57613196,186.32606143)(682.93526758,187.68519705)(682.93526758,189.36177581)
\closepath
}
}
{
\newrgbcolor{curcolor}{0 0 0}
\pscustom[linewidth=2,linecolor=curcolor]
{
\newpath
\moveto(682.93526758,189.36177581)
\curveto(682.93526758,191.03835457)(681.57613196,192.3974902)(679.8995532,192.3974902)
\curveto(678.22297443,192.3974902)(676.86383881,191.03835457)(676.86383881,189.36177581)
\curveto(676.86383881,187.68519705)(678.22297443,186.32606143)(679.8995532,186.32606143)
\curveto(681.57613196,186.32606143)(682.93526758,187.68519705)(682.93526758,189.36177581)
\closepath
}
}
{
\newrgbcolor{curcolor}{0 0 0}
\pscustom[linewidth=2,linecolor=curcolor]
{
\newpath
\moveto(683.38662,200.34392)
\lineto(683.38662,207.66534)
}
}
{
\newrgbcolor{curcolor}{0 0 0}
\pscustom[linewidth=2,linecolor=curcolor]
{
\newpath
\moveto(676.59109,200.37007)
\lineto(676.59109,207.95935)
}
}
{
\newrgbcolor{curcolor}{1 1 1}
\pscustom[linestyle=none,fillstyle=solid,fillcolor=curcolor]
{
\newpath
\moveto(211.35031,124.75384)
\lineto(202.82715,129.93087)
\lineto(211.35031,135.1079)
\closepath
}
}
{
\newrgbcolor{curcolor}{0 0 0}
\pscustom[linewidth=2,linecolor=curcolor]
{
\newpath
\moveto(211.35031,124.75384)
\lineto(202.82715,129.93087)
\lineto(211.35031,135.1079)
\closepath
}
}
{
\newrgbcolor{curcolor}{1 1 1}
\pscustom[linestyle=none,fillstyle=solid,fillcolor=curcolor]
{
\newpath
\moveto(200.43676581,132.86613758)
\curveto(202.11334457,132.86613758)(203.4724802,131.50700196)(203.4724802,129.8304232)
\curveto(203.4724802,128.15384443)(202.11334457,126.79470881)(200.43676581,126.79470881)
\curveto(198.76018705,126.79470881)(197.40105143,128.15384443)(197.40105143,129.8304232)
\curveto(197.40105143,131.50700196)(198.76018705,132.86613758)(200.43676581,132.86613758)
\closepath
}
}
{
\newrgbcolor{curcolor}{0 0 0}
\pscustom[linewidth=2,linecolor=curcolor]
{
\newpath
\moveto(200.43676581,132.86613758)
\curveto(202.11334457,132.86613758)(203.4724802,131.50700196)(203.4724802,129.8304232)
\curveto(203.4724802,128.15384443)(202.11334457,126.79470881)(200.43676581,126.79470881)
\curveto(198.76018705,126.79470881)(197.40105143,128.15384443)(197.40105143,129.8304232)
\curveto(197.40105143,131.50700196)(198.76018705,132.86613758)(200.43676581,132.86613758)
\closepath
}
}
{
\newrgbcolor{curcolor}{0 0 0}
\pscustom[linewidth=2,linecolor=curcolor]
{
\newpath
\moveto(211.41891,133.31749)
\lineto(218.74033,133.31749)
}
}
{
\newrgbcolor{curcolor}{0 0 0}
\pscustom[linewidth=2,linecolor=curcolor]
{
\newpath
\moveto(211.44506,126.52196)
\lineto(219.03434,126.52196)
}
}
{
\newrgbcolor{curcolor}{1 1 1}
\pscustom[linestyle=none,fillstyle=solid,fillcolor=curcolor]
{
\newpath
\moveto(205.01129419,448.04968242)
\curveto(203.33471543,448.04968242)(201.9755798,449.40881804)(201.9755798,451.0853968)
\curveto(201.9755798,452.76197557)(203.33471543,454.12111119)(205.01129419,454.12111119)
\curveto(206.68787295,454.12111119)(208.04700857,452.76197557)(208.04700857,451.0853968)
\curveto(208.04700857,449.40881804)(206.68787295,448.04968242)(205.01129419,448.04968242)
\closepath
}
}
{
\newrgbcolor{curcolor}{0 0 0}
\pscustom[linewidth=2,linecolor=curcolor]
{
\newpath
\moveto(205.01129419,448.04968242)
\curveto(203.33471543,448.04968242)(201.9755798,449.40881804)(201.9755798,451.0853968)
\curveto(201.9755798,452.76197557)(203.33471543,454.12111119)(205.01129419,454.12111119)
\curveto(206.68787295,454.12111119)(208.04700857,452.76197557)(208.04700857,451.0853968)
\curveto(208.04700857,449.40881804)(206.68787295,448.04968242)(205.01129419,448.04968242)
\closepath
}
}
{
\newrgbcolor{curcolor}{0 0 0}
\pscustom[linewidth=2,linecolor=curcolor]
{
\newpath
\moveto(193.48864,456.21014)
\lineto(202.01182,451.03311)
\lineto(193.48864,445.85608)
}
}
{
\newrgbcolor{curcolor}{1 1 1}
\pscustom[linestyle=none,fillstyle=solid,fillcolor=curcolor]
{
\newpath
\moveto(140.18012,264.18881)
\lineto(145.35715,272.71197)
\lineto(150.53418,264.18881)
\closepath
}
}
{
\newrgbcolor{curcolor}{0 0 0}
\pscustom[linewidth=2,linecolor=curcolor]
{
\newpath
\moveto(140.18012,264.18881)
\lineto(145.35715,272.71197)
\lineto(150.53418,264.18881)
\closepath
}
}
{
\newrgbcolor{curcolor}{1 1 1}
\pscustom[linestyle=none,fillstyle=solid,fillcolor=curcolor]
{
\newpath
\moveto(148.29241758,275.10235419)
\curveto(148.29241758,273.42577543)(146.93328196,272.0666398)(145.2567032,272.0666398)
\curveto(143.58012443,272.0666398)(142.22098881,273.42577543)(142.22098881,275.10235419)
\curveto(142.22098881,276.77893295)(143.58012443,278.13806857)(145.2567032,278.13806857)
\curveto(146.93328196,278.13806857)(148.29241758,276.77893295)(148.29241758,275.10235419)
\closepath
}
}
{
\newrgbcolor{curcolor}{0 0 0}
\pscustom[linewidth=2,linecolor=curcolor]
{
\newpath
\moveto(148.29241758,275.10235419)
\curveto(148.29241758,273.42577543)(146.93328196,272.0666398)(145.2567032,272.0666398)
\curveto(143.58012443,272.0666398)(142.22098881,273.42577543)(142.22098881,275.10235419)
\curveto(142.22098881,276.77893295)(143.58012443,278.13806857)(145.2567032,278.13806857)
\curveto(146.93328196,278.13806857)(148.29241758,276.77893295)(148.29241758,275.10235419)
\closepath
}
}
{
\newrgbcolor{curcolor}{0 0 0}
\pscustom[linewidth=2,linecolor=curcolor]
{
\newpath
\moveto(148.74377,264.12021)
\lineto(148.74377,256.79879)
}
}
{
\newrgbcolor{curcolor}{0 0 0}
\pscustom[linewidth=2,linecolor=curcolor]
{
\newpath
\moveto(141.94824,264.09406)
\lineto(141.94824,256.50478)
}
}
\end{pspicture}

\caption{Modelo Entidad-Relación de las entidades que apoyan al proceso de
evaluación por parte del docente.}
\label{modelo3}
\end{figure}

En la figura (\ref{modelo3}), se presenta la parte del modelo entidad-relación
del sistema, que comprende a las entidades que facilitan la evaluación de los
estudiantes por parte del docente.

\subsection{\emph{evaluations}: Los sistemas de evaluación}
En un comienzo se notó que si bien, los docentes siempre presentan las
calificaciones con un formato único (Primer parcial, Segundo parcial, Promedio
de los parciales, Examen final y Segunda instancia), estas solo son la parte
final de un proceso aun mas complejo, por lo cual se creo un paquete que pueda
manejar criterios de evaluación mas elaborados, según el docente que imparta la
materia.

Las principales funciones de este paquete son:

\begin{itemize}
\item Creación y edición de criterios de evaluación.
\item Publicación y aplicación de un criterio de evaluación sobre un grupo
determinado por un docente.
\item Aplicación de calificaciones según un criterio de evaluación especifico.
\end{itemize}

\subsection{\emph{califications}: Las calificaciones}
A partir del paquete que administra y proporciona diferentes tipos de evaluación
para un grupo, se han creado las funcionalidades necesarias para establecer las
calificaciones de un grupo.

Las principales funciones de este paquete son:

\begin{itemize}
\item Edición de las calificaciones de los alumnos de un determinado grupo,
según un criterio de evaluación establecido.
\item Importación automática a partir de una archivo en formato CSV, de las
calificaciones de una grupo.
\item Exportación a formato CSV de las calificaciones de una grupo, según un
criterio de evaluación establecido.
\end{itemize}

\section{Recursos}
Ya creados los espacios virtuales, se han construido los paquetes necesarios
para la publicación e intercambio de recursos en el sistema.

\subsection{\emph{resources}: La abstracción de todos los recursos}
Al igual que en los espacios virtuales, los recursos también comparten un gran
conjunto de funcionalidad común a todos ellos (ya sean de presentación, como de
funcionalidad), es así como se han abstraído estas funcionalidades en este
paquete.

Las principales funciones de este paquete son:

\begin{itemize}
\item Concentrar las funciones de valoración sobre los recursos.
\item Generación de una capa de abstracción para la creación de recursos con
condiciones de uso diferente.
\item Administración de los recursos (creación, modificación, presentación,
valoración, eliminación).
\item Control de permisos sobre los recursos.
\end{itemize}

\subsection{\emph{notes}: Los recursos mas básicos}
El tipo de recurso mas básico es la nota, que representa esencialmente texto.
Este recurso además se diseño para ser utilizado por otros tipo de recursos mas
enriquecidos y especializados.

\subsection{\emph{links}: El administrador de marcadores}
Este recurso fue construido para compartir recursos, a partir de enlaces sobre
la red de Internet, se pensó para trabajos posterior, hacer de este tipo de
recurso una funcionalidad de reconocimiento de recurso y una reenderización mas
apropiada.

\subsection{\emph{files}: El administrador de archivos}
Este recurso fue construido para compartir archivos en un espacio virtual,
también se pensó para trabajo posterior, una especialización sobre la
reenderización y reconocimiento de los formatos.

\subsection{\emph{photos}, \emph{videos}: Los recursos especiales}
Estos recursos están basados en el recurso tipo archivo, pero poseen
características que les permiten ser reenderizados apropiadamente.

\subsection{\emph{events}: El recurso espacio-temporales}
Este recurso representa un evento sobre un espacio virtual, a partir de este
recurso, se considero para trabajo posterior, el que sea tanto un recurso, como
un espacio virtual independiente, además de facilitar el manejo de
notificaciones.

\subsection{\emph{feedback}: El manejador de sugerencias}
Este fue el ultimo paquete que se construyo, y fue creado exclusivamente para
que los usuarios del sistema puedan hacer sus sugerencias sobre nueva
funcionalidad que podrían implementarse sobre el sistema.

\section{Los usuarios y su red de contactos}
Una vez establecidos los conceptos de espacio virtual, y recurso, se han
construido los paquetes necesarios para la representación y manejo de usuarios
del sistema.

\subsection{\emph{users}: El espacio personal}
Representa al usuario final del sistema (docentes, estudiantes, auxiliares,
etc.), estos además están provistos de las funcionalidades para espacios
virtuales.

Las principales funciones de este paquete son:

\begin{itemize}
\item Creación de un espacio virtual propio del usuario.
\item Manejo y visualización de sus valoraciones en el sistema.
\item Administración de los datos personales del usuario.
\item Manejo de los recursos creados por el usuario sobre los diferentes
espacios virtuales a los que tiene acceso.
\end{itemize}

\subsection{\emph{roles}: El controlador de las privilegios}
Un rol representa el conjunto de privilegios que posee un usuario sobre los
paquetes del sistema.
Inicialmente de considero categorizar a los usuarios de forma estática, según
los criterios del modelo de la Universidad, estos son:

\begin{itemize}
\item Visitante
\item Invitado
\item Estudiante
\item Auxiliar
\item Docente
\item Moderador
\item Desarrollador
\item Administrador
\end{itemize}

Pero para facilitar la adaptación a otros tipos de organización, se vio
conveniente crear roles dinámicamente, así como poder administrar el conjunto
de privilegios que estos posean.

Las principales funciones de este paquete son:

\begin{itemize}
\item Administración de roles (creación, visualización, edición, eliminación).
\item Asignación de privilegios dinámicamente.
\item Control de acceso a los espacios, recursos, y rutas según un rol
establecido.
\end{itemize}

\subsection{\emph{contacts}: Las redes sociales}
Para proveer las características de una red social, se construyo un paquete que
pudiese manejar las relaciones entre diferentes tipos de usuarios.

Las principales funciones de este paquete son:

\begin{itemize}
\item Agregación de un contacto.
\item Eliminación de un contacto.
\item Visualización de los recursos compartidos por los contactos.
\end{itemize}

\subsection{\emph{invitations}: Estrategia de propagación del sistema}
Se considero que la creación de cuentas en el sistema nunca sea abierta, con el
objetivo de hacer que los usuarios en el sistema, tengan al menos un contacto
con otro usuario, por tal motivo, se crearon las invitaciones.

Las principales funciones de este paquete son:

\begin{itemize}
\item Envío de una invitación de un usuario a un correo electrónico.
\item Gestión de la caducidad de una invitación.
\item Manejo del contacto con el usuario que atiende a la invitación.
\end{itemize}

\section{Fomento a la participación}
Como medidas para fomentar la participación de parte de los usuarios, se han
implementado algunas funcionalidad propias de la web 2.0.

\subsection{\emph{comments}: Los comentarios}
\subsection{\emph{ratings}: La calidad del recurso}
\subsection{\emph{tags}: Las nuevas interpretaciones}
\subsection{\emph{valorations}: Los sistemas de reputación}

\section{Sistemas de control}
\subsection{\emph{stats}: Los indicadores medibles}

