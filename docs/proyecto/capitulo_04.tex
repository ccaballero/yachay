\chapter{Desarrollo del proyecto}

En este capítulo, trataremos los asuntos concernientes a la construcción de las
funciones del sistema, sobre las que recaerán el control de los recursos, y la
extensibilidad que pueda darsele a todo el proyecto. Si bien en el anterior
capitulo el tema fundamental era el \emph{proceso de desarrollo}, este capitulo
esta centrado en el \emph{producto de software}.

\section{Base funcional del sistema}

Una de las caracteristicas deseables a tomar en cuenta para el desarrollo del
sistema, era obtener un perfecto equilibrio entre modularidad y rendimiento,
pero sin incrementar la complejidad del sistema de modo apreciable.

Para conseguir tal caracteristica, se opto por utilizar una arquitectura basada
en capas, muy similar a como son diseñados los sistemas operativos, pero sin
llegar a la complejidad que estos mismos poseen. En la capa mas básica de la
arquitectura del sistema, se encuentran tres paquetes que son fundamentales para
cualquier funcion que el sistema quiera desempeñar. En esta sección se tratan
estos tres paquetes, ademas de la solución que se plantea para proveer al
usuario final de una personalización mas atractiva.

\subsection{Gestion de paquetes \emph{packages}}


\subsection{Manejo de privilegios \emph{privileges}}
\subsection{Manejo de rutas y navegación \emph{routes}}
\subsection{Sistema de plantillas \emph{templates}}

%\section{Construcción de espacios virtuales}

%\subsection{Espacios virtuales}
%\subsubsection{El espacio genérico \emph{spaces}}
%\section{Los espacios formales \emph{gestions}, \emph{careers}, \emph{areas}}
%\subsubsection{El espacio informal \emph{communities}}
%\section{Los espacios jerárquicos \emph{subjects}, \emph{groups}, \emph{teams}}
%\subsubsection{Las utilidades sobre grupos \emph{groupsets}}

%\subsection{B-learning}
%\subsubsection{Los sistemas de evaluación \emph{evaluations}}
%\subsubsection{Las calificaciones \emph{califications}}

%\section{Intercambio de recursos}

%\subsection{Recursos}
%\subsubsection{Los componentes genéricos \emph{resources}}
%\subsubsection{El recurso mas básico \emph{notes}}
%\subsubsection{Los archivos en general \emph{files}}
%\subsubsection{Los archivos especiales \emph{photos}, \emph{videos}}
%\subsubsection{Los recursos espacio-temporales \emph{events}}
%\subsubsection{La reenderizacion personalizada \emph{links}}
%\subsubsection{Las sugerencias \emph{feedback}}

%\section{Instancias múltiples del sistema}
%\subsection{Conectitividad}
%\subsubsection{La inteligencia colectiva \emph{axon}}

%\section{Canales de comunicación}

%\subsection{Las Personas}
%\subsubsection{El espacio personal \emph{users}}
%\subsubsection{El control de las funciones \emph{roles}}

%\subsection{Red de contactos}
%\subsubsection{Las redes sociales \emph{contacts}}
%\subsubsection{Los círculos \emph{sets}}
%\subsubsection{La propagación \emph{invitations}}

%\section{Fomento a la participación}

%\subsection{Web 2.0}
%\subsubsection{Los comentarios \emph{comments}}
%\subsubsection{La calidad del recurso \emph{ratings}}
%\subsubsection{Las nuevas interpretaciones \emph{tags}}
%\subsubsection{Los sistemas de reputación \emph{valorations}}

%\subsection{Sistemas de control}
%\subsubsection{Los indicadores medibles \emph{stats}}
%\subsubsection{El panel de control \emph{panels}}

