\chapter{Desarrollo del proyecto}

En este capítulo, trataremos los asuntos concernientes a la construcción de las
funciones del sistema, sobre las que recaerán el control de los recursos, y la
expansibilidad que pueda darse a todo el proyecto. Si bien en el anterior
capitulo el tema fundamental era el \emph{proceso de desarrollo}, el objeto
central de este capitulo es el \emph{producto de software}.

\section{Base funcional del sistema}
Una de las características deseables a tomar en cuenta para el desarrollo del
sistema, era obtener un perfecto equilibrio entre modularidad y rendimiento,
pero sin incrementar la complejidad del sistema de modo apreciable.

Para conseguir tal característica, se opto por utilizar una arquitectura basada
en capas, muy similar a como son diseñados los sistemas operativos, pero sin
llegar a la complejidad que estos mismos poseen. En la capa mas básica de la
arquitectura del sistema, se encuentran tres paquetes que son fundamentales para
cualquier función que el sistema quiera desempeñar. En esta sección se tratan
estos tres paquetes, además de la solución que se plantea para proveer al
usuario final de una personalización mas atractiva.

\subsection{\emph{packages}: Manejador de paquetes}
Las principales funciones de este paquete son:
\begin{itemize}
\item Instalación de paquetes en el sistema.
\item Manejo de dependencias entre paquetes del sistema.
\item Establecimiento de rutas para un paquete determinado.
\end{itemize}

\subsection{\emph{privileges}: Manejador de privilegios}
Las principales funciones de este paquete son:
\begin{itemize}
\item Registro de privilegios reservados por cada paquete.
\item Control de acceso a recursos y acciones especificas.
\end{itemize}

\subsection{\emph{routes}: Manejador de rutas de navegación}
Las principales funciones de este paquete son:
\begin{itemize}
\item Registro de rutas reservadas por cada paquete.
\end{itemize}

\subsection{\emph{templates}: Manejador de plantillas}
Las principales funciones de este paquete son:
\begin{itemize}
\item Gestión de la presentación del sistema.
\item Administración de las utilidades adicionales que presentan las paginas
del sistema.
\end{itemize}

\section{Construcción de espacios virtuales}
Uno de los puntos fundamentales en la construcción del sistema, es el control y
manejo organizado de los espacios disponibles del sistema, estos espacios
constituyen los lugares, de intercambio, producción, y discusión de los
recursos que posea el sistema.

\subsection{\emph{spaces}: La abstracción de todos los espacios}
Las principales funciones de este paquete son:
\begin{itemize}
\item Concentrar las funciones sobre los recursos del sistema.
\item Generación de una capa de abstracción para la creación de espacios con
condiciones de uso diferente.
\end{itemize}

\subsection{\emph{gestions}, \emph{careers}, \emph{areas}: Los espacios
formales}
Las principales funciones de estos paquetes son:
\begin{itemize}
\item Conformación de los espacios organizaciones acordes a la estructura real
observada en el contexto del sistema.
\end{itemize}

\subsection{\emph{subjects}, \emph{groups}, \emph{teams}: Los espacios
jerárquicos}
Las principales funciones de estos paquetes son:
\begin{itemize}
\item Creación de la estructura jerárquica que es también un reflejo de la
estructura real observada en el contexto del sistema.
\end{itemize}

\subsection{\emph{communities}: El espacio informal}
Las principales funciones de estos paquetes son:
\begin{itemize}
\item Creación de un espacio común, transversal a las estructuras del contexto
del sistema.
\end{itemize}

\section{B-learning}
\subsection{\emph{evaluations}: Los sistemas de evaluación}
\subsection{\emph{califications}: Las calificaciones}

\section{Recursos}
\subsection{\emph{resources}: La abstracción de todos los recursos}
\subsection{\emph{notes}, \emph{links}: Los recursos mas básicos}
\subsection{\emph{files}: El administrador de archivos}
\subsection{\emph{photos}, \emph{videos}: Los recursos especiales}
\subsection{\emph{events}: El recurso espacio-temporales}
\subsection{\emph{feedback}: El manejador de sugerencias}

\section{Los usuarios y su red de contactos}
\subsection{\emph{users}: El espacio personal}
\subsection{\emph{roles}: El controlador de las privilegios}
\subsection{\emph{contacts}: Las redes sociales}
\subsection{\emph{invitations}: Estrategia de propagación del sistema}

\section{Fomento a la participación}
\subsection{\emph{comments}: Los comentarios}
\subsection{\emph{ratings}: La calidad del recurso}
\subsection{\emph{tags}: Las nuevas interpretaciones}
\subsection{\emph{valorations}: Los sistemas de reputación}

\section{Sistemas de control}
\subsection{\emph{stats}: Los indicadores medibles}

