\chapter{Desarrollo del proyecto}

En este capítulo, trataremos los asuntos concernientes a la construcción de las
funciones del sistema, sobre las que recaerán el control de los recursos, y la
expansibilidad que pueda darse a todo el proyecto. Si bien en el anterior
capitulo el tema fundamental era el \emph{proceso de desarrollo}, el objeto
central de este capitulo es el \emph{producto de software}.

\section{Base funcional del sistema}
Una de las características deseables a tomar en cuenta para el desarrollo del
sistema, era obtener un perfecto equilibrio entre modularidad y rendimiento,
pero sin incrementar la complejidad del sistema de modo apreciable.

Para conseguir tal característica, se opto por utilizar una arquitectura basada
en capas, muy similar al diseño de los sistemas operativos, pero sin llegar a la
complejidad que estos mismos poseen. En la capa mas básica de la arquitectura
del sistema, se encuentran tres paquetes que son fundamentales para cualquier
función que el sistema quiera desempeñar. En esta sección se tratan estos tres
paquetes, además de la solución que se plantea para proveer al usuario final de
una personalización mas atractiva.

\begin{figure}
\centering
%LaTeX with PSTricks extensions
%%Creator: inkscape 0.48.5
%%Please note this file requires PSTricks extensions
\psset{xunit=.5pt,yunit=.5pt,runit=.5pt}
\begin{pspicture}(650,500)
{
\newrgbcolor{curcolor}{0 0 0}
\pscustom[linewidth=1.84937644,linecolor=curcolor]
{
\newpath
\moveto(54.07824701,405.22349423)
\lineto(202.22888177,405.22349423)
\lineto(202.22888177,357.07287091)
\lineto(54.07824701,357.07287091)
\closepath
}
}
{
\newrgbcolor{curcolor}{0 0 0}
\pscustom[linestyle=none,fillstyle=solid,fillcolor=curcolor]
{
\newpath
\moveto(80.0492695,381.65818158)
\lineto(83.1992695,381.65818158)
\curveto(84.15926101,381.65817036)(84.98926018,381.99817002)(85.6892695,382.67818158)
\curveto(86.38925878,383.35816866)(86.73925843,384.40816761)(86.7392695,385.82818158)
\curveto(86.73925843,387.02816499)(86.44925872,387.96816405)(85.8692695,388.64818158)
\curveto(85.28925988,389.34816267)(84.33926083,389.69816232)(83.0192695,389.69818158)
\lineto(80.0492695,389.69818158)
\lineto(80.0492695,381.65818158)
\moveto(77.4092695,391.85818158)
\lineto(82.8692695,391.85818158)
\curveto(83.169262,391.85816016)(83.53926163,391.84816017)(83.9792695,391.82818158)
\curveto(84.43926073,391.80816021)(84.90926026,391.73816028)(85.3892695,391.61818158)
\curveto(85.88925928,391.5181605)(86.37925879,391.33816068)(86.8592695,391.07818158)
\curveto(87.35925781,390.83816118)(87.79925737,390.48816153)(88.1792695,390.02818158)
\curveto(88.57925659,389.56816245)(88.89925627,388.98816303)(89.1392695,388.28818158)
\curveto(89.37925579,387.58816443)(89.49925567,386.72816529)(89.4992695,385.70818158)
\curveto(89.49925567,384.70816731)(89.34925582,383.8181682)(89.0492695,383.03818158)
\curveto(88.74925642,382.27816974)(88.31925685,381.62817039)(87.7592695,381.08818158)
\curveto(87.21925795,380.56817145)(86.5692586,380.16817185)(85.8092695,379.88818158)
\curveto(85.04926012,379.62817239)(84.21926095,379.49817252)(83.3192695,379.49818158)
\lineto(80.0492695,379.49818158)
\lineto(80.0492695,370.43818158)
\lineto(77.4092695,370.43818158)
\lineto(77.4092695,391.85818158)
}
}
{
\newrgbcolor{curcolor}{0 0 0}
\pscustom[linestyle=none,fillstyle=solid,fillcolor=curcolor]
{
\newpath
\moveto(93.64934762,378.47818158)
\lineto(99.40934762,378.47818158)
\lineto(96.67934762,388.82818158)
\lineto(96.61934762,388.82818158)
\lineto(93.64934762,378.47818158)
\moveto(94.93934762,391.85818158)
\lineto(98.47934762,391.85818158)
\lineto(104.23934762,370.43818158)
\lineto(101.47934762,370.43818158)
\lineto(99.94934762,376.31818158)
\lineto(93.10934762,376.31818158)
\lineto(91.51934762,370.43818158)
\lineto(88.75934762,370.43818158)
\lineto(94.93934762,391.85818158)
}
}
{
\newrgbcolor{curcolor}{0 0 0}
\pscustom[linestyle=none,fillstyle=solid,fillcolor=curcolor]
{
\newpath
\moveto(118.8786445,377.78818158)
\curveto(118.81862971,376.76817525)(118.65862987,375.78817623)(118.3986445,374.84818158)
\curveto(118.15863037,373.92817809)(117.78863074,373.10817891)(117.2886445,372.38818158)
\curveto(116.78863174,371.66818035)(116.1286324,371.08818093)(115.3086445,370.64818158)
\curveto(114.50863402,370.22818179)(113.528635,370.018182)(112.3686445,370.01818158)
\curveto(110.84863768,370.018182)(109.6286389,370.33818168)(108.7086445,370.97818158)
\curveto(107.80864072,371.63818038)(107.11864141,372.49817952)(106.6386445,373.55818158)
\curveto(106.15864237,374.6181774)(105.83864269,375.80817621)(105.6786445,377.12818158)
\curveto(105.53864299,378.44817357)(105.46864306,379.78817223)(105.4686445,381.14818158)
\curveto(105.46864306,382.48816953)(105.54864298,383.8181682)(105.7086445,385.13818158)
\curveto(105.86864266,386.47816554)(106.19864233,387.67816434)(106.6986445,388.73818158)
\curveto(107.19864133,389.79816222)(107.89864063,390.64816137)(108.7986445,391.28818158)
\curveto(109.69863883,391.94816007)(110.88863764,392.27815974)(112.3686445,392.27818158)
\curveto(114.54863398,392.27815974)(116.1286324,391.69816032)(117.1086445,390.53818158)
\curveto(118.10863042,389.37816264)(118.63862989,387.73816428)(118.6986445,385.61818158)
\lineto(115.9386445,385.61818158)
\curveto(115.91863261,386.2181658)(115.84863268,386.78816523)(115.7286445,387.32818158)
\curveto(115.60863292,387.88816413)(115.40863312,388.36816365)(115.1286445,388.76818158)
\curveto(114.86863366,389.18816283)(114.50863402,389.5181625)(114.0486445,389.75818158)
\curveto(113.60863492,389.99816202)(113.04863548,390.1181619)(112.3686445,390.11818158)
\curveto(111.44863708,390.1181619)(110.71863781,389.87816214)(110.1786445,389.39818158)
\curveto(109.63863889,388.93816308)(109.21863931,388.29816372)(108.9186445,387.47818158)
\curveto(108.63863989,386.67816534)(108.44864008,385.72816629)(108.3486445,384.62818158)
\curveto(108.26864026,383.54816847)(108.2286403,382.38816963)(108.2286445,381.14818158)
\curveto(108.2286403,379.90817211)(108.26864026,378.73817328)(108.3486445,377.63818158)
\curveto(108.44864008,376.55817546)(108.63863989,375.60817641)(108.9186445,374.78818158)
\curveto(109.21863931,373.98817803)(109.63863889,373.34817867)(110.1786445,372.86818158)
\curveto(110.71863781,372.40817961)(111.44863708,372.17817984)(112.3686445,372.17818158)
\curveto(113.16863536,372.17817984)(113.80863472,372.34817967)(114.2886445,372.68818158)
\curveto(114.76863376,373.02817899)(115.13863339,373.46817855)(115.3986445,374.00818158)
\curveto(115.67863285,374.54817747)(115.85863267,375.14817687)(115.9386445,375.80818158)
\curveto(116.03863249,376.46817555)(116.09863243,377.12817489)(116.1186445,377.78818158)
\lineto(118.8786445,377.78818158)
}
}
{
\newrgbcolor{curcolor}{0 0 0}
\pscustom[linestyle=none,fillstyle=solid,fillcolor=curcolor]
{
\newpath
\moveto(121.354582,391.85818158)
\lineto(123.994582,391.85818158)
\lineto(123.994582,381.47818158)
\lineto(124.054582,381.47818158)
\lineto(131.464582,391.85818158)
\lineto(134.404582,391.85818158)
\lineto(127.924582,382.91818158)
\lineto(134.944582,370.43818158)
\lineto(132.004582,370.43818158)
\lineto(126.214582,380.75818158)
\lineto(123.994582,377.72818158)
\lineto(123.994582,370.43818158)
\lineto(121.354582,370.43818158)
\lineto(121.354582,391.85818158)
}
}
{
\newrgbcolor{curcolor}{0 0 0}
\pscustom[linestyle=none,fillstyle=solid,fillcolor=curcolor]
{
\newpath
\moveto(139.264582,378.47818158)
\lineto(145.024582,378.47818158)
\lineto(142.294582,388.82818158)
\lineto(142.234582,388.82818158)
\lineto(139.264582,378.47818158)
\moveto(140.554582,391.85818158)
\lineto(144.094582,391.85818158)
\lineto(149.854582,370.43818158)
\lineto(147.094582,370.43818158)
\lineto(145.564582,376.31818158)
\lineto(138.724582,376.31818158)
\lineto(137.134582,370.43818158)
\lineto(134.374582,370.43818158)
\lineto(140.554582,391.85818158)
}
}
{
\newrgbcolor{curcolor}{0 0 0}
\pscustom[linestyle=none,fillstyle=solid,fillcolor=curcolor]
{
\newpath
\moveto(161.524582,385.91818158)
\curveto(161.48457013,386.47816554)(161.39457022,387.00816501)(161.254582,387.50818158)
\curveto(161.13457048,388.02816399)(160.93457068,388.47816354)(160.654582,388.85818158)
\curveto(160.39457122,389.23816278)(160.04457157,389.53816248)(159.604582,389.75818158)
\curveto(159.16457245,389.99816202)(158.614573,390.1181619)(157.954582,390.11818158)
\curveto(157.03457458,390.1181619)(156.30457531,389.87816214)(155.764582,389.39818158)
\curveto(155.22457639,388.93816308)(154.80457681,388.29816372)(154.504582,387.47818158)
\curveto(154.22457739,386.67816534)(154.03457758,385.72816629)(153.934582,384.62818158)
\curveto(153.85457776,383.54816847)(153.8145778,382.38816963)(153.814582,381.14818158)
\curveto(153.8145778,379.90817211)(153.85457776,378.73817328)(153.934582,377.63818158)
\curveto(154.03457758,376.55817546)(154.22457739,375.60817641)(154.504582,374.78818158)
\curveto(154.80457681,373.98817803)(155.22457639,373.34817867)(155.764582,372.86818158)
\curveto(156.30457531,372.40817961)(157.03457458,372.17817984)(157.954582,372.17818158)
\curveto(158.87457274,372.17817984)(159.59457202,372.4181796)(160.114582,372.89818158)
\curveto(160.63457098,373.39817862)(161.02457059,374.00817801)(161.284582,374.72818158)
\curveto(161.54457007,375.44817657)(161.70456991,376.22817579)(161.764582,377.06818158)
\curveto(161.84456977,377.90817411)(161.88456973,378.68817333)(161.884582,379.40818158)
\lineto(157.654582,379.40818158)
\lineto(157.654582,381.56818158)
\lineto(164.284582,381.56818158)
\lineto(164.284582,370.43818158)
\lineto(162.304582,370.43818158)
\lineto(162.304582,373.34818158)
\lineto(162.244582,373.34818158)
\curveto(161.96456965,372.42817959)(161.42457019,371.63818038)(160.624582,370.97818158)
\curveto(159.82457179,370.33818168)(158.82457279,370.018182)(157.624582,370.01818158)
\curveto(156.22457539,370.018182)(155.09457652,370.32818169)(154.234582,370.94818158)
\curveto(153.37457824,371.56818045)(152.70457891,372.38817963)(152.224582,373.40818158)
\curveto(151.76457985,374.44817757)(151.45458016,375.63817638)(151.294582,376.97818158)
\curveto(151.13458048,378.3181737)(151.05458056,379.70817231)(151.054582,381.14818158)
\curveto(151.05458056,382.48816953)(151.13458048,383.8181682)(151.294582,385.13818158)
\curveto(151.45458016,386.47816554)(151.78457983,387.67816434)(152.284582,388.73818158)
\curveto(152.78457883,389.79816222)(153.48457813,390.64816137)(154.384582,391.28818158)
\curveto(155.28457633,391.94816007)(156.47457514,392.27815974)(157.954582,392.27818158)
\curveto(158.97457264,392.27815974)(159.83457178,392.14815987)(160.534582,391.88818158)
\curveto(161.25457036,391.62816039)(161.84456977,391.28816073)(162.304582,390.86818158)
\curveto(162.78456883,390.46816155)(163.15456846,390.018162)(163.414582,389.51818158)
\curveto(163.67456794,389.018163)(163.86456775,388.52816349)(163.984582,388.04818158)
\curveto(164.12456749,387.58816443)(164.20456741,387.15816486)(164.224582,386.75818158)
\curveto(164.26456735,386.37816564)(164.28456733,386.09816592)(164.284582,385.91818158)
\lineto(161.524582,385.91818158)
}
}
{
\newrgbcolor{curcolor}{0 0 0}
\pscustom[linestyle=none,fillstyle=solid,fillcolor=curcolor]
{
\newpath
\moveto(167.46786325,391.85818158)
\lineto(178.53786325,391.85818158)
\lineto(178.53786325,389.51818158)
\lineto(170.10786325,389.51818158)
\lineto(170.10786325,382.79818158)
\lineto(178.05786325,382.79818158)
\lineto(178.05786325,380.45818158)
\lineto(170.10786325,380.45818158)
\lineto(170.10786325,372.77818158)
\lineto(178.89786325,372.77818158)
\lineto(178.89786325,370.43818158)
\lineto(167.46786325,370.43818158)
\lineto(167.46786325,391.85818158)
}
}
{
\newrgbcolor{curcolor}{0 0 0}
\pscustom[linestyle=none,fillstyle=solid,fillcolor=curcolor]
{
\newpath
\moveto(367.32515495,381.65818529)
\lineto(370.47515495,381.65818529)
\curveto(371.43514646,381.65817407)(372.26514563,381.99817373)(372.96515495,382.67818529)
\curveto(373.66514423,383.35817237)(374.01514388,384.40817132)(374.01515495,385.82818529)
\curveto(374.01514388,387.0281687)(373.72514417,387.96816776)(373.14515495,388.64818529)
\curveto(372.56514533,389.34816638)(371.61514628,389.69816603)(370.29515495,389.69818529)
\lineto(367.32515495,389.69818529)
\lineto(367.32515495,381.65818529)
\moveto(364.68515495,391.85818529)
\lineto(370.14515495,391.85818529)
\curveto(370.44514745,391.85816387)(370.81514708,391.84816388)(371.25515495,391.82818529)
\curveto(371.71514618,391.80816392)(372.18514571,391.73816399)(372.66515495,391.61818529)
\curveto(373.16514473,391.51816421)(373.65514424,391.33816439)(374.13515495,391.07818529)
\curveto(374.63514326,390.83816489)(375.07514282,390.48816524)(375.45515495,390.02818529)
\curveto(375.85514204,389.56816616)(376.17514172,388.98816674)(376.41515495,388.28818529)
\curveto(376.65514124,387.58816814)(376.77514112,386.728169)(376.77515495,385.70818529)
\curveto(376.77514112,384.70817102)(376.62514127,383.81817191)(376.32515495,383.03818529)
\curveto(376.02514187,382.27817345)(375.5951423,381.6281741)(375.03515495,381.08818529)
\curveto(374.4951434,380.56817516)(373.84514405,380.16817556)(373.08515495,379.88818529)
\curveto(372.32514557,379.6281761)(371.4951464,379.49817623)(370.59515495,379.49818529)
\lineto(367.32515495,379.49818529)
\lineto(367.32515495,370.43818529)
\lineto(364.68515495,370.43818529)
\lineto(364.68515495,391.85818529)
}
}
{
\newrgbcolor{curcolor}{0 0 0}
\pscustom[linestyle=none,fillstyle=solid,fillcolor=curcolor]
{
\newpath
\moveto(381.73921745,382.13818529)
\lineto(384.34921745,382.13818529)
\curveto(384.72921008,382.13817359)(385.16920964,382.15817357)(385.66921745,382.19818529)
\curveto(386.18920862,382.23817349)(386.67920813,382.38817334)(387.13921745,382.64818529)
\curveto(387.59920721,382.90817282)(387.97920683,383.31817241)(388.27921745,383.87818529)
\curveto(388.59920621,384.43817129)(388.75920605,385.23817049)(388.75921745,386.27818529)
\curveto(388.75920605,387.35816837)(388.42920638,388.19816753)(387.76921745,388.79818529)
\curveto(387.1092077,389.39816633)(386.14920866,389.69816603)(384.88921745,389.69818529)
\lineto(381.73921745,389.69818529)
\lineto(381.73921745,382.13818529)
\moveto(379.09921745,391.85818529)
\lineto(386.02921745,391.85818529)
\curveto(387.72920708,391.85816387)(389.06920574,391.38816434)(390.04921745,390.44818529)
\curveto(391.02920378,389.50816622)(391.51920329,388.18816754)(391.51921745,386.48818529)
\curveto(391.51920329,385.90816982)(391.45920335,385.3281704)(391.33921745,384.74818529)
\curveto(391.23920357,384.16817156)(391.05920375,383.6281721)(390.79921745,383.12818529)
\curveto(390.53920427,382.64817308)(390.19920461,382.21817351)(389.77921745,381.83818529)
\curveto(389.35920545,381.47817425)(388.83920597,381.2281745)(388.21921745,381.08818529)
\lineto(388.21921745,381.02818529)
\curveto(389.15920565,380.9281748)(389.87920493,380.53817519)(390.37921745,379.85818529)
\curveto(390.89920391,379.17817655)(391.18920362,378.37817735)(391.24921745,377.45818529)
\lineto(391.42921745,373.79818529)
\curveto(391.44920336,373.19818253)(391.48920332,372.70818302)(391.54921745,372.32818529)
\curveto(391.6092032,371.94818378)(391.68920312,371.6281841)(391.78921745,371.36818529)
\curveto(391.88920292,371.1281846)(391.99920281,370.93818479)(392.11921745,370.79818529)
\curveto(392.25920255,370.65818507)(392.4092024,370.53818519)(392.56921745,370.43818529)
\lineto(389.38921745,370.43818529)
\curveto(389.26920554,370.55818517)(389.16920564,370.73818499)(389.08921745,370.97818529)
\curveto(389.0092058,371.21818451)(388.93920587,371.47818425)(388.87921745,371.75818529)
\curveto(388.81920599,372.05818367)(388.76920604,372.35818337)(388.72921745,372.65818529)
\curveto(388.7092061,372.97818275)(388.68920612,373.26818246)(388.66921745,373.52818529)
\lineto(388.48921745,376.85818529)
\curveto(388.42920638,377.59817813)(388.28920652,378.16817756)(388.06921745,378.56818529)
\curveto(387.86920694,378.98817674)(387.61920719,379.29817643)(387.31921745,379.49818529)
\curveto(387.01920779,379.71817601)(386.68920812,379.84817588)(386.32921745,379.88818529)
\curveto(385.98920882,379.94817578)(385.64920916,379.97817575)(385.30921745,379.97818529)
\lineto(381.73921745,379.97818529)
\lineto(381.73921745,370.43818529)
\lineto(379.09921745,370.43818529)
\lineto(379.09921745,391.85818529)
}
}
{
\newrgbcolor{curcolor}{0 0 0}
\pscustom[linestyle=none,fillstyle=solid,fillcolor=curcolor]
{
\newpath
\moveto(394.68515495,391.85818529)
\lineto(397.32515495,391.85818529)
\lineto(397.32515495,370.43818529)
\lineto(394.68515495,370.43818529)
\lineto(394.68515495,391.85818529)
}
}
{
\newrgbcolor{curcolor}{0 0 0}
\pscustom[linestyle=none,fillstyle=solid,fillcolor=curcolor]
{
\newpath
\moveto(398.97890495,391.85818529)
\lineto(401.73890495,391.85818529)
\lineto(405.93890495,373.46818529)
\lineto(405.99890495,373.46818529)
\lineto(410.19890495,391.85818529)
\lineto(412.95890495,391.85818529)
\lineto(407.55890495,370.43818529)
\lineto(404.19890495,370.43818529)
\lineto(398.97890495,391.85818529)
}
}
{
\newrgbcolor{curcolor}{0 0 0}
\pscustom[linestyle=none,fillstyle=solid,fillcolor=curcolor]
{
\newpath
\moveto(414.6656237,391.85818529)
\lineto(417.3056237,391.85818529)
\lineto(417.3056237,370.43818529)
\lineto(414.6656237,370.43818529)
\lineto(414.6656237,391.85818529)
}
}
{
\newrgbcolor{curcolor}{0 0 0}
\pscustom[linestyle=none,fillstyle=solid,fillcolor=curcolor]
{
\newpath
\moveto(420.7593737,391.85818529)
\lineto(423.3993737,391.85818529)
\lineto(423.3993737,372.77818529)
\lineto(432.0993737,372.77818529)
\lineto(432.0993737,370.43818529)
\lineto(420.7593737,370.43818529)
\lineto(420.7593737,391.85818529)
}
}
{
\newrgbcolor{curcolor}{0 0 0}
\pscustom[linestyle=none,fillstyle=solid,fillcolor=curcolor]
{
\newpath
\moveto(434.06015495,391.85818529)
\lineto(445.13015495,391.85818529)
\lineto(445.13015495,389.51818529)
\lineto(436.70015495,389.51818529)
\lineto(436.70015495,382.79818529)
\lineto(444.65015495,382.79818529)
\lineto(444.65015495,380.45818529)
\lineto(436.70015495,380.45818529)
\lineto(436.70015495,372.77818529)
\lineto(445.49015495,372.77818529)
\lineto(445.49015495,370.43818529)
\lineto(434.06015495,370.43818529)
\lineto(434.06015495,391.85818529)
}
}
{
\newrgbcolor{curcolor}{0 0 0}
\pscustom[linestyle=none,fillstyle=solid,fillcolor=curcolor]
{
\newpath
\moveto(458.1168737,385.91818529)
\curveto(458.07686183,386.47816925)(457.98686192,387.00816872)(457.8468737,387.50818529)
\curveto(457.72686218,388.0281677)(457.52686238,388.47816725)(457.2468737,388.85818529)
\curveto(456.98686292,389.23816649)(456.63686327,389.53816619)(456.1968737,389.75818529)
\curveto(455.75686415,389.99816573)(455.2068647,390.11816561)(454.5468737,390.11818529)
\curveto(453.62686628,390.11816561)(452.89686701,389.87816585)(452.3568737,389.39818529)
\curveto(451.81686809,388.93816679)(451.39686851,388.29816743)(451.0968737,387.47818529)
\curveto(450.81686909,386.67816905)(450.62686928,385.72817)(450.5268737,384.62818529)
\curveto(450.44686946,383.54817218)(450.4068695,382.38817334)(450.4068737,381.14818529)
\curveto(450.4068695,379.90817582)(450.44686946,378.73817699)(450.5268737,377.63818529)
\curveto(450.62686928,376.55817917)(450.81686909,375.60818012)(451.0968737,374.78818529)
\curveto(451.39686851,373.98818174)(451.81686809,373.34818238)(452.3568737,372.86818529)
\curveto(452.89686701,372.40818332)(453.62686628,372.17818355)(454.5468737,372.17818529)
\curveto(455.46686444,372.17818355)(456.18686372,372.41818331)(456.7068737,372.89818529)
\curveto(457.22686268,373.39818233)(457.61686229,374.00818172)(457.8768737,374.72818529)
\curveto(458.13686177,375.44818028)(458.29686161,376.2281795)(458.3568737,377.06818529)
\curveto(458.43686147,377.90817782)(458.47686143,378.68817704)(458.4768737,379.40818529)
\lineto(454.2468737,379.40818529)
\lineto(454.2468737,381.56818529)
\lineto(460.8768737,381.56818529)
\lineto(460.8768737,370.43818529)
\lineto(458.8968737,370.43818529)
\lineto(458.8968737,373.34818529)
\lineto(458.8368737,373.34818529)
\curveto(458.55686135,372.4281833)(458.01686189,371.63818409)(457.2168737,370.97818529)
\curveto(456.41686349,370.33818539)(455.41686449,370.01818571)(454.2168737,370.01818529)
\curveto(452.81686709,370.01818571)(451.68686822,370.3281854)(450.8268737,370.94818529)
\curveto(449.96686994,371.56818416)(449.29687061,372.38818334)(448.8168737,373.40818529)
\curveto(448.35687155,374.44818128)(448.04687186,375.63818009)(447.8868737,376.97818529)
\curveto(447.72687218,378.31817741)(447.64687226,379.70817602)(447.6468737,381.14818529)
\curveto(447.64687226,382.48817324)(447.72687218,383.81817191)(447.8868737,385.13818529)
\curveto(448.04687186,386.47816925)(448.37687153,387.67816805)(448.8768737,388.73818529)
\curveto(449.37687053,389.79816593)(450.07686983,390.64816508)(450.9768737,391.28818529)
\curveto(451.87686803,391.94816378)(453.06686684,392.27816345)(454.5468737,392.27818529)
\curveto(455.56686434,392.27816345)(456.42686348,392.14816358)(457.1268737,391.88818529)
\curveto(457.84686206,391.6281641)(458.43686147,391.28816444)(458.8968737,390.86818529)
\curveto(459.37686053,390.46816526)(459.74686016,390.01816571)(460.0068737,389.51818529)
\curveto(460.26685964,389.01816671)(460.45685945,388.5281672)(460.5768737,388.04818529)
\curveto(460.71685919,387.58816814)(460.79685911,387.15816857)(460.8168737,386.75818529)
\curveto(460.85685905,386.37816935)(460.87685903,386.09816963)(460.8768737,385.91818529)
\lineto(458.1168737,385.91818529)
}
}
{
\newrgbcolor{curcolor}{0 0 0}
\pscustom[linestyle=none,fillstyle=solid,fillcolor=curcolor]
{
\newpath
\moveto(464.06015495,391.85818529)
\lineto(475.13015495,391.85818529)
\lineto(475.13015495,389.51818529)
\lineto(466.70015495,389.51818529)
\lineto(466.70015495,382.79818529)
\lineto(474.65015495,382.79818529)
\lineto(474.65015495,380.45818529)
\lineto(466.70015495,380.45818529)
\lineto(466.70015495,372.77818529)
\lineto(475.49015495,372.77818529)
\lineto(475.49015495,370.43818529)
\lineto(464.06015495,370.43818529)
\lineto(464.06015495,391.85818529)
}
}
{
\newrgbcolor{curcolor}{0 0 0}
\pscustom[linewidth=1.84937644,linecolor=curcolor]
{
\newpath
\moveto(346.01232413,405.22349413)
\lineto(494.1629589,405.22349413)
\lineto(494.1629589,357.07287081)
\lineto(346.01232413,357.07287081)
\closepath
}
}
{
\newrgbcolor{curcolor}{0 0 0}
\pscustom[linestyle=none,fillstyle=solid,fillcolor=curcolor]
{
\newpath
\moveto(372.00287084,248.23967999)
\lineto(374.61287084,248.23967999)
\curveto(374.99286347,248.23966829)(375.43286303,248.25966827)(375.93287084,248.29967999)
\curveto(376.45286201,248.33966819)(376.94286152,248.48966804)(377.40287084,248.74967999)
\curveto(377.8628606,249.00966752)(378.24286022,249.41966711)(378.54287084,249.97967999)
\curveto(378.8628596,250.53966599)(379.02285944,251.33966519)(379.02287084,252.37967999)
\curveto(379.02285944,253.45966307)(378.69285977,254.29966223)(378.03287084,254.89967999)
\curveto(377.37286109,255.49966103)(376.41286205,255.79966073)(375.15287084,255.79967999)
\lineto(372.00287084,255.79967999)
\lineto(372.00287084,248.23967999)
\moveto(369.36287084,257.95967999)
\lineto(376.29287084,257.95967999)
\curveto(377.99286047,257.95965857)(379.33285913,257.48965904)(380.31287084,256.54967999)
\curveto(381.29285717,255.60966092)(381.78285668,254.28966224)(381.78287084,252.58967999)
\curveto(381.78285668,252.00966452)(381.72285674,251.4296651)(381.60287084,250.84967999)
\curveto(381.50285696,250.26966626)(381.32285714,249.7296668)(381.06287084,249.22967999)
\curveto(380.80285766,248.74966778)(380.462858,248.31966821)(380.04287084,247.93967999)
\curveto(379.62285884,247.57966895)(379.10285936,247.3296692)(378.48287084,247.18967999)
\lineto(378.48287084,247.12967999)
\curveto(379.42285904,247.0296695)(380.14285832,246.63966989)(380.64287084,245.95967999)
\curveto(381.1628573,245.27967125)(381.45285701,244.47967205)(381.51287084,243.55967999)
\lineto(381.69287084,239.89967999)
\curveto(381.71285675,239.29967723)(381.75285671,238.80967772)(381.81287084,238.42967999)
\curveto(381.87285659,238.04967848)(381.95285651,237.7296788)(382.05287084,237.46967999)
\curveto(382.15285631,237.2296793)(382.2628562,237.03967949)(382.38287084,236.89967999)
\curveto(382.52285594,236.75967977)(382.67285579,236.63967989)(382.83287084,236.53967999)
\lineto(379.65287084,236.53967999)
\curveto(379.53285893,236.65967987)(379.43285903,236.83967969)(379.35287084,237.07967999)
\curveto(379.27285919,237.31967921)(379.20285926,237.57967895)(379.14287084,237.85967999)
\curveto(379.08285938,238.15967837)(379.03285943,238.45967807)(378.99287084,238.75967999)
\curveto(378.97285949,239.07967745)(378.95285951,239.36967716)(378.93287084,239.62967999)
\lineto(378.75287084,242.95967999)
\curveto(378.69285977,243.69967283)(378.55285991,244.26967226)(378.33287084,244.66967999)
\curveto(378.13286033,245.08967144)(377.88286058,245.39967113)(377.58287084,245.59967999)
\curveto(377.28286118,245.81967071)(376.95286151,245.94967058)(376.59287084,245.98967999)
\curveto(376.25286221,246.04967048)(375.91286255,246.07967045)(375.57287084,246.07967999)
\lineto(372.00287084,246.07967999)
\lineto(372.00287084,236.53967999)
\lineto(369.36287084,236.53967999)
\lineto(369.36287084,257.95967999)
}
}
{
\newrgbcolor{curcolor}{0 0 0}
\pscustom[linestyle=none,fillstyle=solid,fillcolor=curcolor]
{
\newpath
\moveto(391.54880834,256.21967999)
\curveto(390.62880092,256.21966031)(389.89880165,255.97966055)(389.35880834,255.49967999)
\curveto(388.81880273,255.03966149)(388.39880315,254.39966213)(388.09880834,253.57967999)
\curveto(387.81880373,252.77966375)(387.62880392,251.8296647)(387.52880834,250.72967999)
\curveto(387.4488041,249.64966688)(387.40880414,248.48966804)(387.40880834,247.24967999)
\curveto(387.40880414,246.00967052)(387.4488041,244.83967169)(387.52880834,243.73967999)
\curveto(387.62880392,242.65967387)(387.81880373,241.70967482)(388.09880834,240.88967999)
\curveto(388.39880315,240.08967644)(388.81880273,239.44967708)(389.35880834,238.96967999)
\curveto(389.89880165,238.50967802)(390.62880092,238.27967825)(391.54880834,238.27967999)
\curveto(392.46879908,238.27967825)(393.19879835,238.50967802)(393.73880834,238.96967999)
\curveto(394.27879727,239.44967708)(394.68879686,240.08967644)(394.96880834,240.88967999)
\curveto(395.26879628,241.70967482)(395.45879609,242.65967387)(395.53880834,243.73967999)
\curveto(395.63879591,244.83967169)(395.68879586,246.00967052)(395.68880834,247.24967999)
\curveto(395.68879586,248.48966804)(395.63879591,249.64966688)(395.53880834,250.72967999)
\curveto(395.45879609,251.8296647)(395.26879628,252.77966375)(394.96880834,253.57967999)
\curveto(394.68879686,254.39966213)(394.27879727,255.03966149)(393.73880834,255.49967999)
\curveto(393.19879835,255.97966055)(392.46879908,256.21966031)(391.54880834,256.21967999)
\moveto(391.54880834,258.37967999)
\curveto(393.02879852,258.37965815)(394.21879733,258.04965848)(395.11880834,257.38967999)
\curveto(396.01879553,256.74965978)(396.71879483,255.89966063)(397.21880834,254.83967999)
\curveto(397.71879383,253.77966275)(398.0487935,252.57966395)(398.20880834,251.23967999)
\curveto(398.36879318,249.91966661)(398.4487931,248.58966794)(398.44880834,247.24967999)
\curveto(398.4487931,245.88967064)(398.36879318,244.54967198)(398.20880834,243.22967999)
\curveto(398.0487935,241.90967462)(397.71879383,240.71967581)(397.21880834,239.65967999)
\curveto(396.71879483,238.59967793)(396.01879553,237.73967879)(395.11880834,237.07967999)
\curveto(394.21879733,236.43968009)(393.02879852,236.11968041)(391.54880834,236.11967999)
\curveto(390.06880148,236.11968041)(388.87880267,236.43968009)(387.97880834,237.07967999)
\curveto(387.07880447,237.73967879)(386.37880517,238.59967793)(385.87880834,239.65967999)
\curveto(385.37880617,240.71967581)(385.0488065,241.90967462)(384.88880834,243.22967999)
\curveto(384.72880682,244.54967198)(384.6488069,245.88967064)(384.64880834,247.24967999)
\curveto(384.6488069,248.58966794)(384.72880682,249.91966661)(384.88880834,251.23967999)
\curveto(385.0488065,252.57966395)(385.37880617,253.77966275)(385.87880834,254.83967999)
\curveto(386.37880517,255.89966063)(387.07880447,256.74965978)(387.97880834,257.38967999)
\curveto(388.87880267,258.04965848)(390.06880148,258.37965815)(391.54880834,258.37967999)
}
}
{
\newrgbcolor{curcolor}{0 0 0}
\pscustom[linestyle=none,fillstyle=solid,fillcolor=curcolor]
{
\newpath
\moveto(401.40802709,257.95967999)
\lineto(404.04802709,257.95967999)
\lineto(404.04802709,242.89967999)
\curveto(404.04802295,241.31967521)(404.31802268,240.14967638)(404.85802709,239.38967999)
\curveto(405.41802158,238.64967788)(406.35802064,238.27967825)(407.67802709,238.27967999)
\curveto(409.07801792,238.27967825)(410.03801696,238.66967786)(410.55802709,239.44967999)
\curveto(411.07801592,240.24967628)(411.33801566,241.39967513)(411.33802709,242.89967999)
\lineto(411.33802709,257.95967999)
\lineto(413.97802709,257.95967999)
\lineto(413.97802709,242.89967999)
\curveto(413.97801302,240.83967569)(413.44801355,239.18967734)(412.38802709,237.94967999)
\curveto(411.34801565,236.7296798)(409.77801722,236.11968041)(407.67802709,236.11967999)
\curveto(405.4980215,236.11968041)(403.90802309,236.68967984)(402.90802709,237.82967999)
\curveto(401.90802509,238.96967756)(401.40802559,240.65967587)(401.40802709,242.89967999)
\lineto(401.40802709,257.95967999)
}
}
{
\newrgbcolor{curcolor}{0 0 0}
\pscustom[linestyle=none,fillstyle=solid,fillcolor=curcolor]
{
\newpath
\moveto(423.74396459,236.53967999)
\lineto(421.10396459,236.53967999)
\lineto(421.10396459,255.61967999)
\lineto(415.73396459,255.61967999)
\lineto(415.73396459,257.95967999)
\lineto(429.14396459,257.95967999)
\lineto(429.14396459,255.61967999)
\lineto(423.74396459,255.61967999)
\lineto(423.74396459,236.53967999)
}
}
{
\newrgbcolor{curcolor}{0 0 0}
\pscustom[linestyle=none,fillstyle=solid,fillcolor=curcolor]
{
\newpath
\moveto(431.12068334,257.95967999)
\lineto(442.19068334,257.95967999)
\lineto(442.19068334,255.61967999)
\lineto(433.76068334,255.61967999)
\lineto(433.76068334,248.89967999)
\lineto(441.71068334,248.89967999)
\lineto(441.71068334,246.55967999)
\lineto(433.76068334,246.55967999)
\lineto(433.76068334,238.87967999)
\lineto(442.55068334,238.87967999)
\lineto(442.55068334,236.53967999)
\lineto(431.12068334,236.53967999)
\lineto(431.12068334,257.95967999)
}
}
{
\newrgbcolor{curcolor}{0 0 0}
\pscustom[linestyle=none,fillstyle=solid,fillcolor=curcolor]
{
\newpath
\moveto(443.26740209,234.28967999)
\lineto(458.26740209,234.28967999)
\lineto(458.26740209,232.78967999)
\lineto(443.26740209,232.78967999)
\lineto(443.26740209,234.28967999)
}
}
{
\newrgbcolor{curcolor}{0 0 0}
\pscustom[linestyle=none,fillstyle=solid,fillcolor=curcolor]
{
\newpath
\moveto(462.64740209,247.75967999)
\lineto(465.79740209,247.75967999)
\curveto(466.7573936,247.75966877)(467.58739277,248.09966843)(468.28740209,248.77967999)
\curveto(468.98739137,249.45966707)(469.33739102,250.50966602)(469.33740209,251.92967999)
\curveto(469.33739102,253.1296634)(469.04739131,254.06966246)(468.46740209,254.74967999)
\curveto(467.88739247,255.44966108)(466.93739342,255.79966073)(465.61740209,255.79967999)
\lineto(462.64740209,255.79967999)
\lineto(462.64740209,247.75967999)
\moveto(460.00740209,257.95967999)
\lineto(465.46740209,257.95967999)
\curveto(465.76739459,257.95965857)(466.13739422,257.94965858)(466.57740209,257.92967999)
\curveto(467.03739332,257.90965862)(467.50739285,257.83965869)(467.98740209,257.71967999)
\curveto(468.48739187,257.61965891)(468.97739138,257.43965909)(469.45740209,257.17967999)
\curveto(469.9573904,256.93965959)(470.39738996,256.58965994)(470.77740209,256.12967999)
\curveto(471.17738918,255.66966086)(471.49738886,255.08966144)(471.73740209,254.38967999)
\curveto(471.97738838,253.68966284)(472.09738826,252.8296637)(472.09740209,251.80967999)
\curveto(472.09738826,250.80966572)(471.94738841,249.91966661)(471.64740209,249.13967999)
\curveto(471.34738901,248.37966815)(470.91738944,247.7296688)(470.35740209,247.18967999)
\curveto(469.81739054,246.66966986)(469.16739119,246.26967026)(468.40740209,245.98967999)
\curveto(467.64739271,245.7296708)(466.81739354,245.59967093)(465.91740209,245.59967999)
\lineto(462.64740209,245.59967999)
\lineto(462.64740209,236.53967999)
\lineto(460.00740209,236.53967999)
\lineto(460.00740209,257.95967999)
}
}
{
\newrgbcolor{curcolor}{0 0 0}
\pscustom[linestyle=none,fillstyle=solid,fillcolor=curcolor]
{
\newpath
\moveto(477.06146459,248.23967999)
\lineto(479.67146459,248.23967999)
\curveto(480.05145722,248.23966829)(480.49145678,248.25966827)(480.99146459,248.29967999)
\curveto(481.51145576,248.33966819)(482.00145527,248.48966804)(482.46146459,248.74967999)
\curveto(482.92145435,249.00966752)(483.30145397,249.41966711)(483.60146459,249.97967999)
\curveto(483.92145335,250.53966599)(484.08145319,251.33966519)(484.08146459,252.37967999)
\curveto(484.08145319,253.45966307)(483.75145352,254.29966223)(483.09146459,254.89967999)
\curveto(482.43145484,255.49966103)(481.4714558,255.79966073)(480.21146459,255.79967999)
\lineto(477.06146459,255.79967999)
\lineto(477.06146459,248.23967999)
\moveto(474.42146459,257.95967999)
\lineto(481.35146459,257.95967999)
\curveto(483.05145422,257.95965857)(484.39145288,257.48965904)(485.37146459,256.54967999)
\curveto(486.35145092,255.60966092)(486.84145043,254.28966224)(486.84146459,252.58967999)
\curveto(486.84145043,252.00966452)(486.78145049,251.4296651)(486.66146459,250.84967999)
\curveto(486.56145071,250.26966626)(486.38145089,249.7296668)(486.12146459,249.22967999)
\curveto(485.86145141,248.74966778)(485.52145175,248.31966821)(485.10146459,247.93967999)
\curveto(484.68145259,247.57966895)(484.16145311,247.3296692)(483.54146459,247.18967999)
\lineto(483.54146459,247.12967999)
\curveto(484.48145279,247.0296695)(485.20145207,246.63966989)(485.70146459,245.95967999)
\curveto(486.22145105,245.27967125)(486.51145076,244.47967205)(486.57146459,243.55967999)
\lineto(486.75146459,239.89967999)
\curveto(486.7714505,239.29967723)(486.81145046,238.80967772)(486.87146459,238.42967999)
\curveto(486.93145034,238.04967848)(487.01145026,237.7296788)(487.11146459,237.46967999)
\curveto(487.21145006,237.2296793)(487.32144995,237.03967949)(487.44146459,236.89967999)
\curveto(487.58144969,236.75967977)(487.73144954,236.63967989)(487.89146459,236.53967999)
\lineto(484.71146459,236.53967999)
\curveto(484.59145268,236.65967987)(484.49145278,236.83967969)(484.41146459,237.07967999)
\curveto(484.33145294,237.31967921)(484.26145301,237.57967895)(484.20146459,237.85967999)
\curveto(484.14145313,238.15967837)(484.09145318,238.45967807)(484.05146459,238.75967999)
\curveto(484.03145324,239.07967745)(484.01145326,239.36967716)(483.99146459,239.62967999)
\lineto(483.81146459,242.95967999)
\curveto(483.75145352,243.69967283)(483.61145366,244.26967226)(483.39146459,244.66967999)
\curveto(483.19145408,245.08967144)(482.94145433,245.39967113)(482.64146459,245.59967999)
\curveto(482.34145493,245.81967071)(482.01145526,245.94967058)(481.65146459,245.98967999)
\curveto(481.31145596,246.04967048)(480.9714563,246.07967045)(480.63146459,246.07967999)
\lineto(477.06146459,246.07967999)
\lineto(477.06146459,236.53967999)
\lineto(474.42146459,236.53967999)
\lineto(474.42146459,257.95967999)
}
}
{
\newrgbcolor{curcolor}{0 0 0}
\pscustom[linestyle=none,fillstyle=solid,fillcolor=curcolor]
{
\newpath
\moveto(490.00740209,257.95967999)
\lineto(492.64740209,257.95967999)
\lineto(492.64740209,236.53967999)
\lineto(490.00740209,236.53967999)
\lineto(490.00740209,257.95967999)
}
}
{
\newrgbcolor{curcolor}{0 0 0}
\pscustom[linestyle=none,fillstyle=solid,fillcolor=curcolor]
{
\newpath
\moveto(494.30115209,257.95967999)
\lineto(497.06115209,257.95967999)
\lineto(501.26115209,239.56967999)
\lineto(501.32115209,239.56967999)
\lineto(505.52115209,257.95967999)
\lineto(508.28115209,257.95967999)
\lineto(502.88115209,236.53967999)
\lineto(499.52115209,236.53967999)
\lineto(494.30115209,257.95967999)
}
}
{
\newrgbcolor{curcolor}{0 0 0}
\pscustom[linestyle=none,fillstyle=solid,fillcolor=curcolor]
{
\newpath
\moveto(509.98787084,257.95967999)
\lineto(512.62787084,257.95967999)
\lineto(512.62787084,236.53967999)
\lineto(509.98787084,236.53967999)
\lineto(509.98787084,257.95967999)
}
}
{
\newrgbcolor{curcolor}{0 0 0}
\pscustom[linestyle=none,fillstyle=solid,fillcolor=curcolor]
{
\newpath
\moveto(516.08162084,257.95967999)
\lineto(518.72162084,257.95967999)
\lineto(518.72162084,238.87967999)
\lineto(527.42162084,238.87967999)
\lineto(527.42162084,236.53967999)
\lineto(516.08162084,236.53967999)
\lineto(516.08162084,257.95967999)
}
}
{
\newrgbcolor{curcolor}{0 0 0}
\pscustom[linestyle=none,fillstyle=solid,fillcolor=curcolor]
{
\newpath
\moveto(529.38240209,257.95967999)
\lineto(540.45240209,257.95967999)
\lineto(540.45240209,255.61967999)
\lineto(532.02240209,255.61967999)
\lineto(532.02240209,248.89967999)
\lineto(539.97240209,248.89967999)
\lineto(539.97240209,246.55967999)
\lineto(532.02240209,246.55967999)
\lineto(532.02240209,238.87967999)
\lineto(540.81240209,238.87967999)
\lineto(540.81240209,236.53967999)
\lineto(529.38240209,236.53967999)
\lineto(529.38240209,257.95967999)
}
}
{
\newrgbcolor{curcolor}{0 0 0}
\pscustom[linestyle=none,fillstyle=solid,fillcolor=curcolor]
{
\newpath
\moveto(553.43912084,252.01967999)
\curveto(553.39910897,252.57966395)(553.30910906,253.10966342)(553.16912084,253.60967999)
\curveto(553.04910932,254.1296624)(552.84910952,254.57966195)(552.56912084,254.95967999)
\curveto(552.30911006,255.33966119)(551.95911041,255.63966089)(551.51912084,255.85967999)
\curveto(551.07911129,256.09966043)(550.52911184,256.21966031)(549.86912084,256.21967999)
\curveto(548.94911342,256.21966031)(548.21911415,255.97966055)(547.67912084,255.49967999)
\curveto(547.13911523,255.03966149)(546.71911565,254.39966213)(546.41912084,253.57967999)
\curveto(546.13911623,252.77966375)(545.94911642,251.8296647)(545.84912084,250.72967999)
\curveto(545.7691166,249.64966688)(545.72911664,248.48966804)(545.72912084,247.24967999)
\curveto(545.72911664,246.00967052)(545.7691166,244.83967169)(545.84912084,243.73967999)
\curveto(545.94911642,242.65967387)(546.13911623,241.70967482)(546.41912084,240.88967999)
\curveto(546.71911565,240.08967644)(547.13911523,239.44967708)(547.67912084,238.96967999)
\curveto(548.21911415,238.50967802)(548.94911342,238.27967825)(549.86912084,238.27967999)
\curveto(550.78911158,238.27967825)(551.50911086,238.51967801)(552.02912084,238.99967999)
\curveto(552.54910982,239.49967703)(552.93910943,240.10967642)(553.19912084,240.82967999)
\curveto(553.45910891,241.54967498)(553.61910875,242.3296742)(553.67912084,243.16967999)
\curveto(553.75910861,244.00967252)(553.79910857,244.78967174)(553.79912084,245.50967999)
\lineto(549.56912084,245.50967999)
\lineto(549.56912084,247.66967999)
\lineto(556.19912084,247.66967999)
\lineto(556.19912084,236.53967999)
\lineto(554.21912084,236.53967999)
\lineto(554.21912084,239.44967999)
\lineto(554.15912084,239.44967999)
\curveto(553.87910849,238.529678)(553.33910903,237.73967879)(552.53912084,237.07967999)
\curveto(551.73911063,236.43968009)(550.73911163,236.11968041)(549.53912084,236.11967999)
\curveto(548.13911423,236.11968041)(547.00911536,236.4296801)(546.14912084,237.04967999)
\curveto(545.28911708,237.66967886)(544.61911775,238.48967804)(544.13912084,239.50967999)
\curveto(543.67911869,240.54967598)(543.369119,241.73967479)(543.20912084,243.07967999)
\curveto(543.04911932,244.41967211)(542.9691194,245.80967072)(542.96912084,247.24967999)
\curveto(542.9691194,248.58966794)(543.04911932,249.91966661)(543.20912084,251.23967999)
\curveto(543.369119,252.57966395)(543.69911867,253.77966275)(544.19912084,254.83967999)
\curveto(544.69911767,255.89966063)(545.39911697,256.74965978)(546.29912084,257.38967999)
\curveto(547.19911517,258.04965848)(548.38911398,258.37965815)(549.86912084,258.37967999)
\curveto(550.88911148,258.37965815)(551.74911062,258.24965828)(552.44912084,257.98967999)
\curveto(553.1691092,257.7296588)(553.75910861,257.38965914)(554.21912084,256.96967999)
\curveto(554.69910767,256.56965996)(555.0691073,256.11966041)(555.32912084,255.61967999)
\curveto(555.58910678,255.11966141)(555.77910659,254.6296619)(555.89912084,254.14967999)
\curveto(556.03910633,253.68966284)(556.11910625,253.25966327)(556.13912084,252.85967999)
\curveto(556.17910619,252.47966405)(556.19910617,252.19966433)(556.19912084,252.01967999)
\lineto(553.43912084,252.01967999)
}
}
{
\newrgbcolor{curcolor}{0 0 0}
\pscustom[linestyle=none,fillstyle=solid,fillcolor=curcolor]
{
\newpath
\moveto(559.38240209,257.95967999)
\lineto(570.45240209,257.95967999)
\lineto(570.45240209,255.61967999)
\lineto(562.02240209,255.61967999)
\lineto(562.02240209,248.89967999)
\lineto(569.97240209,248.89967999)
\lineto(569.97240209,246.55967999)
\lineto(562.02240209,246.55967999)
\lineto(562.02240209,238.87967999)
\lineto(570.81240209,238.87967999)
\lineto(570.81240209,236.53967999)
\lineto(559.38240209,236.53967999)
\lineto(559.38240209,257.95967999)
}
}
{
\newrgbcolor{curcolor}{0 0 0}
\pscustom[linewidth=2.37778282,linecolor=curcolor]
{
\newpath
\moveto(346.27652685,269.39578961)
\lineto(593.89874975,269.39578961)
\lineto(593.89874975,221.77357052)
\lineto(346.27652685,221.77357052)
\closepath
}
}
{
\newrgbcolor{curcolor}{0 0 0}
\pscustom[linestyle=none,fillstyle=solid,fillcolor=curcolor]
{
\newpath
\moveto(135.15906759,246.57468354)
\lineto(137.76906759,246.57468354)
\curveto(138.14906022,246.57467184)(138.58905978,246.59467182)(139.08906759,246.63468354)
\curveto(139.60905876,246.67467174)(140.09905827,246.82467159)(140.55906759,247.08468354)
\curveto(141.01905735,247.34467107)(141.39905697,247.75467066)(141.69906759,248.31468354)
\curveto(142.01905635,248.87466954)(142.17905619,249.67466874)(142.17906759,250.71468354)
\curveto(142.17905619,251.79466662)(141.84905652,252.63466578)(141.18906759,253.23468354)
\curveto(140.52905784,253.83466458)(139.5690588,254.13466428)(138.30906759,254.13468354)
\lineto(135.15906759,254.13468354)
\lineto(135.15906759,246.57468354)
\moveto(132.51906759,256.29468354)
\lineto(139.44906759,256.29468354)
\curveto(141.14905722,256.29466212)(142.48905588,255.82466259)(143.46906759,254.88468354)
\curveto(144.44905392,253.94466447)(144.93905343,252.62466579)(144.93906759,250.92468354)
\curveto(144.93905343,250.34466807)(144.87905349,249.76466865)(144.75906759,249.18468354)
\curveto(144.65905371,248.60466981)(144.47905389,248.06467035)(144.21906759,247.56468354)
\curveto(143.95905441,247.08467133)(143.61905475,246.65467176)(143.19906759,246.27468354)
\curveto(142.77905559,245.9146725)(142.25905611,245.66467275)(141.63906759,245.52468354)
\lineto(141.63906759,245.46468354)
\curveto(142.57905579,245.36467305)(143.29905507,244.97467344)(143.79906759,244.29468354)
\curveto(144.31905405,243.6146748)(144.60905376,242.8146756)(144.66906759,241.89468354)
\lineto(144.84906759,238.23468354)
\curveto(144.8690535,237.63468078)(144.90905346,237.14468127)(144.96906759,236.76468354)
\curveto(145.02905334,236.38468203)(145.10905326,236.06468235)(145.20906759,235.80468354)
\curveto(145.30905306,235.56468285)(145.41905295,235.37468304)(145.53906759,235.23468354)
\curveto(145.67905269,235.09468332)(145.82905254,234.97468344)(145.98906759,234.87468354)
\lineto(142.80906759,234.87468354)
\curveto(142.68905568,234.99468342)(142.58905578,235.17468324)(142.50906759,235.41468354)
\curveto(142.42905594,235.65468276)(142.35905601,235.9146825)(142.29906759,236.19468354)
\curveto(142.23905613,236.49468192)(142.18905618,236.79468162)(142.14906759,237.09468354)
\curveto(142.12905624,237.414681)(142.10905626,237.70468071)(142.08906759,237.96468354)
\lineto(141.90906759,241.29468354)
\curveto(141.84905652,242.03467638)(141.70905666,242.60467581)(141.48906759,243.00468354)
\curveto(141.28905708,243.42467499)(141.03905733,243.73467468)(140.73906759,243.93468354)
\curveto(140.43905793,244.15467426)(140.10905826,244.28467413)(139.74906759,244.32468354)
\curveto(139.40905896,244.38467403)(139.0690593,244.414674)(138.72906759,244.41468354)
\lineto(135.15906759,244.41468354)
\lineto(135.15906759,234.87468354)
\lineto(132.51906759,234.87468354)
\lineto(132.51906759,256.29468354)
}
}
{
\newrgbcolor{curcolor}{0 0 0}
\pscustom[linestyle=none,fillstyle=solid,fillcolor=curcolor]
{
\newpath
\moveto(154.70500509,254.55468354)
\curveto(153.78499767,254.55466386)(153.0549984,254.3146641)(152.51500509,253.83468354)
\curveto(151.97499948,253.37466504)(151.5549999,252.73466568)(151.25500509,251.91468354)
\curveto(150.97500048,251.1146673)(150.78500067,250.16466825)(150.68500509,249.06468354)
\curveto(150.60500085,247.98467043)(150.56500089,246.82467159)(150.56500509,245.58468354)
\curveto(150.56500089,244.34467407)(150.60500085,243.17467524)(150.68500509,242.07468354)
\curveto(150.78500067,240.99467742)(150.97500048,240.04467837)(151.25500509,239.22468354)
\curveto(151.5549999,238.42467999)(151.97499948,237.78468063)(152.51500509,237.30468354)
\curveto(153.0549984,236.84468157)(153.78499767,236.6146818)(154.70500509,236.61468354)
\curveto(155.62499583,236.6146818)(156.3549951,236.84468157)(156.89500509,237.30468354)
\curveto(157.43499402,237.78468063)(157.84499361,238.42467999)(158.12500509,239.22468354)
\curveto(158.42499303,240.04467837)(158.61499284,240.99467742)(158.69500509,242.07468354)
\curveto(158.79499266,243.17467524)(158.84499261,244.34467407)(158.84500509,245.58468354)
\curveto(158.84499261,246.82467159)(158.79499266,247.98467043)(158.69500509,249.06468354)
\curveto(158.61499284,250.16466825)(158.42499303,251.1146673)(158.12500509,251.91468354)
\curveto(157.84499361,252.73466568)(157.43499402,253.37466504)(156.89500509,253.83468354)
\curveto(156.3549951,254.3146641)(155.62499583,254.55466386)(154.70500509,254.55468354)
\moveto(154.70500509,256.71468354)
\curveto(156.18499527,256.7146617)(157.37499408,256.38466203)(158.27500509,255.72468354)
\curveto(159.17499228,255.08466333)(159.87499158,254.23466418)(160.37500509,253.17468354)
\curveto(160.87499058,252.1146663)(161.20499025,250.9146675)(161.36500509,249.57468354)
\curveto(161.52498993,248.25467016)(161.60498985,246.92467149)(161.60500509,245.58468354)
\curveto(161.60498985,244.22467419)(161.52498993,242.88467553)(161.36500509,241.56468354)
\curveto(161.20499025,240.24467817)(160.87499058,239.05467936)(160.37500509,237.99468354)
\curveto(159.87499158,236.93468148)(159.17499228,236.07468234)(158.27500509,235.41468354)
\curveto(157.37499408,234.77468364)(156.18499527,234.45468396)(154.70500509,234.45468354)
\curveto(153.22499823,234.45468396)(152.03499942,234.77468364)(151.13500509,235.41468354)
\curveto(150.23500122,236.07468234)(149.53500192,236.93468148)(149.03500509,237.99468354)
\curveto(148.53500292,239.05467936)(148.20500325,240.24467817)(148.04500509,241.56468354)
\curveto(147.88500357,242.88467553)(147.80500365,244.22467419)(147.80500509,245.58468354)
\curveto(147.80500365,246.92467149)(147.88500357,248.25467016)(148.04500509,249.57468354)
\curveto(148.20500325,250.9146675)(148.53500292,252.1146663)(149.03500509,253.17468354)
\curveto(149.53500192,254.23466418)(150.23500122,255.08466333)(151.13500509,255.72468354)
\curveto(152.03499942,256.38466203)(153.22499823,256.7146617)(154.70500509,256.71468354)
}
}
{
\newrgbcolor{curcolor}{0 0 0}
\pscustom[linestyle=none,fillstyle=solid,fillcolor=curcolor]
{
\newpath
\moveto(164.56422384,256.29468354)
\lineto(167.20422384,256.29468354)
\lineto(167.20422384,241.23468354)
\curveto(167.2042197,239.65467876)(167.47421943,238.48467993)(168.01422384,237.72468354)
\curveto(168.57421833,236.98468143)(169.51421739,236.6146818)(170.83422384,236.61468354)
\curveto(172.23421467,236.6146818)(173.19421371,237.00468141)(173.71422384,237.78468354)
\curveto(174.23421267,238.58467983)(174.49421241,239.73467868)(174.49422384,241.23468354)
\lineto(174.49422384,256.29468354)
\lineto(177.13422384,256.29468354)
\lineto(177.13422384,241.23468354)
\curveto(177.13420977,239.17467924)(176.6042103,237.52468089)(175.54422384,236.28468354)
\curveto(174.5042124,235.06468335)(172.93421397,234.45468396)(170.83422384,234.45468354)
\curveto(168.65421825,234.45468396)(167.06421984,235.02468339)(166.06422384,236.16468354)
\curveto(165.06422184,237.30468111)(164.56422234,238.99467942)(164.56422384,241.23468354)
\lineto(164.56422384,256.29468354)
}
}
{
\newrgbcolor{curcolor}{0 0 0}
\pscustom[linestyle=none,fillstyle=solid,fillcolor=curcolor]
{
\newpath
\moveto(186.90016134,234.87468354)
\lineto(184.26016134,234.87468354)
\lineto(184.26016134,253.95468354)
\lineto(178.89016134,253.95468354)
\lineto(178.89016134,256.29468354)
\lineto(192.30016134,256.29468354)
\lineto(192.30016134,253.95468354)
\lineto(186.90016134,253.95468354)
\lineto(186.90016134,234.87468354)
}
}
{
\newrgbcolor{curcolor}{0 0 0}
\pscustom[linestyle=none,fillstyle=solid,fillcolor=curcolor]
{
\newpath
\moveto(194.27688009,256.29468354)
\lineto(205.34688009,256.29468354)
\lineto(205.34688009,253.95468354)
\lineto(196.91688009,253.95468354)
\lineto(196.91688009,247.23468354)
\lineto(204.86688009,247.23468354)
\lineto(204.86688009,244.89468354)
\lineto(196.91688009,244.89468354)
\lineto(196.91688009,237.21468354)
\lineto(205.70688009,237.21468354)
\lineto(205.70688009,234.87468354)
\lineto(194.27688009,234.87468354)
\lineto(194.27688009,256.29468354)
}
}
{
\newrgbcolor{curcolor}{0 0 0}
\pscustom[linewidth=1.84937644,linecolor=curcolor]
{
\newpath
\moveto(95.03764303,269.65999237)
\lineto(243.1882778,269.65999237)
\lineto(243.1882778,221.50936905)
\lineto(95.03764303,221.50936905)
\closepath
}
}
{
\newrgbcolor{curcolor}{0 0 0}
\pscustom[linestyle=none,fillstyle=solid,fillcolor=curcolor]
{
\newpath
\moveto(397.15931505,169.65084021)
\lineto(399.76931505,169.65084021)
\curveto(400.14930768,169.65082851)(400.58930724,169.67082849)(401.08931505,169.71084021)
\curveto(401.60930622,169.75082841)(402.09930573,169.90082826)(402.55931505,170.16084021)
\curveto(403.01930481,170.42082774)(403.39930443,170.83082733)(403.69931505,171.39084021)
\curveto(404.01930381,171.95082621)(404.17930365,172.75082541)(404.17931505,173.79084021)
\curveto(404.17930365,174.87082329)(403.84930398,175.71082245)(403.18931505,176.31084021)
\curveto(402.5293053,176.91082125)(401.56930626,177.21082095)(400.30931505,177.21084021)
\lineto(397.15931505,177.21084021)
\lineto(397.15931505,169.65084021)
\moveto(394.51931505,179.37084021)
\lineto(401.44931505,179.37084021)
\curveto(403.14930468,179.37081879)(404.48930334,178.90081926)(405.46931505,177.96084021)
\curveto(406.44930138,177.02082114)(406.93930089,175.70082246)(406.93931505,174.00084021)
\curveto(406.93930089,173.42082474)(406.87930095,172.84082532)(406.75931505,172.26084021)
\curveto(406.65930117,171.68082648)(406.47930135,171.14082702)(406.21931505,170.64084021)
\curveto(405.95930187,170.160828)(405.61930221,169.73082843)(405.19931505,169.35084021)
\curveto(404.77930305,168.99082917)(404.25930357,168.74082942)(403.63931505,168.60084021)
\lineto(403.63931505,168.54084021)
\curveto(404.57930325,168.44082972)(405.29930253,168.05083011)(405.79931505,167.37084021)
\curveto(406.31930151,166.69083147)(406.60930122,165.89083227)(406.66931505,164.97084021)
\lineto(406.84931505,161.31084021)
\curveto(406.86930096,160.71083745)(406.90930092,160.22083794)(406.96931505,159.84084021)
\curveto(407.0293008,159.4608387)(407.10930072,159.14083902)(407.20931505,158.88084021)
\curveto(407.30930052,158.64083952)(407.41930041,158.45083971)(407.53931505,158.31084021)
\curveto(407.67930015,158.17083999)(407.8293,158.05084011)(407.98931505,157.95084021)
\lineto(404.80931505,157.95084021)
\curveto(404.68930314,158.07084009)(404.58930324,158.25083991)(404.50931505,158.49084021)
\curveto(404.4293034,158.73083943)(404.35930347,158.99083917)(404.29931505,159.27084021)
\curveto(404.23930359,159.57083859)(404.18930364,159.87083829)(404.14931505,160.17084021)
\curveto(404.1293037,160.49083767)(404.10930372,160.78083738)(404.08931505,161.04084021)
\lineto(403.90931505,164.37084021)
\curveto(403.84930398,165.11083305)(403.70930412,165.68083248)(403.48931505,166.08084021)
\curveto(403.28930454,166.50083166)(403.03930479,166.81083135)(402.73931505,167.01084021)
\curveto(402.43930539,167.23083093)(402.10930572,167.3608308)(401.74931505,167.40084021)
\curveto(401.40930642,167.4608307)(401.06930676,167.49083067)(400.72931505,167.49084021)
\lineto(397.15931505,167.49084021)
\lineto(397.15931505,157.95084021)
\lineto(394.51931505,157.95084021)
\lineto(394.51931505,179.37084021)
}
}
{
\newrgbcolor{curcolor}{0 0 0}
\pscustom[linestyle=none,fillstyle=solid,fillcolor=curcolor]
{
\newpath
\moveto(416.70525255,177.63084021)
\curveto(415.78524513,177.63082053)(415.05524586,177.39082077)(414.51525255,176.91084021)
\curveto(413.97524694,176.45082171)(413.55524736,175.81082235)(413.25525255,174.99084021)
\curveto(412.97524794,174.19082397)(412.78524813,173.24082492)(412.68525255,172.14084021)
\curveto(412.60524831,171.0608271)(412.56524835,169.90082826)(412.56525255,168.66084021)
\curveto(412.56524835,167.42083074)(412.60524831,166.25083191)(412.68525255,165.15084021)
\curveto(412.78524813,164.07083409)(412.97524794,163.12083504)(413.25525255,162.30084021)
\curveto(413.55524736,161.50083666)(413.97524694,160.8608373)(414.51525255,160.38084021)
\curveto(415.05524586,159.92083824)(415.78524513,159.69083847)(416.70525255,159.69084021)
\curveto(417.62524329,159.69083847)(418.35524256,159.92083824)(418.89525255,160.38084021)
\curveto(419.43524148,160.8608373)(419.84524107,161.50083666)(420.12525255,162.30084021)
\curveto(420.42524049,163.12083504)(420.6152403,164.07083409)(420.69525255,165.15084021)
\curveto(420.79524012,166.25083191)(420.84524007,167.42083074)(420.84525255,168.66084021)
\curveto(420.84524007,169.90082826)(420.79524012,171.0608271)(420.69525255,172.14084021)
\curveto(420.6152403,173.24082492)(420.42524049,174.19082397)(420.12525255,174.99084021)
\curveto(419.84524107,175.81082235)(419.43524148,176.45082171)(418.89525255,176.91084021)
\curveto(418.35524256,177.39082077)(417.62524329,177.63082053)(416.70525255,177.63084021)
\moveto(416.70525255,179.79084021)
\curveto(418.18524273,179.79081837)(419.37524154,179.4608187)(420.27525255,178.80084021)
\curveto(421.17523974,178.16082)(421.87523904,177.31082085)(422.37525255,176.25084021)
\curveto(422.87523804,175.19082297)(423.20523771,173.99082417)(423.36525255,172.65084021)
\curveto(423.52523739,171.33082683)(423.60523731,170.00082816)(423.60525255,168.66084021)
\curveto(423.60523731,167.30083086)(423.52523739,165.9608322)(423.36525255,164.64084021)
\curveto(423.20523771,163.32083484)(422.87523804,162.13083603)(422.37525255,161.07084021)
\curveto(421.87523904,160.01083815)(421.17523974,159.15083901)(420.27525255,158.49084021)
\curveto(419.37524154,157.85084031)(418.18524273,157.53084063)(416.70525255,157.53084021)
\curveto(415.22524569,157.53084063)(414.03524688,157.85084031)(413.13525255,158.49084021)
\curveto(412.23524868,159.15083901)(411.53524938,160.01083815)(411.03525255,161.07084021)
\curveto(410.53525038,162.13083603)(410.20525071,163.32083484)(410.04525255,164.64084021)
\curveto(409.88525103,165.9608322)(409.80525111,167.30083086)(409.80525255,168.66084021)
\curveto(409.80525111,170.00082816)(409.88525103,171.33082683)(410.04525255,172.65084021)
\curveto(410.20525071,173.99082417)(410.53525038,175.19082297)(411.03525255,176.25084021)
\curveto(411.53524938,177.31082085)(412.23524868,178.16082)(413.13525255,178.80084021)
\curveto(414.03524688,179.4608187)(415.22524569,179.79081837)(416.70525255,179.79084021)
}
}
{
\newrgbcolor{curcolor}{0 0 0}
\pscustom[linestyle=none,fillstyle=solid,fillcolor=curcolor]
{
\newpath
\moveto(426.5644713,179.37084021)
\lineto(429.2044713,179.37084021)
\lineto(429.2044713,164.31084021)
\curveto(429.20446716,162.73083543)(429.47446689,161.5608366)(430.0144713,160.80084021)
\curveto(430.57446579,160.0608381)(431.51446485,159.69083847)(432.8344713,159.69084021)
\curveto(434.23446213,159.69083847)(435.19446117,160.08083808)(435.7144713,160.86084021)
\curveto(436.23446013,161.6608365)(436.49445987,162.81083535)(436.4944713,164.31084021)
\lineto(436.4944713,179.37084021)
\lineto(439.1344713,179.37084021)
\lineto(439.1344713,164.31084021)
\curveto(439.13445723,162.25083591)(438.60445776,160.60083756)(437.5444713,159.36084021)
\curveto(436.50445986,158.14084002)(434.93446143,157.53084063)(432.8344713,157.53084021)
\curveto(430.65446571,157.53084063)(429.0644673,158.10084006)(428.0644713,159.24084021)
\curveto(427.0644693,160.38083778)(426.5644698,162.07083609)(426.5644713,164.31084021)
\lineto(426.5644713,179.37084021)
}
}
{
\newrgbcolor{curcolor}{0 0 0}
\pscustom[linestyle=none,fillstyle=solid,fillcolor=curcolor]
{
\newpath
\moveto(448.9004088,157.95084021)
\lineto(446.2604088,157.95084021)
\lineto(446.2604088,177.03084021)
\lineto(440.8904088,177.03084021)
\lineto(440.8904088,179.37084021)
\lineto(454.3004088,179.37084021)
\lineto(454.3004088,177.03084021)
\lineto(448.9004088,177.03084021)
\lineto(448.9004088,157.95084021)
}
}
{
\newrgbcolor{curcolor}{0 0 0}
\pscustom[linestyle=none,fillstyle=solid,fillcolor=curcolor]
{
\newpath
\moveto(456.27712755,179.37084021)
\lineto(467.34712755,179.37084021)
\lineto(467.34712755,177.03084021)
\lineto(458.91712755,177.03084021)
\lineto(458.91712755,170.31084021)
\lineto(466.86712755,170.31084021)
\lineto(466.86712755,167.97084021)
\lineto(458.91712755,167.97084021)
\lineto(458.91712755,160.29084021)
\lineto(467.70712755,160.29084021)
\lineto(467.70712755,157.95084021)
\lineto(456.27712755,157.95084021)
\lineto(456.27712755,179.37084021)
}
}
{
\newrgbcolor{curcolor}{0 0 0}
\pscustom[linestyle=none,fillstyle=solid,fillcolor=curcolor]
{
\newpath
\moveto(468.4238463,155.70084021)
\lineto(483.4238463,155.70084021)
\lineto(483.4238463,154.20084021)
\lineto(468.4238463,154.20084021)
\lineto(468.4238463,155.70084021)
}
}
{
\newrgbcolor{curcolor}{0 0 0}
\pscustom[linestyle=none,fillstyle=solid,fillcolor=curcolor]
{
\newpath
\moveto(485.2238463,179.37084021)
\lineto(489.6638463,179.37084021)
\lineto(493.9538463,162.39084021)
\lineto(494.0138463,162.39084021)
\lineto(498.3038463,179.37084021)
\lineto(502.7438463,179.37084021)
\lineto(502.7438463,157.95084021)
\lineto(500.1038463,157.95084021)
\lineto(500.1038463,176.67084021)
\lineto(500.0438463,176.67084021)
\lineto(495.3038463,157.95084021)
\lineto(492.6638463,157.95084021)
\lineto(487.9238463,176.67084021)
\lineto(487.8638463,176.67084021)
\lineto(487.8638463,157.95084021)
\lineto(485.2238463,157.95084021)
\lineto(485.2238463,179.37084021)
}
}
{
\newrgbcolor{curcolor}{0 0 0}
\pscustom[linestyle=none,fillstyle=solid,fillcolor=curcolor]
{
\newpath
\moveto(506.2575963,179.37084021)
\lineto(517.3275963,179.37084021)
\lineto(517.3275963,177.03084021)
\lineto(508.8975963,177.03084021)
\lineto(508.8975963,170.31084021)
\lineto(516.8475963,170.31084021)
\lineto(516.8475963,167.97084021)
\lineto(508.8975963,167.97084021)
\lineto(508.8975963,160.29084021)
\lineto(517.6875963,160.29084021)
\lineto(517.6875963,157.95084021)
\lineto(506.2575963,157.95084021)
\lineto(506.2575963,179.37084021)
}
}
{
\newrgbcolor{curcolor}{0 0 0}
\pscustom[linestyle=none,fillstyle=solid,fillcolor=curcolor]
{
\newpath
\moveto(520.14431505,179.37084021)
\lineto(522.75431505,179.37084021)
\lineto(523.56431505,179.37084021)
\lineto(530.64431505,161.49084021)
\lineto(530.70431505,161.49084021)
\lineto(530.70431505,179.37084021)
\lineto(533.34431505,179.37084021)
\lineto(533.34431505,157.95084021)
\lineto(530.73431505,157.95084021)
\lineto(529.71431505,157.95084021)
\lineto(522.84431505,175.29084021)
\lineto(522.78431505,175.29084021)
\lineto(522.78431505,157.95084021)
\lineto(520.14431505,157.95084021)
\lineto(520.14431505,179.37084021)
}
}
{
\newrgbcolor{curcolor}{0 0 0}
\pscustom[linestyle=none,fillstyle=solid,fillcolor=curcolor]
{
\newpath
\moveto(536.6035338,179.37084021)
\lineto(539.2435338,179.37084021)
\lineto(539.2435338,164.31084021)
\curveto(539.24352966,162.73083543)(539.51352939,161.5608366)(540.0535338,160.80084021)
\curveto(540.61352829,160.0608381)(541.55352735,159.69083847)(542.8735338,159.69084021)
\curveto(544.27352463,159.69083847)(545.23352367,160.08083808)(545.7535338,160.86084021)
\curveto(546.27352263,161.6608365)(546.53352237,162.81083535)(546.5335338,164.31084021)
\lineto(546.5335338,179.37084021)
\lineto(549.1735338,179.37084021)
\lineto(549.1735338,164.31084021)
\curveto(549.17351973,162.25083591)(548.64352026,160.60083756)(547.5835338,159.36084021)
\curveto(546.54352236,158.14084002)(544.97352393,157.53084063)(542.8735338,157.53084021)
\curveto(540.69352821,157.53084063)(539.1035298,158.10084006)(538.1035338,159.24084021)
\curveto(537.1035318,160.38083778)(536.6035323,162.07083609)(536.6035338,164.31084021)
\lineto(536.6035338,179.37084021)
}
}
{
\newrgbcolor{curcolor}{0 0 0}
\pscustom[linewidth=2.37778306,linecolor=curcolor]
{
\newpath
\moveto(348.03532701,190.80691932)
\lineto(595.65754991,190.80691932)
\lineto(595.65754991,143.18470405)
\lineto(348.03532701,143.18470405)
\closepath
}
}
{
\newrgbcolor{curcolor}{0 0 0}
\pscustom[linestyle=none,fillstyle=solid,fillcolor=curcolor]
{
\newpath
\moveto(382.82284273,96.89941627)
\lineto(385.46284273,96.89941627)
\lineto(388.73284273,79.37941627)
\lineto(388.79284273,79.37941627)
\lineto(391.82284273,96.89941627)
\lineto(395.00284273,96.89941627)
\lineto(398.03284273,79.37941627)
\lineto(398.09284273,79.37941627)
\lineto(401.36284273,96.89941627)
\lineto(404.00284273,96.89941627)
\lineto(399.59284273,75.47941627)
\lineto(396.38284273,75.47941627)
\lineto(393.44284273,92.81941627)
\lineto(393.38284273,92.81941627)
\lineto(390.29284273,75.47941627)
\lineto(387.08284273,75.47941627)
\lineto(382.82284273,96.89941627)
}
}
{
\newrgbcolor{curcolor}{0 0 0}
\pscustom[linestyle=none,fillstyle=solid,fillcolor=curcolor]
{
\newpath
\moveto(406.00253023,96.89941627)
\lineto(408.64253023,96.89941627)
\lineto(408.64253023,75.47941627)
\lineto(406.00253023,75.47941627)
\lineto(406.00253023,96.89941627)
}
}
{
\newrgbcolor{curcolor}{0 0 0}
\pscustom[linestyle=none,fillstyle=solid,fillcolor=curcolor]
{
\newpath
\moveto(412.36628023,96.89941627)
\lineto(418.09628023,96.89941627)
\curveto(419.75627083,96.89939485)(421.07626951,96.61939513)(422.05628023,96.05941627)
\curveto(423.05626753,95.49939625)(423.81626677,94.72939702)(424.33628023,93.74941627)
\curveto(424.85626573,92.78939896)(425.19626539,91.65940009)(425.35628023,90.35941627)
\curveto(425.51626507,89.05940269)(425.59626499,87.66940408)(425.59628023,86.18941627)
\curveto(425.59626499,84.82940692)(425.49626509,83.50940824)(425.29628023,82.22941627)
\curveto(425.09626549,80.9494108)(424.72626586,79.80941194)(424.18628023,78.80941627)
\curveto(423.64626694,77.80941394)(422.90626768,76.99941475)(421.96628023,76.37941627)
\curveto(421.02626956,75.77941597)(419.82627076,75.47941627)(418.36628023,75.47941627)
\lineto(412.36628023,75.47941627)
\lineto(412.36628023,96.89941627)
\moveto(415.00628023,77.63941627)
\lineto(417.76628023,77.63941627)
\curveto(418.90627168,77.63941411)(419.80627078,77.89941385)(420.46628023,78.41941627)
\curveto(421.14626944,78.95941279)(421.65626893,79.63941211)(421.99628023,80.45941627)
\curveto(422.35626823,81.27941047)(422.586268,82.18940956)(422.68628023,83.18941627)
\curveto(422.7862678,84.20940754)(422.83626775,85.19940655)(422.83628023,86.15941627)
\curveto(422.83626775,87.19940455)(422.79626779,88.22940352)(422.71628023,89.24941627)
\curveto(422.63626795,90.26940148)(422.42626816,91.17940057)(422.08628023,91.97941627)
\curveto(421.76626882,92.79939895)(421.26626932,93.45939829)(420.58628023,93.95941627)
\curveto(419.90627068,94.47939727)(418.96627162,94.73939701)(417.76628023,94.73941627)
\lineto(415.00628023,94.73941627)
\lineto(415.00628023,77.63941627)
}
}
{
\newrgbcolor{curcolor}{0 0 0}
\pscustom[linestyle=none,fillstyle=solid,fillcolor=curcolor]
{
\newpath
\moveto(438.96549898,90.95941627)
\curveto(438.92548711,91.51940023)(438.8354872,92.0493997)(438.69549898,92.54941627)
\curveto(438.57548746,93.06939868)(438.37548766,93.51939823)(438.09549898,93.89941627)
\curveto(437.8354882,94.27939747)(437.48548855,94.57939717)(437.04549898,94.79941627)
\curveto(436.60548943,95.03939671)(436.05548998,95.15939659)(435.39549898,95.15941627)
\curveto(434.47549156,95.15939659)(433.74549229,94.91939683)(433.20549898,94.43941627)
\curveto(432.66549337,93.97939777)(432.24549379,93.33939841)(431.94549898,92.51941627)
\curveto(431.66549437,91.71940003)(431.47549456,90.76940098)(431.37549898,89.66941627)
\curveto(431.29549474,88.58940316)(431.25549478,87.42940432)(431.25549898,86.18941627)
\curveto(431.25549478,84.9494068)(431.29549474,83.77940797)(431.37549898,82.67941627)
\curveto(431.47549456,81.59941015)(431.66549437,80.6494111)(431.94549898,79.82941627)
\curveto(432.24549379,79.02941272)(432.66549337,78.38941336)(433.20549898,77.90941627)
\curveto(433.74549229,77.4494143)(434.47549156,77.21941453)(435.39549898,77.21941627)
\curveto(436.31548972,77.21941453)(437.035489,77.45941429)(437.55549898,77.93941627)
\curveto(438.07548796,78.43941331)(438.46548757,79.0494127)(438.72549898,79.76941627)
\curveto(438.98548705,80.48941126)(439.14548689,81.26941048)(439.20549898,82.10941627)
\curveto(439.28548675,82.9494088)(439.32548671,83.72940802)(439.32549898,84.44941627)
\lineto(435.09549898,84.44941627)
\lineto(435.09549898,86.60941627)
\lineto(441.72549898,86.60941627)
\lineto(441.72549898,75.47941627)
\lineto(439.74549898,75.47941627)
\lineto(439.74549898,78.38941627)
\lineto(439.68549898,78.38941627)
\curveto(439.40548663,77.46941428)(438.86548717,76.67941507)(438.06549898,76.01941627)
\curveto(437.26548877,75.37941637)(436.26548977,75.05941669)(435.06549898,75.05941627)
\curveto(433.66549237,75.05941669)(432.5354935,75.36941638)(431.67549898,75.98941627)
\curveto(430.81549522,76.60941514)(430.14549589,77.42941432)(429.66549898,78.44941627)
\curveto(429.20549683,79.48941226)(428.89549714,80.67941107)(428.73549898,82.01941627)
\curveto(428.57549746,83.35940839)(428.49549754,84.749407)(428.49549898,86.18941627)
\curveto(428.49549754,87.52940422)(428.57549746,88.85940289)(428.73549898,90.17941627)
\curveto(428.89549714,91.51940023)(429.22549681,92.71939903)(429.72549898,93.77941627)
\curveto(430.22549581,94.83939691)(430.92549511,95.68939606)(431.82549898,96.32941627)
\curveto(432.72549331,96.98939476)(433.91549212,97.31939443)(435.39549898,97.31941627)
\curveto(436.41548962,97.31939443)(437.27548876,97.18939456)(437.97549898,96.92941627)
\curveto(438.69548734,96.66939508)(439.28548675,96.32939542)(439.74549898,95.90941627)
\curveto(440.22548581,95.50939624)(440.59548544,95.05939669)(440.85549898,94.55941627)
\curveto(441.11548492,94.05939769)(441.30548473,93.56939818)(441.42549898,93.08941627)
\curveto(441.56548447,92.62939912)(441.64548439,92.19939955)(441.66549898,91.79941627)
\curveto(441.70548433,91.41940033)(441.72548431,91.13940061)(441.72549898,90.95941627)
\lineto(438.96549898,90.95941627)
}
}
{
\newrgbcolor{curcolor}{0 0 0}
\pscustom[linestyle=none,fillstyle=solid,fillcolor=curcolor]
{
\newpath
\moveto(444.90878023,96.89941627)
\lineto(455.97878023,96.89941627)
\lineto(455.97878023,94.55941627)
\lineto(447.54878023,94.55941627)
\lineto(447.54878023,87.83941627)
\lineto(455.49878023,87.83941627)
\lineto(455.49878023,85.49941627)
\lineto(447.54878023,85.49941627)
\lineto(447.54878023,77.81941627)
\lineto(456.33878023,77.81941627)
\lineto(456.33878023,75.47941627)
\lineto(444.90878023,75.47941627)
\lineto(444.90878023,96.89941627)
}
}
{
\newrgbcolor{curcolor}{0 0 0}
\pscustom[linestyle=none,fillstyle=solid,fillcolor=curcolor]
{
\newpath
\moveto(465.30549898,75.47941627)
\lineto(462.66549898,75.47941627)
\lineto(462.66549898,94.55941627)
\lineto(457.29549898,94.55941627)
\lineto(457.29549898,96.89941627)
\lineto(470.70549898,96.89941627)
\lineto(470.70549898,94.55941627)
\lineto(465.30549898,94.55941627)
\lineto(465.30549898,75.47941627)
}
}
{
\newrgbcolor{curcolor}{0 0 0}
\pscustom[linestyle=none,fillstyle=solid,fillcolor=curcolor]
{
\newpath
\moveto(470.94221773,73.22941627)
\lineto(485.94221773,73.22941627)
\lineto(485.94221773,71.72941627)
\lineto(470.94221773,71.72941627)
\lineto(470.94221773,73.22941627)
}
}
{
\newrgbcolor{curcolor}{0 0 0}
\pscustom[linestyle=none,fillstyle=solid,fillcolor=curcolor]
{
\newpath
\moveto(490.32221773,87.17941627)
\lineto(492.93221773,87.17941627)
\curveto(493.31221036,87.17940457)(493.75220992,87.19940455)(494.25221773,87.23941627)
\curveto(494.7722089,87.27940447)(495.26220841,87.42940432)(495.72221773,87.68941627)
\curveto(496.18220749,87.9494038)(496.56220711,88.35940339)(496.86221773,88.91941627)
\curveto(497.18220649,89.47940227)(497.34220633,90.27940147)(497.34221773,91.31941627)
\curveto(497.34220633,92.39939935)(497.01220666,93.23939851)(496.35221773,93.83941627)
\curveto(495.69220798,94.43939731)(494.73220894,94.73939701)(493.47221773,94.73941627)
\lineto(490.32221773,94.73941627)
\lineto(490.32221773,87.17941627)
\moveto(487.68221773,96.89941627)
\lineto(494.61221773,96.89941627)
\curveto(496.31220736,96.89939485)(497.65220602,96.42939532)(498.63221773,95.48941627)
\curveto(499.61220406,94.5493972)(500.10220357,93.22939852)(500.10221773,91.52941627)
\curveto(500.10220357,90.9494008)(500.04220363,90.36940138)(499.92221773,89.78941627)
\curveto(499.82220385,89.20940254)(499.64220403,88.66940308)(499.38221773,88.16941627)
\curveto(499.12220455,87.68940406)(498.78220489,87.25940449)(498.36221773,86.87941627)
\curveto(497.94220573,86.51940523)(497.42220625,86.26940548)(496.80221773,86.12941627)
\lineto(496.80221773,86.06941627)
\curveto(497.74220593,85.96940578)(498.46220521,85.57940617)(498.96221773,84.89941627)
\curveto(499.48220419,84.21940753)(499.7722039,83.41940833)(499.83221773,82.49941627)
\lineto(500.01221773,78.83941627)
\curveto(500.03220364,78.23941351)(500.0722036,77.749414)(500.13221773,77.36941627)
\curveto(500.19220348,76.98941476)(500.2722034,76.66941508)(500.37221773,76.40941627)
\curveto(500.4722032,76.16941558)(500.58220309,75.97941577)(500.70221773,75.83941627)
\curveto(500.84220283,75.69941605)(500.99220268,75.57941617)(501.15221773,75.47941627)
\lineto(497.97221773,75.47941627)
\curveto(497.85220582,75.59941615)(497.75220592,75.77941597)(497.67221773,76.01941627)
\curveto(497.59220608,76.25941549)(497.52220615,76.51941523)(497.46221773,76.79941627)
\curveto(497.40220627,77.09941465)(497.35220632,77.39941435)(497.31221773,77.69941627)
\curveto(497.29220638,78.01941373)(497.2722064,78.30941344)(497.25221773,78.56941627)
\lineto(497.07221773,81.89941627)
\curveto(497.01220666,82.63940911)(496.8722068,83.20940854)(496.65221773,83.60941627)
\curveto(496.45220722,84.02940772)(496.20220747,84.33940741)(495.90221773,84.53941627)
\curveto(495.60220807,84.75940699)(495.2722084,84.88940686)(494.91221773,84.92941627)
\curveto(494.5722091,84.98940676)(494.23220944,85.01940673)(493.89221773,85.01941627)
\lineto(490.32221773,85.01941627)
\lineto(490.32221773,75.47941627)
\lineto(487.68221773,75.47941627)
\lineto(487.68221773,96.89941627)
}
}
{
\newrgbcolor{curcolor}{0 0 0}
\pscustom[linestyle=none,fillstyle=solid,fillcolor=curcolor]
{
\newpath
\moveto(509.86815523,95.15941627)
\curveto(508.94814781,95.15939659)(508.21814854,94.91939683)(507.67815523,94.43941627)
\curveto(507.13814962,93.97939777)(506.71815004,93.33939841)(506.41815523,92.51941627)
\curveto(506.13815062,91.71940003)(505.94815081,90.76940098)(505.84815523,89.66941627)
\curveto(505.76815099,88.58940316)(505.72815103,87.42940432)(505.72815523,86.18941627)
\curveto(505.72815103,84.9494068)(505.76815099,83.77940797)(505.84815523,82.67941627)
\curveto(505.94815081,81.59941015)(506.13815062,80.6494111)(506.41815523,79.82941627)
\curveto(506.71815004,79.02941272)(507.13814962,78.38941336)(507.67815523,77.90941627)
\curveto(508.21814854,77.4494143)(508.94814781,77.21941453)(509.86815523,77.21941627)
\curveto(510.78814597,77.21941453)(511.51814524,77.4494143)(512.05815523,77.90941627)
\curveto(512.59814416,78.38941336)(513.00814375,79.02941272)(513.28815523,79.82941627)
\curveto(513.58814317,80.6494111)(513.77814298,81.59941015)(513.85815523,82.67941627)
\curveto(513.9581428,83.77940797)(514.00814275,84.9494068)(514.00815523,86.18941627)
\curveto(514.00814275,87.42940432)(513.9581428,88.58940316)(513.85815523,89.66941627)
\curveto(513.77814298,90.76940098)(513.58814317,91.71940003)(513.28815523,92.51941627)
\curveto(513.00814375,93.33939841)(512.59814416,93.97939777)(512.05815523,94.43941627)
\curveto(511.51814524,94.91939683)(510.78814597,95.15939659)(509.86815523,95.15941627)
\moveto(509.86815523,97.31941627)
\curveto(511.34814541,97.31939443)(512.53814422,96.98939476)(513.43815523,96.32941627)
\curveto(514.33814242,95.68939606)(515.03814172,94.83939691)(515.53815523,93.77941627)
\curveto(516.03814072,92.71939903)(516.36814039,91.51940023)(516.52815523,90.17941627)
\curveto(516.68814007,88.85940289)(516.76813999,87.52940422)(516.76815523,86.18941627)
\curveto(516.76813999,84.82940692)(516.68814007,83.48940826)(516.52815523,82.16941627)
\curveto(516.36814039,80.8494109)(516.03814072,79.65941209)(515.53815523,78.59941627)
\curveto(515.03814172,77.53941421)(514.33814242,76.67941507)(513.43815523,76.01941627)
\curveto(512.53814422,75.37941637)(511.34814541,75.05941669)(509.86815523,75.05941627)
\curveto(508.38814837,75.05941669)(507.19814956,75.37941637)(506.29815523,76.01941627)
\curveto(505.39815136,76.67941507)(504.69815206,77.53941421)(504.19815523,78.59941627)
\curveto(503.69815306,79.65941209)(503.36815339,80.8494109)(503.20815523,82.16941627)
\curveto(503.04815371,83.48940826)(502.96815379,84.82940692)(502.96815523,86.18941627)
\curveto(502.96815379,87.52940422)(503.04815371,88.85940289)(503.20815523,90.17941627)
\curveto(503.36815339,91.51940023)(503.69815306,92.71939903)(504.19815523,93.77941627)
\curveto(504.69815206,94.83939691)(505.39815136,95.68939606)(506.29815523,96.32941627)
\curveto(507.19814956,96.98939476)(508.38814837,97.31939443)(509.86815523,97.31941627)
}
}
{
\newrgbcolor{curcolor}{0 0 0}
\pscustom[linestyle=none,fillstyle=solid,fillcolor=curcolor]
{
\newpath
\moveto(519.72737398,96.89941627)
\lineto(522.36737398,96.89941627)
\lineto(522.36737398,81.83941627)
\curveto(522.36736984,80.25941149)(522.63736957,79.08941266)(523.17737398,78.32941627)
\curveto(523.73736847,77.58941416)(524.67736753,77.21941453)(525.99737398,77.21941627)
\curveto(527.39736481,77.21941453)(528.35736385,77.60941414)(528.87737398,78.38941627)
\curveto(529.39736281,79.18941256)(529.65736255,80.33941141)(529.65737398,81.83941627)
\lineto(529.65737398,96.89941627)
\lineto(532.29737398,96.89941627)
\lineto(532.29737398,81.83941627)
\curveto(532.29735991,79.77941197)(531.76736044,78.12941362)(530.70737398,76.88941627)
\curveto(529.66736254,75.66941608)(528.09736411,75.05941669)(525.99737398,75.05941627)
\curveto(523.81736839,75.05941669)(522.22736998,75.62941612)(521.22737398,76.76941627)
\curveto(520.22737198,77.90941384)(519.72737248,79.59941215)(519.72737398,81.83941627)
\lineto(519.72737398,96.89941627)
}
}
{
\newrgbcolor{curcolor}{0 0 0}
\pscustom[linestyle=none,fillstyle=solid,fillcolor=curcolor]
{
\newpath
\moveto(542.06331148,75.47941627)
\lineto(539.42331148,75.47941627)
\lineto(539.42331148,94.55941627)
\lineto(534.05331148,94.55941627)
\lineto(534.05331148,96.89941627)
\lineto(547.46331148,96.89941627)
\lineto(547.46331148,94.55941627)
\lineto(542.06331148,94.55941627)
\lineto(542.06331148,75.47941627)
}
}
{
\newrgbcolor{curcolor}{0 0 0}
\pscustom[linestyle=none,fillstyle=solid,fillcolor=curcolor]
{
\newpath
\moveto(549.44003023,96.89941627)
\lineto(560.51003023,96.89941627)
\lineto(560.51003023,94.55941627)
\lineto(552.08003023,94.55941627)
\lineto(552.08003023,87.83941627)
\lineto(560.03003023,87.83941627)
\lineto(560.03003023,85.49941627)
\lineto(552.08003023,85.49941627)
\lineto(552.08003023,77.81941627)
\lineto(560.87003023,77.81941627)
\lineto(560.87003023,75.47941627)
\lineto(549.44003023,75.47941627)
\lineto(549.44003023,96.89941627)
}
}
{
\newrgbcolor{curcolor}{0 0 0}
\pscustom[linewidth=2.37778306,linecolor=curcolor]
{
\newpath
\moveto(348.03532687,108.33549537)
\lineto(595.65754977,108.33549537)
\lineto(595.65754977,60.7132801)
\lineto(348.03532687,60.7132801)
\closepath
}
}
{
\newrgbcolor{curcolor}{0 0 0}
\pscustom[linestyle=none,fillstyle=solid,fillcolor=curcolor]
{
\newpath
\moveto(125.17163234,95.23435559)
\lineto(127.81163234,95.23435559)
\lineto(131.08163234,77.71435559)
\lineto(131.14163234,77.71435559)
\lineto(134.17163234,95.23435559)
\lineto(137.35163234,95.23435559)
\lineto(140.38163234,77.71435559)
\lineto(140.44163234,77.71435559)
\lineto(143.71163234,95.23435559)
\lineto(146.35163234,95.23435559)
\lineto(141.94163234,73.81435559)
\lineto(138.73163234,73.81435559)
\lineto(135.79163234,91.15435559)
\lineto(135.73163234,91.15435559)
\lineto(132.64163234,73.81435559)
\lineto(129.43163234,73.81435559)
\lineto(125.17163234,95.23435559)
}
}
{
\newrgbcolor{curcolor}{0 0 0}
\pscustom[linestyle=none,fillstyle=solid,fillcolor=curcolor]
{
\newpath
\moveto(148.35131984,95.23435559)
\lineto(150.99131984,95.23435559)
\lineto(150.99131984,73.81435559)
\lineto(148.35131984,73.81435559)
\lineto(148.35131984,95.23435559)
}
}
{
\newrgbcolor{curcolor}{0 0 0}
\pscustom[linestyle=none,fillstyle=solid,fillcolor=curcolor]
{
\newpath
\moveto(154.71506984,95.23435559)
\lineto(160.44506984,95.23435559)
\curveto(162.10506044,95.23433417)(163.42505912,94.95433445)(164.40506984,94.39435559)
\curveto(165.40505714,93.83433557)(166.16505638,93.06433634)(166.68506984,92.08435559)
\curveto(167.20505534,91.12433828)(167.545055,89.99433941)(167.70506984,88.69435559)
\curveto(167.86505468,87.39434201)(167.9450546,86.0043434)(167.94506984,84.52435559)
\curveto(167.9450546,83.16434624)(167.8450547,81.84434756)(167.64506984,80.56435559)
\curveto(167.4450551,79.28435012)(167.07505547,78.14435126)(166.53506984,77.14435559)
\curveto(165.99505655,76.14435326)(165.25505729,75.33435407)(164.31506984,74.71435559)
\curveto(163.37505917,74.11435529)(162.17506037,73.81435559)(160.71506984,73.81435559)
\lineto(154.71506984,73.81435559)
\lineto(154.71506984,95.23435559)
\moveto(157.35506984,75.97435559)
\lineto(160.11506984,75.97435559)
\curveto(161.25506129,75.97435343)(162.15506039,76.23435317)(162.81506984,76.75435559)
\curveto(163.49505905,77.29435211)(164.00505854,77.97435143)(164.34506984,78.79435559)
\curveto(164.70505784,79.61434979)(164.93505761,80.52434888)(165.03506984,81.52435559)
\curveto(165.13505741,82.54434686)(165.18505736,83.53434587)(165.18506984,84.49435559)
\curveto(165.18505736,85.53434387)(165.1450574,86.56434284)(165.06506984,87.58435559)
\curveto(164.98505756,88.6043408)(164.77505777,89.51433989)(164.43506984,90.31435559)
\curveto(164.11505843,91.13433827)(163.61505893,91.79433761)(162.93506984,92.29435559)
\curveto(162.25506029,92.81433659)(161.31506123,93.07433633)(160.11506984,93.07435559)
\lineto(157.35506984,93.07435559)
\lineto(157.35506984,75.97435559)
}
}
{
\newrgbcolor{curcolor}{0 0 0}
\pscustom[linestyle=none,fillstyle=solid,fillcolor=curcolor]
{
\newpath
\moveto(181.31428859,89.29435559)
\curveto(181.27427672,89.85433955)(181.18427681,90.38433902)(181.04428859,90.88435559)
\curveto(180.92427707,91.404338)(180.72427727,91.85433755)(180.44428859,92.23435559)
\curveto(180.18427781,92.61433679)(179.83427816,92.91433649)(179.39428859,93.13435559)
\curveto(178.95427904,93.37433603)(178.40427959,93.49433591)(177.74428859,93.49435559)
\curveto(176.82428117,93.49433591)(176.0942819,93.25433615)(175.55428859,92.77435559)
\curveto(175.01428298,92.31433709)(174.5942834,91.67433773)(174.29428859,90.85435559)
\curveto(174.01428398,90.05433935)(173.82428417,89.1043403)(173.72428859,88.00435559)
\curveto(173.64428435,86.92434248)(173.60428439,85.76434364)(173.60428859,84.52435559)
\curveto(173.60428439,83.28434612)(173.64428435,82.11434729)(173.72428859,81.01435559)
\curveto(173.82428417,79.93434947)(174.01428398,78.98435042)(174.29428859,78.16435559)
\curveto(174.5942834,77.36435204)(175.01428298,76.72435268)(175.55428859,76.24435559)
\curveto(176.0942819,75.78435362)(176.82428117,75.55435385)(177.74428859,75.55435559)
\curveto(178.66427933,75.55435385)(179.38427861,75.79435361)(179.90428859,76.27435559)
\curveto(180.42427757,76.77435263)(180.81427718,77.38435202)(181.07428859,78.10435559)
\curveto(181.33427666,78.82435058)(181.4942765,79.6043498)(181.55428859,80.44435559)
\curveto(181.63427636,81.28434812)(181.67427632,82.06434734)(181.67428859,82.78435559)
\lineto(177.44428859,82.78435559)
\lineto(177.44428859,84.94435559)
\lineto(184.07428859,84.94435559)
\lineto(184.07428859,73.81435559)
\lineto(182.09428859,73.81435559)
\lineto(182.09428859,76.72435559)
\lineto(182.03428859,76.72435559)
\curveto(181.75427624,75.8043536)(181.21427678,75.01435439)(180.41428859,74.35435559)
\curveto(179.61427838,73.71435569)(178.61427938,73.39435601)(177.41428859,73.39435559)
\curveto(176.01428198,73.39435601)(174.88428311,73.7043557)(174.02428859,74.32435559)
\curveto(173.16428483,74.94435446)(172.4942855,75.76435364)(172.01428859,76.78435559)
\curveto(171.55428644,77.82435158)(171.24428675,79.01435039)(171.08428859,80.35435559)
\curveto(170.92428707,81.69434771)(170.84428715,83.08434632)(170.84428859,84.52435559)
\curveto(170.84428715,85.86434354)(170.92428707,87.19434221)(171.08428859,88.51435559)
\curveto(171.24428675,89.85433955)(171.57428642,91.05433835)(172.07428859,92.11435559)
\curveto(172.57428542,93.17433623)(173.27428472,94.02433538)(174.17428859,94.66435559)
\curveto(175.07428292,95.32433408)(176.26428173,95.65433375)(177.74428859,95.65435559)
\curveto(178.76427923,95.65433375)(179.62427837,95.52433388)(180.32428859,95.26435559)
\curveto(181.04427695,95.0043344)(181.63427636,94.66433474)(182.09428859,94.24435559)
\curveto(182.57427542,93.84433556)(182.94427505,93.39433601)(183.20428859,92.89435559)
\curveto(183.46427453,92.39433701)(183.65427434,91.9043375)(183.77428859,91.42435559)
\curveto(183.91427408,90.96433844)(183.994274,90.53433887)(184.01428859,90.13435559)
\curveto(184.05427394,89.75433965)(184.07427392,89.47433993)(184.07428859,89.29435559)
\lineto(181.31428859,89.29435559)
}
}
{
\newrgbcolor{curcolor}{0 0 0}
\pscustom[linestyle=none,fillstyle=solid,fillcolor=curcolor]
{
\newpath
\moveto(187.25756984,95.23435559)
\lineto(198.32756984,95.23435559)
\lineto(198.32756984,92.89435559)
\lineto(189.89756984,92.89435559)
\lineto(189.89756984,86.17435559)
\lineto(197.84756984,86.17435559)
\lineto(197.84756984,83.83435559)
\lineto(189.89756984,83.83435559)
\lineto(189.89756984,76.15435559)
\lineto(198.68756984,76.15435559)
\lineto(198.68756984,73.81435559)
\lineto(187.25756984,73.81435559)
\lineto(187.25756984,95.23435559)
}
}
{
\newrgbcolor{curcolor}{0 0 0}
\pscustom[linestyle=none,fillstyle=solid,fillcolor=curcolor]
{
\newpath
\moveto(207.65428859,73.81435559)
\lineto(205.01428859,73.81435559)
\lineto(205.01428859,92.89435559)
\lineto(199.64428859,92.89435559)
\lineto(199.64428859,95.23435559)
\lineto(213.05428859,95.23435559)
\lineto(213.05428859,92.89435559)
\lineto(207.65428859,92.89435559)
\lineto(207.65428859,73.81435559)
}
}
{
\newrgbcolor{curcolor}{0 0 0}
\pscustom[linewidth=1.84937644,linecolor=curcolor]
{
\newpath
\moveto(95.03764278,108.59969494)
\lineto(243.18827754,108.59969494)
\lineto(243.18827754,60.44907162)
\lineto(95.03764278,60.44907162)
\closepath
}
}
{
\newrgbcolor{curcolor}{0 0 0}
\pscustom[linewidth=2,linecolor=curcolor]
{
\newpath
\moveto(209.843326,376.71835)
\lineto(209.843326,387.0755)
}
}
{
\newrgbcolor{curcolor}{1 1 1}
\pscustom[linestyle=none,fillstyle=solid,fillcolor=curcolor]
{
\newpath
\moveto(202.527406,381.81691)
\lineto(345.655026,381.81691)
}
}
{
\newrgbcolor{curcolor}{0 0 0}
\pscustom[linewidth=2,linecolor=curcolor]
{
\newpath
\moveto(202.527406,381.81691)
\lineto(345.655026,381.81691)
}
}
{
\newrgbcolor{curcolor}{1 1 1}
\pscustom[linestyle=none,fillstyle=solid,fillcolor=curcolor]
{
\newpath
\moveto(337.646279,376.57861)
\lineto(329.123117,381.75564)
\lineto(337.646279,386.93267)
\closepath
}
}
{
\newrgbcolor{curcolor}{0 0 0}
\pscustom[linewidth=2,linecolor=curcolor]
{
\newpath
\moveto(337.646279,376.57861)
\lineto(329.123117,381.75564)
\lineto(337.646279,386.93267)
\closepath
}
}
{
\newrgbcolor{curcolor}{1 1 1}
\pscustom[linestyle=none,fillstyle=solid,fillcolor=curcolor]
{
\newpath
\moveto(326.73273481,384.69090758)
\curveto(328.40931357,384.69090758)(329.7684492,383.33177196)(329.7684492,381.6551932)
\curveto(329.7684492,379.97861443)(328.40931357,378.61947881)(326.73273481,378.61947881)
\curveto(325.05615605,378.61947881)(323.69702043,379.97861443)(323.69702043,381.6551932)
\curveto(323.69702043,383.33177196)(325.05615605,384.69090758)(326.73273481,384.69090758)
\closepath
}
}
{
\newrgbcolor{curcolor}{0 0 0}
\pscustom[linewidth=2,linecolor=curcolor]
{
\newpath
\moveto(326.73273481,384.69090758)
\curveto(328.40931357,384.69090758)(329.7684492,383.33177196)(329.7684492,381.6551932)
\curveto(329.7684492,379.97861443)(328.40931357,378.61947881)(326.73273481,378.61947881)
\curveto(325.05615605,378.61947881)(323.69702043,379.97861443)(323.69702043,381.6551932)
\curveto(323.69702043,383.33177196)(325.05615605,384.69090758)(326.73273481,384.69090758)
\closepath
}
}
{
\newrgbcolor{curcolor}{1 1 1}
\pscustom[linestyle=none,fillstyle=solid,fillcolor=curcolor]
{
\newpath
\moveto(337.714872,385.14226)
\lineto(345.0363,385.14226)
}
}
{
\newrgbcolor{curcolor}{0 0 0}
\pscustom[linewidth=2,linecolor=curcolor]
{
\newpath
\moveto(337.714872,385.14226)
\lineto(345.0363,385.14226)
}
}
{
\newrgbcolor{curcolor}{1 1 1}
\pscustom[linestyle=none,fillstyle=solid,fillcolor=curcolor]
{
\newpath
\moveto(337.741024,378.34673)
\lineto(345.330309,378.34673)
}
}
{
\newrgbcolor{curcolor}{0 0 0}
\pscustom[linewidth=2,linecolor=curcolor]
{
\newpath
\moveto(337.741024,378.34673)
\lineto(345.330309,378.34673)
}
}
{
\newrgbcolor{curcolor}{0 0 0}
\pscustom[linewidth=2,linecolor=curcolor]
{
\newpath
\moveto(62.739852,406.140582)
\lineto(62.739852,439.475616)
\lineto(181.937856,439.475616)
\lineto(181.937856,405.509237)
}
}
{
\newrgbcolor{curcolor}{1 1 1}
\pscustom[linestyle=none,fillstyle=solid,fillcolor=curcolor]
{
\newpath
\moveto(65.73982038,413.87047971)
\curveto(65.73982038,412.19390095)(64.38068476,410.83476532)(62.704106,410.83476532)
\curveto(61.02752723,410.83476532)(59.66839161,412.19390095)(59.66839161,413.87047971)
\curveto(59.66839161,415.54705847)(61.02752723,416.90619409)(62.704106,416.90619409)
\curveto(64.38068476,416.90619409)(65.73982038,415.54705847)(65.73982038,413.87047971)
\closepath
}
}
{
\newrgbcolor{curcolor}{0 0 0}
\pscustom[linewidth=2,linecolor=curcolor]
{
\newpath
\moveto(65.73982038,413.87047971)
\curveto(65.73982038,412.19390095)(64.38068476,410.83476532)(62.704106,410.83476532)
\curveto(61.02752723,410.83476532)(59.66839161,412.19390095)(59.66839161,413.87047971)
\curveto(59.66839161,415.54705847)(61.02752723,416.90619409)(62.704106,416.90619409)
\curveto(64.38068476,416.90619409)(65.73982038,415.54705847)(65.73982038,413.87047971)
\closepath
}
}
{
\newrgbcolor{curcolor}{1 1 1}
\pscustom[linestyle=none,fillstyle=solid,fillcolor=curcolor]
{
\newpath
\moveto(184.78565358,417.05368069)
\curveto(184.78565358,415.37710193)(183.42651796,414.0179663)(181.7499392,414.0179663)
\curveto(180.07336043,414.0179663)(178.71422481,415.37710193)(178.71422481,417.05368069)
\curveto(178.71422481,418.73025945)(180.07336043,420.08939507)(181.7499392,420.08939507)
\curveto(183.42651796,420.08939507)(184.78565358,418.73025945)(184.78565358,417.05368069)
\closepath
}
}
{
\newrgbcolor{curcolor}{0 0 0}
\pscustom[linewidth=2,linecolor=curcolor]
{
\newpath
\moveto(184.78565358,417.05368069)
\curveto(184.78565358,415.37710193)(183.42651796,414.0179663)(181.7499392,414.0179663)
\curveto(180.07336043,414.0179663)(178.71422481,415.37710193)(178.71422481,417.05368069)
\curveto(178.71422481,418.73025945)(180.07336043,420.08939507)(181.7499392,420.08939507)
\curveto(183.42651796,420.08939507)(184.78565358,418.73025945)(184.78565358,417.05368069)
\closepath
}
}
{
\newrgbcolor{curcolor}{0 0 0}
\pscustom[linewidth=2,linecolor=curcolor]
{
\newpath
\moveto(176.625206,405.531035)
\lineto(181.802236,414.054208)
\lineto(186.979266,405.531035)
}
}
{
\newrgbcolor{curcolor}{0 0 0}
\pscustom[linewidth=2,linecolor=curcolor]
{
\newpath
\moveto(107.739852,271.082162)
\lineto(107.739852,304.417196)
\lineto(226.937856,304.417196)
\lineto(226.937856,270.450817)
}
}
{
\newrgbcolor{curcolor}{1 1 1}
\pscustom[linestyle=none,fillstyle=solid,fillcolor=curcolor]
{
\newpath
\moveto(110.73982038,278.81205971)
\curveto(110.73982038,277.13548095)(109.38068476,275.77634532)(107.704106,275.77634532)
\curveto(106.02752723,275.77634532)(104.66839161,277.13548095)(104.66839161,278.81205971)
\curveto(104.66839161,280.48863847)(106.02752723,281.84777409)(107.704106,281.84777409)
\curveto(109.38068476,281.84777409)(110.73982038,280.48863847)(110.73982038,278.81205971)
\closepath
}
}
{
\newrgbcolor{curcolor}{0 0 0}
\pscustom[linewidth=2,linecolor=curcolor]
{
\newpath
\moveto(110.73982038,278.81205971)
\curveto(110.73982038,277.13548095)(109.38068476,275.77634532)(107.704106,275.77634532)
\curveto(106.02752723,275.77634532)(104.66839161,277.13548095)(104.66839161,278.81205971)
\curveto(104.66839161,280.48863847)(106.02752723,281.84777409)(107.704106,281.84777409)
\curveto(109.38068476,281.84777409)(110.73982038,280.48863847)(110.73982038,278.81205971)
\closepath
}
}
{
\newrgbcolor{curcolor}{1 1 1}
\pscustom[linestyle=none,fillstyle=solid,fillcolor=curcolor]
{
\newpath
\moveto(229.78565358,281.99526069)
\curveto(229.78565358,280.31868193)(228.42651796,278.9595463)(226.7499392,278.9595463)
\curveto(225.07336043,278.9595463)(223.71422481,280.31868193)(223.71422481,281.99526069)
\curveto(223.71422481,283.67183945)(225.07336043,285.03097507)(226.7499392,285.03097507)
\curveto(228.42651796,285.03097507)(229.78565358,283.67183945)(229.78565358,281.99526069)
\closepath
}
}
{
\newrgbcolor{curcolor}{0 0 0}
\pscustom[linewidth=2,linecolor=curcolor]
{
\newpath
\moveto(229.78565358,281.99526069)
\curveto(229.78565358,280.31868193)(228.42651796,278.9595463)(226.7499392,278.9595463)
\curveto(225.07336043,278.9595463)(223.71422481,280.31868193)(223.71422481,281.99526069)
\curveto(223.71422481,283.67183945)(225.07336043,285.03097507)(226.7499392,285.03097507)
\curveto(228.42651796,285.03097507)(229.78565358,283.67183945)(229.78565358,281.99526069)
\closepath
}
}
{
\newrgbcolor{curcolor}{0 0 0}
\pscustom[linewidth=2,linecolor=curcolor]
{
\newpath
\moveto(221.625206,270.472615)
\lineto(226.802236,278.995788)
\lineto(231.979266,270.472615)
}
}
{
\newrgbcolor{curcolor}{0 0 0}
\pscustom[linewidth=2,linecolor=curcolor]
{
\newpath
\moveto(63.954116,357.88834)
\lineto(63.954116,245.0312)
\lineto(93.954116,245.0312)
}
}
{
\newrgbcolor{curcolor}{0 0 0}
\pscustom[linewidth=2,linecolor=curcolor]
{
\newpath
\moveto(58.239823,349.13834)
\lineto(68.596986,349.13834)
}
}
{
\newrgbcolor{curcolor}{1 1 1}
\pscustom[linestyle=none,fillstyle=solid,fillcolor=curcolor]
{
\newpath
\moveto(86.623371,239.95547)
\lineto(78.100209,245.1325)
\lineto(86.623371,250.30953)
\closepath
}
}
{
\newrgbcolor{curcolor}{0 0 0}
\pscustom[linewidth=2,linecolor=curcolor]
{
\newpath
\moveto(86.623371,239.95547)
\lineto(78.100209,245.1325)
\lineto(86.623371,250.30953)
\closepath
}
}
{
\newrgbcolor{curcolor}{1 1 1}
\pscustom[linestyle=none,fillstyle=solid,fillcolor=curcolor]
{
\newpath
\moveto(75.70982681,248.06776758)
\curveto(77.38640557,248.06776758)(78.7455412,246.70863196)(78.7455412,245.0320532)
\curveto(78.7455412,243.35547443)(77.38640557,241.99633881)(75.70982681,241.99633881)
\curveto(74.03324805,241.99633881)(72.67411243,243.35547443)(72.67411243,245.0320532)
\curveto(72.67411243,246.70863196)(74.03324805,248.06776758)(75.70982681,248.06776758)
\closepath
}
}
{
\newrgbcolor{curcolor}{0 0 0}
\pscustom[linewidth=2,linecolor=curcolor]
{
\newpath
\moveto(75.70982681,248.06776758)
\curveto(77.38640557,248.06776758)(78.7455412,246.70863196)(78.7455412,245.0320532)
\curveto(78.7455412,243.35547443)(77.38640557,241.99633881)(75.70982681,241.99633881)
\curveto(74.03324805,241.99633881)(72.67411243,243.35547443)(72.67411243,245.0320532)
\curveto(72.67411243,246.70863196)(74.03324805,248.06776758)(75.70982681,248.06776758)
\closepath
}
}
{
\newrgbcolor{curcolor}{1 1 1}
\pscustom[linestyle=none,fillstyle=solid,fillcolor=curcolor]
{
\newpath
\moveto(86.691964,248.51912)
\lineto(94.013392,248.51912)
}
}
{
\newrgbcolor{curcolor}{0 0 0}
\pscustom[linewidth=2,linecolor=curcolor]
{
\newpath
\moveto(86.691964,248.51912)
\lineto(94.013392,248.51912)
}
}
{
\newrgbcolor{curcolor}{1 1 1}
\pscustom[linestyle=none,fillstyle=solid,fillcolor=curcolor]
{
\newpath
\moveto(86.718116,241.72359)
\lineto(94.307401,241.72359)
}
}
{
\newrgbcolor{curcolor}{0 0 0}
\pscustom[linewidth=2,linecolor=curcolor]
{
\newpath
\moveto(86.718116,241.72359)
\lineto(94.307401,241.72359)
}
}
{
\newrgbcolor{curcolor}{0 0 0}
\pscustom[linewidth=2,linecolor=curcolor]
{
\newpath
\moveto(243.557156,244.0111)
\lineto(345.582566,244.0111)
}
}
{
\newrgbcolor{curcolor}{0 0 0}
\pscustom[linewidth=2,linecolor=curcolor]
{
\newpath
\moveto(412.252636,356.64048)
\lineto(412.252636,270.27506)
}
}
{
\newrgbcolor{curcolor}{1 1 1}
\pscustom[linestyle=none,fillstyle=solid,fillcolor=curcolor]
{
\newpath
\moveto(337.646289,238.94531)
\lineto(329.123127,244.12234)
\lineto(337.646289,249.29937)
\closepath
}
}
{
\newrgbcolor{curcolor}{0 0 0}
\pscustom[linewidth=2,linecolor=curcolor]
{
\newpath
\moveto(337.646289,238.94531)
\lineto(329.123127,244.12234)
\lineto(337.646289,249.29937)
\closepath
}
}
{
\newrgbcolor{curcolor}{1 1 1}
\pscustom[linestyle=none,fillstyle=solid,fillcolor=curcolor]
{
\newpath
\moveto(326.73274481,247.05760758)
\curveto(328.40932357,247.05760758)(329.7684592,245.69847196)(329.7684592,244.0218932)
\curveto(329.7684592,242.34531443)(328.40932357,240.98617881)(326.73274481,240.98617881)
\curveto(325.05616605,240.98617881)(323.69703043,242.34531443)(323.69703043,244.0218932)
\curveto(323.69703043,245.69847196)(325.05616605,247.05760758)(326.73274481,247.05760758)
\closepath
}
}
{
\newrgbcolor{curcolor}{0 0 0}
\pscustom[linewidth=2,linecolor=curcolor]
{
\newpath
\moveto(326.73274481,247.05760758)
\curveto(328.40932357,247.05760758)(329.7684592,245.69847196)(329.7684592,244.0218932)
\curveto(329.7684592,242.34531443)(328.40932357,240.98617881)(326.73274481,240.98617881)
\curveto(325.05616605,240.98617881)(323.69703043,242.34531443)(323.69703043,244.0218932)
\curveto(323.69703043,245.69847196)(325.05616605,247.05760758)(326.73274481,247.05760758)
\closepath
}
}
{
\newrgbcolor{curcolor}{1 1 1}
\pscustom[linestyle=none,fillstyle=solid,fillcolor=curcolor]
{
\newpath
\moveto(337.714882,247.50896)
\lineto(345.03631,247.50896)
}
}
{
\newrgbcolor{curcolor}{0 0 0}
\pscustom[linewidth=2,linecolor=curcolor]
{
\newpath
\moveto(337.714882,247.50896)
\lineto(345.03631,247.50896)
}
}
{
\newrgbcolor{curcolor}{1 1 1}
\pscustom[linestyle=none,fillstyle=solid,fillcolor=curcolor]
{
\newpath
\moveto(337.741034,240.71343)
\lineto(345.330319,240.71343)
}
}
{
\newrgbcolor{curcolor}{0 0 0}
\pscustom[linewidth=2,linecolor=curcolor]
{
\newpath
\moveto(337.741034,240.71343)
\lineto(345.330319,240.71343)
}
}
{
\newrgbcolor{curcolor}{0 0 0}
\pscustom[linewidth=2,linecolor=curcolor]
{
\newpath
\moveto(250.249426,239.3376)
\lineto(250.249426,249.69475)
}
}
{
\newrgbcolor{curcolor}{0 0 0}
\pscustom[linewidth=2,linecolor=curcolor]
{
\newpath
\moveto(406.742476,349.13834)
\lineto(417.099636,349.13834)
}
}
{
\newrgbcolor{curcolor}{1 1 1}
\pscustom[linestyle=none,fillstyle=solid,fillcolor=curcolor]
{
\newpath
\moveto(407.118376,278.112817)
\lineto(412.295406,286.635979)
\lineto(417.472436,278.112817)
\closepath
}
}
{
\newrgbcolor{curcolor}{0 0 0}
\pscustom[linewidth=2,linecolor=curcolor]
{
\newpath
\moveto(407.118376,278.112817)
\lineto(412.295406,286.635979)
\lineto(417.472436,278.112817)
\closepath
}
}
{
\newrgbcolor{curcolor}{1 1 1}
\pscustom[linestyle=none,fillstyle=solid,fillcolor=curcolor]
{
\newpath
\moveto(415.23067358,289.02636119)
\curveto(415.23067358,287.34978243)(413.87153796,285.9906468)(412.1949592,285.9906468)
\curveto(410.51838043,285.9906468)(409.15924481,287.34978243)(409.15924481,289.02636119)
\curveto(409.15924481,290.70293995)(410.51838043,292.06207557)(412.1949592,292.06207557)
\curveto(413.87153796,292.06207557)(415.23067358,290.70293995)(415.23067358,289.02636119)
\closepath
}
}
{
\newrgbcolor{curcolor}{0 0 0}
\pscustom[linewidth=2,linecolor=curcolor]
{
\newpath
\moveto(415.23067358,289.02636119)
\curveto(415.23067358,287.34978243)(413.87153796,285.9906468)(412.1949592,285.9906468)
\curveto(410.51838043,285.9906468)(409.15924481,287.34978243)(409.15924481,289.02636119)
\curveto(409.15924481,290.70293995)(410.51838043,292.06207557)(412.1949592,292.06207557)
\curveto(413.87153796,292.06207557)(415.23067358,290.70293995)(415.23067358,289.02636119)
\closepath
}
}
{
\newrgbcolor{curcolor}{1 1 1}
\pscustom[linestyle=none,fillstyle=solid,fillcolor=curcolor]
{
\newpath
\moveto(415.682026,278.044224)
\lineto(415.682026,270.722796)
}
}
{
\newrgbcolor{curcolor}{0 0 0}
\pscustom[linewidth=2,linecolor=curcolor]
{
\newpath
\moveto(415.682026,278.044224)
\lineto(415.682026,270.722796)
}
}
{
\newrgbcolor{curcolor}{1 1 1}
\pscustom[linestyle=none,fillstyle=solid,fillcolor=curcolor]
{
\newpath
\moveto(408.886496,278.018072)
\lineto(408.886496,270.428787)
}
}
{
\newrgbcolor{curcolor}{0 0 0}
\pscustom[linewidth=2,linecolor=curcolor]
{
\newpath
\moveto(408.886496,278.018072)
\lineto(408.886496,270.428787)
}
}
{
\newrgbcolor{curcolor}{0 0 0}
\pscustom[linewidth=2,linecolor=curcolor]
{
\newpath
\moveto(220.527756,220.50759)
\lineto(220.527756,166.49184)
\lineto(347.754936,166.49184)
}
}
{
\newrgbcolor{curcolor}{0 0 0}
\pscustom[linewidth=2,linecolor=curcolor]
{
\newpath
\moveto(214.813466,211.75759)
\lineto(225.170626,211.75759)
}
}
{
\newrgbcolor{curcolor}{1 1 1}
\pscustom[linestyle=none,fillstyle=solid,fillcolor=curcolor]
{
\newpath
\moveto(339.919109,161.41611)
\lineto(331.395947,166.59314)
\lineto(339.919109,171.77017)
\closepath
}
}
{
\newrgbcolor{curcolor}{0 0 0}
\pscustom[linewidth=2,linecolor=curcolor]
{
\newpath
\moveto(339.919109,161.41611)
\lineto(331.395947,166.59314)
\lineto(339.919109,171.77017)
\closepath
}
}
{
\newrgbcolor{curcolor}{1 1 1}
\pscustom[linestyle=none,fillstyle=solid,fillcolor=curcolor]
{
\newpath
\moveto(329.00556481,169.52840758)
\curveto(330.68214357,169.52840758)(332.0412792,168.16927196)(332.0412792,166.4926932)
\curveto(332.0412792,164.81611443)(330.68214357,163.45697881)(329.00556481,163.45697881)
\curveto(327.32898605,163.45697881)(325.96985043,164.81611443)(325.96985043,166.4926932)
\curveto(325.96985043,168.16927196)(327.32898605,169.52840758)(329.00556481,169.52840758)
\closepath
}
}
{
\newrgbcolor{curcolor}{0 0 0}
\pscustom[linewidth=2,linecolor=curcolor]
{
\newpath
\moveto(329.00556481,169.52840758)
\curveto(330.68214357,169.52840758)(332.0412792,168.16927196)(332.0412792,166.4926932)
\curveto(332.0412792,164.81611443)(330.68214357,163.45697881)(329.00556481,163.45697881)
\curveto(327.32898605,163.45697881)(325.96985043,164.81611443)(325.96985043,166.4926932)
\curveto(325.96985043,168.16927196)(327.32898605,169.52840758)(329.00556481,169.52840758)
\closepath
}
}
{
\newrgbcolor{curcolor}{1 1 1}
\pscustom[linestyle=none,fillstyle=solid,fillcolor=curcolor]
{
\newpath
\moveto(339.987702,169.97976)
\lineto(347.30913,169.97976)
}
}
{
\newrgbcolor{curcolor}{0 0 0}
\pscustom[linewidth=2,linecolor=curcolor]
{
\newpath
\moveto(339.987702,169.97976)
\lineto(347.30913,169.97976)
}
}
{
\newrgbcolor{curcolor}{1 1 1}
\pscustom[linestyle=none,fillstyle=solid,fillcolor=curcolor]
{
\newpath
\moveto(340.013854,163.18423)
\lineto(347.603139,163.18423)
}
}
{
\newrgbcolor{curcolor}{0 0 0}
\pscustom[linewidth=2,linecolor=curcolor]
{
\newpath
\moveto(340.013854,163.18423)
\lineto(347.603139,163.18423)
}
}
{
\newrgbcolor{curcolor}{0 0 0}
\pscustom[linewidth=2,linecolor=curcolor]
{
\newpath
\moveto(245.0723848,83.29681)
\lineto(347.0977948,83.29681)
}
}
{
\newrgbcolor{curcolor}{1 1 1}
\pscustom[linestyle=none,fillstyle=solid,fillcolor=curcolor]
{
\newpath
\moveto(339.1615178,78.23102)
\lineto(330.6383558,83.40805)
\lineto(339.1615178,88.58508)
\closepath
}
}
{
\newrgbcolor{curcolor}{0 0 0}
\pscustom[linewidth=2,linecolor=curcolor]
{
\newpath
\moveto(339.1615178,78.23102)
\lineto(330.6383558,83.40805)
\lineto(339.1615178,88.58508)
\closepath
}
}
{
\newrgbcolor{curcolor}{1 1 1}
\pscustom[linestyle=none,fillstyle=solid,fillcolor=curcolor]
{
\newpath
\moveto(328.24797361,86.34331758)
\curveto(329.92455237,86.34331758)(331.283688,84.98418196)(331.283688,83.3076032)
\curveto(331.283688,81.63102443)(329.92455237,80.27188881)(328.24797361,80.27188881)
\curveto(326.57139485,80.27188881)(325.21225923,81.63102443)(325.21225923,83.3076032)
\curveto(325.21225923,84.98418196)(326.57139485,86.34331758)(328.24797361,86.34331758)
\closepath
}
}
{
\newrgbcolor{curcolor}{0 0 0}
\pscustom[linewidth=2,linecolor=curcolor]
{
\newpath
\moveto(328.24797361,86.34331758)
\curveto(329.92455237,86.34331758)(331.283688,84.98418196)(331.283688,83.3076032)
\curveto(331.283688,81.63102443)(329.92455237,80.27188881)(328.24797361,80.27188881)
\curveto(326.57139485,80.27188881)(325.21225923,81.63102443)(325.21225923,83.3076032)
\curveto(325.21225923,84.98418196)(326.57139485,86.34331758)(328.24797361,86.34331758)
\closepath
}
}
{
\newrgbcolor{curcolor}{1 1 1}
\pscustom[linestyle=none,fillstyle=solid,fillcolor=curcolor]
{
\newpath
\moveto(339.2301108,86.79467)
\lineto(346.5515388,86.79467)
}
}
{
\newrgbcolor{curcolor}{0 0 0}
\pscustom[linewidth=2,linecolor=curcolor]
{
\newpath
\moveto(339.2301108,86.79467)
\lineto(346.5515388,86.79467)
}
}
{
\newrgbcolor{curcolor}{1 1 1}
\pscustom[linestyle=none,fillstyle=solid,fillcolor=curcolor]
{
\newpath
\moveto(339.2562628,79.99914)
\lineto(346.8455478,79.99914)
}
}
{
\newrgbcolor{curcolor}{0 0 0}
\pscustom[linewidth=2,linecolor=curcolor]
{
\newpath
\moveto(339.2562628,79.99914)
\lineto(346.8455478,79.99914)
}
}
{
\newrgbcolor{curcolor}{0 0 0}
\pscustom[linewidth=2,linecolor=curcolor]
{
\newpath
\moveto(251.7646548,78.62331)
\lineto(251.7646548,88.98046)
}
}
{
\newrgbcolor{curcolor}{0 0 0}
\pscustom[linewidth=2,linecolor=curcolor]
{
\newpath
\moveto(63.954116,357.88834)
\lineto(63.954116,84.16441)
\lineto(93.954116,84.16441)
}
}
{
\newrgbcolor{curcolor}{0 0 0}
\pscustom[linewidth=2,linecolor=curcolor]
{
\newpath
\moveto(58.239823,349.13834)
\lineto(68.596986,349.13834)
}
}
{
\newrgbcolor{curcolor}{1 1 1}
\pscustom[linestyle=none,fillstyle=solid,fillcolor=curcolor]
{
\newpath
\moveto(86.623371,79.08868)
\lineto(78.100209,84.26571)
\lineto(86.623371,89.44274)
\closepath
}
}
{
\newrgbcolor{curcolor}{0 0 0}
\pscustom[linewidth=2,linecolor=curcolor]
{
\newpath
\moveto(86.623371,79.08868)
\lineto(78.100209,84.26571)
\lineto(86.623371,89.44274)
\closepath
}
}
{
\newrgbcolor{curcolor}{1 1 1}
\pscustom[linestyle=none,fillstyle=solid,fillcolor=curcolor]
{
\newpath
\moveto(75.70982681,87.20097758)
\curveto(77.38640557,87.20097758)(78.7455412,85.84184196)(78.7455412,84.1652632)
\curveto(78.7455412,82.48868443)(77.38640557,81.12954881)(75.70982681,81.12954881)
\curveto(74.03324805,81.12954881)(72.67411243,82.48868443)(72.67411243,84.1652632)
\curveto(72.67411243,85.84184196)(74.03324805,87.20097758)(75.70982681,87.20097758)
\closepath
}
}
{
\newrgbcolor{curcolor}{0 0 0}
\pscustom[linewidth=2,linecolor=curcolor]
{
\newpath
\moveto(75.70982681,87.20097758)
\curveto(77.38640557,87.20097758)(78.7455412,85.84184196)(78.7455412,84.1652632)
\curveto(78.7455412,82.48868443)(77.38640557,81.12954881)(75.70982681,81.12954881)
\curveto(74.03324805,81.12954881)(72.67411243,82.48868443)(72.67411243,84.1652632)
\curveto(72.67411243,85.84184196)(74.03324805,87.20097758)(75.70982681,87.20097758)
\closepath
}
}
{
\newrgbcolor{curcolor}{1 1 1}
\pscustom[linestyle=none,fillstyle=solid,fillcolor=curcolor]
{
\newpath
\moveto(86.691964,87.65233)
\lineto(94.013392,87.65233)
}
}
{
\newrgbcolor{curcolor}{0 0 0}
\pscustom[linewidth=2,linecolor=curcolor]
{
\newpath
\moveto(86.691964,87.65233)
\lineto(94.013392,87.65233)
}
}
{
\newrgbcolor{curcolor}{1 1 1}
\pscustom[linestyle=none,fillstyle=solid,fillcolor=curcolor]
{
\newpath
\moveto(86.718116,80.8568)
\lineto(94.307401,80.8568)
}
}
{
\newrgbcolor{curcolor}{0 0 0}
\pscustom[linewidth=2,linecolor=curcolor]
{
\newpath
\moveto(86.718116,80.8568)
\lineto(94.307401,80.8568)
}
}
\end{pspicture}

\caption{Modelo Entidad-Relación de la base funcional del sistema.}
\label{modelo1}
\end{figure}

En la figura (\ref{modelo1}), pueden verse los relacionamientos entre estas
entidades a nivel de base de datos.

\subsection{\emph{packages}: Manejador de paquetes}
Las principales funciones de este paquete son:

\begin{itemize}
\item Instalación de paquetes en el sistema.
\item Manejo de dependencias entre paquetes del sistema.
\item Establecimiento de rutas de acceso para un paquete determinado.
\end{itemize}

\subsection{\emph{privileges}: Manejador de privilegios}
Las principales funciones de este paquete son:

\begin{itemize}
\item Registro de privilegios reservados por cada paquete.
\item Control de acceso a recursos y acciones especificas.
\end{itemize}

\subsection{\emph{routes}: Manejador de rutas de navegación}
Las principales funciones de este paquete son:

\begin{itemize}
\item Registro de rutas reservadas por cada paquete.
\end{itemize}

\subsection{Creación de un paquete del sistema}
Despues de contruidas las funcionalidades antes mencionadas se llego a una
definición precisa en la construcción de funcionalidad nueva, ahora detallaremos
tal proceso a partir del ultimo paquete construido en el sistema, el paquete de
sugerencias.

Puede verse en el ejemplo, que este registra cuatro diferentes tipos de
privilegios; cuatro diferentes tipos de rutas de acceso, aquí puede notarse que
una ruta no necesariamente atiende exclusivamente a la petición GET, sino a
cualquier tipo de petición que sea necesario realizar (ya sea POST, PUT, DELETE,
etc); además puede verse el establecimiento de privilegios para el acceso a las
rutas.

Este proceso consiste en los siguientes pasos:

\begin{figure}
\centering
\begin{SQL}
CREATE TABLE `feedback` (
    `resource`    int unsigned NOT NULL,
    `description` text         NOT NULL,
    `resolved`    boolean      NOT NULL DEFAULT FALSE,
    `mark`        boolean      NOT NULL DEFAULT FALSE,
    PRIMARY KEY (`resource`),
    INDEX (`resource`),
    FOREIGN KEY (`resource`) REFERENCES `resource`(`ident`)
        ON UPDATE CASCADE ON DELETE RESTRICT
) DEFAULT CHARACTER SET UTF8;
\end{SQL}
\caption{Definición de datos para el modulo de sugerencias}
\label{code1}
\end{figure}

\begin{figure}
\centering
\begin{SQL}
INSERT INTO `package` (
    `label`, `url`, `type`, `parent`,
    `tsregister`, `description`)
VALUES
    ('feedback', 'feedback', 'app', 'notes',
    UNIX_TIMESTAMP(),
   'Modulo de registro de sugerencias del sistema');
\end{SQL}
\caption{Inserción del paquete en el registro}
\label{code2}
\end{figure}

\begin{figure}
\centering
\begin{SQL}
INSERT INTO `privilege` (
    `description`, `package`, `label`)
VALUES
    ('Ver sugerencias',                 'feedback', 'list'),
    ('Marcar sugerencias solucionadas', 'feedback', 'resolv'),
    ('Marcar sugerencias interesantes', 'feedback', 'mark'),
    ('Eliminar sugerencias inutiles',   'feedback', 'delete');
\end{SQL}
\caption{Inserciones en el registro de privilegios}
\label{code3}
\end{figure}

\begin{figure}
\centering
\begin{SQL}
INSERT INTO `route` (
    `label`,
    `type`, `parent`, `route`,
    `mapping`,
    `module`, `controller`, `action`)
VALUES
    ('Lista de sugerencias',
     'list', '', 'feedback_list',
     'feedback',
     'feedback', 'index',   'index'),
    ('Administrador de sugerencias',
     'list', '', 'feedback_manager',
     'feedback/manager',
     'feedback', 'manager', 'index'),
    ('Nueva sugerencia',
     'view', '', 'feedback_new',
     'feedback/new',
     'feedback', 'manager', 'new'),
    ('Sugerencia: \$entry',
     'view', '', 'feedback_entry_view',
     'feedback/:entry',
     'feedback', 'entry', 'view'),
    ('Editar: \$entry',
     'view', '', 'feedback_entry_edit',
     'feedback/:entry/edit',
     'feedback', 'entry', 'edit'),
    ('', 'action', '', 'feedback_entry_resolv',
     'feedback/:entry/resolv',
     'feedback', 'entry', 'resolv'),
    ('', 'action', '', 'feedback_entry_unresolv',
     'feedback/:entry/unresolv',
     'feedback', 'entry', 'unresolv'),
    ('', 'action', '', 'feedback_entry_mark',
     'feedback/:entry/mark',
     'feedback', 'entry', 'mark'),
    ('', 'action', '', 'feedback_entry_unmark',
     'feedback/:entry/unmark',
     'feedback', 'entry', 'unmark'),
    ('', 'action', '', 'feedback_entry_delete',
     'feedback/:entry/delete',
     'feedback', 'entry', 'delete'),
    ('', 'action', '', 'feedback_entry_drop',
     'feedback/:entry/drop',
     'feedback', 'entry', 'drop');
\end{SQL}
\caption{Inserciones en el registro de rutas}
\label{code4}
\end{figure}

\begin{figure}
\centering
\begin{SQL}
INSERT INTO `route_privilege`
(`route`, `package`, `privilege`)
VALUES
('feedback_list',    'feedback', 'list'),
('feedback_manager', 'feedback', 'resolv'),
('feedback_manager', 'feedback', 'mark'),
('feedback_manager', 'feedback', 'delete');
\end{SQL}
\caption{Inserciones en el registro de rutas-privilegio}
\label{code5}
\end{figure}

\begin{itemize}
\item Creación de un archivo SQL con la definición de las tablas necesarias para
el nuevo paquete, estas deben estar prefijadas con el nombre del paquete
(figura \ref{code1}).
\item Inserción un nuevo registro de paquete (figura \ref{code2}).
\item Inserción de los registros de privilegios para el paquete nuevo
(figura \ref{code3}).
\item Inserción de los registros de ruta para el paquete nuevo
(figura \ref{code4}).
\item Registro de los permisos necesarios por ruta creada
(figura \ref{code5}).
\end{itemize}

Esta diseño esta basado en la forma de control que puede verse en cualquier
sistema administrador de contenido básico (CMS).

\subsection{\emph{templates}: Manejador de plantillas}
Las principales funciones de este paquete son:

\begin{itemize}
\item Gestión de la presentación del sistema.
\item Gestión de la presentación personalizada del usuario final.
\item Administración de las utilidades adicionales que presentan las paginas
del sistema.
\end{itemize}

\section{Construcción de espacios virtuales}
Para la correcta navegación sobre el sistema se han establecido, varios tipos de
espacios agrupadores de recursos.

\subsection{\emph{spaces}: La abstracción de todos los espacios}
Para facilitar el manejo de múltiples tipos de espacios, se ha optado por
abstraer muchas de las funcionalidades comunes (ya sean de presentación, o de
funcionalidad), es así como se ha creado este paquete que acumula la mayor parte
de la funcionalidad para los demás espacios.

Las principales funciones de este paquete son:

\begin{itemize}
\item Concentrar las funciones sobre los recursos del sistema.
\item Generación de una capa de abstracción para la creación de espacios con
condiciones de uso diferente.
\item Administración de los espacios (creación, modificación, presentación,
eliminación).
\item Control de permisos sobre los espacios.
\end{itemize}

\subsection{\emph{gestions}, \emph{careers}, \emph{areas}: Los espacios
formales}
Al comienzo del diseño se observo que una gran parte de los espacios vitales que
eran acordes al modelo al uso en la Universidad, eran dependientes de la
gestión, es decir, que una materia/grupo acumula recursos únicamente validos por
un determinado periodo de tiempo, es así como se crearon estos espacios.

\begin{description}
\item [Gestión] Es el espacio que determina el inicio y final de la validez de
un recurso temporal, esta muy asociado a la idea que se tiene de gestión
académica en la Universidad.
\item [Área] Un área es un tipo de espacio temporal, se creo para agrupar otros
tipos de espacio de forma transversal, y de esta forma flexibilizar las
jerarquías que poseía el modelo tradicional; un ejemplo de esto se ve en las
diferentes relaciones que existen entre diferentes materias o grupos en el
modelo.
\item [Carrera] Se creo este espacio para agrupar los espacios que tienen como
característica principal estar dentro de una malla curricular especifica.
\end{description}

Las principales funciones de estos paquetes son:

\begin{itemize}
\item Conformación de los espacios organizados acorde a la estructura real
observada en el contexto del sistema.
\item Agrupación de otros espacios que comparten algún grado de relacionamiento
entre si.
\end{itemize}

\subsection{\emph{subjects}, \emph{groups}, \emph{teams}: Los espacios
jerárquicos}
Estos espacios representan el eje central sobre el que se ha construido el
sistema, también están modelados acorde a la estructura observada en el contexto
del sistema, estas poseen una estructura de tipo jerárquico (es decir, cada una
esta contenida dentro de otra).

\begin{description}
\item [Materia] Representa el espacio de materia y a su vez puede poseer varios
grupos en su propio contexto.
\item [Grupo] Representa el espacio de trabajo de un grupo especifico de una
materia, esta a su vez puede contener varios equipos, según la organización y
métodos que utilice el instructor del grupo.
\item [Equipo] Representa el mínimo espacio organizacional en un grupo, se
construyo para facilitar la organización de estudiantes dentro de un grupo.
\end{description}

Las principales funciones de estos paquetes son:

\begin{itemize}
\item Creación de la estructura jerárquica que es también un reflejo de la
estructura real observada en el contexto del sistema.
\item Asignación de usuarios (con algún rol especifico), a los sub-espacios
establecidos.
\end{itemize}

\subsection{\emph{communities}: El espacio informal}
Como una forma de crear transversales al modelo jerárquico presentado
anteriormente, se construyo el espacio de comunidad, que agrupa usuarios a
partir de criterios no relacionados a una malla curricular, es decir, como una
forma alternativa de crear relaciones entre usuarios, a partir de un interés
particular. Este es un espacio atemporal, es decir, esta no depende de la
gestión.

Las principales funciones de estos paquetes son:

\begin{itemize}
\item Creación de un espacio común, transversal a las estructuras del contexto
del sistema.
\item Manejo de permisos para el espacio (una comunidad puede ser abierta a
cualquier usuario; o cerrada, de forma que un usuario pueda ser aceptado o
rechazado en la comunidad).
\end{itemize}

\section{B-learning}
Para facilitar el manejo por parte del docente asignado a un grupo, se crearon
dos paquetes que administren las formas de calificación, y la presentación de
calificaciones.

\subsection{\emph{evaluations}: Los sistemas de evaluación}
En un comienzo se notó que si bien, los docentes siempre presentan las
calificaciones con un formato único (Primer parcial, Segundo parcial, Promedio
de los parciales, Examen final y Segunda instancia), estas solo son la parte
final de un proceso aun mas complejo, por lo cual se creo un paquete que pueda
manejar criterios de evaluación mas elaborados, según el docente que imparta la
materia.

Las principales funciones de este paquete son:

\begin{itemize}
\item Creación y edición de criterios de evaluación.
\item Publicación y aplicación de un criterio de evaluación sobre un grupo
determinado por un docente.
\item Aplicación de calificaciones según un criterio de evaluación especifico.
\end{itemize}

\subsection{\emph{califications}: Las calificaciones}
A partir del paquete que administra y proporciona diferentes tipos de evaluación
para un grupo, se han creado las funcionalidades necesarias para establecer las
calificaciones de un grupo.

Las principales funciones de este paquete son:

\begin{itemize}
\item Edición de las calificaciones de los alumnos de un determinado grupo,
según un criterio de evaluación establecido.
\item Importación automática a partir de una archivo en formato CSV, de las
calificaciones de una grupo.
\item Exportación a formato CSV de las calificaciones de una grupo, según un
criterio de evaluación establecido.
\end{itemize}

\section{Recursos}
Ya creados los espacios virtuales, se han construido los paquetes necesarios
para la publicación e intercambio de recursos en el sistema.

\subsection{\emph{resources}: La abstracción de todos los recursos}
Al igual que en los espacios virtuales, los recursos también comparten un gran
conjunto de funcionalidad común a todos ellos (ya sean de presentación, como de
funcionalidad), es así como se han abstraído estas funcionalidades en este
paquete.

Las principales funciones de este paquete son:

\begin{itemize}
\item Concentrar las funciones de valoración sobre los recursos.
\item Generación de una capa de abstracción para la creación de recursos con
condiciones de uso diferente.
\item Administración de los recursos (creación, modificación, presentación,
valoración, eliminación).
\item Control de permisos sobre los recursos.
\end{itemize}

\subsection{\emph{notes}: Los recursos mas básicos}
El tipo de recurso mas básico es la nota, que representa esencialmente texto.
Este recurso además se diseño para ser utilizado por otros tipo de recursos mas
enriquecidos y especializados.

\subsection{\emph{links}: El administrador de marcadores}
Este recurso fue construido para compartir recursos, a partir de enlaces sobre
la red de Internet, se pensó para trabajos posterior, hacer de este tipo de
recurso una funcionalidad de reconocimiento de recurso y una reenderización mas
apropiada.

\subsection{\emph{files}: El administrador de archivos}
Este recurso fue construido para compartir archivos en un espacio virtual,
también se pensó para trabajo posterior, una especialización sobre la
reenderización y reconocimiento de los formatos.

\subsection{\emph{photos}, \emph{videos}: Los recursos especiales}
Estos recursos están basados en el recurso tipo archivo, pero poseen
características que les permiten ser reenderizados apropiadamente.

\subsection{\emph{events}: El recurso espacio-temporales}
Este recurso representa un evento sobre un espacio virtual, a partir de este
recurso, se considero para trabajo posterior, el que sea tanto un recurso, como
un espacio virtual independiente, además de facilitar el manejo de
notificaciones.

\subsection{\emph{feedback}: El manejador de sugerencias}
Este fue el ultimo paquete que se construyo, y fue creado exclusivamente para
que los usuarios del sistema puedan hacer sus sugerencias sobre nueva
funcionalidad que podrían implementarse sobre el sistema.

\section{Los usuarios y su red de contactos}
Una vez establecidos los conceptos de espacio virtual, y recurso, se han
construido los paquetes necesarios para la representación y manejo de usuarios
del sistema.

\subsection{\emph{users}: El espacio personal}
Representa al usuario final del sistema (docentes, estudiantes, auxiliares,
etc.), estos además están provistos de las funcionalidades para espacios
virtuales.

Las principales funciones de este paquete son:

\begin{itemize}
\item Creación de un espacio virtual propio del usuario.
\item Manejo y visualización de sus valoraciones en el sistema.
\item Administración de los datos personales del usuario.
\item Manejo de los recursos creados por el usuario sobre los diferentes
espacios virtuales a los que tiene acceso.
\end{itemize}

\subsection{\emph{roles}: El controlador de las privilegios}
Un rol representa el conjunto de privilegios que posee un usuario sobre los
paquetes del sistema.
Inicialmente de considero categorizar a los usuarios de forma estática, según
los criterios del modelo de la Universidad, estos son:

\begin{itemize}
\item Visitante
\item Invitado
\item Estudiante
\item Auxiliar
\item Docente
\item Moderador
\item Desarrollador
\item Administrador
\end{itemize}

Pero para facilitar la adaptación a otros tipos de organización, se vio
conveniente crear roles dinámicamente, así como poder administrar el conjunto
de privilegios que estos posean.

Las principales funciones de este paquete son:

\begin{itemize}
\item Administración de roles (creación, visualización, edición, eliminación).
\item Asignación de privilegios dinámicamente.
\item Control de acceso a los espacios, recursos, y rutas según un rol
establecido.
\end{itemize}

\subsection{\emph{contacts}: Las redes sociales}
Para proveer las características de una red social, se construyo un paquete que
pudiese manejar las relaciones entre diferentes tipos de usuarios.

Las principales funciones de este paquete son:

\begin{itemize}
\item Agregación de un contacto.
\item Eliminación de un contacto.
\item Visualización de los recursos compartidos por los contactos.
\end{itemize}

\subsection{\emph{invitations}: Estrategia de propagación del sistema}
Se considero que la creación de cuentas en el sistema nunca sea abierta, con el
objetivo de hacer que los usuarios en el sistema, tengan al menos un contacto
con otro usuario, por tal motivo, se crearon las invitaciones.

Las principales funciones de este paquete son:

\begin{itemize}
\item Envío de una invitación de un usuario a un correo electrónico.
\item Gestión de la caducidad de una invitación.
\item Manejo del contacto con el usuario que atiende a la invitación.
\end{itemize}

\section{Fomento a la participación}
Como medidas para fomentar la participación de parte de los usuarios, se han
implementado algunas funcionalidad propias de la web 2.0.

\subsection{\emph{comments}: Los comentarios}
\subsection{\emph{ratings}: La calidad del recurso}
\subsection{\emph{tags}: Las nuevas interpretaciones}
\subsection{\emph{valorations}: Los sistemas de reputación}

\section{Sistemas de control}
\subsection{\emph{stats}: Los indicadores medibles}

