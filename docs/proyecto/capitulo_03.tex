\chapter{Metodología de desarrollo}

En este capítulo, se desarrollan los aspectos necesarios para la definición del 
proceso de desarrollo, primeramente se hará referencia a las cuestiones 
relacionadas con la metodología de desarrollo, posteriormente se verán los
documentos propios que se han definido para las diferentes etapas; luego se
tratarán las etapas de la planificación ha tomarse en cuenta.

\section{Modelo iterativo}

Considerando el contexto de desarrollo (el contexto esta descrito mas adelante
en este capitulo), se ha visto conveniente seguir un modelo de desarrollo que
sea iterativo e incremental\footnote{Para una definición exacta puede
consultarse: https://es.wikipedia.org/wiki/Desarrollo\_iterativo\_y\_creciente}.

La idea central es que, en cada una de esas iteraciones, se construye una parte
pequeña del sistema. Para esa parte del sistema, se realiza todo el proceso:
análisis, diseño, programación y pruebas. Se acaba la iteración con un
prototipo funcional, que incluya todas las partes del sistema construidas hasta
el momento. Los aspectos del sistema con más riesgo (por ejemplo, la
arquitectura) se definen y construyen en las primeras iteraciones.

Las ventajas de este tipo de modelo son las siguientes:

\begin{description}
\item [Flexibilidad] Los requerimientos no quedan totalmente fijados hasta el
final del proyecto de desarrollo. Por ello, se pueden realizar cambios de forma
flexible. Por una parte, el conocimiento que se adquiere en una iteración sirve
para plantear de forma más realista los requerimientos de la siguiente. Por
otra parte, este conocimiento nos puede hacer reformar partes del sistema
construidas en iteraciones anteriores. En una palabra, todos los documentos del
sistema (requerimientos, diseño y código) no son rígidos sino que pueden
cambiarse durante todo el proceso de desarrollo. (Típicamente suelen ser
modificados en mayor medida en las primeras iteraciones y en menor medida en
las últimas).
\item [Mitigación de riesgos] Como las pruebas se hacen desde el principio del
proyecto, puede determinarse la viabilidad o eficiencia de las decisiones de
diseño. Además, los elementos con más riesgo se tratan en las primeras
iteraciones, con lo cual se puede implementar una mitigación de riesgos más 
temprana y exitosa.
\item [Retroalimentación] Como hay prototipos desde el mismo comienzo del
proyecto, estos pueden examinarse, y revalorizarse. También existe una rápida
retroalimentación de lo que funciona y lo que no, ya que las pruebas se 
realizan desde el comienzo mismo del proyecto y no se debe esperar al final
para hacer las modificaciones necesarias.
\end{description}

\section{Requerimientos funcionales}

A partir de los objetivos del proyecto, se pasa a establecer el conjunto de
funciones que el sistema debe tener, para esto se ha optado por crear un
conjunto de categorias, de forma que agrupen las tareas que son comunes a un
objetivo especifico.

\subsection{Espacios virtuales}

Para la correcta navegación sobre el sistema se han establecido, varios tipos de
espacios agrupadores de recursos.

Estos recursos ademas pueden clasificarse según su temporalidad, es decir si
poseen alguna forma de caducidad, o si no poseen tal cualidad. Estos son:

\begin{description}
\item [Gestiones] Una gestión representa la division básica de periodos
academicos, estos trazan un marco de referencia temporal (es decir, su valor de
caducidad) para muchos de los espacios restantes.
\item [Materias] Una materia es el espacio que concentra todos los recursos de
una materia, (esta a su vez concentra a otros sub-espacios). Este espacio es a
su vez un sub-espacio de algún espacio de gestión.
\item [Grupos] Los grupos son espacios de separación de una materia, esta está
basada en el sistema utilizado en el dominio de implementacion del sistema
(UMSS).
\item [Equipos] Los equipos son espacios opcionales de creación, que pueden
utilizarse para dividir aún mas un grupo de estudio, segun el metodo que el
docente pretenda utilizar.
\item [Carreras] Las carreras representan una concentración de materias que a su
vez estan agrupadas segun gestiones especificas.
\item [Areas] Un area es otra forma de agrupación de materias, que carecen de
una cualidad temporal, es decir, que no poseen caducidad.
\item [Comunidades] Una comunidad es una forma de espacio virtual independiente
de toda gestión (lo que implica que no tiene caducidad), y la intención es poder
agrupar a los usuarios segun un interes en particular.
\end{description}

Los elementos pueden resurmirse como se muestra en la figura \ref{espacios}.
Como se verá a la larga la clasificación de los espacios según su temporalidad,
es imprescindible para corregir los incovenientes creados por la formalidad que
poseen algunos espacios, brindando espacios que poseen un caracter mas libre.

\begin{figure}
\centering
%LaTeX with PSTricks extensions
%%Creator: inkscape 0.48.4
%%Please note this file requires PSTricks extensions
\psset{xunit=.5pt,yunit=.5pt,runit=.5pt}
\begin{pspicture}(531.49603271,425.19683838)
{
\newrgbcolor{curcolor}{0 0 0}
\pscustom[linestyle=none,fillstyle=solid,fillcolor=curcolor]
{
\newpath
\moveto(50.2411504,357.87891102)
\lineto(52.0411504,357.87891102)
\lineto(52.0411504,369.15891102)
\lineto(42.6511504,369.15891102)
\lineto(42.6511504,366.75891102)
\lineto(49.4911504,366.75891102)
\curveto(49.67115022,362.85891492)(47.03114599,359.76891102)(42.6211504,359.76891102)
\curveto(37.85115517,359.76891102)(35.2711504,363.84891549)(35.2711504,368.31891102)
\curveto(35.2711504,372.90890643)(37.43115559,377.40891102)(42.6211504,377.40891102)
\curveto(45.80114722,377.40891102)(48.41115097,375.93890778)(48.9811504,372.69891102)
\lineto(51.8311504,372.69891102)
\curveto(51.02115121,377.70890601)(47.30114572,379.80891102)(42.6211504,379.80891102)
\curveto(35.84115718,379.80891102)(32.4211504,374.40890478)(32.4211504,368.16891102)
\curveto(32.4211504,362.5889166)(36.23115679,357.36891102)(42.6211504,357.36891102)
\curveto(45.14114788,357.36891102)(47.84115205,358.29891327)(49.4911504,360.54891102)
\lineto(50.2411504,357.87891102)
}
}
{
\newrgbcolor{curcolor}{0 0 0}
\pscustom[linestyle=none,fillstyle=solid,fillcolor=curcolor]
{
\newpath
\moveto(66.55411915,362.79891102)
\curveto(66.1041196,360.78891303)(64.63411705,359.76891102)(62.53411915,359.76891102)
\curveto(59.14412254,359.76891102)(57.61411924,362.16891372)(57.70411915,364.86891102)
\lineto(69.31411915,364.86891102)
\curveto(69.464119,368.61890727)(67.78411366,373.74891102)(62.29411915,373.74891102)
\curveto(58.06412338,373.74891102)(55.00411915,370.32890637)(55.00411915,365.67891102)
\curveto(55.154119,360.93891576)(57.4941241,357.51891102)(62.44411915,357.51891102)
\curveto(65.92411567,357.51891102)(68.38411984,359.37891444)(69.07411915,362.79891102)
\lineto(66.55411915,362.79891102)
\moveto(57.70411915,367.11891102)
\curveto(57.88411897,369.48890865)(59.47412182,371.49891102)(62.14411915,371.49891102)
\curveto(64.66411663,371.49891102)(66.49411927,369.54890859)(66.61411915,367.11891102)
\lineto(57.70411915,367.11891102)
}
}
{
\newrgbcolor{curcolor}{0 0 0}
\pscustom[linestyle=none,fillstyle=solid,fillcolor=curcolor]
{
\newpath
\moveto(70.9674004,362.76891102)
\curveto(71.11740025,358.92891486)(74.05740388,357.51891102)(77.5374004,357.51891102)
\curveto(80.68739725,357.51891102)(84.1374004,358.71891471)(84.1374004,362.40891102)
\curveto(84.1374004,365.40890802)(81.61739785,366.24891159)(79.0674004,366.81891102)
\curveto(76.69740277,367.38891045)(73.9974004,367.68891285)(73.9974004,369.51891102)
\curveto(73.9974004,371.07890946)(75.76740193,371.49891102)(77.2974004,371.49891102)
\curveto(78.97739872,371.49891102)(80.71740058,370.86890904)(80.8974004,368.88891102)
\lineto(83.4474004,368.88891102)
\curveto(83.23740061,372.66890724)(80.50739698,373.74891102)(77.0874004,373.74891102)
\curveto(74.3874031,373.74891102)(71.2974004,372.4589079)(71.2974004,369.33891102)
\curveto(71.2974004,366.36891399)(73.84740292,365.52891045)(76.3674004,364.95891102)
\curveto(78.91739785,364.38891159)(81.4374004,364.05890904)(81.4374004,362.07891102)
\curveto(81.4374004,360.12891297)(79.27739881,359.76891102)(77.6874004,359.76891102)
\curveto(75.5874025,359.76891102)(73.60740031,360.4889133)(73.5174004,362.76891102)
\lineto(70.9674004,362.76891102)
}
}
{
\newrgbcolor{curcolor}{0 0 0}
\pscustom[linestyle=none,fillstyle=solid,fillcolor=curcolor]
{
\newpath
\moveto(90.4974004,378.03891102)
\lineto(87.9474004,378.03891102)
\lineto(87.9474004,373.38891102)
\lineto(85.3074004,373.38891102)
\lineto(85.3074004,371.13891102)
\lineto(87.9474004,371.13891102)
\lineto(87.9474004,361.26891102)
\curveto(87.9474004,358.41891387)(88.99740304,357.87891102)(91.6374004,357.87891102)
\lineto(93.5874004,357.87891102)
\lineto(93.5874004,360.12891102)
\lineto(92.4174004,360.12891102)
\curveto(90.82740199,360.12891102)(90.4974004,360.33891219)(90.4974004,361.50891102)
\lineto(90.4974004,371.13891102)
\lineto(93.5874004,371.13891102)
\lineto(93.5874004,373.38891102)
\lineto(90.4974004,373.38891102)
\lineto(90.4974004,378.03891102)
}
}
{
\newrgbcolor{curcolor}{0 0 0}
\pscustom[linestyle=none,fillstyle=solid,fillcolor=curcolor]
{
\newpath
\moveto(96.54099415,357.87891102)
\lineto(99.09099415,357.87891102)
\lineto(99.09099415,373.38891102)
\lineto(96.54099415,373.38891102)
\lineto(96.54099415,357.87891102)
\moveto(99.09099415,379.29891102)
\lineto(96.54099415,379.29891102)
\lineto(96.54099415,376.17891102)
\lineto(99.09099415,376.17891102)
\lineto(99.09099415,379.29891102)
}
}
{
\newrgbcolor{curcolor}{0 0 0}
\pscustom[linestyle=none,fillstyle=solid,fillcolor=curcolor]
{
\newpath
\moveto(102.23068165,365.61891102)
\curveto(102.23068165,361.08891555)(104.84068657,357.51891102)(109.76068165,357.51891102)
\curveto(114.68067673,357.51891102)(117.29068165,361.08891555)(117.29068165,365.61891102)
\curveto(117.29068165,370.17890646)(114.68067673,373.74891102)(109.76068165,373.74891102)
\curveto(104.84068657,373.74891102)(102.23068165,370.17890646)(102.23068165,365.61891102)
\moveto(104.93068165,365.61891102)
\curveto(104.93068165,369.39890724)(107.09068432,371.49891102)(109.76068165,371.49891102)
\curveto(112.43067898,371.49891102)(114.59068165,369.39890724)(114.59068165,365.61891102)
\curveto(114.59068165,361.86891477)(112.43067898,359.76891102)(109.76068165,359.76891102)
\curveto(107.09068432,359.76891102)(104.93068165,361.86891477)(104.93068165,365.61891102)
\moveto(107.93068165,375.51891102)
\lineto(109.85068165,375.51891102)
\lineto(113.78068165,379.80891102)
\lineto(110.51068165,379.80891102)
\lineto(107.93068165,375.51891102)
}
}
{
\newrgbcolor{curcolor}{0 0 0}
\pscustom[linestyle=none,fillstyle=solid,fillcolor=curcolor]
{
\newpath
\moveto(120.29724415,357.87891102)
\lineto(122.84724415,357.87891102)
\lineto(122.84724415,366.63891102)
\curveto(122.84724415,369.42890823)(124.34724724,371.49891102)(127.43724415,371.49891102)
\curveto(129.3872422,371.49891102)(130.58724415,370.26890913)(130.58724415,368.37891102)
\lineto(130.58724415,357.87891102)
\lineto(133.13724415,357.87891102)
\lineto(133.13724415,368.07891102)
\curveto(133.13724415,371.40890769)(131.87724007,373.74891102)(127.79724415,373.74891102)
\curveto(125.57724637,373.74891102)(123.83724307,372.8489091)(122.75724415,370.92891102)
\lineto(122.69724415,370.92891102)
\lineto(122.69724415,373.38891102)
\lineto(120.29724415,373.38891102)
\lineto(120.29724415,357.87891102)
}
}
{
\newrgbcolor{curcolor}{0 0 0}
\pscustom[linestyle=none,fillstyle=solid,fillcolor=curcolor]
{
\newpath
\moveto(82.93620258,305.22198963)
\lineto(85.63620258,305.22198963)
\lineto(85.63620258,323.04198963)
\lineto(85.69620258,323.04198963)
\lineto(92.38620258,305.22198963)
\lineto(94.81620258,305.22198963)
\lineto(101.50620258,323.04198963)
\lineto(101.56620258,323.04198963)
\lineto(101.56620258,305.22198963)
\lineto(104.26620258,305.22198963)
\lineto(104.26620258,326.64198963)
\lineto(100.36620258,326.64198963)
\lineto(93.58620258,308.64198963)
\lineto(86.83620258,326.64198963)
\lineto(82.93620258,326.64198963)
\lineto(82.93620258,305.22198963)
}
}
{
\newrgbcolor{curcolor}{0 0 0}
\pscustom[linestyle=none,fillstyle=solid,fillcolor=curcolor]
{
\newpath
\moveto(118.18901508,310.62198963)
\curveto(118.18901508,309.21199104)(116.80901172,307.11198963)(113.44901508,307.11198963)
\curveto(111.88901664,307.11198963)(110.44901508,307.71199131)(110.44901508,309.39198963)
\curveto(110.44901508,311.28198774)(111.88901676,311.88198993)(113.56901508,312.18198963)
\curveto(115.27901337,312.48198933)(117.19901607,312.51199035)(118.18901508,313.23198963)
\lineto(118.18901508,310.62198963)
\moveto(122.32901508,307.26198963)
\curveto(121.99901541,307.14198975)(121.75901487,307.11198963)(121.54901508,307.11198963)
\curveto(120.73901589,307.11198963)(120.73901508,307.65199083)(120.73901508,308.85198963)
\lineto(120.73901508,316.83198963)
\curveto(120.73901508,320.461986)(117.70901229,321.09198963)(114.91901508,321.09198963)
\curveto(111.46901853,321.09198963)(108.49901493,319.74198579)(108.34901508,315.90198963)
\lineto(110.89901508,315.90198963)
\curveto(111.01901496,318.18198735)(112.60901724,318.84198963)(114.76901508,318.84198963)
\curveto(116.38901346,318.84198963)(118.21901508,318.48198741)(118.21901508,316.26198963)
\curveto(118.21901508,314.34199155)(115.81901226,314.52198909)(112.99901508,313.98198963)
\curveto(110.35901772,313.47199014)(107.74901508,312.72198612)(107.74901508,309.21198963)
\curveto(107.74901508,306.12199272)(110.0590179,304.86198963)(112.87901508,304.86198963)
\curveto(115.03901292,304.86198963)(116.92901649,305.61199128)(118.33901508,307.26198963)
\curveto(118.33901508,305.58199131)(119.1790164,304.86198963)(120.49901508,304.86198963)
\curveto(121.30901427,304.86198963)(121.87901553,305.0119899)(122.32901508,305.28198963)
\lineto(122.32901508,307.26198963)
}
}
{
\newrgbcolor{curcolor}{0 0 0}
\pscustom[linestyle=none,fillstyle=solid,fillcolor=curcolor]
{
\newpath
\moveto(128.24229633,325.38198963)
\lineto(125.69229633,325.38198963)
\lineto(125.69229633,320.73198963)
\lineto(123.05229633,320.73198963)
\lineto(123.05229633,318.48198963)
\lineto(125.69229633,318.48198963)
\lineto(125.69229633,308.61198963)
\curveto(125.69229633,305.76199248)(126.74229897,305.22198963)(129.38229633,305.22198963)
\lineto(131.33229633,305.22198963)
\lineto(131.33229633,307.47198963)
\lineto(130.16229633,307.47198963)
\curveto(128.57229792,307.47198963)(128.24229633,307.6819908)(128.24229633,308.85198963)
\lineto(128.24229633,318.48198963)
\lineto(131.33229633,318.48198963)
\lineto(131.33229633,320.73198963)
\lineto(128.24229633,320.73198963)
\lineto(128.24229633,325.38198963)
}
}
{
\newrgbcolor{curcolor}{0 0 0}
\pscustom[linestyle=none,fillstyle=solid,fillcolor=curcolor]
{
\newpath
\moveto(144.84589008,310.14198963)
\curveto(144.39589053,308.13199164)(142.92588798,307.11198963)(140.82589008,307.11198963)
\curveto(137.43589347,307.11198963)(135.90589017,309.51199233)(135.99589008,312.21198963)
\lineto(147.60589008,312.21198963)
\curveto(147.75588993,315.96198588)(146.07588459,321.09198963)(140.58589008,321.09198963)
\curveto(136.35589431,321.09198963)(133.29589008,317.67198498)(133.29589008,313.02198963)
\curveto(133.44588993,308.28199437)(135.78589503,304.86198963)(140.73589008,304.86198963)
\curveto(144.2158866,304.86198963)(146.67589077,306.72199305)(147.36589008,310.14198963)
\lineto(144.84589008,310.14198963)
\moveto(135.99589008,314.46198963)
\curveto(136.1758899,316.83198726)(137.76589275,318.84198963)(140.43589008,318.84198963)
\curveto(142.95588756,318.84198963)(144.7858902,316.8919872)(144.90589008,314.46198963)
\lineto(135.99589008,314.46198963)
}
}
{
\newrgbcolor{curcolor}{0 0 0}
\pscustom[linestyle=none,fillstyle=solid,fillcolor=curcolor]
{
\newpath
\moveto(150.15917133,305.22198963)
\lineto(152.70917133,305.22198963)
\lineto(152.70917133,312.12198963)
\curveto(152.70917133,316.0519857)(154.20917544,318.39198963)(158.31917133,318.39198963)
\lineto(158.31917133,321.09198963)
\curveto(155.55917409,321.18198954)(153.8491701,319.95198714)(152.61917133,317.46198963)
\lineto(152.55917133,317.46198963)
\lineto(152.55917133,320.73198963)
\lineto(150.15917133,320.73198963)
\lineto(150.15917133,305.22198963)
}
}
{
\newrgbcolor{curcolor}{0 0 0}
\pscustom[linestyle=none,fillstyle=solid,fillcolor=curcolor]
{
\newpath
\moveto(160.41870258,305.22198963)
\lineto(162.96870258,305.22198963)
\lineto(162.96870258,320.73198963)
\lineto(160.41870258,320.73198963)
\lineto(160.41870258,305.22198963)
\moveto(162.96870258,326.64198963)
\lineto(160.41870258,326.64198963)
\lineto(160.41870258,323.52198963)
\lineto(162.96870258,323.52198963)
\lineto(162.96870258,326.64198963)
}
}
{
\newrgbcolor{curcolor}{0 0 0}
\pscustom[linestyle=none,fillstyle=solid,fillcolor=curcolor]
{
\newpath
\moveto(176.54839008,310.62198963)
\curveto(176.54839008,309.21199104)(175.16838672,307.11198963)(171.80839008,307.11198963)
\curveto(170.24839164,307.11198963)(168.80839008,307.71199131)(168.80839008,309.39198963)
\curveto(168.80839008,311.28198774)(170.24839176,311.88198993)(171.92839008,312.18198963)
\curveto(173.63838837,312.48198933)(175.55839107,312.51199035)(176.54839008,313.23198963)
\lineto(176.54839008,310.62198963)
\moveto(180.68839008,307.26198963)
\curveto(180.35839041,307.14198975)(180.11838987,307.11198963)(179.90839008,307.11198963)
\curveto(179.09839089,307.11198963)(179.09839008,307.65199083)(179.09839008,308.85198963)
\lineto(179.09839008,316.83198963)
\curveto(179.09839008,320.461986)(176.06838729,321.09198963)(173.27839008,321.09198963)
\curveto(169.82839353,321.09198963)(166.85838993,319.74198579)(166.70839008,315.90198963)
\lineto(169.25839008,315.90198963)
\curveto(169.37838996,318.18198735)(170.96839224,318.84198963)(173.12839008,318.84198963)
\curveto(174.74838846,318.84198963)(176.57839008,318.48198741)(176.57839008,316.26198963)
\curveto(176.57839008,314.34199155)(174.17838726,314.52198909)(171.35839008,313.98198963)
\curveto(168.71839272,313.47199014)(166.10839008,312.72198612)(166.10839008,309.21198963)
\curveto(166.10839008,306.12199272)(168.4183929,304.86198963)(171.23839008,304.86198963)
\curveto(173.39838792,304.86198963)(175.28839149,305.61199128)(176.69839008,307.26198963)
\curveto(176.69839008,305.58199131)(177.5383914,304.86198963)(178.85839008,304.86198963)
\curveto(179.66838927,304.86198963)(180.23839053,305.0119899)(180.68839008,305.28198963)
\lineto(180.68839008,307.26198963)
}
}
{
\newrgbcolor{curcolor}{0 0 0}
\pscustom[linestyle=none,fillstyle=solid,fillcolor=curcolor]
{
\newpath
\moveto(148.30735065,252.20502186)
\lineto(150.10735065,252.20502186)
\lineto(150.10735065,263.48502186)
\lineto(140.71735065,263.48502186)
\lineto(140.71735065,261.08502186)
\lineto(147.55735065,261.08502186)
\curveto(147.73735047,257.18502576)(145.09734624,254.09502186)(140.68735065,254.09502186)
\curveto(135.91735542,254.09502186)(133.33735065,258.17502633)(133.33735065,262.64502186)
\curveto(133.33735065,267.23501727)(135.49735584,271.73502186)(140.68735065,271.73502186)
\curveto(143.86734747,271.73502186)(146.47735122,270.26501862)(147.04735065,267.02502186)
\lineto(149.89735065,267.02502186)
\curveto(149.08735146,272.03501685)(145.36734597,274.13502186)(140.68735065,274.13502186)
\curveto(133.90735743,274.13502186)(130.48735065,268.73501562)(130.48735065,262.49502186)
\curveto(130.48735065,256.91502744)(134.29735704,251.69502186)(140.68735065,251.69502186)
\curveto(143.20734813,251.69502186)(145.9073523,252.62502411)(147.55735065,254.87502186)
\lineto(148.30735065,252.20502186)
}
}
{
\newrgbcolor{curcolor}{0 0 0}
\pscustom[linestyle=none,fillstyle=solid,fillcolor=curcolor]
{
\newpath
\moveto(153.8203194,252.20502186)
\lineto(156.3703194,252.20502186)
\lineto(156.3703194,259.10502186)
\curveto(156.3703194,263.03501793)(157.87032351,265.37502186)(161.9803194,265.37502186)
\lineto(161.9803194,268.07502186)
\curveto(159.22032216,268.16502177)(157.51031817,266.93501937)(156.2803194,264.44502186)
\lineto(156.2203194,264.44502186)
\lineto(156.2203194,267.71502186)
\lineto(153.8203194,267.71502186)
\lineto(153.8203194,252.20502186)
}
}
{
\newrgbcolor{curcolor}{0 0 0}
\pscustom[linestyle=none,fillstyle=solid,fillcolor=curcolor]
{
\newpath
\moveto(176.76985065,267.71502186)
\lineto(174.21985065,267.71502186)
\lineto(174.21985065,258.95502186)
\curveto(174.21985065,256.16502465)(172.71984756,254.09502186)(169.62985065,254.09502186)
\curveto(167.6798526,254.09502186)(166.47985065,255.32502375)(166.47985065,257.21502186)
\lineto(166.47985065,267.71502186)
\lineto(163.92985065,267.71502186)
\lineto(163.92985065,257.51502186)
\curveto(163.92985065,254.18502519)(165.18985473,251.84502186)(169.26985065,251.84502186)
\curveto(171.48984843,251.84502186)(173.22985173,252.74502378)(174.30985065,254.66502186)
\lineto(174.36985065,254.66502186)
\lineto(174.36985065,252.20502186)
\lineto(176.76985065,252.20502186)
\lineto(176.76985065,267.71502186)
}
}
{
\newrgbcolor{curcolor}{0 0 0}
\pscustom[linestyle=none,fillstyle=solid,fillcolor=curcolor]
{
\newpath
\moveto(192.7190694,260.09502186)
\curveto(192.7190694,257.06502489)(191.54906592,254.09502186)(188.0690694,254.09502186)
\curveto(184.55907291,254.09502186)(183.1790694,256.91502492)(183.1790694,259.97502186)
\curveto(183.1790694,262.88501895)(184.49907282,265.82502186)(187.9190694,265.82502186)
\curveto(191.2190661,265.82502186)(192.7190694,263.00501895)(192.7190694,260.09502186)
\moveto(180.7190694,246.26502186)
\lineto(183.2690694,246.26502186)
\lineto(183.2690694,254.27502186)
\lineto(183.3290694,254.27502186)
\curveto(184.46906826,252.44502369)(186.74907099,251.84502186)(188.3390694,251.84502186)
\curveto(193.07906466,251.84502186)(195.4190694,255.53502624)(195.4190694,259.91502186)
\curveto(195.4190694,264.29501748)(193.04906463,268.07502186)(188.2790694,268.07502186)
\curveto(186.14907153,268.07502186)(184.16906856,267.32502015)(183.3290694,265.61502186)
\lineto(183.2690694,265.61502186)
\lineto(183.2690694,267.71502186)
\lineto(180.7190694,267.71502186)
\lineto(180.7190694,246.26502186)
}
}
{
\newrgbcolor{curcolor}{0 0 0}
\pscustom[linestyle=none,fillstyle=solid,fillcolor=curcolor]
{
\newpath
\moveto(197.6015694,259.94502186)
\curveto(197.6015694,255.41502639)(200.21157432,251.84502186)(205.1315694,251.84502186)
\curveto(210.05156448,251.84502186)(212.6615694,255.41502639)(212.6615694,259.94502186)
\curveto(212.6615694,264.5050173)(210.05156448,268.07502186)(205.1315694,268.07502186)
\curveto(200.21157432,268.07502186)(197.6015694,264.5050173)(197.6015694,259.94502186)
\moveto(200.3015694,259.94502186)
\curveto(200.3015694,263.72501808)(202.46157207,265.82502186)(205.1315694,265.82502186)
\curveto(207.80156673,265.82502186)(209.9615694,263.72501808)(209.9615694,259.94502186)
\curveto(209.9615694,256.19502561)(207.80156673,254.09502186)(205.1315694,254.09502186)
\curveto(202.46157207,254.09502186)(200.3015694,256.19502561)(200.3015694,259.94502186)
}
}
{
\newrgbcolor{curcolor}{0 0 0}
\pscustom[linestyle=none,fillstyle=solid,fillcolor=curcolor]
{
\newpath
\moveto(162.46053791,194.11804676)
\lineto(177.34053791,194.11804676)
\lineto(177.34053791,196.51804676)
\lineto(165.31053791,196.51804676)
\lineto(165.31053791,203.92804676)
\lineto(176.44053791,203.92804676)
\lineto(176.44053791,206.32804676)
\lineto(165.31053791,206.32804676)
\lineto(165.31053791,213.13804676)
\lineto(177.25053791,213.13804676)
\lineto(177.25053791,215.53804676)
\lineto(162.46053791,215.53804676)
\lineto(162.46053791,194.11804676)
}
}
{
\newrgbcolor{curcolor}{0 0 0}
\pscustom[linestyle=none,fillstyle=solid,fillcolor=curcolor]
{
\newpath
\moveto(194.24038166,209.62804676)
\lineto(191.69038166,209.62804676)
\lineto(191.69038166,207.55804676)
\lineto(191.63038166,207.55804676)
\curveto(190.4903828,209.38804493)(188.21038007,209.98804676)(186.62038166,209.98804676)
\curveto(181.8803864,209.98804676)(179.54038166,206.29804238)(179.54038166,201.91804676)
\curveto(179.54038166,197.53805114)(181.91038643,193.75804676)(186.68038166,193.75804676)
\curveto(188.81037953,193.75804676)(190.7903825,194.50804847)(191.63038166,196.21804676)
\lineto(191.69038166,196.21804676)
\lineto(191.69038166,188.17804676)
\lineto(194.24038166,188.17804676)
\lineto(194.24038166,209.62804676)
\moveto(182.24038166,201.73804676)
\curveto(182.24038166,204.76804373)(183.41038514,207.73804676)(186.89038166,207.73804676)
\curveto(190.40037815,207.73804676)(191.78038166,204.9180437)(191.78038166,201.85804676)
\curveto(191.78038166,198.94804967)(190.46037824,196.00804676)(187.04038166,196.00804676)
\curveto(183.74038496,196.00804676)(182.24038166,198.82804967)(182.24038166,201.73804676)
}
}
{
\newrgbcolor{curcolor}{0 0 0}
\pscustom[linestyle=none,fillstyle=solid,fillcolor=curcolor]
{
\newpath
\moveto(211.03288166,209.62804676)
\lineto(208.48288166,209.62804676)
\lineto(208.48288166,200.86804676)
\curveto(208.48288166,198.07804955)(206.98287857,196.00804676)(203.89288166,196.00804676)
\curveto(201.94288361,196.00804676)(200.74288166,197.23804865)(200.74288166,199.12804676)
\lineto(200.74288166,209.62804676)
\lineto(198.19288166,209.62804676)
\lineto(198.19288166,199.42804676)
\curveto(198.19288166,196.09805009)(199.45288574,193.75804676)(203.53288166,193.75804676)
\curveto(205.75287944,193.75804676)(207.49288274,194.65804868)(208.57288166,196.57804676)
\lineto(208.63288166,196.57804676)
\lineto(208.63288166,194.11804676)
\lineto(211.03288166,194.11804676)
\lineto(211.03288166,209.62804676)
}
}
{
\newrgbcolor{curcolor}{0 0 0}
\pscustom[linestyle=none,fillstyle=solid,fillcolor=curcolor]
{
\newpath
\moveto(215.04210041,194.11804676)
\lineto(217.59210041,194.11804676)
\lineto(217.59210041,209.62804676)
\lineto(215.04210041,209.62804676)
\lineto(215.04210041,194.11804676)
\moveto(217.59210041,215.53804676)
\lineto(215.04210041,215.53804676)
\lineto(215.04210041,212.41804676)
\lineto(217.59210041,212.41804676)
\lineto(217.59210041,215.53804676)
}
}
{
\newrgbcolor{curcolor}{0 0 0}
\pscustom[linestyle=none,fillstyle=solid,fillcolor=curcolor]
{
\newpath
\moveto(233.66178791,202.00804676)
\curveto(233.66178791,198.97804979)(232.49178443,196.00804676)(229.01178791,196.00804676)
\curveto(225.50179142,196.00804676)(224.12178791,198.82804982)(224.12178791,201.88804676)
\curveto(224.12178791,204.79804385)(225.44179133,207.73804676)(228.86178791,207.73804676)
\curveto(232.16178461,207.73804676)(233.66178791,204.91804385)(233.66178791,202.00804676)
\moveto(221.66178791,188.17804676)
\lineto(224.21178791,188.17804676)
\lineto(224.21178791,196.18804676)
\lineto(224.27178791,196.18804676)
\curveto(225.41178677,194.35804859)(227.6917895,193.75804676)(229.28178791,193.75804676)
\curveto(234.02178317,193.75804676)(236.36178791,197.44805114)(236.36178791,201.82804676)
\curveto(236.36178791,206.20804238)(233.99178314,209.98804676)(229.22178791,209.98804676)
\curveto(227.09179004,209.98804676)(225.11178707,209.23804505)(224.27178791,207.52804676)
\lineto(224.21178791,207.52804676)
\lineto(224.21178791,209.62804676)
\lineto(221.66178791,209.62804676)
\lineto(221.66178791,188.17804676)
}
}
{
\newrgbcolor{curcolor}{0 0 0}
\pscustom[linestyle=none,fillstyle=solid,fillcolor=curcolor]
{
\newpath
\moveto(238.54428791,201.85804676)
\curveto(238.54428791,197.32805129)(241.15429283,193.75804676)(246.07428791,193.75804676)
\curveto(250.99428299,193.75804676)(253.60428791,197.32805129)(253.60428791,201.85804676)
\curveto(253.60428791,206.4180422)(250.99428299,209.98804676)(246.07428791,209.98804676)
\curveto(241.15429283,209.98804676)(238.54428791,206.4180422)(238.54428791,201.85804676)
\moveto(241.24428791,201.85804676)
\curveto(241.24428791,205.63804298)(243.40429058,207.73804676)(246.07428791,207.73804676)
\curveto(248.74428524,207.73804676)(250.90428791,205.63804298)(250.90428791,201.85804676)
\curveto(250.90428791,198.10805051)(248.74428524,196.00804676)(246.07428791,196.00804676)
\curveto(243.40429058,196.00804676)(241.24428791,198.10805051)(241.24428791,201.85804676)
}
}
{
\newrgbcolor{curcolor}{0 0 0}
\pscustom[linewidth=2.65748024,linecolor=curcolor]
{
\newpath
\moveto(149.79512897,227.43689442)
\lineto(266.26969257,227.43689442)
\lineto(266.26969257,176.27921009)
\lineto(149.79512897,176.27921009)
\closepath
}
}
{
\newrgbcolor{curcolor}{0 0 0}
\pscustom[linewidth=2.65748024,linecolor=curcolor]
{
\newpath
\moveto(121.62025562,284.20666981)
\lineto(276.46554676,284.20666981)
\lineto(276.46554676,164.87876606)
\lineto(121.62025562,164.87876606)
\closepath
}
}
{
\newrgbcolor{curcolor}{0 0 0}
\pscustom[linewidth=2.65748024,linecolor=curcolor]
{
\newpath
\moveto(71.49931446,337.21338368)
\lineto(289.23117176,337.21338368)
\lineto(289.23117176,152.97859669)
\lineto(71.49931446,152.97859669)
\closepath
}
}
{
\newrgbcolor{curcolor}{0 0 0}
\pscustom[linewidth=2.65748024,linecolor=curcolor]
{
\newpath
\moveto(21.24184147,392.00745106)
\lineto(301.73921696,392.00745106)
\lineto(301.73921696,139.45129872)
\lineto(21.24184147,139.45129872)
\closepath
}
}
{
\newrgbcolor{curcolor}{0 0 0}
\pscustom[linewidth=2.48031497,linecolor=curcolor,linestyle=dashed,dash=2.48031496 9.92125984]
{
\newpath
\moveto(318.73283,412.24300838)
\lineto(318.73283,22.04274838)
}
}
{
\newrgbcolor{curcolor}{0 0 0}
\pscustom[linestyle=none,fillstyle=solid,fillcolor=curcolor]
{
\newpath
\moveto(76.25002533,61.54572387)
\lineto(86.17002533,61.54572387)
\lineto(86.17002533,63.14572387)
\lineto(78.15002533,63.14572387)
\lineto(78.15002533,68.08572387)
\lineto(85.57002533,68.08572387)
\lineto(85.57002533,69.68572387)
\lineto(78.15002533,69.68572387)
\lineto(78.15002533,74.22572387)
\lineto(86.11002533,74.22572387)
\lineto(86.11002533,75.82572387)
\lineto(76.25002533,75.82572387)
\lineto(76.25002533,61.54572387)
}
}
{
\newrgbcolor{curcolor}{0 0 0}
\pscustom[linestyle=none,fillstyle=solid,fillcolor=curcolor]
{
\newpath
\moveto(87.53658783,64.80572387)
\curveto(87.63658773,62.24572643)(89.59659015,61.30572387)(91.91658783,61.30572387)
\curveto(94.01658573,61.30572387)(96.31658783,62.10572633)(96.31658783,64.56572387)
\curveto(96.31658783,66.56572187)(94.63658613,67.12572425)(92.93658783,67.50572387)
\curveto(91.35658941,67.88572349)(89.55658783,68.08572509)(89.55658783,69.30572387)
\curveto(89.55658783,70.34572283)(90.73658885,70.62572387)(91.75658783,70.62572387)
\curveto(92.87658671,70.62572387)(94.03658795,70.20572255)(94.15658783,68.88572387)
\lineto(95.85658783,68.88572387)
\curveto(95.71658797,71.40572135)(93.89658555,72.12572387)(91.61658783,72.12572387)
\curveto(89.81658963,72.12572387)(87.75658783,71.26572179)(87.75658783,69.18572387)
\curveto(87.75658783,67.20572585)(89.45658951,66.64572349)(91.13658783,66.26572387)
\curveto(92.83658613,65.88572425)(94.51658783,65.66572255)(94.51658783,64.34572387)
\curveto(94.51658783,63.04572517)(93.07658677,62.80572387)(92.01658783,62.80572387)
\curveto(90.61658923,62.80572387)(89.29658777,63.28572539)(89.23658783,64.80572387)
\lineto(87.53658783,64.80572387)
}
}
{
\newrgbcolor{curcolor}{0 0 0}
\pscustom[linestyle=none,fillstyle=solid,fillcolor=curcolor]
{
\newpath
\moveto(106.25658783,66.80572387)
\curveto(106.25658783,64.78572589)(105.47658551,62.80572387)(103.15658783,62.80572387)
\curveto(100.81659017,62.80572387)(99.89658783,64.68572591)(99.89658783,66.72572387)
\curveto(99.89658783,68.66572193)(100.77659011,70.62572387)(103.05658783,70.62572387)
\curveto(105.25658563,70.62572387)(106.25658783,68.74572193)(106.25658783,66.80572387)
\moveto(98.25658783,57.58572387)
\lineto(99.95658783,57.58572387)
\lineto(99.95658783,62.92572387)
\lineto(99.99658783,62.92572387)
\curveto(100.75658707,61.70572509)(102.27658889,61.30572387)(103.33658783,61.30572387)
\curveto(106.49658467,61.30572387)(108.05658783,63.76572679)(108.05658783,66.68572387)
\curveto(108.05658783,69.60572095)(106.47658465,72.12572387)(103.29658783,72.12572387)
\curveto(101.87658925,72.12572387)(100.55658727,71.62572273)(99.99658783,70.48572387)
\lineto(99.95658783,70.48572387)
\lineto(99.95658783,71.88572387)
\lineto(98.25658783,71.88572387)
\lineto(98.25658783,57.58572387)
}
}
{
\newrgbcolor{curcolor}{0 0 0}
\pscustom[linestyle=none,fillstyle=solid,fillcolor=curcolor]
{
\newpath
\moveto(116.47158783,65.14572387)
\curveto(116.47158783,64.20572481)(115.55158559,62.80572387)(113.31158783,62.80572387)
\curveto(112.27158887,62.80572387)(111.31158783,63.20572499)(111.31158783,64.32572387)
\curveto(111.31158783,65.58572261)(112.27158895,65.98572407)(113.39158783,66.18572387)
\curveto(114.53158669,66.38572367)(115.81158849,66.40572435)(116.47158783,66.88572387)
\lineto(116.47158783,65.14572387)
\moveto(119.23158783,62.90572387)
\curveto(119.01158805,62.82572395)(118.85158769,62.80572387)(118.71158783,62.80572387)
\curveto(118.17158837,62.80572387)(118.17158783,63.16572467)(118.17158783,63.96572387)
\lineto(118.17158783,69.28572387)
\curveto(118.17158783,71.70572145)(116.15158597,72.12572387)(114.29158783,72.12572387)
\curveto(111.99159013,72.12572387)(110.01158773,71.22572131)(109.91158783,68.66572387)
\lineto(111.61158783,68.66572387)
\curveto(111.69158775,70.18572235)(112.75158927,70.62572387)(114.19158783,70.62572387)
\curveto(115.27158675,70.62572387)(116.49158783,70.38572239)(116.49158783,68.90572387)
\curveto(116.49158783,67.62572515)(114.89158595,67.74572351)(113.01158783,67.38572387)
\curveto(111.25158959,67.04572421)(109.51158783,66.54572153)(109.51158783,64.20572387)
\curveto(109.51158783,62.14572593)(111.05158971,61.30572387)(112.93158783,61.30572387)
\curveto(114.37158639,61.30572387)(115.63158877,61.80572497)(116.57158783,62.90572387)
\curveto(116.57158783,61.78572499)(117.13158871,61.30572387)(118.01158783,61.30572387)
\curveto(118.55158729,61.30572387)(118.93158813,61.40572405)(119.23158783,61.58572387)
\lineto(119.23158783,62.90572387)
}
}
{
\newrgbcolor{curcolor}{0 0 0}
\pscustom[linestyle=none,fillstyle=solid,fillcolor=curcolor]
{
\newpath
\moveto(129.59377533,68.56572387)
\curveto(129.35377557,71.02572141)(127.47377299,72.12572387)(125.13377533,72.12572387)
\curveto(121.85377861,72.12572387)(120.25377533,69.68572077)(120.25377533,66.58572387)
\curveto(120.25377533,63.50572695)(121.93377849,61.30572387)(125.09377533,61.30572387)
\curveto(127.69377273,61.30572387)(129.27377571,62.80572639)(129.65377533,65.32572387)
\lineto(127.91377533,65.32572387)
\curveto(127.69377555,63.76572543)(126.71377369,62.80572387)(125.07377533,62.80572387)
\curveto(122.91377749,62.80572387)(122.05377533,64.68572577)(122.05377533,66.58572387)
\curveto(122.05377533,68.68572177)(122.81377779,70.62572387)(125.27377533,70.62572387)
\curveto(126.67377393,70.62572387)(127.57377559,69.86572257)(127.83377533,68.56572387)
\lineto(129.59377533,68.56572387)
}
}
{
\newrgbcolor{curcolor}{0 0 0}
\pscustom[linestyle=none,fillstyle=solid,fillcolor=curcolor]
{
\newpath
\moveto(131.65596283,61.54572387)
\lineto(133.35596283,61.54572387)
\lineto(133.35596283,71.88572387)
\lineto(131.65596283,71.88572387)
\lineto(131.65596283,61.54572387)
\moveto(133.35596283,75.82572387)
\lineto(131.65596283,75.82572387)
\lineto(131.65596283,73.74572387)
\lineto(133.35596283,73.74572387)
\lineto(133.35596283,75.82572387)
}
}
{
\newrgbcolor{curcolor}{0 0 0}
\pscustom[linestyle=none,fillstyle=solid,fillcolor=curcolor]
{
\newpath
\moveto(135.44908783,66.70572387)
\curveto(135.44908783,63.68572689)(137.18909111,61.30572387)(140.46908783,61.30572387)
\curveto(143.74908455,61.30572387)(145.48908783,63.68572689)(145.48908783,66.70572387)
\curveto(145.48908783,69.74572083)(143.74908455,72.12572387)(140.46908783,72.12572387)
\curveto(137.18909111,72.12572387)(135.44908783,69.74572083)(135.44908783,66.70572387)
\moveto(137.24908783,66.70572387)
\curveto(137.24908783,69.22572135)(138.68908961,70.62572387)(140.46908783,70.62572387)
\curveto(142.24908605,70.62572387)(143.68908783,69.22572135)(143.68908783,66.70572387)
\curveto(143.68908783,64.20572637)(142.24908605,62.80572387)(140.46908783,62.80572387)
\curveto(138.68908961,62.80572387)(137.24908783,64.20572637)(137.24908783,66.70572387)
}
}
{
\newrgbcolor{curcolor}{0 0 0}
\pscustom[linestyle=none,fillstyle=solid,fillcolor=curcolor]
{
\newpath
\moveto(146.83346283,64.80572387)
\curveto(146.93346273,62.24572643)(148.89346515,61.30572387)(151.21346283,61.30572387)
\curveto(153.31346073,61.30572387)(155.61346283,62.10572633)(155.61346283,64.56572387)
\curveto(155.61346283,66.56572187)(153.93346113,67.12572425)(152.23346283,67.50572387)
\curveto(150.65346441,67.88572349)(148.85346283,68.08572509)(148.85346283,69.30572387)
\curveto(148.85346283,70.34572283)(150.03346385,70.62572387)(151.05346283,70.62572387)
\curveto(152.17346171,70.62572387)(153.33346295,70.20572255)(153.45346283,68.88572387)
\lineto(155.15346283,68.88572387)
\curveto(155.01346297,71.40572135)(153.19346055,72.12572387)(150.91346283,72.12572387)
\curveto(149.11346463,72.12572387)(147.05346283,71.26572179)(147.05346283,69.18572387)
\curveto(147.05346283,67.20572585)(148.75346451,66.64572349)(150.43346283,66.26572387)
\curveto(152.13346113,65.88572425)(153.81346283,65.66572255)(153.81346283,64.34572387)
\curveto(153.81346283,63.04572517)(152.37346177,62.80572387)(151.31346283,62.80572387)
\curveto(149.91346423,62.80572387)(148.59346277,63.28572539)(148.53346283,64.80572387)
\lineto(146.83346283,64.80572387)
}
}
{
\newrgbcolor{curcolor}{0 0 0}
\pscustom[linestyle=none,fillstyle=solid,fillcolor=curcolor]
{
\newpath
\moveto(78.33002533,49.98572387)
\lineto(76.63002533,49.98572387)
\lineto(76.63002533,46.88572387)
\lineto(74.87002533,46.88572387)
\lineto(74.87002533,45.38572387)
\lineto(76.63002533,45.38572387)
\lineto(76.63002533,38.80572387)
\curveto(76.63002533,36.90572577)(77.33002709,36.54572387)(79.09002533,36.54572387)
\lineto(80.39002533,36.54572387)
\lineto(80.39002533,38.04572387)
\lineto(79.61002533,38.04572387)
\curveto(78.55002639,38.04572387)(78.33002533,38.18572465)(78.33002533,38.96572387)
\lineto(78.33002533,45.38572387)
\lineto(80.39002533,45.38572387)
\lineto(80.39002533,46.88572387)
\lineto(78.33002533,46.88572387)
\lineto(78.33002533,49.98572387)
}
}
{
\newrgbcolor{curcolor}{0 0 0}
\pscustom[linestyle=none,fillstyle=solid,fillcolor=curcolor]
{
\newpath
\moveto(89.39908783,39.82572387)
\curveto(89.09908813,38.48572521)(88.11908643,37.80572387)(86.71908783,37.80572387)
\curveto(84.45909009,37.80572387)(83.43908789,39.40572567)(83.49908783,41.20572387)
\lineto(91.23908783,41.20572387)
\curveto(91.33908773,43.70572137)(90.21908417,47.12572387)(86.55908783,47.12572387)
\curveto(83.73909065,47.12572387)(81.69908783,44.84572077)(81.69908783,41.74572387)
\curveto(81.79908773,38.58572703)(83.35909113,36.30572387)(86.65908783,36.30572387)
\curveto(88.97908551,36.30572387)(90.61908829,37.54572615)(91.07908783,39.82572387)
\lineto(89.39908783,39.82572387)
\moveto(83.49908783,42.70572387)
\curveto(83.61908771,44.28572229)(84.67908961,45.62572387)(86.45908783,45.62572387)
\curveto(88.13908615,45.62572387)(89.35908791,44.32572225)(89.43908783,42.70572387)
\lineto(83.49908783,42.70572387)
}
}
{
\newrgbcolor{curcolor}{0 0 0}
\pscustom[linestyle=none,fillstyle=solid,fillcolor=curcolor]
{
\newpath
\moveto(93.00127533,36.54572387)
\lineto(94.70127533,36.54572387)
\lineto(94.70127533,42.98572387)
\curveto(94.70127533,43.76572309)(95.44127735,45.62572387)(97.46127533,45.62572387)
\curveto(98.98127381,45.62572387)(99.40127533,44.66572253)(99.40127533,43.32572387)
\lineto(99.40127533,36.54572387)
\lineto(101.10127533,36.54572387)
\lineto(101.10127533,42.98572387)
\curveto(101.10127533,44.58572227)(102.16127697,45.62572387)(103.80127533,45.62572387)
\curveto(105.46127367,45.62572387)(105.80127533,44.60572259)(105.80127533,43.32572387)
\lineto(105.80127533,36.54572387)
\lineto(107.50127533,36.54572387)
\lineto(107.50127533,44.12572387)
\curveto(107.50127533,46.26572173)(106.12127327,47.12572387)(104.06127533,47.12572387)
\curveto(102.74127665,47.12572387)(101.52127463,46.46572277)(100.82127533,45.36572387)
\curveto(100.40127575,46.62572261)(99.24127407,47.12572387)(97.98127533,47.12572387)
\curveto(96.56127675,47.12572387)(95.40127457,46.52572271)(94.64127533,45.36572387)
\lineto(94.60127533,45.36572387)
\lineto(94.60127533,46.88572387)
\lineto(93.00127533,46.88572387)
\lineto(93.00127533,36.54572387)
}
}
{
\newrgbcolor{curcolor}{0 0 0}
\pscustom[linestyle=none,fillstyle=solid,fillcolor=curcolor]
{
\newpath
\moveto(118.13158783,41.80572387)
\curveto(118.13158783,39.78572589)(117.35158551,37.80572387)(115.03158783,37.80572387)
\curveto(112.69159017,37.80572387)(111.77158783,39.68572591)(111.77158783,41.72572387)
\curveto(111.77158783,43.66572193)(112.65159011,45.62572387)(114.93158783,45.62572387)
\curveto(117.13158563,45.62572387)(118.13158783,43.74572193)(118.13158783,41.80572387)
\moveto(110.13158783,32.58572387)
\lineto(111.83158783,32.58572387)
\lineto(111.83158783,37.92572387)
\lineto(111.87158783,37.92572387)
\curveto(112.63158707,36.70572509)(114.15158889,36.30572387)(115.21158783,36.30572387)
\curveto(118.37158467,36.30572387)(119.93158783,38.76572679)(119.93158783,41.68572387)
\curveto(119.93158783,44.60572095)(118.35158465,47.12572387)(115.17158783,47.12572387)
\curveto(113.75158925,47.12572387)(112.43158727,46.62572273)(111.87158783,45.48572387)
\lineto(111.83158783,45.48572387)
\lineto(111.83158783,46.88572387)
\lineto(110.13158783,46.88572387)
\lineto(110.13158783,32.58572387)
}
}
{
\newrgbcolor{curcolor}{0 0 0}
\pscustom[linestyle=none,fillstyle=solid,fillcolor=curcolor]
{
\newpath
\moveto(121.38658783,41.70572387)
\curveto(121.38658783,38.68572689)(123.12659111,36.30572387)(126.40658783,36.30572387)
\curveto(129.68658455,36.30572387)(131.42658783,38.68572689)(131.42658783,41.70572387)
\curveto(131.42658783,44.74572083)(129.68658455,47.12572387)(126.40658783,47.12572387)
\curveto(123.12659111,47.12572387)(121.38658783,44.74572083)(121.38658783,41.70572387)
\moveto(123.18658783,41.70572387)
\curveto(123.18658783,44.22572135)(124.62658961,45.62572387)(126.40658783,45.62572387)
\curveto(128.18658605,45.62572387)(129.62658783,44.22572135)(129.62658783,41.70572387)
\curveto(129.62658783,39.20572637)(128.18658605,37.80572387)(126.40658783,37.80572387)
\curveto(124.62658961,37.80572387)(123.18658783,39.20572637)(123.18658783,41.70572387)
}
}
{
\newrgbcolor{curcolor}{0 0 0}
\pscustom[linestyle=none,fillstyle=solid,fillcolor=curcolor]
{
\newpath
\moveto(133.37096283,36.54572387)
\lineto(135.07096283,36.54572387)
\lineto(135.07096283,41.14572387)
\curveto(135.07096283,43.76572125)(136.07096557,45.32572387)(138.81096283,45.32572387)
\lineto(138.81096283,47.12572387)
\curveto(136.97096467,47.18572381)(135.83096201,46.36572221)(135.01096283,44.70572387)
\lineto(134.97096283,44.70572387)
\lineto(134.97096283,46.88572387)
\lineto(133.37096283,46.88572387)
\lineto(133.37096283,36.54572387)
}
}
{
\newrgbcolor{curcolor}{0 0 0}
\pscustom[linestyle=none,fillstyle=solid,fillcolor=curcolor]
{
\newpath
\moveto(146.51065033,40.14572387)
\curveto(146.51065033,39.20572481)(145.59064809,37.80572387)(143.35065033,37.80572387)
\curveto(142.31065137,37.80572387)(141.35065033,38.20572499)(141.35065033,39.32572387)
\curveto(141.35065033,40.58572261)(142.31065145,40.98572407)(143.43065033,41.18572387)
\curveto(144.57064919,41.38572367)(145.85065099,41.40572435)(146.51065033,41.88572387)
\lineto(146.51065033,40.14572387)
\moveto(149.27065033,37.90572387)
\curveto(149.05065055,37.82572395)(148.89065019,37.80572387)(148.75065033,37.80572387)
\curveto(148.21065087,37.80572387)(148.21065033,38.16572467)(148.21065033,38.96572387)
\lineto(148.21065033,44.28572387)
\curveto(148.21065033,46.70572145)(146.19064847,47.12572387)(144.33065033,47.12572387)
\curveto(142.03065263,47.12572387)(140.05065023,46.22572131)(139.95065033,43.66572387)
\lineto(141.65065033,43.66572387)
\curveto(141.73065025,45.18572235)(142.79065177,45.62572387)(144.23065033,45.62572387)
\curveto(145.31064925,45.62572387)(146.53065033,45.38572239)(146.53065033,43.90572387)
\curveto(146.53065033,42.62572515)(144.93064845,42.74572351)(143.05065033,42.38572387)
\curveto(141.29065209,42.04572421)(139.55065033,41.54572153)(139.55065033,39.20572387)
\curveto(139.55065033,37.14572593)(141.09065221,36.30572387)(142.97065033,36.30572387)
\curveto(144.41064889,36.30572387)(145.67065127,36.80572497)(146.61065033,37.90572387)
\curveto(146.61065033,36.78572499)(147.17065121,36.30572387)(148.05065033,36.30572387)
\curveto(148.59064979,36.30572387)(148.97065063,36.40572405)(149.27065033,36.58572387)
\lineto(149.27065033,37.90572387)
}
}
{
\newrgbcolor{curcolor}{0 0 0}
\pscustom[linestyle=none,fillstyle=solid,fillcolor=curcolor]
{
\newpath
\moveto(150.95283783,36.54572387)
\lineto(152.65283783,36.54572387)
\lineto(152.65283783,50.82572387)
\lineto(150.95283783,50.82572387)
\lineto(150.95283783,36.54572387)
}
}
{
\newrgbcolor{curcolor}{0 0 0}
\pscustom[linestyle=none,fillstyle=solid,fillcolor=curcolor]
{
\newpath
\moveto(162.44596283,39.82572387)
\curveto(162.14596313,38.48572521)(161.16596143,37.80572387)(159.76596283,37.80572387)
\curveto(157.50596509,37.80572387)(156.48596289,39.40572567)(156.54596283,41.20572387)
\lineto(164.28596283,41.20572387)
\curveto(164.38596273,43.70572137)(163.26595917,47.12572387)(159.60596283,47.12572387)
\curveto(156.78596565,47.12572387)(154.74596283,44.84572077)(154.74596283,41.74572387)
\curveto(154.84596273,38.58572703)(156.40596613,36.30572387)(159.70596283,36.30572387)
\curveto(162.02596051,36.30572387)(163.66596329,37.54572615)(164.12596283,39.82572387)
\lineto(162.44596283,39.82572387)
\moveto(156.54596283,42.70572387)
\curveto(156.66596271,44.28572229)(157.72596461,45.62572387)(159.50596283,45.62572387)
\curveto(161.18596115,45.62572387)(162.40596291,44.32572225)(162.48596283,42.70572387)
\lineto(156.54596283,42.70572387)
}
}
{
\newrgbcolor{curcolor}{0 0 0}
\pscustom[linestyle=none,fillstyle=solid,fillcolor=curcolor]
{
\newpath
\moveto(165.38815033,39.80572387)
\curveto(165.48815023,37.24572643)(167.44815265,36.30572387)(169.76815033,36.30572387)
\curveto(171.86814823,36.30572387)(174.16815033,37.10572633)(174.16815033,39.56572387)
\curveto(174.16815033,41.56572187)(172.48814863,42.12572425)(170.78815033,42.50572387)
\curveto(169.20815191,42.88572349)(167.40815033,43.08572509)(167.40815033,44.30572387)
\curveto(167.40815033,45.34572283)(168.58815135,45.62572387)(169.60815033,45.62572387)
\curveto(170.72814921,45.62572387)(171.88815045,45.20572255)(172.00815033,43.88572387)
\lineto(173.70815033,43.88572387)
\curveto(173.56815047,46.40572135)(171.74814805,47.12572387)(169.46815033,47.12572387)
\curveto(167.66815213,47.12572387)(165.60815033,46.26572179)(165.60815033,44.18572387)
\curveto(165.60815033,42.20572585)(167.30815201,41.64572349)(168.98815033,41.26572387)
\curveto(170.68814863,40.88572425)(172.36815033,40.66572255)(172.36815033,39.34572387)
\curveto(172.36815033,38.04572517)(170.92814927,37.80572387)(169.86815033,37.80572387)
\curveto(168.46815173,37.80572387)(167.14815027,38.28572539)(167.08815033,39.80572387)
\lineto(165.38815033,39.80572387)
}
}
{
\newrgbcolor{curcolor}{0 0 0}
\pscustom[linestyle=none,fillstyle=solid,fillcolor=curcolor]
{
\newpath
\moveto(382.96151978,61.54572387)
\lineto(392.88151978,61.54572387)
\lineto(392.88151978,63.14572387)
\lineto(384.86151978,63.14572387)
\lineto(384.86151978,68.08572387)
\lineto(392.28151978,68.08572387)
\lineto(392.28151978,69.68572387)
\lineto(384.86151978,69.68572387)
\lineto(384.86151978,74.22572387)
\lineto(392.82151978,74.22572387)
\lineto(392.82151978,75.82572387)
\lineto(382.96151978,75.82572387)
\lineto(382.96151978,61.54572387)
}
}
{
\newrgbcolor{curcolor}{0 0 0}
\pscustom[linestyle=none,fillstyle=solid,fillcolor=curcolor]
{
\newpath
\moveto(394.24808228,64.80572387)
\curveto(394.34808218,62.24572643)(396.3080846,61.30572387)(398.62808228,61.30572387)
\curveto(400.72808018,61.30572387)(403.02808228,62.10572633)(403.02808228,64.56572387)
\curveto(403.02808228,66.56572187)(401.34808058,67.12572425)(399.64808228,67.50572387)
\curveto(398.06808386,67.88572349)(396.26808228,68.08572509)(396.26808228,69.30572387)
\curveto(396.26808228,70.34572283)(397.4480833,70.62572387)(398.46808228,70.62572387)
\curveto(399.58808116,70.62572387)(400.7480824,70.20572255)(400.86808228,68.88572387)
\lineto(402.56808228,68.88572387)
\curveto(402.42808242,71.40572135)(400.60808,72.12572387)(398.32808228,72.12572387)
\curveto(396.52808408,72.12572387)(394.46808228,71.26572179)(394.46808228,69.18572387)
\curveto(394.46808228,67.20572585)(396.16808396,66.64572349)(397.84808228,66.26572387)
\curveto(399.54808058,65.88572425)(401.22808228,65.66572255)(401.22808228,64.34572387)
\curveto(401.22808228,63.04572517)(399.78808122,62.80572387)(398.72808228,62.80572387)
\curveto(397.32808368,62.80572387)(396.00808222,63.28572539)(395.94808228,64.80572387)
\lineto(394.24808228,64.80572387)
}
}
{
\newrgbcolor{curcolor}{0 0 0}
\pscustom[linestyle=none,fillstyle=solid,fillcolor=curcolor]
{
\newpath
\moveto(412.96808228,66.80572387)
\curveto(412.96808228,64.78572589)(412.18807996,62.80572387)(409.86808228,62.80572387)
\curveto(407.52808462,62.80572387)(406.60808228,64.68572591)(406.60808228,66.72572387)
\curveto(406.60808228,68.66572193)(407.48808456,70.62572387)(409.76808228,70.62572387)
\curveto(411.96808008,70.62572387)(412.96808228,68.74572193)(412.96808228,66.80572387)
\moveto(404.96808228,57.58572387)
\lineto(406.66808228,57.58572387)
\lineto(406.66808228,62.92572387)
\lineto(406.70808228,62.92572387)
\curveto(407.46808152,61.70572509)(408.98808334,61.30572387)(410.04808228,61.30572387)
\curveto(413.20807912,61.30572387)(414.76808228,63.76572679)(414.76808228,66.68572387)
\curveto(414.76808228,69.60572095)(413.1880791,72.12572387)(410.00808228,72.12572387)
\curveto(408.5880837,72.12572387)(407.26808172,71.62572273)(406.70808228,70.48572387)
\lineto(406.66808228,70.48572387)
\lineto(406.66808228,71.88572387)
\lineto(404.96808228,71.88572387)
\lineto(404.96808228,57.58572387)
}
}
{
\newrgbcolor{curcolor}{0 0 0}
\pscustom[linestyle=none,fillstyle=solid,fillcolor=curcolor]
{
\newpath
\moveto(423.18308228,65.14572387)
\curveto(423.18308228,64.20572481)(422.26308004,62.80572387)(420.02308228,62.80572387)
\curveto(418.98308332,62.80572387)(418.02308228,63.20572499)(418.02308228,64.32572387)
\curveto(418.02308228,65.58572261)(418.9830834,65.98572407)(420.10308228,66.18572387)
\curveto(421.24308114,66.38572367)(422.52308294,66.40572435)(423.18308228,66.88572387)
\lineto(423.18308228,65.14572387)
\moveto(425.94308228,62.90572387)
\curveto(425.7230825,62.82572395)(425.56308214,62.80572387)(425.42308228,62.80572387)
\curveto(424.88308282,62.80572387)(424.88308228,63.16572467)(424.88308228,63.96572387)
\lineto(424.88308228,69.28572387)
\curveto(424.88308228,71.70572145)(422.86308042,72.12572387)(421.00308228,72.12572387)
\curveto(418.70308458,72.12572387)(416.72308218,71.22572131)(416.62308228,68.66572387)
\lineto(418.32308228,68.66572387)
\curveto(418.4030822,70.18572235)(419.46308372,70.62572387)(420.90308228,70.62572387)
\curveto(421.9830812,70.62572387)(423.20308228,70.38572239)(423.20308228,68.90572387)
\curveto(423.20308228,67.62572515)(421.6030804,67.74572351)(419.72308228,67.38572387)
\curveto(417.96308404,67.04572421)(416.22308228,66.54572153)(416.22308228,64.20572387)
\curveto(416.22308228,62.14572593)(417.76308416,61.30572387)(419.64308228,61.30572387)
\curveto(421.08308084,61.30572387)(422.34308322,61.80572497)(423.28308228,62.90572387)
\curveto(423.28308228,61.78572499)(423.84308316,61.30572387)(424.72308228,61.30572387)
\curveto(425.26308174,61.30572387)(425.64308258,61.40572405)(425.94308228,61.58572387)
\lineto(425.94308228,62.90572387)
}
}
{
\newrgbcolor{curcolor}{0 0 0}
\pscustom[linestyle=none,fillstyle=solid,fillcolor=curcolor]
{
\newpath
\moveto(436.30526978,68.56572387)
\curveto(436.06527002,71.02572141)(434.18526744,72.12572387)(431.84526978,72.12572387)
\curveto(428.56527306,72.12572387)(426.96526978,69.68572077)(426.96526978,66.58572387)
\curveto(426.96526978,63.50572695)(428.64527294,61.30572387)(431.80526978,61.30572387)
\curveto(434.40526718,61.30572387)(435.98527016,62.80572639)(436.36526978,65.32572387)
\lineto(434.62526978,65.32572387)
\curveto(434.40527,63.76572543)(433.42526814,62.80572387)(431.78526978,62.80572387)
\curveto(429.62527194,62.80572387)(428.76526978,64.68572577)(428.76526978,66.58572387)
\curveto(428.76526978,68.68572177)(429.52527224,70.62572387)(431.98526978,70.62572387)
\curveto(433.38526838,70.62572387)(434.28527004,69.86572257)(434.54526978,68.56572387)
\lineto(436.30526978,68.56572387)
}
}
{
\newrgbcolor{curcolor}{0 0 0}
\pscustom[linestyle=none,fillstyle=solid,fillcolor=curcolor]
{
\newpath
\moveto(438.36745728,61.54572387)
\lineto(440.06745728,61.54572387)
\lineto(440.06745728,71.88572387)
\lineto(438.36745728,71.88572387)
\lineto(438.36745728,61.54572387)
\moveto(440.06745728,75.82572387)
\lineto(438.36745728,75.82572387)
\lineto(438.36745728,73.74572387)
\lineto(440.06745728,73.74572387)
\lineto(440.06745728,75.82572387)
}
}
{
\newrgbcolor{curcolor}{0 0 0}
\pscustom[linestyle=none,fillstyle=solid,fillcolor=curcolor]
{
\newpath
\moveto(442.16058228,66.70572387)
\curveto(442.16058228,63.68572689)(443.90058556,61.30572387)(447.18058228,61.30572387)
\curveto(450.460579,61.30572387)(452.20058228,63.68572689)(452.20058228,66.70572387)
\curveto(452.20058228,69.74572083)(450.460579,72.12572387)(447.18058228,72.12572387)
\curveto(443.90058556,72.12572387)(442.16058228,69.74572083)(442.16058228,66.70572387)
\moveto(443.96058228,66.70572387)
\curveto(443.96058228,69.22572135)(445.40058406,70.62572387)(447.18058228,70.62572387)
\curveto(448.9605805,70.62572387)(450.40058228,69.22572135)(450.40058228,66.70572387)
\curveto(450.40058228,64.20572637)(448.9605805,62.80572387)(447.18058228,62.80572387)
\curveto(445.40058406,62.80572387)(443.96058228,64.20572637)(443.96058228,66.70572387)
}
}
{
\newrgbcolor{curcolor}{0 0 0}
\pscustom[linestyle=none,fillstyle=solid,fillcolor=curcolor]
{
\newpath
\moveto(453.54495728,64.80572387)
\curveto(453.64495718,62.24572643)(455.6049596,61.30572387)(457.92495728,61.30572387)
\curveto(460.02495518,61.30572387)(462.32495728,62.10572633)(462.32495728,64.56572387)
\curveto(462.32495728,66.56572187)(460.64495558,67.12572425)(458.94495728,67.50572387)
\curveto(457.36495886,67.88572349)(455.56495728,68.08572509)(455.56495728,69.30572387)
\curveto(455.56495728,70.34572283)(456.7449583,70.62572387)(457.76495728,70.62572387)
\curveto(458.88495616,70.62572387)(460.0449574,70.20572255)(460.16495728,68.88572387)
\lineto(461.86495728,68.88572387)
\curveto(461.72495742,71.40572135)(459.904955,72.12572387)(457.62495728,72.12572387)
\curveto(455.82495908,72.12572387)(453.76495728,71.26572179)(453.76495728,69.18572387)
\curveto(453.76495728,67.20572585)(455.46495896,66.64572349)(457.14495728,66.26572387)
\curveto(458.84495558,65.88572425)(460.52495728,65.66572255)(460.52495728,64.34572387)
\curveto(460.52495728,63.04572517)(459.08495622,62.80572387)(458.02495728,62.80572387)
\curveto(456.62495868,62.80572387)(455.30495722,63.28572539)(455.24495728,64.80572387)
\lineto(453.54495728,64.80572387)
}
}
{
\newrgbcolor{curcolor}{0 0 0}
\pscustom[linestyle=none,fillstyle=solid,fillcolor=curcolor]
{
\newpath
\moveto(389.08151978,40.14572387)
\curveto(389.08151978,39.20572481)(388.16151754,37.80572387)(385.92151978,37.80572387)
\curveto(384.88152082,37.80572387)(383.92151978,38.20572499)(383.92151978,39.32572387)
\curveto(383.92151978,40.58572261)(384.8815209,40.98572407)(386.00151978,41.18572387)
\curveto(387.14151864,41.38572367)(388.42152044,41.40572435)(389.08151978,41.88572387)
\lineto(389.08151978,40.14572387)
\moveto(391.84151978,37.90572387)
\curveto(391.62152,37.82572395)(391.46151964,37.80572387)(391.32151978,37.80572387)
\curveto(390.78152032,37.80572387)(390.78151978,38.16572467)(390.78151978,38.96572387)
\lineto(390.78151978,44.28572387)
\curveto(390.78151978,46.70572145)(388.76151792,47.12572387)(386.90151978,47.12572387)
\curveto(384.60152208,47.12572387)(382.62151968,46.22572131)(382.52151978,43.66572387)
\lineto(384.22151978,43.66572387)
\curveto(384.3015197,45.18572235)(385.36152122,45.62572387)(386.80151978,45.62572387)
\curveto(387.8815187,45.62572387)(389.10151978,45.38572239)(389.10151978,43.90572387)
\curveto(389.10151978,42.62572515)(387.5015179,42.74572351)(385.62151978,42.38572387)
\curveto(383.86152154,42.04572421)(382.12151978,41.54572153)(382.12151978,39.20572387)
\curveto(382.12151978,37.14572593)(383.66152166,36.30572387)(385.54151978,36.30572387)
\curveto(386.98151834,36.30572387)(388.24152072,36.80572497)(389.18151978,37.90572387)
\curveto(389.18151978,36.78572499)(389.74152066,36.30572387)(390.62151978,36.30572387)
\curveto(391.16151924,36.30572387)(391.54152008,36.40572405)(391.84151978,36.58572387)
\lineto(391.84151978,37.90572387)
}
}
{
\newrgbcolor{curcolor}{0 0 0}
\pscustom[linestyle=none,fillstyle=solid,fillcolor=curcolor]
{
\newpath
\moveto(395.78370728,49.98572387)
\lineto(394.08370728,49.98572387)
\lineto(394.08370728,46.88572387)
\lineto(392.32370728,46.88572387)
\lineto(392.32370728,45.38572387)
\lineto(394.08370728,45.38572387)
\lineto(394.08370728,38.80572387)
\curveto(394.08370728,36.90572577)(394.78370904,36.54572387)(396.54370728,36.54572387)
\lineto(397.84370728,36.54572387)
\lineto(397.84370728,38.04572387)
\lineto(397.06370728,38.04572387)
\curveto(396.00370834,38.04572387)(395.78370728,38.18572465)(395.78370728,38.96572387)
\lineto(395.78370728,45.38572387)
\lineto(397.84370728,45.38572387)
\lineto(397.84370728,46.88572387)
\lineto(395.78370728,46.88572387)
\lineto(395.78370728,49.98572387)
}
}
{
\newrgbcolor{curcolor}{0 0 0}
\pscustom[linestyle=none,fillstyle=solid,fillcolor=curcolor]
{
\newpath
\moveto(406.85276978,39.82572387)
\curveto(406.55277008,38.48572521)(405.57276838,37.80572387)(404.17276978,37.80572387)
\curveto(401.91277204,37.80572387)(400.89276984,39.40572567)(400.95276978,41.20572387)
\lineto(408.69276978,41.20572387)
\curveto(408.79276968,43.70572137)(407.67276612,47.12572387)(404.01276978,47.12572387)
\curveto(401.1927726,47.12572387)(399.15276978,44.84572077)(399.15276978,41.74572387)
\curveto(399.25276968,38.58572703)(400.81277308,36.30572387)(404.11276978,36.30572387)
\curveto(406.43276746,36.30572387)(408.07277024,37.54572615)(408.53276978,39.82572387)
\lineto(406.85276978,39.82572387)
\moveto(400.95276978,42.70572387)
\curveto(401.07276966,44.28572229)(402.13277156,45.62572387)(403.91276978,45.62572387)
\curveto(405.5927681,45.62572387)(406.81276986,44.32572225)(406.89276978,42.70572387)
\lineto(400.95276978,42.70572387)
}
}
{
\newrgbcolor{curcolor}{0 0 0}
\pscustom[linestyle=none,fillstyle=solid,fillcolor=curcolor]
{
\newpath
\moveto(410.45495728,36.54572387)
\lineto(412.15495728,36.54572387)
\lineto(412.15495728,42.98572387)
\curveto(412.15495728,43.76572309)(412.8949593,45.62572387)(414.91495728,45.62572387)
\curveto(416.43495576,45.62572387)(416.85495728,44.66572253)(416.85495728,43.32572387)
\lineto(416.85495728,36.54572387)
\lineto(418.55495728,36.54572387)
\lineto(418.55495728,42.98572387)
\curveto(418.55495728,44.58572227)(419.61495892,45.62572387)(421.25495728,45.62572387)
\curveto(422.91495562,45.62572387)(423.25495728,44.60572259)(423.25495728,43.32572387)
\lineto(423.25495728,36.54572387)
\lineto(424.95495728,36.54572387)
\lineto(424.95495728,44.12572387)
\curveto(424.95495728,46.26572173)(423.57495522,47.12572387)(421.51495728,47.12572387)
\curveto(420.1949586,47.12572387)(418.97495658,46.46572277)(418.27495728,45.36572387)
\curveto(417.8549577,46.62572261)(416.69495602,47.12572387)(415.43495728,47.12572387)
\curveto(414.0149587,47.12572387)(412.85495652,46.52572271)(412.09495728,45.36572387)
\lineto(412.05495728,45.36572387)
\lineto(412.05495728,46.88572387)
\lineto(410.45495728,46.88572387)
\lineto(410.45495728,36.54572387)
}
}
{
\newrgbcolor{curcolor}{0 0 0}
\pscustom[linestyle=none,fillstyle=solid,fillcolor=curcolor]
{
\newpath
\moveto(435.58526978,41.80572387)
\curveto(435.58526978,39.78572589)(434.80526746,37.80572387)(432.48526978,37.80572387)
\curveto(430.14527212,37.80572387)(429.22526978,39.68572591)(429.22526978,41.72572387)
\curveto(429.22526978,43.66572193)(430.10527206,45.62572387)(432.38526978,45.62572387)
\curveto(434.58526758,45.62572387)(435.58526978,43.74572193)(435.58526978,41.80572387)
\moveto(427.58526978,32.58572387)
\lineto(429.28526978,32.58572387)
\lineto(429.28526978,37.92572387)
\lineto(429.32526978,37.92572387)
\curveto(430.08526902,36.70572509)(431.60527084,36.30572387)(432.66526978,36.30572387)
\curveto(435.82526662,36.30572387)(437.38526978,38.76572679)(437.38526978,41.68572387)
\curveto(437.38526978,44.60572095)(435.8052666,47.12572387)(432.62526978,47.12572387)
\curveto(431.2052712,47.12572387)(429.88526922,46.62572273)(429.32526978,45.48572387)
\lineto(429.28526978,45.48572387)
\lineto(429.28526978,46.88572387)
\lineto(427.58526978,46.88572387)
\lineto(427.58526978,32.58572387)
}
}
{
\newrgbcolor{curcolor}{0 0 0}
\pscustom[linestyle=none,fillstyle=solid,fillcolor=curcolor]
{
\newpath
\moveto(438.84026978,41.70572387)
\curveto(438.84026978,38.68572689)(440.58027306,36.30572387)(443.86026978,36.30572387)
\curveto(447.1402665,36.30572387)(448.88026978,38.68572689)(448.88026978,41.70572387)
\curveto(448.88026978,44.74572083)(447.1402665,47.12572387)(443.86026978,47.12572387)
\curveto(440.58027306,47.12572387)(438.84026978,44.74572083)(438.84026978,41.70572387)
\moveto(440.64026978,41.70572387)
\curveto(440.64026978,44.22572135)(442.08027156,45.62572387)(443.86026978,45.62572387)
\curveto(445.640268,45.62572387)(447.08026978,44.22572135)(447.08026978,41.70572387)
\curveto(447.08026978,39.20572637)(445.640268,37.80572387)(443.86026978,37.80572387)
\curveto(442.08027156,37.80572387)(440.64026978,39.20572637)(440.64026978,41.70572387)
}
}
{
\newrgbcolor{curcolor}{0 0 0}
\pscustom[linestyle=none,fillstyle=solid,fillcolor=curcolor]
{
\newpath
\moveto(450.82464478,36.54572387)
\lineto(452.52464478,36.54572387)
\lineto(452.52464478,41.14572387)
\curveto(452.52464478,43.76572125)(453.52464752,45.32572387)(456.26464478,45.32572387)
\lineto(456.26464478,47.12572387)
\curveto(454.42464662,47.18572381)(453.28464396,46.36572221)(452.46464478,44.70572387)
\lineto(452.42464478,44.70572387)
\lineto(452.42464478,46.88572387)
\lineto(450.82464478,46.88572387)
\lineto(450.82464478,36.54572387)
}
}
{
\newrgbcolor{curcolor}{0 0 0}
\pscustom[linestyle=none,fillstyle=solid,fillcolor=curcolor]
{
\newpath
\moveto(463.96433228,40.14572387)
\curveto(463.96433228,39.20572481)(463.04433004,37.80572387)(460.80433228,37.80572387)
\curveto(459.76433332,37.80572387)(458.80433228,38.20572499)(458.80433228,39.32572387)
\curveto(458.80433228,40.58572261)(459.7643334,40.98572407)(460.88433228,41.18572387)
\curveto(462.02433114,41.38572367)(463.30433294,41.40572435)(463.96433228,41.88572387)
\lineto(463.96433228,40.14572387)
\moveto(466.72433228,37.90572387)
\curveto(466.5043325,37.82572395)(466.34433214,37.80572387)(466.20433228,37.80572387)
\curveto(465.66433282,37.80572387)(465.66433228,38.16572467)(465.66433228,38.96572387)
\lineto(465.66433228,44.28572387)
\curveto(465.66433228,46.70572145)(463.64433042,47.12572387)(461.78433228,47.12572387)
\curveto(459.48433458,47.12572387)(457.50433218,46.22572131)(457.40433228,43.66572387)
\lineto(459.10433228,43.66572387)
\curveto(459.1843322,45.18572235)(460.24433372,45.62572387)(461.68433228,45.62572387)
\curveto(462.7643312,45.62572387)(463.98433228,45.38572239)(463.98433228,43.90572387)
\curveto(463.98433228,42.62572515)(462.3843304,42.74572351)(460.50433228,42.38572387)
\curveto(458.74433404,42.04572421)(457.00433228,41.54572153)(457.00433228,39.20572387)
\curveto(457.00433228,37.14572593)(458.54433416,36.30572387)(460.42433228,36.30572387)
\curveto(461.86433084,36.30572387)(463.12433322,36.80572497)(464.06433228,37.90572387)
\curveto(464.06433228,36.78572499)(464.62433316,36.30572387)(465.50433228,36.30572387)
\curveto(466.04433174,36.30572387)(466.42433258,36.40572405)(466.72433228,36.58572387)
\lineto(466.72433228,37.90572387)
}
}
{
\newrgbcolor{curcolor}{0 0 0}
\pscustom[linestyle=none,fillstyle=solid,fillcolor=curcolor]
{
\newpath
\moveto(468.40651978,36.54572387)
\lineto(470.10651978,36.54572387)
\lineto(470.10651978,50.82572387)
\lineto(468.40651978,50.82572387)
\lineto(468.40651978,36.54572387)
}
}
{
\newrgbcolor{curcolor}{0 0 0}
\pscustom[linestyle=none,fillstyle=solid,fillcolor=curcolor]
{
\newpath
\moveto(479.89964478,39.82572387)
\curveto(479.59964508,38.48572521)(478.61964338,37.80572387)(477.21964478,37.80572387)
\curveto(474.95964704,37.80572387)(473.93964484,39.40572567)(473.99964478,41.20572387)
\lineto(481.73964478,41.20572387)
\curveto(481.83964468,43.70572137)(480.71964112,47.12572387)(477.05964478,47.12572387)
\curveto(474.2396476,47.12572387)(472.19964478,44.84572077)(472.19964478,41.74572387)
\curveto(472.29964468,38.58572703)(473.85964808,36.30572387)(477.15964478,36.30572387)
\curveto(479.47964246,36.30572387)(481.11964524,37.54572615)(481.57964478,39.82572387)
\lineto(479.89964478,39.82572387)
\moveto(473.99964478,42.70572387)
\curveto(474.11964466,44.28572229)(475.17964656,45.62572387)(476.95964478,45.62572387)
\curveto(478.6396431,45.62572387)(479.85964486,44.32572225)(479.93964478,42.70572387)
\lineto(473.99964478,42.70572387)
}
}
{
\newrgbcolor{curcolor}{0 0 0}
\pscustom[linestyle=none,fillstyle=solid,fillcolor=curcolor]
{
\newpath
\moveto(482.84183228,39.80572387)
\curveto(482.94183218,37.24572643)(484.9018346,36.30572387)(487.22183228,36.30572387)
\curveto(489.32183018,36.30572387)(491.62183228,37.10572633)(491.62183228,39.56572387)
\curveto(491.62183228,41.56572187)(489.94183058,42.12572425)(488.24183228,42.50572387)
\curveto(486.66183386,42.88572349)(484.86183228,43.08572509)(484.86183228,44.30572387)
\curveto(484.86183228,45.34572283)(486.0418333,45.62572387)(487.06183228,45.62572387)
\curveto(488.18183116,45.62572387)(489.3418324,45.20572255)(489.46183228,43.88572387)
\lineto(491.16183228,43.88572387)
\curveto(491.02183242,46.40572135)(489.20183,47.12572387)(486.92183228,47.12572387)
\curveto(485.12183408,47.12572387)(483.06183228,46.26572179)(483.06183228,44.18572387)
\curveto(483.06183228,42.20572585)(484.76183396,41.64572349)(486.44183228,41.26572387)
\curveto(488.14183058,40.88572425)(489.82183228,40.66572255)(489.82183228,39.34572387)
\curveto(489.82183228,38.04572517)(488.38183122,37.80572387)(487.32183228,37.80572387)
\curveto(485.92183368,37.80572387)(484.60183222,38.28572539)(484.54183228,39.80572387)
\lineto(482.84183228,39.80572387)
}
}
{
\newrgbcolor{curcolor}{0 0 0}
\pscustom[linewidth=2.48031497,linecolor=curcolor,linestyle=dashed,dash=2.48031496 9.92125984]
{
\newpath
\moveto(505.93645,100.84821838)
\lineto(30.368497,100.84821838)
}
}
{
\newrgbcolor{curcolor}{0 0 0}
\pscustom[linestyle=none,fillstyle=solid,fillcolor=curcolor]
{
\newpath
\moveto(428.79632602,370.6886327)
\curveto(428.13632668,375.24862814)(424.44632155,377.6486327)(419.97632602,377.6486327)
\curveto(413.37633262,377.6486327)(409.83632602,372.57862649)(409.83632602,366.3686327)
\curveto(409.83632602,360.12863894)(413.07633268,355.2086327)(419.73632602,355.2086327)
\curveto(425.13632062,355.2086327)(428.46632656,358.44863804)(429.00632602,363.7886327)
\lineto(426.15632602,363.7886327)
\curveto(425.88632629,360.24863624)(423.72632227,357.6086327)(419.97632602,357.6086327)
\curveto(414.84633115,357.6086327)(412.68632602,361.68863759)(412.68632602,366.5786327)
\curveto(412.68632602,371.04862823)(414.84633112,375.2486327)(419.94632602,375.2486327)
\curveto(422.91632305,375.2486327)(425.34632662,373.71862967)(425.94632602,370.6886327)
\lineto(428.79632602,370.6886327)
}
}
{
\newrgbcolor{curcolor}{0 0 0}
\pscustom[linestyle=none,fillstyle=solid,fillcolor=curcolor]
{
\newpath
\moveto(441.74601352,361.1186327)
\curveto(441.74601352,359.70863411)(440.36601016,357.6086327)(437.00601352,357.6086327)
\curveto(435.44601508,357.6086327)(434.00601352,358.20863438)(434.00601352,359.8886327)
\curveto(434.00601352,361.77863081)(435.4460152,362.378633)(437.12601352,362.6786327)
\curveto(438.83601181,362.9786324)(440.75601451,363.00863342)(441.74601352,363.7286327)
\lineto(441.74601352,361.1186327)
\moveto(445.88601352,357.7586327)
\curveto(445.55601385,357.63863282)(445.31601331,357.6086327)(445.10601352,357.6086327)
\curveto(444.29601433,357.6086327)(444.29601352,358.1486339)(444.29601352,359.3486327)
\lineto(444.29601352,367.3286327)
\curveto(444.29601352,370.95862907)(441.26601073,371.5886327)(438.47601352,371.5886327)
\curveto(435.02601697,371.5886327)(432.05601337,370.23862886)(431.90601352,366.3986327)
\lineto(434.45601352,366.3986327)
\curveto(434.5760134,368.67863042)(436.16601568,369.3386327)(438.32601352,369.3386327)
\curveto(439.9460119,369.3386327)(441.77601352,368.97863048)(441.77601352,366.7586327)
\curveto(441.77601352,364.83863462)(439.3760107,365.01863216)(436.55601352,364.4786327)
\curveto(433.91601616,363.96863321)(431.30601352,363.21862919)(431.30601352,359.7086327)
\curveto(431.30601352,356.61863579)(433.61601634,355.3586327)(436.43601352,355.3586327)
\curveto(438.59601136,355.3586327)(440.48601493,356.10863435)(441.89601352,357.7586327)
\curveto(441.89601352,356.07863438)(442.73601484,355.3586327)(444.05601352,355.3586327)
\curveto(444.86601271,355.3586327)(445.43601397,355.50863297)(445.88601352,355.7786327)
\lineto(445.88601352,357.7586327)
}
}
{
\newrgbcolor{curcolor}{0 0 0}
\pscustom[linestyle=none,fillstyle=solid,fillcolor=curcolor]
{
\newpath
\moveto(448.16929477,355.7186327)
\lineto(450.71929477,355.7186327)
\lineto(450.71929477,362.6186327)
\curveto(450.71929477,366.54862877)(452.21929888,368.8886327)(456.32929477,368.8886327)
\lineto(456.32929477,371.5886327)
\curveto(453.56929753,371.67863261)(451.85929354,370.44863021)(450.62929477,367.9586327)
\lineto(450.56929477,367.9586327)
\lineto(450.56929477,371.2286327)
\lineto(448.16929477,371.2286327)
\lineto(448.16929477,355.7186327)
}
}
{
\newrgbcolor{curcolor}{0 0 0}
\pscustom[linestyle=none,fillstyle=solid,fillcolor=curcolor]
{
\newpath
\moveto(458.18882602,355.7186327)
\lineto(460.73882602,355.7186327)
\lineto(460.73882602,362.6186327)
\curveto(460.73882602,366.54862877)(462.23883013,368.8886327)(466.34882602,368.8886327)
\lineto(466.34882602,371.5886327)
\curveto(463.58882878,371.67863261)(461.87882479,370.44863021)(460.64882602,367.9586327)
\lineto(460.58882602,367.9586327)
\lineto(460.58882602,371.2286327)
\lineto(458.18882602,371.2286327)
\lineto(458.18882602,355.7186327)
}
}
{
\newrgbcolor{curcolor}{0 0 0}
\pscustom[linestyle=none,fillstyle=solid,fillcolor=curcolor]
{
\newpath
\moveto(479.00835727,360.6386327)
\curveto(478.55835772,358.62863471)(477.08835517,357.6086327)(474.98835727,357.6086327)
\curveto(471.59836066,357.6086327)(470.06835736,360.0086354)(470.15835727,362.7086327)
\lineto(481.76835727,362.7086327)
\curveto(481.91835712,366.45862895)(480.23835178,371.5886327)(474.74835727,371.5886327)
\curveto(470.5183615,371.5886327)(467.45835727,368.16862805)(467.45835727,363.5186327)
\curveto(467.60835712,358.77863744)(469.94836222,355.3586327)(474.89835727,355.3586327)
\curveto(478.37835379,355.3586327)(480.83835796,357.21863612)(481.52835727,360.6386327)
\lineto(479.00835727,360.6386327)
\moveto(470.15835727,364.9586327)
\curveto(470.33835709,367.32863033)(471.92835994,369.3386327)(474.59835727,369.3386327)
\curveto(477.11835475,369.3386327)(478.94835739,367.38863027)(479.06835727,364.9586327)
\lineto(470.15835727,364.9586327)
}
}
{
\newrgbcolor{curcolor}{0 0 0}
\pscustom[linestyle=none,fillstyle=solid,fillcolor=curcolor]
{
\newpath
\moveto(484.32163852,355.7186327)
\lineto(486.87163852,355.7186327)
\lineto(486.87163852,362.6186327)
\curveto(486.87163852,366.54862877)(488.37164263,368.8886327)(492.48163852,368.8886327)
\lineto(492.48163852,371.5886327)
\curveto(489.72164128,371.67863261)(488.01163729,370.44863021)(486.78163852,367.9586327)
\lineto(486.72163852,367.9586327)
\lineto(486.72163852,371.2286327)
\lineto(484.32163852,371.2286327)
\lineto(484.32163852,355.7186327)
}
}
{
\newrgbcolor{curcolor}{0 0 0}
\pscustom[linestyle=none,fillstyle=solid,fillcolor=curcolor]
{
\newpath
\moveto(504.03116977,361.1186327)
\curveto(504.03116977,359.70863411)(502.65116641,357.6086327)(499.29116977,357.6086327)
\curveto(497.73117133,357.6086327)(496.29116977,358.20863438)(496.29116977,359.8886327)
\curveto(496.29116977,361.77863081)(497.73117145,362.378633)(499.41116977,362.6786327)
\curveto(501.12116806,362.9786324)(503.04117076,363.00863342)(504.03116977,363.7286327)
\lineto(504.03116977,361.1186327)
\moveto(508.17116977,357.7586327)
\curveto(507.8411701,357.63863282)(507.60116956,357.6086327)(507.39116977,357.6086327)
\curveto(506.58117058,357.6086327)(506.58116977,358.1486339)(506.58116977,359.3486327)
\lineto(506.58116977,367.3286327)
\curveto(506.58116977,370.95862907)(503.55116698,371.5886327)(500.76116977,371.5886327)
\curveto(497.31117322,371.5886327)(494.34116962,370.23862886)(494.19116977,366.3986327)
\lineto(496.74116977,366.3986327)
\curveto(496.86116965,368.67863042)(498.45117193,369.3386327)(500.61116977,369.3386327)
\curveto(502.23116815,369.3386327)(504.06116977,368.97863048)(504.06116977,366.7586327)
\curveto(504.06116977,364.83863462)(501.66116695,365.01863216)(498.84116977,364.4786327)
\curveto(496.20117241,363.96863321)(493.59116977,363.21862919)(493.59116977,359.7086327)
\curveto(493.59116977,356.61863579)(495.90117259,355.3586327)(498.72116977,355.3586327)
\curveto(500.88116761,355.3586327)(502.77117118,356.10863435)(504.18116977,357.7586327)
\curveto(504.18116977,356.07863438)(505.02117109,355.3586327)(506.34116977,355.3586327)
\curveto(507.15116896,355.3586327)(507.72117022,355.50863297)(508.17116977,355.7786327)
\lineto(508.17116977,357.7586327)
}
}
{
\newrgbcolor{curcolor}{0 0 0}
\pscustom[linewidth=2.65748024,linecolor=curcolor]
{
\newpath
\moveto(400.76647983,392.00745106)
\lineto(517.24104343,392.00745106)
\lineto(517.24104343,340.84976673)
\lineto(400.76647983,340.84976673)
\closepath
}
}
{
\newrgbcolor{curcolor}{0 0 0}
\pscustom[linestyle=none,fillstyle=solid,fillcolor=curcolor]
{
\newpath
\moveto(444.86088428,279.27615453)
\lineto(447.80088428,279.27615453)
\lineto(450.20088428,285.72615453)
\lineto(459.26088428,285.72615453)
\lineto(461.60088428,279.27615453)
\lineto(464.75088428,279.27615453)
\lineto(456.38088428,300.69615452)
\lineto(453.23088428,300.69615452)
\lineto(444.86088428,279.27615453)
\moveto(454.73088428,298.11615453)
\lineto(454.79088428,298.11615453)
\lineto(458.36088428,288.12615453)
\lineto(451.10088428,288.12615453)
\lineto(454.73088428,298.11615453)
\moveto(452.96088428,302.22615453)
\lineto(454.88088428,302.22615453)
\lineto(458.81088428,306.51615453)
\lineto(455.54088428,306.51615453)
\lineto(452.96088428,302.22615453)
}
}
{
\newrgbcolor{curcolor}{0 0 0}
\pscustom[linestyle=none,fillstyle=solid,fillcolor=curcolor]
{
\newpath
\moveto(466.35400928,279.27615453)
\lineto(468.90400928,279.27615453)
\lineto(468.90400928,286.17615453)
\curveto(468.90400928,290.1061506)(470.40401339,292.44615452)(474.51400928,292.44615452)
\lineto(474.51400928,295.14615453)
\curveto(471.75401204,295.23615444)(470.04400805,294.00615204)(468.81400928,291.51615453)
\lineto(468.75400928,291.51615453)
\lineto(468.75400928,294.78615453)
\lineto(466.35400928,294.78615453)
\lineto(466.35400928,279.27615453)
}
}
{
\newrgbcolor{curcolor}{0 0 0}
\pscustom[linestyle=none,fillstyle=solid,fillcolor=curcolor]
{
\newpath
\moveto(487.17354053,284.19615452)
\curveto(486.72354098,282.18615654)(485.25353843,281.16615453)(483.15354053,281.16615453)
\curveto(479.76354392,281.16615453)(478.23354062,283.56615722)(478.32354053,286.26615453)
\lineto(489.93354053,286.26615453)
\curveto(490.08354038,290.01615078)(488.40353504,295.14615453)(482.91354053,295.14615453)
\curveto(478.68354476,295.14615453)(475.62354053,291.72614988)(475.62354053,287.07615452)
\curveto(475.77354038,282.33615926)(478.11354548,278.91615453)(483.06354053,278.91615453)
\curveto(486.54353705,278.91615453)(489.00354122,280.77615795)(489.69354053,284.19615452)
\lineto(487.17354053,284.19615452)
\moveto(478.32354053,288.51615453)
\curveto(478.50354035,290.88615215)(480.0935432,292.89615453)(482.76354053,292.89615453)
\curveto(485.28353801,292.89615453)(487.11354065,290.9461521)(487.23354053,288.51615453)
\lineto(478.32354053,288.51615453)
}
}
{
\newrgbcolor{curcolor}{0 0 0}
\pscustom[linestyle=none,fillstyle=solid,fillcolor=curcolor]
{
\newpath
\moveto(502.17682178,284.67615453)
\curveto(502.17682178,283.26615594)(500.79681842,281.16615453)(497.43682178,281.16615453)
\curveto(495.87682334,281.16615453)(494.43682178,281.76615621)(494.43682178,283.44615452)
\curveto(494.43682178,285.33615264)(495.87682346,285.93615483)(497.55682178,286.23615453)
\curveto(499.26682007,286.53615423)(501.18682277,286.56615525)(502.17682178,287.28615453)
\lineto(502.17682178,284.67615453)
\moveto(506.31682178,281.31615453)
\curveto(505.98682211,281.19615465)(505.74682157,281.16615453)(505.53682178,281.16615453)
\curveto(504.72682259,281.16615453)(504.72682178,281.70615573)(504.72682178,282.90615453)
\lineto(504.72682178,290.88615453)
\curveto(504.72682178,294.5161509)(501.69681899,295.14615453)(498.90682178,295.14615453)
\curveto(495.45682523,295.14615453)(492.48682163,293.79615069)(492.33682178,289.95615452)
\lineto(494.88682178,289.95615452)
\curveto(495.00682166,292.23615225)(496.59682394,292.89615453)(498.75682178,292.89615453)
\curveto(500.37682016,292.89615453)(502.20682178,292.53615231)(502.20682178,290.31615453)
\curveto(502.20682178,288.39615645)(499.80681896,288.57615399)(496.98682178,288.03615453)
\curveto(494.34682442,287.52615503)(491.73682178,286.77615101)(491.73682178,283.26615453)
\curveto(491.73682178,280.17615762)(494.0468246,278.91615453)(496.86682178,278.91615453)
\curveto(499.02681962,278.91615453)(500.91682319,279.66615618)(502.32682178,281.31615453)
\curveto(502.32682178,279.63615621)(503.1668231,278.91615453)(504.48682178,278.91615453)
\curveto(505.29682097,278.91615453)(505.86682223,279.06615479)(506.31682178,279.33615452)
\lineto(506.31682178,281.31615453)
}
}
{
\newrgbcolor{curcolor}{0 0 0}
\pscustom[linewidth=2.65748024,linecolor=curcolor]
{
\newpath
\moveto(433.93666797,318.49563695)
\lineto(517.24104267,318.49563695)
\lineto(517.24104267,266.93671132)
\lineto(433.93666797,266.93671132)
\closepath
}
}
{
\newrgbcolor{curcolor}{0 0 0}
\pscustom[linestyle=none,fillstyle=solid,fillcolor=curcolor]
{
\newpath
\moveto(373.17144775,223.54569213)
\curveto(372.51144841,228.10568757)(368.82144328,230.50569213)(364.35144775,230.50569213)
\curveto(357.75145435,230.50569213)(354.21144775,225.43568592)(354.21144775,219.22569213)
\curveto(354.21144775,212.98569837)(357.45145441,208.06569213)(364.11144775,208.06569213)
\curveto(369.51144235,208.06569213)(372.84144829,211.30569747)(373.38144775,216.64569213)
\lineto(370.53144775,216.64569213)
\curveto(370.26144802,213.10569567)(368.101444,210.46569213)(364.35144775,210.46569213)
\curveto(359.22145288,210.46569213)(357.06144775,214.54569702)(357.06144775,219.43569213)
\curveto(357.06144775,223.90568766)(359.22145285,228.10569213)(364.32144775,228.10569213)
\curveto(367.29144478,228.10569213)(369.72144835,226.5756891)(370.32144775,223.54569213)
\lineto(373.17144775,223.54569213)
}
}
{
\newrgbcolor{curcolor}{0 0 0}
\pscustom[linestyle=none,fillstyle=solid,fillcolor=curcolor]
{
\newpath
\moveto(375.68113525,216.31569213)
\curveto(375.68113525,211.78569666)(378.29114017,208.21569213)(383.21113525,208.21569213)
\curveto(388.13113033,208.21569213)(390.74113525,211.78569666)(390.74113525,216.31569213)
\curveto(390.74113525,220.87568757)(388.13113033,224.44569213)(383.21113525,224.44569213)
\curveto(378.29114017,224.44569213)(375.68113525,220.87568757)(375.68113525,216.31569213)
\moveto(378.38113525,216.31569213)
\curveto(378.38113525,220.09568835)(380.54113792,222.19569213)(383.21113525,222.19569213)
\curveto(385.88113258,222.19569213)(388.04113525,220.09568835)(388.04113525,216.31569213)
\curveto(388.04113525,212.56569588)(385.88113258,210.46569213)(383.21113525,210.46569213)
\curveto(380.54113792,210.46569213)(378.38113525,212.56569588)(378.38113525,216.31569213)
}
}
{
\newrgbcolor{curcolor}{0 0 0}
\pscustom[linestyle=none,fillstyle=solid,fillcolor=curcolor]
{
\newpath
\moveto(393.74769775,208.57569213)
\lineto(396.29769775,208.57569213)
\lineto(396.29769775,218.23569213)
\curveto(396.29769775,219.40569096)(397.40770078,222.19569213)(400.43769775,222.19569213)
\curveto(402.71769547,222.19569213)(403.34769775,220.75569012)(403.34769775,218.74569213)
\lineto(403.34769775,208.57569213)
\lineto(405.89769775,208.57569213)
\lineto(405.89769775,218.23569213)
\curveto(405.89769775,220.63568973)(407.48770021,222.19569213)(409.94769775,222.19569213)
\curveto(412.43769526,222.19569213)(412.94769775,220.66569021)(412.94769775,218.74569213)
\lineto(412.94769775,208.57569213)
\lineto(415.49769775,208.57569213)
\lineto(415.49769775,219.94569213)
\curveto(415.49769775,223.15568892)(413.42769466,224.44569213)(410.33769775,224.44569213)
\curveto(408.35769973,224.44569213)(406.5276967,223.45569048)(405.47769775,221.80569213)
\curveto(404.84769838,223.69569024)(403.10769586,224.44569213)(401.21769775,224.44569213)
\curveto(399.08769988,224.44569213)(397.34769661,223.54569039)(396.20769775,221.80569213)
\lineto(396.14769775,221.80569213)
\lineto(396.14769775,224.08569213)
\lineto(393.74769775,224.08569213)
\lineto(393.74769775,208.57569213)
}
}
{
\newrgbcolor{curcolor}{0 0 0}
\pscustom[linestyle=none,fillstyle=solid,fillcolor=curcolor]
{
\newpath
\moveto(432.1931665,224.08569213)
\lineto(429.6431665,224.08569213)
\lineto(429.6431665,215.32569213)
\curveto(429.6431665,212.53569492)(428.14316341,210.46569213)(425.0531665,210.46569213)
\curveto(423.10316845,210.46569213)(421.9031665,211.69569402)(421.9031665,213.58569213)
\lineto(421.9031665,224.08569213)
\lineto(419.3531665,224.08569213)
\lineto(419.3531665,213.88569213)
\curveto(419.3531665,210.55569546)(420.61317058,208.21569213)(424.6931665,208.21569213)
\curveto(426.91316428,208.21569213)(428.65316758,209.11569405)(429.7331665,211.03569213)
\lineto(429.7931665,211.03569213)
\lineto(429.7931665,208.57569213)
\lineto(432.1931665,208.57569213)
\lineto(432.1931665,224.08569213)
}
}
{
\newrgbcolor{curcolor}{0 0 0}
\pscustom[linestyle=none,fillstyle=solid,fillcolor=curcolor]
{
\newpath
\moveto(436.05238525,208.57569213)
\lineto(438.60238525,208.57569213)
\lineto(438.60238525,217.33569213)
\curveto(438.60238525,220.12568934)(440.10238834,222.19569213)(443.19238525,222.19569213)
\curveto(445.1423833,222.19569213)(446.34238525,220.96569024)(446.34238525,219.07569213)
\lineto(446.34238525,208.57569213)
\lineto(448.89238525,208.57569213)
\lineto(448.89238525,218.77569213)
\curveto(448.89238525,222.1056888)(447.63238117,224.44569213)(443.55238525,224.44569213)
\curveto(441.33238747,224.44569213)(439.59238417,223.54569021)(438.51238525,221.62569213)
\lineto(438.45238525,221.62569213)
\lineto(438.45238525,224.08569213)
\lineto(436.05238525,224.08569213)
\lineto(436.05238525,208.57569213)
}
}
{
\newrgbcolor{curcolor}{0 0 0}
\pscustom[linestyle=none,fillstyle=solid,fillcolor=curcolor]
{
\newpath
\moveto(452.901604,208.57569213)
\lineto(455.451604,208.57569213)
\lineto(455.451604,224.08569213)
\lineto(452.901604,224.08569213)
\lineto(452.901604,208.57569213)
\moveto(455.451604,229.99569213)
\lineto(452.901604,229.99569213)
\lineto(452.901604,226.87569213)
\lineto(455.451604,226.87569213)
\lineto(455.451604,229.99569213)
}
}
{
\newrgbcolor{curcolor}{0 0 0}
\pscustom[linestyle=none,fillstyle=solid,fillcolor=curcolor]
{
\newpath
\moveto(473.2912915,229.99569213)
\lineto(470.7412915,229.99569213)
\lineto(470.7412915,222.01569213)
\lineto(470.6812915,222.01569213)
\curveto(469.54129264,223.8456903)(467.26128991,224.44569213)(465.6712915,224.44569213)
\curveto(460.93129624,224.44569213)(458.5912915,220.75568775)(458.5912915,216.37569213)
\curveto(458.5912915,211.99569651)(460.96129627,208.21569213)(465.7312915,208.21569213)
\curveto(467.86128937,208.21569213)(469.84129234,208.96569384)(470.6812915,210.67569213)
\lineto(470.7412915,210.67569213)
\lineto(470.7412915,208.57569213)
\lineto(473.2912915,208.57569213)
\lineto(473.2912915,229.99569213)
\moveto(461.2912915,216.19569213)
\curveto(461.2912915,219.2256891)(462.46129498,222.19569213)(465.9412915,222.19569213)
\curveto(469.45128799,222.19569213)(470.8312915,219.37568907)(470.8312915,216.31569213)
\curveto(470.8312915,213.40569504)(469.51128808,210.46569213)(466.0912915,210.46569213)
\curveto(462.7912948,210.46569213)(461.2912915,213.28569504)(461.2912915,216.19569213)
}
}
{
\newrgbcolor{curcolor}{0 0 0}
\pscustom[linestyle=none,fillstyle=solid,fillcolor=curcolor]
{
\newpath
\moveto(486.8437915,213.97569213)
\curveto(486.8437915,212.56569354)(485.46378814,210.46569213)(482.1037915,210.46569213)
\curveto(480.54379306,210.46569213)(479.1037915,211.06569381)(479.1037915,212.74569213)
\curveto(479.1037915,214.63569024)(480.54379318,215.23569243)(482.2237915,215.53569213)
\curveto(483.93378979,215.83569183)(485.85379249,215.86569285)(486.8437915,216.58569213)
\lineto(486.8437915,213.97569213)
\moveto(490.9837915,210.61569213)
\curveto(490.65379183,210.49569225)(490.41379129,210.46569213)(490.2037915,210.46569213)
\curveto(489.39379231,210.46569213)(489.3937915,211.00569333)(489.3937915,212.20569213)
\lineto(489.3937915,220.18569213)
\curveto(489.3937915,223.8156885)(486.36378871,224.44569213)(483.5737915,224.44569213)
\curveto(480.12379495,224.44569213)(477.15379135,223.09568829)(477.0037915,219.25569213)
\lineto(479.5537915,219.25569213)
\curveto(479.67379138,221.53568985)(481.26379366,222.19569213)(483.4237915,222.19569213)
\curveto(485.04378988,222.19569213)(486.8737915,221.83568991)(486.8737915,219.61569213)
\curveto(486.8737915,217.69569405)(484.47378868,217.87569159)(481.6537915,217.33569213)
\curveto(479.01379414,216.82569264)(476.4037915,216.07568862)(476.4037915,212.56569213)
\curveto(476.4037915,209.47569522)(478.71379432,208.21569213)(481.5337915,208.21569213)
\curveto(483.69378934,208.21569213)(485.58379291,208.96569378)(486.9937915,210.61569213)
\curveto(486.9937915,208.93569381)(487.83379282,208.21569213)(489.1537915,208.21569213)
\curveto(489.96379069,208.21569213)(490.53379195,208.3656924)(490.9837915,208.63569213)
\lineto(490.9837915,210.61569213)
}
}
{
\newrgbcolor{curcolor}{0 0 0}
\pscustom[linestyle=none,fillstyle=solid,fillcolor=curcolor]
{
\newpath
\moveto(507.21707275,229.99569213)
\lineto(504.66707275,229.99569213)
\lineto(504.66707275,222.01569213)
\lineto(504.60707275,222.01569213)
\curveto(503.46707389,223.8456903)(501.18707116,224.44569213)(499.59707275,224.44569213)
\curveto(494.85707749,224.44569213)(492.51707275,220.75568775)(492.51707275,216.37569213)
\curveto(492.51707275,211.99569651)(494.88707752,208.21569213)(499.65707275,208.21569213)
\curveto(501.78707062,208.21569213)(503.76707359,208.96569384)(504.60707275,210.67569213)
\lineto(504.66707275,210.67569213)
\lineto(504.66707275,208.57569213)
\lineto(507.21707275,208.57569213)
\lineto(507.21707275,229.99569213)
\moveto(495.21707275,216.19569213)
\curveto(495.21707275,219.2256891)(496.38707623,222.19569213)(499.86707275,222.19569213)
\curveto(503.37706924,222.19569213)(504.75707275,219.37568907)(504.75707275,216.31569213)
\curveto(504.75707275,213.40569504)(503.43706933,210.46569213)(500.01707275,210.46569213)
\curveto(496.71707605,210.46569213)(495.21707275,213.28569504)(495.21707275,216.19569213)
}
}
{
\newrgbcolor{curcolor}{0 0 0}
\pscustom[linewidth=2.65748024,linecolor=curcolor]
{
\newpath
\moveto(344.1875,244.5825891)
\lineto(517.24104309,244.5825891)
\lineto(517.24104309,193.98877425)
\lineto(344.1875,193.98877425)
\closepath
}
}
{
\newrgbcolor{curcolor}{0 0 0}
\pscustom[linewidth=2.65748024,linecolor=curcolor]
{
\newpath
\moveto(375,366.42857138)
\lineto(170.71429,366.42857138)
}
}
{
\newrgbcolor{curcolor}{0 0 0}
\pscustom[linestyle=none,fillstyle=solid,fillcolor=curcolor]
{
\newpath
\moveto(184.61597488,359.99682843)
\lineto(167.19479306,366.40303843)
\lineto(184.61597584,372.80924702)
\curveto(181.83279933,369.02702404)(181.84883606,363.85228657)(184.61597488,359.99682843)
\closepath
}
}
{
\newrgbcolor{curcolor}{0 0 0}
\pscustom[linewidth=2.65748024,linecolor=curcolor]
{
\newpath
\moveto(412.85714,305.71427838)
\lineto(220.71429,305.71427838)
}
}
{
\newrgbcolor{curcolor}{0 0 0}
\pscustom[linestyle=none,fillstyle=solid,fillcolor=curcolor]
{
\newpath
\moveto(234.61597488,299.28253543)
\lineto(217.19479306,305.68874543)
\lineto(234.61597584,312.09495402)
\curveto(231.83279933,308.31273104)(231.84883606,303.13799357)(234.61597488,299.28253543)
\closepath
}
}
\end{pspicture}

\caption{Espacios virtuales que se han de contruir y su clasificación en el
sistema}
\label{espacios}
\end{figure}

Cada uno de ellos posee la funcionalidad caracteristica de un recurso
administrable, es decir, posee las siguientes operaciones\footnote{Para una
definición exacta puede consultarse:
https://es.wikipedia.org/wiki/Create,\_read,\_update\_and\_delete}:

\begin{itemize}
\item Crear un nuevo elemento (CREATE).
\item Visualizar el elemento a detalle (READ).
\item Editar las caracteristicas del elemento (UPDATE).
\item Eliminación del elemento (DELETE).
\end{itemize}

A su vez se han establecido un conjunto de tareas por lote para facilitar la
correcta manipulación de amplios volumenes de información. Estas tareas son:

\begin{itemize}
\item Importación de datos desde un archivo CSV.
\item Exportación de datos hacia un archivo CSV.
\item Habilitación/Inhabilitación de elementos, ya sea individualmente o en
      grupos de elementos.
\end{itemize}

\subsection{Intercambio de recursos}

Cada espacio virtual debe poseer la capacidad de contener información en
distintos formatos, y para diversos propositos. El objetivo principal es poder
compartir piezas de información entre usuarios del sistema.

Para este proposito, se han definido piezas atomicas de información básica,
estas son:

\begin{description}
\item [Notas] Son piezas de texto que no poseen formato, y representan la unidad
de información mas básica que ha de construirse.
\item [Archivos] Los archivos representan recursos que los usuarios suben al
sistema, y no esta contemplado ninguna tarea adicional, a parte de alojarlos y
brindar la capacidad de ser descargados por otros usuarios.
\item [Imagenes] Una imágen es la unica pieza provista para representación
visual en el sistema. Esta adicionalmente a ser subida por un usuario debe
poder ser visualizada y descargada por parte de los demas usuarios.
\item [Videos] Inicialmente los videos representan archivos en el formato flv,
que puedan ser reproducibles en un player de adobe flash.
\item [Eventos] Los eventos son piezas que demarcan la iniciación y duración de
una actividad, estas pueden ser creadas por algún usuario y visualizados por
otros usuarios, según el espacio virtual en el que se encuentre.
\item [Enlaces] Inicialmente se contempla unicamente la publicación de enlaces,
sin analisis del lugar a donde conducen, a la larga se plantea la posibilidad de
analizar el recurso destino y poder reenderizarse la información segun tal
inspecccióni (es deseable, pero no esta contemplado en los alcances de este
proyecto).
\end{description}

Todos estos tipos de recursos poseen tambien caracteristicas de espacio virtual,
es decir, que cada una de ellas posee operaciones CRUD, ademas de
funcionalidades para el fomento a la participación.

\subsection{Instancias multiples}

Se considera la creación de feeds de
sindicación\footnote{Definición disponible en:
http://es.wikipedia.org/wiki/Redifusión\_web}, para ser exactos se utilizará
el formato RSS.

\subsection{Canales de comunicación}

Para la mejora de los canales de comunicación se ha definido el manejo de otros
tipos de espacios-recursos adicionales que poseen diferentes propositos
utilitarios, estos son:

\begin{description}
\item [Usuarios] Para incrementar la afinidad de los usuarios hacia el sistema
que conforma este proyecto, se han definido la construcción de espacios propios
para cada usuario. De modo que este pueda controlar los recursos que produzca y
que sean realmente suyos.
\item [Roles] Un rol define el tipo de participación que puede poseer un usuario
en el sistema, inicialmente se han creado un conjunto de roles, que esta acordes
a la logica del contexto de implantación (UMSS):
    \begin{itemize}
    \item Administrador
    \item Desarrollador
    \item Moderador
    \item Docente
    \item Auxiliar
    \item Estudiante
    \item Invitado
    \end{itemize}
\item [Contactos] La caracteristica mas propia de una red social esta basada en
la creación de vinculos entre usuarios del sistema, para esto se ha creado una
cadena de contactos, estos vinculos pueden ser de tres tipos (estos estan
basados en la forma que son manejados por la red social twitter) (figura
\ref{contactos}):
    \begin{description}
    \item [Follower] Representa una relación uni-direccional, de un usuario que
    ve los recursos que produce otro usuario.
    \item [Following] Representa una relación uni-direccional, de un usuario que
    produce los recursos que otros usuarios pueden ver.
    \item [Friend] Representa una relación bi-direccional, entre dos usuarios,
    que comparten los recursos que producen.
    \end{description}
    \begin{figure}
    \centering
    %LaTeX with PSTricks extensions
%%Creator: inkscape 0.48.4
%%Please note this file requires PSTricks extensions
\psset{xunit=.5pt,yunit=.5pt,runit=.5pt}
\begin{pspicture}(531.49603271,248.03149414)
{
\newrgbcolor{curcolor}{0 0 0}
\pscustom[linestyle=none,fillstyle=solid,fillcolor=curcolor]
{
\newpath
\moveto(48.30160622,53.29359723)
\lineto(45.45160622,53.29359723)
\lineto(45.45160622,39.61359723)
\curveto(45.45160622,35.83360101)(43.47160262,33.76359723)(39.87160622,33.76359723)
\curveto(36.09161,33.76359723)(33.93160622,35.83360101)(33.93160622,39.61359723)
\lineto(33.93160622,53.29359723)
\lineto(31.08160622,53.29359723)
\lineto(31.08160622,39.61359723)
\curveto(31.08160622,33.91360293)(34.35161174,31.36359723)(39.87160622,31.36359723)
\curveto(45.21160088,31.36359723)(48.30160622,34.21360263)(48.30160622,39.61359723)
\lineto(48.30160622,53.29359723)
}
}
{
\newrgbcolor{curcolor}{0 0 0}
\pscustom[linestyle=none,fillstyle=solid,fillcolor=curcolor]
{
\newpath
\moveto(51.47129372,36.76359723)
\curveto(51.62129357,32.92360107)(54.5612972,31.51359723)(58.04129372,31.51359723)
\curveto(61.19129057,31.51359723)(64.64129372,32.71360092)(64.64129372,36.40359723)
\curveto(64.64129372,39.40359423)(62.12129117,40.2435978)(59.57129372,40.81359723)
\curveto(57.20129609,41.38359666)(54.50129372,41.68359906)(54.50129372,43.51359723)
\curveto(54.50129372,45.07359567)(56.27129525,45.49359723)(57.80129372,45.49359723)
\curveto(59.48129204,45.49359723)(61.2212939,44.86359525)(61.40129372,42.88359723)
\lineto(63.95129372,42.88359723)
\curveto(63.74129393,46.66359345)(61.0112903,47.74359723)(57.59129372,47.74359723)
\curveto(54.89129642,47.74359723)(51.80129372,46.45359411)(51.80129372,43.33359723)
\curveto(51.80129372,40.3636002)(54.35129624,39.52359666)(56.87129372,38.95359723)
\curveto(59.42129117,38.3835978)(61.94129372,38.05359525)(61.94129372,36.07359723)
\curveto(61.94129372,34.12359918)(59.78129213,33.76359723)(58.19129372,33.76359723)
\curveto(56.09129582,33.76359723)(54.11129363,34.48359951)(54.02129372,36.76359723)
\lineto(51.47129372,36.76359723)
}
}
{
\newrgbcolor{curcolor}{0 0 0}
\pscustom[linestyle=none,fillstyle=solid,fillcolor=curcolor]
{
\newpath
\moveto(80.30129372,47.38359723)
\lineto(77.75129372,47.38359723)
\lineto(77.75129372,38.62359723)
\curveto(77.75129372,35.83360002)(76.25129063,33.76359723)(73.16129372,33.76359723)
\curveto(71.21129567,33.76359723)(70.01129372,34.99359912)(70.01129372,36.88359723)
\lineto(70.01129372,47.38359723)
\lineto(67.46129372,47.38359723)
\lineto(67.46129372,37.18359723)
\curveto(67.46129372,33.85360056)(68.7212978,31.51359723)(72.80129372,31.51359723)
\curveto(75.0212915,31.51359723)(76.7612948,32.41359915)(77.84129372,34.33359723)
\lineto(77.90129372,34.33359723)
\lineto(77.90129372,31.87359723)
\lineto(80.30129372,31.87359723)
\lineto(80.30129372,47.38359723)
}
}
{
\newrgbcolor{curcolor}{0 0 0}
\pscustom[linestyle=none,fillstyle=solid,fillcolor=curcolor]
{
\newpath
\moveto(93.76051247,37.27359723)
\curveto(93.76051247,35.86359864)(92.38050911,33.76359723)(89.02051247,33.76359723)
\curveto(87.46051403,33.76359723)(86.02051247,34.36359891)(86.02051247,36.04359723)
\curveto(86.02051247,37.93359534)(87.46051415,38.53359753)(89.14051247,38.83359723)
\curveto(90.85051076,39.13359693)(92.77051346,39.16359795)(93.76051247,39.88359723)
\lineto(93.76051247,37.27359723)
\moveto(97.90051247,33.91359723)
\curveto(97.5705128,33.79359735)(97.33051226,33.76359723)(97.12051247,33.76359723)
\curveto(96.31051328,33.76359723)(96.31051247,34.30359843)(96.31051247,35.50359723)
\lineto(96.31051247,43.48359723)
\curveto(96.31051247,47.1135936)(93.28050968,47.74359723)(90.49051247,47.74359723)
\curveto(87.04051592,47.74359723)(84.07051232,46.39359339)(83.92051247,42.55359723)
\lineto(86.47051247,42.55359723)
\curveto(86.59051235,44.83359495)(88.18051463,45.49359723)(90.34051247,45.49359723)
\curveto(91.96051085,45.49359723)(93.79051247,45.13359501)(93.79051247,42.91359723)
\curveto(93.79051247,40.99359915)(91.39050965,41.17359669)(88.57051247,40.63359723)
\curveto(85.93051511,40.12359774)(83.32051247,39.37359372)(83.32051247,35.86359723)
\curveto(83.32051247,32.77360032)(85.63051529,31.51359723)(88.45051247,31.51359723)
\curveto(90.61051031,31.51359723)(92.50051388,32.26359888)(93.91051247,33.91359723)
\curveto(93.91051247,32.23359891)(94.75051379,31.51359723)(96.07051247,31.51359723)
\curveto(96.88051166,31.51359723)(97.45051292,31.6635975)(97.90051247,31.93359723)
\lineto(97.90051247,33.91359723)
}
}
{
\newrgbcolor{curcolor}{0 0 0}
\pscustom[linestyle=none,fillstyle=solid,fillcolor=curcolor]
{
\newpath
\moveto(100.18379372,31.87359723)
\lineto(102.73379372,31.87359723)
\lineto(102.73379372,38.77359723)
\curveto(102.73379372,42.7035933)(104.23379783,45.04359723)(108.34379372,45.04359723)
\lineto(108.34379372,47.74359723)
\curveto(105.58379648,47.83359714)(103.87379249,46.60359474)(102.64379372,44.11359723)
\lineto(102.58379372,44.11359723)
\lineto(102.58379372,47.38359723)
\lineto(100.18379372,47.38359723)
\lineto(100.18379372,31.87359723)
}
}
{
\newrgbcolor{curcolor}{0 0 0}
\pscustom[linestyle=none,fillstyle=solid,fillcolor=curcolor]
{
\newpath
\moveto(110.44332497,31.87359723)
\lineto(112.99332497,31.87359723)
\lineto(112.99332497,47.38359723)
\lineto(110.44332497,47.38359723)
\lineto(110.44332497,31.87359723)
\moveto(112.99332497,53.29359723)
\lineto(110.44332497,53.29359723)
\lineto(110.44332497,50.17359723)
\lineto(112.99332497,50.17359723)
\lineto(112.99332497,53.29359723)
}
}
{
\newrgbcolor{curcolor}{0 0 0}
\pscustom[linestyle=none,fillstyle=solid,fillcolor=curcolor]
{
\newpath
\moveto(116.13301247,39.61359723)
\curveto(116.13301247,35.08360176)(118.74301739,31.51359723)(123.66301247,31.51359723)
\curveto(128.58300755,31.51359723)(131.19301247,35.08360176)(131.19301247,39.61359723)
\curveto(131.19301247,44.17359267)(128.58300755,47.74359723)(123.66301247,47.74359723)
\curveto(118.74301739,47.74359723)(116.13301247,44.17359267)(116.13301247,39.61359723)
\moveto(118.83301247,39.61359723)
\curveto(118.83301247,43.39359345)(120.99301514,45.49359723)(123.66301247,45.49359723)
\curveto(126.3330098,45.49359723)(128.49301247,43.39359345)(128.49301247,39.61359723)
\curveto(128.49301247,35.86360098)(126.3330098,33.76359723)(123.66301247,33.76359723)
\curveto(120.99301514,33.76359723)(118.83301247,35.86360098)(118.83301247,39.61359723)
}
}
{
\newrgbcolor{curcolor}{0 0 0}
\pscustom[linestyle=none,fillstyle=solid,fillcolor=curcolor]
{
}
}
{
\newrgbcolor{curcolor}{0 0 0}
\pscustom[linestyle=none,fillstyle=solid,fillcolor=curcolor]
{
\newpath
\moveto(151.27988747,53.14359723)
\lineto(149.32988747,53.14359723)
\curveto(148.75988804,49.93360044)(146.11988456,49.15359723)(143.20988747,49.15359723)
\lineto(143.20988747,47.11359723)
\lineto(148.72988747,47.11359723)
\lineto(148.72988747,31.87359723)
\lineto(151.27988747,31.87359723)
\lineto(151.27988747,53.14359723)
}
}
{
\newrgbcolor{curcolor}{0 0 0}
\pscustom[linewidth=2.65748024,linecolor=curcolor]
{
\newpath
\moveto(91.1807452,180.25557914)
\lineto(91.1807452,115.64372914)
}
}
{
\newrgbcolor{curcolor}{1 1 1}
\pscustom[linestyle=none,fillstyle=solid,fillcolor=curcolor]
{
\newpath
\moveto(114.0378901,192.12487159)
\curveto(114.0378901,179.30397606)(103.64450073,168.91058669)(90.8236052,168.91058669)
\curveto(78.00270967,168.91058669)(67.6093203,179.30397606)(67.6093203,192.12487159)
\curveto(67.6093203,204.94576711)(78.00270967,215.33915648)(90.8236052,215.33915648)
\curveto(103.63037851,215.33915648)(114.01790889,204.96779736)(114.03786192,192.16103959)
}
}
{
\newrgbcolor{curcolor}{0 0 0}
\pscustom[linewidth=2.65748024,linecolor=curcolor]
{
\newpath
\moveto(114.0378901,192.12487159)
\curveto(114.0378901,179.30397606)(103.64450073,168.91058669)(90.8236052,168.91058669)
\curveto(78.00270967,168.91058669)(67.6093203,179.30397606)(67.6093203,192.12487159)
\curveto(67.6093203,204.94576711)(78.00270967,215.33915648)(90.8236052,215.33915648)
\curveto(103.63037851,215.33915648)(114.01790889,204.96779736)(114.03786192,192.16103959)
}
}
{
\newrgbcolor{curcolor}{0 0 0}
\pscustom[linewidth=2.65748024,linecolor=curcolor]
{
\newpath
\moveto(70.0480442,79.26156914)
\lineto(91.3286852,117.51609914)
\lineto(111.4153452,79.26156914)
}
}
{
\newrgbcolor{curcolor}{0 0 0}
\pscustom[linewidth=2.65748024,linecolor=curcolor]
{
\newpath
\moveto(58.6807482,155.08045914)
\lineto(123.6807452,155.08045914)
}
}
{
\newrgbcolor{curcolor}{0 0 0}
\pscustom[linestyle=none,fillstyle=solid,fillcolor=curcolor]
{
\newpath
\moveto(397.43614522,53.29359723)
\lineto(394.58614522,53.29359723)
\lineto(394.58614522,39.61359723)
\curveto(394.58614522,35.83360101)(392.60614162,33.76359723)(389.00614522,33.76359723)
\curveto(385.226149,33.76359723)(383.06614522,35.83360101)(383.06614522,39.61359723)
\lineto(383.06614522,53.29359723)
\lineto(380.21614522,53.29359723)
\lineto(380.21614522,39.61359723)
\curveto(380.21614522,33.91360293)(383.48615074,31.36359723)(389.00614522,31.36359723)
\curveto(394.34613988,31.36359723)(397.43614522,34.21360263)(397.43614522,39.61359723)
\lineto(397.43614522,53.29359723)
}
}
{
\newrgbcolor{curcolor}{0 0 0}
\pscustom[linestyle=none,fillstyle=solid,fillcolor=curcolor]
{
\newpath
\moveto(400.60583272,36.76359723)
\curveto(400.75583257,32.92360107)(403.6958362,31.51359723)(407.17583272,31.51359723)
\curveto(410.32582957,31.51359723)(413.77583272,32.71360092)(413.77583272,36.40359723)
\curveto(413.77583272,39.40359423)(411.25583017,40.2435978)(408.70583272,40.81359723)
\curveto(406.33583509,41.38359666)(403.63583272,41.68359906)(403.63583272,43.51359723)
\curveto(403.63583272,45.07359567)(405.40583425,45.49359723)(406.93583272,45.49359723)
\curveto(408.61583104,45.49359723)(410.3558329,44.86359525)(410.53583272,42.88359723)
\lineto(413.08583272,42.88359723)
\curveto(412.87583293,46.66359345)(410.1458293,47.74359723)(406.72583272,47.74359723)
\curveto(404.02583542,47.74359723)(400.93583272,46.45359411)(400.93583272,43.33359723)
\curveto(400.93583272,40.3636002)(403.48583524,39.52359666)(406.00583272,38.95359723)
\curveto(408.55583017,38.3835978)(411.07583272,38.05359525)(411.07583272,36.07359723)
\curveto(411.07583272,34.12359918)(408.91583113,33.76359723)(407.32583272,33.76359723)
\curveto(405.22583482,33.76359723)(403.24583263,34.48359951)(403.15583272,36.76359723)
\lineto(400.60583272,36.76359723)
}
}
{
\newrgbcolor{curcolor}{0 0 0}
\pscustom[linestyle=none,fillstyle=solid,fillcolor=curcolor]
{
\newpath
\moveto(429.43583272,47.38359723)
\lineto(426.88583272,47.38359723)
\lineto(426.88583272,38.62359723)
\curveto(426.88583272,35.83360002)(425.38582963,33.76359723)(422.29583272,33.76359723)
\curveto(420.34583467,33.76359723)(419.14583272,34.99359912)(419.14583272,36.88359723)
\lineto(419.14583272,47.38359723)
\lineto(416.59583272,47.38359723)
\lineto(416.59583272,37.18359723)
\curveto(416.59583272,33.85360056)(417.8558368,31.51359723)(421.93583272,31.51359723)
\curveto(424.1558305,31.51359723)(425.8958338,32.41359915)(426.97583272,34.33359723)
\lineto(427.03583272,34.33359723)
\lineto(427.03583272,31.87359723)
\lineto(429.43583272,31.87359723)
\lineto(429.43583272,47.38359723)
}
}
{
\newrgbcolor{curcolor}{0 0 0}
\pscustom[linestyle=none,fillstyle=solid,fillcolor=curcolor]
{
\newpath
\moveto(442.89505147,37.27359723)
\curveto(442.89505147,35.86359864)(441.51504811,33.76359723)(438.15505147,33.76359723)
\curveto(436.59505303,33.76359723)(435.15505147,34.36359891)(435.15505147,36.04359723)
\curveto(435.15505147,37.93359534)(436.59505315,38.53359753)(438.27505147,38.83359723)
\curveto(439.98504976,39.13359693)(441.90505246,39.16359795)(442.89505147,39.88359723)
\lineto(442.89505147,37.27359723)
\moveto(447.03505147,33.91359723)
\curveto(446.7050518,33.79359735)(446.46505126,33.76359723)(446.25505147,33.76359723)
\curveto(445.44505228,33.76359723)(445.44505147,34.30359843)(445.44505147,35.50359723)
\lineto(445.44505147,43.48359723)
\curveto(445.44505147,47.1135936)(442.41504868,47.74359723)(439.62505147,47.74359723)
\curveto(436.17505492,47.74359723)(433.20505132,46.39359339)(433.05505147,42.55359723)
\lineto(435.60505147,42.55359723)
\curveto(435.72505135,44.83359495)(437.31505363,45.49359723)(439.47505147,45.49359723)
\curveto(441.09504985,45.49359723)(442.92505147,45.13359501)(442.92505147,42.91359723)
\curveto(442.92505147,40.99359915)(440.52504865,41.17359669)(437.70505147,40.63359723)
\curveto(435.06505411,40.12359774)(432.45505147,39.37359372)(432.45505147,35.86359723)
\curveto(432.45505147,32.77360032)(434.76505429,31.51359723)(437.58505147,31.51359723)
\curveto(439.74504931,31.51359723)(441.63505288,32.26359888)(443.04505147,33.91359723)
\curveto(443.04505147,32.23359891)(443.88505279,31.51359723)(445.20505147,31.51359723)
\curveto(446.01505066,31.51359723)(446.58505192,31.6635975)(447.03505147,31.93359723)
\lineto(447.03505147,33.91359723)
}
}
{
\newrgbcolor{curcolor}{0 0 0}
\pscustom[linestyle=none,fillstyle=solid,fillcolor=curcolor]
{
\newpath
\moveto(449.31833272,31.87359723)
\lineto(451.86833272,31.87359723)
\lineto(451.86833272,38.77359723)
\curveto(451.86833272,42.7035933)(453.36833683,45.04359723)(457.47833272,45.04359723)
\lineto(457.47833272,47.74359723)
\curveto(454.71833548,47.83359714)(453.00833149,46.60359474)(451.77833272,44.11359723)
\lineto(451.71833272,44.11359723)
\lineto(451.71833272,47.38359723)
\lineto(449.31833272,47.38359723)
\lineto(449.31833272,31.87359723)
}
}
{
\newrgbcolor{curcolor}{0 0 0}
\pscustom[linestyle=none,fillstyle=solid,fillcolor=curcolor]
{
\newpath
\moveto(459.57786397,31.87359723)
\lineto(462.12786397,31.87359723)
\lineto(462.12786397,47.38359723)
\lineto(459.57786397,47.38359723)
\lineto(459.57786397,31.87359723)
\moveto(462.12786397,53.29359723)
\lineto(459.57786397,53.29359723)
\lineto(459.57786397,50.17359723)
\lineto(462.12786397,50.17359723)
\lineto(462.12786397,53.29359723)
}
}
{
\newrgbcolor{curcolor}{0 0 0}
\pscustom[linestyle=none,fillstyle=solid,fillcolor=curcolor]
{
\newpath
\moveto(465.26755147,39.61359723)
\curveto(465.26755147,35.08360176)(467.87755639,31.51359723)(472.79755147,31.51359723)
\curveto(477.71754655,31.51359723)(480.32755147,35.08360176)(480.32755147,39.61359723)
\curveto(480.32755147,44.17359267)(477.71754655,47.74359723)(472.79755147,47.74359723)
\curveto(467.87755639,47.74359723)(465.26755147,44.17359267)(465.26755147,39.61359723)
\moveto(467.96755147,39.61359723)
\curveto(467.96755147,43.39359345)(470.12755414,45.49359723)(472.79755147,45.49359723)
\curveto(475.4675488,45.49359723)(477.62755147,43.39359345)(477.62755147,39.61359723)
\curveto(477.62755147,35.86360098)(475.4675488,33.76359723)(472.79755147,33.76359723)
\curveto(470.12755414,33.76359723)(467.96755147,35.86360098)(467.96755147,39.61359723)
}
}
{
\newrgbcolor{curcolor}{0 0 0}
\pscustom[linestyle=none,fillstyle=solid,fillcolor=curcolor]
{
}
}
{
\newrgbcolor{curcolor}{0 0 0}
\pscustom[linestyle=none,fillstyle=solid,fillcolor=curcolor]
{
\newpath
\moveto(493.60442647,45.61359723)
\curveto(493.51442656,48.16359468)(494.77442944,50.89359723)(497.74442647,50.89359723)
\curveto(499.99442422,50.89359723)(501.85442647,49.36359492)(501.85442647,47.05359723)
\curveto(501.85442647,44.11360017)(500.02442287,42.79359501)(496.42442647,40.57359723)
\curveto(493.42442947,38.71359909)(490.87442605,36.91359219)(490.45442647,31.87359723)
\lineto(504.34442647,31.87359723)
\lineto(504.34442647,34.12359723)
\lineto(493.42442647,34.12359723)
\curveto(493.93442596,36.76359459)(496.72442914,38.11359885)(499.39442647,39.73359723)
\curveto(502.03442383,41.38359558)(504.55442647,43.27360098)(504.55442647,47.02359723)
\curveto(504.55442647,50.98359327)(501.61442275,53.14359723)(497.89442647,53.14359723)
\curveto(493.39443097,53.14359723)(490.84442668,49.93359291)(491.05442647,45.61359723)
\lineto(493.60442647,45.61359723)
}
}
{
\newrgbcolor{curcolor}{0 0 0}
\pscustom[linewidth=2.65748024,linecolor=curcolor]
{
\newpath
\moveto(440.3152842,180.25557914)
\lineto(440.3152842,115.64372914)
}
}
{
\newrgbcolor{curcolor}{1 1 1}
\pscustom[linestyle=none,fillstyle=solid,fillcolor=curcolor]
{
\newpath
\moveto(463.1724291,192.12487159)
\curveto(463.1724291,179.30397606)(452.77903973,168.91058669)(439.9581442,168.91058669)
\curveto(427.13724867,168.91058669)(416.7438593,179.30397606)(416.7438593,192.12487159)
\curveto(416.7438593,204.94576711)(427.13724867,215.33915648)(439.9581442,215.33915648)
\curveto(452.76491751,215.33915648)(463.15244789,204.96779736)(463.17240092,192.16103959)
}
}
{
\newrgbcolor{curcolor}{0 0 0}
\pscustom[linewidth=2.65748024,linecolor=curcolor]
{
\newpath
\moveto(463.1724291,192.12487159)
\curveto(463.1724291,179.30397606)(452.77903973,168.91058669)(439.9581442,168.91058669)
\curveto(427.13724867,168.91058669)(416.7438593,179.30397606)(416.7438593,192.12487159)
\curveto(416.7438593,204.94576711)(427.13724867,215.33915648)(439.9581442,215.33915648)
\curveto(452.76491751,215.33915648)(463.15244789,204.96779736)(463.17240092,192.16103959)
}
}
{
\newrgbcolor{curcolor}{0 0 0}
\pscustom[linewidth=2.65748024,linecolor=curcolor]
{
\newpath
\moveto(419.1825832,79.26156914)
\lineto(440.4632242,117.51609914)
\lineto(460.5498842,79.26156914)
}
}
{
\newrgbcolor{curcolor}{0 0 0}
\pscustom[linewidth=2.65748024,linecolor=curcolor]
{
\newpath
\moveto(407.8152872,155.08045914)
\lineto(472.8152842,155.08045914)
}
}
{
\newrgbcolor{curcolor}{1 1 1}
\pscustom[linestyle=none,fillstyle=solid,fillcolor=curcolor]
{
\newpath
\moveto(158.9749,187.85983414)
\lineto(372.52114,187.85983414)
}
}
{
\newrgbcolor{curcolor}{0 0 0}
\pscustom[linewidth=2.65748024,linecolor=curcolor]
{
\newpath
\moveto(158.9749,187.85983414)
\lineto(372.52114,187.85983414)
}
}
{
\newrgbcolor{curcolor}{0 0 0}
\pscustom[linestyle=none,fillstyle=solid,fillcolor=curcolor]
{
\newpath
\moveto(172.87658488,181.42809119)
\lineto(155.45540306,187.83430119)
\lineto(172.87658584,194.24050978)
\curveto(170.09340933,190.4582868)(170.10944606,185.28354933)(172.87658488,181.42809119)
\closepath
}
}
{
\newrgbcolor{curcolor}{1 1 1}
\pscustom[linestyle=none,fillstyle=solid,fillcolor=curcolor]
{
\newpath
\moveto(374.28891,141.16527414)
\lineto(157.20713,141.16527414)
}
}
{
\newrgbcolor{curcolor}{0 0 0}
\pscustom[linewidth=2.65748024,linecolor=curcolor]
{
\newpath
\moveto(374.28891,141.16527414)
\lineto(157.20713,141.16527414)
}
}
{
\newrgbcolor{curcolor}{0 0 0}
\pscustom[linestyle=none,fillstyle=solid,fillcolor=curcolor]
{
\newpath
\moveto(360.38722512,147.59701709)
\lineto(377.80840694,141.19080709)
\lineto(360.38722416,134.7845985)
\curveto(363.17040067,138.56682148)(363.15436394,143.74155895)(360.38722512,147.59701709)
\closepath
}
}
{
\newrgbcolor{curcolor}{1 1 1}
\pscustom[linestyle=none,fillstyle=solid,fillcolor=curcolor]
{
\newpath
\moveto(157.20712,94.52176414)
\lineto(374.28891,94.52176414)
}
}
{
\newrgbcolor{curcolor}{0 0 0}
\pscustom[linewidth=2.65748024,linecolor=curcolor]
{
\newpath
\moveto(157.20712,94.52176414)
\lineto(374.28891,94.52176414)
}
}
{
\newrgbcolor{curcolor}{0 0 0}
\pscustom[linestyle=none,fillstyle=solid,fillcolor=curcolor]
{
\newpath
\moveto(171.10880488,88.09002119)
\lineto(153.68762306,94.49623119)
\lineto(171.10880584,100.90243978)
\curveto(168.32562933,97.1202168)(168.34166606,91.94547933)(171.10880488,88.09002119)
\closepath
}
}
{
\newrgbcolor{curcolor}{0 0 0}
\pscustom[linestyle=none,fillstyle=solid,fillcolor=curcolor]
{
\newpath
\moveto(360.38722512,100.95350709)
\lineto(377.80840694,94.54729709)
\lineto(360.38722416,88.1410885)
\curveto(363.17040067,91.92331148)(363.15436394,97.09804895)(360.38722512,100.95350709)
\closepath
}
}
{
\newrgbcolor{curcolor}{0 0 0}
\pscustom[linestyle=none,fillstyle=solid,fillcolor=curcolor]
{
\newpath
\moveto(228.40670776,206.9548334)
\lineto(228.40670776,194.93090762)
\lineto(229.3588562,194.93090762)
\lineto(229.3588562,206.9548334)
\lineto(232.03219604,206.9548334)
\lineto(232.03219604,207.78491152)
\lineto(229.3588562,207.78491152)
\lineto(229.3588562,210.36059512)
\curveto(229.35885295,211.0767248)(229.55009625,211.55686755)(229.93258667,211.8010248)
\curveto(230.26624137,212.0370103)(230.70162505,212.15501149)(231.23873901,212.15502871)
\curveto(231.62122048,212.15501149)(231.99556907,212.11432142)(232.36178589,212.0329584)
\lineto(232.36178589,212.85082949)
\curveto(231.99556907,212.94032971)(231.62122048,212.98508878)(231.23873901,212.98510684)
\curveto(230.44120864,212.98508878)(229.78202961,212.78977648)(229.26119995,212.39916934)
\curveto(228.69153591,211.98411322)(228.40670547,211.3290032)(228.40670776,210.4338373)
\lineto(228.40670776,207.78491152)
\lineto(226.11178589,207.78491152)
\lineto(226.11178589,206.9548334)
\lineto(228.40670776,206.9548334)
}
}
{
\newrgbcolor{curcolor}{0 0 0}
\pscustom[linestyle=none,fillstyle=solid,fillcolor=curcolor]
{
\newpath
\moveto(232.97213745,201.35180605)
\curveto(232.97213665,199.48819473)(233.48483145,197.91348927)(234.51022339,196.62768496)
\curveto(235.52747263,195.33373925)(236.9801079,194.6664222)(238.86813354,194.62573184)
\curveto(240.78055983,194.6664222)(242.2372641,195.33373925)(243.23825073,196.62768496)
\curveto(244.25549125,197.91348927)(244.76411704,199.48819473)(244.76412964,201.35180605)
\curveto(244.76411704,203.23981858)(244.25549125,204.82266205)(243.23825073,206.10034121)
\curveto(242.2372641,207.38613605)(240.78055983,208.04531508)(238.86813354,208.07788027)
\curveto(236.9801079,208.04531508)(235.52747263,207.38613605)(234.51022339,206.10034121)
\curveto(233.48483145,204.82266205)(232.97213665,203.23981858)(232.97213745,201.35180605)
\lineto(232.97213745,201.35180605)
\moveto(233.91207886,201.35180605)
\curveto(233.91207711,202.95498813)(234.34746079,204.33031228)(235.2182312,205.47778262)
\curveto(236.04830544,206.62523186)(237.26493834,207.21930679)(238.86813354,207.26000918)
\curveto(240.5038674,207.21930679)(241.73677632,206.62523186)(242.56686401,205.47778262)
\curveto(243.39693091,204.33031228)(243.81196956,202.95498813)(243.8119812,201.35180605)
\curveto(243.81196956,199.77302517)(243.39693091,198.41397705)(242.56686401,197.27465762)
\curveto(241.73677632,196.09464343)(240.5038674,195.48836149)(238.86813354,195.45580996)
\curveto(237.26493834,195.48836149)(236.04830544,196.09464343)(235.2182312,197.27465762)
\curveto(234.34746079,198.41397705)(233.91207711,199.77302517)(233.91207886,201.35180605)
\lineto(233.91207886,201.35180605)
}
}
{
\newrgbcolor{curcolor}{0 0 0}
\pscustom[linestyle=none,fillstyle=solid,fillcolor=curcolor]
{
\newpath
\moveto(247.21774292,212.7775873)
\lineto(247.21774292,194.93090762)
\lineto(248.16989136,194.93090762)
\lineto(248.16989136,212.7775873)
\lineto(247.21774292,212.7775873)
}
}
{
\newrgbcolor{curcolor}{0 0 0}
\pscustom[linestyle=none,fillstyle=solid,fillcolor=curcolor]
{
\newpath
\moveto(251.41696167,212.7775873)
\lineto(251.41696167,194.93090762)
\lineto(252.36911011,194.93090762)
\lineto(252.36911011,212.7775873)
\lineto(251.41696167,212.7775873)
}
}
{
\newrgbcolor{curcolor}{0 0 0}
\pscustom[linestyle=none,fillstyle=solid,fillcolor=curcolor]
{
\newpath
\moveto(254.79830933,201.35180605)
\curveto(254.79830852,199.48819473)(255.31100332,197.91348927)(256.33639526,196.62768496)
\curveto(257.35364451,195.33373925)(258.80627977,194.6664222)(260.69430542,194.62573184)
\curveto(262.6067317,194.6664222)(264.06343597,195.33373925)(265.06442261,196.62768496)
\curveto(266.08166312,197.91348927)(266.59028892,199.48819473)(266.59030151,201.35180605)
\curveto(266.59028892,203.23981858)(266.08166312,204.82266205)(265.06442261,206.10034121)
\curveto(264.06343597,207.38613605)(262.6067317,208.04531508)(260.69430542,208.07788027)
\curveto(258.80627977,208.04531508)(257.35364451,207.38613605)(256.33639526,206.10034121)
\curveto(255.31100332,204.82266205)(254.79830852,203.23981858)(254.79830933,201.35180605)
\lineto(254.79830933,201.35180605)
\moveto(255.73825073,201.35180605)
\curveto(255.73824899,202.95498813)(256.17363267,204.33031228)(257.04440308,205.47778262)
\curveto(257.87447732,206.62523186)(259.09111022,207.21930679)(260.69430542,207.26000918)
\curveto(262.33003927,207.21930679)(263.56294819,206.62523186)(264.39303589,205.47778262)
\curveto(265.22310278,204.33031228)(265.63814143,202.95498813)(265.63815308,201.35180605)
\curveto(265.63814143,199.77302517)(265.22310278,198.41397705)(264.39303589,197.27465762)
\curveto(263.56294819,196.09464343)(262.33003927,195.48836149)(260.69430542,195.45580996)
\curveto(259.09111022,195.48836149)(257.87447732,196.09464343)(257.04440308,197.27465762)
\curveto(256.17363267,198.41397705)(255.73824899,199.77302517)(255.73825073,201.35180605)
\lineto(255.73825073,201.35180605)
}
}
{
\newrgbcolor{curcolor}{0 0 0}
\pscustom[linestyle=none,fillstyle=solid,fillcolor=curcolor]
{
\newpath
\moveto(268.39694214,207.78491152)
\lineto(267.37155151,207.78491152)
\lineto(271.44869995,194.93090762)
\lineto(272.64498901,194.93090762)
\lineto(276.19723511,206.57641543)
\lineto(276.24606323,206.57641543)
\lineto(279.77389526,194.93090762)
\lineto(280.94577026,194.93090762)
\lineto(285.07174683,207.78491152)
\lineto(284.02194214,207.78491152)
\lineto(280.42086792,196.02954043)
\lineto(280.37203979,196.02954043)
\lineto(276.84420776,207.78491152)
\lineto(275.57467651,207.78491152)
\lineto(272.07125854,196.02954043)
\lineto(272.02243042,196.02954043)
\lineto(268.39694214,207.78491152)
}
}
{
\newrgbcolor{curcolor}{0 0 0}
\pscustom[linestyle=none,fillstyle=solid,fillcolor=curcolor]
{
\newpath
\moveto(297.19332886,201.22973574)
\curveto(297.29911088,203.15030044)(296.89221024,204.75755795)(295.97262573,206.05151309)
\curveto(295.03674335,207.36986002)(293.55969404,208.04531508)(291.54147339,208.07788027)
\curveto(289.62903391,208.04531508)(288.19674368,207.33730797)(287.24459839,205.95385684)
\curveto(286.29244871,204.58665968)(285.83672,202.97940217)(285.87741089,201.13207949)
\curveto(285.86113403,199.31729646)(286.33313877,197.79141908)(287.29342651,196.55444277)
\curveto(288.24557175,195.30932521)(289.66158596,194.6664222)(291.54147339,194.62573184)
\curveto(294.67460178,194.6664222)(296.52599967,196.21671362)(297.09567261,199.27661074)
\lineto(296.14352417,199.27661074)
\curveto(295.59826622,196.76196047)(294.06425083,195.48836149)(291.54147339,195.45580996)
\curveto(289.96269243,195.47208546)(288.78268059,196.06209138)(288.00143433,197.22582949)
\curveto(287.19576812,198.34887295)(286.8010745,199.68350703)(286.81735229,201.22973574)
\lineto(297.19332886,201.22973574)
\moveto(286.81735229,202.05981387)
\curveto(286.93942072,203.38630281)(287.39921843,204.57445266)(288.19674683,205.62426699)
\curveto(288.99426892,206.67405994)(290.10917666,207.21930679)(291.54147339,207.26000918)
\curveto(293.03072322,207.21930679)(294.18225201,206.69440497)(294.99606323,205.68530215)
\curveto(295.79357853,204.66803981)(296.20861718,203.45954492)(296.24118042,202.05981387)
\lineto(286.81735229,202.05981387)
}
}
{
\newrgbcolor{curcolor}{0 0 0}
\pscustom[linestyle=none,fillstyle=solid,fillcolor=curcolor]
{
\newpath
\moveto(300.37936401,207.78491152)
\lineto(299.43942261,207.78491152)
\lineto(299.43942261,194.93090762)
\lineto(300.37936401,194.93090762)
\lineto(300.37936401,201.94995059)
\curveto(300.43632776,202.90209105)(300.57874298,203.64671921)(300.80661011,204.1838373)
\curveto(301.02633368,204.71279888)(301.36813021,205.20514865)(301.83200073,205.66088809)
\curveto(302.43420988,206.22240023)(303.05269884,206.59674882)(303.68746948,206.78393496)
\curveto(304.3059528,206.94668336)(304.87154468,206.98737343)(305.38424683,206.90600527)
\lineto(305.38424683,207.85815371)
\curveto(304.20422764,207.89069283)(303.16256201,207.63027643)(302.25924683,207.07690371)
\curveto(301.33964717,206.51536869)(300.74150323,205.79922357)(300.46481323,204.92846621)
\lineto(300.37936401,204.92846621)
\lineto(300.37936401,207.78491152)
}
}
{
\newrgbcolor{curcolor}{0 0 0}
\pscustom[linestyle=none,fillstyle=solid,fillcolor=curcolor]
{
\newpath
\moveto(223.9694519,160.60363711)
\lineto(223.9694519,148.57971133)
\lineto(224.92160034,148.57971133)
\lineto(224.92160034,160.60363711)
\lineto(227.59494019,160.60363711)
\lineto(227.59494019,161.43371523)
\lineto(224.92160034,161.43371523)
\lineto(224.92160034,164.00939883)
\curveto(224.92159709,164.72552852)(225.11284039,165.20567126)(225.49533081,165.44982852)
\curveto(225.82898551,165.68581401)(226.26436919,165.8038152)(226.80148315,165.80383242)
\curveto(227.18396462,165.8038152)(227.55831321,165.76312513)(227.92453003,165.68176211)
\lineto(227.92453003,166.4996332)
\curveto(227.55831321,166.58913342)(227.18396462,166.63389249)(226.80148315,166.63391055)
\curveto(226.00395278,166.63389249)(225.34477375,166.43858019)(224.82394409,166.04797305)
\curveto(224.25428005,165.63291693)(223.96944961,164.97780691)(223.9694519,164.08264102)
\lineto(223.9694519,161.43371523)
\lineto(221.67453003,161.43371523)
\lineto(221.67453003,160.60363711)
\lineto(223.9694519,160.60363711)
}
}
{
\newrgbcolor{curcolor}{0 0 0}
\pscustom[linestyle=none,fillstyle=solid,fillcolor=curcolor]
{
\newpath
\moveto(228.53488159,155.00060977)
\curveto(228.53488079,153.13699844)(229.04757559,151.56229298)(230.07296753,150.27648867)
\curveto(231.09021677,148.98254296)(232.54285204,148.31522592)(234.43087769,148.27453555)
\curveto(236.34330397,148.31522592)(237.80000824,148.98254296)(238.80099487,150.27648867)
\curveto(239.81823539,151.56229298)(240.32686118,153.13699844)(240.32687378,155.00060977)
\curveto(240.32686118,156.88862229)(239.81823539,158.47146576)(238.80099487,159.74914492)
\curveto(237.80000824,161.03493976)(236.34330397,161.69411879)(234.43087769,161.72668398)
\curveto(232.54285204,161.69411879)(231.09021677,161.03493976)(230.07296753,159.74914492)
\curveto(229.04757559,158.47146576)(228.53488079,156.88862229)(228.53488159,155.00060977)
\lineto(228.53488159,155.00060977)
\moveto(229.474823,155.00060977)
\curveto(229.47482125,156.60379185)(229.91020493,157.97911599)(230.78097534,159.12658633)
\curveto(231.61104958,160.27403557)(232.82768248,160.8681105)(234.43087769,160.90881289)
\curveto(236.06661154,160.8681105)(237.29952046,160.27403557)(238.12960815,159.12658633)
\curveto(238.95967505,157.97911599)(239.3747137,156.60379185)(239.37472534,155.00060977)
\curveto(239.3747137,153.42182888)(238.95967505,152.06278076)(238.12960815,150.92346133)
\curveto(237.29952046,149.74344714)(236.06661154,149.1371652)(234.43087769,149.10461367)
\curveto(232.82768248,149.1371652)(231.61104958,149.74344714)(230.78097534,150.92346133)
\curveto(229.91020493,152.06278076)(229.47482125,153.42182888)(229.474823,155.00060977)
\lineto(229.474823,155.00060977)
}
}
{
\newrgbcolor{curcolor}{0 0 0}
\pscustom[linestyle=none,fillstyle=solid,fillcolor=curcolor]
{
\newpath
\moveto(242.78048706,166.42639102)
\lineto(242.78048706,148.57971133)
\lineto(243.7326355,148.57971133)
\lineto(243.7326355,166.42639102)
\lineto(242.78048706,166.42639102)
}
}
{
\newrgbcolor{curcolor}{0 0 0}
\pscustom[linestyle=none,fillstyle=solid,fillcolor=curcolor]
{
\newpath
\moveto(246.97970581,166.42639102)
\lineto(246.97970581,148.57971133)
\lineto(247.93185425,148.57971133)
\lineto(247.93185425,166.42639102)
\lineto(246.97970581,166.42639102)
}
}
{
\newrgbcolor{curcolor}{0 0 0}
\pscustom[linestyle=none,fillstyle=solid,fillcolor=curcolor]
{
\newpath
\moveto(250.36105347,155.00060977)
\curveto(250.36105266,153.13699844)(250.87374746,151.56229298)(251.8991394,150.27648867)
\curveto(252.91638865,148.98254296)(254.36902391,148.31522592)(256.25704956,148.27453555)
\curveto(258.16947584,148.31522592)(259.62618011,148.98254296)(260.62716675,150.27648867)
\curveto(261.64440726,151.56229298)(262.15303306,153.13699844)(262.15304565,155.00060977)
\curveto(262.15303306,156.88862229)(261.64440726,158.47146576)(260.62716675,159.74914492)
\curveto(259.62618011,161.03493976)(258.16947584,161.69411879)(256.25704956,161.72668398)
\curveto(254.36902391,161.69411879)(252.91638865,161.03493976)(251.8991394,159.74914492)
\curveto(250.87374746,158.47146576)(250.36105266,156.88862229)(250.36105347,155.00060977)
\lineto(250.36105347,155.00060977)
\moveto(251.30099487,155.00060977)
\curveto(251.30099313,156.60379185)(251.73637681,157.97911599)(252.60714722,159.12658633)
\curveto(253.43722146,160.27403557)(254.65385436,160.8681105)(256.25704956,160.90881289)
\curveto(257.89278341,160.8681105)(259.12569233,160.27403557)(259.95578003,159.12658633)
\curveto(260.78584692,157.97911599)(261.20088557,156.60379185)(261.20089722,155.00060977)
\curveto(261.20088557,153.42182888)(260.78584692,152.06278076)(259.95578003,150.92346133)
\curveto(259.12569233,149.74344714)(257.89278341,149.1371652)(256.25704956,149.10461367)
\curveto(254.65385436,149.1371652)(253.43722146,149.74344714)(252.60714722,150.92346133)
\curveto(251.73637681,152.06278076)(251.30099313,153.42182888)(251.30099487,155.00060977)
\lineto(251.30099487,155.00060977)
}
}
{
\newrgbcolor{curcolor}{0 0 0}
\pscustom[linestyle=none,fillstyle=solid,fillcolor=curcolor]
{
\newpath
\moveto(263.95968628,161.43371523)
\lineto(262.93429565,161.43371523)
\lineto(267.01144409,148.57971133)
\lineto(268.20773315,148.57971133)
\lineto(271.75997925,160.22521914)
\lineto(271.80880737,160.22521914)
\lineto(275.3366394,148.57971133)
\lineto(276.5085144,148.57971133)
\lineto(280.63449097,161.43371523)
\lineto(279.58468628,161.43371523)
\lineto(275.98361206,149.67834414)
\lineto(275.93478394,149.67834414)
\lineto(272.4069519,161.43371523)
\lineto(271.13742065,161.43371523)
\lineto(267.63400269,149.67834414)
\lineto(267.58517456,149.67834414)
\lineto(263.95968628,161.43371523)
}
}
{
\newrgbcolor{curcolor}{0 0 0}
\pscustom[linestyle=none,fillstyle=solid,fillcolor=curcolor]
{
\newpath
\moveto(283.1857605,148.57971133)
\lineto(283.1857605,161.43371523)
\lineto(282.23361206,161.43371523)
\lineto(282.23361206,148.57971133)
\lineto(283.1857605,148.57971133)
\moveto(283.1857605,163.85070742)
\lineto(283.1857605,166.42639102)
\lineto(282.23361206,166.42639102)
\lineto(282.23361206,163.85070742)
\lineto(283.1857605,163.85070742)
}
}
{
\newrgbcolor{curcolor}{0 0 0}
\pscustom[linestyle=none,fillstyle=solid,fillcolor=curcolor]
{
\newpath
\moveto(286.21310425,148.57971133)
\lineto(287.15304565,148.57971133)
\lineto(287.15304565,155.52551211)
\curveto(287.15304331,157.12869366)(287.57215096,158.42263768)(288.41036987,159.40734805)
\curveto(289.22416754,160.39203676)(290.34721329,160.89252454)(291.7795105,160.90881289)
\curveto(292.65027089,160.89252454)(293.34200197,160.7419713)(293.85470581,160.45715273)
\curveto(294.35925355,160.17231041)(294.71732611,159.79796183)(294.92892456,159.33410586)
\curveto(295.16491681,158.87836639)(295.31547004,158.41856868)(295.38058472,157.95471133)
\curveto(295.4131262,157.4745592)(295.42940222,157.05952056)(295.42941284,156.70959414)
\lineto(295.42941284,148.57971133)
\lineto(296.38156128,148.57971133)
\lineto(296.38156128,156.52648867)
\curveto(296.38154971,156.94151937)(296.36527368,157.44200715)(296.33273315,158.02795352)
\curveto(296.26761753,158.59760496)(296.09265026,159.16319684)(295.80783081,159.72473086)
\curveto(295.53926539,160.2943806)(295.08760569,160.77045435)(294.45285034,161.15295352)
\curveto(293.82621372,161.51915151)(292.94323934,161.71039481)(291.80392456,161.72668398)
\curveto(290.78666598,161.72667084)(289.85486353,161.47032344)(289.0085144,160.95764102)
\curveto(288.16215688,160.4205198)(287.56401295,159.65961561)(287.21408081,158.67492617)
\lineto(287.15304565,158.67492617)
\lineto(287.15304565,161.43371523)
\lineto(286.21310425,161.43371523)
\lineto(286.21310425,148.57971133)
}
}
{
\newrgbcolor{curcolor}{0 0 0}
\pscustom[linestyle=none,fillstyle=solid,fillcolor=curcolor]
{
\newpath
\moveto(304.16964722,160.90881289)
\curveto(305.75655332,160.8681105)(306.93249615,160.31472563)(307.69747925,159.24865664)
\curveto(308.46244254,158.18256631)(308.84492914,156.90082931)(308.84494019,155.4034418)
\curveto(308.84492914,153.90604064)(308.46244254,152.63244165)(307.69747925,151.58264102)
\curveto(306.93249615,150.46773027)(305.75655332,149.89806938)(304.16964722,149.87365664)
\curveto(302.67224651,149.89806938)(301.53292473,150.45145425)(300.75167847,151.53381289)
\curveto(299.96228828,152.58361357)(299.56759467,153.87348859)(299.56759644,155.4034418)
\curveto(299.56759467,156.83572521)(299.95415027,158.10118618)(300.7272644,159.19982852)
\curveto(301.47595864,160.29844961)(302.62341843,160.8681105)(304.16964722,160.90881289)
\lineto(304.16964722,160.90881289)
\moveto(308.86935425,161.43371523)
\lineto(308.86935425,158.74816836)
\lineto(308.82052612,158.74816836)
\curveto(308.45430453,159.73285773)(307.84802258,160.47748589)(307.00167847,160.98205508)
\curveto(306.14717793,161.47846145)(305.20316846,161.72667084)(304.16964722,161.72668398)
\curveto(302.37114004,161.69411879)(301.00395391,161.07156082)(300.06808472,159.8590082)
\curveto(299.10779695,158.65457104)(298.6276542,157.16938373)(298.62765503,155.4034418)
\curveto(298.6276542,153.58865814)(299.08338291,152.0871948)(299.99484253,150.89904727)
\curveto(300.91443577,149.68648105)(302.30603594,149.0720611)(304.16964722,149.05578555)
\curveto(306.35062825,149.0720611)(307.90091967,150.04862262)(308.82052612,151.98547305)
\lineto(308.86935425,151.98547305)
\lineto(308.86935425,149.03137148)
\curveto(308.86934318,148.81164469)(308.85306715,148.46984816)(308.82052612,148.00598086)
\curveto(308.755411,147.52583868)(308.58044373,147.00907488)(308.29562378,146.45568789)
\curveto(308.01078284,145.9348572)(307.55098512,145.49133551)(306.91622925,145.12512148)
\curveto(306.28959315,144.72636232)(305.41475679,144.51884299)(304.29171753,144.50256289)
\curveto(303.20935535,144.50256697)(302.29382892,144.75484536)(301.5451355,145.25939883)
\curveto(300.78015856,145.75582092)(300.33256786,146.56148418)(300.20236206,147.67639102)
\lineto(299.25021362,147.67639102)
\curveto(299.34786832,146.22782566)(299.86463213,145.19429805)(300.80050659,144.57580508)
\curveto(301.71196101,143.98173416)(302.86755881,143.68469669)(304.26730347,143.6846918)
\curveto(305.65075916,143.68469669)(306.72904584,143.91663005)(307.50216675,144.38049258)
\curveto(308.26713024,144.82808748)(308.81644609,145.38554135)(309.15011597,146.05285586)
\curveto(309.48376313,146.72017543)(309.68314445,147.35494042)(309.7482605,147.95715273)
\curveto(309.79707662,148.5593663)(309.82149066,148.99068097)(309.82150269,149.25109805)
\lineto(309.82150269,161.43371523)
\lineto(308.86935425,161.43371523)
}
}
{
\newrgbcolor{curcolor}{0 0 0}
\pscustom[linestyle=none,fillstyle=solid,fillcolor=curcolor]
{
\newpath
\moveto(240.32687378,111.81329287)
\lineto(240.32687378,99.78936709)
\lineto(241.27902222,99.78936709)
\lineto(241.27902222,111.81329287)
\lineto(243.95236206,111.81329287)
\lineto(243.95236206,112.643371)
\lineto(241.27902222,112.643371)
\lineto(241.27902222,115.21905459)
\curveto(241.27901897,115.93518428)(241.47026227,116.41532703)(241.85275269,116.65948428)
\curveto(242.18640739,116.89546978)(242.62179106,117.01347096)(243.15890503,117.01348818)
\curveto(243.5413865,117.01347096)(243.91573508,116.9727809)(244.2819519,116.89141787)
\lineto(244.2819519,117.70928896)
\curveto(243.91573508,117.79878918)(243.5413865,117.84354825)(243.15890503,117.84356631)
\curveto(242.36137466,117.84354825)(241.70219563,117.64823595)(241.18136597,117.25762881)
\curveto(240.61170193,116.84257269)(240.32687148,116.18746267)(240.32687378,115.29229678)
\lineto(240.32687378,112.643371)
\lineto(238.0319519,112.643371)
\lineto(238.0319519,111.81329287)
\lineto(240.32687378,111.81329287)
}
}
{
\newrgbcolor{curcolor}{0 0 0}
\pscustom[linestyle=none,fillstyle=solid,fillcolor=curcolor]
{
\newpath
\moveto(246.4303894,112.643371)
\lineto(245.490448,112.643371)
\lineto(245.490448,99.78936709)
\lineto(246.4303894,99.78936709)
\lineto(246.4303894,106.80841006)
\curveto(246.48735315,107.76055052)(246.62976837,108.50517869)(246.8576355,109.04229678)
\curveto(247.07735907,109.57125835)(247.4191556,110.06360812)(247.88302612,110.51934756)
\curveto(248.48523527,111.0808597)(249.10372423,111.45520829)(249.73849487,111.64239443)
\curveto(250.35697819,111.80514283)(250.92257007,111.8458329)(251.43527222,111.76446475)
\lineto(251.43527222,112.71661318)
\curveto(250.25525303,112.74915231)(249.2135874,112.4887359)(248.31027222,111.93536318)
\curveto(247.39067256,111.37382816)(246.79252863,110.65768304)(246.51583862,109.78692568)
\lineto(246.4303894,109.78692568)
\lineto(246.4303894,112.643371)
}
}
{
\newrgbcolor{curcolor}{0 0 0}
\pscustom[linestyle=none,fillstyle=solid,fillcolor=curcolor]
{
\newpath
\moveto(253.59591675,99.78936709)
\lineto(253.59591675,112.643371)
\lineto(252.64376831,112.643371)
\lineto(252.64376831,99.78936709)
\lineto(253.59591675,99.78936709)
\moveto(253.59591675,115.06036318)
\lineto(253.59591675,117.63604678)
\lineto(252.64376831,117.63604678)
\lineto(252.64376831,115.06036318)
\lineto(253.59591675,115.06036318)
}
}
{
\newrgbcolor{curcolor}{0 0 0}
\pscustom[linestyle=none,fillstyle=solid,fillcolor=curcolor]
{
\newpath
\moveto(267.365448,106.08819521)
\curveto(267.47123002,108.00875991)(267.06432938,109.61601742)(266.14474487,110.90997256)
\curveto(265.20886249,112.22831949)(263.73181318,112.90377455)(261.71359253,112.93633975)
\curveto(259.80115305,112.90377455)(258.36886282,112.19576744)(257.41671753,110.81231631)
\curveto(256.46456785,109.44511915)(256.00883914,107.83786165)(256.04953003,105.99053896)
\curveto(256.03325317,104.17575593)(256.50525791,102.64987855)(257.46554565,101.41290225)
\curveto(258.41769089,100.16778468)(259.8337051,99.52488168)(261.71359253,99.48419131)
\curveto(264.84672092,99.52488168)(266.69811881,101.0751731)(267.26779175,104.13507021)
\lineto(266.31564331,104.13507021)
\curveto(265.77038536,101.62041995)(264.23636997,100.34682096)(261.71359253,100.31426943)
\curveto(260.13481157,100.33054493)(258.95479973,100.92055085)(258.17355347,102.08428896)
\curveto(257.36788726,103.20733242)(256.97319364,104.5419665)(256.98947144,106.08819521)
\lineto(267.365448,106.08819521)
\moveto(256.98947144,106.91827334)
\curveto(257.11153986,108.24476228)(257.57133757,109.43291213)(258.36886597,110.48272646)
\curveto(259.16638806,111.53251941)(260.2812958,112.07776626)(261.71359253,112.11846865)
\curveto(263.20284236,112.07776626)(264.35437115,111.55286444)(265.16818237,110.54376162)
\curveto(265.96569767,109.52649928)(266.38073632,108.31800439)(266.41329956,106.91827334)
\lineto(256.98947144,106.91827334)
}
}
{
\newrgbcolor{curcolor}{0 0 0}
\pscustom[linestyle=none,fillstyle=solid,fillcolor=curcolor]
{
\newpath
\moveto(269.61154175,99.78936709)
\lineto(270.55148315,99.78936709)
\lineto(270.55148315,106.73516787)
\curveto(270.55148081,108.33834943)(270.97058846,109.63229344)(271.80880737,110.61700381)
\curveto(272.62260504,111.60169252)(273.74565079,112.1021803)(275.177948,112.11846865)
\curveto(276.04870839,112.1021803)(276.74043947,111.95162706)(277.25314331,111.6668085)
\curveto(277.75769105,111.38196617)(278.11576361,111.00761759)(278.32736206,110.54376162)
\curveto(278.56335431,110.08802216)(278.71390754,109.62822444)(278.77902222,109.16436709)
\curveto(278.8115637,108.68421497)(278.82783972,108.26917632)(278.82785034,107.9192499)
\lineto(278.82785034,99.78936709)
\lineto(279.77999878,99.78936709)
\lineto(279.77999878,107.73614443)
\curveto(279.77998721,108.15117513)(279.76371118,108.65166292)(279.73117065,109.23760928)
\curveto(279.66605503,109.80726072)(279.49108776,110.3728526)(279.20626831,110.93438662)
\curveto(278.93770289,111.50403636)(278.48604319,111.98011011)(277.85128784,112.36260928)
\curveto(277.22465122,112.72880728)(276.34167684,112.92005057)(275.20236206,112.93633975)
\curveto(274.18510348,112.9363266)(273.25330103,112.6799792)(272.4069519,112.16729678)
\curveto(271.56059438,111.63017556)(270.96245045,110.86927137)(270.61251831,109.88458193)
\lineto(270.55148315,109.88458193)
\lineto(270.55148315,112.643371)
\lineto(269.61154175,112.643371)
\lineto(269.61154175,99.78936709)
}
}
{
\newrgbcolor{curcolor}{0 0 0}
\pscustom[linestyle=none,fillstyle=solid,fillcolor=curcolor]
{
\newpath
\moveto(287.75119019,100.31426943)
\curveto(286.09916705,100.34682096)(284.89067217,100.95310291)(284.1257019,102.13311709)
\curveto(283.33631174,103.27243652)(282.94161813,104.62334663)(282.94161987,106.18585146)
\curveto(282.9253421,107.80530959)(283.30375969,109.18877175)(284.07687378,110.33624209)
\curveto(284.82556807,111.48369133)(286.03813196,112.07776626)(287.71456909,112.11846865)
\curveto(289.36657915,112.07776626)(290.58321205,111.46741531)(291.36447144,110.28741396)
\curveto(292.12129645,109.12366765)(292.49971404,107.75648152)(292.49972534,106.18585146)
\curveto(292.49971404,104.65589868)(292.11315844,103.31719559)(291.34005737,102.16973818)
\curveto(290.57507404,100.96530992)(289.37878617,100.34682096)(287.75119019,100.31426943)
\lineto(287.75119019,100.31426943)
\moveto(292.5241394,99.78936709)
\lineto(293.46408081,99.78936709)
\lineto(293.46408081,117.63604678)
\lineto(292.5241394,117.63604678)
\lineto(292.5241394,109.76251162)
\lineto(292.47531128,109.76251162)
\curveto(292.32067776,110.28333446)(292.08467539,110.74313218)(291.76730347,111.14190615)
\curveto(291.4499104,111.54065742)(291.07556182,111.87431594)(290.64425659,112.14288271)
\curveto(289.7816178,112.67184119)(288.80505627,112.9363266)(287.71456909,112.93633975)
\curveto(285.80212959,112.92005057)(284.36983936,112.27714757)(283.41769409,111.00762881)
\curveto(282.4736824,109.74622562)(282.00167766,108.13896812)(282.00167847,106.18585146)
\curveto(281.97726362,104.35479221)(282.42485432,102.79636278)(283.3444519,101.5105585)
\curveto(284.23149314,100.20033673)(285.63123132,99.52488168)(287.54367065,99.48419131)
\curveto(289.81416985,99.48419161)(291.45804841,100.5014432)(292.47531128,102.53594912)
\lineto(292.5241394,102.53594912)
\lineto(292.5241394,99.78936709)
}
}
\end{pspicture}

    \caption{Tipos de relacionamientos entre usuarios}
    \label{contactos}
    \end{figure}
\end{description}

Estos recursos, son los componentes propios de un sitio web, ademas de darle las
caracteristicas de una red social propiamente dicha.

\subsection{Fomento a la participación}

La parte mas fundamental del sistema, y el factor clave para el exito de toda
red social, son las funcionalidades que propicien la cultura de participación.

Estos elementos, que estan inspirados en la tendencia de los sitios considerados
dentro de la web 2.0\footnote{Definición disponible en:
http://es.wikipedia.org/wiki/Web\_2.0}, han sido considerados como base para el
establecimiento de las definiciónes siguientes.

Para tal proposito se han definido los siguientes elementos:

\begin{itemize}
\item Comentarios
\item Valoraciones
\item Etiquetado
\item Sistema de reputación\footnote{Si bien vamos a ahondar en este concepto,
pueden verse los detalles introductorios en:
http://en.wikipedia.org/wiki/Reputation\_system}.
\end{itemize}

Estos elementos deben estar disponibles para cualquiera de los recursos
intercambiables definidos anteriormente.

\section{Requerimientos no funcionales}

\section{Estandarización}
\subsection{Estándares de análisis}
\subsection{Estándares de diseño}
\subsection{Estándares de codificación}
\subsection{Estándares de pruebas}

\section{Planificación}
\subsection{Iteraciones}

