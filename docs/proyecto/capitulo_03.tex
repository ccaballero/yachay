\chapter{Metodología de desarrollo}

En este capítulo, se desarrollan los aspectos necesarios para la definición del 
proceso de desarrollo, primeramente se hará referencia a las cuestiones 
relacionadas con la metodología de desarrollo, posteriormente se detallarán los
requisitos del sistema, introduciendo los conceptos claves que se utilizan en el
producto de software; para terminar con una descripción de las etapas
establecidas según la planificación del proyecto.

\section{Modelo iterativo}

Considerando el contexto de desarrollo (el contexto esta descrito mas adelante
en este capítulo), se ha visto conveniente seguir un modelo de desarrollo que
sea iterativo e incremental\footnote{Para una definición exacta puede
consultarse:
https://es.wikipedia.org/wiki/Desarrollo\_iterativo\_y\_creciente}, estos dos
conceptos se definen a continuación\cite{Cockburn}:

\begin{description}
\item [Desarrollo Incremental] Es una estrategia de desarrollo en la cual se
desarrollan diversas partes del sistema en diferentes etapas, y estas son
integradas a medida que son completadas.
\item [Desarrollo Iterativo] Es una estrategia de desarrollo en la cual se
reserva una parte del tiempo para la revision y mejora de partes del sistema.
\end{description}

La idea central es que, en cada una de esas iteraciones, se construye una parte
pequeña del sistema. Para esa parte del sistema, se realiza todo el proceso:
análisis, diseño, programación y pruebas. Se acaba la iteración con un
prototipo funcional, que incluya todas las partes del sistema construidas hasta
el momento. Los aspectos del sistema con más riesgo (por ejemplo, la
arquitectura) se definen y construyen en las primeras iteraciones.

Las ventajas de este modelo son las siguientes:

\begin{description}
\item [Flexibilidad] Los requisitos funcionales no quedan totalmente fijados
hasta el final del proyecto de desarrollo. Por ello, se pueden realizar cambios
de forma flexible. Por una parte, el conocimiento que se adquiere en una
iteración sirve para plantear de forma más realista los requisitos de la
siguiente. Por otra parte, este conocimiento nos puede hacer reformar partes del
sistema construidas en iteraciones anteriores. En una palabra, todos los
documentos del sistema (requisitos funcionales, diseño de datos y código fuente)
son flexibles y pueden cambiar durante todo el proceso de desarrollo
(Típicamente suelen ser modificados en mayor medida en las primeras iteraciones
y en menor medida en las últimas).
\item [Mitigación de riesgos] Como las pruebas se hacen desde el principio del
proyecto, puede determinarse la viabilidad o eficiencia de las decisiones de
diseño. Además, los elementos con más riesgo se tratan en las primeras
iteraciones, con lo cual se puede implementar una mitigación de riesgos más 
temprana y exitosa.
\item [Retroalimentación] Como hay prototipos desde el mismo comienzo del
proyecto, estos pueden examinarse, y revalorizarse. También existe una rápida
retroalimentación de lo que funciona y lo que no, ya que las pruebas se 
realizan desde el comienzo mismo del proyecto y no se debe esperar al final
para hacer las modificaciones necesarias.
\end{description}

\section{Requisitos funcionales}

Un requisito funcional define una función del sistema de software o sus
componentes\footnote{Para una definición exacta puede consultarse:
https://es.wikipedia.org/wiki/Requisito\_funcional}, y estos detallan
completamente las capacidades que un sistema posee; en nuestro caso, estos han
sido clasificados según el objetivo especifico del proyecto al cual estos
contribuyen.

\subsection{Espacios virtuales}

Uno de los puntos fundamentales en la construcción del sistema, es el control y
manejo organizado de los espacios disponibles del sistema, estos espacios
constituyen los lugares, de intercambio, producción, y discusión de los recursos
que posea el sistema.

Estos recursos además pueden clasificarse según su temporalidad, es decir si
poseen alguna forma de caducidad, o si no poseen tal cualidad. Estos son:

\begin{description}
\item [Temporales] Es un espacio temporal todo aquel que depende de la gestión
en la que uno se encuentre; tanto su acceso, como su visibilidad están
delimitadas por la gestión que este presentándose, estos espacios son lo que se
construyeron inicialmente, ejemplos de estos son: materias, grupos, etc.
\item [Atemporales] Es un espacio atemporal aquel que no esta englobado en una
gestión determinada, su acceso y visibilidad es independiente, ejemplos de
estos son: comunidades, áreas, etc.
\end{description}

Los espacios virtuales que se han planificado construir son los siguientes:

\begin{description}
\item [Gestiones] Una gestión representa la división básica de periodos
académicos, estos trazan un marco de referencia temporal (es decir, su valor de
caducidad) para muchos de los espacios restantes.
\item [Materias] Una materia es el espacio que concentra todos los recursos de
una materia, (esta a su vez concentra a otros sub-espacios). Este espacio es a
su vez un sub-espacio de algún espacio de gestión.
\item [Grupos] Los grupos son espacios de separación de una materia, esta está
basada en el sistema utilizado en el dominio de implementación del sistema
(UMSS).
\item [Equipos] Los equipos son espacios opcionales de creación, que pueden
utilizarse para dividir aún mas un grupo de estudio, según el método que el
docente pretenda utilizar.
\item [Carreras] Las carreras representan una concentración de materias que a su
vez están agrupadas según gestiones especificas.
\item [Áreas] Un área es otra forma de agrupación de materias, que carecen de
una cualidad temporal, es decir, que no poseen caducidad.
\item [Comunidades] Una comunidad es una forma de espacio virtual independiente
de toda gestión (lo que implica que no tiene caducidad), y la intención es poder
agrupar a los usuarios según un interés en particular.
\end{description}

En la figura (\ref{espacios}) pueden apreciarse los espacios virtuales que se
construirán, remarcando su característica temporal. Como se verá posteriormente
la clasificación de los espacios según su temporalidad es imprescindible para
corregir los inconvenientes creados por la formalidad que poseen algunos
espacios, brindando espacios que poseen un carácter mas libre.

\begin{figure}
\centering
%LaTeX with PSTricks extensions
%%Creator: inkscape 0.48.4
%%Please note this file requires PSTricks extensions
\psset{xunit=.5pt,yunit=.5pt,runit=.5pt}
\begin{pspicture}(531.49603271,425.19683838)
{
\newrgbcolor{curcolor}{0 0 0}
\pscustom[linestyle=none,fillstyle=solid,fillcolor=curcolor]
{
\newpath
\moveto(50.2411504,357.87891102)
\lineto(52.0411504,357.87891102)
\lineto(52.0411504,369.15891102)
\lineto(42.6511504,369.15891102)
\lineto(42.6511504,366.75891102)
\lineto(49.4911504,366.75891102)
\curveto(49.67115022,362.85891492)(47.03114599,359.76891102)(42.6211504,359.76891102)
\curveto(37.85115517,359.76891102)(35.2711504,363.84891549)(35.2711504,368.31891102)
\curveto(35.2711504,372.90890643)(37.43115559,377.40891102)(42.6211504,377.40891102)
\curveto(45.80114722,377.40891102)(48.41115097,375.93890778)(48.9811504,372.69891102)
\lineto(51.8311504,372.69891102)
\curveto(51.02115121,377.70890601)(47.30114572,379.80891102)(42.6211504,379.80891102)
\curveto(35.84115718,379.80891102)(32.4211504,374.40890478)(32.4211504,368.16891102)
\curveto(32.4211504,362.5889166)(36.23115679,357.36891102)(42.6211504,357.36891102)
\curveto(45.14114788,357.36891102)(47.84115205,358.29891327)(49.4911504,360.54891102)
\lineto(50.2411504,357.87891102)
}
}
{
\newrgbcolor{curcolor}{0 0 0}
\pscustom[linestyle=none,fillstyle=solid,fillcolor=curcolor]
{
\newpath
\moveto(66.55411915,362.79891102)
\curveto(66.1041196,360.78891303)(64.63411705,359.76891102)(62.53411915,359.76891102)
\curveto(59.14412254,359.76891102)(57.61411924,362.16891372)(57.70411915,364.86891102)
\lineto(69.31411915,364.86891102)
\curveto(69.464119,368.61890727)(67.78411366,373.74891102)(62.29411915,373.74891102)
\curveto(58.06412338,373.74891102)(55.00411915,370.32890637)(55.00411915,365.67891102)
\curveto(55.154119,360.93891576)(57.4941241,357.51891102)(62.44411915,357.51891102)
\curveto(65.92411567,357.51891102)(68.38411984,359.37891444)(69.07411915,362.79891102)
\lineto(66.55411915,362.79891102)
\moveto(57.70411915,367.11891102)
\curveto(57.88411897,369.48890865)(59.47412182,371.49891102)(62.14411915,371.49891102)
\curveto(64.66411663,371.49891102)(66.49411927,369.54890859)(66.61411915,367.11891102)
\lineto(57.70411915,367.11891102)
}
}
{
\newrgbcolor{curcolor}{0 0 0}
\pscustom[linestyle=none,fillstyle=solid,fillcolor=curcolor]
{
\newpath
\moveto(70.9674004,362.76891102)
\curveto(71.11740025,358.92891486)(74.05740388,357.51891102)(77.5374004,357.51891102)
\curveto(80.68739725,357.51891102)(84.1374004,358.71891471)(84.1374004,362.40891102)
\curveto(84.1374004,365.40890802)(81.61739785,366.24891159)(79.0674004,366.81891102)
\curveto(76.69740277,367.38891045)(73.9974004,367.68891285)(73.9974004,369.51891102)
\curveto(73.9974004,371.07890946)(75.76740193,371.49891102)(77.2974004,371.49891102)
\curveto(78.97739872,371.49891102)(80.71740058,370.86890904)(80.8974004,368.88891102)
\lineto(83.4474004,368.88891102)
\curveto(83.23740061,372.66890724)(80.50739698,373.74891102)(77.0874004,373.74891102)
\curveto(74.3874031,373.74891102)(71.2974004,372.4589079)(71.2974004,369.33891102)
\curveto(71.2974004,366.36891399)(73.84740292,365.52891045)(76.3674004,364.95891102)
\curveto(78.91739785,364.38891159)(81.4374004,364.05890904)(81.4374004,362.07891102)
\curveto(81.4374004,360.12891297)(79.27739881,359.76891102)(77.6874004,359.76891102)
\curveto(75.5874025,359.76891102)(73.60740031,360.4889133)(73.5174004,362.76891102)
\lineto(70.9674004,362.76891102)
}
}
{
\newrgbcolor{curcolor}{0 0 0}
\pscustom[linestyle=none,fillstyle=solid,fillcolor=curcolor]
{
\newpath
\moveto(90.4974004,378.03891102)
\lineto(87.9474004,378.03891102)
\lineto(87.9474004,373.38891102)
\lineto(85.3074004,373.38891102)
\lineto(85.3074004,371.13891102)
\lineto(87.9474004,371.13891102)
\lineto(87.9474004,361.26891102)
\curveto(87.9474004,358.41891387)(88.99740304,357.87891102)(91.6374004,357.87891102)
\lineto(93.5874004,357.87891102)
\lineto(93.5874004,360.12891102)
\lineto(92.4174004,360.12891102)
\curveto(90.82740199,360.12891102)(90.4974004,360.33891219)(90.4974004,361.50891102)
\lineto(90.4974004,371.13891102)
\lineto(93.5874004,371.13891102)
\lineto(93.5874004,373.38891102)
\lineto(90.4974004,373.38891102)
\lineto(90.4974004,378.03891102)
}
}
{
\newrgbcolor{curcolor}{0 0 0}
\pscustom[linestyle=none,fillstyle=solid,fillcolor=curcolor]
{
\newpath
\moveto(96.54099415,357.87891102)
\lineto(99.09099415,357.87891102)
\lineto(99.09099415,373.38891102)
\lineto(96.54099415,373.38891102)
\lineto(96.54099415,357.87891102)
\moveto(99.09099415,379.29891102)
\lineto(96.54099415,379.29891102)
\lineto(96.54099415,376.17891102)
\lineto(99.09099415,376.17891102)
\lineto(99.09099415,379.29891102)
}
}
{
\newrgbcolor{curcolor}{0 0 0}
\pscustom[linestyle=none,fillstyle=solid,fillcolor=curcolor]
{
\newpath
\moveto(102.23068165,365.61891102)
\curveto(102.23068165,361.08891555)(104.84068657,357.51891102)(109.76068165,357.51891102)
\curveto(114.68067673,357.51891102)(117.29068165,361.08891555)(117.29068165,365.61891102)
\curveto(117.29068165,370.17890646)(114.68067673,373.74891102)(109.76068165,373.74891102)
\curveto(104.84068657,373.74891102)(102.23068165,370.17890646)(102.23068165,365.61891102)
\moveto(104.93068165,365.61891102)
\curveto(104.93068165,369.39890724)(107.09068432,371.49891102)(109.76068165,371.49891102)
\curveto(112.43067898,371.49891102)(114.59068165,369.39890724)(114.59068165,365.61891102)
\curveto(114.59068165,361.86891477)(112.43067898,359.76891102)(109.76068165,359.76891102)
\curveto(107.09068432,359.76891102)(104.93068165,361.86891477)(104.93068165,365.61891102)
\moveto(107.93068165,375.51891102)
\lineto(109.85068165,375.51891102)
\lineto(113.78068165,379.80891102)
\lineto(110.51068165,379.80891102)
\lineto(107.93068165,375.51891102)
}
}
{
\newrgbcolor{curcolor}{0 0 0}
\pscustom[linestyle=none,fillstyle=solid,fillcolor=curcolor]
{
\newpath
\moveto(120.29724415,357.87891102)
\lineto(122.84724415,357.87891102)
\lineto(122.84724415,366.63891102)
\curveto(122.84724415,369.42890823)(124.34724724,371.49891102)(127.43724415,371.49891102)
\curveto(129.3872422,371.49891102)(130.58724415,370.26890913)(130.58724415,368.37891102)
\lineto(130.58724415,357.87891102)
\lineto(133.13724415,357.87891102)
\lineto(133.13724415,368.07891102)
\curveto(133.13724415,371.40890769)(131.87724007,373.74891102)(127.79724415,373.74891102)
\curveto(125.57724637,373.74891102)(123.83724307,372.8489091)(122.75724415,370.92891102)
\lineto(122.69724415,370.92891102)
\lineto(122.69724415,373.38891102)
\lineto(120.29724415,373.38891102)
\lineto(120.29724415,357.87891102)
}
}
{
\newrgbcolor{curcolor}{0 0 0}
\pscustom[linestyle=none,fillstyle=solid,fillcolor=curcolor]
{
\newpath
\moveto(82.93620258,305.22198963)
\lineto(85.63620258,305.22198963)
\lineto(85.63620258,323.04198963)
\lineto(85.69620258,323.04198963)
\lineto(92.38620258,305.22198963)
\lineto(94.81620258,305.22198963)
\lineto(101.50620258,323.04198963)
\lineto(101.56620258,323.04198963)
\lineto(101.56620258,305.22198963)
\lineto(104.26620258,305.22198963)
\lineto(104.26620258,326.64198963)
\lineto(100.36620258,326.64198963)
\lineto(93.58620258,308.64198963)
\lineto(86.83620258,326.64198963)
\lineto(82.93620258,326.64198963)
\lineto(82.93620258,305.22198963)
}
}
{
\newrgbcolor{curcolor}{0 0 0}
\pscustom[linestyle=none,fillstyle=solid,fillcolor=curcolor]
{
\newpath
\moveto(118.18901508,310.62198963)
\curveto(118.18901508,309.21199104)(116.80901172,307.11198963)(113.44901508,307.11198963)
\curveto(111.88901664,307.11198963)(110.44901508,307.71199131)(110.44901508,309.39198963)
\curveto(110.44901508,311.28198774)(111.88901676,311.88198993)(113.56901508,312.18198963)
\curveto(115.27901337,312.48198933)(117.19901607,312.51199035)(118.18901508,313.23198963)
\lineto(118.18901508,310.62198963)
\moveto(122.32901508,307.26198963)
\curveto(121.99901541,307.14198975)(121.75901487,307.11198963)(121.54901508,307.11198963)
\curveto(120.73901589,307.11198963)(120.73901508,307.65199083)(120.73901508,308.85198963)
\lineto(120.73901508,316.83198963)
\curveto(120.73901508,320.461986)(117.70901229,321.09198963)(114.91901508,321.09198963)
\curveto(111.46901853,321.09198963)(108.49901493,319.74198579)(108.34901508,315.90198963)
\lineto(110.89901508,315.90198963)
\curveto(111.01901496,318.18198735)(112.60901724,318.84198963)(114.76901508,318.84198963)
\curveto(116.38901346,318.84198963)(118.21901508,318.48198741)(118.21901508,316.26198963)
\curveto(118.21901508,314.34199155)(115.81901226,314.52198909)(112.99901508,313.98198963)
\curveto(110.35901772,313.47199014)(107.74901508,312.72198612)(107.74901508,309.21198963)
\curveto(107.74901508,306.12199272)(110.0590179,304.86198963)(112.87901508,304.86198963)
\curveto(115.03901292,304.86198963)(116.92901649,305.61199128)(118.33901508,307.26198963)
\curveto(118.33901508,305.58199131)(119.1790164,304.86198963)(120.49901508,304.86198963)
\curveto(121.30901427,304.86198963)(121.87901553,305.0119899)(122.32901508,305.28198963)
\lineto(122.32901508,307.26198963)
}
}
{
\newrgbcolor{curcolor}{0 0 0}
\pscustom[linestyle=none,fillstyle=solid,fillcolor=curcolor]
{
\newpath
\moveto(128.24229633,325.38198963)
\lineto(125.69229633,325.38198963)
\lineto(125.69229633,320.73198963)
\lineto(123.05229633,320.73198963)
\lineto(123.05229633,318.48198963)
\lineto(125.69229633,318.48198963)
\lineto(125.69229633,308.61198963)
\curveto(125.69229633,305.76199248)(126.74229897,305.22198963)(129.38229633,305.22198963)
\lineto(131.33229633,305.22198963)
\lineto(131.33229633,307.47198963)
\lineto(130.16229633,307.47198963)
\curveto(128.57229792,307.47198963)(128.24229633,307.6819908)(128.24229633,308.85198963)
\lineto(128.24229633,318.48198963)
\lineto(131.33229633,318.48198963)
\lineto(131.33229633,320.73198963)
\lineto(128.24229633,320.73198963)
\lineto(128.24229633,325.38198963)
}
}
{
\newrgbcolor{curcolor}{0 0 0}
\pscustom[linestyle=none,fillstyle=solid,fillcolor=curcolor]
{
\newpath
\moveto(144.84589008,310.14198963)
\curveto(144.39589053,308.13199164)(142.92588798,307.11198963)(140.82589008,307.11198963)
\curveto(137.43589347,307.11198963)(135.90589017,309.51199233)(135.99589008,312.21198963)
\lineto(147.60589008,312.21198963)
\curveto(147.75588993,315.96198588)(146.07588459,321.09198963)(140.58589008,321.09198963)
\curveto(136.35589431,321.09198963)(133.29589008,317.67198498)(133.29589008,313.02198963)
\curveto(133.44588993,308.28199437)(135.78589503,304.86198963)(140.73589008,304.86198963)
\curveto(144.2158866,304.86198963)(146.67589077,306.72199305)(147.36589008,310.14198963)
\lineto(144.84589008,310.14198963)
\moveto(135.99589008,314.46198963)
\curveto(136.1758899,316.83198726)(137.76589275,318.84198963)(140.43589008,318.84198963)
\curveto(142.95588756,318.84198963)(144.7858902,316.8919872)(144.90589008,314.46198963)
\lineto(135.99589008,314.46198963)
}
}
{
\newrgbcolor{curcolor}{0 0 0}
\pscustom[linestyle=none,fillstyle=solid,fillcolor=curcolor]
{
\newpath
\moveto(150.15917133,305.22198963)
\lineto(152.70917133,305.22198963)
\lineto(152.70917133,312.12198963)
\curveto(152.70917133,316.0519857)(154.20917544,318.39198963)(158.31917133,318.39198963)
\lineto(158.31917133,321.09198963)
\curveto(155.55917409,321.18198954)(153.8491701,319.95198714)(152.61917133,317.46198963)
\lineto(152.55917133,317.46198963)
\lineto(152.55917133,320.73198963)
\lineto(150.15917133,320.73198963)
\lineto(150.15917133,305.22198963)
}
}
{
\newrgbcolor{curcolor}{0 0 0}
\pscustom[linestyle=none,fillstyle=solid,fillcolor=curcolor]
{
\newpath
\moveto(160.41870258,305.22198963)
\lineto(162.96870258,305.22198963)
\lineto(162.96870258,320.73198963)
\lineto(160.41870258,320.73198963)
\lineto(160.41870258,305.22198963)
\moveto(162.96870258,326.64198963)
\lineto(160.41870258,326.64198963)
\lineto(160.41870258,323.52198963)
\lineto(162.96870258,323.52198963)
\lineto(162.96870258,326.64198963)
}
}
{
\newrgbcolor{curcolor}{0 0 0}
\pscustom[linestyle=none,fillstyle=solid,fillcolor=curcolor]
{
\newpath
\moveto(176.54839008,310.62198963)
\curveto(176.54839008,309.21199104)(175.16838672,307.11198963)(171.80839008,307.11198963)
\curveto(170.24839164,307.11198963)(168.80839008,307.71199131)(168.80839008,309.39198963)
\curveto(168.80839008,311.28198774)(170.24839176,311.88198993)(171.92839008,312.18198963)
\curveto(173.63838837,312.48198933)(175.55839107,312.51199035)(176.54839008,313.23198963)
\lineto(176.54839008,310.62198963)
\moveto(180.68839008,307.26198963)
\curveto(180.35839041,307.14198975)(180.11838987,307.11198963)(179.90839008,307.11198963)
\curveto(179.09839089,307.11198963)(179.09839008,307.65199083)(179.09839008,308.85198963)
\lineto(179.09839008,316.83198963)
\curveto(179.09839008,320.461986)(176.06838729,321.09198963)(173.27839008,321.09198963)
\curveto(169.82839353,321.09198963)(166.85838993,319.74198579)(166.70839008,315.90198963)
\lineto(169.25839008,315.90198963)
\curveto(169.37838996,318.18198735)(170.96839224,318.84198963)(173.12839008,318.84198963)
\curveto(174.74838846,318.84198963)(176.57839008,318.48198741)(176.57839008,316.26198963)
\curveto(176.57839008,314.34199155)(174.17838726,314.52198909)(171.35839008,313.98198963)
\curveto(168.71839272,313.47199014)(166.10839008,312.72198612)(166.10839008,309.21198963)
\curveto(166.10839008,306.12199272)(168.4183929,304.86198963)(171.23839008,304.86198963)
\curveto(173.39838792,304.86198963)(175.28839149,305.61199128)(176.69839008,307.26198963)
\curveto(176.69839008,305.58199131)(177.5383914,304.86198963)(178.85839008,304.86198963)
\curveto(179.66838927,304.86198963)(180.23839053,305.0119899)(180.68839008,305.28198963)
\lineto(180.68839008,307.26198963)
}
}
{
\newrgbcolor{curcolor}{0 0 0}
\pscustom[linestyle=none,fillstyle=solid,fillcolor=curcolor]
{
\newpath
\moveto(148.30735065,252.20502186)
\lineto(150.10735065,252.20502186)
\lineto(150.10735065,263.48502186)
\lineto(140.71735065,263.48502186)
\lineto(140.71735065,261.08502186)
\lineto(147.55735065,261.08502186)
\curveto(147.73735047,257.18502576)(145.09734624,254.09502186)(140.68735065,254.09502186)
\curveto(135.91735542,254.09502186)(133.33735065,258.17502633)(133.33735065,262.64502186)
\curveto(133.33735065,267.23501727)(135.49735584,271.73502186)(140.68735065,271.73502186)
\curveto(143.86734747,271.73502186)(146.47735122,270.26501862)(147.04735065,267.02502186)
\lineto(149.89735065,267.02502186)
\curveto(149.08735146,272.03501685)(145.36734597,274.13502186)(140.68735065,274.13502186)
\curveto(133.90735743,274.13502186)(130.48735065,268.73501562)(130.48735065,262.49502186)
\curveto(130.48735065,256.91502744)(134.29735704,251.69502186)(140.68735065,251.69502186)
\curveto(143.20734813,251.69502186)(145.9073523,252.62502411)(147.55735065,254.87502186)
\lineto(148.30735065,252.20502186)
}
}
{
\newrgbcolor{curcolor}{0 0 0}
\pscustom[linestyle=none,fillstyle=solid,fillcolor=curcolor]
{
\newpath
\moveto(153.8203194,252.20502186)
\lineto(156.3703194,252.20502186)
\lineto(156.3703194,259.10502186)
\curveto(156.3703194,263.03501793)(157.87032351,265.37502186)(161.9803194,265.37502186)
\lineto(161.9803194,268.07502186)
\curveto(159.22032216,268.16502177)(157.51031817,266.93501937)(156.2803194,264.44502186)
\lineto(156.2203194,264.44502186)
\lineto(156.2203194,267.71502186)
\lineto(153.8203194,267.71502186)
\lineto(153.8203194,252.20502186)
}
}
{
\newrgbcolor{curcolor}{0 0 0}
\pscustom[linestyle=none,fillstyle=solid,fillcolor=curcolor]
{
\newpath
\moveto(176.76985065,267.71502186)
\lineto(174.21985065,267.71502186)
\lineto(174.21985065,258.95502186)
\curveto(174.21985065,256.16502465)(172.71984756,254.09502186)(169.62985065,254.09502186)
\curveto(167.6798526,254.09502186)(166.47985065,255.32502375)(166.47985065,257.21502186)
\lineto(166.47985065,267.71502186)
\lineto(163.92985065,267.71502186)
\lineto(163.92985065,257.51502186)
\curveto(163.92985065,254.18502519)(165.18985473,251.84502186)(169.26985065,251.84502186)
\curveto(171.48984843,251.84502186)(173.22985173,252.74502378)(174.30985065,254.66502186)
\lineto(174.36985065,254.66502186)
\lineto(174.36985065,252.20502186)
\lineto(176.76985065,252.20502186)
\lineto(176.76985065,267.71502186)
}
}
{
\newrgbcolor{curcolor}{0 0 0}
\pscustom[linestyle=none,fillstyle=solid,fillcolor=curcolor]
{
\newpath
\moveto(192.7190694,260.09502186)
\curveto(192.7190694,257.06502489)(191.54906592,254.09502186)(188.0690694,254.09502186)
\curveto(184.55907291,254.09502186)(183.1790694,256.91502492)(183.1790694,259.97502186)
\curveto(183.1790694,262.88501895)(184.49907282,265.82502186)(187.9190694,265.82502186)
\curveto(191.2190661,265.82502186)(192.7190694,263.00501895)(192.7190694,260.09502186)
\moveto(180.7190694,246.26502186)
\lineto(183.2690694,246.26502186)
\lineto(183.2690694,254.27502186)
\lineto(183.3290694,254.27502186)
\curveto(184.46906826,252.44502369)(186.74907099,251.84502186)(188.3390694,251.84502186)
\curveto(193.07906466,251.84502186)(195.4190694,255.53502624)(195.4190694,259.91502186)
\curveto(195.4190694,264.29501748)(193.04906463,268.07502186)(188.2790694,268.07502186)
\curveto(186.14907153,268.07502186)(184.16906856,267.32502015)(183.3290694,265.61502186)
\lineto(183.2690694,265.61502186)
\lineto(183.2690694,267.71502186)
\lineto(180.7190694,267.71502186)
\lineto(180.7190694,246.26502186)
}
}
{
\newrgbcolor{curcolor}{0 0 0}
\pscustom[linestyle=none,fillstyle=solid,fillcolor=curcolor]
{
\newpath
\moveto(197.6015694,259.94502186)
\curveto(197.6015694,255.41502639)(200.21157432,251.84502186)(205.1315694,251.84502186)
\curveto(210.05156448,251.84502186)(212.6615694,255.41502639)(212.6615694,259.94502186)
\curveto(212.6615694,264.5050173)(210.05156448,268.07502186)(205.1315694,268.07502186)
\curveto(200.21157432,268.07502186)(197.6015694,264.5050173)(197.6015694,259.94502186)
\moveto(200.3015694,259.94502186)
\curveto(200.3015694,263.72501808)(202.46157207,265.82502186)(205.1315694,265.82502186)
\curveto(207.80156673,265.82502186)(209.9615694,263.72501808)(209.9615694,259.94502186)
\curveto(209.9615694,256.19502561)(207.80156673,254.09502186)(205.1315694,254.09502186)
\curveto(202.46157207,254.09502186)(200.3015694,256.19502561)(200.3015694,259.94502186)
}
}
{
\newrgbcolor{curcolor}{0 0 0}
\pscustom[linestyle=none,fillstyle=solid,fillcolor=curcolor]
{
\newpath
\moveto(162.46053791,194.11804676)
\lineto(177.34053791,194.11804676)
\lineto(177.34053791,196.51804676)
\lineto(165.31053791,196.51804676)
\lineto(165.31053791,203.92804676)
\lineto(176.44053791,203.92804676)
\lineto(176.44053791,206.32804676)
\lineto(165.31053791,206.32804676)
\lineto(165.31053791,213.13804676)
\lineto(177.25053791,213.13804676)
\lineto(177.25053791,215.53804676)
\lineto(162.46053791,215.53804676)
\lineto(162.46053791,194.11804676)
}
}
{
\newrgbcolor{curcolor}{0 0 0}
\pscustom[linestyle=none,fillstyle=solid,fillcolor=curcolor]
{
\newpath
\moveto(194.24038166,209.62804676)
\lineto(191.69038166,209.62804676)
\lineto(191.69038166,207.55804676)
\lineto(191.63038166,207.55804676)
\curveto(190.4903828,209.38804493)(188.21038007,209.98804676)(186.62038166,209.98804676)
\curveto(181.8803864,209.98804676)(179.54038166,206.29804238)(179.54038166,201.91804676)
\curveto(179.54038166,197.53805114)(181.91038643,193.75804676)(186.68038166,193.75804676)
\curveto(188.81037953,193.75804676)(190.7903825,194.50804847)(191.63038166,196.21804676)
\lineto(191.69038166,196.21804676)
\lineto(191.69038166,188.17804676)
\lineto(194.24038166,188.17804676)
\lineto(194.24038166,209.62804676)
\moveto(182.24038166,201.73804676)
\curveto(182.24038166,204.76804373)(183.41038514,207.73804676)(186.89038166,207.73804676)
\curveto(190.40037815,207.73804676)(191.78038166,204.9180437)(191.78038166,201.85804676)
\curveto(191.78038166,198.94804967)(190.46037824,196.00804676)(187.04038166,196.00804676)
\curveto(183.74038496,196.00804676)(182.24038166,198.82804967)(182.24038166,201.73804676)
}
}
{
\newrgbcolor{curcolor}{0 0 0}
\pscustom[linestyle=none,fillstyle=solid,fillcolor=curcolor]
{
\newpath
\moveto(211.03288166,209.62804676)
\lineto(208.48288166,209.62804676)
\lineto(208.48288166,200.86804676)
\curveto(208.48288166,198.07804955)(206.98287857,196.00804676)(203.89288166,196.00804676)
\curveto(201.94288361,196.00804676)(200.74288166,197.23804865)(200.74288166,199.12804676)
\lineto(200.74288166,209.62804676)
\lineto(198.19288166,209.62804676)
\lineto(198.19288166,199.42804676)
\curveto(198.19288166,196.09805009)(199.45288574,193.75804676)(203.53288166,193.75804676)
\curveto(205.75287944,193.75804676)(207.49288274,194.65804868)(208.57288166,196.57804676)
\lineto(208.63288166,196.57804676)
\lineto(208.63288166,194.11804676)
\lineto(211.03288166,194.11804676)
\lineto(211.03288166,209.62804676)
}
}
{
\newrgbcolor{curcolor}{0 0 0}
\pscustom[linestyle=none,fillstyle=solid,fillcolor=curcolor]
{
\newpath
\moveto(215.04210041,194.11804676)
\lineto(217.59210041,194.11804676)
\lineto(217.59210041,209.62804676)
\lineto(215.04210041,209.62804676)
\lineto(215.04210041,194.11804676)
\moveto(217.59210041,215.53804676)
\lineto(215.04210041,215.53804676)
\lineto(215.04210041,212.41804676)
\lineto(217.59210041,212.41804676)
\lineto(217.59210041,215.53804676)
}
}
{
\newrgbcolor{curcolor}{0 0 0}
\pscustom[linestyle=none,fillstyle=solid,fillcolor=curcolor]
{
\newpath
\moveto(233.66178791,202.00804676)
\curveto(233.66178791,198.97804979)(232.49178443,196.00804676)(229.01178791,196.00804676)
\curveto(225.50179142,196.00804676)(224.12178791,198.82804982)(224.12178791,201.88804676)
\curveto(224.12178791,204.79804385)(225.44179133,207.73804676)(228.86178791,207.73804676)
\curveto(232.16178461,207.73804676)(233.66178791,204.91804385)(233.66178791,202.00804676)
\moveto(221.66178791,188.17804676)
\lineto(224.21178791,188.17804676)
\lineto(224.21178791,196.18804676)
\lineto(224.27178791,196.18804676)
\curveto(225.41178677,194.35804859)(227.6917895,193.75804676)(229.28178791,193.75804676)
\curveto(234.02178317,193.75804676)(236.36178791,197.44805114)(236.36178791,201.82804676)
\curveto(236.36178791,206.20804238)(233.99178314,209.98804676)(229.22178791,209.98804676)
\curveto(227.09179004,209.98804676)(225.11178707,209.23804505)(224.27178791,207.52804676)
\lineto(224.21178791,207.52804676)
\lineto(224.21178791,209.62804676)
\lineto(221.66178791,209.62804676)
\lineto(221.66178791,188.17804676)
}
}
{
\newrgbcolor{curcolor}{0 0 0}
\pscustom[linestyle=none,fillstyle=solid,fillcolor=curcolor]
{
\newpath
\moveto(238.54428791,201.85804676)
\curveto(238.54428791,197.32805129)(241.15429283,193.75804676)(246.07428791,193.75804676)
\curveto(250.99428299,193.75804676)(253.60428791,197.32805129)(253.60428791,201.85804676)
\curveto(253.60428791,206.4180422)(250.99428299,209.98804676)(246.07428791,209.98804676)
\curveto(241.15429283,209.98804676)(238.54428791,206.4180422)(238.54428791,201.85804676)
\moveto(241.24428791,201.85804676)
\curveto(241.24428791,205.63804298)(243.40429058,207.73804676)(246.07428791,207.73804676)
\curveto(248.74428524,207.73804676)(250.90428791,205.63804298)(250.90428791,201.85804676)
\curveto(250.90428791,198.10805051)(248.74428524,196.00804676)(246.07428791,196.00804676)
\curveto(243.40429058,196.00804676)(241.24428791,198.10805051)(241.24428791,201.85804676)
}
}
{
\newrgbcolor{curcolor}{0 0 0}
\pscustom[linewidth=2.65748024,linecolor=curcolor]
{
\newpath
\moveto(149.79512897,227.43689442)
\lineto(266.26969257,227.43689442)
\lineto(266.26969257,176.27921009)
\lineto(149.79512897,176.27921009)
\closepath
}
}
{
\newrgbcolor{curcolor}{0 0 0}
\pscustom[linewidth=2.65748024,linecolor=curcolor]
{
\newpath
\moveto(121.62025562,284.20666981)
\lineto(276.46554676,284.20666981)
\lineto(276.46554676,164.87876606)
\lineto(121.62025562,164.87876606)
\closepath
}
}
{
\newrgbcolor{curcolor}{0 0 0}
\pscustom[linewidth=2.65748024,linecolor=curcolor]
{
\newpath
\moveto(71.49931446,337.21338368)
\lineto(289.23117176,337.21338368)
\lineto(289.23117176,152.97859669)
\lineto(71.49931446,152.97859669)
\closepath
}
}
{
\newrgbcolor{curcolor}{0 0 0}
\pscustom[linewidth=2.65748024,linecolor=curcolor]
{
\newpath
\moveto(21.24184147,392.00745106)
\lineto(301.73921696,392.00745106)
\lineto(301.73921696,139.45129872)
\lineto(21.24184147,139.45129872)
\closepath
}
}
{
\newrgbcolor{curcolor}{0 0 0}
\pscustom[linewidth=2.48031497,linecolor=curcolor,linestyle=dashed,dash=2.48031496 9.92125984]
{
\newpath
\moveto(318.73283,412.24300838)
\lineto(318.73283,22.04274838)
}
}
{
\newrgbcolor{curcolor}{0 0 0}
\pscustom[linestyle=none,fillstyle=solid,fillcolor=curcolor]
{
\newpath
\moveto(76.25002533,61.54572387)
\lineto(86.17002533,61.54572387)
\lineto(86.17002533,63.14572387)
\lineto(78.15002533,63.14572387)
\lineto(78.15002533,68.08572387)
\lineto(85.57002533,68.08572387)
\lineto(85.57002533,69.68572387)
\lineto(78.15002533,69.68572387)
\lineto(78.15002533,74.22572387)
\lineto(86.11002533,74.22572387)
\lineto(86.11002533,75.82572387)
\lineto(76.25002533,75.82572387)
\lineto(76.25002533,61.54572387)
}
}
{
\newrgbcolor{curcolor}{0 0 0}
\pscustom[linestyle=none,fillstyle=solid,fillcolor=curcolor]
{
\newpath
\moveto(87.53658783,64.80572387)
\curveto(87.63658773,62.24572643)(89.59659015,61.30572387)(91.91658783,61.30572387)
\curveto(94.01658573,61.30572387)(96.31658783,62.10572633)(96.31658783,64.56572387)
\curveto(96.31658783,66.56572187)(94.63658613,67.12572425)(92.93658783,67.50572387)
\curveto(91.35658941,67.88572349)(89.55658783,68.08572509)(89.55658783,69.30572387)
\curveto(89.55658783,70.34572283)(90.73658885,70.62572387)(91.75658783,70.62572387)
\curveto(92.87658671,70.62572387)(94.03658795,70.20572255)(94.15658783,68.88572387)
\lineto(95.85658783,68.88572387)
\curveto(95.71658797,71.40572135)(93.89658555,72.12572387)(91.61658783,72.12572387)
\curveto(89.81658963,72.12572387)(87.75658783,71.26572179)(87.75658783,69.18572387)
\curveto(87.75658783,67.20572585)(89.45658951,66.64572349)(91.13658783,66.26572387)
\curveto(92.83658613,65.88572425)(94.51658783,65.66572255)(94.51658783,64.34572387)
\curveto(94.51658783,63.04572517)(93.07658677,62.80572387)(92.01658783,62.80572387)
\curveto(90.61658923,62.80572387)(89.29658777,63.28572539)(89.23658783,64.80572387)
\lineto(87.53658783,64.80572387)
}
}
{
\newrgbcolor{curcolor}{0 0 0}
\pscustom[linestyle=none,fillstyle=solid,fillcolor=curcolor]
{
\newpath
\moveto(106.25658783,66.80572387)
\curveto(106.25658783,64.78572589)(105.47658551,62.80572387)(103.15658783,62.80572387)
\curveto(100.81659017,62.80572387)(99.89658783,64.68572591)(99.89658783,66.72572387)
\curveto(99.89658783,68.66572193)(100.77659011,70.62572387)(103.05658783,70.62572387)
\curveto(105.25658563,70.62572387)(106.25658783,68.74572193)(106.25658783,66.80572387)
\moveto(98.25658783,57.58572387)
\lineto(99.95658783,57.58572387)
\lineto(99.95658783,62.92572387)
\lineto(99.99658783,62.92572387)
\curveto(100.75658707,61.70572509)(102.27658889,61.30572387)(103.33658783,61.30572387)
\curveto(106.49658467,61.30572387)(108.05658783,63.76572679)(108.05658783,66.68572387)
\curveto(108.05658783,69.60572095)(106.47658465,72.12572387)(103.29658783,72.12572387)
\curveto(101.87658925,72.12572387)(100.55658727,71.62572273)(99.99658783,70.48572387)
\lineto(99.95658783,70.48572387)
\lineto(99.95658783,71.88572387)
\lineto(98.25658783,71.88572387)
\lineto(98.25658783,57.58572387)
}
}
{
\newrgbcolor{curcolor}{0 0 0}
\pscustom[linestyle=none,fillstyle=solid,fillcolor=curcolor]
{
\newpath
\moveto(116.47158783,65.14572387)
\curveto(116.47158783,64.20572481)(115.55158559,62.80572387)(113.31158783,62.80572387)
\curveto(112.27158887,62.80572387)(111.31158783,63.20572499)(111.31158783,64.32572387)
\curveto(111.31158783,65.58572261)(112.27158895,65.98572407)(113.39158783,66.18572387)
\curveto(114.53158669,66.38572367)(115.81158849,66.40572435)(116.47158783,66.88572387)
\lineto(116.47158783,65.14572387)
\moveto(119.23158783,62.90572387)
\curveto(119.01158805,62.82572395)(118.85158769,62.80572387)(118.71158783,62.80572387)
\curveto(118.17158837,62.80572387)(118.17158783,63.16572467)(118.17158783,63.96572387)
\lineto(118.17158783,69.28572387)
\curveto(118.17158783,71.70572145)(116.15158597,72.12572387)(114.29158783,72.12572387)
\curveto(111.99159013,72.12572387)(110.01158773,71.22572131)(109.91158783,68.66572387)
\lineto(111.61158783,68.66572387)
\curveto(111.69158775,70.18572235)(112.75158927,70.62572387)(114.19158783,70.62572387)
\curveto(115.27158675,70.62572387)(116.49158783,70.38572239)(116.49158783,68.90572387)
\curveto(116.49158783,67.62572515)(114.89158595,67.74572351)(113.01158783,67.38572387)
\curveto(111.25158959,67.04572421)(109.51158783,66.54572153)(109.51158783,64.20572387)
\curveto(109.51158783,62.14572593)(111.05158971,61.30572387)(112.93158783,61.30572387)
\curveto(114.37158639,61.30572387)(115.63158877,61.80572497)(116.57158783,62.90572387)
\curveto(116.57158783,61.78572499)(117.13158871,61.30572387)(118.01158783,61.30572387)
\curveto(118.55158729,61.30572387)(118.93158813,61.40572405)(119.23158783,61.58572387)
\lineto(119.23158783,62.90572387)
}
}
{
\newrgbcolor{curcolor}{0 0 0}
\pscustom[linestyle=none,fillstyle=solid,fillcolor=curcolor]
{
\newpath
\moveto(129.59377533,68.56572387)
\curveto(129.35377557,71.02572141)(127.47377299,72.12572387)(125.13377533,72.12572387)
\curveto(121.85377861,72.12572387)(120.25377533,69.68572077)(120.25377533,66.58572387)
\curveto(120.25377533,63.50572695)(121.93377849,61.30572387)(125.09377533,61.30572387)
\curveto(127.69377273,61.30572387)(129.27377571,62.80572639)(129.65377533,65.32572387)
\lineto(127.91377533,65.32572387)
\curveto(127.69377555,63.76572543)(126.71377369,62.80572387)(125.07377533,62.80572387)
\curveto(122.91377749,62.80572387)(122.05377533,64.68572577)(122.05377533,66.58572387)
\curveto(122.05377533,68.68572177)(122.81377779,70.62572387)(125.27377533,70.62572387)
\curveto(126.67377393,70.62572387)(127.57377559,69.86572257)(127.83377533,68.56572387)
\lineto(129.59377533,68.56572387)
}
}
{
\newrgbcolor{curcolor}{0 0 0}
\pscustom[linestyle=none,fillstyle=solid,fillcolor=curcolor]
{
\newpath
\moveto(131.65596283,61.54572387)
\lineto(133.35596283,61.54572387)
\lineto(133.35596283,71.88572387)
\lineto(131.65596283,71.88572387)
\lineto(131.65596283,61.54572387)
\moveto(133.35596283,75.82572387)
\lineto(131.65596283,75.82572387)
\lineto(131.65596283,73.74572387)
\lineto(133.35596283,73.74572387)
\lineto(133.35596283,75.82572387)
}
}
{
\newrgbcolor{curcolor}{0 0 0}
\pscustom[linestyle=none,fillstyle=solid,fillcolor=curcolor]
{
\newpath
\moveto(135.44908783,66.70572387)
\curveto(135.44908783,63.68572689)(137.18909111,61.30572387)(140.46908783,61.30572387)
\curveto(143.74908455,61.30572387)(145.48908783,63.68572689)(145.48908783,66.70572387)
\curveto(145.48908783,69.74572083)(143.74908455,72.12572387)(140.46908783,72.12572387)
\curveto(137.18909111,72.12572387)(135.44908783,69.74572083)(135.44908783,66.70572387)
\moveto(137.24908783,66.70572387)
\curveto(137.24908783,69.22572135)(138.68908961,70.62572387)(140.46908783,70.62572387)
\curveto(142.24908605,70.62572387)(143.68908783,69.22572135)(143.68908783,66.70572387)
\curveto(143.68908783,64.20572637)(142.24908605,62.80572387)(140.46908783,62.80572387)
\curveto(138.68908961,62.80572387)(137.24908783,64.20572637)(137.24908783,66.70572387)
}
}
{
\newrgbcolor{curcolor}{0 0 0}
\pscustom[linestyle=none,fillstyle=solid,fillcolor=curcolor]
{
\newpath
\moveto(146.83346283,64.80572387)
\curveto(146.93346273,62.24572643)(148.89346515,61.30572387)(151.21346283,61.30572387)
\curveto(153.31346073,61.30572387)(155.61346283,62.10572633)(155.61346283,64.56572387)
\curveto(155.61346283,66.56572187)(153.93346113,67.12572425)(152.23346283,67.50572387)
\curveto(150.65346441,67.88572349)(148.85346283,68.08572509)(148.85346283,69.30572387)
\curveto(148.85346283,70.34572283)(150.03346385,70.62572387)(151.05346283,70.62572387)
\curveto(152.17346171,70.62572387)(153.33346295,70.20572255)(153.45346283,68.88572387)
\lineto(155.15346283,68.88572387)
\curveto(155.01346297,71.40572135)(153.19346055,72.12572387)(150.91346283,72.12572387)
\curveto(149.11346463,72.12572387)(147.05346283,71.26572179)(147.05346283,69.18572387)
\curveto(147.05346283,67.20572585)(148.75346451,66.64572349)(150.43346283,66.26572387)
\curveto(152.13346113,65.88572425)(153.81346283,65.66572255)(153.81346283,64.34572387)
\curveto(153.81346283,63.04572517)(152.37346177,62.80572387)(151.31346283,62.80572387)
\curveto(149.91346423,62.80572387)(148.59346277,63.28572539)(148.53346283,64.80572387)
\lineto(146.83346283,64.80572387)
}
}
{
\newrgbcolor{curcolor}{0 0 0}
\pscustom[linestyle=none,fillstyle=solid,fillcolor=curcolor]
{
\newpath
\moveto(78.33002533,49.98572387)
\lineto(76.63002533,49.98572387)
\lineto(76.63002533,46.88572387)
\lineto(74.87002533,46.88572387)
\lineto(74.87002533,45.38572387)
\lineto(76.63002533,45.38572387)
\lineto(76.63002533,38.80572387)
\curveto(76.63002533,36.90572577)(77.33002709,36.54572387)(79.09002533,36.54572387)
\lineto(80.39002533,36.54572387)
\lineto(80.39002533,38.04572387)
\lineto(79.61002533,38.04572387)
\curveto(78.55002639,38.04572387)(78.33002533,38.18572465)(78.33002533,38.96572387)
\lineto(78.33002533,45.38572387)
\lineto(80.39002533,45.38572387)
\lineto(80.39002533,46.88572387)
\lineto(78.33002533,46.88572387)
\lineto(78.33002533,49.98572387)
}
}
{
\newrgbcolor{curcolor}{0 0 0}
\pscustom[linestyle=none,fillstyle=solid,fillcolor=curcolor]
{
\newpath
\moveto(89.39908783,39.82572387)
\curveto(89.09908813,38.48572521)(88.11908643,37.80572387)(86.71908783,37.80572387)
\curveto(84.45909009,37.80572387)(83.43908789,39.40572567)(83.49908783,41.20572387)
\lineto(91.23908783,41.20572387)
\curveto(91.33908773,43.70572137)(90.21908417,47.12572387)(86.55908783,47.12572387)
\curveto(83.73909065,47.12572387)(81.69908783,44.84572077)(81.69908783,41.74572387)
\curveto(81.79908773,38.58572703)(83.35909113,36.30572387)(86.65908783,36.30572387)
\curveto(88.97908551,36.30572387)(90.61908829,37.54572615)(91.07908783,39.82572387)
\lineto(89.39908783,39.82572387)
\moveto(83.49908783,42.70572387)
\curveto(83.61908771,44.28572229)(84.67908961,45.62572387)(86.45908783,45.62572387)
\curveto(88.13908615,45.62572387)(89.35908791,44.32572225)(89.43908783,42.70572387)
\lineto(83.49908783,42.70572387)
}
}
{
\newrgbcolor{curcolor}{0 0 0}
\pscustom[linestyle=none,fillstyle=solid,fillcolor=curcolor]
{
\newpath
\moveto(93.00127533,36.54572387)
\lineto(94.70127533,36.54572387)
\lineto(94.70127533,42.98572387)
\curveto(94.70127533,43.76572309)(95.44127735,45.62572387)(97.46127533,45.62572387)
\curveto(98.98127381,45.62572387)(99.40127533,44.66572253)(99.40127533,43.32572387)
\lineto(99.40127533,36.54572387)
\lineto(101.10127533,36.54572387)
\lineto(101.10127533,42.98572387)
\curveto(101.10127533,44.58572227)(102.16127697,45.62572387)(103.80127533,45.62572387)
\curveto(105.46127367,45.62572387)(105.80127533,44.60572259)(105.80127533,43.32572387)
\lineto(105.80127533,36.54572387)
\lineto(107.50127533,36.54572387)
\lineto(107.50127533,44.12572387)
\curveto(107.50127533,46.26572173)(106.12127327,47.12572387)(104.06127533,47.12572387)
\curveto(102.74127665,47.12572387)(101.52127463,46.46572277)(100.82127533,45.36572387)
\curveto(100.40127575,46.62572261)(99.24127407,47.12572387)(97.98127533,47.12572387)
\curveto(96.56127675,47.12572387)(95.40127457,46.52572271)(94.64127533,45.36572387)
\lineto(94.60127533,45.36572387)
\lineto(94.60127533,46.88572387)
\lineto(93.00127533,46.88572387)
\lineto(93.00127533,36.54572387)
}
}
{
\newrgbcolor{curcolor}{0 0 0}
\pscustom[linestyle=none,fillstyle=solid,fillcolor=curcolor]
{
\newpath
\moveto(118.13158783,41.80572387)
\curveto(118.13158783,39.78572589)(117.35158551,37.80572387)(115.03158783,37.80572387)
\curveto(112.69159017,37.80572387)(111.77158783,39.68572591)(111.77158783,41.72572387)
\curveto(111.77158783,43.66572193)(112.65159011,45.62572387)(114.93158783,45.62572387)
\curveto(117.13158563,45.62572387)(118.13158783,43.74572193)(118.13158783,41.80572387)
\moveto(110.13158783,32.58572387)
\lineto(111.83158783,32.58572387)
\lineto(111.83158783,37.92572387)
\lineto(111.87158783,37.92572387)
\curveto(112.63158707,36.70572509)(114.15158889,36.30572387)(115.21158783,36.30572387)
\curveto(118.37158467,36.30572387)(119.93158783,38.76572679)(119.93158783,41.68572387)
\curveto(119.93158783,44.60572095)(118.35158465,47.12572387)(115.17158783,47.12572387)
\curveto(113.75158925,47.12572387)(112.43158727,46.62572273)(111.87158783,45.48572387)
\lineto(111.83158783,45.48572387)
\lineto(111.83158783,46.88572387)
\lineto(110.13158783,46.88572387)
\lineto(110.13158783,32.58572387)
}
}
{
\newrgbcolor{curcolor}{0 0 0}
\pscustom[linestyle=none,fillstyle=solid,fillcolor=curcolor]
{
\newpath
\moveto(121.38658783,41.70572387)
\curveto(121.38658783,38.68572689)(123.12659111,36.30572387)(126.40658783,36.30572387)
\curveto(129.68658455,36.30572387)(131.42658783,38.68572689)(131.42658783,41.70572387)
\curveto(131.42658783,44.74572083)(129.68658455,47.12572387)(126.40658783,47.12572387)
\curveto(123.12659111,47.12572387)(121.38658783,44.74572083)(121.38658783,41.70572387)
\moveto(123.18658783,41.70572387)
\curveto(123.18658783,44.22572135)(124.62658961,45.62572387)(126.40658783,45.62572387)
\curveto(128.18658605,45.62572387)(129.62658783,44.22572135)(129.62658783,41.70572387)
\curveto(129.62658783,39.20572637)(128.18658605,37.80572387)(126.40658783,37.80572387)
\curveto(124.62658961,37.80572387)(123.18658783,39.20572637)(123.18658783,41.70572387)
}
}
{
\newrgbcolor{curcolor}{0 0 0}
\pscustom[linestyle=none,fillstyle=solid,fillcolor=curcolor]
{
\newpath
\moveto(133.37096283,36.54572387)
\lineto(135.07096283,36.54572387)
\lineto(135.07096283,41.14572387)
\curveto(135.07096283,43.76572125)(136.07096557,45.32572387)(138.81096283,45.32572387)
\lineto(138.81096283,47.12572387)
\curveto(136.97096467,47.18572381)(135.83096201,46.36572221)(135.01096283,44.70572387)
\lineto(134.97096283,44.70572387)
\lineto(134.97096283,46.88572387)
\lineto(133.37096283,46.88572387)
\lineto(133.37096283,36.54572387)
}
}
{
\newrgbcolor{curcolor}{0 0 0}
\pscustom[linestyle=none,fillstyle=solid,fillcolor=curcolor]
{
\newpath
\moveto(146.51065033,40.14572387)
\curveto(146.51065033,39.20572481)(145.59064809,37.80572387)(143.35065033,37.80572387)
\curveto(142.31065137,37.80572387)(141.35065033,38.20572499)(141.35065033,39.32572387)
\curveto(141.35065033,40.58572261)(142.31065145,40.98572407)(143.43065033,41.18572387)
\curveto(144.57064919,41.38572367)(145.85065099,41.40572435)(146.51065033,41.88572387)
\lineto(146.51065033,40.14572387)
\moveto(149.27065033,37.90572387)
\curveto(149.05065055,37.82572395)(148.89065019,37.80572387)(148.75065033,37.80572387)
\curveto(148.21065087,37.80572387)(148.21065033,38.16572467)(148.21065033,38.96572387)
\lineto(148.21065033,44.28572387)
\curveto(148.21065033,46.70572145)(146.19064847,47.12572387)(144.33065033,47.12572387)
\curveto(142.03065263,47.12572387)(140.05065023,46.22572131)(139.95065033,43.66572387)
\lineto(141.65065033,43.66572387)
\curveto(141.73065025,45.18572235)(142.79065177,45.62572387)(144.23065033,45.62572387)
\curveto(145.31064925,45.62572387)(146.53065033,45.38572239)(146.53065033,43.90572387)
\curveto(146.53065033,42.62572515)(144.93064845,42.74572351)(143.05065033,42.38572387)
\curveto(141.29065209,42.04572421)(139.55065033,41.54572153)(139.55065033,39.20572387)
\curveto(139.55065033,37.14572593)(141.09065221,36.30572387)(142.97065033,36.30572387)
\curveto(144.41064889,36.30572387)(145.67065127,36.80572497)(146.61065033,37.90572387)
\curveto(146.61065033,36.78572499)(147.17065121,36.30572387)(148.05065033,36.30572387)
\curveto(148.59064979,36.30572387)(148.97065063,36.40572405)(149.27065033,36.58572387)
\lineto(149.27065033,37.90572387)
}
}
{
\newrgbcolor{curcolor}{0 0 0}
\pscustom[linestyle=none,fillstyle=solid,fillcolor=curcolor]
{
\newpath
\moveto(150.95283783,36.54572387)
\lineto(152.65283783,36.54572387)
\lineto(152.65283783,50.82572387)
\lineto(150.95283783,50.82572387)
\lineto(150.95283783,36.54572387)
}
}
{
\newrgbcolor{curcolor}{0 0 0}
\pscustom[linestyle=none,fillstyle=solid,fillcolor=curcolor]
{
\newpath
\moveto(162.44596283,39.82572387)
\curveto(162.14596313,38.48572521)(161.16596143,37.80572387)(159.76596283,37.80572387)
\curveto(157.50596509,37.80572387)(156.48596289,39.40572567)(156.54596283,41.20572387)
\lineto(164.28596283,41.20572387)
\curveto(164.38596273,43.70572137)(163.26595917,47.12572387)(159.60596283,47.12572387)
\curveto(156.78596565,47.12572387)(154.74596283,44.84572077)(154.74596283,41.74572387)
\curveto(154.84596273,38.58572703)(156.40596613,36.30572387)(159.70596283,36.30572387)
\curveto(162.02596051,36.30572387)(163.66596329,37.54572615)(164.12596283,39.82572387)
\lineto(162.44596283,39.82572387)
\moveto(156.54596283,42.70572387)
\curveto(156.66596271,44.28572229)(157.72596461,45.62572387)(159.50596283,45.62572387)
\curveto(161.18596115,45.62572387)(162.40596291,44.32572225)(162.48596283,42.70572387)
\lineto(156.54596283,42.70572387)
}
}
{
\newrgbcolor{curcolor}{0 0 0}
\pscustom[linestyle=none,fillstyle=solid,fillcolor=curcolor]
{
\newpath
\moveto(165.38815033,39.80572387)
\curveto(165.48815023,37.24572643)(167.44815265,36.30572387)(169.76815033,36.30572387)
\curveto(171.86814823,36.30572387)(174.16815033,37.10572633)(174.16815033,39.56572387)
\curveto(174.16815033,41.56572187)(172.48814863,42.12572425)(170.78815033,42.50572387)
\curveto(169.20815191,42.88572349)(167.40815033,43.08572509)(167.40815033,44.30572387)
\curveto(167.40815033,45.34572283)(168.58815135,45.62572387)(169.60815033,45.62572387)
\curveto(170.72814921,45.62572387)(171.88815045,45.20572255)(172.00815033,43.88572387)
\lineto(173.70815033,43.88572387)
\curveto(173.56815047,46.40572135)(171.74814805,47.12572387)(169.46815033,47.12572387)
\curveto(167.66815213,47.12572387)(165.60815033,46.26572179)(165.60815033,44.18572387)
\curveto(165.60815033,42.20572585)(167.30815201,41.64572349)(168.98815033,41.26572387)
\curveto(170.68814863,40.88572425)(172.36815033,40.66572255)(172.36815033,39.34572387)
\curveto(172.36815033,38.04572517)(170.92814927,37.80572387)(169.86815033,37.80572387)
\curveto(168.46815173,37.80572387)(167.14815027,38.28572539)(167.08815033,39.80572387)
\lineto(165.38815033,39.80572387)
}
}
{
\newrgbcolor{curcolor}{0 0 0}
\pscustom[linestyle=none,fillstyle=solid,fillcolor=curcolor]
{
\newpath
\moveto(382.96151978,61.54572387)
\lineto(392.88151978,61.54572387)
\lineto(392.88151978,63.14572387)
\lineto(384.86151978,63.14572387)
\lineto(384.86151978,68.08572387)
\lineto(392.28151978,68.08572387)
\lineto(392.28151978,69.68572387)
\lineto(384.86151978,69.68572387)
\lineto(384.86151978,74.22572387)
\lineto(392.82151978,74.22572387)
\lineto(392.82151978,75.82572387)
\lineto(382.96151978,75.82572387)
\lineto(382.96151978,61.54572387)
}
}
{
\newrgbcolor{curcolor}{0 0 0}
\pscustom[linestyle=none,fillstyle=solid,fillcolor=curcolor]
{
\newpath
\moveto(394.24808228,64.80572387)
\curveto(394.34808218,62.24572643)(396.3080846,61.30572387)(398.62808228,61.30572387)
\curveto(400.72808018,61.30572387)(403.02808228,62.10572633)(403.02808228,64.56572387)
\curveto(403.02808228,66.56572187)(401.34808058,67.12572425)(399.64808228,67.50572387)
\curveto(398.06808386,67.88572349)(396.26808228,68.08572509)(396.26808228,69.30572387)
\curveto(396.26808228,70.34572283)(397.4480833,70.62572387)(398.46808228,70.62572387)
\curveto(399.58808116,70.62572387)(400.7480824,70.20572255)(400.86808228,68.88572387)
\lineto(402.56808228,68.88572387)
\curveto(402.42808242,71.40572135)(400.60808,72.12572387)(398.32808228,72.12572387)
\curveto(396.52808408,72.12572387)(394.46808228,71.26572179)(394.46808228,69.18572387)
\curveto(394.46808228,67.20572585)(396.16808396,66.64572349)(397.84808228,66.26572387)
\curveto(399.54808058,65.88572425)(401.22808228,65.66572255)(401.22808228,64.34572387)
\curveto(401.22808228,63.04572517)(399.78808122,62.80572387)(398.72808228,62.80572387)
\curveto(397.32808368,62.80572387)(396.00808222,63.28572539)(395.94808228,64.80572387)
\lineto(394.24808228,64.80572387)
}
}
{
\newrgbcolor{curcolor}{0 0 0}
\pscustom[linestyle=none,fillstyle=solid,fillcolor=curcolor]
{
\newpath
\moveto(412.96808228,66.80572387)
\curveto(412.96808228,64.78572589)(412.18807996,62.80572387)(409.86808228,62.80572387)
\curveto(407.52808462,62.80572387)(406.60808228,64.68572591)(406.60808228,66.72572387)
\curveto(406.60808228,68.66572193)(407.48808456,70.62572387)(409.76808228,70.62572387)
\curveto(411.96808008,70.62572387)(412.96808228,68.74572193)(412.96808228,66.80572387)
\moveto(404.96808228,57.58572387)
\lineto(406.66808228,57.58572387)
\lineto(406.66808228,62.92572387)
\lineto(406.70808228,62.92572387)
\curveto(407.46808152,61.70572509)(408.98808334,61.30572387)(410.04808228,61.30572387)
\curveto(413.20807912,61.30572387)(414.76808228,63.76572679)(414.76808228,66.68572387)
\curveto(414.76808228,69.60572095)(413.1880791,72.12572387)(410.00808228,72.12572387)
\curveto(408.5880837,72.12572387)(407.26808172,71.62572273)(406.70808228,70.48572387)
\lineto(406.66808228,70.48572387)
\lineto(406.66808228,71.88572387)
\lineto(404.96808228,71.88572387)
\lineto(404.96808228,57.58572387)
}
}
{
\newrgbcolor{curcolor}{0 0 0}
\pscustom[linestyle=none,fillstyle=solid,fillcolor=curcolor]
{
\newpath
\moveto(423.18308228,65.14572387)
\curveto(423.18308228,64.20572481)(422.26308004,62.80572387)(420.02308228,62.80572387)
\curveto(418.98308332,62.80572387)(418.02308228,63.20572499)(418.02308228,64.32572387)
\curveto(418.02308228,65.58572261)(418.9830834,65.98572407)(420.10308228,66.18572387)
\curveto(421.24308114,66.38572367)(422.52308294,66.40572435)(423.18308228,66.88572387)
\lineto(423.18308228,65.14572387)
\moveto(425.94308228,62.90572387)
\curveto(425.7230825,62.82572395)(425.56308214,62.80572387)(425.42308228,62.80572387)
\curveto(424.88308282,62.80572387)(424.88308228,63.16572467)(424.88308228,63.96572387)
\lineto(424.88308228,69.28572387)
\curveto(424.88308228,71.70572145)(422.86308042,72.12572387)(421.00308228,72.12572387)
\curveto(418.70308458,72.12572387)(416.72308218,71.22572131)(416.62308228,68.66572387)
\lineto(418.32308228,68.66572387)
\curveto(418.4030822,70.18572235)(419.46308372,70.62572387)(420.90308228,70.62572387)
\curveto(421.9830812,70.62572387)(423.20308228,70.38572239)(423.20308228,68.90572387)
\curveto(423.20308228,67.62572515)(421.6030804,67.74572351)(419.72308228,67.38572387)
\curveto(417.96308404,67.04572421)(416.22308228,66.54572153)(416.22308228,64.20572387)
\curveto(416.22308228,62.14572593)(417.76308416,61.30572387)(419.64308228,61.30572387)
\curveto(421.08308084,61.30572387)(422.34308322,61.80572497)(423.28308228,62.90572387)
\curveto(423.28308228,61.78572499)(423.84308316,61.30572387)(424.72308228,61.30572387)
\curveto(425.26308174,61.30572387)(425.64308258,61.40572405)(425.94308228,61.58572387)
\lineto(425.94308228,62.90572387)
}
}
{
\newrgbcolor{curcolor}{0 0 0}
\pscustom[linestyle=none,fillstyle=solid,fillcolor=curcolor]
{
\newpath
\moveto(436.30526978,68.56572387)
\curveto(436.06527002,71.02572141)(434.18526744,72.12572387)(431.84526978,72.12572387)
\curveto(428.56527306,72.12572387)(426.96526978,69.68572077)(426.96526978,66.58572387)
\curveto(426.96526978,63.50572695)(428.64527294,61.30572387)(431.80526978,61.30572387)
\curveto(434.40526718,61.30572387)(435.98527016,62.80572639)(436.36526978,65.32572387)
\lineto(434.62526978,65.32572387)
\curveto(434.40527,63.76572543)(433.42526814,62.80572387)(431.78526978,62.80572387)
\curveto(429.62527194,62.80572387)(428.76526978,64.68572577)(428.76526978,66.58572387)
\curveto(428.76526978,68.68572177)(429.52527224,70.62572387)(431.98526978,70.62572387)
\curveto(433.38526838,70.62572387)(434.28527004,69.86572257)(434.54526978,68.56572387)
\lineto(436.30526978,68.56572387)
}
}
{
\newrgbcolor{curcolor}{0 0 0}
\pscustom[linestyle=none,fillstyle=solid,fillcolor=curcolor]
{
\newpath
\moveto(438.36745728,61.54572387)
\lineto(440.06745728,61.54572387)
\lineto(440.06745728,71.88572387)
\lineto(438.36745728,71.88572387)
\lineto(438.36745728,61.54572387)
\moveto(440.06745728,75.82572387)
\lineto(438.36745728,75.82572387)
\lineto(438.36745728,73.74572387)
\lineto(440.06745728,73.74572387)
\lineto(440.06745728,75.82572387)
}
}
{
\newrgbcolor{curcolor}{0 0 0}
\pscustom[linestyle=none,fillstyle=solid,fillcolor=curcolor]
{
\newpath
\moveto(442.16058228,66.70572387)
\curveto(442.16058228,63.68572689)(443.90058556,61.30572387)(447.18058228,61.30572387)
\curveto(450.460579,61.30572387)(452.20058228,63.68572689)(452.20058228,66.70572387)
\curveto(452.20058228,69.74572083)(450.460579,72.12572387)(447.18058228,72.12572387)
\curveto(443.90058556,72.12572387)(442.16058228,69.74572083)(442.16058228,66.70572387)
\moveto(443.96058228,66.70572387)
\curveto(443.96058228,69.22572135)(445.40058406,70.62572387)(447.18058228,70.62572387)
\curveto(448.9605805,70.62572387)(450.40058228,69.22572135)(450.40058228,66.70572387)
\curveto(450.40058228,64.20572637)(448.9605805,62.80572387)(447.18058228,62.80572387)
\curveto(445.40058406,62.80572387)(443.96058228,64.20572637)(443.96058228,66.70572387)
}
}
{
\newrgbcolor{curcolor}{0 0 0}
\pscustom[linestyle=none,fillstyle=solid,fillcolor=curcolor]
{
\newpath
\moveto(453.54495728,64.80572387)
\curveto(453.64495718,62.24572643)(455.6049596,61.30572387)(457.92495728,61.30572387)
\curveto(460.02495518,61.30572387)(462.32495728,62.10572633)(462.32495728,64.56572387)
\curveto(462.32495728,66.56572187)(460.64495558,67.12572425)(458.94495728,67.50572387)
\curveto(457.36495886,67.88572349)(455.56495728,68.08572509)(455.56495728,69.30572387)
\curveto(455.56495728,70.34572283)(456.7449583,70.62572387)(457.76495728,70.62572387)
\curveto(458.88495616,70.62572387)(460.0449574,70.20572255)(460.16495728,68.88572387)
\lineto(461.86495728,68.88572387)
\curveto(461.72495742,71.40572135)(459.904955,72.12572387)(457.62495728,72.12572387)
\curveto(455.82495908,72.12572387)(453.76495728,71.26572179)(453.76495728,69.18572387)
\curveto(453.76495728,67.20572585)(455.46495896,66.64572349)(457.14495728,66.26572387)
\curveto(458.84495558,65.88572425)(460.52495728,65.66572255)(460.52495728,64.34572387)
\curveto(460.52495728,63.04572517)(459.08495622,62.80572387)(458.02495728,62.80572387)
\curveto(456.62495868,62.80572387)(455.30495722,63.28572539)(455.24495728,64.80572387)
\lineto(453.54495728,64.80572387)
}
}
{
\newrgbcolor{curcolor}{0 0 0}
\pscustom[linestyle=none,fillstyle=solid,fillcolor=curcolor]
{
\newpath
\moveto(389.08151978,40.14572387)
\curveto(389.08151978,39.20572481)(388.16151754,37.80572387)(385.92151978,37.80572387)
\curveto(384.88152082,37.80572387)(383.92151978,38.20572499)(383.92151978,39.32572387)
\curveto(383.92151978,40.58572261)(384.8815209,40.98572407)(386.00151978,41.18572387)
\curveto(387.14151864,41.38572367)(388.42152044,41.40572435)(389.08151978,41.88572387)
\lineto(389.08151978,40.14572387)
\moveto(391.84151978,37.90572387)
\curveto(391.62152,37.82572395)(391.46151964,37.80572387)(391.32151978,37.80572387)
\curveto(390.78152032,37.80572387)(390.78151978,38.16572467)(390.78151978,38.96572387)
\lineto(390.78151978,44.28572387)
\curveto(390.78151978,46.70572145)(388.76151792,47.12572387)(386.90151978,47.12572387)
\curveto(384.60152208,47.12572387)(382.62151968,46.22572131)(382.52151978,43.66572387)
\lineto(384.22151978,43.66572387)
\curveto(384.3015197,45.18572235)(385.36152122,45.62572387)(386.80151978,45.62572387)
\curveto(387.8815187,45.62572387)(389.10151978,45.38572239)(389.10151978,43.90572387)
\curveto(389.10151978,42.62572515)(387.5015179,42.74572351)(385.62151978,42.38572387)
\curveto(383.86152154,42.04572421)(382.12151978,41.54572153)(382.12151978,39.20572387)
\curveto(382.12151978,37.14572593)(383.66152166,36.30572387)(385.54151978,36.30572387)
\curveto(386.98151834,36.30572387)(388.24152072,36.80572497)(389.18151978,37.90572387)
\curveto(389.18151978,36.78572499)(389.74152066,36.30572387)(390.62151978,36.30572387)
\curveto(391.16151924,36.30572387)(391.54152008,36.40572405)(391.84151978,36.58572387)
\lineto(391.84151978,37.90572387)
}
}
{
\newrgbcolor{curcolor}{0 0 0}
\pscustom[linestyle=none,fillstyle=solid,fillcolor=curcolor]
{
\newpath
\moveto(395.78370728,49.98572387)
\lineto(394.08370728,49.98572387)
\lineto(394.08370728,46.88572387)
\lineto(392.32370728,46.88572387)
\lineto(392.32370728,45.38572387)
\lineto(394.08370728,45.38572387)
\lineto(394.08370728,38.80572387)
\curveto(394.08370728,36.90572577)(394.78370904,36.54572387)(396.54370728,36.54572387)
\lineto(397.84370728,36.54572387)
\lineto(397.84370728,38.04572387)
\lineto(397.06370728,38.04572387)
\curveto(396.00370834,38.04572387)(395.78370728,38.18572465)(395.78370728,38.96572387)
\lineto(395.78370728,45.38572387)
\lineto(397.84370728,45.38572387)
\lineto(397.84370728,46.88572387)
\lineto(395.78370728,46.88572387)
\lineto(395.78370728,49.98572387)
}
}
{
\newrgbcolor{curcolor}{0 0 0}
\pscustom[linestyle=none,fillstyle=solid,fillcolor=curcolor]
{
\newpath
\moveto(406.85276978,39.82572387)
\curveto(406.55277008,38.48572521)(405.57276838,37.80572387)(404.17276978,37.80572387)
\curveto(401.91277204,37.80572387)(400.89276984,39.40572567)(400.95276978,41.20572387)
\lineto(408.69276978,41.20572387)
\curveto(408.79276968,43.70572137)(407.67276612,47.12572387)(404.01276978,47.12572387)
\curveto(401.1927726,47.12572387)(399.15276978,44.84572077)(399.15276978,41.74572387)
\curveto(399.25276968,38.58572703)(400.81277308,36.30572387)(404.11276978,36.30572387)
\curveto(406.43276746,36.30572387)(408.07277024,37.54572615)(408.53276978,39.82572387)
\lineto(406.85276978,39.82572387)
\moveto(400.95276978,42.70572387)
\curveto(401.07276966,44.28572229)(402.13277156,45.62572387)(403.91276978,45.62572387)
\curveto(405.5927681,45.62572387)(406.81276986,44.32572225)(406.89276978,42.70572387)
\lineto(400.95276978,42.70572387)
}
}
{
\newrgbcolor{curcolor}{0 0 0}
\pscustom[linestyle=none,fillstyle=solid,fillcolor=curcolor]
{
\newpath
\moveto(410.45495728,36.54572387)
\lineto(412.15495728,36.54572387)
\lineto(412.15495728,42.98572387)
\curveto(412.15495728,43.76572309)(412.8949593,45.62572387)(414.91495728,45.62572387)
\curveto(416.43495576,45.62572387)(416.85495728,44.66572253)(416.85495728,43.32572387)
\lineto(416.85495728,36.54572387)
\lineto(418.55495728,36.54572387)
\lineto(418.55495728,42.98572387)
\curveto(418.55495728,44.58572227)(419.61495892,45.62572387)(421.25495728,45.62572387)
\curveto(422.91495562,45.62572387)(423.25495728,44.60572259)(423.25495728,43.32572387)
\lineto(423.25495728,36.54572387)
\lineto(424.95495728,36.54572387)
\lineto(424.95495728,44.12572387)
\curveto(424.95495728,46.26572173)(423.57495522,47.12572387)(421.51495728,47.12572387)
\curveto(420.1949586,47.12572387)(418.97495658,46.46572277)(418.27495728,45.36572387)
\curveto(417.8549577,46.62572261)(416.69495602,47.12572387)(415.43495728,47.12572387)
\curveto(414.0149587,47.12572387)(412.85495652,46.52572271)(412.09495728,45.36572387)
\lineto(412.05495728,45.36572387)
\lineto(412.05495728,46.88572387)
\lineto(410.45495728,46.88572387)
\lineto(410.45495728,36.54572387)
}
}
{
\newrgbcolor{curcolor}{0 0 0}
\pscustom[linestyle=none,fillstyle=solid,fillcolor=curcolor]
{
\newpath
\moveto(435.58526978,41.80572387)
\curveto(435.58526978,39.78572589)(434.80526746,37.80572387)(432.48526978,37.80572387)
\curveto(430.14527212,37.80572387)(429.22526978,39.68572591)(429.22526978,41.72572387)
\curveto(429.22526978,43.66572193)(430.10527206,45.62572387)(432.38526978,45.62572387)
\curveto(434.58526758,45.62572387)(435.58526978,43.74572193)(435.58526978,41.80572387)
\moveto(427.58526978,32.58572387)
\lineto(429.28526978,32.58572387)
\lineto(429.28526978,37.92572387)
\lineto(429.32526978,37.92572387)
\curveto(430.08526902,36.70572509)(431.60527084,36.30572387)(432.66526978,36.30572387)
\curveto(435.82526662,36.30572387)(437.38526978,38.76572679)(437.38526978,41.68572387)
\curveto(437.38526978,44.60572095)(435.8052666,47.12572387)(432.62526978,47.12572387)
\curveto(431.2052712,47.12572387)(429.88526922,46.62572273)(429.32526978,45.48572387)
\lineto(429.28526978,45.48572387)
\lineto(429.28526978,46.88572387)
\lineto(427.58526978,46.88572387)
\lineto(427.58526978,32.58572387)
}
}
{
\newrgbcolor{curcolor}{0 0 0}
\pscustom[linestyle=none,fillstyle=solid,fillcolor=curcolor]
{
\newpath
\moveto(438.84026978,41.70572387)
\curveto(438.84026978,38.68572689)(440.58027306,36.30572387)(443.86026978,36.30572387)
\curveto(447.1402665,36.30572387)(448.88026978,38.68572689)(448.88026978,41.70572387)
\curveto(448.88026978,44.74572083)(447.1402665,47.12572387)(443.86026978,47.12572387)
\curveto(440.58027306,47.12572387)(438.84026978,44.74572083)(438.84026978,41.70572387)
\moveto(440.64026978,41.70572387)
\curveto(440.64026978,44.22572135)(442.08027156,45.62572387)(443.86026978,45.62572387)
\curveto(445.640268,45.62572387)(447.08026978,44.22572135)(447.08026978,41.70572387)
\curveto(447.08026978,39.20572637)(445.640268,37.80572387)(443.86026978,37.80572387)
\curveto(442.08027156,37.80572387)(440.64026978,39.20572637)(440.64026978,41.70572387)
}
}
{
\newrgbcolor{curcolor}{0 0 0}
\pscustom[linestyle=none,fillstyle=solid,fillcolor=curcolor]
{
\newpath
\moveto(450.82464478,36.54572387)
\lineto(452.52464478,36.54572387)
\lineto(452.52464478,41.14572387)
\curveto(452.52464478,43.76572125)(453.52464752,45.32572387)(456.26464478,45.32572387)
\lineto(456.26464478,47.12572387)
\curveto(454.42464662,47.18572381)(453.28464396,46.36572221)(452.46464478,44.70572387)
\lineto(452.42464478,44.70572387)
\lineto(452.42464478,46.88572387)
\lineto(450.82464478,46.88572387)
\lineto(450.82464478,36.54572387)
}
}
{
\newrgbcolor{curcolor}{0 0 0}
\pscustom[linestyle=none,fillstyle=solid,fillcolor=curcolor]
{
\newpath
\moveto(463.96433228,40.14572387)
\curveto(463.96433228,39.20572481)(463.04433004,37.80572387)(460.80433228,37.80572387)
\curveto(459.76433332,37.80572387)(458.80433228,38.20572499)(458.80433228,39.32572387)
\curveto(458.80433228,40.58572261)(459.7643334,40.98572407)(460.88433228,41.18572387)
\curveto(462.02433114,41.38572367)(463.30433294,41.40572435)(463.96433228,41.88572387)
\lineto(463.96433228,40.14572387)
\moveto(466.72433228,37.90572387)
\curveto(466.5043325,37.82572395)(466.34433214,37.80572387)(466.20433228,37.80572387)
\curveto(465.66433282,37.80572387)(465.66433228,38.16572467)(465.66433228,38.96572387)
\lineto(465.66433228,44.28572387)
\curveto(465.66433228,46.70572145)(463.64433042,47.12572387)(461.78433228,47.12572387)
\curveto(459.48433458,47.12572387)(457.50433218,46.22572131)(457.40433228,43.66572387)
\lineto(459.10433228,43.66572387)
\curveto(459.1843322,45.18572235)(460.24433372,45.62572387)(461.68433228,45.62572387)
\curveto(462.7643312,45.62572387)(463.98433228,45.38572239)(463.98433228,43.90572387)
\curveto(463.98433228,42.62572515)(462.3843304,42.74572351)(460.50433228,42.38572387)
\curveto(458.74433404,42.04572421)(457.00433228,41.54572153)(457.00433228,39.20572387)
\curveto(457.00433228,37.14572593)(458.54433416,36.30572387)(460.42433228,36.30572387)
\curveto(461.86433084,36.30572387)(463.12433322,36.80572497)(464.06433228,37.90572387)
\curveto(464.06433228,36.78572499)(464.62433316,36.30572387)(465.50433228,36.30572387)
\curveto(466.04433174,36.30572387)(466.42433258,36.40572405)(466.72433228,36.58572387)
\lineto(466.72433228,37.90572387)
}
}
{
\newrgbcolor{curcolor}{0 0 0}
\pscustom[linestyle=none,fillstyle=solid,fillcolor=curcolor]
{
\newpath
\moveto(468.40651978,36.54572387)
\lineto(470.10651978,36.54572387)
\lineto(470.10651978,50.82572387)
\lineto(468.40651978,50.82572387)
\lineto(468.40651978,36.54572387)
}
}
{
\newrgbcolor{curcolor}{0 0 0}
\pscustom[linestyle=none,fillstyle=solid,fillcolor=curcolor]
{
\newpath
\moveto(479.89964478,39.82572387)
\curveto(479.59964508,38.48572521)(478.61964338,37.80572387)(477.21964478,37.80572387)
\curveto(474.95964704,37.80572387)(473.93964484,39.40572567)(473.99964478,41.20572387)
\lineto(481.73964478,41.20572387)
\curveto(481.83964468,43.70572137)(480.71964112,47.12572387)(477.05964478,47.12572387)
\curveto(474.2396476,47.12572387)(472.19964478,44.84572077)(472.19964478,41.74572387)
\curveto(472.29964468,38.58572703)(473.85964808,36.30572387)(477.15964478,36.30572387)
\curveto(479.47964246,36.30572387)(481.11964524,37.54572615)(481.57964478,39.82572387)
\lineto(479.89964478,39.82572387)
\moveto(473.99964478,42.70572387)
\curveto(474.11964466,44.28572229)(475.17964656,45.62572387)(476.95964478,45.62572387)
\curveto(478.6396431,45.62572387)(479.85964486,44.32572225)(479.93964478,42.70572387)
\lineto(473.99964478,42.70572387)
}
}
{
\newrgbcolor{curcolor}{0 0 0}
\pscustom[linestyle=none,fillstyle=solid,fillcolor=curcolor]
{
\newpath
\moveto(482.84183228,39.80572387)
\curveto(482.94183218,37.24572643)(484.9018346,36.30572387)(487.22183228,36.30572387)
\curveto(489.32183018,36.30572387)(491.62183228,37.10572633)(491.62183228,39.56572387)
\curveto(491.62183228,41.56572187)(489.94183058,42.12572425)(488.24183228,42.50572387)
\curveto(486.66183386,42.88572349)(484.86183228,43.08572509)(484.86183228,44.30572387)
\curveto(484.86183228,45.34572283)(486.0418333,45.62572387)(487.06183228,45.62572387)
\curveto(488.18183116,45.62572387)(489.3418324,45.20572255)(489.46183228,43.88572387)
\lineto(491.16183228,43.88572387)
\curveto(491.02183242,46.40572135)(489.20183,47.12572387)(486.92183228,47.12572387)
\curveto(485.12183408,47.12572387)(483.06183228,46.26572179)(483.06183228,44.18572387)
\curveto(483.06183228,42.20572585)(484.76183396,41.64572349)(486.44183228,41.26572387)
\curveto(488.14183058,40.88572425)(489.82183228,40.66572255)(489.82183228,39.34572387)
\curveto(489.82183228,38.04572517)(488.38183122,37.80572387)(487.32183228,37.80572387)
\curveto(485.92183368,37.80572387)(484.60183222,38.28572539)(484.54183228,39.80572387)
\lineto(482.84183228,39.80572387)
}
}
{
\newrgbcolor{curcolor}{0 0 0}
\pscustom[linewidth=2.48031497,linecolor=curcolor,linestyle=dashed,dash=2.48031496 9.92125984]
{
\newpath
\moveto(505.93645,100.84821838)
\lineto(30.368497,100.84821838)
}
}
{
\newrgbcolor{curcolor}{0 0 0}
\pscustom[linestyle=none,fillstyle=solid,fillcolor=curcolor]
{
\newpath
\moveto(428.79632602,370.6886327)
\curveto(428.13632668,375.24862814)(424.44632155,377.6486327)(419.97632602,377.6486327)
\curveto(413.37633262,377.6486327)(409.83632602,372.57862649)(409.83632602,366.3686327)
\curveto(409.83632602,360.12863894)(413.07633268,355.2086327)(419.73632602,355.2086327)
\curveto(425.13632062,355.2086327)(428.46632656,358.44863804)(429.00632602,363.7886327)
\lineto(426.15632602,363.7886327)
\curveto(425.88632629,360.24863624)(423.72632227,357.6086327)(419.97632602,357.6086327)
\curveto(414.84633115,357.6086327)(412.68632602,361.68863759)(412.68632602,366.5786327)
\curveto(412.68632602,371.04862823)(414.84633112,375.2486327)(419.94632602,375.2486327)
\curveto(422.91632305,375.2486327)(425.34632662,373.71862967)(425.94632602,370.6886327)
\lineto(428.79632602,370.6886327)
}
}
{
\newrgbcolor{curcolor}{0 0 0}
\pscustom[linestyle=none,fillstyle=solid,fillcolor=curcolor]
{
\newpath
\moveto(441.74601352,361.1186327)
\curveto(441.74601352,359.70863411)(440.36601016,357.6086327)(437.00601352,357.6086327)
\curveto(435.44601508,357.6086327)(434.00601352,358.20863438)(434.00601352,359.8886327)
\curveto(434.00601352,361.77863081)(435.4460152,362.378633)(437.12601352,362.6786327)
\curveto(438.83601181,362.9786324)(440.75601451,363.00863342)(441.74601352,363.7286327)
\lineto(441.74601352,361.1186327)
\moveto(445.88601352,357.7586327)
\curveto(445.55601385,357.63863282)(445.31601331,357.6086327)(445.10601352,357.6086327)
\curveto(444.29601433,357.6086327)(444.29601352,358.1486339)(444.29601352,359.3486327)
\lineto(444.29601352,367.3286327)
\curveto(444.29601352,370.95862907)(441.26601073,371.5886327)(438.47601352,371.5886327)
\curveto(435.02601697,371.5886327)(432.05601337,370.23862886)(431.90601352,366.3986327)
\lineto(434.45601352,366.3986327)
\curveto(434.5760134,368.67863042)(436.16601568,369.3386327)(438.32601352,369.3386327)
\curveto(439.9460119,369.3386327)(441.77601352,368.97863048)(441.77601352,366.7586327)
\curveto(441.77601352,364.83863462)(439.3760107,365.01863216)(436.55601352,364.4786327)
\curveto(433.91601616,363.96863321)(431.30601352,363.21862919)(431.30601352,359.7086327)
\curveto(431.30601352,356.61863579)(433.61601634,355.3586327)(436.43601352,355.3586327)
\curveto(438.59601136,355.3586327)(440.48601493,356.10863435)(441.89601352,357.7586327)
\curveto(441.89601352,356.07863438)(442.73601484,355.3586327)(444.05601352,355.3586327)
\curveto(444.86601271,355.3586327)(445.43601397,355.50863297)(445.88601352,355.7786327)
\lineto(445.88601352,357.7586327)
}
}
{
\newrgbcolor{curcolor}{0 0 0}
\pscustom[linestyle=none,fillstyle=solid,fillcolor=curcolor]
{
\newpath
\moveto(448.16929477,355.7186327)
\lineto(450.71929477,355.7186327)
\lineto(450.71929477,362.6186327)
\curveto(450.71929477,366.54862877)(452.21929888,368.8886327)(456.32929477,368.8886327)
\lineto(456.32929477,371.5886327)
\curveto(453.56929753,371.67863261)(451.85929354,370.44863021)(450.62929477,367.9586327)
\lineto(450.56929477,367.9586327)
\lineto(450.56929477,371.2286327)
\lineto(448.16929477,371.2286327)
\lineto(448.16929477,355.7186327)
}
}
{
\newrgbcolor{curcolor}{0 0 0}
\pscustom[linestyle=none,fillstyle=solid,fillcolor=curcolor]
{
\newpath
\moveto(458.18882602,355.7186327)
\lineto(460.73882602,355.7186327)
\lineto(460.73882602,362.6186327)
\curveto(460.73882602,366.54862877)(462.23883013,368.8886327)(466.34882602,368.8886327)
\lineto(466.34882602,371.5886327)
\curveto(463.58882878,371.67863261)(461.87882479,370.44863021)(460.64882602,367.9586327)
\lineto(460.58882602,367.9586327)
\lineto(460.58882602,371.2286327)
\lineto(458.18882602,371.2286327)
\lineto(458.18882602,355.7186327)
}
}
{
\newrgbcolor{curcolor}{0 0 0}
\pscustom[linestyle=none,fillstyle=solid,fillcolor=curcolor]
{
\newpath
\moveto(479.00835727,360.6386327)
\curveto(478.55835772,358.62863471)(477.08835517,357.6086327)(474.98835727,357.6086327)
\curveto(471.59836066,357.6086327)(470.06835736,360.0086354)(470.15835727,362.7086327)
\lineto(481.76835727,362.7086327)
\curveto(481.91835712,366.45862895)(480.23835178,371.5886327)(474.74835727,371.5886327)
\curveto(470.5183615,371.5886327)(467.45835727,368.16862805)(467.45835727,363.5186327)
\curveto(467.60835712,358.77863744)(469.94836222,355.3586327)(474.89835727,355.3586327)
\curveto(478.37835379,355.3586327)(480.83835796,357.21863612)(481.52835727,360.6386327)
\lineto(479.00835727,360.6386327)
\moveto(470.15835727,364.9586327)
\curveto(470.33835709,367.32863033)(471.92835994,369.3386327)(474.59835727,369.3386327)
\curveto(477.11835475,369.3386327)(478.94835739,367.38863027)(479.06835727,364.9586327)
\lineto(470.15835727,364.9586327)
}
}
{
\newrgbcolor{curcolor}{0 0 0}
\pscustom[linestyle=none,fillstyle=solid,fillcolor=curcolor]
{
\newpath
\moveto(484.32163852,355.7186327)
\lineto(486.87163852,355.7186327)
\lineto(486.87163852,362.6186327)
\curveto(486.87163852,366.54862877)(488.37164263,368.8886327)(492.48163852,368.8886327)
\lineto(492.48163852,371.5886327)
\curveto(489.72164128,371.67863261)(488.01163729,370.44863021)(486.78163852,367.9586327)
\lineto(486.72163852,367.9586327)
\lineto(486.72163852,371.2286327)
\lineto(484.32163852,371.2286327)
\lineto(484.32163852,355.7186327)
}
}
{
\newrgbcolor{curcolor}{0 0 0}
\pscustom[linestyle=none,fillstyle=solid,fillcolor=curcolor]
{
\newpath
\moveto(504.03116977,361.1186327)
\curveto(504.03116977,359.70863411)(502.65116641,357.6086327)(499.29116977,357.6086327)
\curveto(497.73117133,357.6086327)(496.29116977,358.20863438)(496.29116977,359.8886327)
\curveto(496.29116977,361.77863081)(497.73117145,362.378633)(499.41116977,362.6786327)
\curveto(501.12116806,362.9786324)(503.04117076,363.00863342)(504.03116977,363.7286327)
\lineto(504.03116977,361.1186327)
\moveto(508.17116977,357.7586327)
\curveto(507.8411701,357.63863282)(507.60116956,357.6086327)(507.39116977,357.6086327)
\curveto(506.58117058,357.6086327)(506.58116977,358.1486339)(506.58116977,359.3486327)
\lineto(506.58116977,367.3286327)
\curveto(506.58116977,370.95862907)(503.55116698,371.5886327)(500.76116977,371.5886327)
\curveto(497.31117322,371.5886327)(494.34116962,370.23862886)(494.19116977,366.3986327)
\lineto(496.74116977,366.3986327)
\curveto(496.86116965,368.67863042)(498.45117193,369.3386327)(500.61116977,369.3386327)
\curveto(502.23116815,369.3386327)(504.06116977,368.97863048)(504.06116977,366.7586327)
\curveto(504.06116977,364.83863462)(501.66116695,365.01863216)(498.84116977,364.4786327)
\curveto(496.20117241,363.96863321)(493.59116977,363.21862919)(493.59116977,359.7086327)
\curveto(493.59116977,356.61863579)(495.90117259,355.3586327)(498.72116977,355.3586327)
\curveto(500.88116761,355.3586327)(502.77117118,356.10863435)(504.18116977,357.7586327)
\curveto(504.18116977,356.07863438)(505.02117109,355.3586327)(506.34116977,355.3586327)
\curveto(507.15116896,355.3586327)(507.72117022,355.50863297)(508.17116977,355.7786327)
\lineto(508.17116977,357.7586327)
}
}
{
\newrgbcolor{curcolor}{0 0 0}
\pscustom[linewidth=2.65748024,linecolor=curcolor]
{
\newpath
\moveto(400.76647983,392.00745106)
\lineto(517.24104343,392.00745106)
\lineto(517.24104343,340.84976673)
\lineto(400.76647983,340.84976673)
\closepath
}
}
{
\newrgbcolor{curcolor}{0 0 0}
\pscustom[linestyle=none,fillstyle=solid,fillcolor=curcolor]
{
\newpath
\moveto(444.86088428,279.27615453)
\lineto(447.80088428,279.27615453)
\lineto(450.20088428,285.72615453)
\lineto(459.26088428,285.72615453)
\lineto(461.60088428,279.27615453)
\lineto(464.75088428,279.27615453)
\lineto(456.38088428,300.69615452)
\lineto(453.23088428,300.69615452)
\lineto(444.86088428,279.27615453)
\moveto(454.73088428,298.11615453)
\lineto(454.79088428,298.11615453)
\lineto(458.36088428,288.12615453)
\lineto(451.10088428,288.12615453)
\lineto(454.73088428,298.11615453)
\moveto(452.96088428,302.22615453)
\lineto(454.88088428,302.22615453)
\lineto(458.81088428,306.51615453)
\lineto(455.54088428,306.51615453)
\lineto(452.96088428,302.22615453)
}
}
{
\newrgbcolor{curcolor}{0 0 0}
\pscustom[linestyle=none,fillstyle=solid,fillcolor=curcolor]
{
\newpath
\moveto(466.35400928,279.27615453)
\lineto(468.90400928,279.27615453)
\lineto(468.90400928,286.17615453)
\curveto(468.90400928,290.1061506)(470.40401339,292.44615452)(474.51400928,292.44615452)
\lineto(474.51400928,295.14615453)
\curveto(471.75401204,295.23615444)(470.04400805,294.00615204)(468.81400928,291.51615453)
\lineto(468.75400928,291.51615453)
\lineto(468.75400928,294.78615453)
\lineto(466.35400928,294.78615453)
\lineto(466.35400928,279.27615453)
}
}
{
\newrgbcolor{curcolor}{0 0 0}
\pscustom[linestyle=none,fillstyle=solid,fillcolor=curcolor]
{
\newpath
\moveto(487.17354053,284.19615452)
\curveto(486.72354098,282.18615654)(485.25353843,281.16615453)(483.15354053,281.16615453)
\curveto(479.76354392,281.16615453)(478.23354062,283.56615722)(478.32354053,286.26615453)
\lineto(489.93354053,286.26615453)
\curveto(490.08354038,290.01615078)(488.40353504,295.14615453)(482.91354053,295.14615453)
\curveto(478.68354476,295.14615453)(475.62354053,291.72614988)(475.62354053,287.07615452)
\curveto(475.77354038,282.33615926)(478.11354548,278.91615453)(483.06354053,278.91615453)
\curveto(486.54353705,278.91615453)(489.00354122,280.77615795)(489.69354053,284.19615452)
\lineto(487.17354053,284.19615452)
\moveto(478.32354053,288.51615453)
\curveto(478.50354035,290.88615215)(480.0935432,292.89615453)(482.76354053,292.89615453)
\curveto(485.28353801,292.89615453)(487.11354065,290.9461521)(487.23354053,288.51615453)
\lineto(478.32354053,288.51615453)
}
}
{
\newrgbcolor{curcolor}{0 0 0}
\pscustom[linestyle=none,fillstyle=solid,fillcolor=curcolor]
{
\newpath
\moveto(502.17682178,284.67615453)
\curveto(502.17682178,283.26615594)(500.79681842,281.16615453)(497.43682178,281.16615453)
\curveto(495.87682334,281.16615453)(494.43682178,281.76615621)(494.43682178,283.44615452)
\curveto(494.43682178,285.33615264)(495.87682346,285.93615483)(497.55682178,286.23615453)
\curveto(499.26682007,286.53615423)(501.18682277,286.56615525)(502.17682178,287.28615453)
\lineto(502.17682178,284.67615453)
\moveto(506.31682178,281.31615453)
\curveto(505.98682211,281.19615465)(505.74682157,281.16615453)(505.53682178,281.16615453)
\curveto(504.72682259,281.16615453)(504.72682178,281.70615573)(504.72682178,282.90615453)
\lineto(504.72682178,290.88615453)
\curveto(504.72682178,294.5161509)(501.69681899,295.14615453)(498.90682178,295.14615453)
\curveto(495.45682523,295.14615453)(492.48682163,293.79615069)(492.33682178,289.95615452)
\lineto(494.88682178,289.95615452)
\curveto(495.00682166,292.23615225)(496.59682394,292.89615453)(498.75682178,292.89615453)
\curveto(500.37682016,292.89615453)(502.20682178,292.53615231)(502.20682178,290.31615453)
\curveto(502.20682178,288.39615645)(499.80681896,288.57615399)(496.98682178,288.03615453)
\curveto(494.34682442,287.52615503)(491.73682178,286.77615101)(491.73682178,283.26615453)
\curveto(491.73682178,280.17615762)(494.0468246,278.91615453)(496.86682178,278.91615453)
\curveto(499.02681962,278.91615453)(500.91682319,279.66615618)(502.32682178,281.31615453)
\curveto(502.32682178,279.63615621)(503.1668231,278.91615453)(504.48682178,278.91615453)
\curveto(505.29682097,278.91615453)(505.86682223,279.06615479)(506.31682178,279.33615452)
\lineto(506.31682178,281.31615453)
}
}
{
\newrgbcolor{curcolor}{0 0 0}
\pscustom[linewidth=2.65748024,linecolor=curcolor]
{
\newpath
\moveto(433.93666797,318.49563695)
\lineto(517.24104267,318.49563695)
\lineto(517.24104267,266.93671132)
\lineto(433.93666797,266.93671132)
\closepath
}
}
{
\newrgbcolor{curcolor}{0 0 0}
\pscustom[linestyle=none,fillstyle=solid,fillcolor=curcolor]
{
\newpath
\moveto(373.17144775,223.54569213)
\curveto(372.51144841,228.10568757)(368.82144328,230.50569213)(364.35144775,230.50569213)
\curveto(357.75145435,230.50569213)(354.21144775,225.43568592)(354.21144775,219.22569213)
\curveto(354.21144775,212.98569837)(357.45145441,208.06569213)(364.11144775,208.06569213)
\curveto(369.51144235,208.06569213)(372.84144829,211.30569747)(373.38144775,216.64569213)
\lineto(370.53144775,216.64569213)
\curveto(370.26144802,213.10569567)(368.101444,210.46569213)(364.35144775,210.46569213)
\curveto(359.22145288,210.46569213)(357.06144775,214.54569702)(357.06144775,219.43569213)
\curveto(357.06144775,223.90568766)(359.22145285,228.10569213)(364.32144775,228.10569213)
\curveto(367.29144478,228.10569213)(369.72144835,226.5756891)(370.32144775,223.54569213)
\lineto(373.17144775,223.54569213)
}
}
{
\newrgbcolor{curcolor}{0 0 0}
\pscustom[linestyle=none,fillstyle=solid,fillcolor=curcolor]
{
\newpath
\moveto(375.68113525,216.31569213)
\curveto(375.68113525,211.78569666)(378.29114017,208.21569213)(383.21113525,208.21569213)
\curveto(388.13113033,208.21569213)(390.74113525,211.78569666)(390.74113525,216.31569213)
\curveto(390.74113525,220.87568757)(388.13113033,224.44569213)(383.21113525,224.44569213)
\curveto(378.29114017,224.44569213)(375.68113525,220.87568757)(375.68113525,216.31569213)
\moveto(378.38113525,216.31569213)
\curveto(378.38113525,220.09568835)(380.54113792,222.19569213)(383.21113525,222.19569213)
\curveto(385.88113258,222.19569213)(388.04113525,220.09568835)(388.04113525,216.31569213)
\curveto(388.04113525,212.56569588)(385.88113258,210.46569213)(383.21113525,210.46569213)
\curveto(380.54113792,210.46569213)(378.38113525,212.56569588)(378.38113525,216.31569213)
}
}
{
\newrgbcolor{curcolor}{0 0 0}
\pscustom[linestyle=none,fillstyle=solid,fillcolor=curcolor]
{
\newpath
\moveto(393.74769775,208.57569213)
\lineto(396.29769775,208.57569213)
\lineto(396.29769775,218.23569213)
\curveto(396.29769775,219.40569096)(397.40770078,222.19569213)(400.43769775,222.19569213)
\curveto(402.71769547,222.19569213)(403.34769775,220.75569012)(403.34769775,218.74569213)
\lineto(403.34769775,208.57569213)
\lineto(405.89769775,208.57569213)
\lineto(405.89769775,218.23569213)
\curveto(405.89769775,220.63568973)(407.48770021,222.19569213)(409.94769775,222.19569213)
\curveto(412.43769526,222.19569213)(412.94769775,220.66569021)(412.94769775,218.74569213)
\lineto(412.94769775,208.57569213)
\lineto(415.49769775,208.57569213)
\lineto(415.49769775,219.94569213)
\curveto(415.49769775,223.15568892)(413.42769466,224.44569213)(410.33769775,224.44569213)
\curveto(408.35769973,224.44569213)(406.5276967,223.45569048)(405.47769775,221.80569213)
\curveto(404.84769838,223.69569024)(403.10769586,224.44569213)(401.21769775,224.44569213)
\curveto(399.08769988,224.44569213)(397.34769661,223.54569039)(396.20769775,221.80569213)
\lineto(396.14769775,221.80569213)
\lineto(396.14769775,224.08569213)
\lineto(393.74769775,224.08569213)
\lineto(393.74769775,208.57569213)
}
}
{
\newrgbcolor{curcolor}{0 0 0}
\pscustom[linestyle=none,fillstyle=solid,fillcolor=curcolor]
{
\newpath
\moveto(432.1931665,224.08569213)
\lineto(429.6431665,224.08569213)
\lineto(429.6431665,215.32569213)
\curveto(429.6431665,212.53569492)(428.14316341,210.46569213)(425.0531665,210.46569213)
\curveto(423.10316845,210.46569213)(421.9031665,211.69569402)(421.9031665,213.58569213)
\lineto(421.9031665,224.08569213)
\lineto(419.3531665,224.08569213)
\lineto(419.3531665,213.88569213)
\curveto(419.3531665,210.55569546)(420.61317058,208.21569213)(424.6931665,208.21569213)
\curveto(426.91316428,208.21569213)(428.65316758,209.11569405)(429.7331665,211.03569213)
\lineto(429.7931665,211.03569213)
\lineto(429.7931665,208.57569213)
\lineto(432.1931665,208.57569213)
\lineto(432.1931665,224.08569213)
}
}
{
\newrgbcolor{curcolor}{0 0 0}
\pscustom[linestyle=none,fillstyle=solid,fillcolor=curcolor]
{
\newpath
\moveto(436.05238525,208.57569213)
\lineto(438.60238525,208.57569213)
\lineto(438.60238525,217.33569213)
\curveto(438.60238525,220.12568934)(440.10238834,222.19569213)(443.19238525,222.19569213)
\curveto(445.1423833,222.19569213)(446.34238525,220.96569024)(446.34238525,219.07569213)
\lineto(446.34238525,208.57569213)
\lineto(448.89238525,208.57569213)
\lineto(448.89238525,218.77569213)
\curveto(448.89238525,222.1056888)(447.63238117,224.44569213)(443.55238525,224.44569213)
\curveto(441.33238747,224.44569213)(439.59238417,223.54569021)(438.51238525,221.62569213)
\lineto(438.45238525,221.62569213)
\lineto(438.45238525,224.08569213)
\lineto(436.05238525,224.08569213)
\lineto(436.05238525,208.57569213)
}
}
{
\newrgbcolor{curcolor}{0 0 0}
\pscustom[linestyle=none,fillstyle=solid,fillcolor=curcolor]
{
\newpath
\moveto(452.901604,208.57569213)
\lineto(455.451604,208.57569213)
\lineto(455.451604,224.08569213)
\lineto(452.901604,224.08569213)
\lineto(452.901604,208.57569213)
\moveto(455.451604,229.99569213)
\lineto(452.901604,229.99569213)
\lineto(452.901604,226.87569213)
\lineto(455.451604,226.87569213)
\lineto(455.451604,229.99569213)
}
}
{
\newrgbcolor{curcolor}{0 0 0}
\pscustom[linestyle=none,fillstyle=solid,fillcolor=curcolor]
{
\newpath
\moveto(473.2912915,229.99569213)
\lineto(470.7412915,229.99569213)
\lineto(470.7412915,222.01569213)
\lineto(470.6812915,222.01569213)
\curveto(469.54129264,223.8456903)(467.26128991,224.44569213)(465.6712915,224.44569213)
\curveto(460.93129624,224.44569213)(458.5912915,220.75568775)(458.5912915,216.37569213)
\curveto(458.5912915,211.99569651)(460.96129627,208.21569213)(465.7312915,208.21569213)
\curveto(467.86128937,208.21569213)(469.84129234,208.96569384)(470.6812915,210.67569213)
\lineto(470.7412915,210.67569213)
\lineto(470.7412915,208.57569213)
\lineto(473.2912915,208.57569213)
\lineto(473.2912915,229.99569213)
\moveto(461.2912915,216.19569213)
\curveto(461.2912915,219.2256891)(462.46129498,222.19569213)(465.9412915,222.19569213)
\curveto(469.45128799,222.19569213)(470.8312915,219.37568907)(470.8312915,216.31569213)
\curveto(470.8312915,213.40569504)(469.51128808,210.46569213)(466.0912915,210.46569213)
\curveto(462.7912948,210.46569213)(461.2912915,213.28569504)(461.2912915,216.19569213)
}
}
{
\newrgbcolor{curcolor}{0 0 0}
\pscustom[linestyle=none,fillstyle=solid,fillcolor=curcolor]
{
\newpath
\moveto(486.8437915,213.97569213)
\curveto(486.8437915,212.56569354)(485.46378814,210.46569213)(482.1037915,210.46569213)
\curveto(480.54379306,210.46569213)(479.1037915,211.06569381)(479.1037915,212.74569213)
\curveto(479.1037915,214.63569024)(480.54379318,215.23569243)(482.2237915,215.53569213)
\curveto(483.93378979,215.83569183)(485.85379249,215.86569285)(486.8437915,216.58569213)
\lineto(486.8437915,213.97569213)
\moveto(490.9837915,210.61569213)
\curveto(490.65379183,210.49569225)(490.41379129,210.46569213)(490.2037915,210.46569213)
\curveto(489.39379231,210.46569213)(489.3937915,211.00569333)(489.3937915,212.20569213)
\lineto(489.3937915,220.18569213)
\curveto(489.3937915,223.8156885)(486.36378871,224.44569213)(483.5737915,224.44569213)
\curveto(480.12379495,224.44569213)(477.15379135,223.09568829)(477.0037915,219.25569213)
\lineto(479.5537915,219.25569213)
\curveto(479.67379138,221.53568985)(481.26379366,222.19569213)(483.4237915,222.19569213)
\curveto(485.04378988,222.19569213)(486.8737915,221.83568991)(486.8737915,219.61569213)
\curveto(486.8737915,217.69569405)(484.47378868,217.87569159)(481.6537915,217.33569213)
\curveto(479.01379414,216.82569264)(476.4037915,216.07568862)(476.4037915,212.56569213)
\curveto(476.4037915,209.47569522)(478.71379432,208.21569213)(481.5337915,208.21569213)
\curveto(483.69378934,208.21569213)(485.58379291,208.96569378)(486.9937915,210.61569213)
\curveto(486.9937915,208.93569381)(487.83379282,208.21569213)(489.1537915,208.21569213)
\curveto(489.96379069,208.21569213)(490.53379195,208.3656924)(490.9837915,208.63569213)
\lineto(490.9837915,210.61569213)
}
}
{
\newrgbcolor{curcolor}{0 0 0}
\pscustom[linestyle=none,fillstyle=solid,fillcolor=curcolor]
{
\newpath
\moveto(507.21707275,229.99569213)
\lineto(504.66707275,229.99569213)
\lineto(504.66707275,222.01569213)
\lineto(504.60707275,222.01569213)
\curveto(503.46707389,223.8456903)(501.18707116,224.44569213)(499.59707275,224.44569213)
\curveto(494.85707749,224.44569213)(492.51707275,220.75568775)(492.51707275,216.37569213)
\curveto(492.51707275,211.99569651)(494.88707752,208.21569213)(499.65707275,208.21569213)
\curveto(501.78707062,208.21569213)(503.76707359,208.96569384)(504.60707275,210.67569213)
\lineto(504.66707275,210.67569213)
\lineto(504.66707275,208.57569213)
\lineto(507.21707275,208.57569213)
\lineto(507.21707275,229.99569213)
\moveto(495.21707275,216.19569213)
\curveto(495.21707275,219.2256891)(496.38707623,222.19569213)(499.86707275,222.19569213)
\curveto(503.37706924,222.19569213)(504.75707275,219.37568907)(504.75707275,216.31569213)
\curveto(504.75707275,213.40569504)(503.43706933,210.46569213)(500.01707275,210.46569213)
\curveto(496.71707605,210.46569213)(495.21707275,213.28569504)(495.21707275,216.19569213)
}
}
{
\newrgbcolor{curcolor}{0 0 0}
\pscustom[linewidth=2.65748024,linecolor=curcolor]
{
\newpath
\moveto(344.1875,244.5825891)
\lineto(517.24104309,244.5825891)
\lineto(517.24104309,193.98877425)
\lineto(344.1875,193.98877425)
\closepath
}
}
{
\newrgbcolor{curcolor}{0 0 0}
\pscustom[linewidth=2.65748024,linecolor=curcolor]
{
\newpath
\moveto(375,366.42857138)
\lineto(170.71429,366.42857138)
}
}
{
\newrgbcolor{curcolor}{0 0 0}
\pscustom[linestyle=none,fillstyle=solid,fillcolor=curcolor]
{
\newpath
\moveto(184.61597488,359.99682843)
\lineto(167.19479306,366.40303843)
\lineto(184.61597584,372.80924702)
\curveto(181.83279933,369.02702404)(181.84883606,363.85228657)(184.61597488,359.99682843)
\closepath
}
}
{
\newrgbcolor{curcolor}{0 0 0}
\pscustom[linewidth=2.65748024,linecolor=curcolor]
{
\newpath
\moveto(412.85714,305.71427838)
\lineto(220.71429,305.71427838)
}
}
{
\newrgbcolor{curcolor}{0 0 0}
\pscustom[linestyle=none,fillstyle=solid,fillcolor=curcolor]
{
\newpath
\moveto(234.61597488,299.28253543)
\lineto(217.19479306,305.68874543)
\lineto(234.61597584,312.09495402)
\curveto(231.83279933,308.31273104)(231.84883606,303.13799357)(234.61597488,299.28253543)
\closepath
}
}
\end{pspicture}

\caption{Espacios virtuales que se han de construir y su clasificación en el
sistema.}
\label{espacios}
\end{figure}

Cada uno de ellos posee la funcionalidad característica de un recurso
administrable, es decir, posee las siguientes operaciones\footnote{Para una
definición exacta puede consultarse:
https://es.wikipedia.org/wiki/Create,\_read,\_update\_and\_delete}:

\begin{itemize}
\item Crear un nuevo elemento (CREATE).
\item Visualizar el elemento a detalle (READ).
\item Editar las características del elemento (UPDATE).
\item Eliminación del elemento (DELETE).
\end{itemize}

A su vez se han establecido un conjunto de tareas por lote para facilitar la
correcta manipulación de amplios volúmenes de información. Estas tareas son:

\begin{itemize}
\item Importación de datos desde un archivo CSV.
\item Exportación de datos hacia un archivo CSV.
\item Habilitación/Inhabilitación de elementos, ya sea individualmente o en
      grupos de elementos.
\end{itemize}

\subsection{Intercambio de recursos}

Cada espacio virtual debe poseer la capacidad de contener información en
distintos formatos, y para diversos propósitos. El objetivo principal es poder
compartir piezas de información entre usuarios del sistema.

Para este propósito, se han definido piezas atómicas de información básica,
estas son:

\begin{description}
\item [Notas] Son piezas de texto que no poseen formato, y representan la unidad
de información mas básica que ha de construirse.
\item [Archivos] Los archivos representan recursos que los usuarios suben al
sistema, y no esta contemplada ninguna tarea adicional, a parte de alojarlos y
brindar la capacidad de ser descargados por otros usuarios.
\item [Imágenes] Una imagen es la única pieza provista para representación
visual en el sistema. Está adicionalmente a ser subida por un usuario debe
poder ser visualizada y descargada por parte de los demás usuarios.
\item [Vídeos] Inicialmente los vídeos representan archivos en el formato flv,
que puedan ser reproducibles en un player de adobe flash.
\item [Eventos] Los eventos son piezas que demarcan la iniciación y duración de
una actividad, estas pueden ser creadas por algún usuario y visualizadas por
otros usuarios, según el espacio virtual en el que se encuentre.
\item [Enlaces] Inicialmente se contempla únicamente la publicación de enlaces,
sin análisis del lugar a donde conducen, a la larga se plantea la posibilidad de
analizar el recurso destino y poder reenderizarse la información según tal
inspección (es deseable, pero no esta contemplado en los alcances de este
proyecto).
\end{description}

Todos estos tipos de recursos poseen también características de espacio virtual,
es decir, que cada una de ellas posee operaciones CRUD, además de
funcionalidades para el fomento a la participación.

\subsection{Canales de comunicación}

Para la mejora de los canales de comunicación se ha definido el manejo de otros
tipos de espacios-recursos adicionales que poseen diferentes propósitos
utilitarios, estos son:

\begin{description}
\item [Usuarios] Para incrementar la afinidad de los usuarios hacia el sistema
que conforma este proyecto, se han definido la construcción de espacios propios
para cada usuario. De modo que este pueda controlar los recursos que produzca y
que sean realmente suyos.
\item [Roles] Un rol define el tipo de participación que puede poseer un usuario
en el sistema, inicialmente se han creado un conjunto de roles, que esta acordes
a la lógica del contexto de implantación (UMSS):
    \begin{itemize}
    \item Administrador
    \item Desarrollador
    \item Moderador
    \item Docente
    \item Auxiliar
    \item Estudiante
    \item Invitado
    \end{itemize}
\item [Contactos] La característica mas propia de una red social esta basada en
la creación de vínculos entre usuarios del sistema, para esto se ha creado una
cadena de contactos, estos vínculos pueden ser de tres tipos (estos están
basados en la forma que son manejados por la red social twitter) (figura
\ref{contactos}):
\begin{description}
\item [Follower] Representa una relación uní-direccional, de un usuario que
ve los recursos que produce otro usuario.
\item [Following] Representa una relación uní-direccional, de un usuario que
produce los recursos que otros usuarios pueden ver.
\item [Friend] Representa una relación bi-direccional, entre dos usuarios,
que comparten los recursos que producen.
\end{description}
Las relaciones de tipo \emph{friend}, se consideran relaciones fuertes, mientras
que las otras dos clases de relaciones, son consideradas relaciones debiles.
\end{description}

\begin{figure}
\centering
%LaTeX with PSTricks extensions
%%Creator: inkscape 0.48.4
%%Please note this file requires PSTricks extensions
\psset{xunit=.5pt,yunit=.5pt,runit=.5pt}
\begin{pspicture}(531.49603271,248.03149414)
{
\newrgbcolor{curcolor}{0 0 0}
\pscustom[linestyle=none,fillstyle=solid,fillcolor=curcolor]
{
\newpath
\moveto(48.30160622,53.29359723)
\lineto(45.45160622,53.29359723)
\lineto(45.45160622,39.61359723)
\curveto(45.45160622,35.83360101)(43.47160262,33.76359723)(39.87160622,33.76359723)
\curveto(36.09161,33.76359723)(33.93160622,35.83360101)(33.93160622,39.61359723)
\lineto(33.93160622,53.29359723)
\lineto(31.08160622,53.29359723)
\lineto(31.08160622,39.61359723)
\curveto(31.08160622,33.91360293)(34.35161174,31.36359723)(39.87160622,31.36359723)
\curveto(45.21160088,31.36359723)(48.30160622,34.21360263)(48.30160622,39.61359723)
\lineto(48.30160622,53.29359723)
}
}
{
\newrgbcolor{curcolor}{0 0 0}
\pscustom[linestyle=none,fillstyle=solid,fillcolor=curcolor]
{
\newpath
\moveto(51.47129372,36.76359723)
\curveto(51.62129357,32.92360107)(54.5612972,31.51359723)(58.04129372,31.51359723)
\curveto(61.19129057,31.51359723)(64.64129372,32.71360092)(64.64129372,36.40359723)
\curveto(64.64129372,39.40359423)(62.12129117,40.2435978)(59.57129372,40.81359723)
\curveto(57.20129609,41.38359666)(54.50129372,41.68359906)(54.50129372,43.51359723)
\curveto(54.50129372,45.07359567)(56.27129525,45.49359723)(57.80129372,45.49359723)
\curveto(59.48129204,45.49359723)(61.2212939,44.86359525)(61.40129372,42.88359723)
\lineto(63.95129372,42.88359723)
\curveto(63.74129393,46.66359345)(61.0112903,47.74359723)(57.59129372,47.74359723)
\curveto(54.89129642,47.74359723)(51.80129372,46.45359411)(51.80129372,43.33359723)
\curveto(51.80129372,40.3636002)(54.35129624,39.52359666)(56.87129372,38.95359723)
\curveto(59.42129117,38.3835978)(61.94129372,38.05359525)(61.94129372,36.07359723)
\curveto(61.94129372,34.12359918)(59.78129213,33.76359723)(58.19129372,33.76359723)
\curveto(56.09129582,33.76359723)(54.11129363,34.48359951)(54.02129372,36.76359723)
\lineto(51.47129372,36.76359723)
}
}
{
\newrgbcolor{curcolor}{0 0 0}
\pscustom[linestyle=none,fillstyle=solid,fillcolor=curcolor]
{
\newpath
\moveto(80.30129372,47.38359723)
\lineto(77.75129372,47.38359723)
\lineto(77.75129372,38.62359723)
\curveto(77.75129372,35.83360002)(76.25129063,33.76359723)(73.16129372,33.76359723)
\curveto(71.21129567,33.76359723)(70.01129372,34.99359912)(70.01129372,36.88359723)
\lineto(70.01129372,47.38359723)
\lineto(67.46129372,47.38359723)
\lineto(67.46129372,37.18359723)
\curveto(67.46129372,33.85360056)(68.7212978,31.51359723)(72.80129372,31.51359723)
\curveto(75.0212915,31.51359723)(76.7612948,32.41359915)(77.84129372,34.33359723)
\lineto(77.90129372,34.33359723)
\lineto(77.90129372,31.87359723)
\lineto(80.30129372,31.87359723)
\lineto(80.30129372,47.38359723)
}
}
{
\newrgbcolor{curcolor}{0 0 0}
\pscustom[linestyle=none,fillstyle=solid,fillcolor=curcolor]
{
\newpath
\moveto(93.76051247,37.27359723)
\curveto(93.76051247,35.86359864)(92.38050911,33.76359723)(89.02051247,33.76359723)
\curveto(87.46051403,33.76359723)(86.02051247,34.36359891)(86.02051247,36.04359723)
\curveto(86.02051247,37.93359534)(87.46051415,38.53359753)(89.14051247,38.83359723)
\curveto(90.85051076,39.13359693)(92.77051346,39.16359795)(93.76051247,39.88359723)
\lineto(93.76051247,37.27359723)
\moveto(97.90051247,33.91359723)
\curveto(97.5705128,33.79359735)(97.33051226,33.76359723)(97.12051247,33.76359723)
\curveto(96.31051328,33.76359723)(96.31051247,34.30359843)(96.31051247,35.50359723)
\lineto(96.31051247,43.48359723)
\curveto(96.31051247,47.1135936)(93.28050968,47.74359723)(90.49051247,47.74359723)
\curveto(87.04051592,47.74359723)(84.07051232,46.39359339)(83.92051247,42.55359723)
\lineto(86.47051247,42.55359723)
\curveto(86.59051235,44.83359495)(88.18051463,45.49359723)(90.34051247,45.49359723)
\curveto(91.96051085,45.49359723)(93.79051247,45.13359501)(93.79051247,42.91359723)
\curveto(93.79051247,40.99359915)(91.39050965,41.17359669)(88.57051247,40.63359723)
\curveto(85.93051511,40.12359774)(83.32051247,39.37359372)(83.32051247,35.86359723)
\curveto(83.32051247,32.77360032)(85.63051529,31.51359723)(88.45051247,31.51359723)
\curveto(90.61051031,31.51359723)(92.50051388,32.26359888)(93.91051247,33.91359723)
\curveto(93.91051247,32.23359891)(94.75051379,31.51359723)(96.07051247,31.51359723)
\curveto(96.88051166,31.51359723)(97.45051292,31.6635975)(97.90051247,31.93359723)
\lineto(97.90051247,33.91359723)
}
}
{
\newrgbcolor{curcolor}{0 0 0}
\pscustom[linestyle=none,fillstyle=solid,fillcolor=curcolor]
{
\newpath
\moveto(100.18379372,31.87359723)
\lineto(102.73379372,31.87359723)
\lineto(102.73379372,38.77359723)
\curveto(102.73379372,42.7035933)(104.23379783,45.04359723)(108.34379372,45.04359723)
\lineto(108.34379372,47.74359723)
\curveto(105.58379648,47.83359714)(103.87379249,46.60359474)(102.64379372,44.11359723)
\lineto(102.58379372,44.11359723)
\lineto(102.58379372,47.38359723)
\lineto(100.18379372,47.38359723)
\lineto(100.18379372,31.87359723)
}
}
{
\newrgbcolor{curcolor}{0 0 0}
\pscustom[linestyle=none,fillstyle=solid,fillcolor=curcolor]
{
\newpath
\moveto(110.44332497,31.87359723)
\lineto(112.99332497,31.87359723)
\lineto(112.99332497,47.38359723)
\lineto(110.44332497,47.38359723)
\lineto(110.44332497,31.87359723)
\moveto(112.99332497,53.29359723)
\lineto(110.44332497,53.29359723)
\lineto(110.44332497,50.17359723)
\lineto(112.99332497,50.17359723)
\lineto(112.99332497,53.29359723)
}
}
{
\newrgbcolor{curcolor}{0 0 0}
\pscustom[linestyle=none,fillstyle=solid,fillcolor=curcolor]
{
\newpath
\moveto(116.13301247,39.61359723)
\curveto(116.13301247,35.08360176)(118.74301739,31.51359723)(123.66301247,31.51359723)
\curveto(128.58300755,31.51359723)(131.19301247,35.08360176)(131.19301247,39.61359723)
\curveto(131.19301247,44.17359267)(128.58300755,47.74359723)(123.66301247,47.74359723)
\curveto(118.74301739,47.74359723)(116.13301247,44.17359267)(116.13301247,39.61359723)
\moveto(118.83301247,39.61359723)
\curveto(118.83301247,43.39359345)(120.99301514,45.49359723)(123.66301247,45.49359723)
\curveto(126.3330098,45.49359723)(128.49301247,43.39359345)(128.49301247,39.61359723)
\curveto(128.49301247,35.86360098)(126.3330098,33.76359723)(123.66301247,33.76359723)
\curveto(120.99301514,33.76359723)(118.83301247,35.86360098)(118.83301247,39.61359723)
}
}
{
\newrgbcolor{curcolor}{0 0 0}
\pscustom[linestyle=none,fillstyle=solid,fillcolor=curcolor]
{
}
}
{
\newrgbcolor{curcolor}{0 0 0}
\pscustom[linestyle=none,fillstyle=solid,fillcolor=curcolor]
{
\newpath
\moveto(151.27988747,53.14359723)
\lineto(149.32988747,53.14359723)
\curveto(148.75988804,49.93360044)(146.11988456,49.15359723)(143.20988747,49.15359723)
\lineto(143.20988747,47.11359723)
\lineto(148.72988747,47.11359723)
\lineto(148.72988747,31.87359723)
\lineto(151.27988747,31.87359723)
\lineto(151.27988747,53.14359723)
}
}
{
\newrgbcolor{curcolor}{0 0 0}
\pscustom[linewidth=2.65748024,linecolor=curcolor]
{
\newpath
\moveto(91.1807452,180.25557914)
\lineto(91.1807452,115.64372914)
}
}
{
\newrgbcolor{curcolor}{1 1 1}
\pscustom[linestyle=none,fillstyle=solid,fillcolor=curcolor]
{
\newpath
\moveto(114.0378901,192.12487159)
\curveto(114.0378901,179.30397606)(103.64450073,168.91058669)(90.8236052,168.91058669)
\curveto(78.00270967,168.91058669)(67.6093203,179.30397606)(67.6093203,192.12487159)
\curveto(67.6093203,204.94576711)(78.00270967,215.33915648)(90.8236052,215.33915648)
\curveto(103.63037851,215.33915648)(114.01790889,204.96779736)(114.03786192,192.16103959)
}
}
{
\newrgbcolor{curcolor}{0 0 0}
\pscustom[linewidth=2.65748024,linecolor=curcolor]
{
\newpath
\moveto(114.0378901,192.12487159)
\curveto(114.0378901,179.30397606)(103.64450073,168.91058669)(90.8236052,168.91058669)
\curveto(78.00270967,168.91058669)(67.6093203,179.30397606)(67.6093203,192.12487159)
\curveto(67.6093203,204.94576711)(78.00270967,215.33915648)(90.8236052,215.33915648)
\curveto(103.63037851,215.33915648)(114.01790889,204.96779736)(114.03786192,192.16103959)
}
}
{
\newrgbcolor{curcolor}{0 0 0}
\pscustom[linewidth=2.65748024,linecolor=curcolor]
{
\newpath
\moveto(70.0480442,79.26156914)
\lineto(91.3286852,117.51609914)
\lineto(111.4153452,79.26156914)
}
}
{
\newrgbcolor{curcolor}{0 0 0}
\pscustom[linewidth=2.65748024,linecolor=curcolor]
{
\newpath
\moveto(58.6807482,155.08045914)
\lineto(123.6807452,155.08045914)
}
}
{
\newrgbcolor{curcolor}{0 0 0}
\pscustom[linestyle=none,fillstyle=solid,fillcolor=curcolor]
{
\newpath
\moveto(397.43614522,53.29359723)
\lineto(394.58614522,53.29359723)
\lineto(394.58614522,39.61359723)
\curveto(394.58614522,35.83360101)(392.60614162,33.76359723)(389.00614522,33.76359723)
\curveto(385.226149,33.76359723)(383.06614522,35.83360101)(383.06614522,39.61359723)
\lineto(383.06614522,53.29359723)
\lineto(380.21614522,53.29359723)
\lineto(380.21614522,39.61359723)
\curveto(380.21614522,33.91360293)(383.48615074,31.36359723)(389.00614522,31.36359723)
\curveto(394.34613988,31.36359723)(397.43614522,34.21360263)(397.43614522,39.61359723)
\lineto(397.43614522,53.29359723)
}
}
{
\newrgbcolor{curcolor}{0 0 0}
\pscustom[linestyle=none,fillstyle=solid,fillcolor=curcolor]
{
\newpath
\moveto(400.60583272,36.76359723)
\curveto(400.75583257,32.92360107)(403.6958362,31.51359723)(407.17583272,31.51359723)
\curveto(410.32582957,31.51359723)(413.77583272,32.71360092)(413.77583272,36.40359723)
\curveto(413.77583272,39.40359423)(411.25583017,40.2435978)(408.70583272,40.81359723)
\curveto(406.33583509,41.38359666)(403.63583272,41.68359906)(403.63583272,43.51359723)
\curveto(403.63583272,45.07359567)(405.40583425,45.49359723)(406.93583272,45.49359723)
\curveto(408.61583104,45.49359723)(410.3558329,44.86359525)(410.53583272,42.88359723)
\lineto(413.08583272,42.88359723)
\curveto(412.87583293,46.66359345)(410.1458293,47.74359723)(406.72583272,47.74359723)
\curveto(404.02583542,47.74359723)(400.93583272,46.45359411)(400.93583272,43.33359723)
\curveto(400.93583272,40.3636002)(403.48583524,39.52359666)(406.00583272,38.95359723)
\curveto(408.55583017,38.3835978)(411.07583272,38.05359525)(411.07583272,36.07359723)
\curveto(411.07583272,34.12359918)(408.91583113,33.76359723)(407.32583272,33.76359723)
\curveto(405.22583482,33.76359723)(403.24583263,34.48359951)(403.15583272,36.76359723)
\lineto(400.60583272,36.76359723)
}
}
{
\newrgbcolor{curcolor}{0 0 0}
\pscustom[linestyle=none,fillstyle=solid,fillcolor=curcolor]
{
\newpath
\moveto(429.43583272,47.38359723)
\lineto(426.88583272,47.38359723)
\lineto(426.88583272,38.62359723)
\curveto(426.88583272,35.83360002)(425.38582963,33.76359723)(422.29583272,33.76359723)
\curveto(420.34583467,33.76359723)(419.14583272,34.99359912)(419.14583272,36.88359723)
\lineto(419.14583272,47.38359723)
\lineto(416.59583272,47.38359723)
\lineto(416.59583272,37.18359723)
\curveto(416.59583272,33.85360056)(417.8558368,31.51359723)(421.93583272,31.51359723)
\curveto(424.1558305,31.51359723)(425.8958338,32.41359915)(426.97583272,34.33359723)
\lineto(427.03583272,34.33359723)
\lineto(427.03583272,31.87359723)
\lineto(429.43583272,31.87359723)
\lineto(429.43583272,47.38359723)
}
}
{
\newrgbcolor{curcolor}{0 0 0}
\pscustom[linestyle=none,fillstyle=solid,fillcolor=curcolor]
{
\newpath
\moveto(442.89505147,37.27359723)
\curveto(442.89505147,35.86359864)(441.51504811,33.76359723)(438.15505147,33.76359723)
\curveto(436.59505303,33.76359723)(435.15505147,34.36359891)(435.15505147,36.04359723)
\curveto(435.15505147,37.93359534)(436.59505315,38.53359753)(438.27505147,38.83359723)
\curveto(439.98504976,39.13359693)(441.90505246,39.16359795)(442.89505147,39.88359723)
\lineto(442.89505147,37.27359723)
\moveto(447.03505147,33.91359723)
\curveto(446.7050518,33.79359735)(446.46505126,33.76359723)(446.25505147,33.76359723)
\curveto(445.44505228,33.76359723)(445.44505147,34.30359843)(445.44505147,35.50359723)
\lineto(445.44505147,43.48359723)
\curveto(445.44505147,47.1135936)(442.41504868,47.74359723)(439.62505147,47.74359723)
\curveto(436.17505492,47.74359723)(433.20505132,46.39359339)(433.05505147,42.55359723)
\lineto(435.60505147,42.55359723)
\curveto(435.72505135,44.83359495)(437.31505363,45.49359723)(439.47505147,45.49359723)
\curveto(441.09504985,45.49359723)(442.92505147,45.13359501)(442.92505147,42.91359723)
\curveto(442.92505147,40.99359915)(440.52504865,41.17359669)(437.70505147,40.63359723)
\curveto(435.06505411,40.12359774)(432.45505147,39.37359372)(432.45505147,35.86359723)
\curveto(432.45505147,32.77360032)(434.76505429,31.51359723)(437.58505147,31.51359723)
\curveto(439.74504931,31.51359723)(441.63505288,32.26359888)(443.04505147,33.91359723)
\curveto(443.04505147,32.23359891)(443.88505279,31.51359723)(445.20505147,31.51359723)
\curveto(446.01505066,31.51359723)(446.58505192,31.6635975)(447.03505147,31.93359723)
\lineto(447.03505147,33.91359723)
}
}
{
\newrgbcolor{curcolor}{0 0 0}
\pscustom[linestyle=none,fillstyle=solid,fillcolor=curcolor]
{
\newpath
\moveto(449.31833272,31.87359723)
\lineto(451.86833272,31.87359723)
\lineto(451.86833272,38.77359723)
\curveto(451.86833272,42.7035933)(453.36833683,45.04359723)(457.47833272,45.04359723)
\lineto(457.47833272,47.74359723)
\curveto(454.71833548,47.83359714)(453.00833149,46.60359474)(451.77833272,44.11359723)
\lineto(451.71833272,44.11359723)
\lineto(451.71833272,47.38359723)
\lineto(449.31833272,47.38359723)
\lineto(449.31833272,31.87359723)
}
}
{
\newrgbcolor{curcolor}{0 0 0}
\pscustom[linestyle=none,fillstyle=solid,fillcolor=curcolor]
{
\newpath
\moveto(459.57786397,31.87359723)
\lineto(462.12786397,31.87359723)
\lineto(462.12786397,47.38359723)
\lineto(459.57786397,47.38359723)
\lineto(459.57786397,31.87359723)
\moveto(462.12786397,53.29359723)
\lineto(459.57786397,53.29359723)
\lineto(459.57786397,50.17359723)
\lineto(462.12786397,50.17359723)
\lineto(462.12786397,53.29359723)
}
}
{
\newrgbcolor{curcolor}{0 0 0}
\pscustom[linestyle=none,fillstyle=solid,fillcolor=curcolor]
{
\newpath
\moveto(465.26755147,39.61359723)
\curveto(465.26755147,35.08360176)(467.87755639,31.51359723)(472.79755147,31.51359723)
\curveto(477.71754655,31.51359723)(480.32755147,35.08360176)(480.32755147,39.61359723)
\curveto(480.32755147,44.17359267)(477.71754655,47.74359723)(472.79755147,47.74359723)
\curveto(467.87755639,47.74359723)(465.26755147,44.17359267)(465.26755147,39.61359723)
\moveto(467.96755147,39.61359723)
\curveto(467.96755147,43.39359345)(470.12755414,45.49359723)(472.79755147,45.49359723)
\curveto(475.4675488,45.49359723)(477.62755147,43.39359345)(477.62755147,39.61359723)
\curveto(477.62755147,35.86360098)(475.4675488,33.76359723)(472.79755147,33.76359723)
\curveto(470.12755414,33.76359723)(467.96755147,35.86360098)(467.96755147,39.61359723)
}
}
{
\newrgbcolor{curcolor}{0 0 0}
\pscustom[linestyle=none,fillstyle=solid,fillcolor=curcolor]
{
}
}
{
\newrgbcolor{curcolor}{0 0 0}
\pscustom[linestyle=none,fillstyle=solid,fillcolor=curcolor]
{
\newpath
\moveto(493.60442647,45.61359723)
\curveto(493.51442656,48.16359468)(494.77442944,50.89359723)(497.74442647,50.89359723)
\curveto(499.99442422,50.89359723)(501.85442647,49.36359492)(501.85442647,47.05359723)
\curveto(501.85442647,44.11360017)(500.02442287,42.79359501)(496.42442647,40.57359723)
\curveto(493.42442947,38.71359909)(490.87442605,36.91359219)(490.45442647,31.87359723)
\lineto(504.34442647,31.87359723)
\lineto(504.34442647,34.12359723)
\lineto(493.42442647,34.12359723)
\curveto(493.93442596,36.76359459)(496.72442914,38.11359885)(499.39442647,39.73359723)
\curveto(502.03442383,41.38359558)(504.55442647,43.27360098)(504.55442647,47.02359723)
\curveto(504.55442647,50.98359327)(501.61442275,53.14359723)(497.89442647,53.14359723)
\curveto(493.39443097,53.14359723)(490.84442668,49.93359291)(491.05442647,45.61359723)
\lineto(493.60442647,45.61359723)
}
}
{
\newrgbcolor{curcolor}{0 0 0}
\pscustom[linewidth=2.65748024,linecolor=curcolor]
{
\newpath
\moveto(440.3152842,180.25557914)
\lineto(440.3152842,115.64372914)
}
}
{
\newrgbcolor{curcolor}{1 1 1}
\pscustom[linestyle=none,fillstyle=solid,fillcolor=curcolor]
{
\newpath
\moveto(463.1724291,192.12487159)
\curveto(463.1724291,179.30397606)(452.77903973,168.91058669)(439.9581442,168.91058669)
\curveto(427.13724867,168.91058669)(416.7438593,179.30397606)(416.7438593,192.12487159)
\curveto(416.7438593,204.94576711)(427.13724867,215.33915648)(439.9581442,215.33915648)
\curveto(452.76491751,215.33915648)(463.15244789,204.96779736)(463.17240092,192.16103959)
}
}
{
\newrgbcolor{curcolor}{0 0 0}
\pscustom[linewidth=2.65748024,linecolor=curcolor]
{
\newpath
\moveto(463.1724291,192.12487159)
\curveto(463.1724291,179.30397606)(452.77903973,168.91058669)(439.9581442,168.91058669)
\curveto(427.13724867,168.91058669)(416.7438593,179.30397606)(416.7438593,192.12487159)
\curveto(416.7438593,204.94576711)(427.13724867,215.33915648)(439.9581442,215.33915648)
\curveto(452.76491751,215.33915648)(463.15244789,204.96779736)(463.17240092,192.16103959)
}
}
{
\newrgbcolor{curcolor}{0 0 0}
\pscustom[linewidth=2.65748024,linecolor=curcolor]
{
\newpath
\moveto(419.1825832,79.26156914)
\lineto(440.4632242,117.51609914)
\lineto(460.5498842,79.26156914)
}
}
{
\newrgbcolor{curcolor}{0 0 0}
\pscustom[linewidth=2.65748024,linecolor=curcolor]
{
\newpath
\moveto(407.8152872,155.08045914)
\lineto(472.8152842,155.08045914)
}
}
{
\newrgbcolor{curcolor}{1 1 1}
\pscustom[linestyle=none,fillstyle=solid,fillcolor=curcolor]
{
\newpath
\moveto(158.9749,187.85983414)
\lineto(372.52114,187.85983414)
}
}
{
\newrgbcolor{curcolor}{0 0 0}
\pscustom[linewidth=2.65748024,linecolor=curcolor]
{
\newpath
\moveto(158.9749,187.85983414)
\lineto(372.52114,187.85983414)
}
}
{
\newrgbcolor{curcolor}{0 0 0}
\pscustom[linestyle=none,fillstyle=solid,fillcolor=curcolor]
{
\newpath
\moveto(172.87658488,181.42809119)
\lineto(155.45540306,187.83430119)
\lineto(172.87658584,194.24050978)
\curveto(170.09340933,190.4582868)(170.10944606,185.28354933)(172.87658488,181.42809119)
\closepath
}
}
{
\newrgbcolor{curcolor}{1 1 1}
\pscustom[linestyle=none,fillstyle=solid,fillcolor=curcolor]
{
\newpath
\moveto(374.28891,141.16527414)
\lineto(157.20713,141.16527414)
}
}
{
\newrgbcolor{curcolor}{0 0 0}
\pscustom[linewidth=2.65748024,linecolor=curcolor]
{
\newpath
\moveto(374.28891,141.16527414)
\lineto(157.20713,141.16527414)
}
}
{
\newrgbcolor{curcolor}{0 0 0}
\pscustom[linestyle=none,fillstyle=solid,fillcolor=curcolor]
{
\newpath
\moveto(360.38722512,147.59701709)
\lineto(377.80840694,141.19080709)
\lineto(360.38722416,134.7845985)
\curveto(363.17040067,138.56682148)(363.15436394,143.74155895)(360.38722512,147.59701709)
\closepath
}
}
{
\newrgbcolor{curcolor}{1 1 1}
\pscustom[linestyle=none,fillstyle=solid,fillcolor=curcolor]
{
\newpath
\moveto(157.20712,94.52176414)
\lineto(374.28891,94.52176414)
}
}
{
\newrgbcolor{curcolor}{0 0 0}
\pscustom[linewidth=2.65748024,linecolor=curcolor]
{
\newpath
\moveto(157.20712,94.52176414)
\lineto(374.28891,94.52176414)
}
}
{
\newrgbcolor{curcolor}{0 0 0}
\pscustom[linestyle=none,fillstyle=solid,fillcolor=curcolor]
{
\newpath
\moveto(171.10880488,88.09002119)
\lineto(153.68762306,94.49623119)
\lineto(171.10880584,100.90243978)
\curveto(168.32562933,97.1202168)(168.34166606,91.94547933)(171.10880488,88.09002119)
\closepath
}
}
{
\newrgbcolor{curcolor}{0 0 0}
\pscustom[linestyle=none,fillstyle=solid,fillcolor=curcolor]
{
\newpath
\moveto(360.38722512,100.95350709)
\lineto(377.80840694,94.54729709)
\lineto(360.38722416,88.1410885)
\curveto(363.17040067,91.92331148)(363.15436394,97.09804895)(360.38722512,100.95350709)
\closepath
}
}
{
\newrgbcolor{curcolor}{0 0 0}
\pscustom[linestyle=none,fillstyle=solid,fillcolor=curcolor]
{
\newpath
\moveto(228.40670776,206.9548334)
\lineto(228.40670776,194.93090762)
\lineto(229.3588562,194.93090762)
\lineto(229.3588562,206.9548334)
\lineto(232.03219604,206.9548334)
\lineto(232.03219604,207.78491152)
\lineto(229.3588562,207.78491152)
\lineto(229.3588562,210.36059512)
\curveto(229.35885295,211.0767248)(229.55009625,211.55686755)(229.93258667,211.8010248)
\curveto(230.26624137,212.0370103)(230.70162505,212.15501149)(231.23873901,212.15502871)
\curveto(231.62122048,212.15501149)(231.99556907,212.11432142)(232.36178589,212.0329584)
\lineto(232.36178589,212.85082949)
\curveto(231.99556907,212.94032971)(231.62122048,212.98508878)(231.23873901,212.98510684)
\curveto(230.44120864,212.98508878)(229.78202961,212.78977648)(229.26119995,212.39916934)
\curveto(228.69153591,211.98411322)(228.40670547,211.3290032)(228.40670776,210.4338373)
\lineto(228.40670776,207.78491152)
\lineto(226.11178589,207.78491152)
\lineto(226.11178589,206.9548334)
\lineto(228.40670776,206.9548334)
}
}
{
\newrgbcolor{curcolor}{0 0 0}
\pscustom[linestyle=none,fillstyle=solid,fillcolor=curcolor]
{
\newpath
\moveto(232.97213745,201.35180605)
\curveto(232.97213665,199.48819473)(233.48483145,197.91348927)(234.51022339,196.62768496)
\curveto(235.52747263,195.33373925)(236.9801079,194.6664222)(238.86813354,194.62573184)
\curveto(240.78055983,194.6664222)(242.2372641,195.33373925)(243.23825073,196.62768496)
\curveto(244.25549125,197.91348927)(244.76411704,199.48819473)(244.76412964,201.35180605)
\curveto(244.76411704,203.23981858)(244.25549125,204.82266205)(243.23825073,206.10034121)
\curveto(242.2372641,207.38613605)(240.78055983,208.04531508)(238.86813354,208.07788027)
\curveto(236.9801079,208.04531508)(235.52747263,207.38613605)(234.51022339,206.10034121)
\curveto(233.48483145,204.82266205)(232.97213665,203.23981858)(232.97213745,201.35180605)
\lineto(232.97213745,201.35180605)
\moveto(233.91207886,201.35180605)
\curveto(233.91207711,202.95498813)(234.34746079,204.33031228)(235.2182312,205.47778262)
\curveto(236.04830544,206.62523186)(237.26493834,207.21930679)(238.86813354,207.26000918)
\curveto(240.5038674,207.21930679)(241.73677632,206.62523186)(242.56686401,205.47778262)
\curveto(243.39693091,204.33031228)(243.81196956,202.95498813)(243.8119812,201.35180605)
\curveto(243.81196956,199.77302517)(243.39693091,198.41397705)(242.56686401,197.27465762)
\curveto(241.73677632,196.09464343)(240.5038674,195.48836149)(238.86813354,195.45580996)
\curveto(237.26493834,195.48836149)(236.04830544,196.09464343)(235.2182312,197.27465762)
\curveto(234.34746079,198.41397705)(233.91207711,199.77302517)(233.91207886,201.35180605)
\lineto(233.91207886,201.35180605)
}
}
{
\newrgbcolor{curcolor}{0 0 0}
\pscustom[linestyle=none,fillstyle=solid,fillcolor=curcolor]
{
\newpath
\moveto(247.21774292,212.7775873)
\lineto(247.21774292,194.93090762)
\lineto(248.16989136,194.93090762)
\lineto(248.16989136,212.7775873)
\lineto(247.21774292,212.7775873)
}
}
{
\newrgbcolor{curcolor}{0 0 0}
\pscustom[linestyle=none,fillstyle=solid,fillcolor=curcolor]
{
\newpath
\moveto(251.41696167,212.7775873)
\lineto(251.41696167,194.93090762)
\lineto(252.36911011,194.93090762)
\lineto(252.36911011,212.7775873)
\lineto(251.41696167,212.7775873)
}
}
{
\newrgbcolor{curcolor}{0 0 0}
\pscustom[linestyle=none,fillstyle=solid,fillcolor=curcolor]
{
\newpath
\moveto(254.79830933,201.35180605)
\curveto(254.79830852,199.48819473)(255.31100332,197.91348927)(256.33639526,196.62768496)
\curveto(257.35364451,195.33373925)(258.80627977,194.6664222)(260.69430542,194.62573184)
\curveto(262.6067317,194.6664222)(264.06343597,195.33373925)(265.06442261,196.62768496)
\curveto(266.08166312,197.91348927)(266.59028892,199.48819473)(266.59030151,201.35180605)
\curveto(266.59028892,203.23981858)(266.08166312,204.82266205)(265.06442261,206.10034121)
\curveto(264.06343597,207.38613605)(262.6067317,208.04531508)(260.69430542,208.07788027)
\curveto(258.80627977,208.04531508)(257.35364451,207.38613605)(256.33639526,206.10034121)
\curveto(255.31100332,204.82266205)(254.79830852,203.23981858)(254.79830933,201.35180605)
\lineto(254.79830933,201.35180605)
\moveto(255.73825073,201.35180605)
\curveto(255.73824899,202.95498813)(256.17363267,204.33031228)(257.04440308,205.47778262)
\curveto(257.87447732,206.62523186)(259.09111022,207.21930679)(260.69430542,207.26000918)
\curveto(262.33003927,207.21930679)(263.56294819,206.62523186)(264.39303589,205.47778262)
\curveto(265.22310278,204.33031228)(265.63814143,202.95498813)(265.63815308,201.35180605)
\curveto(265.63814143,199.77302517)(265.22310278,198.41397705)(264.39303589,197.27465762)
\curveto(263.56294819,196.09464343)(262.33003927,195.48836149)(260.69430542,195.45580996)
\curveto(259.09111022,195.48836149)(257.87447732,196.09464343)(257.04440308,197.27465762)
\curveto(256.17363267,198.41397705)(255.73824899,199.77302517)(255.73825073,201.35180605)
\lineto(255.73825073,201.35180605)
}
}
{
\newrgbcolor{curcolor}{0 0 0}
\pscustom[linestyle=none,fillstyle=solid,fillcolor=curcolor]
{
\newpath
\moveto(268.39694214,207.78491152)
\lineto(267.37155151,207.78491152)
\lineto(271.44869995,194.93090762)
\lineto(272.64498901,194.93090762)
\lineto(276.19723511,206.57641543)
\lineto(276.24606323,206.57641543)
\lineto(279.77389526,194.93090762)
\lineto(280.94577026,194.93090762)
\lineto(285.07174683,207.78491152)
\lineto(284.02194214,207.78491152)
\lineto(280.42086792,196.02954043)
\lineto(280.37203979,196.02954043)
\lineto(276.84420776,207.78491152)
\lineto(275.57467651,207.78491152)
\lineto(272.07125854,196.02954043)
\lineto(272.02243042,196.02954043)
\lineto(268.39694214,207.78491152)
}
}
{
\newrgbcolor{curcolor}{0 0 0}
\pscustom[linestyle=none,fillstyle=solid,fillcolor=curcolor]
{
\newpath
\moveto(297.19332886,201.22973574)
\curveto(297.29911088,203.15030044)(296.89221024,204.75755795)(295.97262573,206.05151309)
\curveto(295.03674335,207.36986002)(293.55969404,208.04531508)(291.54147339,208.07788027)
\curveto(289.62903391,208.04531508)(288.19674368,207.33730797)(287.24459839,205.95385684)
\curveto(286.29244871,204.58665968)(285.83672,202.97940217)(285.87741089,201.13207949)
\curveto(285.86113403,199.31729646)(286.33313877,197.79141908)(287.29342651,196.55444277)
\curveto(288.24557175,195.30932521)(289.66158596,194.6664222)(291.54147339,194.62573184)
\curveto(294.67460178,194.6664222)(296.52599967,196.21671362)(297.09567261,199.27661074)
\lineto(296.14352417,199.27661074)
\curveto(295.59826622,196.76196047)(294.06425083,195.48836149)(291.54147339,195.45580996)
\curveto(289.96269243,195.47208546)(288.78268059,196.06209138)(288.00143433,197.22582949)
\curveto(287.19576812,198.34887295)(286.8010745,199.68350703)(286.81735229,201.22973574)
\lineto(297.19332886,201.22973574)
\moveto(286.81735229,202.05981387)
\curveto(286.93942072,203.38630281)(287.39921843,204.57445266)(288.19674683,205.62426699)
\curveto(288.99426892,206.67405994)(290.10917666,207.21930679)(291.54147339,207.26000918)
\curveto(293.03072322,207.21930679)(294.18225201,206.69440497)(294.99606323,205.68530215)
\curveto(295.79357853,204.66803981)(296.20861718,203.45954492)(296.24118042,202.05981387)
\lineto(286.81735229,202.05981387)
}
}
{
\newrgbcolor{curcolor}{0 0 0}
\pscustom[linestyle=none,fillstyle=solid,fillcolor=curcolor]
{
\newpath
\moveto(300.37936401,207.78491152)
\lineto(299.43942261,207.78491152)
\lineto(299.43942261,194.93090762)
\lineto(300.37936401,194.93090762)
\lineto(300.37936401,201.94995059)
\curveto(300.43632776,202.90209105)(300.57874298,203.64671921)(300.80661011,204.1838373)
\curveto(301.02633368,204.71279888)(301.36813021,205.20514865)(301.83200073,205.66088809)
\curveto(302.43420988,206.22240023)(303.05269884,206.59674882)(303.68746948,206.78393496)
\curveto(304.3059528,206.94668336)(304.87154468,206.98737343)(305.38424683,206.90600527)
\lineto(305.38424683,207.85815371)
\curveto(304.20422764,207.89069283)(303.16256201,207.63027643)(302.25924683,207.07690371)
\curveto(301.33964717,206.51536869)(300.74150323,205.79922357)(300.46481323,204.92846621)
\lineto(300.37936401,204.92846621)
\lineto(300.37936401,207.78491152)
}
}
{
\newrgbcolor{curcolor}{0 0 0}
\pscustom[linestyle=none,fillstyle=solid,fillcolor=curcolor]
{
\newpath
\moveto(223.9694519,160.60363711)
\lineto(223.9694519,148.57971133)
\lineto(224.92160034,148.57971133)
\lineto(224.92160034,160.60363711)
\lineto(227.59494019,160.60363711)
\lineto(227.59494019,161.43371523)
\lineto(224.92160034,161.43371523)
\lineto(224.92160034,164.00939883)
\curveto(224.92159709,164.72552852)(225.11284039,165.20567126)(225.49533081,165.44982852)
\curveto(225.82898551,165.68581401)(226.26436919,165.8038152)(226.80148315,165.80383242)
\curveto(227.18396462,165.8038152)(227.55831321,165.76312513)(227.92453003,165.68176211)
\lineto(227.92453003,166.4996332)
\curveto(227.55831321,166.58913342)(227.18396462,166.63389249)(226.80148315,166.63391055)
\curveto(226.00395278,166.63389249)(225.34477375,166.43858019)(224.82394409,166.04797305)
\curveto(224.25428005,165.63291693)(223.96944961,164.97780691)(223.9694519,164.08264102)
\lineto(223.9694519,161.43371523)
\lineto(221.67453003,161.43371523)
\lineto(221.67453003,160.60363711)
\lineto(223.9694519,160.60363711)
}
}
{
\newrgbcolor{curcolor}{0 0 0}
\pscustom[linestyle=none,fillstyle=solid,fillcolor=curcolor]
{
\newpath
\moveto(228.53488159,155.00060977)
\curveto(228.53488079,153.13699844)(229.04757559,151.56229298)(230.07296753,150.27648867)
\curveto(231.09021677,148.98254296)(232.54285204,148.31522592)(234.43087769,148.27453555)
\curveto(236.34330397,148.31522592)(237.80000824,148.98254296)(238.80099487,150.27648867)
\curveto(239.81823539,151.56229298)(240.32686118,153.13699844)(240.32687378,155.00060977)
\curveto(240.32686118,156.88862229)(239.81823539,158.47146576)(238.80099487,159.74914492)
\curveto(237.80000824,161.03493976)(236.34330397,161.69411879)(234.43087769,161.72668398)
\curveto(232.54285204,161.69411879)(231.09021677,161.03493976)(230.07296753,159.74914492)
\curveto(229.04757559,158.47146576)(228.53488079,156.88862229)(228.53488159,155.00060977)
\lineto(228.53488159,155.00060977)
\moveto(229.474823,155.00060977)
\curveto(229.47482125,156.60379185)(229.91020493,157.97911599)(230.78097534,159.12658633)
\curveto(231.61104958,160.27403557)(232.82768248,160.8681105)(234.43087769,160.90881289)
\curveto(236.06661154,160.8681105)(237.29952046,160.27403557)(238.12960815,159.12658633)
\curveto(238.95967505,157.97911599)(239.3747137,156.60379185)(239.37472534,155.00060977)
\curveto(239.3747137,153.42182888)(238.95967505,152.06278076)(238.12960815,150.92346133)
\curveto(237.29952046,149.74344714)(236.06661154,149.1371652)(234.43087769,149.10461367)
\curveto(232.82768248,149.1371652)(231.61104958,149.74344714)(230.78097534,150.92346133)
\curveto(229.91020493,152.06278076)(229.47482125,153.42182888)(229.474823,155.00060977)
\lineto(229.474823,155.00060977)
}
}
{
\newrgbcolor{curcolor}{0 0 0}
\pscustom[linestyle=none,fillstyle=solid,fillcolor=curcolor]
{
\newpath
\moveto(242.78048706,166.42639102)
\lineto(242.78048706,148.57971133)
\lineto(243.7326355,148.57971133)
\lineto(243.7326355,166.42639102)
\lineto(242.78048706,166.42639102)
}
}
{
\newrgbcolor{curcolor}{0 0 0}
\pscustom[linestyle=none,fillstyle=solid,fillcolor=curcolor]
{
\newpath
\moveto(246.97970581,166.42639102)
\lineto(246.97970581,148.57971133)
\lineto(247.93185425,148.57971133)
\lineto(247.93185425,166.42639102)
\lineto(246.97970581,166.42639102)
}
}
{
\newrgbcolor{curcolor}{0 0 0}
\pscustom[linestyle=none,fillstyle=solid,fillcolor=curcolor]
{
\newpath
\moveto(250.36105347,155.00060977)
\curveto(250.36105266,153.13699844)(250.87374746,151.56229298)(251.8991394,150.27648867)
\curveto(252.91638865,148.98254296)(254.36902391,148.31522592)(256.25704956,148.27453555)
\curveto(258.16947584,148.31522592)(259.62618011,148.98254296)(260.62716675,150.27648867)
\curveto(261.64440726,151.56229298)(262.15303306,153.13699844)(262.15304565,155.00060977)
\curveto(262.15303306,156.88862229)(261.64440726,158.47146576)(260.62716675,159.74914492)
\curveto(259.62618011,161.03493976)(258.16947584,161.69411879)(256.25704956,161.72668398)
\curveto(254.36902391,161.69411879)(252.91638865,161.03493976)(251.8991394,159.74914492)
\curveto(250.87374746,158.47146576)(250.36105266,156.88862229)(250.36105347,155.00060977)
\lineto(250.36105347,155.00060977)
\moveto(251.30099487,155.00060977)
\curveto(251.30099313,156.60379185)(251.73637681,157.97911599)(252.60714722,159.12658633)
\curveto(253.43722146,160.27403557)(254.65385436,160.8681105)(256.25704956,160.90881289)
\curveto(257.89278341,160.8681105)(259.12569233,160.27403557)(259.95578003,159.12658633)
\curveto(260.78584692,157.97911599)(261.20088557,156.60379185)(261.20089722,155.00060977)
\curveto(261.20088557,153.42182888)(260.78584692,152.06278076)(259.95578003,150.92346133)
\curveto(259.12569233,149.74344714)(257.89278341,149.1371652)(256.25704956,149.10461367)
\curveto(254.65385436,149.1371652)(253.43722146,149.74344714)(252.60714722,150.92346133)
\curveto(251.73637681,152.06278076)(251.30099313,153.42182888)(251.30099487,155.00060977)
\lineto(251.30099487,155.00060977)
}
}
{
\newrgbcolor{curcolor}{0 0 0}
\pscustom[linestyle=none,fillstyle=solid,fillcolor=curcolor]
{
\newpath
\moveto(263.95968628,161.43371523)
\lineto(262.93429565,161.43371523)
\lineto(267.01144409,148.57971133)
\lineto(268.20773315,148.57971133)
\lineto(271.75997925,160.22521914)
\lineto(271.80880737,160.22521914)
\lineto(275.3366394,148.57971133)
\lineto(276.5085144,148.57971133)
\lineto(280.63449097,161.43371523)
\lineto(279.58468628,161.43371523)
\lineto(275.98361206,149.67834414)
\lineto(275.93478394,149.67834414)
\lineto(272.4069519,161.43371523)
\lineto(271.13742065,161.43371523)
\lineto(267.63400269,149.67834414)
\lineto(267.58517456,149.67834414)
\lineto(263.95968628,161.43371523)
}
}
{
\newrgbcolor{curcolor}{0 0 0}
\pscustom[linestyle=none,fillstyle=solid,fillcolor=curcolor]
{
\newpath
\moveto(283.1857605,148.57971133)
\lineto(283.1857605,161.43371523)
\lineto(282.23361206,161.43371523)
\lineto(282.23361206,148.57971133)
\lineto(283.1857605,148.57971133)
\moveto(283.1857605,163.85070742)
\lineto(283.1857605,166.42639102)
\lineto(282.23361206,166.42639102)
\lineto(282.23361206,163.85070742)
\lineto(283.1857605,163.85070742)
}
}
{
\newrgbcolor{curcolor}{0 0 0}
\pscustom[linestyle=none,fillstyle=solid,fillcolor=curcolor]
{
\newpath
\moveto(286.21310425,148.57971133)
\lineto(287.15304565,148.57971133)
\lineto(287.15304565,155.52551211)
\curveto(287.15304331,157.12869366)(287.57215096,158.42263768)(288.41036987,159.40734805)
\curveto(289.22416754,160.39203676)(290.34721329,160.89252454)(291.7795105,160.90881289)
\curveto(292.65027089,160.89252454)(293.34200197,160.7419713)(293.85470581,160.45715273)
\curveto(294.35925355,160.17231041)(294.71732611,159.79796183)(294.92892456,159.33410586)
\curveto(295.16491681,158.87836639)(295.31547004,158.41856868)(295.38058472,157.95471133)
\curveto(295.4131262,157.4745592)(295.42940222,157.05952056)(295.42941284,156.70959414)
\lineto(295.42941284,148.57971133)
\lineto(296.38156128,148.57971133)
\lineto(296.38156128,156.52648867)
\curveto(296.38154971,156.94151937)(296.36527368,157.44200715)(296.33273315,158.02795352)
\curveto(296.26761753,158.59760496)(296.09265026,159.16319684)(295.80783081,159.72473086)
\curveto(295.53926539,160.2943806)(295.08760569,160.77045435)(294.45285034,161.15295352)
\curveto(293.82621372,161.51915151)(292.94323934,161.71039481)(291.80392456,161.72668398)
\curveto(290.78666598,161.72667084)(289.85486353,161.47032344)(289.0085144,160.95764102)
\curveto(288.16215688,160.4205198)(287.56401295,159.65961561)(287.21408081,158.67492617)
\lineto(287.15304565,158.67492617)
\lineto(287.15304565,161.43371523)
\lineto(286.21310425,161.43371523)
\lineto(286.21310425,148.57971133)
}
}
{
\newrgbcolor{curcolor}{0 0 0}
\pscustom[linestyle=none,fillstyle=solid,fillcolor=curcolor]
{
\newpath
\moveto(304.16964722,160.90881289)
\curveto(305.75655332,160.8681105)(306.93249615,160.31472563)(307.69747925,159.24865664)
\curveto(308.46244254,158.18256631)(308.84492914,156.90082931)(308.84494019,155.4034418)
\curveto(308.84492914,153.90604064)(308.46244254,152.63244165)(307.69747925,151.58264102)
\curveto(306.93249615,150.46773027)(305.75655332,149.89806938)(304.16964722,149.87365664)
\curveto(302.67224651,149.89806938)(301.53292473,150.45145425)(300.75167847,151.53381289)
\curveto(299.96228828,152.58361357)(299.56759467,153.87348859)(299.56759644,155.4034418)
\curveto(299.56759467,156.83572521)(299.95415027,158.10118618)(300.7272644,159.19982852)
\curveto(301.47595864,160.29844961)(302.62341843,160.8681105)(304.16964722,160.90881289)
\lineto(304.16964722,160.90881289)
\moveto(308.86935425,161.43371523)
\lineto(308.86935425,158.74816836)
\lineto(308.82052612,158.74816836)
\curveto(308.45430453,159.73285773)(307.84802258,160.47748589)(307.00167847,160.98205508)
\curveto(306.14717793,161.47846145)(305.20316846,161.72667084)(304.16964722,161.72668398)
\curveto(302.37114004,161.69411879)(301.00395391,161.07156082)(300.06808472,159.8590082)
\curveto(299.10779695,158.65457104)(298.6276542,157.16938373)(298.62765503,155.4034418)
\curveto(298.6276542,153.58865814)(299.08338291,152.0871948)(299.99484253,150.89904727)
\curveto(300.91443577,149.68648105)(302.30603594,149.0720611)(304.16964722,149.05578555)
\curveto(306.35062825,149.0720611)(307.90091967,150.04862262)(308.82052612,151.98547305)
\lineto(308.86935425,151.98547305)
\lineto(308.86935425,149.03137148)
\curveto(308.86934318,148.81164469)(308.85306715,148.46984816)(308.82052612,148.00598086)
\curveto(308.755411,147.52583868)(308.58044373,147.00907488)(308.29562378,146.45568789)
\curveto(308.01078284,145.9348572)(307.55098512,145.49133551)(306.91622925,145.12512148)
\curveto(306.28959315,144.72636232)(305.41475679,144.51884299)(304.29171753,144.50256289)
\curveto(303.20935535,144.50256697)(302.29382892,144.75484536)(301.5451355,145.25939883)
\curveto(300.78015856,145.75582092)(300.33256786,146.56148418)(300.20236206,147.67639102)
\lineto(299.25021362,147.67639102)
\curveto(299.34786832,146.22782566)(299.86463213,145.19429805)(300.80050659,144.57580508)
\curveto(301.71196101,143.98173416)(302.86755881,143.68469669)(304.26730347,143.6846918)
\curveto(305.65075916,143.68469669)(306.72904584,143.91663005)(307.50216675,144.38049258)
\curveto(308.26713024,144.82808748)(308.81644609,145.38554135)(309.15011597,146.05285586)
\curveto(309.48376313,146.72017543)(309.68314445,147.35494042)(309.7482605,147.95715273)
\curveto(309.79707662,148.5593663)(309.82149066,148.99068097)(309.82150269,149.25109805)
\lineto(309.82150269,161.43371523)
\lineto(308.86935425,161.43371523)
}
}
{
\newrgbcolor{curcolor}{0 0 0}
\pscustom[linestyle=none,fillstyle=solid,fillcolor=curcolor]
{
\newpath
\moveto(240.32687378,111.81329287)
\lineto(240.32687378,99.78936709)
\lineto(241.27902222,99.78936709)
\lineto(241.27902222,111.81329287)
\lineto(243.95236206,111.81329287)
\lineto(243.95236206,112.643371)
\lineto(241.27902222,112.643371)
\lineto(241.27902222,115.21905459)
\curveto(241.27901897,115.93518428)(241.47026227,116.41532703)(241.85275269,116.65948428)
\curveto(242.18640739,116.89546978)(242.62179106,117.01347096)(243.15890503,117.01348818)
\curveto(243.5413865,117.01347096)(243.91573508,116.9727809)(244.2819519,116.89141787)
\lineto(244.2819519,117.70928896)
\curveto(243.91573508,117.79878918)(243.5413865,117.84354825)(243.15890503,117.84356631)
\curveto(242.36137466,117.84354825)(241.70219563,117.64823595)(241.18136597,117.25762881)
\curveto(240.61170193,116.84257269)(240.32687148,116.18746267)(240.32687378,115.29229678)
\lineto(240.32687378,112.643371)
\lineto(238.0319519,112.643371)
\lineto(238.0319519,111.81329287)
\lineto(240.32687378,111.81329287)
}
}
{
\newrgbcolor{curcolor}{0 0 0}
\pscustom[linestyle=none,fillstyle=solid,fillcolor=curcolor]
{
\newpath
\moveto(246.4303894,112.643371)
\lineto(245.490448,112.643371)
\lineto(245.490448,99.78936709)
\lineto(246.4303894,99.78936709)
\lineto(246.4303894,106.80841006)
\curveto(246.48735315,107.76055052)(246.62976837,108.50517869)(246.8576355,109.04229678)
\curveto(247.07735907,109.57125835)(247.4191556,110.06360812)(247.88302612,110.51934756)
\curveto(248.48523527,111.0808597)(249.10372423,111.45520829)(249.73849487,111.64239443)
\curveto(250.35697819,111.80514283)(250.92257007,111.8458329)(251.43527222,111.76446475)
\lineto(251.43527222,112.71661318)
\curveto(250.25525303,112.74915231)(249.2135874,112.4887359)(248.31027222,111.93536318)
\curveto(247.39067256,111.37382816)(246.79252863,110.65768304)(246.51583862,109.78692568)
\lineto(246.4303894,109.78692568)
\lineto(246.4303894,112.643371)
}
}
{
\newrgbcolor{curcolor}{0 0 0}
\pscustom[linestyle=none,fillstyle=solid,fillcolor=curcolor]
{
\newpath
\moveto(253.59591675,99.78936709)
\lineto(253.59591675,112.643371)
\lineto(252.64376831,112.643371)
\lineto(252.64376831,99.78936709)
\lineto(253.59591675,99.78936709)
\moveto(253.59591675,115.06036318)
\lineto(253.59591675,117.63604678)
\lineto(252.64376831,117.63604678)
\lineto(252.64376831,115.06036318)
\lineto(253.59591675,115.06036318)
}
}
{
\newrgbcolor{curcolor}{0 0 0}
\pscustom[linestyle=none,fillstyle=solid,fillcolor=curcolor]
{
\newpath
\moveto(267.365448,106.08819521)
\curveto(267.47123002,108.00875991)(267.06432938,109.61601742)(266.14474487,110.90997256)
\curveto(265.20886249,112.22831949)(263.73181318,112.90377455)(261.71359253,112.93633975)
\curveto(259.80115305,112.90377455)(258.36886282,112.19576744)(257.41671753,110.81231631)
\curveto(256.46456785,109.44511915)(256.00883914,107.83786165)(256.04953003,105.99053896)
\curveto(256.03325317,104.17575593)(256.50525791,102.64987855)(257.46554565,101.41290225)
\curveto(258.41769089,100.16778468)(259.8337051,99.52488168)(261.71359253,99.48419131)
\curveto(264.84672092,99.52488168)(266.69811881,101.0751731)(267.26779175,104.13507021)
\lineto(266.31564331,104.13507021)
\curveto(265.77038536,101.62041995)(264.23636997,100.34682096)(261.71359253,100.31426943)
\curveto(260.13481157,100.33054493)(258.95479973,100.92055085)(258.17355347,102.08428896)
\curveto(257.36788726,103.20733242)(256.97319364,104.5419665)(256.98947144,106.08819521)
\lineto(267.365448,106.08819521)
\moveto(256.98947144,106.91827334)
\curveto(257.11153986,108.24476228)(257.57133757,109.43291213)(258.36886597,110.48272646)
\curveto(259.16638806,111.53251941)(260.2812958,112.07776626)(261.71359253,112.11846865)
\curveto(263.20284236,112.07776626)(264.35437115,111.55286444)(265.16818237,110.54376162)
\curveto(265.96569767,109.52649928)(266.38073632,108.31800439)(266.41329956,106.91827334)
\lineto(256.98947144,106.91827334)
}
}
{
\newrgbcolor{curcolor}{0 0 0}
\pscustom[linestyle=none,fillstyle=solid,fillcolor=curcolor]
{
\newpath
\moveto(269.61154175,99.78936709)
\lineto(270.55148315,99.78936709)
\lineto(270.55148315,106.73516787)
\curveto(270.55148081,108.33834943)(270.97058846,109.63229344)(271.80880737,110.61700381)
\curveto(272.62260504,111.60169252)(273.74565079,112.1021803)(275.177948,112.11846865)
\curveto(276.04870839,112.1021803)(276.74043947,111.95162706)(277.25314331,111.6668085)
\curveto(277.75769105,111.38196617)(278.11576361,111.00761759)(278.32736206,110.54376162)
\curveto(278.56335431,110.08802216)(278.71390754,109.62822444)(278.77902222,109.16436709)
\curveto(278.8115637,108.68421497)(278.82783972,108.26917632)(278.82785034,107.9192499)
\lineto(278.82785034,99.78936709)
\lineto(279.77999878,99.78936709)
\lineto(279.77999878,107.73614443)
\curveto(279.77998721,108.15117513)(279.76371118,108.65166292)(279.73117065,109.23760928)
\curveto(279.66605503,109.80726072)(279.49108776,110.3728526)(279.20626831,110.93438662)
\curveto(278.93770289,111.50403636)(278.48604319,111.98011011)(277.85128784,112.36260928)
\curveto(277.22465122,112.72880728)(276.34167684,112.92005057)(275.20236206,112.93633975)
\curveto(274.18510348,112.9363266)(273.25330103,112.6799792)(272.4069519,112.16729678)
\curveto(271.56059438,111.63017556)(270.96245045,110.86927137)(270.61251831,109.88458193)
\lineto(270.55148315,109.88458193)
\lineto(270.55148315,112.643371)
\lineto(269.61154175,112.643371)
\lineto(269.61154175,99.78936709)
}
}
{
\newrgbcolor{curcolor}{0 0 0}
\pscustom[linestyle=none,fillstyle=solid,fillcolor=curcolor]
{
\newpath
\moveto(287.75119019,100.31426943)
\curveto(286.09916705,100.34682096)(284.89067217,100.95310291)(284.1257019,102.13311709)
\curveto(283.33631174,103.27243652)(282.94161813,104.62334663)(282.94161987,106.18585146)
\curveto(282.9253421,107.80530959)(283.30375969,109.18877175)(284.07687378,110.33624209)
\curveto(284.82556807,111.48369133)(286.03813196,112.07776626)(287.71456909,112.11846865)
\curveto(289.36657915,112.07776626)(290.58321205,111.46741531)(291.36447144,110.28741396)
\curveto(292.12129645,109.12366765)(292.49971404,107.75648152)(292.49972534,106.18585146)
\curveto(292.49971404,104.65589868)(292.11315844,103.31719559)(291.34005737,102.16973818)
\curveto(290.57507404,100.96530992)(289.37878617,100.34682096)(287.75119019,100.31426943)
\lineto(287.75119019,100.31426943)
\moveto(292.5241394,99.78936709)
\lineto(293.46408081,99.78936709)
\lineto(293.46408081,117.63604678)
\lineto(292.5241394,117.63604678)
\lineto(292.5241394,109.76251162)
\lineto(292.47531128,109.76251162)
\curveto(292.32067776,110.28333446)(292.08467539,110.74313218)(291.76730347,111.14190615)
\curveto(291.4499104,111.54065742)(291.07556182,111.87431594)(290.64425659,112.14288271)
\curveto(289.7816178,112.67184119)(288.80505627,112.9363266)(287.71456909,112.93633975)
\curveto(285.80212959,112.92005057)(284.36983936,112.27714757)(283.41769409,111.00762881)
\curveto(282.4736824,109.74622562)(282.00167766,108.13896812)(282.00167847,106.18585146)
\curveto(281.97726362,104.35479221)(282.42485432,102.79636278)(283.3444519,101.5105585)
\curveto(284.23149314,100.20033673)(285.63123132,99.52488168)(287.54367065,99.48419131)
\curveto(289.81416985,99.48419161)(291.45804841,100.5014432)(292.47531128,102.53594912)
\lineto(292.5241394,102.53594912)
\lineto(292.5241394,99.78936709)
}
}
\end{pspicture}

\caption{Tipos de relacionamientos entre usuarios.}
\label{contactos}
\end{figure}

Estos recursos, son los componentes propios de un sitio web, además de darle las
características de una red social propiamente dicha.

\subsection{Fomento a la participación}

Una parte fundamental del sistema, y el factor clave para el éxito de toda red
social, son sus políticas que propician la cultura de participación.

Estos elementos, que están inspirados en la tendencia de los sitios considerados
dentro de la web 2.0\footnote{Definición disponible en:
http://es.wikipedia.org/wiki/Web\_2.0}, han sido considerados como base para el
establecimiento de las definiciones siguientes:

\begin{description}
\item [Comentarios] Los comentarios crean el espacio de debate entre usuarios,
estos se encuentran en cada tipo de recursos que es provisto por el sistema.
\item [Valoraciones] Una valoración es un voto a favor o en contra de un recurso
determinado, y define la calidad misma del recurso segun sus consumidores.
\item [Etiquetado] Las etiquetas\footnote{Puede verse la definición extendida
en: http://es.wikipedia.org/wiki/Etiqueta\_(metadato)} son palabras clave que
son asignadas a un recurso, estas son de tipo informal (es decir, definidas por
el creador del recurso), y sirven como un medio alternativo de clasificación de
los recursos conocido como ``folcsonomías''\footnote{Puede verse la definición
extendida en http://es.wikipedia.org/wiki/Folcsonomía}.
\item [Sistema de reputación] Un sistema de reputación\footnote{Si bien vamos a
ahondar en este concepto, pueden verse los detalles introductorios en:
http://en.wikipedia.org/wiki/Reputation\_system} define un conjunto de políticas
de fomento a la interacción o participación de los usuarios, entre los múltiples
métodos que pueden encontrarse en los sitios web actuales, se han definido
cuatro indicadores a ser tomados en cuenta:
\begin{description}
    \item [Actividad] El indicador de actividad, se basa en el número de
    recursos que un usuario ha creado en el sistema.
    \item [Participación] El indicador de participación mide el número de
    comentarios creados por el usuarios en los recursos del sistema.
    \item [Popularidad] El indicador de popularidad mide el apoyo de los
    usuarios hacia la calidad de los recursos que crean otros usuarios, es
    medida a partir de las valoraciones realizadas en el recurso.
    \item [Sociabilidad] El indicador de sociabilidad mide el número de
    conexiones (contactos) que posee un usuario.
    \end{description}
\end{description}

Estos elementos deben estar disponibles para cualquiera de los recursos
intercambiables definidos anteriormente.

\section{Requisitos no funcionales}

Un requisito no funcional es aquel que especifica criterios que pueden usarse
para juzgar la operación de un sistema en lugar de sus comportamientos
específicos\footnote{Para una definición exacta puede consultarse:
http://es.wikipedia.org/wiki/Requisito\_no\_funcional}. Para la construcción de
esta aplicación web, se han definido un conjunto de herramientas, estas se
describen en esta sección.

\subsection{Contexto de despliegue}

Se ha de construir el sistema, considerando a disposición del desarrollo un
servidor que se ejecute en un sistema operativo GNU/Linux, esto implica varias
consideraciones; como la disposición de un amplio conjunto de
herramientas disponibles para tareas de automatización y scripting.

\subsection{Servidor web}

Se ha determinado que el sitio web, puede ser ejecutado en el servidor web mas
popular que existe, además de en el segundo, estos son:

\begin{itemize}
\item Apache Web Server 2
\item Nginx\footnote{Pueden apreciarse sus características en:
http://es.wikipedia.org/wiki/Nginx}
\end{itemize}

Se crearán las pruebas necesarias para asegurar el correcto funcionamiento del
sistema en estos dos servidores HTTP.

\subsection{Base de datos}

Se ha determinado usar el DBMS mas popular: MySQL, además de probar el sistema
en la versión alternativa de este: MariaDB.

\subsection{Lenguaje de programación}

Para la creación del sistema, se ha optado por el uso del lenguaje PHP 5, con la
utilización de las librerías, estándares, y conceptos  que componen el marco
de trabajo denominado Zend Framework.

Para la implementación con Zend Framework, se ha establecido un conjunto de
módulos a ser desarrollados, estos se detallan en la sección siguiente.

\section{Diseño de paquetes}

Para la construcción del sistema se han determinado un conjunto de módulos ha
ser desarrollados, estos se detallan en la figura \ref{paquetes}.

\begin{figure}
\centering
%LaTeX with PSTricks extensions
%%Creator: inkscape 0.48.4
%%Please note this file requires PSTricks extensions
\psset{xunit=.5pt,yunit=.5pt,runit=.5pt}
\begin{pspicture}(885.82672119,992.12597656)
{
\newrgbcolor{curcolor}{1 0.50196081 0.50196081}
\pscustom[linewidth=2.63455725,linecolor=curcolor]
{
\newpath
\moveto(284.31857,881.72407456)
\lineto(368.62441,881.72407456)
\lineto(368.62441,934.41520456)
\lineto(313.29871,934.41520456)
\lineto(313.29871,944.95344456)
\lineto(284.31857,944.95344456)
\lineto(284.31857,881.72407456)
\closepath
}
}
{
\newrgbcolor{curcolor}{0 0 0}
\pscustom[linestyle=none,fillstyle=solid,fillcolor=curcolor]
{
\newpath
\moveto(296.89445526,915.8163008)
\lineto(296.89445526,916.89688096)
\lineto(303.1399513,919.53401111)
\lineto(303.1399513,918.38267868)
\lineto(298.18729224,916.35015885)
\lineto(303.1399513,914.29834295)
\lineto(303.1399513,913.14701052)
\lineto(296.89445526,915.8163008)
}
}
{
\newrgbcolor{curcolor}{0 0 0}
\pscustom[linestyle=none,fillstyle=solid,fillcolor=curcolor]
{
\newpath
\moveto(304.58715708,915.8163008)
\lineto(304.58715708,916.89688096)
\lineto(310.83265312,919.53401111)
\lineto(310.83265312,918.38267868)
\lineto(305.87999405,916.35015885)
\lineto(310.83265312,914.29834295)
\lineto(310.83265312,913.14701052)
\lineto(304.58715708,915.8163008)
}
}
{
\newrgbcolor{curcolor}{0 0 0}
\pscustom[linestyle=none,fillstyle=solid,fillcolor=curcolor]
{
\newpath
\moveto(313.49551157,911.69337292)
\lineto(312.42136344,911.69337292)
\lineto(312.42136344,921.12272123)
\lineto(313.57912789,921.12272123)
\lineto(313.57912789,917.75877227)
\curveto(314.06795927,918.37195196)(314.69186505,918.67854483)(315.4508471,918.67855182)
\curveto(315.87106841,918.67854483)(316.26770954,918.59278459)(316.64077168,918.42127083)
\curveto(317.01811168,918.25403162)(317.32684856,918.01604694)(317.56698325,917.70731608)
\curveto(317.81139394,917.4028612)(318.00221048,917.03409214)(318.13943345,916.60100782)
\curveto(318.27664326,916.16791368)(318.34525146,915.70480836)(318.34525824,915.21169047)
\curveto(318.34525146,914.04105961)(318.05581063,913.13628904)(317.4769349,912.49737602)
\curveto(316.89804733,911.85846139)(316.20338935,911.53900448)(315.39295888,911.53900433)
\curveto(314.58680875,911.53900448)(313.95432694,911.87561344)(313.49551157,912.54883222)
\lineto(313.49551157,911.69337292)
\moveto(313.48264752,915.16023427)
\curveto(313.4826456,914.34122047)(313.59413392,913.74947478)(313.81711281,913.38499544)
\curveto(314.18159159,912.78896005)(314.674713,912.4909432)(315.29647851,912.49094399)
\curveto(315.80246021,912.4909432)(316.23983746,912.70963182)(316.60861156,913.14701052)
\curveto(316.97737556,913.58867433)(317.16176008,914.2447402)(317.16176569,915.1152101)
\curveto(317.16176008,916.00711322)(316.98380758,916.66532309)(316.62790763,917.0898417)
\curveto(316.27628556,917.51434951)(315.84962835,917.72660612)(315.3479347,917.72661215)
\curveto(314.84194547,917.72660612)(314.40456823,917.50577349)(314.03580165,917.0641136)
\curveto(313.66703013,916.62673098)(313.4826456,915.99210517)(313.48264752,915.16023427)
}
}
{
\newrgbcolor{curcolor}{0 0 0}
\pscustom[linestyle=none,fillstyle=solid,fillcolor=curcolor]
{
\newpath
\moveto(324.21769576,912.53596817)
\curveto(323.78888921,912.17148629)(323.37509603,911.91420555)(322.97631498,911.7641252)
\curveto(322.58181377,911.6140447)(322.15730056,911.53900448)(321.70277407,911.53900433)
\curveto(320.95236912,911.53900448)(320.37563148,911.721245)(319.97255941,912.08572643)
\curveto(319.56948518,912.45449509)(319.36794861,912.92403243)(319.36794909,913.49433986)
\curveto(319.36794861,913.82880301)(319.44298882,914.13325188)(319.59306995,914.40768737)
\curveto(319.74743769,914.68640545)(319.94683026,914.90938209)(320.19124825,915.07661795)
\curveto(320.43995166,915.24384704)(320.71867246,915.3703434)(321.02741147,915.45610741)
\curveto(321.25467399,915.51613582)(321.59771496,915.57402398)(322.05653544,915.62977208)
\curveto(322.99131894,915.74125646)(323.6795449,915.87418484)(324.12121538,916.02855761)
\curveto(324.12549817,916.18720973)(324.12764217,916.28797802)(324.12764741,916.33086278)
\curveto(324.12764217,916.80253948)(324.01829786,917.13486043)(323.79961415,917.32782662)
\curveto(323.5037364,917.58938973)(323.06421514,917.7201741)(322.48104907,917.72018013)
\curveto(321.93646793,917.7201741)(321.53339478,917.62369382)(321.27182841,917.43073901)
\curveto(321.0145453,917.24206074)(320.82372876,916.90545178)(320.69937821,916.42091112)
\lineto(319.56734185,916.57527972)
\curveto(319.67025347,917.05982022)(319.83962995,917.45002933)(320.07547181,917.74590823)
\curveto(320.3113113,918.04606303)(320.65220827,918.27547168)(321.09816375,918.43413487)
\curveto(321.54411481,918.5970726)(322.06082028,918.67854483)(322.64828171,918.67855182)
\curveto(323.23144762,918.67854483)(323.70527297,918.60993664)(324.06975919,918.47272702)
\curveto(324.43423505,918.33550385)(324.70223581,918.16183936)(324.87376228,917.95173302)
\curveto(325.04527679,917.74590217)(325.16534113,917.48433343)(325.23395567,917.167026)
\curveto(325.27254144,916.96977196)(325.29183749,916.61386695)(325.29184389,916.09930989)
\lineto(325.29184389,914.55562394)
\curveto(325.29183749,913.47933001)(325.31542156,912.79753607)(325.36259616,912.51024007)
\curveto(325.41404584,912.22723045)(325.51267012,911.95494167)(325.6584693,911.69337292)
\lineto(324.44924865,911.69337292)
\curveto(324.32917875,911.93350161)(324.25199453,912.21436641)(324.21769576,912.53596817)
\moveto(324.12121538,915.12164212)
\curveto(323.70098496,914.95011821)(323.07064716,914.80432579)(322.2302001,914.68426444)
\curveto(321.75422741,914.61565325)(321.41761845,914.53846903)(321.22037222,914.45271155)
\curveto(321.02312133,914.36694854)(320.87089689,914.24045218)(320.76369846,914.07322209)
\curveto(320.65649628,913.91027524)(320.60289613,913.72803472)(320.60289784,913.52649998)
\curveto(320.60289613,913.21776127)(320.71867246,912.96048054)(320.95022718,912.75465701)
\curveto(321.18606579,912.54883136)(321.52910677,912.44591907)(321.97935114,912.44591982)
\curveto(322.42530132,912.44591907)(322.82194245,912.54239934)(323.16927572,912.73536094)
\curveto(323.51660043,912.93260846)(323.77173716,913.20060922)(323.93468667,913.53936403)
\curveto(324.05903398,913.80093093)(324.12121016,914.18685203)(324.12121538,914.69712849)
\lineto(324.12121538,915.12164212)
}
}
{
\newrgbcolor{curcolor}{0 0 0}
\pscustom[linestyle=none,fillstyle=solid,fillcolor=curcolor]
{
\newpath
\moveto(326.6297049,913.73232477)
\lineto(327.77460531,913.91242147)
\curveto(327.83892394,913.45360194)(328.01687645,913.10198494)(328.30846336,912.85756941)
\curveto(328.60433412,912.61315155)(329.0159833,912.4909432)(329.54341212,912.49094399)
\curveto(330.07512231,912.4909432)(330.46961944,912.5981435)(330.72690467,912.81254523)
\curveto(330.9841809,913.03123274)(331.11282127,913.28636946)(331.11282616,913.57795618)
\curveto(331.11282127,913.83952304)(330.99918895,914.04534763)(330.77192885,914.19543056)
\curveto(330.61326785,914.29834035)(330.21877072,914.42912472)(329.58843629,914.58778407)
\curveto(328.73940651,914.80218178)(328.14980483,914.98656631)(327.81962948,915.1409382)
\curveto(327.49373896,915.2995912)(327.24503425,915.51613582)(327.07351461,915.7905727)
\curveto(326.90628128,916.06928939)(326.82266504,916.37588227)(326.82266564,916.71035224)
\curveto(326.82266504,917.01479609)(326.89127324,917.29566089)(327.02849044,917.55294748)
\curveto(327.16999403,917.81451037)(327.36081058,918.03105499)(327.60094064,918.20258198)
\curveto(327.78103578,918.33550385)(328.02545247,918.44699217)(328.33419146,918.53704727)
\curveto(328.64721424,918.6313767)(328.9816792,918.67854483)(329.33758732,918.67855182)
\curveto(329.87358574,918.67854483)(330.34312308,918.60136061)(330.74620075,918.44699892)
\curveto(331.15355739,918.29262373)(331.45371824,918.08251113)(331.64668421,917.8166605)
\curveto(331.83963934,917.55508563)(331.97256772,917.20346863)(332.04546975,916.76180844)
\lineto(330.91343339,916.60743984)
\curveto(330.86197256,916.95905193)(330.71189213,917.23348471)(330.46319166,917.43073901)
\curveto(330.21877072,917.62798184)(329.87144173,917.72660612)(329.42120365,917.72661215)
\curveto(328.88948693,917.72660612)(328.50999785,917.63870187)(328.28273526,917.46289914)
\curveto(328.05546856,917.28708487)(327.94183623,917.08126028)(327.94183795,916.84542476)
\curveto(327.94183623,916.69533918)(327.98900437,916.56026679)(328.0833425,916.4402072)
\curveto(328.17767691,916.3158501)(328.32561333,916.2129378)(328.5271522,916.13147001)
\curveto(328.64292623,916.08858545)(328.9838232,915.98996117)(329.54984414,915.83559687)
\curveto(330.36885115,915.61690411)(330.93915678,915.43680759)(331.26076273,915.29530679)
\curveto(331.58664662,915.1580868)(331.84178335,914.95655022)(332.02617368,914.69069646)
\curveto(332.2105524,914.42483671)(332.30274466,914.09465977)(332.30275074,913.70016465)
\curveto(332.30274466,913.31424154)(332.18911234,912.9497605)(331.96185343,912.60672044)
\curveto(331.73887106,912.26796656)(331.41512613,912.00425381)(330.99061769,911.81558139)
\curveto(330.56609971,911.63119675)(330.08584234,911.53900448)(329.54984414,911.53900433)
\curveto(328.66222229,911.53900448)(327.98471636,911.72338901)(327.51732432,912.09215846)
\curveto(327.0542177,912.46092711)(326.75834486,913.00764867)(326.6297049,913.73232477)
}
}
{
\newrgbcolor{curcolor}{0 0 0}
\pscustom[linestyle=none,fillstyle=solid,fillcolor=curcolor]
{
\newpath
\moveto(338.35528795,913.89312539)
\lineto(339.55164455,913.74518882)
\curveto(339.36296528,913.04624078)(339.01349228,912.50380723)(338.50322452,912.11788656)
\curveto(337.99294537,911.73196503)(337.34116751,911.53900448)(336.54788899,911.53900433)
\curveto(335.54877841,911.53900448)(334.75549615,911.84559736)(334.16803983,912.45878387)
\curveto(333.58486881,913.07625687)(333.29328398,913.94029133)(333.29328446,915.05088985)
\curveto(333.29328398,916.20007377)(333.58915682,917.09198031)(334.18090388,917.72661215)
\curveto(334.77264819,918.36123193)(335.54020238,918.67854483)(336.48356874,918.67855182)
\curveto(337.39691167,918.67854483)(338.1430258,918.36766394)(338.72191336,917.74590823)
\curveto(339.3007891,917.1241404)(339.59022992,916.24938591)(339.5902367,915.12164212)
\curveto(339.59022992,915.0530305)(339.58808592,914.95011821)(339.58380468,914.81290493)
\lineto(334.48964107,914.81290493)
\curveto(334.53251951,914.06249968)(334.74477611,913.48790604)(335.12641152,913.0891223)
\curveto(335.50804229,912.69033577)(335.98401165,912.4909432)(336.55432102,912.49094399)
\curveto(336.97883048,912.4909432)(337.34116751,912.60243151)(337.6413332,912.82540928)
\curveto(337.94148922,913.04838479)(338.1794739,913.4042898)(338.35528795,913.89312539)
\moveto(334.55396131,915.7648446)
\lineto(338.368152,915.7648446)
\curveto(338.31669029,916.33943416)(338.17089788,916.77037939)(337.93077431,917.05768158)
\curveto(337.56200014,917.50362948)(337.08388678,917.72660612)(336.49643279,917.72661215)
\curveto(335.96471559,917.72660612)(335.51661831,917.54865361)(335.15213962,917.1927541)
\curveto(334.79194425,916.83684358)(334.59255168,916.36087423)(334.55396131,915.7648446)
}
}
{
\newrgbcolor{curcolor}{0 0 0}
\pscustom[linestyle=none,fillstyle=solid,fillcolor=curcolor]
{
\newpath
\moveto(347.10927351,915.8163008)
\lineto(340.86377746,913.14701052)
\lineto(340.86377746,914.29834295)
\lineto(345.81000451,916.35015885)
\lineto(340.86377746,918.38267868)
\lineto(340.86377746,919.53401111)
\lineto(347.10927351,916.89688096)
\lineto(347.10927351,915.8163008)
}
}
{
\newrgbcolor{curcolor}{0 0 0}
\pscustom[linestyle=none,fillstyle=solid,fillcolor=curcolor]
{
\newpath
\moveto(354.80197294,915.8163008)
\lineto(348.5564769,913.14701052)
\lineto(348.5564769,914.29834295)
\lineto(353.50270394,916.35015885)
\lineto(348.5564769,918.38267868)
\lineto(348.5564769,919.53401111)
\lineto(354.80197294,916.89688096)
\lineto(354.80197294,915.8163008)
}
}
{
\newrgbcolor{curcolor}{0 0 0}
\pscustom[linestyle=none,fillstyle=solid,fillcolor=curcolor]
{
\newpath
\moveto(295.75082867,898.76500506)
\lineto(297.43601916,898.76500506)
\lineto(297.43601916,897.7616092)
\curveto(297.6547052,898.10464435)(297.95057804,898.38336514)(298.32363857,898.59777242)
\curveto(298.69669217,898.81216636)(299.11048535,898.91936667)(299.56501935,898.91937365)
\curveto(300.3582969,898.91936667)(301.03151482,898.60848578)(301.58467513,897.98673006)
\curveto(302.13782198,897.36496224)(302.41439876,896.49878377)(302.41440632,895.38819206)
\curveto(302.41439876,894.24757736)(302.13567797,893.35995883)(301.5782431,892.72533381)
\curveto(301.02079479,892.09499522)(300.34543287,891.77982632)(299.5521553,891.77982617)
\curveto(299.17480553,891.77982632)(298.83176455,891.85486654)(298.52303134,892.00494704)
\curveto(298.21857881,892.15502739)(297.89697789,892.41230813)(297.55822763,892.77679001)
\lineto(297.55822763,889.33565676)
\lineto(295.75082867,889.33565676)
\lineto(295.75082867,898.76500506)
\moveto(297.53893155,895.46537636)
\curveto(297.53892887,894.69781864)(297.6911533,894.12965702)(297.99560531,893.7608898)
\curveto(298.30005104,893.39640693)(298.6709641,893.21416641)(299.10834559,893.21416769)
\curveto(299.52856654,893.21416641)(299.87803954,893.38139889)(300.15676563,893.71586562)
\curveto(300.43548112,894.05461681)(300.57484152,894.60777038)(300.57484724,895.37532801)
\curveto(300.57484152,896.09142261)(300.43119311,896.62313613)(300.14390158,896.97047015)
\curveto(299.85659948,897.3177941)(299.50069446,897.4914586)(299.07618547,897.49146416)
\curveto(298.63451599,897.4914586)(298.26789095,897.31993811)(297.97630924,896.97690218)
\curveto(297.68472129,896.63814417)(297.53892887,896.13430273)(297.53893155,895.46537636)
}
}
{
\newrgbcolor{curcolor}{0 0 0}
\pscustom[linestyle=none,fillstyle=solid,fillcolor=curcolor]
{
\newpath
\moveto(305.20590466,896.68102904)
\lineto(303.56573834,896.97690218)
\curveto(303.75012221,897.63725102)(304.06743512,898.12608441)(304.51767801,898.44340382)
\curveto(304.96791768,898.76071022)(305.63684759,898.91936667)(306.52446974,898.91937365)
\curveto(307.33061242,898.91936667)(307.93093413,898.82288639)(308.32543667,898.62993254)
\curveto(308.71992838,898.44125331)(308.99650517,898.19898062)(309.15516786,897.90311374)
\curveto(309.31810608,897.61152294)(309.39957831,897.07337741)(309.3995848,896.28867553)
\lineto(309.38028873,894.1789714)
\curveto(309.38028226,893.57864745)(309.40815434,893.13483818)(309.46390505,892.84754228)
\curveto(309.52393067,892.56453256)(309.63327498,892.26008369)(309.79193831,891.93419476)
\lineto(308.00383543,891.93419476)
\curveto(307.9566622,892.0542591)(307.89877404,892.23221161)(307.83017076,892.46805282)
\curveto(307.80014976,892.57525259)(307.77870969,892.64600479)(307.76585051,892.68030964)
\curveto(307.45710878,892.38014803)(307.12693184,892.15502739)(306.7753187,892.00494704)
\curveto(306.42369783,891.85486654)(306.04849676,891.77982632)(305.64971437,891.77982617)
\curveto(304.94647762,891.77982632)(304.39118004,891.97064287)(303.98381995,892.35227637)
\curveto(303.58074573,892.73390904)(303.37920916,893.21631042)(303.37920963,893.79948194)
\curveto(303.37920916,894.18540118)(303.47140142,894.52844216)(303.65578669,894.82860591)
\curveto(303.84017047,895.13305188)(304.0974512,895.36460454)(304.42762966,895.52326458)
\curveto(304.7620931,895.68620546)(305.24235047,895.82770986)(305.86840321,895.94777821)
\curveto(306.71313866,896.10643065)(307.29845233,896.25436707)(307.62434597,896.39158792)
\lineto(307.62434597,896.57168462)
\curveto(307.62434125,896.91900897)(307.53858101,897.16556967)(307.36706498,897.31136746)
\curveto(307.19554003,897.46144251)(306.87179511,897.53648273)(306.39582924,897.53648833)
\curveto(306.07422484,897.53648273)(305.82337612,897.47216254)(305.64328234,897.34352759)
\curveto(305.4631831,897.21916982)(305.31739068,896.99833719)(305.20590466,896.68102904)
\moveto(307.62434597,895.21452739)
\curveto(307.39278859,895.13733989)(307.02616355,895.04514763)(306.52446974,894.93795033)
\curveto(306.02276869,894.83074702)(305.69473576,894.72569072)(305.54036995,894.62278111)
\curveto(305.30452664,894.45554595)(305.18660631,894.24328934)(305.18660858,893.98601066)
\curveto(305.18660631,893.73301589)(305.28094258,893.51432727)(305.46961767,893.32994414)
\curveto(305.65828765,893.14555821)(305.89841634,893.05336595)(306.19000445,893.05336707)
\curveto(306.5158901,893.05336595)(306.82677098,893.16056626)(307.12264804,893.37496831)
\curveto(307.34133245,893.53791133)(307.48498086,893.7373039)(307.5535937,893.97314661)
\curveto(307.60075719,894.12751301)(307.62434125,894.42124185)(307.62434597,894.854334)
\lineto(307.62434597,895.21452739)
}
}
{
\newrgbcolor{curcolor}{0 0 0}
\pscustom[linestyle=none,fillstyle=solid,fillcolor=curcolor]
{
\newpath
\moveto(317.14374248,896.74534928)
\lineto(315.36207162,896.42374805)
\curveto(315.30203433,896.77964857)(315.16481793,897.04764934)(314.95042203,897.22775114)
\curveto(314.74030472,897.40784236)(314.46587194,897.49789062)(314.12712286,897.49789618)
\curveto(313.67687769,897.49789062)(313.31668467,897.34137817)(313.0465427,897.02835837)
\curveto(312.78068314,896.7196164)(312.64775476,896.20076692)(312.64775717,895.47180838)
\curveto(312.64775476,894.66137053)(312.78282715,894.0889209)(313.05297473,893.75445777)
\curveto(313.3274047,893.419991)(313.69402974,893.25275852)(314.15285096,893.25275984)
\curveto(314.49588803,893.25275852)(314.77675283,893.3492388)(314.99544621,893.54220095)
\curveto(315.21413008,893.73944791)(315.36849852,894.07605687)(315.45855199,894.55202884)
\lineto(317.23379082,894.24972368)
\curveto(317.04939931,893.43499904)(316.6956383,892.81966929)(316.17250674,892.40373257)
\curveto(315.64936332,891.98779492)(314.94827332,891.77982632)(314.06923464,891.77982617)
\curveto(313.07012397,891.77982632)(312.27255369,892.09499522)(311.67652143,892.72533381)
\curveto(311.08477431,893.35567081)(310.78890146,894.2282813)(310.78890201,895.34316789)
\curveto(310.78890146,896.47091169)(311.08691831,897.34781019)(311.68295345,897.97386601)
\curveto(312.27898571,898.60419777)(313.08513201,898.91936667)(314.10139476,898.91937365)
\curveto(314.93326527,898.91936667)(315.59361916,898.73927016)(316.08245839,898.37908357)
\curveto(316.57557395,898.02317212)(316.92933496,897.47859456)(317.14374248,896.74534928)
}
}
{
\newrgbcolor{curcolor}{0 0 0}
\pscustom[linestyle=none,fillstyle=solid,fillcolor=curcolor]
{
\newpath
\moveto(318.45587539,891.93419476)
\lineto(318.45587539,901.36354306)
\lineto(320.26327434,901.36354306)
\lineto(320.26327434,896.3594278)
\lineto(322.37941049,898.76500506)
\lineto(324.60489106,898.76500506)
\lineto(322.27006607,896.26937945)
\lineto(324.7721237,891.93419476)
\lineto(322.8232202,891.93419476)
\lineto(321.10586959,895.00227057)
\lineto(320.26327434,894.12108318)
\lineto(320.26327434,891.93419476)
\lineto(318.45587539,891.93419476)
}
}
{
\newrgbcolor{curcolor}{0 0 0}
\pscustom[linestyle=none,fillstyle=solid,fillcolor=curcolor]
{
\newpath
\moveto(327.20342892,896.68102904)
\lineto(325.56326261,896.97690218)
\curveto(325.74764648,897.63725102)(326.06495938,898.12608441)(326.51520227,898.44340382)
\curveto(326.96544195,898.76071022)(327.63437185,898.91936667)(328.521994,898.91937365)
\curveto(329.32813668,898.91936667)(329.92845839,898.82288639)(330.32296093,898.62993254)
\curveto(330.71745264,898.44125331)(330.99402943,898.19898062)(331.15269212,897.90311374)
\curveto(331.31563034,897.61152294)(331.39710258,897.07337741)(331.39710907,896.28867553)
\lineto(331.37781299,894.1789714)
\curveto(331.37780652,893.57864745)(331.4056786,893.13483818)(331.46142931,892.84754228)
\curveto(331.52145493,892.56453256)(331.63079924,892.26008369)(331.78946258,891.93419476)
\lineto(330.00135969,891.93419476)
\curveto(329.95418646,892.0542591)(329.8962983,892.23221161)(329.82769502,892.46805282)
\curveto(329.79767402,892.57525259)(329.77623396,892.64600479)(329.76337478,892.68030964)
\curveto(329.45463304,892.38014803)(329.1244561,892.15502739)(328.77284296,892.00494704)
\curveto(328.42122209,891.85486654)(328.04602103,891.77982632)(327.64723863,891.77982617)
\curveto(326.94400188,891.77982632)(326.3887043,891.97064287)(325.98134422,892.35227637)
\curveto(325.57826999,892.73390904)(325.37673342,893.21631042)(325.37673389,893.79948194)
\curveto(325.37673342,894.18540118)(325.46892568,894.52844216)(325.65331095,894.82860591)
\curveto(325.83769473,895.13305188)(326.09497547,895.36460454)(326.42515392,895.52326458)
\curveto(326.75961736,895.68620546)(327.23987473,895.82770986)(327.86592747,895.94777821)
\curveto(328.71066292,896.10643065)(329.29597659,896.25436707)(329.62187023,896.39158792)
\lineto(329.62187023,896.57168462)
\curveto(329.62186552,896.91900897)(329.53610527,897.16556967)(329.36458924,897.31136746)
\curveto(329.19306429,897.46144251)(328.86931937,897.53648273)(328.3933535,897.53648833)
\curveto(328.0717491,897.53648273)(327.82090038,897.47216254)(327.6408066,897.34352759)
\curveto(327.46070736,897.21916982)(327.31491494,896.99833719)(327.20342892,896.68102904)
\moveto(329.62187023,895.21452739)
\curveto(329.39031286,895.13733989)(329.02368781,895.04514763)(328.521994,894.93795033)
\curveto(328.02029295,894.83074702)(327.69226002,894.72569072)(327.53789421,894.62278111)
\curveto(327.30205091,894.45554595)(327.18413057,894.24328934)(327.18413285,893.98601066)
\curveto(327.18413057,893.73301589)(327.27846684,893.51432727)(327.46714194,893.32994414)
\curveto(327.65581191,893.14555821)(327.8959406,893.05336595)(328.18752871,893.05336707)
\curveto(328.51341436,893.05336595)(328.82429524,893.16056626)(329.1201723,893.37496831)
\curveto(329.33885671,893.53791133)(329.48250512,893.7373039)(329.55111796,893.97314661)
\curveto(329.59828145,894.12751301)(329.62186552,894.42124185)(329.62187023,894.854334)
\lineto(329.62187023,895.21452739)
}
}
{
\newrgbcolor{curcolor}{0 0 0}
\pscustom[linestyle=none,fillstyle=solid,fillcolor=curcolor]
{
\newpath
\moveto(333.01797916,891.48395303)
\lineto(335.08265911,891.23310406)
\curveto(335.11696037,890.99297608)(335.19628859,890.82788761)(335.32064403,890.73783816)
\curveto(335.49216144,890.60919899)(335.76230621,890.5448788)(336.13107915,890.54487741)
\curveto(336.6027566,890.5448788)(336.95651761,890.61563101)(337.19236323,890.75713423)
\curveto(337.35101473,890.85147168)(337.47107908,891.00369611)(337.55255662,891.21380799)
\curveto(337.60829547,891.36388914)(337.63616755,891.64046593)(337.63617294,892.04353918)
\lineto(337.63617294,893.04050302)
\curveto(337.09587801,892.30296381)(336.41408406,891.93419476)(335.59078907,891.93419476)
\curveto(334.6731511,891.93419476)(333.94633303,892.32225987)(333.41033267,893.09839124)
\curveto(332.99010631,893.71157583)(332.77999371,894.474842)(332.77999425,895.38819206)
\curveto(332.77999371,896.53308787)(333.05442649,897.40784236)(333.60329342,898.01245816)
\curveto(334.15644563,898.61706181)(334.84252759,898.91936667)(335.66154134,898.91937365)
\curveto(336.50627633,898.91936667)(337.20307831,898.54845361)(337.75194939,897.80663337)
\lineto(337.75194939,898.76500506)
\lineto(339.4435719,898.76500506)
\lineto(339.4435719,892.63528546)
\curveto(339.4435647,891.82913846)(339.37710051,891.22667275)(339.24417913,890.8278865)
\curveto(339.11124375,890.42910247)(338.92471522,890.11607758)(338.68459298,889.88881089)
\curveto(338.44445785,889.66154829)(338.12285693,889.48359578)(337.71978926,889.35495283)
\curveto(337.32099865,889.22631505)(336.81501321,889.16199486)(336.20183142,889.16199209)
\curveto(335.04406416,889.16199486)(334.22290982,889.36138743)(333.73836594,889.76017039)
\curveto(333.25381906,890.15466969)(333.01154637,890.65636712)(333.01154714,891.26526419)
\curveto(333.01154637,891.32529703)(333.01369037,891.39819324)(333.01797916,891.48395303)
\moveto(334.63241738,895.49110446)
\curveto(334.63241499,894.76642683)(334.77177538,894.23471332)(335.05049899,893.89596232)
\curveto(335.33350498,893.5614954)(335.68083397,893.39426292)(336.092487,893.39426438)
\curveto(336.53414841,893.39426292)(336.90720547,893.56578341)(337.21165931,893.90882636)
\curveto(337.5161032,894.25615338)(337.66832764,894.76857084)(337.66833307,895.44608028)
\curveto(337.66832764,896.15359879)(337.52253522,896.67888028)(337.23095538,897.02192635)
\curveto(336.93936556,897.36496224)(336.57059651,897.53648273)(336.12464712,897.53648833)
\curveto(335.691554,897.53648273)(335.33350498,897.36710625)(335.05049899,897.02835837)
\curveto(334.77177538,896.69388833)(334.63241499,896.18147087)(334.63241738,895.49110446)
}
}
{
\newrgbcolor{curcolor}{0 0 0}
\pscustom[linestyle=none,fillstyle=solid,fillcolor=curcolor]
{
\newpath
\moveto(345.19380163,894.10821913)
\lineto(346.99476856,893.80591397)
\curveto(346.7632092,893.14555821)(346.39658415,892.64171678)(345.89489233,892.29438815)
\curveto(345.39747731,891.95134681)(344.77357153,891.77982632)(344.02317312,891.77982617)
\curveto(342.83539001,891.77982632)(341.9563475,892.16789143)(341.38604297,892.94402265)
\curveto(340.93580059,893.56578341)(340.71067995,894.35048965)(340.71068037,895.29814371)
\curveto(340.71067995,896.43017558)(341.00655279,897.3156501)(341.59829979,897.95456994)
\curveto(342.19004417,898.59776575)(342.9383023,898.91936667)(343.84307643,898.91937365)
\curveto(344.85933177,898.91936667)(345.66119006,898.58275771)(346.24865369,897.90954577)
\curveto(346.83610541,897.24060988)(347.11697021,896.21363096)(347.09124893,894.82860591)
\lineto(342.5631035,894.82860591)
\curveto(342.57596527,894.29260148)(342.72175768,893.87452029)(343.00048118,893.57436108)
\curveto(343.27919927,893.27848659)(343.62652826,893.13055017)(344.0424692,893.13055137)
\curveto(344.32547425,893.13055017)(344.56345893,893.20773439)(344.75642394,893.36210426)
\curveto(344.94938003,893.51647127)(345.09517245,893.76517598)(345.19380163,894.10821913)
\moveto(345.29671402,895.93491416)
\curveto(345.28384498,896.45804765)(345.1487726,896.85468879)(344.89149646,897.12483875)
\curveto(344.63421113,897.39926634)(344.32118624,897.53648273)(343.95242085,897.53648833)
\curveto(343.55792007,897.53648273)(343.23203114,897.39283432)(342.97475309,897.10554267)
\curveto(342.71746967,896.81824068)(342.59097331,896.42803157)(342.59526362,895.93491416)
\lineto(345.29671402,895.93491416)
}
}
{
\newrgbcolor{curcolor}{0 0 0}
\pscustom[linestyle=none,fillstyle=solid,fillcolor=curcolor]
{
\newpath
\moveto(347.93384404,893.88309827)
\lineto(349.74767502,894.15967533)
\curveto(349.82485712,893.8080561)(349.98136956,893.54005534)(350.21721283,893.35567224)
\curveto(350.45305091,893.1755743)(350.78322785,893.08552604)(351.20774464,893.0855272)
\curveto(351.67513439,893.08552604)(352.02675139,893.17128629)(352.2625967,893.34280819)
\curveto(352.42124852,893.46287112)(352.50057674,893.62367158)(352.50058162,893.82521004)
\curveto(352.50057674,893.96242454)(352.45769662,894.07605687)(352.37194112,894.16610736)
\curveto(352.28188812,894.25186537)(352.08035154,894.33119359)(351.76733079,894.40409227)
\curveto(350.3094025,894.72569072)(349.38533586,895.01941955)(348.99512812,895.28527966)
\curveto(348.45483721,895.65404536)(348.18469244,896.16646282)(348.184693,896.82253358)
\curveto(348.18469244,897.41427438)(348.41838911,897.9116838)(348.8857837,898.31476333)
\curveto(349.35317577,898.71783009)(350.07784984,898.91936667)(351.05980807,898.91937365)
\curveto(351.9945913,898.91936667)(352.68924928,898.76714224)(353.14378409,898.4626999)
\curveto(353.59830787,898.1582445)(353.91133276,897.70800322)(354.08285971,897.1119747)
\lineto(352.37837315,896.79680548)
\curveto(352.30547219,897.06265738)(352.16611179,897.26633796)(351.96029154,897.40784784)
\curveto(351.75875063,897.54934676)(351.4693098,897.62009897)(351.09196819,897.62010465)
\curveto(350.61599537,897.62009897)(350.2750984,897.55363478)(350.06927626,897.42071188)
\curveto(349.93205742,897.32637013)(349.86344923,897.20416178)(349.86345146,897.05408647)
\curveto(349.86344923,896.92544099)(349.9234814,896.81609668)(350.04354816,896.72605321)
\curveto(350.2064902,896.60598408)(350.7682198,896.43660759)(351.72873865,896.21792325)
\curveto(352.69353729,895.99923035)(353.36675521,895.73122958)(353.74839442,895.41392016)
\curveto(354.12573337,895.09231576)(354.31440591,894.64421849)(354.3144126,894.06962698)
\curveto(354.31440591,893.44357506)(354.05283716,892.90542953)(353.52970558,892.45518877)
\curveto(353.00656218,892.00494696)(352.23257598,891.77982632)(351.20774464,891.77982617)
\curveto(350.2772424,891.77982632)(349.5397043,891.96849886)(348.99512812,892.34584435)
\curveto(348.45483721,892.72318901)(348.1010762,893.23560647)(347.93384404,893.88309827)
}
}
{
\newrgbcolor{curcolor}{1 0.50196081 0.50196081}
\pscustom[linewidth=2.63455725,linecolor=curcolor]
{
\newpath
\moveto(160.17532,881.72413456)
\lineto(244.48116,881.72413456)
\lineto(244.48116,934.41527456)
\lineto(187.83818,934.41527456)
\lineto(187.83818,944.95350456)
\lineto(160.17532,944.95350456)
\lineto(160.17532,881.72413456)
\closepath
}
}
{
\newrgbcolor{curcolor}{0 0 0}
\pscustom[linestyle=none,fillstyle=solid,fillcolor=curcolor]
{
\newpath
\moveto(172.75123079,915.81636199)
\lineto(172.75123079,916.89694215)
\lineto(178.99672683,919.5340723)
\lineto(178.99672683,918.38273987)
\lineto(174.04406776,916.35022004)
\lineto(178.99672683,914.29840414)
\lineto(178.99672683,913.14707171)
\lineto(172.75123079,915.81636199)
}
}
{
\newrgbcolor{curcolor}{0 0 0}
\pscustom[linestyle=none,fillstyle=solid,fillcolor=curcolor]
{
\newpath
\moveto(180.4439326,915.81636199)
\lineto(180.4439326,916.89694215)
\lineto(186.68942865,919.5340723)
\lineto(186.68942865,918.38273987)
\lineto(181.73676958,916.35022004)
\lineto(186.68942865,914.29840414)
\lineto(186.68942865,913.14707171)
\lineto(180.4439326,915.81636199)
}
}
{
\newrgbcolor{curcolor}{0 0 0}
\pscustom[linestyle=none,fillstyle=solid,fillcolor=curcolor]
{
\newpath
\moveto(189.3522871,911.69343411)
\lineto(188.27813896,911.69343411)
\lineto(188.27813896,921.12278241)
\lineto(189.43590342,921.12278241)
\lineto(189.43590342,917.75883346)
\curveto(189.9247348,918.37201315)(190.54864057,918.67860602)(191.30762263,918.678613)
\curveto(191.72784393,918.67860602)(192.12448506,918.59284578)(192.49754721,918.42133201)
\curveto(192.8748872,918.25409281)(193.18362408,918.01610813)(193.42375877,917.70737727)
\curveto(193.66816946,917.40292238)(193.85898601,917.03415333)(193.99620898,916.60106901)
\curveto(194.13341879,916.16797486)(194.20202698,915.70486955)(194.20203377,915.21175166)
\curveto(194.20202698,914.0411208)(193.91258616,913.13635023)(193.33371043,912.49743721)
\curveto(192.75482286,911.85852258)(192.06016488,911.53906567)(191.2497344,911.53906552)
\curveto(190.44358427,911.53906567)(189.81110247,911.87567463)(189.3522871,912.54889341)
\lineto(189.3522871,911.69343411)
\moveto(189.33942305,915.16029546)
\curveto(189.33942113,914.34128166)(189.45090945,913.74953597)(189.67388834,913.38505663)
\curveto(190.03836712,912.78902124)(190.53148852,912.49100439)(191.15325403,912.49100518)
\curveto(191.65923574,912.49100439)(192.09661298,912.70969301)(192.46538708,913.14707171)
\curveto(192.83415109,913.58873551)(193.01853561,914.24480138)(193.01854121,915.11527129)
\curveto(193.01853561,916.00717441)(192.84058311,916.66538428)(192.48468316,917.08990289)
\curveto(192.13306109,917.5144107)(191.70640387,917.72666731)(191.20471023,917.72667334)
\curveto(190.698721,917.72666731)(190.26134375,917.50583468)(189.89257718,917.06417479)
\curveto(189.52380565,916.62679217)(189.33942113,915.99216636)(189.33942305,915.16029546)
}
}
{
\newrgbcolor{curcolor}{0 0 0}
\pscustom[linestyle=none,fillstyle=solid,fillcolor=curcolor]
{
\newpath
\moveto(200.07447128,912.53602936)
\curveto(199.64566474,912.17154748)(199.23187156,911.91426674)(198.8330905,911.76418639)
\curveto(198.4385893,911.61410589)(198.01407609,911.53906567)(197.5595496,911.53906552)
\curveto(196.80914465,911.53906567)(196.23240701,911.72130619)(195.82933494,912.08578762)
\curveto(195.42626071,912.45455628)(195.22472414,912.92409362)(195.22472461,913.49440105)
\curveto(195.22472414,913.8288642)(195.29976435,914.13331307)(195.44984548,914.40774856)
\curveto(195.60421322,914.68646664)(195.80360579,914.90944328)(196.04802378,915.07667914)
\curveto(196.29672719,915.24390823)(196.57544799,915.37040459)(196.884187,915.4561686)
\curveto(197.11144951,915.51619701)(197.45449049,915.57408517)(197.91331096,915.62983327)
\curveto(198.84809446,915.74131765)(199.53632042,915.87424603)(199.97799091,916.0286188)
\curveto(199.98227369,916.18727092)(199.9844177,916.28803921)(199.98442294,916.33092397)
\curveto(199.9844177,916.80260067)(199.87507339,917.13492162)(199.65638967,917.3278878)
\curveto(199.36051192,917.58945092)(198.92099067,917.72023529)(198.3378246,917.72024132)
\curveto(197.79324346,917.72023529)(197.39017031,917.62375501)(197.12860394,917.4308002)
\curveto(196.87132083,917.24212193)(196.68050429,916.90551297)(196.55615374,916.42097231)
\lineto(195.42411738,916.57534091)
\curveto(195.527029,917.05988141)(195.69640548,917.45009052)(195.93224734,917.74596941)
\curveto(196.16808683,918.04612422)(196.5089838,918.27553287)(196.95493927,918.43419606)
\curveto(197.40089034,918.59713379)(197.91759581,918.67860602)(198.50505724,918.678613)
\curveto(199.08822315,918.67860602)(199.5620485,918.60999782)(199.92653471,918.47278821)
\curveto(200.29101057,918.33556504)(200.55901134,918.16190055)(200.73053781,917.95179421)
\curveto(200.90205232,917.74596336)(201.02211666,917.48439462)(201.0907312,917.16708719)
\curveto(201.12931696,916.96983315)(201.14861302,916.61392814)(201.14861942,916.09937108)
\lineto(201.14861942,914.55568513)
\curveto(201.14861302,913.4793912)(201.17219709,912.79759726)(201.21937169,912.51030126)
\curveto(201.27082137,912.22729163)(201.36944565,911.95500286)(201.51524483,911.69343411)
\lineto(200.30602417,911.69343411)
\curveto(200.18595428,911.9335628)(200.10877006,912.2144276)(200.07447128,912.53602936)
\moveto(199.97799091,915.12170331)
\curveto(199.55776049,914.95017939)(198.92742269,914.80438698)(198.08697563,914.68432563)
\curveto(197.61100294,914.61571444)(197.27439398,914.53853022)(197.07714774,914.45277274)
\curveto(196.87989685,914.36700973)(196.72767242,914.24051337)(196.62047399,914.07328328)
\curveto(196.51327181,913.91033643)(196.45967166,913.72809591)(196.45967337,913.52656117)
\curveto(196.45967166,913.21782246)(196.57544799,912.96054172)(196.8070027,912.7547182)
\curveto(197.04284132,912.54889255)(197.3858823,912.44598026)(197.83612667,912.44598101)
\curveto(198.28207685,912.44598026)(198.67871798,912.54246053)(199.02605125,912.73542212)
\curveto(199.37337596,912.93266964)(199.62851269,913.20067041)(199.79146219,913.53942522)
\curveto(199.91580951,913.80099212)(199.97798568,914.18691322)(199.97799091,914.69718968)
\lineto(199.97799091,915.12170331)
}
}
{
\newrgbcolor{curcolor}{0 0 0}
\pscustom[linestyle=none,fillstyle=solid,fillcolor=curcolor]
{
\newpath
\moveto(202.48648043,913.73238596)
\lineto(203.63138084,913.91248266)
\curveto(203.69569947,913.45366313)(203.87365198,913.10204613)(204.16523889,912.8576306)
\curveto(204.46110965,912.61321273)(204.87275882,912.49100439)(205.40018764,912.49100518)
\curveto(205.93189784,912.49100439)(206.32639497,912.59820469)(206.5836802,912.81260642)
\curveto(206.84095643,913.03129393)(206.9695968,913.28643065)(206.96960169,913.57801737)
\curveto(206.9695968,913.83958423)(206.85596447,914.04540882)(206.62870437,914.19549175)
\curveto(206.47004337,914.29840154)(206.07554625,914.42918591)(205.44521182,914.58784526)
\curveto(204.59618203,914.80224297)(204.00658035,914.9866275)(203.67640501,915.14099939)
\curveto(203.35051448,915.29965239)(203.10180978,915.51619701)(202.93029014,915.79063389)
\curveto(202.76305681,916.06935058)(202.67944057,916.37594346)(202.67944117,916.71041343)
\curveto(202.67944057,917.01485728)(202.74804877,917.29572208)(202.88526596,917.55300867)
\curveto(203.02676956,917.81457156)(203.21758611,918.03111617)(203.45771617,918.20264317)
\curveto(203.6378113,918.33556504)(203.882228,918.44705336)(204.19096699,918.53710846)
\curveto(204.50398977,918.63143789)(204.83845472,918.67860602)(205.19436285,918.678613)
\curveto(205.73036127,918.67860602)(206.1998986,918.6014218)(206.60297628,918.44706011)
\curveto(207.01033291,918.29268492)(207.31049377,918.08257232)(207.50345974,917.81672169)
\curveto(207.69641487,917.55514682)(207.82934325,917.20352982)(207.90224528,916.76186963)
\lineto(206.77020892,916.60750103)
\curveto(206.71874808,916.95911312)(206.56866766,917.2335459)(206.31996719,917.4308002)
\curveto(206.07554625,917.62804303)(205.72821726,917.72666731)(205.27797917,917.72667334)
\curveto(204.74626246,917.72666731)(204.36677338,917.63876306)(204.13951079,917.46296032)
\curveto(203.91224409,917.28714605)(203.79861176,917.08132147)(203.79861348,916.84548595)
\curveto(203.79861176,916.69540037)(203.8457799,916.56032798)(203.94011802,916.44026839)
\curveto(204.03445243,916.31591129)(204.18238886,916.21299899)(204.38392773,916.1315312)
\curveto(204.49970176,916.08864664)(204.84059873,915.99002236)(205.40661967,915.83565806)
\curveto(206.22562668,915.61696529)(206.7959323,915.43686878)(207.11753826,915.29536798)
\curveto(207.44342215,915.15814799)(207.69855888,914.95661141)(207.8829492,914.69075765)
\curveto(208.06732793,914.4248979)(208.15952019,914.09472096)(208.15952627,913.70022584)
\curveto(208.15952019,913.31430273)(208.04588787,912.94982169)(207.81862896,912.60678163)
\curveto(207.59564658,912.26802775)(207.27190166,912.004315)(206.84739322,911.81564258)
\curveto(206.42287524,911.63125794)(205.94261787,911.53906567)(205.40661967,911.53906552)
\curveto(204.51899781,911.53906567)(203.84149188,911.7234502)(203.37409984,912.09221965)
\curveto(202.91099323,912.4609883)(202.61512039,913.00770986)(202.48648043,913.73238596)
}
}
{
\newrgbcolor{curcolor}{0 0 0}
\pscustom[linestyle=none,fillstyle=solid,fillcolor=curcolor]
{
\newpath
\moveto(214.21206348,913.89318658)
\lineto(215.40842008,913.74525001)
\curveto(215.2197408,913.04630197)(214.87026781,912.50386842)(214.36000005,912.11794775)
\curveto(213.8497209,911.73202622)(213.19794304,911.53906567)(212.40466452,911.53906552)
\curveto(211.40555393,911.53906567)(210.61227167,911.84565855)(210.02481536,912.45884506)
\curveto(209.44164434,913.07631805)(209.15005951,913.94035252)(209.15005999,915.05095104)
\curveto(209.15005951,916.20013496)(209.44593235,917.0920415)(210.03767941,917.72667334)
\curveto(210.62942372,918.36129312)(211.39697791,918.67860602)(212.34034427,918.678613)
\curveto(213.2536872,918.67860602)(213.99980133,918.36772513)(214.57868889,917.74596941)
\curveto(215.15756463,917.12420159)(215.44700545,916.2494471)(215.44701223,915.12170331)
\curveto(215.44700545,915.05309169)(215.44486144,914.95017939)(215.44058021,914.81296612)
\lineto(210.34641659,914.81296612)
\curveto(210.38929504,914.06256086)(210.60155164,913.48796723)(210.98318705,913.08918349)
\curveto(211.36481782,912.69039695)(211.84078717,912.49100439)(212.41109654,912.49100518)
\curveto(212.83560601,912.49100439)(213.19794304,912.6024927)(213.49810873,912.82547047)
\curveto(213.79826475,913.04844597)(214.03624943,913.40435099)(214.21206348,913.89318658)
\moveto(210.41073684,915.76490579)
\lineto(214.22492753,915.76490579)
\curveto(214.17346582,916.33949535)(214.02767341,916.77044058)(213.78754984,917.05774276)
\curveto(213.41877567,917.50369067)(212.94066231,917.72666731)(212.35320832,917.72667334)
\curveto(211.82149112,917.72666731)(211.37339384,917.5487148)(211.00891514,917.19281528)
\curveto(210.64871978,916.83690477)(210.44932721,916.36093541)(210.41073684,915.76490579)
}
}
{
\newrgbcolor{curcolor}{0 0 0}
\pscustom[linestyle=none,fillstyle=solid,fillcolor=curcolor]
{
\newpath
\moveto(222.96604904,915.81636199)
\lineto(216.72055299,913.14707171)
\lineto(216.72055299,914.29840414)
\lineto(221.66678003,916.35022004)
\lineto(216.72055299,918.38273987)
\lineto(216.72055299,919.5340723)
\lineto(222.96604904,916.89694215)
\lineto(222.96604904,915.81636199)
}
}
{
\newrgbcolor{curcolor}{0 0 0}
\pscustom[linestyle=none,fillstyle=solid,fillcolor=curcolor]
{
\newpath
\moveto(230.65874847,915.81636199)
\lineto(224.41325242,913.14707171)
\lineto(224.41325242,914.29840414)
\lineto(229.35947947,916.35022004)
\lineto(224.41325242,918.38273987)
\lineto(224.41325242,919.5340723)
\lineto(230.65874847,916.89694215)
\lineto(230.65874847,915.81636199)
}
}
{
\newrgbcolor{curcolor}{0 0 0}
\pscustom[linestyle=none,fillstyle=solid,fillcolor=curcolor]
{
\newpath
\moveto(170.29034345,898.76507388)
\lineto(171.97553394,898.76507388)
\lineto(171.97553394,897.76167802)
\curveto(172.19421999,898.10471317)(172.49009283,898.38343396)(172.86315336,898.59784123)
\curveto(173.23620695,898.81223518)(173.65000013,898.91943549)(174.10453414,898.91944247)
\curveto(174.89781169,898.91943549)(175.57102961,898.6085546)(176.12418991,897.98679888)
\curveto(176.67733676,897.36503106)(176.95391355,896.49885259)(176.95392111,895.38826088)
\curveto(176.95391355,894.24764617)(176.67519276,893.36002764)(176.11775789,892.72540263)
\curveto(175.56030958,892.09506404)(174.88494765,891.77989514)(174.09167009,891.77989499)
\curveto(173.71432032,891.77989514)(173.37127934,891.85493535)(173.06254613,892.00501585)
\curveto(172.75809359,892.15509621)(172.43649268,892.41237694)(172.09774241,892.77685882)
\lineto(172.09774241,889.33572558)
\lineto(170.29034345,889.33572558)
\lineto(170.29034345,898.76507388)
\moveto(172.07844634,895.46544518)
\curveto(172.07844366,894.69788746)(172.23066809,894.12972584)(172.5351201,893.76095861)
\curveto(172.83956582,893.39647575)(173.21047888,893.21423523)(173.64786038,893.21423651)
\curveto(174.06808133,893.21423523)(174.41755432,893.38146771)(174.69628042,893.71593444)
\curveto(174.97499591,894.05468562)(175.11435631,894.6078392)(175.11436203,895.37539683)
\curveto(175.11435631,896.09149143)(174.9707079,896.62320494)(174.68341637,896.97053897)
\curveto(174.39611426,897.31786292)(174.04020925,897.49152742)(173.61570026,897.49153298)
\curveto(173.17403078,897.49152742)(172.80740573,897.32000693)(172.51582402,896.97697099)
\curveto(172.22423607,896.63821299)(172.07844366,896.13437155)(172.07844634,895.46544518)
}
}
{
\newrgbcolor{curcolor}{0 0 0}
\pscustom[linestyle=none,fillstyle=solid,fillcolor=curcolor]
{
\newpath
\moveto(180.12490891,891.93426358)
\lineto(178.31750995,891.93426358)
\lineto(178.31750995,898.76507388)
\lineto(179.99626841,898.76507388)
\lineto(179.99626841,897.79383814)
\curveto(180.28356268,898.25264959)(180.54084341,898.55495445)(180.76811138,898.70075363)
\curveto(180.99966072,898.84653928)(181.26122947,898.91943549)(181.5528184,898.91944247)
\curveto(181.96446347,898.91943549)(182.3611046,898.80580316)(182.74274298,898.57854516)
\lineto(182.18315683,897.00269909)
\curveto(181.87870323,897.19994259)(181.59569442,897.29856687)(181.33412956,897.29857223)
\curveto(181.08113295,897.29856687)(180.86673234,897.22781467)(180.69092708,897.08631542)
\curveto(180.51511534,896.94909387)(180.37575494,896.69824516)(180.27284547,896.33376852)
\curveto(180.17421837,895.96928308)(180.12490623,895.2060169)(180.12490891,894.0439677)
\lineto(180.12490891,891.93426358)
}
}
{
\newrgbcolor{curcolor}{0 0 0}
\pscustom[linestyle=none,fillstyle=solid,fillcolor=curcolor]
{
\newpath
\moveto(183.51458636,899.69128544)
\lineto(183.51458636,901.36361188)
\lineto(185.32198532,901.36361188)
\lineto(185.32198532,899.69128544)
\lineto(183.51458636,899.69128544)
\moveto(183.51458636,891.93426358)
\lineto(183.51458636,898.76507388)
\lineto(185.32198532,898.76507388)
\lineto(185.32198532,891.93426358)
\lineto(183.51458636,891.93426358)
}
}
{
\newrgbcolor{curcolor}{0 0 0}
\pscustom[linestyle=none,fillstyle=solid,fillcolor=curcolor]
{
\newpath
\moveto(189.04612657,891.93426358)
\lineto(186.29321997,898.76507388)
\lineto(188.19066728,898.76507388)
\lineto(189.47707223,895.27891646)
\lineto(189.85012967,894.11471998)
\curveto(189.94875032,894.41059064)(190.0109265,894.60569519)(190.03665839,894.70003423)
\curveto(190.09668674,894.89299201)(190.16100693,895.08595256)(190.22961913,895.27891646)
\lineto(191.52888813,898.76507388)
\lineto(193.38774329,898.76507388)
\lineto(190.67342884,891.93426358)
\lineto(189.04612657,891.93426358)
}
}
{
\newrgbcolor{curcolor}{0 0 0}
\pscustom[linestyle=none,fillstyle=solid,fillcolor=curcolor]
{
\newpath
\moveto(194.52621165,899.69128544)
\lineto(194.52621165,901.36361188)
\lineto(196.33361061,901.36361188)
\lineto(196.33361061,899.69128544)
\lineto(194.52621165,899.69128544)
\moveto(194.52621165,891.93426358)
\lineto(194.52621165,898.76507388)
\lineto(196.33361061,898.76507388)
\lineto(196.33361061,891.93426358)
\lineto(194.52621165,891.93426358)
}
}
{
\newrgbcolor{curcolor}{0 0 0}
\pscustom[linestyle=none,fillstyle=solid,fillcolor=curcolor]
{
\newpath
\moveto(198.17960254,891.93426358)
\lineto(198.17960254,901.36361188)
\lineto(199.9870015,901.36361188)
\lineto(199.9870015,891.93426358)
\lineto(198.17960254,891.93426358)
}
}
{
\newrgbcolor{curcolor}{0 0 0}
\pscustom[linestyle=none,fillstyle=solid,fillcolor=curcolor]
{
\newpath
\moveto(205.78868865,894.10828795)
\lineto(207.58965559,893.80598279)
\curveto(207.35809622,893.14562703)(206.99147118,892.6417856)(206.48977935,892.29445697)
\curveto(205.99236433,891.95141563)(205.36845855,891.77989514)(204.61806015,891.77989499)
\curveto(203.43027703,891.77989514)(202.55123452,892.16796025)(201.98092999,892.94409147)
\curveto(201.53068762,893.56585223)(201.30556697,894.35055847)(201.30556739,895.29821253)
\curveto(201.30556697,896.43024439)(201.60143982,897.31571892)(202.19318681,897.95463876)
\curveto(202.78493119,898.59783457)(203.53318932,898.91943549)(204.43796345,898.91944247)
\curveto(205.4542188,898.91943549)(206.25607708,898.58282653)(206.84354071,897.90961458)
\curveto(207.43099243,897.2406787)(207.71185723,896.21369978)(207.68613596,894.82867472)
\lineto(203.15799052,894.82867472)
\curveto(203.17085229,894.2926703)(203.31664471,893.87458911)(203.59536821,893.5744299)
\curveto(203.87408629,893.27855541)(204.22141528,893.13061899)(204.63735622,893.13062019)
\curveto(204.92036128,893.13061899)(205.15834595,893.20780321)(205.35131097,893.36217308)
\curveto(205.54426705,893.51654009)(205.69005947,893.7652448)(205.78868865,894.10828795)
\moveto(205.89160105,895.93498298)
\curveto(205.87873201,896.45811647)(205.74365962,896.8547576)(205.48638349,897.12490756)
\curveto(205.22909816,897.39933516)(204.91607326,897.53655155)(204.54730787,897.53655715)
\curveto(204.15280709,897.53655155)(203.82691816,897.39290314)(203.56964011,897.10561149)
\curveto(203.31235669,896.8183095)(203.18586033,896.42810039)(203.19015065,895.93498298)
\lineto(205.89160105,895.93498298)
}
}
{
\newrgbcolor{curcolor}{0 0 0}
\pscustom[linestyle=none,fillstyle=solid,fillcolor=curcolor]
{
\newpath
\moveto(208.99826696,891.48402185)
\lineto(211.06294691,891.23317288)
\curveto(211.09724816,890.9930449)(211.17657639,890.82795643)(211.30093182,890.73790698)
\curveto(211.47244923,890.60926781)(211.742594,890.54494762)(212.11136694,890.54494623)
\curveto(212.5830444,890.54494762)(212.9368054,890.61569982)(213.17265103,890.75720305)
\curveto(213.33130253,890.8515405)(213.45136687,891.00376493)(213.53284442,891.21387681)
\curveto(213.58858326,891.36395796)(213.61645534,891.64053474)(213.61646074,892.043608)
\lineto(213.61646074,893.04057184)
\curveto(213.0761658,892.30303263)(212.39437186,891.93426358)(211.57107686,891.93426358)
\curveto(210.6534389,891.93426358)(209.92662083,892.32232869)(209.39062047,893.09846006)
\curveto(208.9703941,893.71164465)(208.7602815,894.47491082)(208.76028204,895.38826088)
\curveto(208.7602815,896.53315669)(209.03471428,897.40791118)(209.58358121,898.01252698)
\curveto(210.13673343,898.61713063)(210.82281538,898.91943549)(211.64182914,898.91944247)
\curveto(212.48656412,898.91943549)(213.18336611,898.54852243)(213.73223718,897.80670219)
\lineto(213.73223718,898.76507388)
\lineto(215.4238597,898.76507388)
\lineto(215.4238597,892.63535428)
\curveto(215.42385249,891.82920728)(215.3573883,891.22674156)(215.22446693,890.82795532)
\curveto(215.09153155,890.42917129)(214.90500301,890.1161464)(214.66488077,889.88887971)
\curveto(214.42474565,889.6616171)(214.10314473,889.4836646)(213.70007706,889.35502165)
\curveto(213.30128644,889.22638386)(212.795301,889.16206368)(212.18211922,889.16206091)
\curveto(211.02435195,889.16206368)(210.20319761,889.36145625)(209.71865373,889.76023921)
\curveto(209.23410685,890.15473851)(208.99183416,890.65643594)(208.99183493,891.26533301)
\curveto(208.99183416,891.32536585)(208.99397817,891.39826205)(208.99826696,891.48402185)
\moveto(210.61270517,895.49117327)
\curveto(210.61270278,894.76649565)(210.75206318,894.23478214)(211.03078678,893.89603113)
\curveto(211.31379278,893.56156422)(211.66112177,893.39433174)(212.0727748,893.3943332)
\curveto(212.5144362,893.39433174)(212.88749326,893.56585223)(213.1919471,893.90889518)
\curveto(213.496391,894.2562222)(213.64861543,894.76863966)(213.64862086,895.4461491)
\curveto(213.64861543,896.15366761)(213.50282302,896.6789491)(213.21124318,897.02199517)
\curveto(212.91965336,897.36503106)(212.55088431,897.53655155)(212.10493492,897.53655715)
\curveto(211.6718418,897.53655155)(211.31379278,897.36717506)(211.03078678,897.02842719)
\curveto(210.75206318,896.69395715)(210.61270278,896.18153968)(210.61270517,895.49117327)
}
}
{
\newrgbcolor{curcolor}{0 0 0}
\pscustom[linestyle=none,fillstyle=solid,fillcolor=curcolor]
{
\newpath
\moveto(221.17408942,894.10828795)
\lineto(222.97505636,893.80598279)
\curveto(222.743497,893.14562703)(222.37687195,892.6417856)(221.87518012,892.29445697)
\curveto(221.3777651,891.95141563)(220.75385933,891.77989514)(220.00346092,891.77989499)
\curveto(218.8156778,891.77989514)(217.9366353,892.16796025)(217.36633076,892.94409147)
\curveto(216.91608839,893.56585223)(216.69096775,894.35055847)(216.69096816,895.29821253)
\curveto(216.69096775,896.43024439)(216.98684059,897.31571892)(217.57858758,897.95463876)
\curveto(218.17033196,898.59783457)(218.91859009,898.91943549)(219.82336422,898.91944247)
\curveto(220.83961957,898.91943549)(221.64147785,898.58282653)(222.22894148,897.90961458)
\curveto(222.8163932,897.2406787)(223.097258,896.21369978)(223.07153673,894.82867472)
\lineto(218.5433913,894.82867472)
\curveto(218.55625306,894.2926703)(218.70204548,893.87458911)(218.98076898,893.5744299)
\curveto(219.25948707,893.27855541)(219.60681606,893.13061899)(220.02275699,893.13062019)
\curveto(220.30576205,893.13061899)(220.54374673,893.20780321)(220.73671174,893.36217308)
\curveto(220.92966783,893.51654009)(221.07546024,893.7652448)(221.17408942,894.10828795)
\moveto(221.27700182,895.93498298)
\curveto(221.26413278,896.45811647)(221.12906039,896.8547576)(220.87178426,897.12490756)
\curveto(220.61449893,897.39933516)(220.30147404,897.53655155)(219.93270864,897.53655715)
\curveto(219.53820786,897.53655155)(219.21231893,897.39290314)(218.95504088,897.10561149)
\curveto(218.69775747,896.8183095)(218.57126111,896.42810039)(218.57555142,895.93498298)
\lineto(221.27700182,895.93498298)
}
}
{
\newrgbcolor{curcolor}{0 0 0}
\pscustom[linestyle=none,fillstyle=solid,fillcolor=curcolor]
{
\newpath
\moveto(223.91413183,893.88316708)
\lineto(225.72796281,894.15974415)
\curveto(225.80514491,893.80812492)(225.96165736,893.54012416)(226.19750062,893.35574105)
\curveto(226.4333387,893.17564312)(226.76351564,893.08559486)(227.18803243,893.08559601)
\curveto(227.65542218,893.08559486)(228.00703919,893.17135511)(228.2428845,893.342877)
\curveto(228.40153631,893.46293994)(228.48086454,893.6237404)(228.48086941,893.82527886)
\curveto(228.48086454,893.96249336)(228.43798441,894.07612568)(228.35222892,894.16617617)
\curveto(228.26217591,894.25193419)(228.06063934,894.33126241)(227.74761859,894.40416109)
\curveto(226.28969029,894.72575954)(225.36562366,895.01948837)(224.97541592,895.28534848)
\curveto(224.43512501,895.65411418)(224.16498024,896.16653164)(224.1649808,896.8226024)
\curveto(224.16498024,897.4143432)(224.3986769,897.91175262)(224.8660715,898.31483214)
\curveto(225.33346357,898.71789891)(226.05813763,898.91943549)(227.04009587,898.91944247)
\curveto(227.97487909,898.91943549)(228.66953707,898.76721105)(229.12407189,898.46276871)
\curveto(229.57859566,898.15831332)(229.89162056,897.70807204)(230.0631475,897.11204351)
\lineto(228.35866094,896.7968743)
\curveto(228.28575998,897.0627262)(228.14639958,897.26640678)(227.94057933,897.40791665)
\curveto(227.73903842,897.54941558)(227.4495976,897.62016778)(227.07225599,897.62017347)
\curveto(226.59628317,897.62016778)(226.25538619,897.5537036)(226.04956405,897.4207807)
\curveto(225.91234522,897.32643895)(225.84373702,897.2042306)(225.84373926,897.05415529)
\curveto(225.84373702,896.92550981)(225.90376919,896.81616549)(226.02383595,896.72612203)
\curveto(226.186778,896.60605289)(226.7485076,896.43667641)(227.70902644,896.21799207)
\curveto(228.67382509,895.99929917)(229.347043,895.7312984)(229.72868222,895.41398898)
\curveto(230.10602117,895.09238458)(230.29469371,894.6442873)(230.29470039,894.0696958)
\curveto(230.29469371,893.44364388)(230.03312496,892.90549835)(229.50999337,892.45525759)
\curveto(228.98684998,892.00501578)(228.21286377,891.77989514)(227.18803243,891.77989499)
\curveto(226.2575302,891.77989514)(225.5199921,891.96856768)(224.97541592,892.34591317)
\curveto(224.43512501,892.72325783)(224.081364,893.23567529)(223.91413183,893.88316708)
}
}
{
\newrgbcolor{curcolor}{1 0.50196081 0.50196081}
\pscustom[linewidth=2.63455725,linecolor=curcolor]
{
\newpath
\moveto(405.82729,881.72407456)
\lineto(490.13312,881.72407456)
\lineto(490.13312,934.41522456)
\lineto(434.80743,934.41522456)
\lineto(434.80743,944.95344456)
\lineto(405.82729,944.95344456)
\lineto(405.82729,881.72407456)
\closepath
}
}
{
\newrgbcolor{curcolor}{0 0 0}
\pscustom[linestyle=none,fillstyle=solid,fillcolor=curcolor]
{
\newpath
\moveto(418.40320101,915.81629977)
\lineto(418.40320101,916.89687993)
\lineto(424.64869706,919.53401008)
\lineto(424.64869706,918.38267765)
\lineto(419.69603799,916.35015782)
\lineto(424.64869706,914.29834193)
\lineto(424.64869706,913.14700949)
\lineto(418.40320101,915.81629977)
}
}
{
\newrgbcolor{curcolor}{0 0 0}
\pscustom[linestyle=none,fillstyle=solid,fillcolor=curcolor]
{
\newpath
\moveto(426.09590283,915.81629977)
\lineto(426.09590283,916.89687993)
\lineto(432.34139887,919.53401008)
\lineto(432.34139887,918.38267765)
\lineto(427.38873981,916.35015782)
\lineto(432.34139887,914.29834193)
\lineto(432.34139887,913.14700949)
\lineto(426.09590283,915.81629977)
}
}
{
\newrgbcolor{curcolor}{0 0 0}
\pscustom[linestyle=none,fillstyle=solid,fillcolor=curcolor]
{
\newpath
\moveto(435.00425733,911.6933719)
\lineto(433.93010919,911.6933719)
\lineto(433.93010919,921.1227202)
\lineto(435.08787365,921.1227202)
\lineto(435.08787365,917.75877125)
\curveto(435.57670502,918.37195093)(436.2006108,918.6785438)(436.95959285,918.67855079)
\curveto(437.37981416,918.6785438)(437.77645529,918.59278356)(438.14951744,918.4212698)
\curveto(438.52685743,918.25403059)(438.83559431,918.01604591)(439.075729,917.70731505)
\curveto(439.32013969,917.40286017)(439.51095623,917.03409112)(439.6481792,916.60100679)
\curveto(439.78538902,916.16791265)(439.85399721,915.70480733)(439.854004,915.21168944)
\curveto(439.85399721,914.04105859)(439.56455639,913.13628801)(438.98568065,912.49737499)
\curveto(438.40679309,911.85846037)(437.71213511,911.53900346)(436.90170463,911.5390033)
\curveto(436.0955545,911.53900346)(435.4630727,911.87561242)(435.00425733,912.54883119)
\lineto(435.00425733,911.6933719)
\moveto(434.99139328,915.16023324)
\curveto(434.99139135,914.34121944)(435.10287967,913.74947376)(435.32585856,913.38499441)
\curveto(435.69033735,912.78895902)(436.18345875,912.49094217)(436.80522426,912.49094297)
\curveto(437.31120597,912.49094217)(437.74858321,912.70963079)(438.11735731,913.14700949)
\curveto(438.48612131,913.5886733)(438.67050584,914.24473917)(438.67051144,915.11520907)
\curveto(438.67050584,916.00711219)(438.49255333,916.66532207)(438.13665339,917.08984067)
\curveto(437.78503132,917.51434849)(437.3583741,917.72660509)(436.85668046,917.72661112)
\curveto(436.35069123,917.72660509)(435.91331398,917.50577246)(435.54454741,917.06411257)
\curveto(435.17577588,916.62672996)(434.99139135,915.99210415)(434.99139328,915.16023324)
}
}
{
\newrgbcolor{curcolor}{0 0 0}
\pscustom[linestyle=none,fillstyle=solid,fillcolor=curcolor]
{
\newpath
\moveto(445.72644151,912.53596714)
\curveto(445.29763496,912.17148526)(444.88384178,911.91420453)(444.48506073,911.76412417)
\curveto(444.09055952,911.61404367)(443.66604631,911.53900346)(443.21151983,911.5390033)
\curveto(442.46111488,911.53900346)(441.88437723,911.72124398)(441.48130517,912.08572541)
\curveto(441.07823094,912.45449407)(440.87669436,912.9240314)(440.87669484,913.49433883)
\curveto(440.87669436,913.82880198)(440.95173458,914.13325085)(441.10181571,914.40768635)
\curveto(441.25618344,914.68640443)(441.45557601,914.90938106)(441.69999401,915.07661692)
\curveto(441.94869742,915.24384601)(442.22741821,915.37034238)(442.53615723,915.45610638)
\curveto(442.76341974,915.51613479)(443.10646072,915.57402296)(443.56528119,915.62977105)
\curveto(444.50006469,915.74125543)(445.18829065,915.87418381)(445.62996114,916.02855659)
\curveto(445.63424392,916.1872087)(445.63638793,916.28797699)(445.63639316,916.33086175)
\curveto(445.63638793,916.80253846)(445.52704362,917.1348594)(445.3083599,917.32782559)
\curveto(445.01248215,917.5893887)(444.5729609,917.72017307)(443.98979482,917.7201791)
\curveto(443.44521368,917.72017307)(443.04214053,917.6236928)(442.78057417,917.43073798)
\curveto(442.52329106,917.24205971)(442.33247451,916.90545075)(442.20812396,916.4209101)
\lineto(441.07608761,916.57527869)
\curveto(441.17899922,917.05981919)(441.34837571,917.4500283)(441.58421756,917.7459072)
\curveto(441.82005705,918.046062)(442.16095402,918.27547065)(442.6069095,918.43413385)
\curveto(443.05286056,918.59707157)(443.56956604,918.6785438)(444.15702747,918.67855079)
\curveto(444.74019337,918.6785438)(445.21401872,918.60993561)(445.57850494,918.472726)
\curveto(445.9429808,918.33550283)(446.21098156,918.16183833)(446.38250804,917.95173199)
\curveto(446.55402254,917.74590115)(446.67408688,917.4843324)(446.74270142,917.16702497)
\curveto(446.78128719,916.96977093)(446.80058325,916.61386592)(446.80058965,916.09930886)
\lineto(446.80058965,914.55562292)
\curveto(446.80058325,913.47932899)(446.82416731,912.79753504)(446.87134192,912.51023904)
\curveto(446.92279159,912.22722942)(447.02141587,911.95494064)(447.16721506,911.6933719)
\lineto(445.9579944,911.6933719)
\curveto(445.8379245,911.93350058)(445.76074028,912.21436538)(445.72644151,912.53596714)
\moveto(445.62996114,915.12164109)
\curveto(445.20973071,914.95011718)(444.57939292,914.80432476)(443.73894586,914.68426341)
\curveto(443.26297316,914.61565222)(442.9263642,914.538468)(442.72911797,914.45271052)
\curveto(442.53186708,914.36694752)(442.37964265,914.24045116)(442.27244421,914.07322106)
\curveto(442.16524204,913.91027421)(442.11164188,913.72803369)(442.11164359,913.52649895)
\curveto(442.11164188,913.21776024)(442.22741821,912.96047951)(442.45897293,912.75465598)
\curveto(442.69481154,912.54883033)(443.03785252,912.44591804)(443.48809689,912.44591879)
\curveto(443.93404708,912.44591804)(444.33068821,912.54239832)(444.67802147,912.73535991)
\curveto(445.02534619,912.93260743)(445.28048291,913.20060819)(445.44343242,913.539363)
\curveto(445.56777973,913.8009299)(445.62995591,914.186851)(445.62996114,914.69712746)
\lineto(445.62996114,915.12164109)
}
}
{
\newrgbcolor{curcolor}{0 0 0}
\pscustom[linestyle=none,fillstyle=solid,fillcolor=curcolor]
{
\newpath
\moveto(448.13845065,913.73232375)
\lineto(449.28335106,913.91242044)
\curveto(449.34766969,913.45360091)(449.5256222,913.10198391)(449.81720912,912.85756838)
\curveto(450.11307988,912.61315052)(450.52472905,912.49094217)(451.05215787,912.49094297)
\curveto(451.58386807,912.49094217)(451.97836519,912.59814247)(452.23565043,912.81254421)
\curveto(452.49292666,913.03123171)(452.62156702,913.28636844)(452.62157191,913.57795515)
\curveto(452.62156702,913.83952201)(452.5079347,914.0453466)(452.2806746,914.19542953)
\curveto(452.1220136,914.29833932)(451.72751648,914.42912369)(451.09718204,914.58778304)
\curveto(450.24815226,914.80218076)(449.65855058,914.98656528)(449.32837523,915.14093717)
\curveto(449.00248471,915.29959017)(448.75378,915.51613479)(448.58226036,915.79057167)
\curveto(448.41502704,916.06928837)(448.3314108,916.37588124)(448.3314114,916.71035121)
\curveto(448.3314108,917.01479506)(448.40001899,917.29565986)(448.53723619,917.55294645)
\curveto(448.67873979,917.81450934)(448.86955633,918.03105396)(449.10968639,918.20258096)
\curveto(449.28978153,918.33550283)(449.53419823,918.44699114)(449.84293722,918.53704624)
\curveto(450.15596,918.63137567)(450.49042495,918.6785438)(450.84633308,918.67855079)
\curveto(451.38233149,918.6785438)(451.85186883,918.60135958)(452.2549465,918.4469979)
\curveto(452.66230314,918.2926227)(452.962464,918.0825101)(453.15542997,917.81665947)
\curveto(453.3483851,917.5550846)(453.48131347,917.2034676)(453.5542155,916.76180741)
\lineto(452.42217915,916.60743881)
\curveto(452.37071831,916.9590509)(452.22063788,917.23348368)(451.97193741,917.43073798)
\curveto(451.72751648,917.62798081)(451.38018749,917.72660509)(450.9299494,917.72661112)
\curveto(450.39823269,917.72660509)(450.01874361,917.63870084)(449.79148102,917.46289811)
\curveto(449.56421431,917.28708384)(449.45058199,917.08125925)(449.45058371,916.84542373)
\curveto(449.45058199,916.69533815)(449.49775012,916.56026577)(449.59208825,916.44020617)
\curveto(449.68642266,916.31584907)(449.83435908,916.21293678)(450.03589796,916.13146898)
\curveto(450.15167199,916.08858442)(450.49256896,915.98996014)(451.0585899,915.83559584)
\curveto(451.8775969,915.61690308)(452.44790253,915.43680656)(452.76950848,915.29530576)
\curveto(453.09539238,915.15808577)(453.3505291,914.9565492)(453.53491943,914.69069544)
\curveto(453.71929815,914.42483568)(453.81149042,914.09465874)(453.81149649,913.70016362)
\curveto(453.81149042,913.31424052)(453.69785809,912.94975948)(453.47059918,912.60671941)
\curveto(453.24761681,912.26796553)(452.92387189,912.00425278)(452.49936344,911.81558037)
\curveto(452.07484547,911.63119572)(451.5945881,911.53900346)(451.0585899,911.5390033)
\curveto(450.17096804,911.53900346)(449.49346211,911.72338798)(449.02607007,912.09215743)
\curveto(448.56296346,912.46092608)(448.26709062,913.00764764)(448.13845065,913.73232375)
}
}
{
\newrgbcolor{curcolor}{0 0 0}
\pscustom[linestyle=none,fillstyle=solid,fillcolor=curcolor]
{
\newpath
\moveto(459.8640337,913.89312437)
\lineto(461.06039031,913.7451878)
\curveto(460.87171103,913.04623975)(460.52223803,912.50380621)(460.01197027,912.11788553)
\curveto(459.50169113,911.73196401)(458.84991327,911.53900346)(458.05663474,911.5390033)
\curveto(457.05752416,911.53900346)(456.2642419,911.84559633)(455.67678558,912.45878284)
\curveto(455.09361456,913.07625584)(454.80202973,913.9402903)(454.80203021,915.05088882)
\curveto(454.80202973,916.20007274)(455.09790258,917.09197928)(455.68964963,917.72661112)
\curveto(456.28139395,918.3612309)(457.04894814,918.6785438)(457.9923145,918.67855079)
\curveto(458.90565743,918.6785438)(459.65177155,918.36766292)(460.23065911,917.7459072)
\curveto(460.80953485,917.12413937)(461.09897568,916.24938488)(461.09898246,915.12164109)
\curveto(461.09897568,915.05302947)(461.09683167,914.95011718)(461.09255043,914.81290391)
\lineto(455.99838682,914.81290391)
\curveto(456.04126526,914.06249865)(456.25352187,913.48790501)(456.63515727,913.08912127)
\curveto(457.01678804,912.69033474)(457.4927574,912.49094217)(458.06306677,912.49094297)
\curveto(458.48757624,912.49094217)(458.84991327,912.60243049)(459.15007895,912.82540825)
\curveto(459.45023498,913.04838376)(459.68821966,913.40428877)(459.8640337,913.89312437)
\moveto(456.06270707,915.76484357)
\lineto(459.87689775,915.76484357)
\curveto(459.82543605,916.33943314)(459.67964363,916.77037836)(459.43952007,917.05768055)
\curveto(459.0707459,917.50362845)(458.59263253,917.72660509)(458.00517855,917.72661112)
\curveto(457.47346135,917.72660509)(457.02536407,917.54865258)(456.66088537,917.19275307)
\curveto(456.30069,916.83684255)(456.10129744,916.3608732)(456.06270707,915.76484357)
}
}
{
\newrgbcolor{curcolor}{0 0 0}
\pscustom[linestyle=none,fillstyle=solid,fillcolor=curcolor]
{
\newpath
\moveto(468.61801926,915.81629977)
\lineto(462.37252322,913.14700949)
\lineto(462.37252322,914.29834193)
\lineto(467.31875026,916.35015782)
\lineto(462.37252322,918.38267765)
\lineto(462.37252322,919.53401008)
\lineto(468.61801926,916.89687993)
\lineto(468.61801926,915.81629977)
}
}
{
\newrgbcolor{curcolor}{0 0 0}
\pscustom[linestyle=none,fillstyle=solid,fillcolor=curcolor]
{
\newpath
\moveto(476.31071869,915.81629977)
\lineto(470.06522265,913.14700949)
\lineto(470.06522265,914.29834193)
\lineto(475.01144969,916.35015782)
\lineto(470.06522265,918.38267765)
\lineto(470.06522265,919.53401008)
\lineto(476.31071869,916.89687993)
\lineto(476.31071869,915.81629977)
}
}
{
\newrgbcolor{curcolor}{0 0 0}
\pscustom[linestyle=none,fillstyle=solid,fillcolor=curcolor]
{
\newpath
\moveto(430.89674455,891.93424714)
\lineto(429.08934559,891.93424714)
\lineto(429.08934559,898.76505744)
\lineto(430.76810405,898.76505744)
\lineto(430.76810405,897.7938217)
\curveto(431.05539833,898.25263315)(431.31267906,898.55493801)(431.53994703,898.70073719)
\curveto(431.77149637,898.84652284)(432.03306511,898.91941905)(432.32465405,898.91942603)
\curveto(432.73629912,898.91941905)(433.13294025,898.80578672)(433.51457863,898.57852872)
\lineto(432.95499247,897.00268265)
\curveto(432.65053887,897.19992615)(432.36753006,897.29855043)(432.1059652,897.29855579)
\curveto(431.8529686,897.29855043)(431.63856799,897.22779823)(431.46276273,897.08629898)
\curveto(431.28695099,896.94907743)(431.14759059,896.69822872)(431.04468112,896.33375208)
\curveto(430.94605401,895.96926664)(430.89674187,895.20600046)(430.89674455,894.04395126)
\lineto(430.89674455,891.93424714)
}
}
{
\newrgbcolor{curcolor}{0 0 0}
\pscustom[linestyle=none,fillstyle=solid,fillcolor=curcolor]
{
\newpath
\moveto(433.86833992,895.44613266)
\curveto(433.86833939,896.04645086)(434.01627582,896.62747652)(434.31214963,897.18921137)
\curveto(434.6080215,897.75093572)(435.02610269,898.17973694)(435.56639446,898.47561632)
\curveto(436.11096979,898.77148263)(436.71772351,898.91941905)(437.38665747,898.91942603)
\curveto(438.42006437,898.91941905)(439.26694678,898.58281009)(439.92730725,897.90959814)
\curveto(440.58765454,897.24066226)(440.91783148,896.39377985)(440.91783906,895.36894836)
\curveto(440.91783148,894.33553398)(440.58336653,893.47793154)(439.9144432,892.79613846)
\curveto(439.24979473,892.11863167)(438.41148834,891.7798787)(437.39952152,891.77987855)
\curveto(436.77346767,891.7798787)(436.17528997,891.9213831)(435.60498661,892.20439218)
\curveto(435.03896673,892.48740072)(434.6080215,892.9011939)(434.31214963,893.44577296)
\curveto(434.01627582,893.99463701)(433.86833939,894.66142291)(433.86833992,895.44613266)
\moveto(435.72076305,895.34965229)
\curveto(435.72076067,894.67214294)(435.88156113,894.15329346)(436.20316491,893.7931023)
\curveto(436.52476296,893.43290741)(436.9214041,893.2528109)(437.39308949,893.25281222)
\curveto(437.86476678,893.2528109)(438.25926391,893.43290741)(438.57658205,893.7931023)
\curveto(438.89817773,894.15329346)(439.05897819,894.67643096)(439.05898391,895.36251634)
\curveto(439.05897819,896.03144282)(438.89817773,896.54600428)(438.57658205,896.90620228)
\curveto(438.25926391,897.26639034)(437.86476678,897.44648685)(437.39308949,897.44649236)
\curveto(436.9214041,897.44648685)(436.52476296,897.26639034)(436.20316491,896.90620228)
\curveto(435.88156113,896.54600428)(435.72076067,896.0271548)(435.72076305,895.34965229)
}
}
{
\newrgbcolor{curcolor}{0 0 0}
\pscustom[linestyle=none,fillstyle=solid,fillcolor=curcolor]
{
\newpath
\moveto(446.83530191,891.93424714)
\lineto(446.83530191,892.95693908)
\curveto(446.58659176,892.59245702)(446.25855882,892.3051602)(445.85120212,892.09504776)
\curveto(445.44812451,891.884935)(445.0214673,891.7798787)(444.57122919,891.77987855)
\curveto(444.11240871,891.7798787)(443.70075954,891.88064699)(443.33628044,892.08218371)
\curveto(442.97179746,892.28372014)(442.70808471,892.56672894)(442.54514139,892.93121098)
\curveto(442.38219578,893.29569102)(442.30072355,893.79953246)(442.30072445,894.4427368)
\lineto(442.30072445,898.76505744)
\lineto(444.10812341,898.76505744)
\lineto(444.10812341,895.62622935)
\curveto(444.1081207,894.66571092)(444.14028079,894.07610924)(444.20460378,893.85742254)
\curveto(444.27320917,893.64302001)(444.39541751,893.47149952)(444.57122919,893.34286056)
\curveto(444.74703452,893.2185068)(444.97001115,893.15633062)(445.24015977,893.15633185)
\curveto(445.5488928,893.15633062)(445.82546959,893.23994686)(446.06989096,893.40718081)
\curveto(446.31430298,893.57869983)(446.48153546,893.78881243)(446.57158889,894.03751924)
\curveto(446.66163197,894.29050986)(446.7066561,894.90583961)(446.70666141,895.88351034)
\lineto(446.70666141,898.76505744)
\lineto(448.51406037,898.76505744)
\lineto(448.51406037,891.93424714)
\lineto(446.83530191,891.93424714)
}
}
{
\newrgbcolor{curcolor}{0 0 0}
\pscustom[linestyle=none,fillstyle=solid,fillcolor=curcolor]
{
\newpath
\moveto(453.52460725,898.76505744)
\lineto(453.52460725,897.32428389)
\lineto(452.2896585,897.32428389)
\lineto(452.2896585,894.57137729)
\curveto(452.28965565,894.01393307)(452.30037569,893.68804414)(452.32181862,893.59370953)
\curveto(452.34754382,893.50365961)(452.40114397,893.4286194)(452.48261924,893.36858866)
\curveto(452.56837645,893.30855506)(452.67128874,893.27853897)(452.79135643,893.27854032)
\curveto(452.95858556,893.27853897)(453.20085825,893.33642714)(453.51817523,893.45220498)
\lineto(453.67254382,892.05002359)
\curveto(453.2523144,891.86992696)(452.77634504,891.7798787)(452.24463432,891.77987855)
\curveto(451.9187426,891.7798787)(451.62501376,891.83347885)(451.36344693,891.94067917)
\curveto(451.10187627,892.05216748)(450.90891572,892.19367188)(450.7845647,892.3651928)
\curveto(450.66449902,892.54100087)(450.58088279,892.77684154)(450.53371574,893.07271552)
\curveto(450.49512254,893.28282698)(450.47582649,893.70734019)(450.47582751,894.34625643)
\lineto(450.47582751,897.32428389)
\lineto(449.64609632,897.32428389)
\lineto(449.64609632,898.76505744)
\lineto(450.47582751,898.76505744)
\lineto(450.47582751,900.12221466)
\lineto(452.2896585,901.17706672)
\lineto(452.2896585,898.76505744)
\lineto(453.52460725,898.76505744)
}
}
{
\newrgbcolor{curcolor}{0 0 0}
\pscustom[linestyle=none,fillstyle=solid,fillcolor=curcolor]
{
\newpath
\moveto(458.74741204,894.10827151)
\lineto(460.54837897,893.80596635)
\curveto(460.31681961,893.14561059)(459.95019456,892.64176916)(459.44850274,892.29444053)
\curveto(458.95108772,891.95139919)(458.32718194,891.7798787)(457.57678353,891.77987855)
\curveto(456.38900041,891.7798787)(455.50995791,892.16794381)(454.93965338,892.94407503)
\curveto(454.489411,893.56583579)(454.26429036,894.35054203)(454.26429078,895.29819609)
\curveto(454.26429036,896.43022795)(454.5601632,897.31570248)(455.15191019,897.95462232)
\curveto(455.74365458,898.59781813)(456.49191271,898.91941905)(457.39668684,898.91942603)
\curveto(458.41294218,898.91941905)(459.21480047,898.58281009)(459.8022641,897.90959814)
\curveto(460.38971582,897.24066226)(460.67058062,896.21368334)(460.64485934,894.82865828)
\lineto(456.11671391,894.82865828)
\curveto(456.12957568,894.29265386)(456.27536809,893.87457267)(456.55409159,893.57441345)
\curveto(456.83280968,893.27853897)(457.18013867,893.13060255)(457.5960796,893.13060375)
\curveto(457.87908466,893.13060255)(458.11706934,893.20778677)(458.31003435,893.36215664)
\curveto(458.50299044,893.51652365)(458.64878285,893.76522836)(458.74741204,894.10827151)
\moveto(458.85032443,895.93496654)
\curveto(458.83745539,896.45810003)(458.70238301,896.85474116)(458.44510687,897.12489112)
\curveto(458.18782154,897.39931872)(457.87479665,897.53653511)(457.50603126,897.53654071)
\curveto(457.11153047,897.53653511)(456.78564155,897.3928867)(456.52836349,897.10559505)
\curveto(456.27108008,896.81829306)(456.14458372,896.42808395)(456.14887403,895.93496654)
\lineto(458.85032443,895.93496654)
}
}
{
\newrgbcolor{curcolor}{0 0 0}
\pscustom[linestyle=none,fillstyle=solid,fillcolor=curcolor]
{
\newpath
\moveto(461.48745444,893.88315064)
\lineto(463.30128543,894.15972771)
\curveto(463.37846752,893.80810848)(463.53497997,893.54010772)(463.77082323,893.35572461)
\curveto(464.00666131,893.17562668)(464.33683826,893.08557842)(464.76135505,893.08557957)
\curveto(465.2287448,893.08557842)(465.5803618,893.17133867)(465.81620711,893.34286056)
\curveto(465.97485892,893.4629235)(466.05418715,893.62372396)(466.05419202,893.82526242)
\curveto(466.05418715,893.96247692)(466.01130703,894.07610924)(465.92555153,894.16615973)
\curveto(465.83549853,894.25191775)(465.63396195,894.33124597)(465.3209412,894.40414465)
\curveto(463.86301291,894.7257431)(462.93894627,895.01947193)(462.54873853,895.28533204)
\curveto(462.00844762,895.65409774)(461.73830285,896.1665152)(461.73830341,896.82258596)
\curveto(461.73830285,897.41432676)(461.97199952,897.91173618)(462.43939411,898.3148157)
\curveto(462.90678618,898.71788247)(463.63146025,898.91941905)(464.61341848,898.91942603)
\curveto(465.54820171,898.91941905)(466.24285969,898.76719461)(466.6973945,898.46275227)
\curveto(467.15191828,898.15829688)(467.46494317,897.70805559)(467.63647012,897.11202707)
\lineto(465.93198355,896.79685786)
\curveto(465.85908259,897.06270976)(465.7197222,897.26639034)(465.51390194,897.40790021)
\curveto(465.31236104,897.54939914)(465.02292021,897.62015134)(464.6455786,897.62015703)
\curveto(464.16960578,897.62015134)(463.82870881,897.55368715)(463.62288666,897.42076426)
\curveto(463.48566783,897.32642251)(463.41705963,897.20421416)(463.41706187,897.05413885)
\curveto(463.41705963,896.92549336)(463.47709181,896.81614905)(463.59715857,896.72610559)
\curveto(463.76010061,896.60603645)(464.32183021,896.43665997)(465.28234905,896.21797563)
\curveto(466.2471477,895.99928273)(466.92036562,895.73128196)(467.30200483,895.41397254)
\curveto(467.67934378,895.09236814)(467.86801632,894.64427086)(467.86802301,894.06967936)
\curveto(467.86801632,893.44362744)(467.60644757,892.90548191)(467.08331599,892.45524115)
\curveto(466.56017259,892.00499934)(465.78618639,891.7798787)(464.76135505,891.77987855)
\curveto(463.83085281,891.7798787)(463.09331471,891.96855124)(462.54873853,892.34589673)
\curveto(462.00844762,892.72324139)(461.65468661,893.23565885)(461.48745444,893.88315064)
}
}
{
\newrgbcolor{curcolor}{1 0.50196081 0.50196081}
\pscustom[linewidth=2.63455725,linecolor=curcolor]
{
\newpath
\moveto(648.84463,762.31953456)
\lineto(733.15046,762.31953456)
\lineto(733.15046,815.01067456)
\lineto(677.82477,815.01067456)
\lineto(677.82477,825.54890456)
\lineto(648.84463,825.54890456)
\lineto(648.84463,762.31953456)
\closepath
}
}
{
\newrgbcolor{curcolor}{0 0 0}
\pscustom[linestyle=none,fillstyle=solid,fillcolor=curcolor]
{
\newpath
\moveto(661.42054551,796.41176199)
\lineto(661.42054551,797.49234215)
\lineto(667.66604156,800.1294723)
\lineto(667.66604156,798.97813987)
\lineto(662.71338249,796.94562004)
\lineto(667.66604156,794.89380414)
\lineto(667.66604156,793.74247171)
\lineto(661.42054551,796.41176199)
}
}
{
\newrgbcolor{curcolor}{0 0 0}
\pscustom[linestyle=none,fillstyle=solid,fillcolor=curcolor]
{
\newpath
\moveto(669.11324733,796.41176199)
\lineto(669.11324733,797.49234215)
\lineto(675.35874337,800.1294723)
\lineto(675.35874337,798.97813987)
\lineto(670.40608431,796.94562004)
\lineto(675.35874337,794.89380414)
\lineto(675.35874337,793.74247171)
\lineto(669.11324733,796.41176199)
}
}
{
\newrgbcolor{curcolor}{0 0 0}
\pscustom[linestyle=none,fillstyle=solid,fillcolor=curcolor]
{
\newpath
\moveto(678.02160183,792.28883411)
\lineto(676.94745369,792.28883411)
\lineto(676.94745369,801.71818241)
\lineto(678.10521815,801.71818241)
\lineto(678.10521815,798.35423346)
\curveto(678.59404952,798.96741315)(679.2179553,799.27400602)(679.97693735,799.274013)
\curveto(680.39715866,799.27400602)(680.79379979,799.18824578)(681.16686194,799.01673201)
\curveto(681.54420193,798.84949281)(681.85293881,798.61150813)(682.0930735,798.30277727)
\curveto(682.33748419,797.99832238)(682.52830073,797.62955333)(682.6655237,797.19646901)
\curveto(682.80273352,796.76337486)(682.87134171,796.30026955)(682.8713485,795.80715166)
\curveto(682.87134171,794.6365208)(682.58190089,793.73175023)(682.00302515,793.09283721)
\curveto(681.42413759,792.45392258)(680.72947961,792.13446567)(679.91904913,792.13446552)
\curveto(679.112899,792.13446567)(678.4804172,792.47107463)(678.02160183,793.14429341)
\lineto(678.02160183,792.28883411)
\moveto(678.00873778,795.75569546)
\curveto(678.00873585,794.93668166)(678.12022417,794.34493597)(678.34320306,793.98045663)
\curveto(678.70768185,793.38442124)(679.20080325,793.08640439)(679.82256876,793.08640518)
\curveto(680.32855047,793.08640439)(680.76592771,793.30509301)(681.13470181,793.74247171)
\curveto(681.50346581,794.18413551)(681.68785034,794.84020138)(681.68785594,795.71067129)
\curveto(681.68785034,796.60257441)(681.50989783,797.26078428)(681.15399789,797.68530289)
\curveto(680.80237582,798.1098107)(680.3757186,798.32206731)(679.87402496,798.32207334)
\curveto(679.36803573,798.32206731)(678.93065848,798.10123468)(678.56189191,797.65957479)
\curveto(678.19312038,797.22219217)(678.00873585,796.58756636)(678.00873778,795.75569546)
}
}
{
\newrgbcolor{curcolor}{0 0 0}
\pscustom[linestyle=none,fillstyle=solid,fillcolor=curcolor]
{
\newpath
\moveto(688.74378601,793.13142936)
\curveto(688.31497946,792.76694748)(687.90118628,792.50966674)(687.50240523,792.35958639)
\curveto(687.10790402,792.20950589)(686.68339081,792.13446567)(686.22886433,792.13446552)
\curveto(685.47845938,792.13446567)(684.90172173,792.31670619)(684.49864967,792.68118762)
\curveto(684.09557544,793.04995628)(683.89403886,793.51949362)(683.89403934,794.08980105)
\curveto(683.89403886,794.4242642)(683.96907908,794.72871307)(684.11916021,795.00314856)
\curveto(684.27352794,795.28186664)(684.47292051,795.50484328)(684.71733851,795.67207914)
\curveto(684.96604192,795.83930823)(685.24476271,795.96580459)(685.55350173,796.0515686)
\curveto(685.78076424,796.11159701)(686.12380522,796.16948517)(686.58262569,796.22523327)
\curveto(687.51740919,796.33671765)(688.20563515,796.46964603)(688.64730564,796.6240188)
\curveto(688.65158842,796.78267092)(688.65373243,796.88343921)(688.65373766,796.92632397)
\curveto(688.65373243,797.39800067)(688.54438812,797.73032162)(688.3257044,797.9232878)
\curveto(688.02982665,798.18485092)(687.5903054,798.31563529)(687.00713932,798.31564132)
\curveto(686.46255818,798.31563529)(686.05948503,798.21915501)(685.79791867,798.0262002)
\curveto(685.54063556,797.83752193)(685.34981901,797.50091297)(685.22546846,797.01637231)
\lineto(684.09343211,797.17074091)
\curveto(684.19634372,797.65528141)(684.36572021,798.04549052)(684.60156206,798.34136941)
\curveto(684.83740155,798.64152422)(685.17829852,798.87093287)(685.624254,799.02959606)
\curveto(686.07020506,799.19253379)(686.58691054,799.27400602)(687.17437197,799.274013)
\curveto(687.75753787,799.27400602)(688.23136322,799.20539782)(688.59584944,799.06818821)
\curveto(688.9603253,798.93096504)(689.22832606,798.75730055)(689.39985254,798.54719421)
\curveto(689.57136704,798.34136336)(689.69143138,798.07979462)(689.76004592,797.76248719)
\curveto(689.79863169,797.56523315)(689.81792775,797.20932814)(689.81793415,796.69477108)
\lineto(689.81793415,795.15108513)
\curveto(689.81792775,794.0747912)(689.84151181,793.39299726)(689.88868642,793.10570126)
\curveto(689.94013609,792.82269163)(690.03876037,792.55040286)(690.18455956,792.28883411)
\lineto(688.9753389,792.28883411)
\curveto(688.855269,792.5289628)(688.77808478,792.8098276)(688.74378601,793.13142936)
\moveto(688.64730564,795.71710331)
\curveto(688.22707521,795.54557939)(687.59673742,795.39978698)(686.75629036,795.27972563)
\curveto(686.28031766,795.21111444)(685.9437087,795.13393022)(685.74646247,795.04817274)
\curveto(685.54921158,794.96240973)(685.39698715,794.83591337)(685.28978871,794.66868328)
\curveto(685.18258654,794.50573643)(685.12898638,794.32349591)(685.12898809,794.12196117)
\curveto(685.12898638,793.81322246)(685.24476271,793.55594172)(685.47631743,793.3501182)
\curveto(685.71215604,793.14429255)(686.05519702,793.04138026)(686.50544139,793.04138101)
\curveto(686.95139158,793.04138026)(687.34803271,793.13786053)(687.69536597,793.33082212)
\curveto(688.04269069,793.52806964)(688.29782741,793.79607041)(688.46077692,794.13482522)
\curveto(688.58512423,794.39639212)(688.64730041,794.78231322)(688.64730564,795.29258968)
\lineto(688.64730564,795.71710331)
}
}
{
\newrgbcolor{curcolor}{0 0 0}
\pscustom[linestyle=none,fillstyle=solid,fillcolor=curcolor]
{
\newpath
\moveto(691.15579515,794.32778596)
\lineto(692.30069556,794.50788266)
\curveto(692.36501419,794.04906313)(692.5429667,793.69744613)(692.83455362,793.4530306)
\curveto(693.13042438,793.20861273)(693.54207355,793.08640439)(694.06950237,793.08640518)
\curveto(694.60121257,793.08640439)(694.99570969,793.19360469)(695.25299493,793.40800642)
\curveto(695.51027116,793.62669393)(695.63891152,793.88183065)(695.63891641,794.17341737)
\curveto(695.63891152,794.43498423)(695.5252792,794.64080882)(695.2980191,794.79089175)
\curveto(695.1393581,794.89380154)(694.74486098,795.02458591)(694.11452654,795.18324526)
\curveto(693.26549676,795.39764297)(692.67589508,795.5820275)(692.34571973,795.73639939)
\curveto(692.01982921,795.89505239)(691.7711245,796.11159701)(691.59960486,796.38603389)
\curveto(691.43237154,796.66475058)(691.3487553,796.97134346)(691.3487559,797.30581343)
\curveto(691.3487553,797.61025728)(691.41736349,797.89112208)(691.55458069,798.14840867)
\curveto(691.69608429,798.40997156)(691.88690083,798.62651617)(692.12703089,798.79804317)
\curveto(692.30712603,798.93096504)(692.55154273,799.04245336)(692.86028172,799.13250846)
\curveto(693.1733045,799.22683789)(693.50776945,799.27400602)(693.86367758,799.274013)
\curveto(694.39967599,799.27400602)(694.86921333,799.1968218)(695.272291,799.04246011)
\curveto(695.67964764,798.88808492)(695.9798085,798.67797232)(696.17277447,798.41212169)
\curveto(696.3657296,798.15054682)(696.49865797,797.79892982)(696.57156,797.35726963)
\lineto(695.43952365,797.20290103)
\curveto(695.38806281,797.55451312)(695.23798238,797.8289459)(694.98928191,798.0262002)
\curveto(694.74486098,798.22344303)(694.39753199,798.32206731)(693.9472939,798.32207334)
\curveto(693.41557719,798.32206731)(693.03608811,798.23416306)(692.80882552,798.05836032)
\curveto(692.58155881,797.88254605)(692.46792649,797.67672147)(692.46792821,797.44088595)
\curveto(692.46792649,797.29080037)(692.51509462,797.15572798)(692.60943275,797.03566839)
\curveto(692.70376716,796.91131129)(692.85170358,796.80839899)(693.05324246,796.7269312)
\curveto(693.16901649,796.68404664)(693.50991346,796.58542236)(694.0759344,796.43105806)
\curveto(694.8949414,796.21236529)(695.46524703,796.03226878)(695.78685298,795.89076798)
\curveto(696.11273688,795.75354799)(696.3678736,795.55201141)(696.55226393,795.28615765)
\curveto(696.73664265,795.0202979)(696.82883492,794.69012096)(696.82884099,794.29562584)
\curveto(696.82883492,793.90970273)(696.71520259,793.54522169)(696.48794368,793.20218163)
\curveto(696.26496131,792.86342775)(695.94121639,792.599715)(695.51670794,792.41104258)
\curveto(695.09218997,792.22665794)(694.6119326,792.13446567)(694.0759344,792.13446552)
\curveto(693.18831254,792.13446567)(692.51080661,792.3188502)(692.04341457,792.68761965)
\curveto(691.58030796,793.0563883)(691.28443512,793.60310986)(691.15579515,794.32778596)
}
}
{
\newrgbcolor{curcolor}{0 0 0}
\pscustom[linestyle=none,fillstyle=solid,fillcolor=curcolor]
{
\newpath
\moveto(702.8813782,794.48858658)
\lineto(704.07773481,794.34065001)
\curveto(703.88905553,793.64170197)(703.53958253,793.09926842)(703.02931477,792.71334775)
\curveto(702.51903563,792.32742622)(701.86725777,792.13446567)(701.07397924,792.13446552)
\curveto(700.07486866,792.13446567)(699.2815864,792.44105855)(698.69413008,793.05424506)
\curveto(698.11095906,793.67171805)(697.81937423,794.53575252)(697.81937471,795.64635104)
\curveto(697.81937423,796.79553496)(698.11524708,797.6874415)(698.70699413,798.32207334)
\curveto(699.29873845,798.95669312)(700.06629264,799.27400602)(701.009659,799.274013)
\curveto(701.92300193,799.27400602)(702.66911605,798.96312513)(703.24800361,798.34136941)
\curveto(703.82687935,797.71960159)(704.11632018,796.8448471)(704.11632696,795.71710331)
\curveto(704.11632018,795.64849169)(704.11417617,795.54557939)(704.10989493,795.40836612)
\lineto(699.01573132,795.40836612)
\curveto(699.05860976,794.65796086)(699.27086637,794.08336723)(699.65250177,793.68458349)
\curveto(700.03413254,793.28579695)(700.5101019,793.08640439)(701.08041127,793.08640518)
\curveto(701.50492074,793.08640439)(701.86725777,793.1978927)(702.16742345,793.42087047)
\curveto(702.46757948,793.64384597)(702.70556416,793.99975099)(702.8813782,794.48858658)
\moveto(699.08005157,796.36030579)
\lineto(702.89424225,796.36030579)
\curveto(702.84278055,796.93489535)(702.69698813,797.36584058)(702.45686457,797.65314276)
\curveto(702.0880904,798.09909067)(701.60997703,798.32206731)(701.02252305,798.32207334)
\curveto(700.49080585,798.32206731)(700.04270857,798.1441148)(699.67822987,797.78821528)
\curveto(699.3180345,797.43230477)(699.11864194,796.95633541)(699.08005157,796.36030579)
}
}
{
\newrgbcolor{curcolor}{0 0 0}
\pscustom[linestyle=none,fillstyle=solid,fillcolor=curcolor]
{
\newpath
\moveto(711.63536376,796.41176199)
\lineto(705.38986772,793.74247171)
\lineto(705.38986772,794.89380414)
\lineto(710.33609476,796.94562004)
\lineto(705.38986772,798.97813987)
\lineto(705.38986772,800.1294723)
\lineto(711.63536376,797.49234215)
\lineto(711.63536376,796.41176199)
}
}
{
\newrgbcolor{curcolor}{0 0 0}
\pscustom[linestyle=none,fillstyle=solid,fillcolor=curcolor]
{
\newpath
\moveto(719.32806319,796.41176199)
\lineto(713.08256715,793.74247171)
\lineto(713.08256715,794.89380414)
\lineto(718.02879419,796.94562004)
\lineto(713.08256715,798.97813987)
\lineto(713.08256715,800.1294723)
\lineto(719.32806319,797.49234215)
\lineto(719.32806319,796.41176199)
}
}
{
\newrgbcolor{curcolor}{0 0 0}
\pscustom[linestyle=none,fillstyle=solid,fillcolor=curcolor]
{
\newpath
\moveto(663.46074066,779.36047388)
\lineto(663.46074066,777.91970033)
\lineto(662.22579191,777.91970033)
\lineto(662.22579191,775.16679373)
\curveto(662.22578906,774.60934951)(662.2365091,774.28346058)(662.25795203,774.18912597)
\curveto(662.28367723,774.09907605)(662.33727738,774.02403584)(662.41875265,773.9640051)
\curveto(662.50450986,773.9039715)(662.60742215,773.87395541)(662.72748984,773.87395676)
\curveto(662.89471897,773.87395541)(663.13699166,773.93184358)(663.45430864,774.04762142)
\lineto(663.60867723,772.64544003)
\curveto(663.18844781,772.4653434)(662.71247845,772.37529514)(662.18076773,772.37529499)
\curveto(661.85487601,772.37529514)(661.56114717,772.42889529)(661.29958034,772.53609561)
\curveto(661.03800968,772.64758392)(660.84504913,772.78908832)(660.72069811,772.96060924)
\curveto(660.60063243,773.13641731)(660.5170162,773.37225798)(660.46984915,773.66813196)
\curveto(660.43125595,773.87824342)(660.4119599,774.30275663)(660.41196092,774.94167287)
\lineto(660.41196092,777.91970033)
\lineto(659.58222973,777.91970033)
\lineto(659.58222973,779.36047388)
\lineto(660.41196092,779.36047388)
\lineto(660.41196092,780.7176311)
\lineto(662.22579191,781.77248316)
\lineto(662.22579191,779.36047388)
\lineto(663.46074066,779.36047388)
}
}
{
\newrgbcolor{curcolor}{0 0 0}
\pscustom[linestyle=none,fillstyle=solid,fillcolor=curcolor]
{
\newpath
\moveto(668.68354497,774.70368795)
\lineto(670.4845119,774.40138279)
\curveto(670.25295254,773.74102703)(669.8863275,773.2371856)(669.38463567,772.88985697)
\curveto(668.88722065,772.54681563)(668.26331487,772.37529514)(667.51291646,772.37529499)
\curveto(666.32513335,772.37529514)(665.44609084,772.76336025)(664.87578631,773.53949147)
\curveto(664.42554393,774.16125223)(664.20042329,774.94595847)(664.20042371,775.89361253)
\curveto(664.20042329,777.02564439)(664.49629614,777.91111892)(665.08804313,778.55003876)
\curveto(665.67978751,779.19323457)(666.42804564,779.51483549)(667.33281977,779.51484247)
\curveto(668.34907512,779.51483549)(669.1509334,779.17822653)(669.73839703,778.50501458)
\curveto(670.32584875,777.8360787)(670.60671355,776.80909978)(670.58099227,775.42407472)
\lineto(666.05284684,775.42407472)
\curveto(666.06570861,774.8880703)(666.21150102,774.46998911)(666.49022453,774.1698299)
\curveto(666.76894261,773.87395541)(667.1162716,773.72601899)(667.53221254,773.72602019)
\curveto(667.81521759,773.72601899)(668.05320227,773.80320321)(668.24616729,773.95757308)
\curveto(668.43912337,774.11194009)(668.58491579,774.3606448)(668.68354497,774.70368795)
\moveto(668.78645737,776.53038298)
\curveto(668.77358833,777.05351647)(668.63851594,777.4501576)(668.38123981,777.72030756)
\curveto(668.12395447,777.99473516)(667.81092958,778.13195155)(667.44216419,778.13195715)
\curveto(667.04766341,778.13195155)(666.72177448,777.98830314)(666.46449643,777.70101149)
\curveto(666.20721301,777.4137095)(666.08071665,777.02350039)(666.08500697,776.53038298)
\lineto(668.78645737,776.53038298)
}
}
{
\newrgbcolor{curcolor}{0 0 0}
\pscustom[linestyle=none,fillstyle=solid,fillcolor=curcolor]
{
\newpath
\moveto(671.92528579,779.36047388)
\lineto(673.5911802,779.36047388)
\lineto(673.5911802,778.42783029)
\curveto(674.18721142,779.15249845)(674.89687744,779.51483549)(675.72018039,779.51484247)
\curveto(676.15755304,779.51483549)(676.53704212,779.42478723)(676.85864878,779.24469743)
\curveto(677.18024395,779.0645942)(677.4439567,778.79230543)(677.64978782,778.42783029)
\curveto(677.94994214,778.79230543)(678.27368707,779.0645942)(678.62102356,779.24469743)
\curveto(678.96834505,779.42478723)(679.3392581,779.51483549)(679.73376385,779.51484247)
\curveto(680.23545266,779.51483549)(680.65996587,779.41192319)(681.00730475,779.20610528)
\curveto(681.35462385,779.00456203)(681.61404859,778.70654518)(681.78557975,778.31205384)
\curveto(681.90992143,778.02046323)(681.97209761,777.54878188)(681.97210846,776.89700839)
\lineto(681.97210846,772.52966358)
\lineto(680.16470951,772.52966358)
\lineto(680.16470951,776.43390261)
\curveto(680.16470046,777.11140464)(680.10252428,777.54878188)(679.97818079,777.74603566)
\curveto(679.81093945,778.00331118)(679.55365871,778.13195155)(679.20633782,778.13195715)
\curveto(678.953337,778.13195155)(678.71535233,778.05476733)(678.49238307,777.90040426)
\curveto(678.26939905,777.74603045)(678.1085986,777.5187658)(678.00998121,777.21860963)
\curveto(677.91135003,776.9227321)(677.86203789,776.45319476)(677.86204464,775.80999621)
\lineto(677.86204464,772.52966358)
\lineto(676.05464568,772.52966358)
\lineto(676.05464568,776.27310199)
\curveto(676.05464074,776.93774014)(676.02248065,777.36654137)(675.95816531,777.55950694)
\curveto(675.89384028,777.75246247)(675.793072,777.89611087)(675.65586015,777.9904526)
\curveto(675.52292723,778.08478341)(675.34068671,778.13195155)(675.10913804,778.13195715)
\curveto(674.83041325,778.13195155)(674.57956454,778.05691133)(674.35659115,777.90683628)
\curveto(674.13361127,777.75675048)(673.97281081,777.54020586)(673.87418929,777.25720178)
\curveto(673.77985026,776.97418825)(673.73268213,776.50465091)(673.73268474,775.84858836)
\lineto(673.73268474,772.52966358)
\lineto(671.92528579,772.52966358)
\lineto(671.92528579,779.36047388)
}
}
{
\newrgbcolor{curcolor}{0 0 0}
\pscustom[linestyle=none,fillstyle=solid,fillcolor=curcolor]
{
\newpath
\moveto(683.71518758,779.36047388)
\lineto(685.40037807,779.36047388)
\lineto(685.40037807,778.35707802)
\curveto(685.61906412,778.70011317)(685.91493696,778.97883396)(686.28799749,779.19324123)
\curveto(686.66105108,779.40763518)(687.07484426,779.51483549)(687.52937827,779.51484247)
\curveto(688.32265582,779.51483549)(688.99587374,779.2039546)(689.54903404,778.58219888)
\curveto(690.10218089,777.96043106)(690.37875768,777.09425259)(690.37876524,775.98366088)
\curveto(690.37875768,774.84304617)(690.10003689,773.95542764)(689.54260202,773.32080263)
\curveto(688.98515371,772.69046404)(688.30979178,772.37529514)(687.51651422,772.37529499)
\curveto(687.13916445,772.37529514)(686.79612347,772.45033535)(686.48739026,772.60041585)
\curveto(686.18293772,772.75049621)(685.86133681,773.00777694)(685.52258654,773.37225882)
\lineto(685.52258654,769.93112558)
\lineto(683.71518758,769.93112558)
\lineto(683.71518758,779.36047388)
\moveto(685.50329047,776.06084518)
\curveto(685.50328779,775.29328746)(685.65551222,774.72512584)(685.95996423,774.35635861)
\curveto(686.26440995,773.99187575)(686.63532301,773.80963523)(687.07270451,773.80963651)
\curveto(687.49292546,773.80963523)(687.84239845,773.97686771)(688.12112455,774.31133444)
\curveto(688.39984004,774.65008562)(688.53920044,775.2032392)(688.53920616,775.97079683)
\curveto(688.53920044,776.68689143)(688.39555203,777.21860494)(688.1082605,777.56593897)
\curveto(687.82095839,777.91326292)(687.46505338,778.08692742)(687.04054439,778.08693298)
\curveto(686.59887491,778.08692742)(686.23224986,777.91540693)(685.94066815,777.57237099)
\curveto(685.6490802,777.23361299)(685.50328779,776.72977155)(685.50329047,776.06084518)
}
}
{
\newrgbcolor{curcolor}{0 0 0}
\pscustom[linestyle=none,fillstyle=solid,fillcolor=curcolor]
{
\newpath
\moveto(691.81953837,772.52966358)
\lineto(691.81953837,781.95901188)
\lineto(693.62693733,781.95901188)
\lineto(693.62693733,772.52966358)
\lineto(691.81953837,772.52966358)
}
}
{
\newrgbcolor{curcolor}{0 0 0}
\pscustom[linestyle=none,fillstyle=solid,fillcolor=curcolor]
{
\newpath
\moveto(696.82365446,777.27649786)
\lineto(695.18348814,777.57237099)
\curveto(695.36787201,778.23271983)(695.68518492,778.72155323)(696.13542781,779.03887264)
\curveto(696.58566748,779.35617904)(697.25459739,779.51483549)(698.14221953,779.51484247)
\curveto(698.94836222,779.51483549)(699.54868393,779.41835521)(699.94318647,779.22540136)
\curveto(700.33767818,779.03672212)(700.61425496,778.79444943)(700.77291766,778.49858256)
\curveto(700.93585588,778.20699176)(701.01732811,777.66884623)(701.0173346,776.88414434)
\lineto(700.99803853,774.77444022)
\curveto(700.99803206,774.17411627)(701.02590414,773.730307)(701.08165485,773.4430111)
\curveto(701.14168047,773.16000138)(701.25102478,772.85555251)(701.40968811,772.52966358)
\lineto(699.62158523,772.52966358)
\curveto(699.574412,772.64972792)(699.51652384,772.82768043)(699.44792056,773.06352164)
\curveto(699.41789956,773.17072141)(699.39645949,773.24147361)(699.38360031,773.27577845)
\curveto(699.07485858,772.97561685)(698.74468164,772.75049621)(698.3930685,772.60041585)
\curveto(698.04144763,772.45033535)(697.66624656,772.37529514)(697.26746417,772.37529499)
\curveto(696.56422742,772.37529514)(696.00892984,772.56611168)(695.60156975,772.94774519)
\curveto(695.19849553,773.32937786)(694.99695896,773.81177923)(694.99695943,774.39495076)
\curveto(694.99695896,774.78087)(695.08915122,775.12391097)(695.27353649,775.42407472)
\curveto(695.45792027,775.7285207)(695.715201,775.96007336)(696.04537946,776.1187334)
\curveto(696.3798429,776.28167427)(696.86010027,776.42317868)(697.48615301,776.54324703)
\curveto(698.33088846,776.70189947)(698.91620213,776.84983589)(699.24209577,776.98705674)
\lineto(699.24209577,777.16715343)
\curveto(699.24209105,777.51447779)(699.15633081,777.76103849)(698.98481478,777.90683628)
\curveto(698.81328983,778.05691133)(698.48954491,778.13195155)(698.01357904,778.13195715)
\curveto(697.69197464,778.13195155)(697.44112592,778.06763136)(697.26103214,777.93899641)
\curveto(697.0809329,777.81463864)(696.93514048,777.59380601)(696.82365446,777.27649786)
\moveto(699.24209577,775.80999621)
\curveto(699.01053839,775.73280871)(698.64391335,775.64061645)(698.14221953,775.53341914)
\curveto(697.64051849,775.42621584)(697.31248556,775.32115954)(697.15811975,775.21824993)
\curveto(696.92227644,775.05101477)(696.80435611,774.83875816)(696.80435838,774.58147948)
\curveto(696.80435611,774.32848471)(696.89869238,774.10979608)(697.08736747,773.92541295)
\curveto(697.27603745,773.74102703)(697.51616614,773.64883477)(697.80775425,773.64883589)
\curveto(698.1336399,773.64883477)(698.44452078,773.75603508)(698.74039784,773.97043713)
\curveto(698.95908225,774.13338015)(699.10273066,774.33277272)(699.1713435,774.56861543)
\curveto(699.21850699,774.72298183)(699.24209105,775.01671067)(699.24209577,775.44980282)
\lineto(699.24209577,775.80999621)
}
}
{
\newrgbcolor{curcolor}{0 0 0}
\pscustom[linestyle=none,fillstyle=solid,fillcolor=curcolor]
{
\newpath
\moveto(705.9378334,779.36047388)
\lineto(705.9378334,777.91970033)
\lineto(704.70288465,777.91970033)
\lineto(704.70288465,775.16679373)
\curveto(704.70288181,774.60934951)(704.71360184,774.28346058)(704.73504477,774.18912597)
\curveto(704.76076997,774.09907605)(704.81437013,774.02403584)(704.89584539,773.9640051)
\curveto(704.9816026,773.9039715)(705.0845149,773.87395541)(705.20458258,773.87395676)
\curveto(705.37181171,773.87395541)(705.6140844,773.93184358)(705.93140138,774.04762142)
\lineto(706.08576997,772.64544003)
\curveto(705.66554055,772.4653434)(705.18957119,772.37529514)(704.65786048,772.37529499)
\curveto(704.33196875,772.37529514)(704.03823991,772.42889529)(703.77667308,772.53609561)
\curveto(703.51510242,772.64758392)(703.32214187,772.78908832)(703.19779086,772.96060924)
\curveto(703.07772518,773.13641731)(702.99410894,773.37225798)(702.94694189,773.66813196)
\curveto(702.90834869,773.87824342)(702.88905264,774.30275663)(702.88905367,774.94167287)
\lineto(702.88905367,777.91970033)
\lineto(702.05932247,777.91970033)
\lineto(702.05932247,779.36047388)
\lineto(702.88905367,779.36047388)
\lineto(702.88905367,780.7176311)
\lineto(704.70288465,781.77248316)
\lineto(704.70288465,779.36047388)
\lineto(705.9378334,779.36047388)
}
}
{
\newrgbcolor{curcolor}{0 0 0}
\pscustom[linestyle=none,fillstyle=solid,fillcolor=curcolor]
{
\newpath
\moveto(711.16063819,774.70368795)
\lineto(712.96160512,774.40138279)
\curveto(712.73004576,773.74102703)(712.36342072,773.2371856)(711.86172889,772.88985697)
\curveto(711.36431387,772.54681563)(710.74040809,772.37529514)(709.99000968,772.37529499)
\curveto(708.80222657,772.37529514)(707.92318406,772.76336025)(707.35287953,773.53949147)
\curveto(706.90263715,774.16125223)(706.67751651,774.94595847)(706.67751693,775.89361253)
\curveto(706.67751651,777.02564439)(706.97338936,777.91111892)(707.56513635,778.55003876)
\curveto(708.15688073,779.19323457)(708.90513886,779.51483549)(709.80991299,779.51484247)
\curveto(710.82616834,779.51483549)(711.62802662,779.17822653)(712.21549025,778.50501458)
\curveto(712.80294197,777.8360787)(713.08380677,776.80909978)(713.05808549,775.42407472)
\lineto(708.52994006,775.42407472)
\curveto(708.54280183,774.8880703)(708.68859424,774.46998911)(708.96731775,774.1698299)
\curveto(709.24603583,773.87395541)(709.59336482,773.72601899)(710.00930576,773.72602019)
\curveto(710.29231081,773.72601899)(710.53029549,773.80320321)(710.72326051,773.95757308)
\curveto(710.91621659,774.11194009)(711.06200901,774.3606448)(711.16063819,774.70368795)
\moveto(711.26355059,776.53038298)
\curveto(711.25068155,777.05351647)(711.11560916,777.4501576)(710.85833303,777.72030756)
\curveto(710.60104769,777.99473516)(710.2880228,778.13195155)(709.91925741,778.13195715)
\curveto(709.52475663,778.13195155)(709.1988677,777.98830314)(708.94158965,777.70101149)
\curveto(708.68430623,777.4137095)(708.55780987,777.02350039)(708.56210019,776.53038298)
\lineto(711.26355059,776.53038298)
}
}
{
\newrgbcolor{curcolor}{0 0 0}
\pscustom[linestyle=none,fillstyle=solid,fillcolor=curcolor]
{
\newpath
\moveto(713.9006806,774.47856708)
\lineto(715.71451158,774.75514415)
\curveto(715.79169368,774.40352492)(715.94820612,774.13552416)(716.18404939,773.95114105)
\curveto(716.41988747,773.77104312)(716.75006441,773.68099486)(717.1745812,773.68099601)
\curveto(717.64197095,773.68099486)(717.99358795,773.76675511)(718.22943326,773.938277)
\curveto(718.38808508,774.05833994)(718.4674133,774.2191404)(718.46741818,774.42067886)
\curveto(718.4674133,774.55789336)(718.42453318,774.67152568)(718.33877768,774.76157617)
\curveto(718.24872468,774.84733419)(718.0471881,774.92666241)(717.73416735,774.99956109)
\curveto(716.27623906,775.32115954)(715.35217242,775.61488837)(714.96196468,775.88074848)
\curveto(714.42167377,776.24951418)(714.151529,776.76193164)(714.15152956,777.4180024)
\curveto(714.151529,778.0097432)(714.38522567,778.50715262)(714.85262026,778.91023214)
\curveto(715.32001233,779.31329891)(716.0446864,779.51483549)(717.02664463,779.51484247)
\curveto(717.96142786,779.51483549)(718.65608584,779.36261105)(719.11062065,779.05816871)
\curveto(719.56514443,778.75371332)(719.87816932,778.30347204)(720.04969627,777.70744351)
\lineto(718.34520971,777.3922743)
\curveto(718.27230875,777.6581262)(718.13294835,777.86180678)(717.9271281,778.00331665)
\curveto(717.72558719,778.14481558)(717.43614636,778.21556778)(717.05880475,778.21557347)
\curveto(716.58283193,778.21556778)(716.24193496,778.1491036)(716.03611282,778.0161807)
\curveto(715.89889398,777.92183895)(715.83028579,777.7996306)(715.83028803,777.64955529)
\curveto(715.83028579,777.52090981)(715.89031796,777.41156549)(716.01038472,777.32152203)
\curveto(716.17332676,777.20145289)(716.73505637,777.03207641)(717.69557521,776.81339207)
\curveto(718.66037385,776.59469917)(719.33359177,776.3266984)(719.71523098,776.00938898)
\curveto(720.09256993,775.68778458)(720.28124247,775.2396873)(720.28124916,774.6650958)
\curveto(720.28124247,774.03904388)(720.01967373,773.50089835)(719.49654214,773.05065759)
\curveto(718.97339874,772.60041578)(718.19941254,772.37529514)(717.1745812,772.37529499)
\curveto(716.24407897,772.37529514)(715.50654086,772.56396768)(714.96196468,772.94131317)
\curveto(714.42167377,773.31865783)(714.06791276,773.83107529)(713.9006806,774.47856708)
}
}
{
\newrgbcolor{curcolor}{0.50196081 0 1}
\pscustom[linewidth=2.63455725,linecolor=curcolor]
{
\newpath
\moveto(37.36788,762.31953456)
\lineto(122.991,762.31953456)
\lineto(122.991,815.01068456)
\lineto(65.03074,815.01068456)
\lineto(65.03074,825.54890456)
\lineto(37.36788,825.54890456)
\lineto(37.36788,762.31953456)
\closepath
}
}
{
\newrgbcolor{curcolor}{0 0 0}
\pscustom[linestyle=none,fillstyle=solid,fillcolor=curcolor]
{
\newpath
\moveto(44.67465627,796.41175903)
\lineto(44.67465627,797.49233919)
\lineto(50.92015231,800.12946934)
\lineto(50.92015231,798.97813691)
\lineto(45.96749324,796.94561708)
\lineto(50.92015231,794.89380118)
\lineto(50.92015231,793.74246875)
\lineto(44.67465627,796.41175903)
}
}
{
\newrgbcolor{curcolor}{0 0 0}
\pscustom[linestyle=none,fillstyle=solid,fillcolor=curcolor]
{
\newpath
\moveto(52.36735808,796.41175903)
\lineto(52.36735808,797.49233919)
\lineto(58.61285413,800.12946934)
\lineto(58.61285413,798.97813691)
\lineto(53.66019506,796.94561708)
\lineto(58.61285413,794.89380118)
\lineto(58.61285413,793.74246875)
\lineto(52.36735808,796.41175903)
}
}
{
\newrgbcolor{curcolor}{0 0 0}
\pscustom[linestyle=none,fillstyle=solid,fillcolor=curcolor]
{
\newpath
\moveto(60.20799647,792.28883115)
\lineto(60.20799647,799.11964145)
\lineto(61.24355245,799.11964145)
\lineto(61.24355245,798.16126976)
\curveto(61.45795116,798.49572884)(61.74310397,798.76372961)(62.09901175,798.96527286)
\curveto(62.454914,799.17109077)(62.86013116,799.27400306)(63.31466443,799.27401005)
\curveto(63.8206459,799.27400306)(64.23443907,799.16894676)(64.55604521,798.95884083)
\curveto(64.88192892,798.74872156)(65.11133757,798.45499273)(65.24427186,798.07765344)
\curveto(65.78455549,798.87521792)(66.4877895,799.27400306)(67.35397598,799.27401005)
\curveto(68.0314739,799.27400306)(68.55246738,799.08533052)(68.916958,798.70799187)
\curveto(69.28142946,798.33492838)(69.46366998,797.75819074)(69.4636801,796.97777721)
\lineto(69.4636801,792.28883115)
\lineto(68.31234767,792.28883115)
\lineto(68.31234767,796.59185572)
\curveto(68.3123387,797.05495674)(68.27374659,797.38727768)(68.19657122,797.58881956)
\curveto(68.12366616,797.79463884)(67.98859377,797.95972732)(67.79135366,798.08408546)
\curveto(67.59409665,798.20843202)(67.36254399,798.2706082)(67.09669499,798.27061418)
\curveto(66.61642986,798.2706082)(66.21764473,798.10980774)(65.90033838,797.78821233)
\curveto(65.58301892,797.47089392)(65.42436247,796.96062047)(65.42436855,796.25739043)
\lineto(65.42436855,792.28883115)
\lineto(64.26660409,792.28883115)
\lineto(64.26660409,796.72692824)
\curveto(64.26659917,797.24148527)(64.1722629,797.62740637)(63.983595,797.8846927)
\curveto(63.79491782,798.14196783)(63.48618094,798.2706082)(63.05738344,798.27061418)
\curveto(62.73149079,798.2706082)(62.42918593,798.18484796)(62.15046795,798.01333319)
\curveto(61.87603235,797.84180698)(61.67663979,797.59095826)(61.55228964,797.2607863)
\curveto(61.42793508,796.93060438)(61.3657589,796.45463503)(61.36576093,795.8328768)
\lineto(61.36576093,792.28883115)
\lineto(60.20799647,792.28883115)
}
}
{
\newrgbcolor{curcolor}{0 0 0}
\pscustom[linestyle=none,fillstyle=solid,fillcolor=curcolor]
{
\newpath
\moveto(71.22605474,800.38675033)
\lineto(71.22605474,801.71817946)
\lineto(72.38381919,801.71817946)
\lineto(72.38381919,800.38675033)
\lineto(71.22605474,800.38675033)
\moveto(71.22605474,792.28883115)
\lineto(71.22605474,799.11964145)
\lineto(72.38381919,799.11964145)
\lineto(72.38381919,792.28883115)
\lineto(71.22605474,792.28883115)
}
}
{
\newrgbcolor{curcolor}{0 0 0}
\pscustom[linestyle=none,fillstyle=solid,fillcolor=curcolor]
{
\newpath
\moveto(78.58429024,792.28883115)
\lineto(78.58429024,793.15072247)
\curveto(78.15119571,792.47321568)(77.5144259,792.13446271)(76.67397889,792.13446256)
\curveto(76.12939795,792.13446271)(75.62770052,792.28454314)(75.1688851,792.58470429)
\curveto(74.71435392,792.88486485)(74.36059291,793.30294604)(74.10760101,793.83894912)
\curveto(73.85889548,794.37923711)(73.73454312,794.99885488)(73.73454357,795.69780428)
\curveto(73.73454312,796.37959481)(73.84817545,796.99706857)(74.07544089,797.55022741)
\curveto(74.30270474,798.10766374)(74.64360171,798.53432095)(75.09813282,798.83020034)
\curveto(75.55266031,799.12606664)(76.06078975,799.27400306)(76.62252269,799.27401005)
\curveto(77.03416853,799.27400306)(77.40079357,799.18609881)(77.72239893,799.01029703)
\curveto(78.04399541,798.83876982)(78.30556415,798.61364918)(78.50710595,798.33493443)
\lineto(78.50710595,801.71817946)
\lineto(79.65843838,801.71817946)
\lineto(79.65843838,792.28883115)
\lineto(78.58429024,792.28883115)
\moveto(74.92446816,795.69780428)
\curveto(74.92446651,794.82304638)(75.10885104,794.16912451)(75.47762228,793.73603673)
\curveto(75.84638914,793.30294604)(76.28162238,793.08640143)(76.78332331,793.08640222)
\curveto(77.28930525,793.08640143)(77.71810648,793.29222601)(78.06972826,793.7038766)
\curveto(78.42562849,794.11981237)(78.603581,794.75229417)(78.60358632,795.60132391)
\curveto(78.603581,796.53610726)(78.42348449,797.22218921)(78.06329624,797.65957183)
\curveto(77.70309843,798.09694371)(77.25928917,798.31563233)(76.73186711,798.31563836)
\curveto(76.2173022,798.31563233)(75.78635697,798.10551973)(75.43903014,797.68529993)
\curveto(75.095987,797.26506934)(74.92446651,796.60257145)(74.92446816,795.69780428)
}
}
{
\newrgbcolor{curcolor}{0 0 0}
\pscustom[linestyle=none,fillstyle=solid,fillcolor=curcolor]
{
\newpath
\moveto(85.91679833,792.28883115)
\lineto(85.91679833,793.15072247)
\curveto(85.4837038,792.47321568)(84.84693398,792.13446271)(84.00648698,792.13446256)
\curveto(83.46190604,792.13446271)(82.96020861,792.28454314)(82.50139318,792.58470429)
\curveto(82.046862,792.88486485)(81.693101,793.30294604)(81.4401091,793.83894912)
\curveto(81.19140357,794.37923711)(81.06705121,794.99885488)(81.06705166,795.69780428)
\curveto(81.06705121,796.37959481)(81.18068353,796.99706857)(81.40794897,797.55022741)
\curveto(81.63521283,798.10766374)(81.9761098,798.53432095)(82.43064091,798.83020034)
\curveto(82.88516839,799.12606664)(83.39329784,799.27400306)(83.95503078,799.27401005)
\curveto(84.36667661,799.27400306)(84.73330166,799.18609881)(85.05490701,799.01029703)
\curveto(85.37650349,798.83876982)(85.63807224,798.61364918)(85.83961403,798.33493443)
\lineto(85.83961403,801.71817946)
\lineto(86.99094647,801.71817946)
\lineto(86.99094647,792.28883115)
\lineto(85.91679833,792.28883115)
\moveto(82.25697624,795.69780428)
\curveto(82.2569746,794.82304638)(82.44135913,794.16912451)(82.81013037,793.73603673)
\curveto(83.17889723,793.30294604)(83.61413047,793.08640143)(84.1158314,793.08640222)
\curveto(84.62181334,793.08640143)(85.05061456,793.29222601)(85.40223635,793.7038766)
\curveto(85.75813658,794.11981237)(85.93608909,794.75229417)(85.93609441,795.60132391)
\curveto(85.93608909,796.53610726)(85.75599257,797.22218921)(85.39580433,797.65957183)
\curveto(85.03560652,798.09694371)(84.59179726,798.31563233)(84.0643752,798.31563836)
\curveto(83.54981029,798.31563233)(83.11886506,798.10551973)(82.77153822,797.68529993)
\curveto(82.42849509,797.26506934)(82.2569746,796.60257145)(82.25697624,795.69780428)
}
}
{
\newrgbcolor{curcolor}{0 0 0}
\pscustom[linestyle=none,fillstyle=solid,fillcolor=curcolor]
{
\newpath
\moveto(88.79191326,792.28883115)
\lineto(88.79191326,801.71817946)
\lineto(89.94967772,801.71817946)
\lineto(89.94967772,792.28883115)
\lineto(88.79191326,792.28883115)
}
}
{
\newrgbcolor{curcolor}{0 0 0}
\pscustom[linestyle=none,fillstyle=solid,fillcolor=curcolor]
{
\newpath
\moveto(96.42672583,794.48858362)
\lineto(97.62308244,794.34064705)
\curveto(97.43440316,793.64169901)(97.08493016,793.09926546)(96.5746624,792.71334479)
\curveto(96.06438325,792.32742326)(95.4126054,792.13446271)(94.61932687,792.13446256)
\curveto(93.62021629,792.13446271)(92.82693403,792.44105559)(92.23947771,793.0542421)
\curveto(91.65630669,793.6717151)(91.36472186,794.53574956)(91.36472234,795.64634808)
\curveto(91.36472186,796.795532)(91.6605947,797.68743854)(92.25234176,798.32207038)
\curveto(92.84408608,798.95669016)(93.61164027,799.27400306)(94.55500663,799.27401005)
\curveto(95.46834956,799.27400306)(96.21446368,798.96312217)(96.79335124,798.34136646)
\curveto(97.37222698,797.71959863)(97.66166781,796.84484414)(97.66167459,795.71710035)
\curveto(97.66166781,795.64848873)(97.6595238,795.54557644)(97.65524256,795.40836316)
\lineto(92.56107895,795.40836316)
\curveto(92.60395739,794.65795791)(92.816214,794.08336427)(93.1978494,793.68458053)
\curveto(93.57948017,793.285794)(94.05544953,793.08640143)(94.6257589,793.08640222)
\curveto(95.05026836,793.08640143)(95.4126054,793.19788974)(95.71277108,793.42086751)
\curveto(96.01292711,793.64384302)(96.25091179,793.99974803)(96.42672583,794.48858362)
\moveto(92.6253992,796.36030283)
\lineto(96.43958988,796.36030283)
\curveto(96.38812818,796.93489239)(96.24233576,797.36583762)(96.0022122,797.65313981)
\curveto(95.63343803,798.09908771)(95.15532466,798.32206435)(94.56787068,798.32207038)
\curveto(94.03615347,798.32206435)(93.5880562,798.14411184)(93.2235775,797.78821233)
\curveto(92.86338213,797.43230181)(92.66398956,796.95633246)(92.6253992,796.36030283)
}
}
{
\newrgbcolor{curcolor}{0 0 0}
\pscustom[linestyle=none,fillstyle=solid,fillcolor=curcolor]
{
\newpath
\moveto(105.18071139,796.41175903)
\lineto(98.93521535,793.74246875)
\lineto(98.93521535,794.89380118)
\lineto(103.88144239,796.94561708)
\lineto(98.93521535,798.97813691)
\lineto(98.93521535,800.12946934)
\lineto(105.18071139,797.49233919)
\lineto(105.18071139,796.41175903)
}
}
{
\newrgbcolor{curcolor}{0 0 0}
\pscustom[linestyle=none,fillstyle=solid,fillcolor=curcolor]
{
\newpath
\moveto(112.87341464,796.41175903)
\lineto(106.62791859,793.74246875)
\lineto(106.62791859,794.89380118)
\lineto(111.57414564,796.94561708)
\lineto(106.62791859,798.97813691)
\lineto(106.62791859,800.12946934)
\lineto(112.87341464,797.49233919)
\lineto(112.87341464,796.41175903)
}
}
{
\newrgbcolor{curcolor}{0 0 0}
\pscustom[linestyle=none,fillstyle=solid,fillcolor=curcolor]
{
\newpath
\moveto(67.70645359,772.52967588)
\lineto(65.89905463,772.52967588)
\lineto(65.89905463,779.36048618)
\lineto(67.57781309,779.36048618)
\lineto(67.57781309,778.38925044)
\curveto(67.86510736,778.84806189)(68.12238809,779.15036675)(68.34965606,779.29616593)
\curveto(68.5812054,779.44195158)(68.84277415,779.51484779)(69.13436308,779.51485477)
\curveto(69.54600815,779.51484779)(69.94264928,779.40121546)(70.32428766,779.17395746)
\lineto(69.76470151,777.59811139)
\curveto(69.46024791,777.79535489)(69.1772391,777.89397917)(68.91567424,777.89398453)
\curveto(68.66267763,777.89397917)(68.44827702,777.82322697)(68.27247176,777.68172772)
\curveto(68.09666002,777.54450617)(67.95729962,777.29365746)(67.85439015,776.92918082)
\curveto(67.75576305,776.56469538)(67.70645091,775.8014292)(67.70645359,774.63938)
\lineto(67.70645359,772.52967588)
}
}
{
\newrgbcolor{curcolor}{0 0 0}
\pscustom[linestyle=none,fillstyle=solid,fillcolor=curcolor]
{
\newpath
\moveto(70.67804896,776.0415614)
\curveto(70.67804843,776.6418796)(70.82598485,777.22290526)(71.12185867,777.78464011)
\curveto(71.41773054,778.34636446)(71.83581173,778.77516568)(72.3761035,779.07104506)
\curveto(72.92067882,779.36691137)(73.52743255,779.51484779)(74.1963665,779.51485477)
\curveto(75.2297734,779.51484779)(76.07665582,779.17823883)(76.73701628,778.50502688)
\curveto(77.39736358,777.836091)(77.72754052,776.98920859)(77.7275481,775.9643771)
\curveto(77.72754052,774.93096272)(77.39307557,774.07336028)(76.72415223,773.3915672)
\curveto(76.05950377,772.71406041)(75.22119738,772.37530744)(74.20923055,772.37530729)
\curveto(73.58317671,772.37530744)(72.98499901,772.51681184)(72.41469564,772.79982092)
\curveto(71.84867577,773.08282946)(71.41773054,773.49662264)(71.12185867,774.0412017)
\curveto(70.82598485,774.59006575)(70.67804843,775.25685165)(70.67804896,776.0415614)
\moveto(72.53047209,775.94508103)
\curveto(72.53046971,775.26757168)(72.69127017,774.7487222)(73.01287395,774.38853104)
\curveto(73.334472,774.02833615)(73.73111313,773.84823964)(74.20279853,773.84824096)
\curveto(74.67447582,773.84823964)(75.06897294,774.02833615)(75.38629108,774.38853104)
\curveto(75.70788676,774.7487222)(75.86868722,775.2718597)(75.86869294,775.95794508)
\curveto(75.86868722,776.62687156)(75.70788676,777.14143302)(75.38629108,777.50163102)
\curveto(75.06897294,777.86181908)(74.67447582,778.04191559)(74.20279853,778.0419211)
\curveto(73.73111313,778.04191559)(73.334472,777.86181908)(73.01287395,777.50163102)
\curveto(72.69127017,777.14143302)(72.53046971,776.62258354)(72.53047209,775.94508103)
}
}
{
\newrgbcolor{curcolor}{0 0 0}
\pscustom[linestyle=none,fillstyle=solid,fillcolor=curcolor]
{
\newpath
\moveto(79.14902564,772.52967588)
\lineto(79.14902564,781.95902418)
\lineto(80.95642459,781.95902418)
\lineto(80.95642459,772.52967588)
\lineto(79.14902564,772.52967588)
}
}
{
\newrgbcolor{curcolor}{0 0 0}
\pscustom[linestyle=none,fillstyle=solid,fillcolor=curcolor]
{
\newpath
\moveto(86.75810984,774.70370025)
\lineto(88.55907678,774.40139509)
\curveto(88.32751741,773.74103933)(87.96089237,773.2371979)(87.45920054,772.88986927)
\curveto(86.96178552,772.54682793)(86.33787974,772.37530744)(85.58748134,772.37530729)
\curveto(84.39969822,772.37530744)(83.52065572,772.76337255)(82.95035118,773.53950377)
\curveto(82.50010881,774.16126453)(82.27498817,774.94597077)(82.27498858,775.89362483)
\curveto(82.27498817,777.02565669)(82.57086101,777.91113122)(83.162608,778.55005106)
\curveto(83.75435238,779.19324687)(84.50261051,779.51484779)(85.40738464,779.51485477)
\curveto(86.42363999,779.51484779)(87.22549827,779.17823883)(87.8129619,778.50502688)
\curveto(88.40041362,777.836091)(88.68127842,776.80911208)(88.65555715,775.42408702)
\lineto(84.12741171,775.42408702)
\curveto(84.14027348,774.8880826)(84.2860659,774.47000141)(84.5647894,774.1698422)
\curveto(84.84350749,773.87396771)(85.19083647,773.72603129)(85.60677741,773.72603249)
\curveto(85.88978247,773.72603129)(86.12776715,773.80321551)(86.32073216,773.95758538)
\curveto(86.51368824,774.11195239)(86.65948066,774.3606571)(86.75810984,774.70370025)
\moveto(86.86102224,776.53039528)
\curveto(86.8481532,777.05352877)(86.71308081,777.4501699)(86.45580468,777.72031986)
\curveto(86.19851935,777.99474746)(85.88549445,778.13196385)(85.51672906,778.13196945)
\curveto(85.12222828,778.13196385)(84.79633935,777.98831544)(84.5390613,777.70102379)
\curveto(84.28177788,777.4137218)(84.15528152,777.02351269)(84.15957184,776.53039528)
\lineto(86.86102224,776.53039528)
}
}
{
\newrgbcolor{curcolor}{0 0 0}
\pscustom[linestyle=none,fillstyle=solid,fillcolor=curcolor]
{
\newpath
\moveto(89.49815225,774.47857938)
\lineto(91.31198323,774.75515645)
\curveto(91.38916533,774.40353722)(91.54567778,774.13553646)(91.78152104,773.95115335)
\curveto(92.01735912,773.77105542)(92.34753606,773.68100716)(92.77205285,773.68100831)
\curveto(93.2394426,773.68100716)(93.5910596,773.76676741)(93.82690491,773.9382893)
\curveto(93.98555673,774.05835224)(94.06488496,774.2191527)(94.06488983,774.42069116)
\curveto(94.06488496,774.55790566)(94.02200483,774.67153798)(93.93624934,774.76158847)
\curveto(93.84619633,774.84734649)(93.64465976,774.92667471)(93.33163901,774.99957339)
\curveto(91.87371071,775.32117184)(90.94964408,775.61490067)(90.55943634,775.88076078)
\curveto(90.01914543,776.24952648)(89.74900066,776.76194394)(89.74900122,777.4180147)
\curveto(89.74900066,778.0097555)(89.98269732,778.50716492)(90.45009191,778.91024444)
\curveto(90.91748399,779.31331121)(91.64215805,779.51484779)(92.62411628,779.51485477)
\curveto(93.55889951,779.51484779)(94.25355749,779.36262335)(94.70809231,779.05818101)
\curveto(95.16261608,778.75372562)(95.47564098,778.30348434)(95.64716792,777.70745581)
\lineto(93.94268136,777.3922866)
\curveto(93.8697804,777.6581385)(93.73042,777.86181908)(93.52459975,778.00332895)
\curveto(93.32305884,778.14482788)(93.03361802,778.21558008)(92.65627641,778.21558577)
\curveto(92.18030358,778.21558008)(91.83940661,778.1491159)(91.63358447,778.016193)
\curveto(91.49636564,777.92185125)(91.42775744,777.7996429)(91.42775968,777.64956759)
\curveto(91.42775744,777.5209221)(91.48778961,777.41157779)(91.60785637,777.32153433)
\curveto(91.77079842,777.20146519)(92.33252802,777.03208871)(93.29304686,776.81340437)
\curveto(94.25784551,776.59471147)(94.93106342,776.3267107)(95.31270263,776.00940128)
\curveto(95.69004159,775.68779688)(95.87871412,775.2396996)(95.87872081,774.6651081)
\curveto(95.87871412,774.03905618)(95.61714538,773.50091065)(95.09401379,773.05066989)
\curveto(94.5708704,772.60042808)(93.79688419,772.37530744)(92.77205285,772.37530729)
\curveto(91.84155062,772.37530744)(91.10401252,772.56397998)(90.55943634,772.94132547)
\curveto(90.01914543,773.31867013)(89.66538442,773.83108759)(89.49815225,774.47857938)
}
}
{
\newrgbcolor{curcolor}{0.50196081 0 1}
\pscustom[linewidth=2.63455725,linecolor=curcolor]
{
\newpath
\moveto(37.36788,642.91493486)
\lineto(122.991,642.91493486)
\lineto(122.991,695.60607486)
\lineto(65.03074,695.60607486)
\lineto(65.03074,706.14430486)
\lineto(37.36788,706.14430486)
\lineto(37.36788,642.91493486)
\closepath
}
}
{
\newrgbcolor{curcolor}{0 0 0}
\pscustom[linestyle=none,fillstyle=solid,fillcolor=curcolor]
{
\newpath
\moveto(44.67465627,677.00716215)
\lineto(44.67465627,678.08774231)
\lineto(50.92015231,680.72487246)
\lineto(50.92015231,679.57354003)
\lineto(45.96749324,677.5410202)
\lineto(50.92015231,675.4892043)
\lineto(50.92015231,674.33787187)
\lineto(44.67465627,677.00716215)
}
}
{
\newrgbcolor{curcolor}{0 0 0}
\pscustom[linestyle=none,fillstyle=solid,fillcolor=curcolor]
{
\newpath
\moveto(52.36735808,677.00716215)
\lineto(52.36735808,678.08774231)
\lineto(58.61285413,680.72487246)
\lineto(58.61285413,679.57354003)
\lineto(53.66019506,677.5410202)
\lineto(58.61285413,675.4892043)
\lineto(58.61285413,674.33787187)
\lineto(52.36735808,677.00716215)
}
}
{
\newrgbcolor{curcolor}{0 0 0}
\pscustom[linestyle=none,fillstyle=solid,fillcolor=curcolor]
{
\newpath
\moveto(60.20799647,672.88423427)
\lineto(60.20799647,679.71504457)
\lineto(61.24355245,679.71504457)
\lineto(61.24355245,678.75667288)
\curveto(61.45795116,679.09113196)(61.74310397,679.35913272)(62.09901175,679.56067598)
\curveto(62.454914,679.76649389)(62.86013116,679.86940618)(63.31466443,679.86941316)
\curveto(63.8206459,679.86940618)(64.23443907,679.76434988)(64.55604521,679.55424395)
\curveto(64.88192892,679.34412468)(65.11133757,679.05039585)(65.24427186,678.67305656)
\curveto(65.78455549,679.47062104)(66.4877895,679.86940618)(67.35397598,679.86941316)
\curveto(68.0314739,679.86940618)(68.55246738,679.68073364)(68.916958,679.30339499)
\curveto(69.28142946,678.9303315)(69.46366998,678.35359386)(69.4636801,677.57318032)
\lineto(69.4636801,672.88423427)
\lineto(68.31234767,672.88423427)
\lineto(68.31234767,677.18725884)
\curveto(68.3123387,677.65035986)(68.27374659,677.9826808)(68.19657122,678.18422268)
\curveto(68.12366616,678.39004196)(67.98859377,678.55513043)(67.79135366,678.67948858)
\curveto(67.59409665,678.80383514)(67.36254399,678.86601132)(67.09669499,678.8660173)
\curveto(66.61642986,678.86601132)(66.21764473,678.70521086)(65.90033838,678.38361544)
\curveto(65.58301892,678.06629704)(65.42436247,677.55602359)(65.42436855,676.85279355)
\lineto(65.42436855,672.88423427)
\lineto(64.26660409,672.88423427)
\lineto(64.26660409,677.32233136)
\curveto(64.26659917,677.83688839)(64.1722629,678.22280949)(63.983595,678.48009582)
\curveto(63.79491782,678.73737095)(63.48618094,678.86601132)(63.05738344,678.8660173)
\curveto(62.73149079,678.86601132)(62.42918593,678.78025108)(62.15046795,678.60873631)
\curveto(61.87603235,678.4372101)(61.67663979,678.18636138)(61.55228964,677.85618941)
\curveto(61.42793508,677.5260075)(61.3657589,677.05003814)(61.36576093,676.42827992)
\lineto(61.36576093,672.88423427)
\lineto(60.20799647,672.88423427)
}
}
{
\newrgbcolor{curcolor}{0 0 0}
\pscustom[linestyle=none,fillstyle=solid,fillcolor=curcolor]
{
\newpath
\moveto(71.22605474,680.98215345)
\lineto(71.22605474,682.31358257)
\lineto(72.38381919,682.31358257)
\lineto(72.38381919,680.98215345)
\lineto(71.22605474,680.98215345)
\moveto(71.22605474,672.88423427)
\lineto(71.22605474,679.71504457)
\lineto(72.38381919,679.71504457)
\lineto(72.38381919,672.88423427)
\lineto(71.22605474,672.88423427)
}
}
{
\newrgbcolor{curcolor}{0 0 0}
\pscustom[linestyle=none,fillstyle=solid,fillcolor=curcolor]
{
\newpath
\moveto(78.58429024,672.88423427)
\lineto(78.58429024,673.74612559)
\curveto(78.15119571,673.0686188)(77.5144259,672.72986583)(76.67397889,672.72986568)
\curveto(76.12939795,672.72986583)(75.62770052,672.87994626)(75.1688851,673.18010741)
\curveto(74.71435392,673.48026797)(74.36059291,673.89834916)(74.10760101,674.43435224)
\curveto(73.85889548,674.97464023)(73.73454312,675.594258)(73.73454357,676.2932074)
\curveto(73.73454312,676.97499793)(73.84817545,677.59247169)(74.07544089,678.14563053)
\curveto(74.30270474,678.70306686)(74.64360171,679.12972407)(75.09813282,679.42560346)
\curveto(75.55266031,679.72146976)(76.06078975,679.86940618)(76.62252269,679.86941316)
\curveto(77.03416853,679.86940618)(77.40079357,679.78150193)(77.72239893,679.60570015)
\curveto(78.04399541,679.43417294)(78.30556415,679.2090523)(78.50710595,678.93033755)
\lineto(78.50710595,682.31358257)
\lineto(79.65843838,682.31358257)
\lineto(79.65843838,672.88423427)
\lineto(78.58429024,672.88423427)
\moveto(74.92446816,676.2932074)
\curveto(74.92446651,675.41844949)(75.10885104,674.76452763)(75.47762228,674.33143984)
\curveto(75.84638914,673.89834916)(76.28162238,673.68180455)(76.78332331,673.68180534)
\curveto(77.28930525,673.68180455)(77.71810648,673.88762913)(78.06972826,674.29927972)
\curveto(78.42562849,674.71521549)(78.603581,675.34769729)(78.60358632,676.19672703)
\curveto(78.603581,677.13151038)(78.42348449,677.81759233)(78.06329624,678.25497495)
\curveto(77.70309843,678.69234682)(77.25928917,678.91103545)(76.73186711,678.91104147)
\curveto(76.2173022,678.91103545)(75.78635697,678.70092285)(75.43903014,678.28070305)
\curveto(75.095987,677.86047245)(74.92446651,677.19797457)(74.92446816,676.2932074)
}
}
{
\newrgbcolor{curcolor}{0 0 0}
\pscustom[linestyle=none,fillstyle=solid,fillcolor=curcolor]
{
\newpath
\moveto(85.91679833,672.88423427)
\lineto(85.91679833,673.74612559)
\curveto(85.4837038,673.0686188)(84.84693398,672.72986583)(84.00648698,672.72986568)
\curveto(83.46190604,672.72986583)(82.96020861,672.87994626)(82.50139318,673.18010741)
\curveto(82.046862,673.48026797)(81.693101,673.89834916)(81.4401091,674.43435224)
\curveto(81.19140357,674.97464023)(81.06705121,675.594258)(81.06705166,676.2932074)
\curveto(81.06705121,676.97499793)(81.18068353,677.59247169)(81.40794897,678.14563053)
\curveto(81.63521283,678.70306686)(81.9761098,679.12972407)(82.43064091,679.42560346)
\curveto(82.88516839,679.72146976)(83.39329784,679.86940618)(83.95503078,679.86941316)
\curveto(84.36667661,679.86940618)(84.73330166,679.78150193)(85.05490701,679.60570015)
\curveto(85.37650349,679.43417294)(85.63807224,679.2090523)(85.83961403,678.93033755)
\lineto(85.83961403,682.31358257)
\lineto(86.99094647,682.31358257)
\lineto(86.99094647,672.88423427)
\lineto(85.91679833,672.88423427)
\moveto(82.25697624,676.2932074)
\curveto(82.2569746,675.41844949)(82.44135913,674.76452763)(82.81013037,674.33143984)
\curveto(83.17889723,673.89834916)(83.61413047,673.68180455)(84.1158314,673.68180534)
\curveto(84.62181334,673.68180455)(85.05061456,673.88762913)(85.40223635,674.29927972)
\curveto(85.75813658,674.71521549)(85.93608909,675.34769729)(85.93609441,676.19672703)
\curveto(85.93608909,677.13151038)(85.75599257,677.81759233)(85.39580433,678.25497495)
\curveto(85.03560652,678.69234682)(84.59179726,678.91103545)(84.0643752,678.91104147)
\curveto(83.54981029,678.91103545)(83.11886506,678.70092285)(82.77153822,678.28070305)
\curveto(82.42849509,677.86047245)(82.2569746,677.19797457)(82.25697624,676.2932074)
}
}
{
\newrgbcolor{curcolor}{0 0 0}
\pscustom[linestyle=none,fillstyle=solid,fillcolor=curcolor]
{
\newpath
\moveto(88.79191326,672.88423427)
\lineto(88.79191326,682.31358257)
\lineto(89.94967772,682.31358257)
\lineto(89.94967772,672.88423427)
\lineto(88.79191326,672.88423427)
}
}
{
\newrgbcolor{curcolor}{0 0 0}
\pscustom[linestyle=none,fillstyle=solid,fillcolor=curcolor]
{
\newpath
\moveto(96.42672583,675.08398674)
\lineto(97.62308244,674.93605017)
\curveto(97.43440316,674.23710213)(97.08493016,673.69466858)(96.5746624,673.30874791)
\curveto(96.06438325,672.92282638)(95.4126054,672.72986583)(94.61932687,672.72986568)
\curveto(93.62021629,672.72986583)(92.82693403,673.03645871)(92.23947771,673.64964522)
\curveto(91.65630669,674.26711821)(91.36472186,675.13115268)(91.36472234,676.2417512)
\curveto(91.36472186,677.39093512)(91.6605947,678.28284166)(92.25234176,678.9174735)
\curveto(92.84408608,679.55209327)(93.61164027,679.86940618)(94.55500663,679.86941316)
\curveto(95.46834956,679.86940618)(96.21446368,679.55852529)(96.79335124,678.93676957)
\curveto(97.37222698,678.31500175)(97.66166781,677.44024726)(97.66167459,676.31250347)
\curveto(97.66166781,676.24389185)(97.6595238,676.14097955)(97.65524256,676.00376628)
\lineto(92.56107895,676.00376628)
\curveto(92.60395739,675.25336102)(92.816214,674.67876739)(93.1978494,674.27998365)
\curveto(93.57948017,673.88119711)(94.05544953,673.68180455)(94.6257589,673.68180534)
\curveto(95.05026836,673.68180455)(95.4126054,673.79329286)(95.71277108,674.01627063)
\curveto(96.01292711,674.23924613)(96.25091179,674.59515115)(96.42672583,675.08398674)
\moveto(92.6253992,676.95570595)
\lineto(96.43958988,676.95570595)
\curveto(96.38812818,677.53029551)(96.24233576,677.96124074)(96.0022122,678.24854292)
\curveto(95.63343803,678.69449083)(95.15532466,678.91746747)(94.56787068,678.9174735)
\curveto(94.03615347,678.91746747)(93.5880562,678.73951496)(93.2235775,678.38361544)
\curveto(92.86338213,678.02770493)(92.66398956,677.55173557)(92.6253992,676.95570595)
}
}
{
\newrgbcolor{curcolor}{0 0 0}
\pscustom[linestyle=none,fillstyle=solid,fillcolor=curcolor]
{
\newpath
\moveto(105.18071139,677.00716215)
\lineto(98.93521535,674.33787187)
\lineto(98.93521535,675.4892043)
\lineto(103.88144239,677.5410202)
\lineto(98.93521535,679.57354003)
\lineto(98.93521535,680.72487246)
\lineto(105.18071139,678.08774231)
\lineto(105.18071139,677.00716215)
}
}
{
\newrgbcolor{curcolor}{0 0 0}
\pscustom[linestyle=none,fillstyle=solid,fillcolor=curcolor]
{
\newpath
\moveto(112.87341464,677.00716215)
\lineto(106.62791859,674.33787187)
\lineto(106.62791859,675.4892043)
\lineto(111.57414564,677.5410202)
\lineto(106.62791859,679.57354003)
\lineto(106.62791859,680.72487246)
\lineto(112.87341464,678.08774231)
\lineto(112.87341464,677.00716215)
}
}
{
\newrgbcolor{curcolor}{0 0 0}
\pscustom[linestyle=none,fillstyle=solid,fillcolor=curcolor]
{
\newpath
\moveto(67.83768749,653.12504848)
\lineto(67.83768749,654.14774042)
\curveto(67.58897734,653.78325836)(67.2609444,653.49596154)(66.8535877,653.2858491)
\curveto(66.4505101,653.07573634)(66.02385288,652.97068004)(65.57361477,652.97067989)
\curveto(65.11479429,652.97068004)(64.70314512,653.07144833)(64.33866602,653.27298505)
\curveto(63.97418304,653.47452148)(63.71047029,653.75753028)(63.54752697,654.12201232)
\curveto(63.38458136,654.48649236)(63.30310913,654.9903338)(63.30311003,655.63353814)
\lineto(63.30311003,659.95585878)
\lineto(65.11050899,659.95585878)
\lineto(65.11050899,656.8170307)
\curveto(65.11050628,655.85651227)(65.14266637,655.26691059)(65.20698936,655.04822389)
\curveto(65.27559475,654.83382135)(65.3978031,654.66230086)(65.57361477,654.5336619)
\curveto(65.7494201,654.40930814)(65.97239673,654.34713196)(66.24254535,654.34713319)
\curveto(66.55127838,654.34713196)(66.82785517,654.4307482)(67.07227654,654.59798215)
\curveto(67.31668856,654.76950117)(67.48392104,654.97961377)(67.57397447,655.22832058)
\curveto(67.66401755,655.4813112)(67.70904168,656.09664095)(67.70904699,657.07431169)
\lineto(67.70904699,659.95585878)
\lineto(69.51644595,659.95585878)
\lineto(69.51644595,653.12504848)
\lineto(67.83768749,653.12504848)
}
}
{
\newrgbcolor{curcolor}{0 0 0}
\pscustom[linestyle=none,fillstyle=solid,fillcolor=curcolor]
{
\newpath
\moveto(70.75782632,655.07395198)
\lineto(72.5716573,655.35052905)
\curveto(72.6488394,654.99890982)(72.80535185,654.73090906)(73.04119511,654.54652595)
\curveto(73.27703319,654.36642802)(73.60721013,654.27637976)(74.03172693,654.27638091)
\curveto(74.49911668,654.27637976)(74.85073368,654.36214001)(75.08657899,654.5336619)
\curveto(75.2452308,654.65372484)(75.32455903,654.8145253)(75.3245639,655.01606376)
\curveto(75.32455903,655.15327826)(75.28167891,655.26691059)(75.19592341,655.35696107)
\curveto(75.1058704,655.44271909)(74.90433383,655.52204731)(74.59131308,655.59494599)
\curveto(73.13338478,655.91654444)(72.20931815,656.21027327)(71.81911041,656.47613338)
\curveto(71.2788195,656.84489908)(71.00867473,657.35731654)(71.00867529,658.0133873)
\curveto(71.00867473,658.6051281)(71.24237139,659.10253752)(71.70976599,659.50561705)
\curveto(72.17715806,659.90868381)(72.90183212,660.11022039)(73.88379036,660.11022737)
\curveto(74.81857359,660.11022039)(75.51323157,659.95799595)(75.96776638,659.65355362)
\curveto(76.42229016,659.34909822)(76.73531505,658.89885694)(76.90684199,658.30282842)
\lineto(75.20235543,657.9876592)
\curveto(75.12945447,658.2535111)(74.99009407,658.45719168)(74.78427382,658.59870155)
\curveto(74.58273291,658.74020048)(74.29329209,658.81095269)(73.91595048,658.81095837)
\curveto(73.43997766,658.81095269)(73.09908069,658.7444885)(72.89325854,658.6115656)
\curveto(72.75603971,658.51722385)(72.68743151,658.3950155)(72.68743375,658.24494019)
\curveto(72.68743151,658.11629471)(72.74746368,658.00695039)(72.86753044,657.91690693)
\curveto(73.03047249,657.7968378)(73.59220209,657.62746131)(74.55272093,657.40877697)
\curveto(75.51751958,657.19008407)(76.1907375,656.9220833)(76.57237671,656.60477388)
\curveto(76.94971566,656.28316948)(77.1383882,655.8350722)(77.13839489,655.2604807)
\curveto(77.1383882,654.63442878)(76.87681945,654.09628325)(76.35368786,653.64604249)
\curveto(75.83054447,653.19580068)(75.05655826,652.97068004)(74.03172693,652.97067989)
\curveto(73.10122469,652.97068004)(72.36368659,653.15935258)(71.81911041,653.53669807)
\curveto(71.2788195,653.91404273)(70.92505849,654.42646019)(70.75782632,655.07395198)
}
}
{
\newrgbcolor{curcolor}{0 0 0}
\pscustom[linestyle=none,fillstyle=solid,fillcolor=curcolor]
{
\newpath
\moveto(82.68280009,655.29907285)
\lineto(84.48376702,654.99676769)
\curveto(84.25220766,654.33641193)(83.88558262,653.8325705)(83.38389079,653.48524187)
\curveto(82.88647577,653.14220053)(82.26256999,652.97068004)(81.51217158,652.97067989)
\curveto(80.32438847,652.97068004)(79.44534596,653.35874515)(78.87504143,654.13487637)
\curveto(78.42479905,654.75663713)(78.19967841,655.54134337)(78.19967883,656.48899743)
\curveto(78.19967841,657.62102929)(78.49555126,658.50650382)(79.08729825,659.14542366)
\curveto(79.67904263,659.78861947)(80.42730076,660.11022039)(81.33207489,660.11022737)
\curveto(82.34833023,660.11022039)(83.15018852,659.77361143)(83.73765215,659.10039949)
\curveto(84.32510387,658.4314636)(84.60596867,657.40448468)(84.58024739,656.01945962)
\lineto(80.05210196,656.01945962)
\curveto(80.06496373,655.4834552)(80.21075614,655.06537401)(80.48947965,654.7652148)
\curveto(80.76819773,654.46934031)(81.11552672,654.32140389)(81.53146766,654.32140509)
\curveto(81.81447271,654.32140389)(82.05245739,654.39858811)(82.24542241,654.55295798)
\curveto(82.43837849,654.70732499)(82.58417091,654.9560297)(82.68280009,655.29907285)
\moveto(82.78571249,657.12576788)
\curveto(82.77284344,657.64890137)(82.63777106,658.0455425)(82.38049493,658.31569246)
\curveto(82.12320959,658.59012006)(81.8101847,658.72733645)(81.44141931,658.72734205)
\curveto(81.04691853,658.72733645)(80.7210296,658.58368804)(80.46375155,658.29639639)
\curveto(80.20646813,658.0090944)(80.07997177,657.61888529)(80.08426209,657.12576788)
\lineto(82.78571249,657.12576788)
}
}
{
\newrgbcolor{curcolor}{0 0 0}
\pscustom[linestyle=none,fillstyle=solid,fillcolor=curcolor]
{
\newpath
\moveto(87.78982761,653.12504848)
\lineto(85.98242865,653.12504848)
\lineto(85.98242865,659.95585878)
\lineto(87.66118711,659.95585878)
\lineto(87.66118711,658.98462304)
\curveto(87.94848139,659.44343449)(88.20576212,659.74573935)(88.43303008,659.89153853)
\curveto(88.66457943,660.03732418)(88.92614817,660.11022039)(89.21773711,660.11022737)
\curveto(89.62938218,660.11022039)(90.02602331,659.99658806)(90.40766169,659.76933006)
\lineto(89.84807553,658.19348399)
\curveto(89.54362193,658.39072749)(89.26061312,658.48935177)(88.99904826,658.48935713)
\curveto(88.74605166,658.48935177)(88.53165105,658.41859957)(88.35584579,658.27710032)
\curveto(88.18003404,658.13987877)(88.04067365,657.88903006)(87.93776418,657.52455342)
\curveto(87.83913707,657.16006798)(87.78982493,656.39680181)(87.78982761,655.2347526)
\lineto(87.78982761,653.12504848)
}
}
{
\newrgbcolor{curcolor}{0 0 0}
\pscustom[linestyle=none,fillstyle=solid,fillcolor=curcolor]
{
\newpath
\moveto(90.54273462,655.07395198)
\lineto(92.3565656,655.35052905)
\curveto(92.4337477,654.99890982)(92.59026014,654.73090906)(92.82610341,654.54652595)
\curveto(93.06194149,654.36642802)(93.39211843,654.27637976)(93.81663522,654.27638091)
\curveto(94.28402497,654.27637976)(94.63564197,654.36214001)(94.87148728,654.5336619)
\curveto(95.0301391,654.65372484)(95.10946732,654.8145253)(95.1094722,655.01606376)
\curveto(95.10946732,655.15327826)(95.0665872,655.26691059)(94.9808317,655.35696107)
\curveto(94.8907787,655.44271909)(94.68924212,655.52204731)(94.37622137,655.59494599)
\curveto(92.91829308,655.91654444)(91.99422644,656.21027327)(91.6040187,656.47613338)
\curveto(91.06372779,656.84489908)(90.79358302,657.35731654)(90.79358358,658.0133873)
\curveto(90.79358302,658.6051281)(91.02727969,659.10253752)(91.49467428,659.50561705)
\curveto(91.96206635,659.90868381)(92.68674042,660.11022039)(93.66869865,660.11022737)
\curveto(94.60348188,660.11022039)(95.29813986,659.95799595)(95.75267467,659.65355362)
\curveto(96.20719845,659.34909822)(96.52022334,658.89885694)(96.69175029,658.30282842)
\lineto(94.98726373,657.9876592)
\curveto(94.91436277,658.2535111)(94.77500237,658.45719168)(94.56918212,658.59870155)
\curveto(94.36764121,658.74020048)(94.07820038,658.81095269)(93.70085877,658.81095837)
\curveto(93.22488595,658.81095269)(92.88398898,658.7444885)(92.67816684,658.6115656)
\curveto(92.540948,658.51722385)(92.47233981,658.3950155)(92.47234205,658.24494019)
\curveto(92.47233981,658.11629471)(92.53237198,658.00695039)(92.65243874,657.91690693)
\curveto(92.81538078,657.7968378)(93.37711039,657.62746131)(94.33762923,657.40877697)
\curveto(95.30242787,657.19008407)(95.97564579,656.9220833)(96.357285,656.60477388)
\curveto(96.73462395,656.28316948)(96.92329649,655.8350722)(96.92330318,655.2604807)
\curveto(96.92329649,654.63442878)(96.66172775,654.09628325)(96.13859616,653.64604249)
\curveto(95.61545276,653.19580068)(94.84146656,652.97068004)(93.81663522,652.97067989)
\curveto(92.88613299,652.97068004)(92.14859488,653.15935258)(91.6040187,653.53669807)
\curveto(91.06372779,653.91404273)(90.70996678,654.42646019)(90.54273462,655.07395198)
}
}
{
\newrgbcolor{curcolor}{0 1 0.25098041}
\pscustom[linewidth=2.63455725,linecolor=curcolor]
{
\newpath
\moveto(37.36788,523.51033456)
\lineto(121.67372,523.51033456)
\lineto(121.67372,576.20147456)
\lineto(65.03074,576.20147456)
\lineto(65.03074,586.73970456)
\lineto(37.36788,586.73970456)
\lineto(37.36788,523.51033456)
\closepath
}
}
{
\newrgbcolor{curcolor}{0 0 0}
\pscustom[linestyle=none,fillstyle=solid,fillcolor=curcolor]
{
\newpath
\moveto(52.57832353,559.19127215)
\lineto(52.57832353,560.27185231)
\lineto(58.82381958,562.90898246)
\lineto(58.82381958,561.75765003)
\lineto(53.87116051,559.7251302)
\lineto(58.82381958,557.6733143)
\lineto(58.82381958,556.52198187)
\lineto(52.57832353,559.19127215)
}
}
{
\newrgbcolor{curcolor}{0 0 0}
\pscustom[linestyle=none,fillstyle=solid,fillcolor=curcolor]
{
\newpath
\moveto(60.27102535,559.19127215)
\lineto(60.27102535,560.27185231)
\lineto(66.51652139,562.90898246)
\lineto(66.51652139,561.75765003)
\lineto(61.56386232,559.7251302)
\lineto(66.51652139,557.6733143)
\lineto(66.51652139,556.52198187)
\lineto(60.27102535,559.19127215)
}
}
{
\newrgbcolor{curcolor}{0 0 0}
\pscustom[linestyle=none,fillstyle=solid,fillcolor=curcolor]
{
\newpath
\moveto(72.56905689,555.91093952)
\curveto(72.14025035,555.54645764)(71.72645717,555.2891769)(71.32767611,555.13909655)
\curveto(70.93317491,554.98901605)(70.5086617,554.91397583)(70.05413521,554.91397568)
\curveto(69.30373026,554.91397583)(68.72699262,555.09621635)(68.32392055,555.46069778)
\curveto(67.92084632,555.82946644)(67.71930975,556.29900378)(67.71931022,556.86931121)
\curveto(67.71930975,557.20377436)(67.79434996,557.50822323)(67.94443109,557.78265872)
\curveto(68.09879883,558.0613768)(68.2981914,558.28435344)(68.54260939,558.4515893)
\curveto(68.7913128,558.61881839)(69.0700336,558.74531475)(69.37877261,558.83107876)
\curveto(69.60603512,558.89110717)(69.9490761,558.94899533)(70.40789657,559.00474343)
\curveto(71.34268007,559.11622781)(72.03090603,559.24915619)(72.47257652,559.40352896)
\curveto(72.47685931,559.56218108)(72.47900331,559.66294937)(72.47900855,559.70583413)
\curveto(72.47900331,560.17751083)(72.369659,560.50983178)(72.15097528,560.70279796)
\curveto(71.85509753,560.96436108)(71.41557628,561.09514545)(70.83241021,561.09515148)
\curveto(70.28782907,561.09514545)(69.88475592,560.99866517)(69.62318955,560.80571036)
\curveto(69.36590644,560.61703209)(69.1750899,560.28042313)(69.05073935,559.79588247)
\lineto(67.91870299,559.95025107)
\curveto(68.02161461,560.43479157)(68.19099109,560.82500068)(68.42683295,561.12087957)
\curveto(68.66267244,561.42103438)(69.00356941,561.65044303)(69.44952488,561.80910622)
\curveto(69.89547595,561.97204395)(70.41218142,562.05351618)(70.99964285,562.05352317)
\curveto(71.58280876,562.05351618)(72.05663411,561.98490798)(72.42112032,561.84769837)
\curveto(72.78559619,561.7104752)(73.05359695,561.53681071)(73.22512342,561.32670437)
\curveto(73.39663793,561.12087352)(73.51670227,560.85930478)(73.58531681,560.54199735)
\curveto(73.62390257,560.34474331)(73.64319863,559.9888383)(73.64320503,559.47428124)
\lineto(73.64320503,557.93059529)
\curveto(73.64319863,556.85430136)(73.6667827,556.17250742)(73.7139573,555.88521142)
\curveto(73.76540698,555.60220179)(73.86403126,555.32991302)(74.00983044,555.06834427)
\lineto(72.80060979,555.06834427)
\curveto(72.68053989,555.30847296)(72.60335567,555.58933776)(72.56905689,555.91093952)
\moveto(72.47257652,558.49661347)
\curveto(72.0523461,558.32508955)(71.4220083,558.17929714)(70.58156124,558.05923579)
\curveto(70.10558855,557.9906246)(69.76897959,557.91344038)(69.57173335,557.8276829)
\curveto(69.37448246,557.74191989)(69.22225803,557.61542353)(69.1150596,557.44819344)
\curveto(69.00785742,557.28524659)(68.95425727,557.10300607)(68.95425898,556.90147133)
\curveto(68.95425727,556.59273262)(69.0700336,556.33545188)(69.30158831,556.12962836)
\curveto(69.53742693,555.92380271)(69.88046791,555.82089042)(70.33071228,555.82089117)
\curveto(70.77666246,555.82089042)(71.17330359,555.91737069)(71.52063686,556.11033228)
\curveto(71.86796157,556.30757981)(72.1230983,556.57558057)(72.2860478,556.91433538)
\curveto(72.41039512,557.17590228)(72.47257129,557.56182338)(72.47257652,558.07209984)
\lineto(72.47257652,558.49661347)
}
}
{
\newrgbcolor{curcolor}{0 0 0}
\pscustom[linestyle=none,fillstyle=solid,fillcolor=curcolor]
{
\newpath
\moveto(75.44417087,552.4505102)
\lineto(75.44417087,561.89915457)
\lineto(76.49902293,561.89915457)
\lineto(76.49902293,561.01153515)
\curveto(76.74772571,561.3588582)(77.02859051,561.61828294)(77.34161817,561.78981015)
\curveto(77.6546403,561.96561193)(78.03412938,562.05351618)(78.48008655,562.05352317)
\curveto(79.06325231,562.05351618)(79.57781378,561.90343575)(80.0237725,561.60328143)
\curveto(80.46972032,561.30311404)(80.80632928,560.87860083)(81.03360039,560.32974053)
\curveto(81.26085858,559.78515772)(81.3744909,559.18698001)(81.3744977,558.53520562)
\curveto(81.3744909,557.83625616)(81.24799454,557.20591837)(80.99500824,556.64419034)
\curveto(80.74629711,556.08674718)(80.38181607,555.65794595)(79.90156403,555.35778539)
\curveto(79.42558935,555.06191226)(78.92389192,554.91397583)(78.39647023,554.91397568)
\curveto(78.01054531,554.91397583)(77.66321632,554.99544807)(77.35448222,555.15839262)
\curveto(77.05003058,555.32133699)(76.79918186,555.52716158)(76.60193532,555.775867)
\lineto(76.60193532,552.4505102)
\lineto(75.44417087,552.4505102)
\moveto(76.4925909,558.44515727)
\curveto(76.49258899,557.56611139)(76.67054149,556.91647754)(77.02644896,556.49625377)
\curveto(77.38235152,556.07602715)(77.81329675,555.86591455)(78.31928594,555.86591534)
\curveto(78.83384366,555.86591455)(79.27336491,556.08245916)(79.63785101,556.51554984)
\curveto(80.006615,556.95292564)(80.19099953,557.62828757)(80.19100514,558.54163764)
\curveto(80.19099953,559.41210065)(80.01090301,560.06387851)(79.65071506,560.49697317)
\curveto(79.29480497,560.93005698)(78.86814776,561.1466016)(78.37074213,561.14660767)
\curveto(77.87761693,561.1466016)(77.44023969,560.91504894)(77.05860908,560.451949)
\curveto(76.68126152,559.99312631)(76.49258899,559.3241964)(76.4925909,558.44515727)
}
}
{
\newrgbcolor{curcolor}{0 0 0}
\pscustom[linestyle=none,fillstyle=solid,fillcolor=curcolor]
{
\newpath
\moveto(82.77667895,552.4505102)
\lineto(82.77667895,561.89915457)
\lineto(83.83153102,561.89915457)
\lineto(83.83153102,561.01153515)
\curveto(84.0802338,561.3588582)(84.3610986,561.61828294)(84.67412626,561.78981015)
\curveto(84.98714839,561.96561193)(85.36663747,562.05351618)(85.81259464,562.05352317)
\curveto(86.3957604,562.05351618)(86.91032187,561.90343575)(87.35628059,561.60328143)
\curveto(87.80222841,561.30311404)(88.13883737,560.87860083)(88.36610847,560.32974053)
\curveto(88.59336666,559.78515772)(88.70699899,559.18698001)(88.70700579,558.53520562)
\curveto(88.70699899,557.83625616)(88.58050263,557.20591837)(88.32751632,556.64419034)
\curveto(88.0788052,556.08674718)(87.71432416,555.65794595)(87.23407211,555.35778539)
\curveto(86.75809743,555.06191226)(86.2564,554.91397583)(85.72897832,554.91397568)
\curveto(85.3430534,554.91397583)(84.99572441,554.99544807)(84.68699031,555.15839262)
\curveto(84.38253866,555.32133699)(84.13168995,555.52716158)(83.93444341,555.775867)
\lineto(83.93444341,552.4505102)
\lineto(82.77667895,552.4505102)
\moveto(83.82509899,558.44515727)
\curveto(83.82509707,557.56611139)(84.00304958,556.91647754)(84.35895705,556.49625377)
\curveto(84.71485961,556.07602715)(85.14580484,555.86591455)(85.65179402,555.86591534)
\curveto(86.16635175,555.86591455)(86.605873,556.08245916)(86.9703591,556.51554984)
\curveto(87.33912309,556.95292564)(87.52350761,557.62828757)(87.52351323,558.54163764)
\curveto(87.52350761,559.41210065)(87.3434111,560.06387851)(86.98322315,560.49697317)
\curveto(86.62731306,560.93005698)(86.20065584,561.1466016)(85.70325022,561.14660767)
\curveto(85.21012502,561.1466016)(84.77274777,560.91504894)(84.39111717,560.451949)
\curveto(84.01376961,559.99312631)(83.82509707,559.3241964)(83.82509899,558.44515727)
}
}
{
\newrgbcolor{curcolor}{0 0 0}
\pscustom[linestyle=none,fillstyle=solid,fillcolor=curcolor]
{
\newpath
\moveto(96.20674652,559.19127215)
\lineto(89.96125047,556.52198187)
\lineto(89.96125047,557.6733143)
\lineto(94.90747751,559.7251302)
\lineto(89.96125047,561.75765003)
\lineto(89.96125047,562.90898246)
\lineto(96.20674652,560.27185231)
\lineto(96.20674652,559.19127215)
}
}
{
\newrgbcolor{curcolor}{0 0 0}
\pscustom[linestyle=none,fillstyle=solid,fillcolor=curcolor]
{
\newpath
\moveto(103.89944976,559.19127215)
\lineto(97.65395372,556.52198187)
\lineto(97.65395372,557.6733143)
\lineto(102.60018076,559.7251302)
\lineto(97.65395372,561.75765003)
\lineto(97.65395372,562.90898246)
\lineto(103.89944976,560.27185231)
\lineto(103.89944976,559.19127215)
}
}
{
\newrgbcolor{curcolor}{0 0 0}
\pscustom[linestyle=none,fillstyle=solid,fillcolor=curcolor]
{
\newpath
\moveto(58.75949933,540.12028249)
\lineto(56.97782847,539.79868125)
\curveto(56.91779118,540.15458177)(56.78057479,540.42258254)(56.56617888,540.60268434)
\curveto(56.35606158,540.78277556)(56.08162879,540.87282382)(55.74287971,540.87282938)
\curveto(55.29263455,540.87282382)(54.93244152,540.71631137)(54.66229955,540.40329158)
\curveto(54.39643999,540.0945496)(54.26351161,539.57570012)(54.26351402,538.84674158)
\curveto(54.26351161,538.03630374)(54.398584,537.46385411)(54.66873158,537.12939097)
\curveto(54.94316155,536.7949242)(55.30978659,536.62769172)(55.76860781,536.62769304)
\curveto(56.11164488,536.62769172)(56.39250968,536.724172)(56.61120306,536.91713416)
\curveto(56.82988693,537.11438111)(56.98425537,537.45099007)(57.07430884,537.92696204)
\lineto(58.84954767,537.62465688)
\curveto(58.66515616,536.80993224)(58.31139515,536.19460249)(57.78826359,535.77866577)
\curveto(57.26512017,535.36272812)(56.56403017,535.15475952)(55.68499149,535.15475937)
\curveto(54.68588082,535.15475952)(53.88831054,535.46992842)(53.29227828,536.10026701)
\curveto(52.70053116,536.73060402)(52.40465832,537.6032145)(52.40465886,538.71810109)
\curveto(52.40465832,539.84584489)(52.70267517,540.72274339)(53.2987103,541.34879922)
\curveto(53.89474256,541.97913097)(54.70088886,542.29429987)(55.71715162,542.29430686)
\curveto(56.54902213,542.29429987)(57.20937601,542.11420336)(57.69821524,541.75401678)
\curveto(58.19133081,541.39810532)(58.54509181,540.85352777)(58.75949933,540.12028249)
}
}
{
\newrgbcolor{curcolor}{0 0 0}
\pscustom[linestyle=none,fillstyle=solid,fillcolor=curcolor]
{
\newpath
\moveto(59.71787088,538.82101348)
\curveto(59.71787035,539.42133168)(59.86580677,540.00235734)(60.16168058,540.5640922)
\curveto(60.45755246,541.12581654)(60.87563365,541.55461776)(61.41592541,541.85049715)
\curveto(61.96050074,542.14636345)(62.56725447,542.29429987)(63.23618842,542.29430686)
\curveto(64.26959532,542.29429987)(65.11647773,541.95769091)(65.7768382,541.28447897)
\curveto(66.4371855,540.61554309)(66.76736244,539.76866067)(66.76737001,538.74382919)
\curveto(66.76736244,537.71041481)(66.43289748,536.85281236)(65.76397415,536.17101928)
\curveto(65.09932568,535.49351249)(64.2610193,535.15475952)(63.24905247,535.15475937)
\curveto(62.62299863,535.15475952)(62.02482092,535.29626393)(61.45451756,535.579273)
\curveto(60.88849768,535.86228154)(60.45755246,536.27607472)(60.16168058,536.82065378)
\curveto(59.86580677,537.36951784)(59.71787035,538.03630374)(59.71787088,538.82101348)
\moveto(61.57029401,538.72453311)
\curveto(61.57029163,538.04702377)(61.73109209,537.52817429)(62.05269586,537.16798312)
\curveto(62.37429392,536.80778824)(62.77093505,536.62769172)(63.24262045,536.62769304)
\curveto(63.71429774,536.62769172)(64.10879486,536.80778824)(64.426113,537.16798312)
\curveto(64.74770868,537.52817429)(64.90850914,538.05131178)(64.90851486,538.73739716)
\curveto(64.90850914,539.40632364)(64.74770868,539.92088511)(64.426113,540.28108311)
\curveto(64.10879486,540.64127116)(63.71429774,540.82136767)(63.24262045,540.82137319)
\curveto(62.77093505,540.82136767)(62.37429392,540.64127116)(62.05269586,540.28108311)
\curveto(61.73109209,539.92088511)(61.57029163,539.40203563)(61.57029401,538.72453311)
}
}
{
\newrgbcolor{curcolor}{0 0 0}
\pscustom[linestyle=none,fillstyle=solid,fillcolor=curcolor]
{
\newpath
\moveto(74.402183,535.30912796)
\lineto(72.59478404,535.30912796)
\lineto(72.59478404,538.79528539)
\curveto(72.59477869,539.53282)(72.55618658,540.00878936)(72.47900759,540.22319488)
\curveto(72.40181814,540.44187859)(72.27532178,540.61125507)(72.09951813,540.73132484)
\curveto(71.92799279,540.85138376)(71.72002419,540.91141593)(71.47561173,540.91142153)
\curveto(71.16258261,540.91141593)(70.88171781,540.82565569)(70.63301649,540.65414054)
\curveto(70.38430839,540.48261471)(70.2127879,540.25535006)(70.11845451,539.97234592)
\curveto(70.02840337,539.68933245)(69.98337925,539.16619496)(69.98338199,538.40293188)
\lineto(69.98338199,535.30912796)
\lineto(68.17598303,535.30912796)
\lineto(68.17598303,542.13993826)
\lineto(69.85474149,542.13993826)
\lineto(69.85474149,541.1365424)
\curveto(70.45077258,541.90837877)(71.20117472,542.29429987)(72.10595016,542.29430686)
\curveto(72.50473043,542.29429987)(72.86921147,542.22140366)(73.19939437,542.07561801)
\curveto(73.52956535,541.93410684)(73.77827006,541.75186633)(73.94550924,541.52889591)
\curveto(74.11702303,541.30591305)(74.23494336,541.05292033)(74.2992706,540.76991699)
\curveto(74.36787174,540.48690272)(74.40217584,540.08168557)(74.402183,539.55426431)
\lineto(74.402183,535.30912796)
}
}
{
\newrgbcolor{curcolor}{0 0 0}
\pscustom[linestyle=none,fillstyle=solid,fillcolor=curcolor]
{
\newpath
\moveto(79.37413773,542.13993826)
\lineto(79.37413773,540.69916472)
\lineto(78.13918897,540.69916472)
\lineto(78.13918897,537.94625812)
\curveto(78.13918613,537.38881389)(78.14990616,537.06292496)(78.1713491,536.96859035)
\curveto(78.1970743,536.87854044)(78.25067445,536.80350022)(78.33214972,536.74346949)
\curveto(78.41790693,536.68343588)(78.52081922,536.6534198)(78.64088691,536.65342114)
\curveto(78.80811604,536.6534198)(79.05038873,536.71130796)(79.3677057,536.82708581)
\lineto(79.5220743,535.42490441)
\curveto(79.10184487,535.24480778)(78.62587552,535.15475952)(78.0941648,535.15475937)
\curveto(77.76827307,535.15475952)(77.47454424,535.20835968)(77.21297741,535.31555999)
\curveto(76.95140675,535.4270483)(76.7584462,535.5685527)(76.63409518,535.74007362)
\curveto(76.5140295,535.91588169)(76.43041326,536.15172237)(76.38324621,536.44759635)
\curveto(76.34465302,536.65770781)(76.32535696,537.08222102)(76.32535799,537.72113725)
\lineto(76.32535799,540.69916472)
\lineto(75.4956268,540.69916472)
\lineto(75.4956268,542.13993826)
\lineto(76.32535799,542.13993826)
\lineto(76.32535799,543.49709549)
\lineto(78.13918897,544.55194755)
\lineto(78.13918897,542.13993826)
\lineto(79.37413773,542.13993826)
}
}
{
\newrgbcolor{curcolor}{0 0 0}
\pscustom[linestyle=none,fillstyle=solid,fillcolor=curcolor]
{
\newpath
\moveto(81.99197248,540.05596224)
\lineto(80.35180617,540.35183538)
\curveto(80.53619004,541.01218422)(80.85350294,541.50101761)(81.30374584,541.81833702)
\curveto(81.75398551,542.13564342)(82.42291542,542.29429987)(83.31053756,542.29430686)
\curveto(84.11668024,542.29429987)(84.71700195,542.1978196)(85.11150449,542.00486574)
\curveto(85.5059962,541.81618651)(85.78257299,541.57391382)(85.94123569,541.27804694)
\curveto(86.10417391,540.98645614)(86.18564614,540.44831061)(86.18565263,539.66360873)
\lineto(86.16635656,537.55390461)
\curveto(86.16635008,536.95358065)(86.19422216,536.50977139)(86.24997288,536.22247548)
\curveto(86.30999849,535.93946576)(86.41934281,535.63501689)(86.57800614,535.30912796)
\lineto(84.78990326,535.30912796)
\curveto(84.74273003,535.42919231)(84.68484186,535.60714481)(84.61623859,535.84298602)
\curveto(84.58621758,535.95018579)(84.56477752,536.02093799)(84.55191834,536.05524284)
\curveto(84.2431766,535.75508124)(83.91299966,535.52996059)(83.56138653,535.37988024)
\curveto(83.20976566,535.22979974)(82.83456459,535.15475952)(82.43578219,535.15475937)
\curveto(81.73254545,535.15475952)(81.17724787,535.34557607)(80.76988778,535.72720957)
\curveto(80.36681356,536.10884224)(80.16527698,536.59124362)(80.16527745,537.17441515)
\curveto(80.16527698,537.56033438)(80.25746925,537.90337536)(80.44185452,538.20353911)
\curveto(80.6262383,538.50798508)(80.88351903,538.73953774)(81.21369749,538.89819778)
\curveto(81.54816092,539.06113866)(82.02841829,539.20264306)(82.65447104,539.32271142)
\curveto(83.49920648,539.48136385)(84.08452015,539.62930028)(84.4104138,539.76652112)
\lineto(84.4104138,539.94661782)
\curveto(84.41040908,540.29394217)(84.32464884,540.54050287)(84.1531328,540.68630067)
\curveto(83.98160786,540.83637572)(83.65786294,540.91141593)(83.18189707,540.91142153)
\curveto(82.86029266,540.91141593)(82.60944395,540.84709575)(82.42935017,540.71846079)
\curveto(82.24925092,540.59410303)(82.10345851,540.3732704)(81.99197248,540.05596224)
\moveto(84.4104138,538.58946059)
\curveto(84.17885642,538.51227309)(83.81223138,538.42008083)(83.31053756,538.31288353)
\curveto(82.80883652,538.20568022)(82.48080358,538.10062392)(82.32643777,537.99771432)
\curveto(82.09059447,537.83047915)(81.97267413,537.61822255)(81.97267641,537.36094386)
\curveto(81.97267413,537.10794909)(82.0670104,536.88926047)(82.2556855,536.70487734)
\curveto(82.44435548,536.52049142)(82.68448416,536.42829915)(82.97607227,536.42830027)
\curveto(83.30195792,536.42829915)(83.61283881,536.53549946)(83.90871586,536.74990151)
\curveto(84.12740027,536.91284454)(84.27104868,537.1122371)(84.33966152,537.34807981)
\curveto(84.38682501,537.50244622)(84.41040908,537.79617505)(84.4104138,538.22926721)
\lineto(84.4104138,538.58946059)
}
}
{
\newrgbcolor{curcolor}{0 0 0}
\pscustom[linestyle=none,fillstyle=solid,fillcolor=curcolor]
{
\newpath
\moveto(93.9298103,540.12028249)
\lineto(92.14813944,539.79868125)
\curveto(92.08810215,540.15458177)(91.95088576,540.42258254)(91.73648986,540.60268434)
\curveto(91.52637255,540.78277556)(91.25193977,540.87282382)(90.91319069,540.87282938)
\curveto(90.46294552,540.87282382)(90.10275249,540.71631137)(89.83261053,540.40329158)
\curveto(89.56675097,540.0945496)(89.43382259,539.57570012)(89.43382499,538.84674158)
\curveto(89.43382259,538.03630374)(89.56889497,537.46385411)(89.83904255,537.12939097)
\curveto(90.11347252,536.7949242)(90.48009757,536.62769172)(90.93891879,536.62769304)
\curveto(91.28195585,536.62769172)(91.56282065,536.724172)(91.78151403,536.91713416)
\curveto(92.0001979,537.11438111)(92.15456634,537.45099007)(92.24461981,537.92696204)
\lineto(94.01985865,537.62465688)
\curveto(93.83546713,536.80993224)(93.48170612,536.19460249)(92.95857456,535.77866577)
\curveto(92.43543114,535.36272812)(91.73434114,535.15475952)(90.85530247,535.15475937)
\curveto(89.85619179,535.15475952)(89.05862152,535.46992842)(88.46258925,536.10026701)
\curveto(87.87084213,536.73060402)(87.57496929,537.6032145)(87.57496984,538.71810109)
\curveto(87.57496929,539.84584489)(87.87298614,540.72274339)(88.46902128,541.34879922)
\curveto(89.06505354,541.97913097)(89.87119983,542.29429987)(90.88746259,542.29430686)
\curveto(91.7193331,542.29429987)(92.37968698,542.11420336)(92.86852622,541.75401678)
\curveto(93.36164178,541.39810532)(93.71540279,540.85352777)(93.9298103,540.12028249)
}
}
{
\newrgbcolor{curcolor}{0 0 0}
\pscustom[linestyle=none,fillstyle=solid,fillcolor=curcolor]
{
\newpath
\moveto(98.43865952,542.13993826)
\lineto(98.43865952,540.69916472)
\lineto(97.20371076,540.69916472)
\lineto(97.20371076,537.94625812)
\curveto(97.20370792,537.38881389)(97.21442795,537.06292496)(97.23587089,536.96859035)
\curveto(97.26159609,536.87854044)(97.31519624,536.80350022)(97.39667151,536.74346949)
\curveto(97.48242872,536.68343588)(97.58534101,536.6534198)(97.7054087,536.65342114)
\curveto(97.87263783,536.6534198)(98.11491052,536.71130796)(98.43222749,536.82708581)
\lineto(98.58659609,535.42490441)
\curveto(98.16636666,535.24480778)(97.69039731,535.15475952)(97.15868659,535.15475937)
\curveto(96.83279486,535.15475952)(96.53906603,535.20835968)(96.2774992,535.31555999)
\curveto(96.01592854,535.4270483)(95.82296799,535.5685527)(95.69861697,535.74007362)
\curveto(95.57855129,535.91588169)(95.49493505,536.15172237)(95.447768,536.44759635)
\curveto(95.40917481,536.65770781)(95.38987875,537.08222102)(95.38987978,537.72113725)
\lineto(95.38987978,540.69916472)
\lineto(94.56014859,540.69916472)
\lineto(94.56014859,542.13993826)
\lineto(95.38987978,542.13993826)
\lineto(95.38987978,543.49709549)
\lineto(97.20371076,544.55194755)
\lineto(97.20371076,542.13993826)
\lineto(98.43865952,542.13993826)
}
}
{
\newrgbcolor{curcolor}{0 0 0}
\pscustom[linestyle=none,fillstyle=solid,fillcolor=curcolor]
{
\newpath
\moveto(99.06899862,537.25803147)
\lineto(100.88282961,537.53460853)
\curveto(100.9600117,537.1829893)(101.11652415,536.91498854)(101.35236741,536.73060544)
\curveto(101.58820549,536.5505075)(101.91838243,536.46045925)(102.34289923,536.4604604)
\curveto(102.81028898,536.46045925)(103.16190598,536.54621949)(103.39775129,536.71774139)
\curveto(103.5564031,536.83780432)(103.63573133,536.99860478)(103.6357362,537.20014324)
\curveto(103.63573133,537.33735774)(103.59285121,537.45099007)(103.50709571,537.54104056)
\curveto(103.41704271,537.62679857)(103.21550613,537.7061268)(102.90248538,537.77902547)
\curveto(101.44455708,538.10062392)(100.52049045,538.39435276)(100.13028271,538.66021287)
\curveto(99.5899918,539.02897857)(99.31984703,539.54139603)(99.31984759,540.19746678)
\curveto(99.31984703,540.78920758)(99.5535437,541.286617)(100.02093829,541.68969653)
\curveto(100.48833036,542.0927633)(101.21300442,542.29429987)(102.19496266,542.29430686)
\curveto(103.12974589,542.29429987)(103.82440387,542.14207544)(104.27893868,541.8376331)
\curveto(104.73346246,541.5331777)(105.04648735,541.08293642)(105.2180143,540.4869079)
\lineto(103.51352773,540.17173868)
\curveto(103.44062677,540.43759058)(103.30126638,540.64127116)(103.09544612,540.78278104)
\curveto(102.89390521,540.92427997)(102.60446439,540.99503217)(102.22712278,540.99503785)
\curveto(101.75114996,540.99503217)(101.41025299,540.92856798)(101.20443084,540.79564509)
\curveto(101.06721201,540.70130333)(100.99860381,540.57909498)(100.99860605,540.42901968)
\curveto(100.99860381,540.30037419)(101.05863598,540.19102988)(101.17870274,540.10098641)
\curveto(101.34164479,539.98091728)(101.90337439,539.8115408)(102.86389323,539.59285646)
\curveto(103.82869188,539.37416355)(104.5019098,539.10616279)(104.88354901,538.78885336)
\curveto(105.26088796,538.46724896)(105.4495605,538.01915169)(105.44956719,537.44456019)
\curveto(105.4495605,536.81850827)(105.18799175,536.28036273)(104.66486017,535.83012197)
\curveto(104.14171677,535.37988017)(103.36773056,535.15475952)(102.34289923,535.15475937)
\curveto(101.41239699,535.15475952)(100.67485889,535.34343206)(100.13028271,535.72077755)
\curveto(99.5899918,536.09812221)(99.23623079,536.61053967)(99.06899862,537.25803147)
}
}
{
\newrgbcolor{curcolor}{0.50196081 0.50196081 0}
\pscustom[linewidth=2.63455725,linecolor=curcolor]
{
\newpath
\moveto(37.36788,404.10576156)
\lineto(121.67372,404.10576156)
\lineto(121.67372,456.79691156)
\lineto(65.03074,456.79691156)
\lineto(65.03074,467.33514156)
\lineto(37.36788,467.33514156)
\lineto(37.36788,404.10576156)
\closepath
}
}
{
\newrgbcolor{curcolor}{0 0 0}
\pscustom[linestyle=none,fillstyle=solid,fillcolor=curcolor]
{
\newpath
\moveto(55.21291005,438.19799885)
\lineto(55.21291005,439.27857901)
\lineto(61.45840609,441.91570916)
\lineto(61.45840609,440.76437673)
\lineto(56.50574703,438.7318569)
\lineto(61.45840609,436.680041)
\lineto(61.45840609,435.52870857)
\lineto(55.21291005,438.19799885)
}
}
{
\newrgbcolor{curcolor}{0 0 0}
\pscustom[linestyle=none,fillstyle=solid,fillcolor=curcolor]
{
\newpath
\moveto(62.90561187,438.19799885)
\lineto(62.90561187,439.27857901)
\lineto(69.15110791,441.91570916)
\lineto(69.15110791,440.76437673)
\lineto(64.19844884,438.7318569)
\lineto(69.15110791,436.680041)
\lineto(69.15110791,435.52870857)
\lineto(62.90561187,438.19799885)
}
}
{
\newrgbcolor{curcolor}{0 0 0}
\pscustom[linestyle=none,fillstyle=solid,fillcolor=curcolor]
{
\newpath
\moveto(75.22293949,434.07507097)
\lineto(75.22293949,435.07846684)
\curveto(74.69122063,434.30662363)(73.96869057,433.92070253)(73.05534714,433.92070238)
\curveto(72.65227082,433.92070253)(72.27492574,433.99788675)(71.92331078,434.15225527)
\curveto(71.57597975,434.30662363)(71.31655501,434.49958418)(71.14503579,434.7311375)
\curveto(70.97780204,434.96697752)(70.85988171,435.25427433)(70.79127443,435.59302882)
\curveto(70.74410538,435.82029195)(70.72052131,436.18048497)(70.72052215,436.67360898)
\lineto(70.72052215,440.90588127)
\lineto(71.87828661,440.90588127)
\lineto(71.87828661,437.11741869)
\curveto(71.87828461,436.51280592)(71.90186868,436.10544476)(71.94903888,435.89533398)
\curveto(72.02193302,435.59088329)(72.17630146,435.35075461)(72.41214467,435.17494721)
\curveto(72.6479828,435.00342562)(72.93956763,434.91766538)(73.28690003,434.91766622)
\curveto(73.63422561,434.91766538)(73.96011454,435.00556963)(74.2645678,435.18137923)
\curveto(74.56901228,435.36147464)(74.78341289,435.60374733)(74.90777027,435.90819803)
\curveto(75.03640561,436.21693308)(75.10072579,436.66288635)(75.10073102,437.24605918)
\lineto(75.10073102,440.90588127)
\lineto(76.25849547,440.90588127)
\lineto(76.25849547,434.07507097)
\lineto(75.22293949,434.07507097)
}
}
{
\newrgbcolor{curcolor}{0 0 0}
\pscustom[linestyle=none,fillstyle=solid,fillcolor=curcolor]
{
\newpath
\moveto(80.60654312,435.11062696)
\lineto(80.77377576,434.08793502)
\curveto(80.44788327,434.01932682)(80.15629844,433.98502272)(79.89902039,433.98502263)
\curveto(79.47879251,433.98502272)(79.15290358,434.05148691)(78.92135263,434.1844154)
\curveto(78.68979826,434.31734366)(78.52685379,434.49100816)(78.43251875,434.7054094)
\curveto(78.33818126,434.92409739)(78.29101312,435.3807707)(78.2910142,436.07543068)
\lineto(78.2910142,440.0053978)
\lineto(77.44198693,440.0053978)
\lineto(77.44198693,440.90588127)
\lineto(78.2910142,440.90588127)
\lineto(78.2910142,442.59750378)
\lineto(79.44234664,443.29216246)
\lineto(79.44234664,440.90588127)
\lineto(80.60654312,440.90588127)
\lineto(80.60654312,440.0053978)
\lineto(79.44234664,440.0053978)
\lineto(79.44234664,436.01111043)
\curveto(79.4423444,435.68093155)(79.46164046,435.46867495)(79.50023486,435.37433998)
\curveto(79.54311269,435.28000241)(79.60957688,435.20496219)(79.69962763,435.14921911)
\curveto(79.79396141,435.09347388)(79.92688978,435.0656018)(80.09841316,435.06560279)
\curveto(80.22705064,435.0656018)(80.39642712,435.08060984)(80.60654312,435.11062696)
}
}
{
\newrgbcolor{curcolor}{0 0 0}
\pscustom[linestyle=none,fillstyle=solid,fillcolor=curcolor]
{
\newpath
\moveto(81.7385803,442.17299015)
\lineto(81.7385803,443.50441928)
\lineto(82.89634475,443.50441928)
\lineto(82.89634475,442.17299015)
\lineto(81.7385803,442.17299015)
\moveto(81.7385803,434.07507097)
\lineto(81.7385803,440.90588127)
\lineto(82.89634475,440.90588127)
\lineto(82.89634475,434.07507097)
\lineto(81.7385803,434.07507097)
}
}
{
\newrgbcolor{curcolor}{0 0 0}
\pscustom[linestyle=none,fillstyle=solid,fillcolor=curcolor]
{
\newpath
\moveto(84.63942455,434.07507097)
\lineto(84.63942455,443.50441928)
\lineto(85.79718901,443.50441928)
\lineto(85.79718901,434.07507097)
\lineto(84.63942455,434.07507097)
}
}
{
\newrgbcolor{curcolor}{0 0 0}
\pscustom[linestyle=none,fillstyle=solid,fillcolor=curcolor]
{
\newpath
\moveto(93.6957146,438.19799885)
\lineto(87.45021855,435.52870857)
\lineto(87.45021855,436.680041)
\lineto(92.39644559,438.7318569)
\lineto(87.45021855,440.76437673)
\lineto(87.45021855,441.91570916)
\lineto(93.6957146,439.27857901)
\lineto(93.6957146,438.19799885)
}
}
{
\newrgbcolor{curcolor}{0 0 0}
\pscustom[linestyle=none,fillstyle=solid,fillcolor=curcolor]
{
\newpath
\moveto(101.38841784,438.19799885)
\lineto(95.1429218,435.52870857)
\lineto(95.1429218,436.680041)
\lineto(100.08914884,438.7318569)
\lineto(95.1429218,440.76437673)
\lineto(95.1429218,441.91570916)
\lineto(101.38841784,439.27857901)
\lineto(101.38841784,438.19799885)
}
}
{
\newrgbcolor{curcolor}{0 0 0}
\pscustom[linestyle=none,fillstyle=solid,fillcolor=curcolor]
{
\newpath
\moveto(46.21706839,422.07293756)
\lineto(46.21706839,423.745264)
\lineto(48.02446735,423.745264)
\lineto(48.02446735,422.07293756)
\lineto(46.21706839,422.07293756)
\moveto(46.21706839,414.3159157)
\lineto(46.21706839,421.146726)
\lineto(48.02446735,421.146726)
\lineto(48.02446735,414.3159157)
\lineto(46.21706839,414.3159157)
}
}
{
\newrgbcolor{curcolor}{0 0 0}
\pscustom[linestyle=none,fillstyle=solid,fillcolor=curcolor]
{
\newpath
\moveto(56.08379448,414.3159157)
\lineto(54.27639552,414.3159157)
\lineto(54.27639552,417.80207312)
\curveto(54.27639017,418.53960774)(54.23779806,419.01557709)(54.16061908,419.22998262)
\curveto(54.08342962,419.44866633)(53.95693326,419.61804281)(53.78112962,419.73811258)
\curveto(53.60960427,419.8581715)(53.40163568,419.91820367)(53.15722321,419.91820927)
\curveto(52.84419409,419.91820367)(52.56332929,419.83244342)(52.31462797,419.66092828)
\curveto(52.06591987,419.48940244)(51.89439938,419.2621378)(51.80006599,418.97913365)
\curveto(51.71001486,418.69612018)(51.66499073,418.17298269)(51.66499347,417.40971961)
\lineto(51.66499347,414.3159157)
\lineto(49.85759451,414.3159157)
\lineto(49.85759451,421.146726)
\lineto(51.53635297,421.146726)
\lineto(51.53635297,420.14333014)
\curveto(52.13238406,420.91516651)(52.8827862,421.30108761)(53.78756164,421.30109459)
\curveto(54.18634192,421.30108761)(54.55082295,421.2281914)(54.88100585,421.08240575)
\curveto(55.21117684,420.94089458)(55.45988154,420.75865406)(55.62712072,420.53568365)
\curveto(55.79863451,420.31270079)(55.91655485,420.05970807)(55.98088209,419.77670472)
\curveto(56.04948322,419.49369046)(56.08378732,419.0884733)(56.08379448,418.56105204)
\lineto(56.08379448,414.3159157)
}
}
{
\newrgbcolor{curcolor}{0 0 0}
\pscustom[linestyle=none,fillstyle=solid,fillcolor=curcolor]
{
\newpath
\moveto(59.80150415,414.3159157)
\lineto(57.04859755,421.146726)
\lineto(58.94604485,421.146726)
\lineto(60.2324498,417.66056858)
\lineto(60.60550724,416.4963721)
\curveto(60.70412789,416.79224276)(60.76630407,416.98734731)(60.79203596,417.08168635)
\curveto(60.85206432,417.27464413)(60.9163845,417.46760468)(60.9849967,417.66056858)
\lineto(62.2842657,421.146726)
\lineto(64.14312086,421.146726)
\lineto(61.42880641,414.3159157)
\lineto(59.80150415,414.3159157)
}
}
{
\newrgbcolor{curcolor}{0 0 0}
\pscustom[linestyle=none,fillstyle=solid,fillcolor=curcolor]
{
\newpath
\moveto(65.28159018,422.07293756)
\lineto(65.28159018,423.745264)
\lineto(67.08898914,423.745264)
\lineto(67.08898914,422.07293756)
\lineto(65.28159018,422.07293756)
\moveto(65.28159018,414.3159157)
\lineto(65.28159018,421.146726)
\lineto(67.08898914,421.146726)
\lineto(67.08898914,414.3159157)
\lineto(65.28159018,414.3159157)
}
}
{
\newrgbcolor{curcolor}{0 0 0}
\pscustom[linestyle=none,fillstyle=solid,fillcolor=curcolor]
{
\newpath
\moveto(72.06737522,421.146726)
\lineto(72.06737522,419.70595245)
\lineto(70.83242646,419.70595245)
\lineto(70.83242646,416.95304585)
\curveto(70.83242362,416.39560163)(70.84314365,416.0697127)(70.86458659,415.97537809)
\curveto(70.89031179,415.88532817)(70.94391194,415.81028796)(71.02538721,415.75025722)
\curveto(71.11114441,415.69022362)(71.21405671,415.66020753)(71.33412439,415.66020888)
\curveto(71.50135353,415.66020753)(71.74362622,415.7180957)(72.06094319,415.83387354)
\lineto(72.21531179,414.43169215)
\curveto(71.79508236,414.25159552)(71.31911301,414.16154726)(70.78740229,414.16154711)
\curveto(70.46151056,414.16154726)(70.16778173,414.21514741)(69.9062149,414.32234773)
\curveto(69.64464424,414.43383604)(69.45168369,414.57534044)(69.32733267,414.74686136)
\curveto(69.20726699,414.92266943)(69.12365075,415.1585101)(69.0764837,415.45438408)
\curveto(69.03789051,415.66449554)(69.01859445,416.08900875)(69.01859548,416.72792499)
\lineto(69.01859548,419.70595245)
\lineto(68.18886429,419.70595245)
\lineto(68.18886429,421.146726)
\lineto(69.01859548,421.146726)
\lineto(69.01859548,422.50388322)
\lineto(70.83242646,423.55873528)
\lineto(70.83242646,421.146726)
\lineto(72.06737522,421.146726)
}
}
{
\newrgbcolor{curcolor}{0 0 0}
\pscustom[linestyle=none,fillstyle=solid,fillcolor=curcolor]
{
\newpath
\moveto(74.68520997,419.06274998)
\lineto(73.04504366,419.35862311)
\curveto(73.22942753,420.01897195)(73.54674043,420.50780535)(73.99698332,420.82512476)
\curveto(74.447223,421.14243116)(75.11615291,421.30108761)(76.00377505,421.30109459)
\curveto(76.80991773,421.30108761)(77.41023944,421.20460733)(77.80474198,421.01165348)
\curveto(78.19923369,420.82297424)(78.47581048,420.58070155)(78.63447318,420.28483468)
\curveto(78.7974114,419.99324388)(78.87888363,419.45509835)(78.87889012,418.67039646)
\lineto(78.85959404,416.56069234)
\curveto(78.85958757,415.96036839)(78.88745965,415.51655912)(78.94321037,415.22926322)
\curveto(79.00323598,414.9462535)(79.11258029,414.64180463)(79.27124363,414.3159157)
\lineto(77.48314075,414.3159157)
\curveto(77.43596752,414.43598004)(77.37807935,414.61393255)(77.30947608,414.84977376)
\curveto(77.27945507,414.95697353)(77.25801501,415.02772573)(77.24515583,415.06203057)
\curveto(76.93641409,414.76186897)(76.60623715,414.53674833)(76.25462402,414.38666797)
\curveto(75.90300315,414.23658747)(75.52780208,414.16154726)(75.12901968,414.16154711)
\curveto(74.42578294,414.16154726)(73.87048536,414.3523638)(73.46312527,414.73399731)
\curveto(73.06005105,415.11562998)(72.85851447,415.59803135)(72.85851494,416.18120288)
\curveto(72.85851447,416.56712212)(72.95070673,416.91016309)(73.13509201,417.21032684)
\curveto(73.31947579,417.51477282)(73.57675652,417.74632548)(73.90693498,417.90498552)
\curveto(74.24139841,418.06792639)(74.72165578,418.2094308)(75.34770852,418.32949915)
\curveto(76.19244397,418.48815159)(76.77775764,418.63608801)(77.10365128,418.77330886)
\lineto(77.10365128,418.95340555)
\curveto(77.10364657,419.30072991)(77.01788633,419.54729061)(76.84637029,419.6930884)
\curveto(76.67484535,419.84316345)(76.35110043,419.91820367)(75.87513455,419.91820927)
\curveto(75.55353015,419.91820367)(75.30268144,419.85388348)(75.12258766,419.72524853)
\curveto(74.94248841,419.60089076)(74.796696,419.38005813)(74.68520997,419.06274998)
\moveto(77.10365128,417.59624833)
\curveto(76.87209391,417.51906083)(76.50546886,417.42686857)(76.00377505,417.31967126)
\curveto(75.50207401,417.21246796)(75.17404107,417.10741166)(75.01967526,417.00450205)
\curveto(74.78383196,416.83726689)(74.66591162,416.62501028)(74.6659139,416.3677316)
\curveto(74.66591162,416.11473683)(74.76024789,415.8960482)(74.94892299,415.71166507)
\curveto(75.13759297,415.52727915)(75.37772165,415.43508689)(75.66930976,415.43508801)
\curveto(75.99519541,415.43508689)(76.3060763,415.5422872)(76.60195335,415.75668925)
\curveto(76.82063776,415.91963227)(76.96428617,416.11902484)(77.03289901,416.35486755)
\curveto(77.0800625,416.50923395)(77.10364657,416.80296279)(77.10365128,417.23605494)
\lineto(77.10365128,417.59624833)
}
}
{
\newrgbcolor{curcolor}{0 0 0}
\pscustom[linestyle=none,fillstyle=solid,fillcolor=curcolor]
{
\newpath
\moveto(83.79938892,421.146726)
\lineto(83.79938892,419.70595245)
\lineto(82.56444017,419.70595245)
\lineto(82.56444017,416.95304585)
\curveto(82.56443732,416.39560163)(82.57515735,416.0697127)(82.59660029,415.97537809)
\curveto(82.62232549,415.88532817)(82.67592564,415.81028796)(82.75740091,415.75025722)
\curveto(82.84315812,415.69022362)(82.94607041,415.66020753)(83.0661381,415.66020888)
\curveto(83.23336723,415.66020753)(83.47563992,415.7180957)(83.7929569,415.83387354)
\lineto(83.94732549,414.43169215)
\curveto(83.52709607,414.25159552)(83.05112671,414.16154726)(82.51941599,414.16154711)
\curveto(82.19352427,414.16154726)(81.89979543,414.21514741)(81.6382286,414.32234773)
\curveto(81.37665794,414.43383604)(81.18369739,414.57534044)(81.05934637,414.74686136)
\curveto(80.93928069,414.92266943)(80.85566445,415.1585101)(80.80849741,415.45438408)
\curveto(80.76990421,415.66449554)(80.75060815,416.08900875)(80.75060918,416.72792499)
\lineto(80.75060918,419.70595245)
\lineto(79.92087799,419.70595245)
\lineto(79.92087799,421.146726)
\lineto(80.75060918,421.146726)
\lineto(80.75060918,422.50388322)
\lineto(82.56444017,423.55873528)
\lineto(82.56444017,421.146726)
\lineto(83.79938892,421.146726)
}
}
{
\newrgbcolor{curcolor}{0 0 0}
\pscustom[linestyle=none,fillstyle=solid,fillcolor=curcolor]
{
\newpath
\moveto(85.06649848,422.07293756)
\lineto(85.06649848,423.745264)
\lineto(86.87389743,423.745264)
\lineto(86.87389743,422.07293756)
\lineto(85.06649848,422.07293756)
\moveto(85.06649848,414.3159157)
\lineto(85.06649848,421.146726)
\lineto(86.87389743,421.146726)
\lineto(86.87389743,414.3159157)
\lineto(85.06649848,414.3159157)
}
}
{
\newrgbcolor{curcolor}{0 0 0}
\pscustom[linestyle=none,fillstyle=solid,fillcolor=curcolor]
{
\newpath
\moveto(88.30180584,417.82780122)
\curveto(88.30180532,418.42811942)(88.44974174,419.00914508)(88.74561555,419.57087993)
\curveto(89.04148742,420.13260428)(89.45956862,420.5614055)(89.99986038,420.85728488)
\curveto(90.54443571,421.15315119)(91.15118944,421.30108761)(91.82012339,421.30109459)
\curveto(92.85353029,421.30108761)(93.7004127,420.96447865)(94.36077317,420.2912667)
\curveto(95.02112047,419.62233082)(95.35129741,418.77544841)(95.35130498,417.75061692)
\curveto(95.35129741,416.71720254)(95.01683245,415.8596001)(94.34790912,415.17780702)
\curveto(93.68326065,414.50030023)(92.84495426,414.16154726)(91.83298744,414.16154711)
\curveto(91.2069336,414.16154726)(90.60875589,414.30305166)(90.03845253,414.58606074)
\curveto(89.47243265,414.86906928)(89.04148742,415.28286246)(88.74561555,415.82744152)
\curveto(88.44974174,416.37630557)(88.30180532,417.04309147)(88.30180584,417.82780122)
\moveto(90.15422898,417.73132085)
\curveto(90.1542266,417.0538115)(90.31502705,416.53496202)(90.63663083,416.17477086)
\curveto(90.95822889,415.81457597)(91.35487002,415.63447946)(91.82655541,415.63448078)
\curveto(92.29823271,415.63447946)(92.69272983,415.81457597)(93.01004797,416.17477086)
\curveto(93.33164365,416.53496202)(93.49244411,417.05809952)(93.49244983,417.7441849)
\curveto(93.49244411,418.41311138)(93.33164365,418.92767284)(93.01004797,419.28787084)
\curveto(92.69272983,419.6480589)(92.29823271,419.82815541)(91.82655541,419.82816092)
\curveto(91.35487002,419.82815541)(90.95822889,419.6480589)(90.63663083,419.28787084)
\curveto(90.31502705,418.92767284)(90.1542266,418.40882336)(90.15422898,417.73132085)
}
}
{
\newrgbcolor{curcolor}{0 0 0}
\pscustom[linestyle=none,fillstyle=solid,fillcolor=curcolor]
{
\newpath
\moveto(102.98611797,414.3159157)
\lineto(101.17871901,414.3159157)
\lineto(101.17871901,417.80207312)
\curveto(101.17871366,418.53960774)(101.14012155,419.01557709)(101.06294256,419.22998262)
\curveto(100.98575311,419.44866633)(100.85925675,419.61804281)(100.6834531,419.73811258)
\curveto(100.51192776,419.8581715)(100.30395916,419.91820367)(100.0595467,419.91820927)
\curveto(99.74651757,419.91820367)(99.46565277,419.83244342)(99.21695146,419.66092828)
\curveto(98.96824336,419.48940244)(98.79672287,419.2621378)(98.70238947,418.97913365)
\curveto(98.61233834,418.69612018)(98.56731421,418.17298269)(98.56731695,417.40971961)
\lineto(98.56731695,414.3159157)
\lineto(96.759918,414.3159157)
\lineto(96.759918,421.146726)
\lineto(98.43867646,421.146726)
\lineto(98.43867646,420.14333014)
\curveto(99.03470755,420.91516651)(99.78510968,421.30108761)(100.68988513,421.30109459)
\curveto(101.0886654,421.30108761)(101.45314644,421.2281914)(101.78332934,421.08240575)
\curveto(102.11350032,420.94089458)(102.36220503,420.75865406)(102.52944421,420.53568365)
\curveto(102.70095799,420.31270079)(102.81887833,420.05970807)(102.88320557,419.77670472)
\curveto(102.95180671,419.49369046)(102.98611081,419.0884733)(102.98611797,418.56105204)
\lineto(102.98611797,414.3159157)
}
}
{
\newrgbcolor{curcolor}{0 0 0}
\pscustom[linestyle=none,fillstyle=solid,fillcolor=curcolor]
{
\newpath
\moveto(104.18890619,416.2648192)
\lineto(106.00273717,416.54139627)
\curveto(106.07991927,416.18977704)(106.23643171,415.92177628)(106.47227498,415.73739317)
\curveto(106.70811306,415.55729524)(107.03829,415.46724698)(107.46280679,415.46724813)
\curveto(107.93019654,415.46724698)(108.28181354,415.55300723)(108.51765885,415.72452912)
\curveto(108.67631067,415.84459206)(108.75563889,416.00539252)(108.75564377,416.20693098)
\curveto(108.75563889,416.34414548)(108.71275877,416.4577778)(108.62700327,416.54782829)
\curveto(108.53695027,416.63358631)(108.33541369,416.71291453)(108.02239294,416.78581321)
\curveto(106.56446465,417.10741166)(105.64039801,417.40114049)(105.25019027,417.6670006)
\curveto(104.70989936,418.0357663)(104.43975459,418.54818376)(104.43975515,419.20425452)
\curveto(104.43975459,419.79599532)(104.67345126,420.29340474)(105.14084585,420.69648426)
\curveto(105.60823792,421.09955103)(106.33291199,421.30108761)(107.31487022,421.30109459)
\curveto(108.24965345,421.30108761)(108.94431143,421.14886317)(109.39884624,420.84442083)
\curveto(109.85337002,420.53996544)(110.16639491,420.08972416)(110.33792186,419.49369563)
\lineto(108.6334353,419.17852642)
\curveto(108.56053434,419.44437832)(108.42117394,419.6480589)(108.21535369,419.78956877)
\curveto(108.01381278,419.9310677)(107.72437195,420.0018199)(107.34703034,420.00182559)
\curveto(106.87105752,420.0018199)(106.53016055,419.93535572)(106.32433841,419.80243282)
\curveto(106.18711957,419.70809107)(106.11851138,419.58588272)(106.11851361,419.43580741)
\curveto(106.11851138,419.30716192)(106.17854355,419.19781761)(106.29861031,419.10777415)
\curveto(106.46155235,418.98770501)(107.02328196,418.81832853)(107.9838008,418.59964419)
\curveto(108.94859944,418.38095129)(109.62181736,418.11295052)(110.00345657,417.7956411)
\curveto(110.38079552,417.4740367)(110.56946806,417.02593942)(110.56947475,416.45134792)
\curveto(110.56946806,415.825296)(110.30789932,415.28715047)(109.78476773,414.83690971)
\curveto(109.26162433,414.3866679)(108.48763813,414.16154726)(107.46280679,414.16154711)
\curveto(106.53230456,414.16154726)(105.79476645,414.3502198)(105.25019027,414.72756529)
\curveto(104.70989936,415.10490995)(104.35613835,415.61732741)(104.18890619,416.2648192)
}
}
{
\newrgbcolor{curcolor}{0.50196081 0 1}
\pscustom[linewidth=2.63455725,linecolor=curcolor]
{
\newpath
\moveto(404.51,642.91493486)
\lineto(490.13312,642.91493486)
\lineto(490.13312,695.60607486)
\lineto(433.49014,695.60607486)
\lineto(433.49014,706.14430486)
\lineto(404.51,706.14430486)
\lineto(404.51,642.91493486)
\closepath
}
}
{
\newrgbcolor{curcolor}{0 0 0}
\pscustom[linestyle=none,fillstyle=solid,fillcolor=curcolor]
{
\newpath
\moveto(411.81677299,677.00716245)
\lineto(411.81677299,678.08774261)
\lineto(418.06226904,680.72487276)
\lineto(418.06226904,679.57354033)
\lineto(413.10960997,677.5410205)
\lineto(418.06226904,675.4892046)
\lineto(418.06226904,674.33787217)
\lineto(411.81677299,677.00716245)
}
}
{
\newrgbcolor{curcolor}{0 0 0}
\pscustom[linestyle=none,fillstyle=solid,fillcolor=curcolor]
{
\newpath
\moveto(419.50947481,677.00716245)
\lineto(419.50947481,678.08774261)
\lineto(425.75497085,680.72487276)
\lineto(425.75497085,679.57354033)
\lineto(420.80231179,677.5410205)
\lineto(425.75497085,675.4892046)
\lineto(425.75497085,674.33787217)
\lineto(419.50947481,677.00716245)
}
}
{
\newrgbcolor{curcolor}{0 0 0}
\pscustom[linestyle=none,fillstyle=solid,fillcolor=curcolor]
{
\newpath
\moveto(427.35011319,672.88423457)
\lineto(427.35011319,679.71504487)
\lineto(428.38566918,679.71504487)
\lineto(428.38566918,678.75667318)
\curveto(428.60006789,679.09113226)(428.8852207,679.35913303)(429.24112847,679.56067628)
\curveto(429.59703073,679.76649419)(430.00224788,679.86940648)(430.45678115,679.86941347)
\curveto(430.96276262,679.86940648)(431.3765558,679.76435018)(431.69816193,679.55424425)
\curveto(432.02404565,679.34412498)(432.2534543,679.05039615)(432.38638858,678.67305686)
\curveto(432.92667222,679.47062134)(433.62990622,679.86940648)(434.49609271,679.86941347)
\curveto(435.17359062,679.86940648)(435.69458411,679.68073394)(436.05907472,679.30339529)
\curveto(436.42354618,678.9303318)(436.6057867,678.35359416)(436.60579683,677.57318062)
\lineto(436.60579683,672.88423457)
\lineto(435.45446439,672.88423457)
\lineto(435.45446439,677.18725914)
\curveto(435.45445542,677.65036016)(435.41586331,677.9826811)(435.33868795,678.18422298)
\curveto(435.26578288,678.39004226)(435.1307105,678.55513073)(434.93347039,678.67948888)
\curveto(434.73621338,678.80383544)(434.50466072,678.86601162)(434.23881171,678.8660176)
\curveto(433.75854659,678.86601162)(433.35976145,678.70521116)(433.04245511,678.38361574)
\curveto(432.72513564,678.06629734)(432.56647919,677.55602389)(432.56648528,676.85279385)
\lineto(432.56648528,672.88423457)
\lineto(431.40872082,672.88423457)
\lineto(431.40872082,677.32233166)
\curveto(431.40871589,677.83688869)(431.31437962,678.22280979)(431.12571173,678.48009612)
\curveto(430.93703455,678.73737125)(430.62829767,678.86601162)(430.19950016,678.8660176)
\curveto(429.87360752,678.86601162)(429.57130266,678.78025138)(429.29258467,678.60873661)
\curveto(429.01814908,678.4372104)(428.81875651,678.18636168)(428.69440637,677.85618971)
\curveto(428.5700518,677.5260078)(428.50787563,677.05003845)(428.50787765,676.42828022)
\lineto(428.50787765,672.88423457)
\lineto(427.35011319,672.88423457)
}
}
{
\newrgbcolor{curcolor}{0 0 0}
\pscustom[linestyle=none,fillstyle=solid,fillcolor=curcolor]
{
\newpath
\moveto(438.36817146,680.98215375)
\lineto(438.36817146,682.31358287)
\lineto(439.52593592,682.31358287)
\lineto(439.52593592,680.98215375)
\lineto(438.36817146,680.98215375)
\moveto(438.36817146,672.88423457)
\lineto(438.36817146,679.71504487)
\lineto(439.52593592,679.71504487)
\lineto(439.52593592,672.88423457)
\lineto(438.36817146,672.88423457)
}
}
{
\newrgbcolor{curcolor}{0 0 0}
\pscustom[linestyle=none,fillstyle=solid,fillcolor=curcolor]
{
\newpath
\moveto(445.72640697,672.88423457)
\lineto(445.72640697,673.74612589)
\curveto(445.29331244,673.0686191)(444.65654262,672.72986613)(443.81609562,672.72986598)
\curveto(443.27151468,672.72986613)(442.76981725,672.87994656)(442.31100182,673.18010771)
\curveto(441.85647064,673.48026827)(441.50270963,673.89834946)(441.24971774,674.43435254)
\curveto(441.0010122,674.97464053)(440.87665985,675.5942583)(440.8766603,676.2932077)
\curveto(440.87665985,676.97499823)(440.99029217,677.59247199)(441.21755761,678.14563083)
\curveto(441.44482147,678.70306716)(441.78571844,679.12972437)(442.24024955,679.42560376)
\curveto(442.69477703,679.72147006)(443.20290648,679.86940648)(443.76463942,679.86941347)
\curveto(444.17628525,679.86940648)(444.5429103,679.78150223)(444.86451565,679.60570045)
\curveto(445.18611213,679.43417324)(445.44768088,679.2090526)(445.64922267,678.93033785)
\lineto(445.64922267,682.31358287)
\lineto(446.80055511,682.31358287)
\lineto(446.80055511,672.88423457)
\lineto(445.72640697,672.88423457)
\moveto(442.06658488,676.2932077)
\curveto(442.06658324,675.4184498)(442.25096777,674.76452793)(442.61973901,674.33144014)
\curveto(442.98850587,673.89834946)(443.42373911,673.68180485)(443.92544004,673.68180564)
\curveto(444.43142198,673.68180485)(444.8602232,673.88762943)(445.21184499,674.29928002)
\curveto(445.56774522,674.71521579)(445.74569773,675.34769759)(445.74570305,676.19672733)
\curveto(445.74569773,677.13151068)(445.56560121,677.81759263)(445.20541297,678.25497525)
\curveto(444.84521516,678.69234713)(444.4014059,678.91103575)(443.87398384,678.91104178)
\curveto(443.35941893,678.91103575)(442.9284737,678.70092315)(442.58114686,678.28070335)
\curveto(442.23810373,677.86047275)(442.06658324,677.19797487)(442.06658488,676.2932077)
}
}
{
\newrgbcolor{curcolor}{0 0 0}
\pscustom[linestyle=none,fillstyle=solid,fillcolor=curcolor]
{
\newpath
\moveto(453.05891506,672.88423457)
\lineto(453.05891506,673.74612589)
\curveto(452.62582052,673.0686191)(451.98905071,672.72986613)(451.1486037,672.72986598)
\curveto(450.60402276,672.72986613)(450.10232533,672.87994656)(449.64350991,673.18010771)
\curveto(449.18897873,673.48026827)(448.83521772,673.89834946)(448.58222582,674.43435254)
\curveto(448.33352029,674.97464053)(448.20916794,675.5942583)(448.20916839,676.2932077)
\curveto(448.20916794,676.97499823)(448.32280026,677.59247199)(448.5500657,678.14563083)
\curveto(448.77732956,678.70306716)(449.11822653,679.12972437)(449.57275764,679.42560376)
\curveto(450.02728512,679.72147006)(450.53541457,679.86940648)(451.09714751,679.86941347)
\curveto(451.50879334,679.86940648)(451.87541839,679.78150223)(452.19702374,679.60570045)
\curveto(452.51862022,679.43417324)(452.78018896,679.2090526)(452.98173076,678.93033785)
\lineto(452.98173076,682.31358287)
\lineto(454.13306319,682.31358287)
\lineto(454.13306319,672.88423457)
\lineto(453.05891506,672.88423457)
\moveto(449.39909297,676.2932077)
\curveto(449.39909133,675.4184498)(449.58347585,674.76452793)(449.9522471,674.33144014)
\curveto(450.32101396,673.89834946)(450.7562472,673.68180485)(451.25794813,673.68180564)
\curveto(451.76393007,673.68180485)(452.19273129,673.88762943)(452.54435308,674.29928002)
\curveto(452.90025331,674.71521579)(453.07820581,675.34769759)(453.07821113,676.19672733)
\curveto(453.07820581,677.13151068)(452.8981093,677.81759263)(452.53792105,678.25497525)
\curveto(452.17772325,678.69234713)(451.73391398,678.91103575)(451.20649193,678.91104178)
\curveto(450.69192701,678.91103575)(450.26098178,678.70092315)(449.91365495,678.28070335)
\curveto(449.57061182,677.86047275)(449.39909133,677.19797487)(449.39909297,676.2932077)
}
}
{
\newrgbcolor{curcolor}{0 0 0}
\pscustom[linestyle=none,fillstyle=solid,fillcolor=curcolor]
{
\newpath
\moveto(455.93402999,672.88423457)
\lineto(455.93402999,682.31358287)
\lineto(457.09179444,682.31358287)
\lineto(457.09179444,672.88423457)
\lineto(455.93402999,672.88423457)
}
}
{
\newrgbcolor{curcolor}{0 0 0}
\pscustom[linestyle=none,fillstyle=solid,fillcolor=curcolor]
{
\newpath
\moveto(463.56884256,675.08398704)
\lineto(464.76519916,674.93605047)
\curveto(464.57651989,674.23710243)(464.22704689,673.69466888)(463.71677913,673.30874821)
\curveto(463.20649998,672.92282668)(462.55472212,672.72986613)(461.7614436,672.72986598)
\curveto(460.76233302,672.72986613)(459.96905076,673.03645901)(459.38159444,673.64964552)
\curveto(458.79842342,674.26711851)(458.50683859,675.13115298)(458.50683907,676.2417515)
\curveto(458.50683859,677.39093542)(458.80271143,678.28284196)(459.39445849,678.9174738)
\curveto(459.9862028,679.55209358)(460.75375699,679.86940648)(461.69712335,679.86941347)
\curveto(462.61046628,679.86940648)(463.35658041,679.55852559)(463.93546797,678.93676987)
\curveto(464.51434371,678.31500205)(464.80378453,677.44024756)(464.80379131,676.31250377)
\curveto(464.80378453,676.24389215)(464.80164053,676.14097985)(464.79735929,676.00376658)
\lineto(459.70319568,676.00376658)
\curveto(459.74607412,675.25336132)(459.95833072,674.67876769)(460.33996613,674.27998395)
\curveto(460.7215969,673.88119741)(461.19756626,673.68180485)(461.76787563,673.68180564)
\curveto(462.19238509,673.68180485)(462.55472212,673.79329316)(462.85488781,674.01627093)
\curveto(463.15504383,674.23924643)(463.39302851,674.59515145)(463.56884256,675.08398704)
\moveto(459.76751592,676.95570625)
\lineto(463.58170661,676.95570625)
\curveto(463.5302449,677.53029581)(463.38445249,677.96124104)(463.14432892,678.24854322)
\curveto(462.77555475,678.69449113)(462.29744139,678.91746777)(461.7099874,678.9174738)
\curveto(461.1782702,678.91746777)(460.73017292,678.73951526)(460.36569423,678.38361574)
\curveto(460.00549886,678.02770523)(459.80610629,677.55173587)(459.76751592,676.95570625)
}
}
{
\newrgbcolor{curcolor}{0 0 0}
\pscustom[linestyle=none,fillstyle=solid,fillcolor=curcolor]
{
\newpath
\moveto(472.32282812,677.00716245)
\lineto(466.07733207,674.33787217)
\lineto(466.07733207,675.4892046)
\lineto(471.02355912,677.5410205)
\lineto(466.07733207,679.57354033)
\lineto(466.07733207,680.72487276)
\lineto(472.32282812,678.08774261)
\lineto(472.32282812,677.00716245)
}
}
{
\newrgbcolor{curcolor}{0 0 0}
\pscustom[linestyle=none,fillstyle=solid,fillcolor=curcolor]
{
\newpath
\moveto(480.01553137,677.00716245)
\lineto(473.77003532,674.33787217)
\lineto(473.77003532,675.4892046)
\lineto(478.71626236,677.5410205)
\lineto(473.77003532,679.57354033)
\lineto(473.77003532,680.72487276)
\lineto(480.01553137,678.08774261)
\lineto(480.01553137,677.00716245)
}
}
{
\newrgbcolor{curcolor}{0 0 0}
\pscustom[linestyle=none,fillstyle=solid,fillcolor=curcolor]
{
\newpath
\moveto(424.57788342,655.07392177)
\lineto(426.3917144,655.35049883)
\curveto(426.4688965,654.9988796)(426.62540894,654.73087884)(426.86125221,654.54649574)
\curveto(427.09709029,654.3663978)(427.42726723,654.27634955)(427.85178402,654.2763507)
\curveto(428.31917377,654.27634955)(428.67079077,654.36210979)(428.90663608,654.53363169)
\curveto(429.0652879,654.65369462)(429.14461612,654.81449508)(429.144621,655.01603354)
\curveto(429.14461612,655.15324804)(429.101736,655.26688037)(429.0159805,655.35693086)
\curveto(428.9259275,655.44268887)(428.72439093,655.5220171)(428.41137018,655.59491577)
\curveto(426.95344188,655.91651422)(426.02937524,656.21024306)(425.6391675,656.47610317)
\curveto(425.09887659,656.84486887)(424.82873182,657.35728633)(424.82873238,658.01335708)
\curveto(424.82873182,658.60509788)(425.06242849,659.1025073)(425.52982308,659.50558683)
\curveto(425.99721515,659.9086536)(426.72188922,660.11019017)(427.70384745,660.11019716)
\curveto(428.63863068,660.11019017)(429.33328866,659.95796574)(429.78782347,659.6535234)
\curveto(430.24234725,659.349068)(430.55537214,658.89882672)(430.72689909,658.3027982)
\lineto(429.02241253,657.98762898)
\curveto(428.94951157,658.25348088)(428.81015117,658.45716146)(428.60433092,658.59867134)
\curveto(428.40279001,658.74017027)(428.11334918,658.81092247)(427.73600758,658.81092815)
\curveto(427.26003475,658.81092247)(426.91913778,658.74445828)(426.71331564,658.61153539)
\curveto(426.5760968,658.51719363)(426.50748861,658.39498528)(426.50749085,658.24490998)
\curveto(426.50748861,658.11626449)(426.56752078,658.00692018)(426.68758754,657.91687671)
\curveto(426.85052958,657.79680758)(427.41225919,657.6274311)(428.37277803,657.40874676)
\curveto(429.33757667,657.19005385)(430.01079459,656.92205309)(430.3924338,656.60474366)
\curveto(430.76977275,656.28313926)(430.95844529,655.83504199)(430.95845198,655.26045049)
\curveto(430.95844529,654.63439857)(430.69687655,654.09625303)(430.17374496,653.64601227)
\curveto(429.65060156,653.19577047)(428.87661536,652.97064982)(427.85178402,652.97064967)
\curveto(426.92128179,652.97064982)(426.18374368,653.15932236)(425.6391675,653.53666785)
\curveto(425.09887659,653.91401251)(424.74511558,654.42642997)(424.57788342,655.07392177)
}
}
{
\newrgbcolor{curcolor}{0 0 0}
\pscustom[linestyle=none,fillstyle=solid,fillcolor=curcolor]
{
\newpath
\moveto(432.49570576,659.95582856)
\lineto(434.18089625,659.95582856)
\lineto(434.18089625,658.9524327)
\curveto(434.39958229,659.29546785)(434.69545513,659.57418864)(435.06851566,659.78859592)
\curveto(435.44156926,660.00298987)(435.85536244,660.11019017)(436.30989644,660.11019716)
\curveto(437.10317399,660.11019017)(437.77639191,659.79930929)(438.32955222,659.17755357)
\curveto(438.88269907,658.55578574)(439.15927585,657.68960727)(439.15928341,656.57901556)
\curveto(439.15927585,655.43840086)(438.88055506,654.55078233)(438.32312019,653.91615731)
\curveto(437.76567188,653.28581872)(437.09030996,652.97064982)(436.29703239,652.97064967)
\curveto(435.91968262,652.97064982)(435.57664164,653.04569004)(435.26790843,653.19577054)
\curveto(434.9634559,653.34585089)(434.64185498,653.60313163)(434.30310472,653.96761351)
\lineto(434.30310472,650.52648026)
\lineto(432.49570576,650.52648026)
\lineto(432.49570576,659.95582856)
\moveto(434.28380864,656.65619986)
\curveto(434.28380596,655.88864214)(434.43603039,655.32048052)(434.7404824,654.9517133)
\curveto(435.04492813,654.58723043)(435.41584119,654.40498991)(435.85322268,654.40499119)
\curveto(436.27344363,654.40498991)(436.62291663,654.57222239)(436.90164272,654.90668912)
\curveto(437.18035821,655.24544031)(437.31971861,655.79859388)(437.31972433,656.56615151)
\curveto(437.31971861,657.28224611)(437.1760702,657.81395963)(436.88877867,658.16129365)
\curveto(436.60147656,658.50861761)(436.24557155,658.6822821)(435.82106256,658.68228766)
\curveto(435.37939308,658.6822821)(435.01276804,658.51076161)(434.72118633,658.16772568)
\curveto(434.42959837,657.82896767)(434.28380596,657.32512623)(434.28380864,656.65619986)
}
}
{
\newrgbcolor{curcolor}{0 0 0}
\pscustom[linestyle=none,fillstyle=solid,fillcolor=curcolor]
{
\newpath
\moveto(441.95078175,657.87185254)
\lineto(440.31061543,658.16772568)
\curveto(440.4949993,658.82807452)(440.81231221,659.31690791)(441.2625551,659.63422732)
\curveto(441.71279477,659.95153372)(442.38172468,660.11019017)(443.26934682,660.11019716)
\curveto(444.07548951,660.11019017)(444.67581122,660.0137099)(445.07031376,659.82075604)
\curveto(445.46480547,659.63207681)(445.74138225,659.38980412)(445.90004495,659.09393724)
\curveto(446.06298317,658.80234644)(446.1444554,658.26420091)(446.14446189,657.47949903)
\lineto(446.12516582,655.36979491)
\curveto(446.12515935,654.76947095)(446.15303143,654.32566169)(446.20878214,654.03836578)
\curveto(446.26880776,653.75535606)(446.37815207,653.45090719)(446.5368154,653.12501826)
\lineto(444.74871252,653.12501826)
\curveto(444.70153929,653.24508261)(444.64365113,653.42303511)(444.57504785,653.65887632)
\curveto(444.54502684,653.76607609)(444.52358678,653.83682829)(444.5107276,653.87113314)
\curveto(444.20198587,653.57097154)(443.87180893,653.34585089)(443.52019579,653.19577054)
\curveto(443.16857492,653.04569004)(442.79337385,652.97064982)(442.39459146,652.97064967)
\curveto(441.69135471,652.97064982)(441.13605713,653.16146637)(440.72869704,653.54309987)
\curveto(440.32562282,653.92473254)(440.12408625,654.40713392)(440.12408672,654.99030545)
\curveto(440.12408625,655.37622468)(440.21627851,655.71926566)(440.40066378,656.01942941)
\curveto(440.58504756,656.32387538)(440.84232829,656.55542804)(441.17250675,656.71408808)
\curveto(441.50697019,656.87702896)(441.98722756,657.01853336)(442.6132803,657.13860172)
\curveto(443.45801575,657.29725415)(444.04332942,657.44519058)(444.36922306,657.58241142)
\lineto(444.36922306,657.76250812)
\curveto(444.36921834,658.10983247)(444.2834581,658.35639317)(444.11194207,658.50219097)
\curveto(443.94041712,658.65226602)(443.6166722,658.72730623)(443.14070633,658.72731183)
\curveto(442.81910193,658.72730623)(442.56825321,658.66298605)(442.38815943,658.53435109)
\curveto(442.20806018,658.40999333)(442.06226777,658.1891607)(441.95078175,657.87185254)
\moveto(444.36922306,656.40535089)
\curveto(444.13766568,656.32816339)(443.77104064,656.23597113)(443.26934682,656.12877383)
\curveto(442.76764578,656.02157052)(442.43961284,655.91651422)(442.28524704,655.81360462)
\curveto(442.04940373,655.64636945)(441.9314834,655.43411285)(441.93148567,655.17683416)
\curveto(441.9314834,654.92383939)(442.02581967,654.70515077)(442.21449476,654.52076764)
\curveto(442.40316474,654.33638172)(442.64329343,654.24418945)(442.93488154,654.24419057)
\curveto(443.26076718,654.24418945)(443.57164807,654.35138976)(443.86752513,654.56579181)
\curveto(444.08620954,654.72873484)(444.22985795,654.9281274)(444.29847079,655.16397011)
\curveto(444.34563428,655.31833652)(444.36921834,655.61206535)(444.36922306,656.04515751)
\lineto(444.36922306,656.40535089)
}
}
{
\newrgbcolor{curcolor}{0 0 0}
\pscustom[linestyle=none,fillstyle=solid,fillcolor=curcolor]
{
\newpath
\moveto(453.88861956,657.93617279)
\lineto(452.10694871,657.61457155)
\curveto(452.04691141,657.97047207)(451.90969502,658.23847284)(451.69529912,658.41857464)
\curveto(451.48518181,658.59866586)(451.21074903,658.68871412)(450.87199995,658.68871968)
\curveto(450.42175478,658.68871412)(450.06156176,658.53220167)(449.79141979,658.21918188)
\curveto(449.52556023,657.9104399)(449.39263185,657.39159042)(449.39263426,656.66263188)
\curveto(449.39263185,655.85219404)(449.52770424,655.27974441)(449.79785182,654.94528127)
\curveto(450.07228179,654.6108145)(450.43890683,654.44358202)(450.89772805,654.44358334)
\curveto(451.24076512,654.44358202)(451.52162992,654.5400623)(451.74032329,654.73302446)
\curveto(451.95900716,654.93027141)(452.1133756,655.26688037)(452.20342908,655.74285234)
\lineto(453.97866791,655.44054718)
\curveto(453.79427639,654.62582254)(453.44051539,654.01049279)(452.91738383,653.59455607)
\curveto(452.3942404,653.17861842)(451.69315041,652.97064982)(450.81411173,652.97064967)
\curveto(449.81500105,652.97064982)(449.01743078,653.28581872)(448.42139852,653.91615731)
\curveto(447.8296514,654.54649432)(447.53377855,655.4191048)(447.5337791,656.53399139)
\curveto(447.53377855,657.66173519)(447.8317954,658.53863369)(448.42783054,659.16468952)
\curveto(449.0238628,659.79502127)(449.8300091,660.11019017)(450.84627185,660.11019716)
\curveto(451.67814236,660.11019017)(452.33849625,659.93009366)(452.82733548,659.56990708)
\curveto(453.32045104,659.21399562)(453.67421205,658.66941807)(453.88861956,657.93617279)
}
}
{
\newrgbcolor{curcolor}{0 0 0}
\pscustom[linestyle=none,fillstyle=solid,fillcolor=curcolor]
{
\newpath
\moveto(459.22076795,655.29904263)
\lineto(461.02173488,654.99673747)
\curveto(460.79017552,654.33638172)(460.42355048,653.83254028)(459.92185865,653.48521165)
\curveto(459.42444363,653.14217031)(458.80053785,652.97064982)(458.05013944,652.97064967)
\curveto(456.86235633,652.97064982)(455.98331382,653.35871493)(455.41300929,654.13484615)
\curveto(454.96276692,654.75660691)(454.73764627,655.54131315)(454.73764669,656.48896722)
\curveto(454.73764627,657.62099908)(455.03351912,658.5064736)(455.62526611,659.14539344)
\curveto(456.21701049,659.78858925)(456.96526862,660.11019017)(457.87004275,660.11019716)
\curveto(458.8862981,660.11019017)(459.68815638,659.77358121)(460.27562001,659.10036927)
\curveto(460.86307173,658.43143339)(461.14393653,657.40445446)(461.11821526,656.01942941)
\lineto(456.59006982,656.01942941)
\curveto(456.60293159,655.48342499)(456.748724,655.06534379)(457.02744751,654.76518458)
\curveto(457.30616559,654.4693101)(457.65349458,654.32137367)(458.06943552,654.32137487)
\curveto(458.35244058,654.32137367)(458.59042525,654.39855789)(458.78339027,654.55292776)
\curveto(458.97634635,654.70729477)(459.12213877,654.95599948)(459.22076795,655.29904263)
\moveto(459.32368035,657.12573767)
\curveto(459.31081131,657.64887116)(459.17573892,658.04551229)(458.91846279,658.31566225)
\curveto(458.66117746,658.59008984)(458.34815256,658.72730623)(457.97938717,658.72731183)
\curveto(457.58488639,658.72730623)(457.25899746,658.58365782)(457.00171941,658.29636617)
\curveto(456.74443599,658.00906418)(456.61793963,657.61885507)(456.62222995,657.12573767)
\lineto(459.32368035,657.12573767)
}
}
{
\newrgbcolor{curcolor}{0 0 0}
\pscustom[linestyle=none,fillstyle=solid,fillcolor=curcolor]
{
\newpath
\moveto(461.96081036,655.07392177)
\lineto(463.77464134,655.35049883)
\curveto(463.85182344,654.9988796)(464.00833588,654.73087884)(464.24417915,654.54649574)
\curveto(464.48001723,654.3663978)(464.81019417,654.27634955)(465.23471096,654.2763507)
\curveto(465.70210071,654.27634955)(466.05371771,654.36210979)(466.28956302,654.53363169)
\curveto(466.44821484,654.65369462)(466.52754306,654.81449508)(466.52754794,655.01603354)
\curveto(466.52754306,655.15324804)(466.48466294,655.26688037)(466.39890744,655.35693086)
\curveto(466.30885444,655.44268887)(466.10731787,655.5220171)(465.79429712,655.59491577)
\curveto(464.33636882,655.91651422)(463.41230219,656.21024306)(463.02209444,656.47610317)
\curveto(462.48180353,656.84486887)(462.21165876,657.35728633)(462.21165932,658.01335708)
\curveto(462.21165876,658.60509788)(462.44535543,659.1025073)(462.91275002,659.50558683)
\curveto(463.38014209,659.9086536)(464.10481616,660.11019017)(465.08677439,660.11019716)
\curveto(466.02155762,660.11019017)(466.7162156,659.95796574)(467.17075042,659.6535234)
\curveto(467.62527419,659.349068)(467.93829908,658.89882672)(468.10982603,658.3027982)
\lineto(466.40533947,657.98762898)
\curveto(466.33243851,658.25348088)(466.19307811,658.45716146)(465.98725786,658.59867134)
\curveto(465.78571695,658.74017027)(465.49627612,658.81092247)(465.11893452,658.81092815)
\curveto(464.64296169,658.81092247)(464.30206472,658.74445828)(464.09624258,658.61153539)
\curveto(463.95902374,658.51719363)(463.89041555,658.39498528)(463.89041779,658.24490998)
\curveto(463.89041555,658.11626449)(463.95044772,658.00692018)(464.07051448,657.91687671)
\curveto(464.23345653,657.79680758)(464.79518613,657.6274311)(465.75570497,657.40874676)
\curveto(466.72050361,657.19005385)(467.39372153,656.92205309)(467.77536074,656.60474366)
\curveto(468.15269969,656.28313926)(468.34137223,655.83504199)(468.34137892,655.26045049)
\curveto(468.34137223,654.63439857)(468.07980349,654.09625303)(467.5566719,653.64601227)
\curveto(467.03352851,653.19577047)(466.2595423,652.97064982)(465.23471096,652.97064967)
\curveto(464.30420873,652.97064982)(463.56667063,653.15932236)(463.02209444,653.53666785)
\curveto(462.48180353,653.91401251)(462.12804253,654.42642997)(461.96081036,655.07392177)
}
}
{
\newrgbcolor{curcolor}{0.50196081 0 1}
\pscustom[linewidth=2.63455725,linecolor=curcolor]
{
\newpath
\moveto(283.00129,523.51032456)
\lineto(368.62441,523.51032456)
\lineto(368.62441,576.20147456)
\lineto(311.98143,576.20147456)
\lineto(311.98143,586.73970456)
\lineto(283.00129,586.73970456)
\lineto(283.00129,523.51032456)
\closepath
}
}
{
\newrgbcolor{curcolor}{0 0 0}
\pscustom[linestyle=none,fillstyle=solid,fillcolor=curcolor]
{
\newpath
\moveto(290.30807576,557.60256185)
\lineto(290.30807576,558.68314201)
\lineto(296.5535718,561.32027216)
\lineto(296.5535718,560.16893973)
\lineto(291.60091273,558.1364199)
\lineto(296.5535718,556.084604)
\lineto(296.5535718,554.93327157)
\lineto(290.30807576,557.60256185)
}
}
{
\newrgbcolor{curcolor}{0 0 0}
\pscustom[linestyle=none,fillstyle=solid,fillcolor=curcolor]
{
\newpath
\moveto(298.00077757,557.60256185)
\lineto(298.00077757,558.68314201)
\lineto(304.24627362,561.32027216)
\lineto(304.24627362,560.16893973)
\lineto(299.29361455,558.1364199)
\lineto(304.24627362,556.084604)
\lineto(304.24627362,554.93327157)
\lineto(298.00077757,557.60256185)
}
}
{
\newrgbcolor{curcolor}{0 0 0}
\pscustom[linestyle=none,fillstyle=solid,fillcolor=curcolor]
{
\newpath
\moveto(305.84141596,553.47963397)
\lineto(305.84141596,560.31044427)
\lineto(306.87697195,560.31044427)
\lineto(306.87697195,559.35207258)
\curveto(307.09137065,559.68653166)(307.37652347,559.95453243)(307.73243124,560.15607568)
\curveto(308.0883335,560.36189359)(308.49355065,560.46480588)(308.94808392,560.46481287)
\curveto(309.45406539,560.46480588)(309.86785857,560.35974958)(310.1894647,560.14964365)
\curveto(310.51534841,559.93952438)(310.74475707,559.64579555)(310.87769135,559.26845626)
\curveto(311.41797498,560.06602074)(312.12120899,560.46480588)(312.98739547,560.46481287)
\curveto(313.66489339,560.46480588)(314.18588687,560.27613334)(314.55037749,559.89879469)
\curveto(314.91484895,559.5257312)(315.09708947,558.94899356)(315.09709959,558.16858003)
\lineto(315.09709959,553.47963397)
\lineto(313.94576716,553.47963397)
\lineto(313.94576716,557.78265854)
\curveto(313.94575819,558.24575956)(313.90716608,558.5780805)(313.82999071,558.77962238)
\curveto(313.75708565,558.98544166)(313.62201327,559.15053014)(313.42477315,559.27488828)
\curveto(313.22751614,559.39923484)(312.99596348,559.46141102)(312.73011448,559.461417)
\curveto(312.24984935,559.46141102)(311.85106422,559.30061056)(311.53375787,558.97901515)
\curveto(311.21643841,558.66169674)(311.05778196,558.15142329)(311.05778804,557.44819325)
\lineto(311.05778804,553.47963397)
\lineto(309.90002359,553.47963397)
\lineto(309.90002359,557.91773106)
\curveto(309.90001866,558.43228809)(309.80568239,558.81820919)(309.6170145,559.07549552)
\curveto(309.42833731,559.33277065)(309.11960043,559.46141102)(308.69080293,559.461417)
\curveto(308.36491028,559.46141102)(308.06260542,559.37565078)(307.78388744,559.20413601)
\curveto(307.50945185,559.0326098)(307.31005928,558.78176108)(307.18570914,558.45158912)
\curveto(307.06135457,558.1214072)(306.99917839,557.64543785)(306.99918042,557.02367962)
\lineto(306.99918042,553.47963397)
\lineto(305.84141596,553.47963397)
}
}
{
\newrgbcolor{curcolor}{0 0 0}
\pscustom[linestyle=none,fillstyle=solid,fillcolor=curcolor]
{
\newpath
\moveto(316.85947423,561.57755315)
\lineto(316.85947423,562.90898228)
\lineto(318.01723869,562.90898228)
\lineto(318.01723869,561.57755315)
\lineto(316.85947423,561.57755315)
\moveto(316.85947423,553.47963397)
\lineto(316.85947423,560.31044427)
\lineto(318.01723869,560.31044427)
\lineto(318.01723869,553.47963397)
\lineto(316.85947423,553.47963397)
}
}
{
\newrgbcolor{curcolor}{0 0 0}
\pscustom[linestyle=none,fillstyle=solid,fillcolor=curcolor]
{
\newpath
\moveto(324.21770974,553.47963397)
\lineto(324.21770974,554.34152529)
\curveto(323.7846152,553.6640185)(323.14784539,553.32526553)(322.30739838,553.32526538)
\curveto(321.76281744,553.32526553)(321.26112001,553.47534596)(320.80230459,553.77550711)
\curveto(320.34777341,554.07566767)(319.9940124,554.49374886)(319.7410205,555.02975194)
\curveto(319.49231497,555.57003993)(319.36796262,556.1896577)(319.36796307,556.8886071)
\curveto(319.36796262,557.57039763)(319.48159494,558.18787139)(319.70886038,558.74103023)
\curveto(319.93612424,559.29846656)(320.27702121,559.72512377)(320.73155232,560.02100316)
\curveto(321.1860798,560.31686946)(321.69420925,560.46480588)(322.25594218,560.46481287)
\curveto(322.66758802,560.46480588)(323.03421306,560.37690163)(323.35581842,560.20109985)
\curveto(323.6774149,560.02957264)(323.93898364,559.804452)(324.14052544,559.52573725)
\lineto(324.14052544,562.90898228)
\lineto(325.29185787,562.90898228)
\lineto(325.29185787,553.47963397)
\lineto(324.21770974,553.47963397)
\moveto(320.55788765,556.8886071)
\curveto(320.55788601,556.0138492)(320.74227053,555.35992733)(321.11104178,554.92683955)
\curveto(321.47980863,554.49374886)(321.91504187,554.27720425)(322.4167428,554.27720504)
\curveto(322.92272475,554.27720425)(323.35152597,554.48302883)(323.70314776,554.89467942)
\curveto(324.05904798,555.31061519)(324.23700049,555.94309699)(324.23700581,556.79212673)
\curveto(324.23700049,557.72691008)(324.05690398,558.41299203)(323.69671573,558.85037465)
\curveto(323.33651793,559.28774653)(322.89270866,559.50643515)(322.36528661,559.50644118)
\curveto(321.85072169,559.50643515)(321.41977646,559.29632255)(321.07244963,558.87610275)
\curveto(320.7294065,558.45587216)(320.55788601,557.79337427)(320.55788765,556.8886071)
}
}
{
\newrgbcolor{curcolor}{0 0 0}
\pscustom[linestyle=none,fillstyle=solid,fillcolor=curcolor]
{
\newpath
\moveto(331.55021782,553.47963397)
\lineto(331.55021782,554.34152529)
\curveto(331.11712329,553.6640185)(330.48035348,553.32526553)(329.63990647,553.32526538)
\curveto(329.09532553,553.32526553)(328.5936281,553.47534596)(328.13481268,553.77550711)
\curveto(327.6802815,554.07566767)(327.32652049,554.49374886)(327.07352859,555.02975194)
\curveto(326.82482306,555.57003993)(326.7004707,556.1896577)(326.70047115,556.8886071)
\curveto(326.7004707,557.57039763)(326.81410303,558.18787139)(327.04136847,558.74103023)
\curveto(327.26863232,559.29846656)(327.60952929,559.72512377)(328.0640604,560.02100316)
\curveto(328.51858788,560.31686946)(329.02671733,560.46480588)(329.58845027,560.46481287)
\curveto(330.00009611,560.46480588)(330.36672115,560.37690163)(330.68832651,560.20109985)
\curveto(331.00992298,560.02957264)(331.27149173,559.804452)(331.47303353,559.52573725)
\lineto(331.47303353,562.90898228)
\lineto(332.62436596,562.90898228)
\lineto(332.62436596,553.47963397)
\lineto(331.55021782,553.47963397)
\moveto(327.89039573,556.8886071)
\curveto(327.89039409,556.0138492)(328.07477862,555.35992733)(328.44354986,554.92683955)
\curveto(328.81231672,554.49374886)(329.24754996,554.27720425)(329.74925089,554.27720504)
\curveto(330.25523283,554.27720425)(330.68403406,554.48302883)(331.03565584,554.89467942)
\curveto(331.39155607,555.31061519)(331.56950858,555.94309699)(331.5695139,556.79212673)
\curveto(331.56950858,557.72691008)(331.38941207,558.41299203)(331.02922382,558.85037465)
\curveto(330.66902601,559.28774653)(330.22521675,559.50643515)(329.69779469,559.50644118)
\curveto(329.18322978,559.50643515)(328.75228455,559.29632255)(328.40495772,558.87610275)
\curveto(328.06191458,558.45587216)(327.89039409,557.79337427)(327.89039573,556.8886071)
}
}
{
\newrgbcolor{curcolor}{0 0 0}
\pscustom[linestyle=none,fillstyle=solid,fillcolor=curcolor]
{
\newpath
\moveto(334.42533275,553.47963397)
\lineto(334.42533275,562.90898228)
\lineto(335.58309721,562.90898228)
\lineto(335.58309721,553.47963397)
\lineto(334.42533275,553.47963397)
}
}
{
\newrgbcolor{curcolor}{0 0 0}
\pscustom[linestyle=none,fillstyle=solid,fillcolor=curcolor]
{
\newpath
\moveto(342.06014532,555.67938644)
\lineto(343.25650193,555.53144987)
\curveto(343.06782265,554.83250183)(342.71834966,554.29006828)(342.20808189,553.90414761)
\curveto(341.69780275,553.51822608)(341.04602489,553.32526553)(340.25274637,553.32526538)
\curveto(339.25363578,553.32526553)(338.46035352,553.63185841)(337.8728972,554.24504492)
\curveto(337.28972618,554.86251792)(336.99814135,555.72655238)(336.99814184,556.8371509)
\curveto(336.99814135,557.98633482)(337.2940142,558.87824136)(337.88576125,559.5128732)
\curveto(338.47750557,560.14749298)(339.24505976,560.46480588)(340.18842612,560.46481287)
\curveto(341.10176905,560.46480588)(341.84788317,560.15392499)(342.42677074,559.53216928)
\curveto(343.00564647,558.91040145)(343.2950873,558.03564696)(343.29509408,556.90790317)
\curveto(343.2950873,556.83929155)(343.29294329,556.73637926)(343.28866205,556.59916598)
\lineto(338.19449844,556.59916598)
\curveto(338.23737689,555.84876073)(338.44963349,555.27416709)(338.83126889,554.87538335)
\curveto(339.21289967,554.47659682)(339.68886902,554.27720425)(340.25917839,554.27720504)
\curveto(340.68368786,554.27720425)(341.04602489,554.38869256)(341.34619058,554.61167033)
\curveto(341.6463466,554.83464584)(341.88433128,555.19055085)(342.06014532,555.67938644)
\moveto(338.25881869,557.55110565)
\lineto(342.07300937,557.55110565)
\curveto(342.02154767,558.12569521)(341.87575525,558.55664044)(341.63563169,558.84394263)
\curveto(341.26685752,559.28989053)(340.78874416,559.51286717)(340.20129017,559.5128732)
\curveto(339.66957297,559.51286717)(339.22147569,559.33491466)(338.85699699,558.97901515)
\curveto(338.49680162,558.62310463)(338.29740906,558.14713528)(338.25881869,557.55110565)
}
}
{
\newrgbcolor{curcolor}{0 0 0}
\pscustom[linestyle=none,fillstyle=solid,fillcolor=curcolor]
{
\newpath
\moveto(350.81413088,557.60256185)
\lineto(344.56863484,554.93327157)
\lineto(344.56863484,556.084604)
\lineto(349.51486188,558.1364199)
\lineto(344.56863484,560.16893973)
\lineto(344.56863484,561.32027216)
\lineto(350.81413088,558.68314201)
\lineto(350.81413088,557.60256185)
}
}
{
\newrgbcolor{curcolor}{0 0 0}
\pscustom[linestyle=none,fillstyle=solid,fillcolor=curcolor]
{
\newpath
\moveto(358.50683413,557.60256185)
\lineto(352.26133809,554.93327157)
\lineto(352.26133809,556.084604)
\lineto(357.20756513,558.1364199)
\lineto(352.26133809,560.16893973)
\lineto(352.26133809,561.32027216)
\lineto(358.50683413,558.68314201)
\lineto(358.50683413,557.60256185)
}
}
{
\newrgbcolor{curcolor}{0 0 0}
\pscustom[linestyle=none,fillstyle=solid,fillcolor=curcolor]
{
\newpath
\moveto(296.21525455,533.7204787)
\lineto(294.40785559,533.7204787)
\lineto(294.40785559,540.551289)
\lineto(296.08661405,540.551289)
\lineto(296.08661405,539.58005326)
\curveto(296.37390832,540.03886471)(296.63118906,540.34116957)(296.85845702,540.48696875)
\curveto(297.09000636,540.6327544)(297.35157511,540.70565061)(297.64316404,540.70565759)
\curveto(298.05480911,540.70565061)(298.45145024,540.59201828)(298.83308863,540.36476028)
\lineto(298.27350247,538.78891421)
\curveto(297.96904887,538.98615771)(297.68604006,539.08478199)(297.4244752,539.08478735)
\curveto(297.1714786,539.08478199)(296.95707799,539.01402979)(296.78127273,538.87253054)
\curveto(296.60546098,538.73530899)(296.46610059,538.48446028)(296.36319112,538.11998364)
\curveto(296.26456401,537.7554982)(296.21525187,536.99223202)(296.21525455,535.83018282)
\lineto(296.21525455,533.7204787)
}
}
{
\newrgbcolor{curcolor}{0 0 0}
\pscustom[linestyle=none,fillstyle=solid,fillcolor=curcolor]
{
\newpath
\moveto(303.56062676,535.89450307)
\lineto(305.36159369,535.59219791)
\curveto(305.13003433,534.93184215)(304.76340928,534.42800072)(304.26171746,534.08067209)
\curveto(303.76430244,533.73763075)(303.14039666,533.56611026)(302.38999825,533.56611011)
\curveto(301.20221514,533.56611026)(300.32317263,533.95417537)(299.7528681,534.73030659)
\curveto(299.30262572,535.35206735)(299.07750508,536.13677359)(299.0775055,537.08442765)
\curveto(299.07750508,538.21645951)(299.37337792,539.10193404)(299.96512492,539.74085388)
\curveto(300.5568693,540.38404969)(301.30512743,540.70565061)(302.20990156,540.70565759)
\curveto(303.2261569,540.70565061)(304.02801519,540.36904165)(304.61547882,539.6958297)
\curveto(305.20293054,539.02689382)(305.48379534,537.9999149)(305.45807406,536.61488984)
\lineto(300.92992863,536.61488984)
\curveto(300.9427904,536.07888542)(301.08858281,535.66080423)(301.36730631,535.36064502)
\curveto(301.6460244,535.06477053)(301.99335339,534.91683411)(302.40929433,534.91683531)
\curveto(302.69229938,534.91683411)(302.93028406,534.99401833)(303.12324907,535.1483882)
\curveto(303.31620516,535.30275521)(303.46199758,535.55145992)(303.56062676,535.89450307)
\moveto(303.66353915,537.7211981)
\curveto(303.65067011,538.24433159)(303.51559773,538.64097272)(303.25832159,538.91112268)
\curveto(303.00103626,539.18555028)(302.68801137,539.32276667)(302.31924598,539.32277227)
\curveto(301.9247452,539.32276667)(301.59885627,539.17911826)(301.34157822,538.89182661)
\curveto(301.0842948,538.60452462)(300.95779844,538.21431551)(300.96208875,537.7211981)
\lineto(303.66353915,537.7211981)
}
}
{
\newrgbcolor{curcolor}{0 0 0}
\pscustom[linestyle=none,fillstyle=solid,fillcolor=curcolor]
{
\newpath
\moveto(306.30066964,535.6693822)
\lineto(308.11450063,535.94595927)
\curveto(308.19168272,535.59434004)(308.34819517,535.32633928)(308.58403843,535.14195617)
\curveto(308.81987651,534.96185824)(309.15005345,534.87180998)(309.57457025,534.87181113)
\curveto(310.04196,534.87180998)(310.393577,534.95757023)(310.62942231,535.12909212)
\curveto(310.78807412,535.24915506)(310.86740235,535.40995552)(310.86740722,535.61149398)
\curveto(310.86740235,535.74870848)(310.82452223,535.8623408)(310.73876673,535.95239129)
\curveto(310.64871372,536.03814931)(310.44717715,536.11747753)(310.1341564,536.19037621)
\curveto(308.6762281,536.51197466)(307.75216147,536.80570349)(307.36195373,537.0715636)
\curveto(306.82166282,537.4403293)(306.55151805,537.95274676)(306.55151861,538.60881752)
\curveto(306.55151805,539.20055832)(306.78521471,539.69796774)(307.25260931,540.10104726)
\curveto(307.72000138,540.50411403)(308.44467544,540.70565061)(309.42663368,540.70565759)
\curveto(310.36141691,540.70565061)(311.05607489,540.55342617)(311.5106097,540.24898383)
\curveto(311.96513348,539.94452844)(312.27815837,539.49428716)(312.44968531,538.89825863)
\lineto(310.74519875,538.58308942)
\curveto(310.67229779,538.84894132)(310.53293739,539.0526219)(310.32711714,539.19413177)
\curveto(310.12557623,539.3356307)(309.83613541,539.4063829)(309.4587938,539.40638859)
\curveto(308.98282098,539.4063829)(308.64192401,539.33991872)(308.43610186,539.20699582)
\curveto(308.29888303,539.11265407)(308.23027483,538.99044572)(308.23027707,538.84037041)
\curveto(308.23027483,538.71172492)(308.290307,538.60238061)(308.41037376,538.51233715)
\curveto(308.57331581,538.39226801)(309.13504541,538.22289153)(310.09556425,538.00420719)
\curveto(311.0603629,537.78551429)(311.73358082,537.51751352)(312.11522003,537.2002041)
\curveto(312.49255898,536.8785997)(312.68123152,536.43050242)(312.68123821,535.85591092)
\curveto(312.68123152,535.229859)(312.41966277,534.69171347)(311.89653119,534.24147271)
\curveto(311.37338779,533.7912309)(310.59940158,533.56611026)(309.57457025,533.56611011)
\curveto(308.64406801,533.56611026)(307.90652991,533.7547828)(307.36195373,534.13212829)
\curveto(306.82166282,534.50947295)(306.46790181,535.02189041)(306.30066964,535.6693822)
}
}
{
\newrgbcolor{curcolor}{0 0 0}
\pscustom[linestyle=none,fillstyle=solid,fillcolor=curcolor]
{
\newpath
\moveto(313.85186657,537.23236422)
\curveto(313.85186604,537.83268242)(313.99980247,538.41370808)(314.29567628,538.97544293)
\curveto(314.59154815,539.53716728)(315.00962934,539.9659685)(315.54992111,540.26184788)
\curveto(316.09449643,540.55771419)(316.70125016,540.70565061)(317.37018412,540.70565759)
\curveto(318.40359102,540.70565061)(319.25047343,540.36904165)(319.9108339,539.6958297)
\curveto(320.57118119,539.02689382)(320.90135813,538.18001141)(320.90136571,537.15517992)
\curveto(320.90135813,536.12176554)(320.56689318,535.2641631)(319.89796985,534.58237002)
\curveto(319.23332138,533.90486323)(318.39501499,533.56611026)(317.38304817,533.56611011)
\curveto(316.75699432,533.56611026)(316.15881662,533.70761466)(315.58851326,533.99062374)
\curveto(315.02249338,534.27363228)(314.59154815,534.68742546)(314.29567628,535.23200452)
\curveto(313.99980247,535.78086857)(313.85186604,536.44765447)(313.85186657,537.23236422)
\moveto(315.7042897,537.13588385)
\curveto(315.70428732,536.4583745)(315.86508778,535.93952502)(316.18669156,535.57933386)
\curveto(316.50828961,535.21913897)(316.90493074,535.03904246)(317.37661614,535.03904378)
\curveto(317.84829343,535.03904246)(318.24279056,535.21913897)(318.5601087,535.57933386)
\curveto(318.88170438,535.93952502)(319.04250484,536.46266252)(319.04251055,537.1487479)
\curveto(319.04250484,537.81767438)(318.88170438,538.33223584)(318.5601087,538.69243384)
\curveto(318.24279056,539.0526219)(317.84829343,539.23271841)(317.37661614,539.23272392)
\curveto(316.90493074,539.23271841)(316.50828961,539.0526219)(316.18669156,538.69243384)
\curveto(315.86508778,538.33223584)(315.70428732,537.81338636)(315.7042897,537.13588385)
}
}
{
\newrgbcolor{curcolor}{0 0 0}
\pscustom[linestyle=none,fillstyle=solid,fillcolor=curcolor]
{
\newpath
\moveto(326.81882808,533.7204787)
\lineto(326.81882808,534.74317064)
\curveto(326.57011793,534.37868858)(326.242085,534.09139176)(325.83472829,533.88127932)
\curveto(325.43165069,533.67116656)(325.00499347,533.56611026)(324.55475537,533.56611011)
\curveto(324.09593488,533.56611026)(323.68428571,533.66687855)(323.31980661,533.86841527)
\curveto(322.95532363,534.0699517)(322.69161088,534.3529605)(322.52866757,534.71744254)
\curveto(322.36572195,535.08192258)(322.28424972,535.58576402)(322.28425062,536.22896836)
\lineto(322.28425062,540.551289)
\lineto(324.09164958,540.551289)
\lineto(324.09164958,537.41246091)
\curveto(324.09164687,536.45194248)(324.12380696,535.8623408)(324.18812995,535.6436541)
\curveto(324.25673534,535.42925157)(324.37894369,535.25773108)(324.55475537,535.12909212)
\curveto(324.73056069,535.00473836)(324.95353732,534.94256218)(325.22368594,534.94256341)
\curveto(325.53241897,534.94256218)(325.80899576,535.02617842)(326.05341714,535.19341237)
\curveto(326.29782916,535.36493139)(326.46506163,535.57504399)(326.55511507,535.8237508)
\curveto(326.64515815,536.07674142)(326.69018227,536.69207117)(326.69018759,537.6697419)
\lineto(326.69018759,540.551289)
\lineto(328.49758654,540.551289)
\lineto(328.49758654,533.7204787)
\lineto(326.81882808,533.7204787)
}
}
{
\newrgbcolor{curcolor}{0 0 0}
\pscustom[linestyle=none,fillstyle=solid,fillcolor=curcolor]
{
\newpath
\moveto(332.10595203,533.7204787)
\lineto(330.29855307,533.7204787)
\lineto(330.29855307,540.551289)
\lineto(331.97731153,540.551289)
\lineto(331.97731153,539.58005326)
\curveto(332.2646058,540.03886471)(332.52188654,540.34116957)(332.7491545,540.48696875)
\curveto(332.98070384,540.6327544)(333.24227259,540.70565061)(333.53386152,540.70565759)
\curveto(333.94550659,540.70565061)(334.34214772,540.59201828)(334.7237861,540.36476028)
\lineto(334.16419995,538.78891421)
\curveto(333.85974635,538.98615771)(333.57673754,539.08478199)(333.31517268,539.08478735)
\curveto(333.06217608,539.08478199)(332.84777546,539.01402979)(332.67197021,538.87253054)
\curveto(332.49615846,538.73530899)(332.35679807,538.48446028)(332.2538886,538.11998364)
\curveto(332.15526149,537.7554982)(332.10594935,536.99223202)(332.10595203,535.83018282)
\lineto(332.10595203,533.7204787)
}
}
{
\newrgbcolor{curcolor}{0 0 0}
\pscustom[linestyle=none,fillstyle=solid,fillcolor=curcolor]
{
\newpath
\moveto(341.45168442,538.53163322)
\lineto(339.67001356,538.21003198)
\curveto(339.60997627,538.56593251)(339.47275987,538.83393327)(339.25836397,539.01403508)
\curveto(339.04824666,539.1941263)(338.77381388,539.28417456)(338.4350648,539.28418012)
\curveto(337.98481963,539.28417456)(337.62462661,539.12766211)(337.35448464,538.81464231)
\curveto(337.08862508,538.50590034)(336.9556967,537.98705086)(336.95569911,537.25809232)
\curveto(336.9556967,536.44765447)(337.09076909,535.87520484)(337.36091667,535.54074171)
\curveto(337.63534664,535.20627494)(338.00197168,535.03904246)(338.4607929,535.03904378)
\curveto(338.80382997,535.03904246)(339.08469477,535.13552273)(339.30338814,535.32848489)
\curveto(339.52207201,535.52573185)(339.67644045,535.8623408)(339.76649393,536.33831278)
\lineto(341.54173276,536.03600762)
\curveto(341.35734124,535.22128298)(341.00358024,534.60595322)(340.48044868,534.19001651)
\curveto(339.95730526,533.77407885)(339.25621526,533.56611026)(338.37717658,533.56611011)
\curveto(337.3780659,533.56611026)(336.58049563,533.88127916)(335.98446337,534.51161775)
\curveto(335.39271625,535.14195475)(335.0968434,536.01456524)(335.09684395,537.12945182)
\curveto(335.0968434,538.25719563)(335.39486025,539.13409413)(335.99089539,539.76014995)
\curveto(336.58692765,540.39048171)(337.39307395,540.70565061)(338.4093367,540.70565759)
\curveto(339.24120721,540.70565061)(339.9015611,540.52555409)(340.39040033,540.16536751)
\curveto(340.88351589,539.80945605)(341.2372769,539.2648785)(341.45168442,538.53163322)
}
}
{
\newrgbcolor{curcolor}{0 0 0}
\pscustom[linestyle=none,fillstyle=solid,fillcolor=curcolor]
{
\newpath
\moveto(346.7838328,535.89450307)
\lineto(348.58479973,535.59219791)
\curveto(348.35324037,534.93184215)(347.98661533,534.42800072)(347.4849235,534.08067209)
\curveto(346.98750848,533.73763075)(346.3636027,533.56611026)(345.61320429,533.56611011)
\curveto(344.42542118,533.56611026)(343.54637867,533.95417537)(342.97607414,534.73030659)
\curveto(342.52583177,535.35206735)(342.30071112,536.13677359)(342.30071154,537.08442765)
\curveto(342.30071112,538.21645951)(342.59658397,539.10193404)(343.18833096,539.74085388)
\curveto(343.78007534,540.38404969)(344.52833347,540.70565061)(345.4331076,540.70565759)
\curveto(346.44936295,540.70565061)(347.25122123,540.36904165)(347.83868486,539.6958297)
\curveto(348.42613658,539.02689382)(348.70700138,537.9999149)(348.68128011,536.61488984)
\lineto(344.15313467,536.61488984)
\curveto(344.16599644,536.07888542)(344.31178886,535.66080423)(344.59051236,535.36064502)
\curveto(344.86923044,535.06477053)(345.21655943,534.91683411)(345.63250037,534.91683531)
\curveto(345.91550543,534.91683411)(346.1534901,534.99401833)(346.34645512,535.1483882)
\curveto(346.5394112,535.30275521)(346.68520362,535.55145992)(346.7838328,535.89450307)
\moveto(346.8867452,537.7211981)
\curveto(346.87387616,538.24433159)(346.73880377,538.64097272)(346.48152764,538.91112268)
\curveto(346.22424231,539.18555028)(345.91121741,539.32276667)(345.54245202,539.32277227)
\curveto(345.14795124,539.32276667)(344.82206231,539.17911826)(344.56478426,538.89182661)
\curveto(344.30750084,538.60452462)(344.18100448,538.21431551)(344.1852948,537.7211981)
\lineto(346.8867452,537.7211981)
}
}
{
\newrgbcolor{curcolor}{0 0 0}
\pscustom[linestyle=none,fillstyle=solid,fillcolor=curcolor]
{
\newpath
\moveto(349.52387521,535.6693822)
\lineto(351.33770619,535.94595927)
\curveto(351.41488829,535.59434004)(351.57140073,535.32633928)(351.807244,535.14195617)
\curveto(352.04308208,534.96185824)(352.37325902,534.87180998)(352.79777581,534.87181113)
\curveto(353.26516556,534.87180998)(353.61678256,534.95757023)(353.85262787,535.12909212)
\curveto(354.01127969,535.24915506)(354.09060791,535.40995552)(354.09061279,535.61149398)
\curveto(354.09060791,535.74870848)(354.04772779,535.8623408)(353.96197229,535.95239129)
\curveto(353.87191929,536.03814931)(353.67038272,536.11747753)(353.35736197,536.19037621)
\curveto(351.89943367,536.51197466)(350.97536704,536.80570349)(350.58515929,537.0715636)
\curveto(350.04486838,537.4403293)(349.77472361,537.95274676)(349.77472417,538.60881752)
\curveto(349.77472361,539.20055832)(350.00842028,539.69796774)(350.47581487,540.10104726)
\curveto(350.94320694,540.50411403)(351.66788101,540.70565061)(352.64983924,540.70565759)
\curveto(353.58462247,540.70565061)(354.27928045,540.55342617)(354.73381527,540.24898383)
\curveto(355.18833904,539.94452844)(355.50136393,539.49428716)(355.67289088,538.89825863)
\lineto(353.96840432,538.58308942)
\curveto(353.89550336,538.84894132)(353.75614296,539.0526219)(353.55032271,539.19413177)
\curveto(353.3487818,539.3356307)(353.05934098,539.4063829)(352.68199937,539.40638859)
\curveto(352.20602654,539.4063829)(351.86512957,539.33991872)(351.65930743,539.20699582)
\curveto(351.52208859,539.11265407)(351.4534804,538.99044572)(351.45348264,538.84037041)
\curveto(351.4534804,538.71172492)(351.51351257,538.60238061)(351.63357933,538.51233715)
\curveto(351.79652138,538.39226801)(352.35825098,538.22289153)(353.31876982,538.00420719)
\curveto(354.28356846,537.78551429)(354.95678638,537.51751352)(355.33842559,537.2002041)
\curveto(355.71576455,536.8785997)(355.90443708,536.43050242)(355.90444377,535.85591092)
\curveto(355.90443708,535.229859)(355.64286834,534.69171347)(355.11973675,534.24147271)
\curveto(354.59659336,533.7912309)(353.82260715,533.56611026)(352.79777581,533.56611011)
\curveto(351.86727358,533.56611026)(351.12973548,533.7547828)(350.58515929,534.13212829)
\curveto(350.04486838,534.50947295)(349.69110738,535.02189041)(349.52387521,535.6693822)
}
}
{
\newrgbcolor{curcolor}{0 1 0.25098041}
\pscustom[linewidth=2.63455725,linecolor=curcolor]
{
\newpath
\moveto(160.17532,404.10576156)
\lineto(244.48116,404.10576156)
\lineto(244.48116,456.79691156)
\lineto(187.83818,456.79691156)
\lineto(187.83818,467.33514156)
\lineto(160.17532,467.33514156)
\lineto(160.17532,404.10576156)
\closepath
}
}
{
\newrgbcolor{curcolor}{0 0 0}
\pscustom[linestyle=none,fillstyle=solid,fillcolor=curcolor]
{
\newpath
\moveto(175.38576753,439.78670731)
\lineto(175.38576753,440.86728747)
\lineto(181.63126357,443.50441762)
\lineto(181.63126357,442.35308519)
\lineto(176.67860451,440.32056536)
\lineto(181.63126357,438.26874946)
\lineto(181.63126357,437.11741703)
\lineto(175.38576753,439.78670731)
}
}
{
\newrgbcolor{curcolor}{0 0 0}
\pscustom[linestyle=none,fillstyle=solid,fillcolor=curcolor]
{
\newpath
\moveto(183.07846935,439.78670731)
\lineto(183.07846935,440.86728747)
\lineto(189.32396539,443.50441762)
\lineto(189.32396539,442.35308519)
\lineto(184.37130632,440.32056536)
\lineto(189.32396539,438.26874946)
\lineto(189.32396539,437.11741703)
\lineto(183.07846935,439.78670731)
}
}
{
\newrgbcolor{curcolor}{0 0 0}
\pscustom[linestyle=none,fillstyle=solid,fillcolor=curcolor]
{
\newpath
\moveto(195.37650089,436.50637468)
\curveto(194.94769435,436.1418928)(194.53390117,435.88461206)(194.13512011,435.73453171)
\curveto(193.74061891,435.58445121)(193.3161057,435.50941099)(192.86157921,435.50941084)
\curveto(192.11117426,435.50941099)(191.53443662,435.69165151)(191.13136455,436.05613294)
\curveto(190.72829032,436.4249016)(190.52675375,436.89443894)(190.52675422,437.46474637)
\curveto(190.52675375,437.79920952)(190.60179396,438.10365839)(190.75187509,438.37809388)
\curveto(190.90624283,438.65681196)(191.1056354,438.8797886)(191.35005339,439.04702446)
\curveto(191.5987568,439.21425355)(191.8774776,439.34074991)(192.18621661,439.42651392)
\curveto(192.41347912,439.48654233)(192.7565201,439.54443049)(193.21534057,439.60017859)
\curveto(194.15012407,439.71166297)(194.83835003,439.84459135)(195.28002052,439.99896412)
\curveto(195.2843033,440.15761624)(195.28644731,440.25838453)(195.28645255,440.30126929)
\curveto(195.28644731,440.77294599)(195.177103,441.10526694)(194.95841928,441.29823312)
\curveto(194.66254153,441.55979624)(194.22302028,441.69058061)(193.63985421,441.69058663)
\curveto(193.09527307,441.69058061)(192.69219992,441.59410033)(192.43063355,441.40114552)
\curveto(192.17335044,441.21246725)(191.9825339,440.87585829)(191.85818335,440.39131763)
\lineto(190.72614699,440.54568623)
\curveto(190.82905861,441.03022673)(190.99843509,441.42043584)(191.23427695,441.71631473)
\curveto(191.47011643,442.01646954)(191.81101341,442.24587819)(192.25696888,442.40454138)
\curveto(192.70291995,442.56747911)(193.21962542,442.64895134)(193.80708685,442.64895832)
\curveto(194.39025276,442.64895134)(194.86407811,442.58034314)(195.22856432,442.44313353)
\curveto(195.59304018,442.30591036)(195.86104095,442.13224587)(196.03256742,441.92213953)
\curveto(196.20408193,441.71630868)(196.32414627,441.45473994)(196.39276081,441.13743251)
\curveto(196.43134657,440.94017847)(196.45064263,440.58427346)(196.45064903,440.06971639)
\lineto(196.45064903,438.52603045)
\curveto(196.45064263,437.44973652)(196.4742267,436.76794258)(196.5214013,436.48064658)
\curveto(196.57285098,436.19763695)(196.67147526,435.92534818)(196.81727444,435.66377943)
\lineto(195.60805378,435.66377943)
\curveto(195.48798388,435.90390812)(195.41079966,436.18477292)(195.37650089,436.50637468)
\moveto(195.28002052,439.09204863)
\curveto(194.85979009,438.92052471)(194.2294523,438.7747323)(193.38900524,438.65467095)
\curveto(192.91303255,438.58605976)(192.57642359,438.50887554)(192.37917735,438.42311806)
\curveto(192.18192646,438.33735505)(192.02970203,438.21085869)(191.9225036,438.04362859)
\curveto(191.81530142,437.88068175)(191.76170127,437.69844123)(191.76170298,437.49690649)
\curveto(191.76170127,437.18816778)(191.8774776,436.93088704)(192.10903231,436.72506352)
\curveto(192.34487093,436.51923787)(192.68791191,436.41632558)(193.13815628,436.41632633)
\curveto(193.58410646,436.41632558)(193.98074759,436.51280585)(194.32808086,436.70576744)
\curveto(194.67540557,436.90301496)(194.9305423,437.17101573)(195.0934918,437.50977054)
\curveto(195.21783912,437.77133744)(195.28001529,438.15725854)(195.28002052,438.667535)
\lineto(195.28002052,439.09204863)
}
}
{
\newrgbcolor{curcolor}{0 0 0}
\pscustom[linestyle=none,fillstyle=solid,fillcolor=curcolor]
{
\newpath
\moveto(198.25161487,433.04594535)
\lineto(198.25161487,442.49458973)
\lineto(199.30646693,442.49458973)
\lineto(199.30646693,441.60697031)
\curveto(199.55516971,441.95429336)(199.83603451,442.2137181)(200.14906217,442.38524531)
\curveto(200.4620843,442.56104709)(200.84157338,442.64895134)(201.28753055,442.64895832)
\curveto(201.87069631,442.64895134)(202.38525778,442.49887091)(202.8312165,442.19871659)
\curveto(203.27716432,441.8985492)(203.61377328,441.47403599)(203.84104438,440.92517569)
\curveto(204.06830257,440.38059287)(204.1819349,439.78241517)(204.1819417,439.13064078)
\curveto(204.1819349,438.43169132)(204.05543854,437.80135352)(203.80245224,437.2396255)
\curveto(203.55374111,436.68218234)(203.18926007,436.25338111)(202.70900803,435.95322055)
\curveto(202.23303334,435.65734741)(201.73133591,435.50941099)(201.20391423,435.50941084)
\curveto(200.81798931,435.50941099)(200.47066032,435.59088323)(200.16192622,435.75382778)
\curveto(199.85747457,435.91677215)(199.60662586,436.12259674)(199.40937932,436.37130216)
\lineto(199.40937932,433.04594535)
\lineto(198.25161487,433.04594535)
\moveto(199.3000349,439.04059243)
\curveto(199.30003299,438.16154655)(199.47798549,437.5119127)(199.83389296,437.09168893)
\curveto(200.18979552,436.6714623)(200.62074075,436.46134971)(201.12672993,436.4613505)
\curveto(201.64128766,436.46134971)(202.08080891,436.67789432)(202.44529501,437.110985)
\curveto(202.814059,437.5483608)(202.99844353,438.22372273)(202.99844914,439.1370728)
\curveto(202.99844353,440.00753581)(202.81834701,440.65931367)(202.45815906,441.09240833)
\curveto(202.10224897,441.52549214)(201.67559176,441.74203675)(201.17818613,441.74204283)
\curveto(200.68506093,441.74203675)(200.24768369,441.51048409)(199.86605308,441.04738416)
\curveto(199.48870552,440.58856147)(199.30003299,439.91963156)(199.3000349,439.04059243)
}
}
{
\newrgbcolor{curcolor}{0 0 0}
\pscustom[linestyle=none,fillstyle=solid,fillcolor=curcolor]
{
\newpath
\moveto(205.58412295,433.04594535)
\lineto(205.58412295,442.49458973)
\lineto(206.63897501,442.49458973)
\lineto(206.63897501,441.60697031)
\curveto(206.8876778,441.95429336)(207.1685426,442.2137181)(207.48157026,442.38524531)
\curveto(207.79459238,442.56104709)(208.17408147,442.64895134)(208.62003864,442.64895832)
\curveto(209.2032044,442.64895134)(209.71776587,442.49887091)(210.16372458,442.19871659)
\curveto(210.60967241,441.8985492)(210.94628137,441.47403599)(211.17355247,440.92517569)
\curveto(211.40081066,440.38059287)(211.51444299,439.78241517)(211.51444978,439.13064078)
\curveto(211.51444299,438.43169132)(211.38794662,437.80135352)(211.13496032,437.2396255)
\curveto(210.8862492,436.68218234)(210.52176816,436.25338111)(210.04151611,435.95322055)
\curveto(209.56554143,435.65734741)(209.063844,435.50941099)(208.53642232,435.50941084)
\curveto(208.1504974,435.50941099)(207.80316841,435.59088323)(207.49443431,435.75382778)
\curveto(207.18998266,435.91677215)(206.93913395,436.12259674)(206.74188741,436.37130216)
\lineto(206.74188741,433.04594535)
\lineto(205.58412295,433.04594535)
\moveto(206.63254299,439.04059243)
\curveto(206.63254107,438.16154655)(206.81049358,437.5119127)(207.16640104,437.09168893)
\curveto(207.52230361,436.6714623)(207.95324884,436.46134971)(208.45923802,436.4613505)
\curveto(208.97379575,436.46134971)(209.413317,436.67789432)(209.7778031,437.110985)
\curveto(210.14656709,437.5483608)(210.33095161,438.22372273)(210.33095723,439.1370728)
\curveto(210.33095161,440.00753581)(210.1508551,440.65931367)(209.79066715,441.09240833)
\curveto(209.43475706,441.52549214)(209.00809984,441.74203675)(208.51069422,441.74204283)
\curveto(208.01756902,441.74203675)(207.58019177,441.51048409)(207.19856117,441.04738416)
\curveto(206.82121361,440.58856147)(206.63254107,439.91963156)(206.63254299,439.04059243)
}
}
{
\newrgbcolor{curcolor}{0 0 0}
\pscustom[linestyle=none,fillstyle=solid,fillcolor=curcolor]
{
\newpath
\moveto(219.01419052,439.78670731)
\lineto(212.76869447,437.11741703)
\lineto(212.76869447,438.26874946)
\lineto(217.71492151,440.32056536)
\lineto(212.76869447,442.35308519)
\lineto(212.76869447,443.50441762)
\lineto(219.01419052,440.86728747)
\lineto(219.01419052,439.78670731)
}
}
{
\newrgbcolor{curcolor}{0 0 0}
\pscustom[linestyle=none,fillstyle=solid,fillcolor=curcolor]
{
\newpath
\moveto(226.70689376,439.78670731)
\lineto(220.46139772,437.11741703)
\lineto(220.46139772,438.26874946)
\lineto(225.40762476,440.32056536)
\lineto(220.46139772,442.35308519)
\lineto(220.46139772,443.50441762)
\lineto(226.70689376,440.86728747)
\lineto(226.70689376,439.78670731)
}
}
{
\newrgbcolor{curcolor}{0 0 0}
\pscustom[linestyle=none,fillstyle=solid,fillcolor=curcolor]
{
\newpath
\moveto(192.3624781,415.90456312)
\lineto(190.55507914,415.90456312)
\lineto(190.55507914,419.39072055)
\curveto(190.55507379,420.12825516)(190.51648168,420.60422452)(190.4393027,420.81863004)
\curveto(190.36211324,421.03731375)(190.23561688,421.20669023)(190.05981324,421.32676)
\curveto(189.88828789,421.44681892)(189.6803193,421.50685109)(189.43590683,421.50685669)
\curveto(189.12287771,421.50685109)(188.84201291,421.42109085)(188.59331159,421.2495757)
\curveto(188.34460349,421.07804987)(188.173083,420.85078522)(188.07874961,420.56778108)
\curveto(187.98869848,420.28476761)(187.94367435,419.76163012)(187.94367709,418.99836703)
\lineto(187.94367709,415.90456312)
\lineto(186.13627813,415.90456312)
\lineto(186.13627813,422.73537342)
\lineto(187.81503659,422.73537342)
\lineto(187.81503659,421.73197756)
\curveto(188.41106768,422.50381393)(189.16146982,422.88973503)(190.06624526,422.88974202)
\curveto(190.46502553,422.88973503)(190.82950657,422.81683882)(191.15968947,422.67105317)
\curveto(191.48986046,422.529542)(191.73856516,422.34730148)(191.90580434,422.12433107)
\curveto(192.07731813,421.90134821)(192.19523847,421.64835549)(192.2595657,421.36535215)
\curveto(192.32816684,421.08233788)(192.36247094,420.67712073)(192.3624781,420.14969947)
\lineto(192.3624781,415.90456312)
}
}
{
\newrgbcolor{curcolor}{0 0 0}
\pscustom[linestyle=none,fillstyle=solid,fillcolor=curcolor]
{
\newpath
\moveto(193.78395516,419.41644864)
\curveto(193.78395464,420.01676684)(193.93189106,420.5977925)(194.22776487,421.15952735)
\curveto(194.52363674,421.7212517)(194.94171794,422.15005292)(195.4820097,422.44593231)
\curveto(196.02658503,422.74179861)(196.63333876,422.88973503)(197.30227271,422.88974202)
\curveto(198.33567961,422.88973503)(199.18256202,422.55312607)(199.84292249,421.87991413)
\curveto(200.50326979,421.21097825)(200.83344673,420.36409583)(200.8334543,419.33926435)
\curveto(200.83344673,418.30584997)(200.49898177,417.44824752)(199.83005844,416.76645444)
\curveto(199.16540997,416.08894765)(198.32710358,415.75019468)(197.31513676,415.75019453)
\curveto(196.68908292,415.75019468)(196.09090521,415.89169909)(195.52060185,416.17470816)
\curveto(194.95458197,416.4577167)(194.52363674,416.87150988)(194.22776487,417.41608894)
\curveto(193.93189106,417.964953)(193.78395464,418.6317389)(193.78395516,419.41644864)
\moveto(195.6363783,419.31996827)
\curveto(195.63637592,418.64245893)(195.79717637,418.12360945)(196.11878015,417.76341828)
\curveto(196.44037821,417.4032234)(196.83701934,417.22312688)(197.30870473,417.2231282)
\curveto(197.78038203,417.22312688)(198.17487915,417.4032234)(198.49219729,417.76341828)
\curveto(198.81379297,418.12360945)(198.97459343,418.64674694)(198.97459915,419.33283232)
\curveto(198.97459343,420.0017588)(198.81379297,420.51632027)(198.49219729,420.87651827)
\curveto(198.17487915,421.23670632)(197.78038203,421.41680283)(197.30870473,421.41680835)
\curveto(196.83701934,421.41680283)(196.44037821,421.23670632)(196.11878015,420.87651827)
\curveto(195.79717637,420.51632027)(195.63637592,419.99747079)(195.6363783,419.31996827)
}
}
{
\newrgbcolor{curcolor}{0 0 0}
\pscustom[linestyle=none,fillstyle=solid,fillcolor=curcolor]
{
\newpath
\moveto(205.38732742,422.73537342)
\lineto(205.38732742,421.29459987)
\lineto(204.15237867,421.29459987)
\lineto(204.15237867,418.54169328)
\curveto(204.15237583,417.98424905)(204.16309586,417.65836012)(204.18453879,417.56402551)
\curveto(204.21026399,417.4739756)(204.26386415,417.39893538)(204.34533941,417.33890465)
\curveto(204.43109662,417.27887104)(204.53400892,417.24885496)(204.6540766,417.2488563)
\curveto(204.82130573,417.24885496)(205.06357842,417.30674312)(205.3808954,417.42252097)
\lineto(205.53526399,416.02033957)
\curveto(205.11503457,415.84024294)(204.63906521,415.75019468)(204.1073545,415.75019453)
\curveto(203.78146277,415.75019468)(203.48773393,415.80379484)(203.2261671,415.91099515)
\curveto(202.96459644,416.02248346)(202.77163589,416.16398786)(202.64728488,416.33550878)
\curveto(202.5272192,416.51131685)(202.44360296,416.74715753)(202.39643591,417.04303151)
\curveto(202.35784271,417.25314297)(202.33854666,417.67765618)(202.33854769,418.31657241)
\lineto(202.33854769,421.29459987)
\lineto(201.50881649,421.29459987)
\lineto(201.50881649,422.73537342)
\lineto(202.33854769,422.73537342)
\lineto(202.33854769,424.09253065)
\lineto(204.15237867,425.14738271)
\lineto(204.15237867,422.73537342)
\lineto(205.38732742,422.73537342)
}
}
{
\newrgbcolor{curcolor}{0 0 0}
\pscustom[linestyle=none,fillstyle=solid,fillcolor=curcolor]
{
\newpath
\moveto(210.61013221,418.07858749)
\lineto(212.41109914,417.77628233)
\curveto(212.17953978,417.11592658)(211.81291474,416.61208514)(211.31122291,416.26475651)
\curveto(210.81380789,415.92171517)(210.18990211,415.75019468)(209.4395037,415.75019453)
\curveto(208.25172059,415.75019468)(207.37267808,416.13825979)(206.80237355,416.91439101)
\curveto(206.35213117,417.53615177)(206.12701053,418.32085801)(206.12701095,419.26851207)
\curveto(206.12701053,420.40054394)(206.42288338,421.28601846)(207.01463037,421.9249383)
\curveto(207.60637475,422.56813411)(208.35463288,422.88973503)(209.25940701,422.88974202)
\curveto(210.27566236,422.88973503)(211.07752064,422.55312607)(211.66498427,421.87991413)
\curveto(212.25243599,421.21097825)(212.53330079,420.18399932)(212.50757951,418.79897427)
\lineto(207.97943408,418.79897427)
\curveto(207.99229585,418.26296985)(208.13808826,417.84488865)(208.41681177,417.54472944)
\curveto(208.69552985,417.24885496)(209.04285884,417.10091853)(209.45879978,417.10091973)
\curveto(209.74180483,417.10091853)(209.97978951,417.17810275)(210.17275453,417.33247262)
\curveto(210.36571061,417.48683963)(210.51150303,417.73554434)(210.61013221,418.07858749)
\moveto(210.71304461,419.90528253)
\curveto(210.70017556,420.42841602)(210.56510318,420.82505715)(210.30782705,421.09520711)
\curveto(210.05054171,421.3696347)(209.73751682,421.50685109)(209.36875143,421.50685669)
\curveto(208.97425065,421.50685109)(208.64836172,421.36320268)(208.39108367,421.07591103)
\curveto(208.13380025,420.78860904)(208.00730389,420.39839993)(208.01159421,419.90528253)
\lineto(210.71304461,419.90528253)
}
}
{
\newrgbcolor{curcolor}{0 0 0}
\pscustom[linestyle=none,fillstyle=solid,fillcolor=curcolor]
{
\newpath
\moveto(213.35017462,417.85346663)
\lineto(215.1640056,418.13004369)
\curveto(215.2411877,417.77842446)(215.39770014,417.5104237)(215.63354341,417.3260406)
\curveto(215.86938149,417.14594266)(216.19955843,417.05589441)(216.62407522,417.05589556)
\curveto(217.09146497,417.05589441)(217.44308197,417.14165465)(217.67892728,417.31317655)
\curveto(217.8375791,417.43323948)(217.91690732,417.59403994)(217.9169122,417.7955784)
\curveto(217.91690732,417.9327929)(217.8740272,418.04642523)(217.7882717,418.13647572)
\curveto(217.6982187,418.22223373)(217.49668212,418.30156196)(217.18366137,418.37446063)
\curveto(215.72573308,418.69605908)(214.80166644,418.98978792)(214.4114587,419.25564803)
\curveto(213.87116779,419.62441373)(213.60102302,420.13683119)(213.60102358,420.79290194)
\curveto(213.60102302,421.38464274)(213.83471969,421.88205216)(214.30211428,422.28513169)
\curveto(214.76950635,422.68819846)(215.49418042,422.88973503)(216.47613865,422.88974202)
\curveto(217.41092188,422.88973503)(218.10557986,422.7375106)(218.56011467,422.43306826)
\curveto(219.01463845,422.12861286)(219.32766334,421.67837158)(219.49919029,421.08234306)
\lineto(217.79470373,420.76717384)
\curveto(217.72180277,421.03302574)(217.58244237,421.23670632)(217.37662212,421.3782162)
\curveto(217.17508121,421.51971513)(216.88564038,421.59046733)(216.50829877,421.59047301)
\curveto(216.03232595,421.59046733)(215.69142898,421.52400314)(215.48560684,421.39108025)
\curveto(215.348388,421.29673849)(215.27977981,421.17453014)(215.27978205,421.02445483)
\curveto(215.27977981,420.89580935)(215.33981198,420.78646504)(215.45987874,420.69642157)
\curveto(215.62282078,420.57635244)(216.18455039,420.40697596)(217.14506923,420.18829162)
\curveto(218.10986787,419.96959871)(218.78308579,419.70159795)(219.164725,419.38428852)
\curveto(219.54206395,419.06268412)(219.73073649,418.61458685)(219.73074318,418.03999535)
\curveto(219.73073649,417.41394343)(219.46916775,416.87579789)(218.94603616,416.42555713)
\curveto(218.42289276,415.97531533)(217.64890656,415.75019468)(216.62407522,415.75019453)
\curveto(215.69357299,415.75019468)(214.95603488,415.93886722)(214.4114587,416.31621271)
\curveto(213.87116779,416.69355737)(213.51740678,417.20597483)(213.35017462,417.85346663)
}
}
{
\newrgbcolor{curcolor}{0 1 0.25098041}
\pscustom[linewidth=2.63455725,linecolor=curcolor]
{
\newpath
\moveto(160.17532,165.29658756)
\lineto(244.48116,165.29658756)
\lineto(244.48116,217.98773756)
\lineto(187.83818,217.98773756)
\lineto(187.83818,228.52596756)
\lineto(160.17532,228.52596756)
\lineto(160.17532,165.29658756)
\closepath
}
}
{
\newrgbcolor{curcolor}{0 0 0}
\pscustom[linestyle=none,fillstyle=solid,fillcolor=curcolor]
{
\newpath
\moveto(175.38576753,200.97752931)
\lineto(175.38576753,202.05810947)
\lineto(181.63126357,204.69523962)
\lineto(181.63126357,203.54390719)
\lineto(176.67860451,201.51138736)
\lineto(181.63126357,199.45957146)
\lineto(181.63126357,198.30823903)
\lineto(175.38576753,200.97752931)
}
}
{
\newrgbcolor{curcolor}{0 0 0}
\pscustom[linestyle=none,fillstyle=solid,fillcolor=curcolor]
{
\newpath
\moveto(183.07846935,200.97752931)
\lineto(183.07846935,202.05810947)
\lineto(189.32396539,204.69523962)
\lineto(189.32396539,203.54390719)
\lineto(184.37130632,201.51138736)
\lineto(189.32396539,199.45957146)
\lineto(189.32396539,198.30823903)
\lineto(183.07846935,200.97752931)
}
}
{
\newrgbcolor{curcolor}{0 0 0}
\pscustom[linestyle=none,fillstyle=solid,fillcolor=curcolor]
{
\newpath
\moveto(195.37650089,197.69719668)
\curveto(194.94769435,197.3327148)(194.53390117,197.07543406)(194.13512011,196.92535371)
\curveto(193.74061891,196.77527321)(193.3161057,196.70023299)(192.86157921,196.70023284)
\curveto(192.11117426,196.70023299)(191.53443662,196.88247351)(191.13136455,197.24695494)
\curveto(190.72829032,197.6157236)(190.52675375,198.08526094)(190.52675422,198.65556837)
\curveto(190.52675375,198.99003152)(190.60179396,199.29448039)(190.75187509,199.56891588)
\curveto(190.90624283,199.84763396)(191.1056354,200.0706106)(191.35005339,200.23784646)
\curveto(191.5987568,200.40507555)(191.8774776,200.53157191)(192.18621661,200.61733592)
\curveto(192.41347912,200.67736433)(192.7565201,200.73525249)(193.21534057,200.79100059)
\curveto(194.15012407,200.90248497)(194.83835003,201.03541335)(195.28002052,201.18978612)
\curveto(195.2843033,201.34843824)(195.28644731,201.44920653)(195.28645255,201.49209129)
\curveto(195.28644731,201.96376799)(195.177103,202.29608894)(194.95841928,202.48905513)
\curveto(194.66254153,202.75061824)(194.22302028,202.88140261)(193.63985421,202.88140864)
\curveto(193.09527307,202.88140261)(192.69219992,202.78492233)(192.43063355,202.59196752)
\curveto(192.17335044,202.40328925)(191.9825339,202.06668029)(191.85818335,201.58213963)
\lineto(190.72614699,201.73650823)
\curveto(190.82905861,202.22104873)(190.99843509,202.61125784)(191.23427695,202.90713674)
\curveto(191.47011643,203.20729154)(191.81101341,203.43670019)(192.25696888,203.59536338)
\curveto(192.70291995,203.75830111)(193.21962542,203.83977334)(193.80708685,203.83978033)
\curveto(194.39025276,203.83977334)(194.86407811,203.77116515)(195.22856432,203.63395553)
\curveto(195.59304018,203.49673236)(195.86104095,203.32306787)(196.03256742,203.11296153)
\curveto(196.20408193,202.90713068)(196.32414627,202.64556194)(196.39276081,202.32825451)
\curveto(196.43134657,202.13100047)(196.45064263,201.77509546)(196.45064903,201.2605384)
\lineto(196.45064903,199.71685245)
\curveto(196.45064263,198.64055852)(196.4742267,197.95876458)(196.5214013,197.67146858)
\curveto(196.57285098,197.38845896)(196.67147526,197.11617018)(196.81727444,196.85460143)
\lineto(195.60805378,196.85460143)
\curveto(195.48798388,197.09473012)(195.41079966,197.37559492)(195.37650089,197.69719668)
\moveto(195.28002052,200.28287063)
\curveto(194.85979009,200.11134672)(194.2294523,199.9655543)(193.38900524,199.84549295)
\curveto(192.91303255,199.77688176)(192.57642359,199.69969754)(192.37917735,199.61394006)
\curveto(192.18192646,199.52817705)(192.02970203,199.40168069)(191.9225036,199.2344506)
\curveto(191.81530142,199.07150375)(191.76170127,198.88926323)(191.76170298,198.68772849)
\curveto(191.76170127,198.37898978)(191.8774776,198.12170905)(192.10903231,197.91588552)
\curveto(192.34487093,197.71005987)(192.68791191,197.60714758)(193.13815628,197.60714833)
\curveto(193.58410646,197.60714758)(193.98074759,197.70362785)(194.32808086,197.89658945)
\curveto(194.67540557,198.09383697)(194.9305423,198.36183773)(195.0934918,198.70059254)
\curveto(195.21783912,198.96215944)(195.28001529,199.34808054)(195.28002052,199.858357)
\lineto(195.28002052,200.28287063)
}
}
{
\newrgbcolor{curcolor}{0 0 0}
\pscustom[linestyle=none,fillstyle=solid,fillcolor=curcolor]
{
\newpath
\moveto(198.25161487,194.23676736)
\lineto(198.25161487,203.68541173)
\lineto(199.30646693,203.68541173)
\lineto(199.30646693,202.79779231)
\curveto(199.55516971,203.14511536)(199.83603451,203.4045401)(200.14906217,203.57606731)
\curveto(200.4620843,203.75186909)(200.84157338,203.83977334)(201.28753055,203.83978033)
\curveto(201.87069631,203.83977334)(202.38525778,203.68969291)(202.8312165,203.38953859)
\curveto(203.27716432,203.0893712)(203.61377328,202.66485799)(203.84104438,202.11599769)
\curveto(204.06830257,201.57141488)(204.1819349,200.97323717)(204.1819417,200.32146278)
\curveto(204.1819349,199.62251332)(204.05543854,198.99217553)(203.80245224,198.4304475)
\curveto(203.55374111,197.87300434)(203.18926007,197.44420311)(202.70900803,197.14404255)
\curveto(202.23303334,196.84816942)(201.73133591,196.70023299)(201.20391423,196.70023284)
\curveto(200.81798931,196.70023299)(200.47066032,196.78170523)(200.16192622,196.94464978)
\curveto(199.85747457,197.10759416)(199.60662586,197.31341874)(199.40937932,197.56212416)
\lineto(199.40937932,194.23676736)
\lineto(198.25161487,194.23676736)
\moveto(199.3000349,200.23141443)
\curveto(199.30003299,199.35236855)(199.47798549,198.7027347)(199.83389296,198.28251093)
\curveto(200.18979552,197.86228431)(200.62074075,197.65217171)(201.12672993,197.6521725)
\curveto(201.64128766,197.65217171)(202.08080891,197.86871632)(202.44529501,198.30180701)
\curveto(202.814059,198.73918281)(202.99844353,199.41454473)(202.99844914,200.32789481)
\curveto(202.99844353,201.19835781)(202.81834701,201.85013567)(202.45815906,202.28323033)
\curveto(202.10224897,202.71631414)(201.67559176,202.93285876)(201.17818613,202.93286483)
\curveto(200.68506093,202.93285876)(200.24768369,202.7013061)(199.86605308,202.23820616)
\curveto(199.48870552,201.77938347)(199.30003299,201.11045356)(199.3000349,200.23141443)
}
}
{
\newrgbcolor{curcolor}{0 0 0}
\pscustom[linestyle=none,fillstyle=solid,fillcolor=curcolor]
{
\newpath
\moveto(205.58412295,194.23676736)
\lineto(205.58412295,203.68541173)
\lineto(206.63897501,203.68541173)
\lineto(206.63897501,202.79779231)
\curveto(206.8876778,203.14511536)(207.1685426,203.4045401)(207.48157026,203.57606731)
\curveto(207.79459238,203.75186909)(208.17408147,203.83977334)(208.62003864,203.83978033)
\curveto(209.2032044,203.83977334)(209.71776587,203.68969291)(210.16372458,203.38953859)
\curveto(210.60967241,203.0893712)(210.94628137,202.66485799)(211.17355247,202.11599769)
\curveto(211.40081066,201.57141488)(211.51444299,200.97323717)(211.51444978,200.32146278)
\curveto(211.51444299,199.62251332)(211.38794662,198.99217553)(211.13496032,198.4304475)
\curveto(210.8862492,197.87300434)(210.52176816,197.44420311)(210.04151611,197.14404255)
\curveto(209.56554143,196.84816942)(209.063844,196.70023299)(208.53642232,196.70023284)
\curveto(208.1504974,196.70023299)(207.80316841,196.78170523)(207.49443431,196.94464978)
\curveto(207.18998266,197.10759416)(206.93913395,197.31341874)(206.74188741,197.56212416)
\lineto(206.74188741,194.23676736)
\lineto(205.58412295,194.23676736)
\moveto(206.63254299,200.23141443)
\curveto(206.63254107,199.35236855)(206.81049358,198.7027347)(207.16640104,198.28251093)
\curveto(207.52230361,197.86228431)(207.95324884,197.65217171)(208.45923802,197.6521725)
\curveto(208.97379575,197.65217171)(209.413317,197.86871632)(209.7778031,198.30180701)
\curveto(210.14656709,198.73918281)(210.33095161,199.41454473)(210.33095723,200.32789481)
\curveto(210.33095161,201.19835781)(210.1508551,201.85013567)(209.79066715,202.28323033)
\curveto(209.43475706,202.71631414)(209.00809984,202.93285876)(208.51069422,202.93286483)
\curveto(208.01756902,202.93285876)(207.58019177,202.7013061)(207.19856117,202.23820616)
\curveto(206.82121361,201.77938347)(206.63254107,201.11045356)(206.63254299,200.23141443)
}
}
{
\newrgbcolor{curcolor}{0 0 0}
\pscustom[linestyle=none,fillstyle=solid,fillcolor=curcolor]
{
\newpath
\moveto(219.01419052,200.97752931)
\lineto(212.76869447,198.30823903)
\lineto(212.76869447,199.45957146)
\lineto(217.71492151,201.51138736)
\lineto(212.76869447,203.54390719)
\lineto(212.76869447,204.69523962)
\lineto(219.01419052,202.05810947)
\lineto(219.01419052,200.97752931)
}
}
{
\newrgbcolor{curcolor}{0 0 0}
\pscustom[linestyle=none,fillstyle=solid,fillcolor=curcolor]
{
\newpath
\moveto(226.70689376,200.97752931)
\lineto(220.46139772,198.30823903)
\lineto(220.46139772,199.45957146)
\lineto(225.40762476,201.51138736)
\lineto(220.46139772,203.54390719)
\lineto(220.46139772,204.69523962)
\lineto(226.70689376,202.05810947)
\lineto(226.70689376,200.97752931)
}
}
{
\newrgbcolor{curcolor}{0 0 0}
\pscustom[linestyle=none,fillstyle=solid,fillcolor=curcolor]
{
\newpath
\moveto(189.30981588,183.92619542)
\lineto(190.31321174,183.92619542)
\lineto(190.31321174,184.4407574)
\curveto(190.31321058,185.0153437)(190.37324276,185.44414492)(190.49330844,185.72716236)
\curveto(190.61765945,186.01016253)(190.84278009,186.23957118)(191.16867104,186.41538901)
\curveto(191.49884596,186.5954762)(191.91478315,186.68552446)(192.41648384,186.68553405)
\curveto(192.93104204,186.68552446)(193.43488348,186.60834024)(193.92800966,186.45398115)
\lineto(193.68359272,185.1933043)
\curveto(193.39629137,185.2619044)(193.11971458,185.2962085)(192.85386152,185.2962167)
\curveto(192.59228908,185.2962085)(192.40361654,185.23403232)(192.28784334,185.10968798)
\curveto(192.17635189,184.98961562)(192.12060774,184.75591896)(192.1206107,184.40859728)
\lineto(192.1206107,183.92619542)
\lineto(193.4713359,183.92619542)
\lineto(193.4713359,182.50471795)
\lineto(192.1206107,182.50471795)
\lineto(192.1206107,177.09538513)
\lineto(190.31321174,177.09538513)
\lineto(190.31321174,182.50471795)
\lineto(189.30981588,182.50471795)
\lineto(189.30981588,183.92619542)
}
}
{
\newrgbcolor{curcolor}{0 0 0}
\pscustom[linestyle=none,fillstyle=solid,fillcolor=curcolor]
{
\newpath
\moveto(194.50046006,184.85240699)
\lineto(194.50046006,186.52473343)
\lineto(196.30785902,186.52473343)
\lineto(196.30785902,184.85240699)
\lineto(194.50046006,184.85240699)
\moveto(194.50046006,177.09538513)
\lineto(194.50046006,183.92619542)
\lineto(196.30785902,183.92619542)
\lineto(196.30785902,177.09538513)
\lineto(194.50046006,177.09538513)
}
}
{
\newrgbcolor{curcolor}{0 0 0}
\pscustom[linestyle=none,fillstyle=solid,fillcolor=curcolor]
{
\newpath
\moveto(198.15385047,177.09538513)
\lineto(198.15385047,186.52473343)
\lineto(199.96124943,186.52473343)
\lineto(199.96124943,177.09538513)
\lineto(198.15385047,177.09538513)
}
}
{
\newrgbcolor{curcolor}{0 0 0}
\pscustom[linestyle=none,fillstyle=solid,fillcolor=curcolor]
{
\newpath
\moveto(205.76293563,179.2694095)
\lineto(207.56390256,178.96710433)
\curveto(207.3323432,178.30674858)(206.96571816,177.80290714)(206.46402633,177.45557851)
\curveto(205.96661131,177.11253717)(205.34270553,176.94101669)(204.59230712,176.94101653)
\curveto(203.40452401,176.94101669)(202.5254815,177.32908179)(201.95517697,178.10521301)
\curveto(201.50493459,178.72697378)(201.27981395,179.51168001)(201.27981437,180.45933408)
\curveto(201.27981395,181.59136594)(201.5756868,182.47684046)(202.16743379,183.1157603)
\curveto(202.75917817,183.75895612)(203.5074363,184.08055703)(204.41221043,184.08056402)
\curveto(205.42846578,184.08055703)(206.23032406,183.74394807)(206.81778769,183.07073613)
\curveto(207.40523941,182.40180025)(207.68610421,181.37482132)(207.66038294,179.98979627)
\lineto(203.1322375,179.98979627)
\curveto(203.14509927,179.45379185)(203.29089168,179.03571066)(203.56961519,178.73555144)
\curveto(203.84833327,178.43967696)(204.19566226,178.29174054)(204.6116032,178.29174173)
\curveto(204.89460825,178.29174054)(205.13259293,178.36892476)(205.32555795,178.52329462)
\curveto(205.51851403,178.67766163)(205.66430645,178.92636634)(205.76293563,179.2694095)
\moveto(205.86584803,181.09610453)
\curveto(205.85297899,181.61923802)(205.7179066,182.01587915)(205.46063047,182.28602911)
\curveto(205.20334513,182.5604567)(204.89032024,182.69767309)(204.52155485,182.69767869)
\curveto(204.12705407,182.69767309)(203.80116514,182.55402468)(203.54388709,182.26673303)
\curveto(203.28660367,181.97943104)(203.16010731,181.58922193)(203.16439763,181.09610453)
\lineto(205.86584803,181.09610453)
}
}
{
\newrgbcolor{curcolor}{0 0 0}
\pscustom[linestyle=none,fillstyle=solid,fillcolor=curcolor]
{
\newpath
\moveto(208.50297804,179.04428863)
\lineto(210.31680902,179.32086569)
\curveto(210.39399112,178.96924647)(210.55050356,178.7012457)(210.78634683,178.5168626)
\curveto(211.02218491,178.33676466)(211.35236185,178.24671641)(211.77687864,178.24671756)
\curveto(212.24426839,178.24671641)(212.59588539,178.33247665)(212.8317307,178.50399855)
\curveto(212.99038252,178.62406148)(213.06971074,178.78486194)(213.06971562,178.98640041)
\curveto(213.06971074,179.12361491)(213.02683062,179.23724723)(212.94107512,179.32729772)
\curveto(212.85102212,179.41305573)(212.64948555,179.49238396)(212.3364648,179.56528263)
\curveto(210.8785365,179.88688108)(209.95446987,180.18060992)(209.56426212,180.44647003)
\curveto(209.02397121,180.81523573)(208.75382644,181.32765319)(208.753827,181.98372394)
\curveto(208.75382644,182.57546474)(208.98752311,183.07287416)(209.4549177,183.47595369)
\curveto(209.92230977,183.87902046)(210.64698384,184.08055703)(211.62894207,184.08056402)
\curveto(212.5637253,184.08055703)(213.25838328,183.9283326)(213.71291809,183.62389026)
\curveto(214.16744187,183.31943486)(214.48046676,182.86919358)(214.65199371,182.27316506)
\lineto(212.94750715,181.95799585)
\curveto(212.87460619,182.22384774)(212.73524579,182.42752832)(212.52942554,182.5690382)
\curveto(212.32788463,182.71053713)(212.0384438,182.78128933)(211.6611022,182.78129502)
\curveto(211.18512937,182.78128933)(210.8442324,182.71482514)(210.63841026,182.58190225)
\curveto(210.50119142,182.48756049)(210.43258323,182.36535214)(210.43258547,182.21527684)
\curveto(210.43258323,182.08663135)(210.4926154,181.97728704)(210.61268216,181.88724357)
\curveto(210.77562421,181.76717444)(211.33735381,181.59779796)(212.29787265,181.37911362)
\curveto(213.26267129,181.16042071)(213.93588921,180.89241995)(214.31752842,180.57511052)
\curveto(214.69486737,180.25350613)(214.88353991,179.80540885)(214.8835466,179.23081735)
\curveto(214.88353991,178.60476543)(214.62197117,178.06661989)(214.09883958,177.61637913)
\curveto(213.57569619,177.16613733)(212.80170998,176.94101669)(211.77687864,176.94101653)
\curveto(210.84637641,176.94101669)(210.10883831,177.12968922)(209.56426212,177.50703471)
\curveto(209.02397121,177.88437937)(208.67021021,178.39679683)(208.50297804,179.04428863)
}
}
{
\newrgbcolor{curcolor}{0 1 0.25098041}
\pscustom[linewidth=2.63455725,linecolor=curcolor]
{
\newpath
\moveto(37.36788,165.29658756)
\lineto(121.67372,165.29658756)
\lineto(121.67372,217.98774756)
\lineto(65.03074,217.98774756)
\lineto(65.03074,228.52596756)
\lineto(37.36788,228.52596756)
\lineto(37.36788,165.29658756)
\closepath
}
}
{
\newrgbcolor{curcolor}{0 0 0}
\pscustom[linestyle=none,fillstyle=solid,fillcolor=curcolor]
{
\newpath
\moveto(52.57830251,200.97752931)
\lineto(52.57830251,202.05810947)
\lineto(58.82379855,204.69523962)
\lineto(58.82379855,203.54390719)
\lineto(53.87113948,201.51138736)
\lineto(58.82379855,199.45957146)
\lineto(58.82379855,198.30823903)
\lineto(52.57830251,200.97752931)
}
}
{
\newrgbcolor{curcolor}{0 0 0}
\pscustom[linestyle=none,fillstyle=solid,fillcolor=curcolor]
{
\newpath
\moveto(60.27100432,200.97752931)
\lineto(60.27100432,202.05810947)
\lineto(66.51650037,204.69523962)
\lineto(66.51650037,203.54390719)
\lineto(61.5638413,201.51138736)
\lineto(66.51650037,199.45957146)
\lineto(66.51650037,198.30823903)
\lineto(60.27100432,200.97752931)
}
}
{
\newrgbcolor{curcolor}{0 0 0}
\pscustom[linestyle=none,fillstyle=solid,fillcolor=curcolor]
{
\newpath
\moveto(72.56903587,197.69719668)
\curveto(72.14022932,197.3327148)(71.72643614,197.07543406)(71.32765509,196.92535371)
\curveto(70.93315388,196.77527321)(70.50864067,196.70023299)(70.05411419,196.70023284)
\curveto(69.30370924,196.70023299)(68.72697159,196.88247351)(68.32389953,197.24695494)
\curveto(67.9208253,197.6157236)(67.71928872,198.08526094)(67.7192892,198.65556837)
\curveto(67.71928872,198.99003152)(67.79432894,199.29448039)(67.94441006,199.56891588)
\curveto(68.0987778,199.84763396)(68.29817037,200.0706106)(68.54258837,200.23784646)
\curveto(68.79129178,200.40507555)(69.07001257,200.53157191)(69.37875159,200.61733592)
\curveto(69.6060141,200.67736433)(69.94905508,200.73525249)(70.40787555,200.79100059)
\curveto(71.34265905,200.90248497)(72.03088501,201.03541335)(72.4725555,201.18978612)
\curveto(72.47683828,201.34843824)(72.47898229,201.44920653)(72.47898752,201.49209129)
\curveto(72.47898229,201.96376799)(72.36963797,202.29608894)(72.15095426,202.48905513)
\curveto(71.85507651,202.75061824)(71.41555526,202.88140261)(70.83238918,202.88140864)
\curveto(70.28780804,202.88140261)(69.88473489,202.78492233)(69.62316853,202.59196752)
\curveto(69.36588541,202.40328925)(69.17506887,202.06668029)(69.05071832,201.58213963)
\lineto(67.91868197,201.73650823)
\curveto(68.02159358,202.22104873)(68.19097007,202.61125784)(68.42681192,202.90713673)
\curveto(68.66265141,203.20729154)(69.00354838,203.43670019)(69.44950386,203.59536338)
\curveto(69.89545492,203.75830111)(70.4121604,203.83977334)(70.99962183,203.83978033)
\curveto(71.58278773,203.83977334)(72.05661308,203.77116514)(72.4210993,203.63395553)
\curveto(72.78557516,203.49673236)(73.05357592,203.32306787)(73.22510239,203.11296153)
\curveto(73.3966169,202.90713068)(73.51668124,202.64556194)(73.58529578,202.32825451)
\curveto(73.62388155,202.13100047)(73.6431776,201.77509546)(73.643184,201.2605384)
\lineto(73.643184,199.71685245)
\curveto(73.6431776,198.64055852)(73.66676167,197.95876458)(73.71393628,197.67146858)
\curveto(73.76538595,197.38845895)(73.86401023,197.11617018)(74.00980942,196.85460143)
\lineto(72.80058876,196.85460143)
\curveto(72.68051886,197.09473012)(72.60333464,197.37559492)(72.56903587,197.69719668)
\moveto(72.4725555,200.28287063)
\curveto(72.05232507,200.11134671)(71.42198727,199.9655543)(70.58154022,199.84549295)
\curveto(70.10556752,199.77688176)(69.76895856,199.69969754)(69.57171233,199.61394006)
\curveto(69.37446144,199.52817705)(69.22223701,199.40168069)(69.11503857,199.2344506)
\curveto(69.00783639,199.07150375)(68.95423624,198.88926323)(68.95423795,198.68772849)
\curveto(68.95423624,198.37898978)(69.07001257,198.12170904)(69.30156729,197.91588552)
\curveto(69.5374059,197.71005987)(69.88044688,197.60714758)(70.33069125,197.60714833)
\curveto(70.77664143,197.60714758)(71.17328257,197.70362785)(71.52061583,197.89658945)
\curveto(71.86794054,198.09383697)(72.12307727,198.36183773)(72.28602678,198.70059254)
\curveto(72.41037409,198.96215944)(72.47255027,199.34808054)(72.4725555,199.858357)
\lineto(72.4725555,200.28287063)
}
}
{
\newrgbcolor{curcolor}{0 0 0}
\pscustom[linestyle=none,fillstyle=solid,fillcolor=curcolor]
{
\newpath
\moveto(75.44414984,194.23676736)
\lineto(75.44414984,203.68541173)
\lineto(76.4990019,203.68541173)
\lineto(76.4990019,202.79779231)
\curveto(76.74770469,203.14511536)(77.02856949,203.4045401)(77.34159715,203.57606731)
\curveto(77.65461927,203.75186909)(78.03410835,203.83977334)(78.48006553,203.83978033)
\curveto(79.06323129,203.83977334)(79.57779275,203.68969291)(80.02375147,203.38953859)
\curveto(80.4696993,203.0893712)(80.80630825,202.66485799)(81.03357936,202.11599769)
\curveto(81.26083755,201.57141488)(81.37446987,200.97323717)(81.37447667,200.32146278)
\curveto(81.37446987,199.62251332)(81.24797351,198.99217553)(80.99498721,198.4304475)
\curveto(80.74627608,197.87300434)(80.38179504,197.44420311)(79.901543,197.14404255)
\curveto(79.42556832,196.84816942)(78.92387089,196.70023299)(78.39644921,196.70023284)
\curveto(78.01052429,196.70023299)(77.6631953,196.78170523)(77.3544612,196.94464978)
\curveto(77.05000955,197.10759415)(76.79916084,197.31341874)(76.6019143,197.56212416)
\lineto(76.6019143,194.23676736)
\lineto(75.44414984,194.23676736)
\moveto(76.49256988,200.23141443)
\curveto(76.49256796,199.35236855)(76.67052047,198.7027347)(77.02642793,198.28251093)
\curveto(77.3823305,197.86228431)(77.81327573,197.65217171)(78.31926491,197.6521725)
\curveto(78.83382263,197.65217171)(79.27334389,197.86871632)(79.63782999,198.30180701)
\curveto(80.00659398,198.7391828)(80.1909785,199.41454473)(80.19098412,200.32789481)
\curveto(80.1909785,201.19835781)(80.01088199,201.85013567)(79.65069404,202.28323033)
\curveto(79.29478395,202.71631414)(78.86812673,202.93285876)(78.37072111,202.93286483)
\curveto(77.87759591,202.93285876)(77.44021866,202.7013061)(77.05858806,202.23820616)
\curveto(76.6812405,201.77938347)(76.49256796,201.11045356)(76.49256988,200.23141443)
}
}
{
\newrgbcolor{curcolor}{0 0 0}
\pscustom[linestyle=none,fillstyle=solid,fillcolor=curcolor]
{
\newpath
\moveto(82.77665793,194.23676736)
\lineto(82.77665793,203.68541173)
\lineto(83.83150999,203.68541173)
\lineto(83.83150999,202.79779231)
\curveto(84.08021278,203.14511536)(84.36107758,203.4045401)(84.67410523,203.57606731)
\curveto(84.98712736,203.75186909)(85.36661644,203.83977334)(85.81257362,203.83978033)
\curveto(86.39573937,203.83977334)(86.91030084,203.68969291)(87.35625956,203.38953859)
\curveto(87.80220738,203.0893712)(88.13881634,202.66485799)(88.36608745,202.11599769)
\curveto(88.59334564,201.57141488)(88.70697796,200.97323717)(88.70698476,200.32146278)
\curveto(88.70697796,199.62251332)(88.5804816,198.99217553)(88.3274953,198.4304475)
\curveto(88.07878417,197.87300434)(87.71430313,197.44420311)(87.23405109,197.14404255)
\curveto(86.75807641,196.84816942)(86.25637898,196.70023299)(85.72895729,196.70023284)
\curveto(85.34303237,196.70023299)(84.99570338,196.78170523)(84.68696928,196.94464978)
\curveto(84.38251764,197.10759415)(84.13166892,197.31341874)(83.93442239,197.56212416)
\lineto(83.93442239,194.23676736)
\lineto(82.77665793,194.23676736)
\moveto(83.82507797,200.23141443)
\curveto(83.82507605,199.35236855)(84.00302856,198.7027347)(84.35893602,198.28251093)
\curveto(84.71483858,197.86228431)(85.14578381,197.65217171)(85.651773,197.6521725)
\curveto(86.16633072,197.65217171)(86.60585197,197.86871632)(86.97033807,198.30180701)
\curveto(87.33910206,198.7391828)(87.52348659,199.41454473)(87.5234922,200.32789481)
\curveto(87.52348659,201.19835781)(87.34339008,201.85013567)(86.98320212,202.28323033)
\curveto(86.62729203,202.71631414)(86.20063482,202.93285876)(85.7032292,202.93286483)
\curveto(85.210104,202.93285876)(84.77272675,202.7013061)(84.39109614,202.23820616)
\curveto(84.01374859,201.77938347)(83.82507605,201.11045356)(83.82507797,200.23141443)
}
}
{
\newrgbcolor{curcolor}{0 0 0}
\pscustom[linestyle=none,fillstyle=solid,fillcolor=curcolor]
{
\newpath
\moveto(96.20672549,200.97752931)
\lineto(89.96122945,198.30823903)
\lineto(89.96122945,199.45957146)
\lineto(94.90745649,201.51138736)
\lineto(89.96122945,203.54390719)
\lineto(89.96122945,204.69523962)
\lineto(96.20672549,202.05810947)
\lineto(96.20672549,200.97752931)
}
}
{
\newrgbcolor{curcolor}{0 0 0}
\pscustom[linestyle=none,fillstyle=solid,fillcolor=curcolor]
{
\newpath
\moveto(103.89942874,200.97752931)
\lineto(97.65393269,198.30823903)
\lineto(97.65393269,199.45957146)
\lineto(102.60015974,201.51138736)
\lineto(97.65393269,203.54390719)
\lineto(97.65393269,204.69523962)
\lineto(103.89942874,202.05810947)
\lineto(103.89942874,200.97752931)
}
}
{
\newrgbcolor{curcolor}{0 0 0}
\pscustom[linestyle=none,fillstyle=solid,fillcolor=curcolor]
{
\newpath
\moveto(62.02825312,179.26940949)
\lineto(63.82922006,178.96710433)
\curveto(63.59766069,178.30674858)(63.23103565,177.80290714)(62.72934382,177.45557851)
\curveto(62.2319288,177.11253717)(61.60802302,176.94101669)(60.85762462,176.94101653)
\curveto(59.6698415,176.94101669)(58.790799,177.32908179)(58.22049446,178.10521301)
\curveto(57.77025209,178.72697377)(57.54513145,179.51168001)(57.54513186,180.45933408)
\curveto(57.54513145,181.59136594)(57.84100429,182.47684046)(58.43275128,183.1157603)
\curveto(59.02449566,183.75895611)(59.77275379,184.08055703)(60.67752792,184.08056402)
\curveto(61.69378327,184.08055703)(62.49564155,183.74394807)(63.08310518,183.07073613)
\curveto(63.6705569,182.40180025)(63.9514217,181.37482132)(63.92570043,179.98979627)
\lineto(59.39755499,179.98979627)
\curveto(59.41041676,179.45379185)(59.55620918,179.03571065)(59.83493268,178.73555144)
\curveto(60.11365077,178.43967696)(60.46097975,178.29174053)(60.87692069,178.29174173)
\curveto(61.15992575,178.29174053)(61.39791043,178.36892475)(61.59087544,178.52329462)
\curveto(61.78383152,178.67766163)(61.92962394,178.92636634)(62.02825312,179.26940949)
\moveto(62.13116552,181.09610453)
\curveto(62.11829648,181.61923802)(61.98322409,182.01587915)(61.72594796,182.28602911)
\curveto(61.46866263,182.5604567)(61.15563773,182.69767309)(60.78687234,182.69767869)
\curveto(60.39237156,182.69767309)(60.06648263,182.55402468)(59.80920458,182.26673303)
\curveto(59.55192116,181.97943104)(59.4254248,181.58922193)(59.42971512,181.09610453)
\lineto(62.13116552,181.09610453)
}
}
{
\newrgbcolor{curcolor}{0 0 0}
\pscustom[linestyle=none,fillstyle=solid,fillcolor=curcolor]
{
\newpath
\moveto(67.28321721,177.09538513)
\lineto(64.53031061,183.92619542)
\lineto(66.42775792,183.92619542)
\lineto(67.71416287,180.440038)
\lineto(68.08722031,179.27584152)
\curveto(68.18584096,179.57171218)(68.24801714,179.76681674)(68.27374902,179.86115577)
\curveto(68.33377738,180.05411356)(68.39809757,180.24707411)(68.46670977,180.440038)
\lineto(69.76597877,183.92619542)
\lineto(71.62483393,183.92619542)
\lineto(68.91051948,177.09538513)
\lineto(67.28321721,177.09538513)
}
}
{
\newrgbcolor{curcolor}{0 0 0}
\pscustom[linestyle=none,fillstyle=solid,fillcolor=curcolor]
{
\newpath
\moveto(76.71899752,179.26940949)
\lineto(78.51996446,178.96710433)
\curveto(78.28840509,178.30674858)(77.92178005,177.80290714)(77.42008822,177.45557851)
\curveto(76.9226732,177.11253717)(76.29876742,176.94101669)(75.54836902,176.94101653)
\curveto(74.3605859,176.94101669)(73.4815434,177.32908179)(72.91123886,178.10521301)
\curveto(72.46099649,178.72697377)(72.23587585,179.51168001)(72.23587626,180.45933408)
\curveto(72.23587585,181.59136594)(72.53174869,182.47684046)(73.12349568,183.1157603)
\curveto(73.71524006,183.75895611)(74.46349819,184.08055703)(75.36827232,184.08056402)
\curveto(76.38452767,184.08055703)(77.18638595,183.74394807)(77.77384958,183.07073613)
\curveto(78.3613013,182.40180025)(78.6421661,181.37482132)(78.61644483,179.98979627)
\lineto(74.08829939,179.98979627)
\curveto(74.10116116,179.45379185)(74.24695358,179.03571065)(74.52567708,178.73555144)
\curveto(74.80439517,178.43967696)(75.15172415,178.29174053)(75.56766509,178.29174173)
\curveto(75.85067015,178.29174053)(76.08865483,178.36892475)(76.28161984,178.52329462)
\curveto(76.47457592,178.67766163)(76.62036834,178.92636634)(76.71899752,179.26940949)
\moveto(76.82190992,181.09610453)
\curveto(76.80904088,181.61923802)(76.67396849,182.01587915)(76.41669236,182.28602911)
\curveto(76.15940703,182.5604567)(75.84638213,182.69767309)(75.47761674,182.69767869)
\curveto(75.08311596,182.69767309)(74.75722703,182.55402468)(74.49994898,182.26673303)
\curveto(74.24266556,181.97943104)(74.1161692,181.58922193)(74.12045952,181.09610453)
\lineto(76.82190992,181.09610453)
}
}
{
\newrgbcolor{curcolor}{0 0 0}
\pscustom[linestyle=none,fillstyle=solid,fillcolor=curcolor]
{
\newpath
\moveto(86.3091463,177.09538513)
\lineto(84.50174734,177.09538513)
\lineto(84.50174734,180.58154255)
\curveto(84.50174199,181.31907716)(84.46314988,181.79504652)(84.3859709,182.00945204)
\curveto(84.30878144,182.22813575)(84.18228508,182.39751224)(84.00648144,182.517582)
\curveto(83.83495609,182.63764092)(83.6269875,182.69767309)(83.38257503,182.69767869)
\curveto(83.06954591,182.69767309)(82.78868111,182.61191285)(82.53997979,182.4403977)
\curveto(82.29127169,182.26887187)(82.1197512,182.04160722)(82.02541781,181.75860308)
\curveto(81.93536668,181.47558961)(81.89034255,180.95245212)(81.89034529,180.18918904)
\lineto(81.89034529,177.09538513)
\lineto(80.08294633,177.09538513)
\lineto(80.08294633,183.92619542)
\lineto(81.76170479,183.92619542)
\lineto(81.76170479,182.92279956)
\curveto(82.35773588,183.69463593)(83.10813802,184.08055703)(84.01291346,184.08056402)
\curveto(84.41169374,184.08055703)(84.77617477,184.00766082)(85.10635767,183.86187517)
\curveto(85.43652866,183.720364)(85.68523336,183.53812349)(85.85247254,183.31515307)
\curveto(86.02398633,183.09217021)(86.14190667,182.83917749)(86.20623391,182.55617415)
\curveto(86.27483504,182.27315988)(86.30913914,181.86794273)(86.3091463,181.34052147)
\lineto(86.3091463,177.09538513)
}
}
{
\newrgbcolor{curcolor}{0 0 0}
\pscustom[linestyle=none,fillstyle=solid,fillcolor=curcolor]
{
\newpath
\moveto(91.28110103,183.92619542)
\lineto(91.28110103,182.48542188)
\lineto(90.04615228,182.48542188)
\lineto(90.04615228,179.73251528)
\curveto(90.04614944,179.17507105)(90.05686947,178.84918212)(90.0783124,178.75484751)
\curveto(90.1040376,178.6647976)(90.15763775,178.58975738)(90.23911302,178.52972665)
\curveto(90.32487023,178.46969304)(90.42778252,178.43967696)(90.54785021,178.4396783)
\curveto(90.71507934,178.43967696)(90.95735203,178.49756512)(91.27466901,178.61334297)
\lineto(91.4290376,177.21116157)
\curveto(91.00880818,177.03106494)(90.53283882,176.94101669)(90.0011281,176.94101653)
\curveto(89.67523638,176.94101669)(89.38150754,176.99461684)(89.11994071,177.10181715)
\curveto(88.85837005,177.21330546)(88.6654095,177.35480986)(88.54105848,177.52633078)
\curveto(88.4209928,177.70213885)(88.33737657,177.93797953)(88.29020952,178.23385351)
\curveto(88.25161632,178.44396497)(88.23232027,178.86847818)(88.2323213,179.50739441)
\lineto(88.2323213,182.48542188)
\lineto(87.4025901,182.48542188)
\lineto(87.4025901,183.92619542)
\lineto(88.2323213,183.92619542)
\lineto(88.2323213,185.28335265)
\lineto(90.04615228,186.33820471)
\lineto(90.04615228,183.92619542)
\lineto(91.28110103,183.92619542)
}
}
{
\newrgbcolor{curcolor}{0 0 0}
\pscustom[linestyle=none,fillstyle=solid,fillcolor=curcolor]
{
\newpath
\moveto(91.91143823,179.04428863)
\lineto(93.72526921,179.32086569)
\curveto(93.80245131,178.96924647)(93.95896376,178.7012457)(94.19480702,178.5168626)
\curveto(94.4306451,178.33676466)(94.76082204,178.24671641)(95.18533883,178.24671756)
\curveto(95.65272858,178.24671641)(96.00434559,178.33247665)(96.24019089,178.50399855)
\curveto(96.39884271,178.62406148)(96.47817094,178.78486194)(96.47817581,178.98640041)
\curveto(96.47817094,179.12361491)(96.43529081,179.23724723)(96.34953532,179.32729772)
\curveto(96.25948231,179.41305573)(96.05794574,179.49238396)(95.74492499,179.56528263)
\curveto(94.28699669,179.88688108)(93.36293006,180.18060992)(92.97272232,180.44647003)
\curveto(92.43243141,180.81523573)(92.16228664,181.32765319)(92.1622872,181.98372394)
\curveto(92.16228664,182.57546474)(92.3959833,183.07287416)(92.86337789,183.47595369)
\curveto(93.33076997,183.87902046)(94.05544403,184.08055703)(95.03740226,184.08056402)
\curveto(95.97218549,184.08055703)(96.66684347,183.9283326)(97.12137829,183.62389026)
\curveto(97.57590206,183.31943486)(97.88892696,182.86919358)(98.0604539,182.27316506)
\lineto(96.35596734,181.95799585)
\curveto(96.28306638,182.22384774)(96.14370598,182.42752832)(95.93788573,182.5690382)
\curveto(95.73634482,182.71053713)(95.446904,182.78128933)(95.06956239,182.78129501)
\curveto(94.59358956,182.78128933)(94.25269259,182.71482514)(94.04687045,182.58190225)
\curveto(93.90965162,182.48756049)(93.84104342,182.36535214)(93.84104566,182.21527684)
\curveto(93.84104342,182.08663135)(93.90107559,181.97728704)(94.02114235,181.88724357)
\curveto(94.1840844,181.76717444)(94.745814,181.59779796)(95.70633284,181.37911362)
\curveto(96.67113149,181.16042071)(97.3443494,180.89241995)(97.72598861,180.57511052)
\curveto(98.10332757,180.25350613)(98.2920001,179.80540885)(98.29200679,179.23081735)
\curveto(98.2920001,178.60476543)(98.03043136,178.06661989)(97.50729977,177.61637913)
\curveto(96.98415638,177.16613733)(96.21017017,176.94101669)(95.18533883,176.94101653)
\curveto(94.2548366,176.94101669)(93.5172985,177.12968922)(92.97272232,177.50703471)
\curveto(92.43243141,177.88437937)(92.0786704,178.39679683)(91.91143823,179.04428863)
}
}
{
\newrgbcolor{curcolor}{0 1 0.25098041}
\pscustom[linewidth=2.63455725,linecolor=curcolor]
{
\newpath
\moveto(37.36788,45.89200856)
\lineto(121.673712,45.89200856)
\lineto(121.673712,98.58315856)
\lineto(66.348009,98.58315856)
\lineto(66.348009,109.12138856)
\lineto(37.36788,109.12138856)
\lineto(37.36788,45.89200856)
\closepath
}
}
{
\newrgbcolor{curcolor}{0 0 0}
\pscustom[linestyle=none,fillstyle=solid,fillcolor=curcolor]
{
\newpath
\moveto(52.57833311,81.57295427)
\lineto(52.57833311,82.65353443)
\lineto(58.82382915,85.29066458)
\lineto(58.82382915,84.13933215)
\lineto(53.87117009,82.10681232)
\lineto(58.82382915,80.05499642)
\lineto(58.82382915,78.90366399)
\lineto(52.57833311,81.57295427)
}
}
{
\newrgbcolor{curcolor}{0 0 0}
\pscustom[linestyle=none,fillstyle=solid,fillcolor=curcolor]
{
\newpath
\moveto(60.27103493,81.57295427)
\lineto(60.27103493,82.65353443)
\lineto(66.51653097,85.29066458)
\lineto(66.51653097,84.13933215)
\lineto(61.5638719,82.10681232)
\lineto(66.51653097,80.05499642)
\lineto(66.51653097,78.90366399)
\lineto(60.27103493,81.57295427)
}
}
{
\newrgbcolor{curcolor}{0 0 0}
\pscustom[linestyle=none,fillstyle=solid,fillcolor=curcolor]
{
\newpath
\moveto(72.56906647,78.29262164)
\curveto(72.14025993,77.92813976)(71.72646675,77.67085902)(71.32768569,77.52077867)
\curveto(70.93318449,77.37069817)(70.50867128,77.29565795)(70.05414479,77.2956578)
\curveto(69.30373984,77.29565795)(68.7270022,77.47789847)(68.32393013,77.8423799)
\curveto(67.9208559,78.21114856)(67.71931933,78.6806859)(67.7193198,79.25099333)
\curveto(67.71931933,79.58545648)(67.79435954,79.88990535)(67.94444067,80.16434084)
\curveto(68.09880841,80.44305892)(68.29820098,80.66603556)(68.54261897,80.83327142)
\curveto(68.79132238,81.00050051)(69.07004318,81.12699687)(69.37878219,81.21276088)
\curveto(69.6060447,81.27278929)(69.94908568,81.33067745)(70.40790615,81.38642555)
\curveto(71.34268965,81.49790993)(72.03091561,81.63083831)(72.4725861,81.78521108)
\curveto(72.47686888,81.9438632)(72.47901289,82.04463149)(72.47901813,82.08751625)
\curveto(72.47901289,82.55919295)(72.36966858,82.8915139)(72.15098486,83.08448008)
\curveto(71.85510711,83.3460432)(71.41558586,83.47682757)(70.83241979,83.47683359)
\curveto(70.28783865,83.47682757)(69.8847655,83.38034729)(69.62319913,83.18739248)
\curveto(69.36591602,82.99871421)(69.17509948,82.66210525)(69.05074893,82.17756459)
\lineto(67.91871257,82.33193319)
\curveto(68.02162419,82.81647369)(68.19100067,83.2066828)(68.42684253,83.50256169)
\curveto(68.66268202,83.8027165)(69.00357899,84.03212515)(69.44953446,84.19078834)
\curveto(69.89548553,84.35372607)(70.412191,84.4351983)(70.99965243,84.43520528)
\curveto(71.58281834,84.4351983)(72.05664369,84.3665901)(72.4211299,84.22938049)
\curveto(72.78560576,84.09215732)(73.05360653,83.91849283)(73.225133,83.70838649)
\curveto(73.39664751,83.50255564)(73.51671185,83.2409869)(73.58532639,82.92367947)
\curveto(73.62391215,82.72642543)(73.64320821,82.37052042)(73.64321461,81.85596335)
\lineto(73.64321461,80.31227741)
\curveto(73.64320821,79.23598348)(73.66679228,78.55418954)(73.71396688,78.26689354)
\curveto(73.76541656,77.98388391)(73.86404084,77.71159514)(74.00984002,77.45002639)
\lineto(72.80061936,77.45002639)
\curveto(72.68054947,77.69015508)(72.60336525,77.97101988)(72.56906647,78.29262164)
\moveto(72.4725861,80.87829559)
\curveto(72.05235568,80.70677167)(71.42201788,80.56097926)(70.58157082,80.44091791)
\curveto(70.10559813,80.37230672)(69.76898917,80.2951225)(69.57174293,80.20936502)
\curveto(69.37449204,80.12360201)(69.22226761,79.99710565)(69.11506918,79.82987555)
\curveto(69.007867,79.66692871)(68.95426685,79.48468819)(68.95426856,79.28315345)
\curveto(68.95426685,78.97441474)(69.07004318,78.717134)(69.30159789,78.51131048)
\curveto(69.53743651,78.30548483)(69.88047749,78.20257254)(70.33072186,78.20257329)
\curveto(70.77667204,78.20257254)(71.17331317,78.29905281)(71.52064644,78.4920144)
\curveto(71.86797115,78.68926192)(72.12310788,78.95726269)(72.28605738,79.2960175)
\curveto(72.4104047,79.5575844)(72.47258087,79.9435055)(72.4725861,80.45378196)
\lineto(72.4725861,80.87829559)
}
}
{
\newrgbcolor{curcolor}{0 0 0}
\pscustom[linestyle=none,fillstyle=solid,fillcolor=curcolor]
{
\newpath
\moveto(75.44418045,74.83219231)
\lineto(75.44418045,84.28083669)
\lineto(76.49903251,84.28083669)
\lineto(76.49903251,83.39321727)
\curveto(76.74773529,83.74054032)(77.02860009,83.99996506)(77.34162775,84.17149227)
\curveto(77.65464988,84.34729405)(78.03413896,84.4351983)(78.48009613,84.43520528)
\curveto(79.06326189,84.4351983)(79.57782336,84.28511787)(80.02378208,83.98496355)
\curveto(80.4697299,83.68479616)(80.80633886,83.26028295)(81.03360996,82.71142265)
\curveto(81.26086815,82.16683983)(81.37450048,81.56866213)(81.37450728,80.91688774)
\curveto(81.37450048,80.21793828)(81.24800412,79.58760048)(80.99501782,79.02587246)
\curveto(80.74630669,78.4684293)(80.38182565,78.03962807)(79.90157361,77.73946751)
\curveto(79.42559892,77.44359437)(78.92390149,77.29565795)(78.39647981,77.2956578)
\curveto(78.01055489,77.29565795)(77.6632259,77.37713019)(77.3544918,77.54007474)
\curveto(77.05004015,77.70301911)(76.79919144,77.9088437)(76.6019449,78.15754912)
\lineto(76.6019449,74.83219231)
\lineto(75.44418045,74.83219231)
\moveto(76.49260048,80.82683939)
\curveto(76.49259857,79.94779351)(76.67055107,79.29815966)(77.02645854,78.87793589)
\curveto(77.3823611,78.45770926)(77.81330633,78.24759667)(78.31929552,78.24759746)
\curveto(78.83385324,78.24759667)(79.27337449,78.46414128)(79.63786059,78.89723196)
\curveto(80.00662458,79.33460776)(80.19100911,80.00996969)(80.19101472,80.92331976)
\curveto(80.19100911,81.79378277)(80.01091259,82.44556063)(79.65072464,82.87865529)
\curveto(79.29481455,83.3117391)(78.86815734,83.52828371)(78.37075171,83.52828979)
\curveto(77.87762651,83.52828371)(77.44024927,83.29673105)(77.05861866,82.83363112)
\curveto(76.6812711,82.37480843)(76.49259857,81.70587852)(76.49260048,80.82683939)
}
}
{
\newrgbcolor{curcolor}{0 0 0}
\pscustom[linestyle=none,fillstyle=solid,fillcolor=curcolor]
{
\newpath
\moveto(82.77668853,74.83219231)
\lineto(82.77668853,84.28083669)
\lineto(83.83154059,84.28083669)
\lineto(83.83154059,83.39321727)
\curveto(84.08024338,83.74054032)(84.36110818,83.99996506)(84.67413584,84.17149227)
\curveto(84.98715796,84.34729405)(85.36664705,84.4351983)(85.81260422,84.43520528)
\curveto(86.39576998,84.4351983)(86.91033145,84.28511787)(87.35629016,83.98496355)
\curveto(87.80223799,83.68479616)(88.13884695,83.26028295)(88.36611805,82.71142265)
\curveto(88.59337624,82.16683983)(88.70700857,81.56866213)(88.70701536,80.91688774)
\curveto(88.70700857,80.21793828)(88.58051221,79.58760048)(88.3275259,79.02587246)
\curveto(88.07881478,78.4684293)(87.71433374,78.03962807)(87.23408169,77.73946751)
\curveto(86.75810701,77.44359437)(86.25640958,77.29565795)(85.7289879,77.2956578)
\curveto(85.34306298,77.29565795)(84.99573399,77.37713019)(84.68699989,77.54007474)
\curveto(84.38254824,77.70301911)(84.13169953,77.9088437)(83.93445299,78.15754912)
\lineto(83.93445299,74.83219231)
\lineto(82.77668853,74.83219231)
\moveto(83.82510857,80.82683939)
\curveto(83.82510665,79.94779351)(84.00305916,79.29815966)(84.35896663,78.87793589)
\curveto(84.71486919,78.45770926)(85.14581442,78.24759667)(85.6518036,78.24759746)
\curveto(86.16636133,78.24759667)(86.60588258,78.46414128)(86.97036868,78.89723196)
\curveto(87.33913267,79.33460776)(87.52351719,80.00996969)(87.52352281,80.92331976)
\curveto(87.52351719,81.79378277)(87.34342068,82.44556063)(86.98323273,82.87865529)
\curveto(86.62732264,83.3117391)(86.20066542,83.52828371)(85.7032598,83.52828979)
\curveto(85.2101346,83.52828371)(84.77275735,83.29673105)(84.39112675,82.83363112)
\curveto(84.01377919,82.37480843)(83.82510665,81.70587852)(83.82510857,80.82683939)
}
}
{
\newrgbcolor{curcolor}{0 0 0}
\pscustom[linestyle=none,fillstyle=solid,fillcolor=curcolor]
{
\newpath
\moveto(96.2067561,81.57295427)
\lineto(89.96126005,78.90366399)
\lineto(89.96126005,80.05499642)
\lineto(94.90748709,82.10681232)
\lineto(89.96126005,84.13933215)
\lineto(89.96126005,85.29066458)
\lineto(96.2067561,82.65353443)
\lineto(96.2067561,81.57295427)
}
}
{
\newrgbcolor{curcolor}{0 0 0}
\pscustom[linestyle=none,fillstyle=solid,fillcolor=curcolor]
{
\newpath
\moveto(103.89945934,81.57295427)
\lineto(97.6539633,78.90366399)
\lineto(97.6539633,80.05499642)
\lineto(102.60019034,82.10681232)
\lineto(97.6539633,84.13933215)
\lineto(97.6539633,85.29066458)
\lineto(103.89945934,82.65353443)
\lineto(103.89945934,81.57295427)
}
}
{
\newrgbcolor{curcolor}{0 0 0}
\pscustom[linestyle=none,fillstyle=solid,fillcolor=curcolor]
{
\newpath
\moveto(59.33838923,64.52174245)
\lineto(61.02357972,64.52174245)
\lineto(61.02357972,63.51834659)
\curveto(61.24226576,63.86138174)(61.5381386,64.14010253)(61.91119913,64.35450981)
\curveto(62.28425273,64.56890376)(62.69804591,64.67610406)(63.15257991,64.67611105)
\curveto(63.94585746,64.67610406)(64.61907538,64.36522318)(65.17223569,63.74346746)
\curveto(65.72538254,63.12169963)(66.00195932,62.25552116)(66.00196688,61.14492945)
\curveto(66.00195932,60.00431475)(65.72323853,59.11669622)(65.16580366,58.4820712)
\curveto(64.60835535,57.85173261)(63.93299343,57.53656371)(63.13971586,57.53656356)
\curveto(62.76236609,57.53656371)(62.41932511,57.61160393)(62.1105919,57.76168443)
\curveto(61.80613937,57.91176478)(61.48453845,58.16904552)(61.14578819,58.5335274)
\lineto(61.14578819,55.09239415)
\lineto(59.33838923,55.09239415)
\lineto(59.33838923,64.52174245)
\moveto(61.12649211,61.22211375)
\curveto(61.12648943,60.45455603)(61.27871386,59.88639441)(61.58316587,59.51762719)
\curveto(61.8876116,59.15314432)(62.25852466,58.9709038)(62.69590615,58.97090508)
\curveto(63.1161271,58.9709038)(63.4656001,59.13813628)(63.74432619,59.47260301)
\curveto(64.02304168,59.8113542)(64.16240208,60.36450777)(64.1624078,61.1320654)
\curveto(64.16240208,61.84816)(64.01875367,62.37987352)(63.73146214,62.72720754)
\curveto(63.44416003,63.0745315)(63.08825502,63.24819599)(62.66374603,63.24820155)
\curveto(62.22207655,63.24819599)(61.85545151,63.0766755)(61.5638698,62.73363957)
\curveto(61.27228185,62.39488156)(61.12648943,61.89104012)(61.12649211,61.22211375)
}
}
{
\newrgbcolor{curcolor}{0 0 0}
\pscustom[linestyle=none,fillstyle=solid,fillcolor=curcolor]
{
\newpath
\moveto(69.24370695,67.12028046)
\lineto(69.24370695,63.65341911)
\curveto(69.82687387,64.33520709)(70.52367585,64.67610406)(71.334115,64.67611105)
\curveto(71.75004735,64.67610406)(72.12524842,64.59891984)(72.45971933,64.44455815)
\curveto(72.79417832,64.29018296)(73.04502704,64.0929344)(73.21226623,63.85281188)
\curveto(73.38378,63.61267703)(73.49955633,63.34682027)(73.55959557,63.05524081)
\curveto(73.62390869,62.76365061)(73.65606878,62.31126532)(73.65607594,61.69808358)
\lineto(73.65607594,57.69093215)
\lineto(71.84867698,57.69093215)
\lineto(71.84867698,61.29929805)
\curveto(71.84867163,62.01539248)(71.81436753,62.46992177)(71.74576458,62.6628873)
\curveto(71.67715114,62.85584287)(71.55494279,63.00806731)(71.37913917,63.11956105)
\curveto(71.2076138,63.23533196)(70.99106918,63.29322012)(70.72950467,63.29322572)
\curveto(70.42933958,63.29322012)(70.16133882,63.22032391)(69.92550158,63.07453688)
\curveto(69.68965748,62.92873908)(69.51599298,62.70790645)(69.40450757,62.41203833)
\curveto(69.29730436,62.12044878)(69.24370421,61.68735954)(69.24370695,61.11276933)
\lineto(69.24370695,57.69093215)
\lineto(67.43630799,57.69093215)
\lineto(67.43630799,67.12028046)
\lineto(69.24370695,67.12028046)
}
}
{
\newrgbcolor{curcolor}{0 0 0}
\pscustom[linestyle=none,fillstyle=solid,fillcolor=curcolor]
{
\newpath
\moveto(75.077553,61.20281767)
\curveto(75.07755247,61.80313587)(75.2254889,62.38416153)(75.52136271,62.94589639)
\curveto(75.81723458,63.50762073)(76.23531577,63.93642195)(76.77560754,64.23230134)
\curveto(77.32018286,64.52816764)(77.92693659,64.67610406)(78.59587055,64.67611105)
\curveto(79.62927745,64.67610406)(80.47615986,64.3394951)(81.13652033,63.66628316)
\curveto(81.79686762,62.99734728)(82.12704456,62.15046486)(82.12705214,61.12563338)
\curveto(82.12704456,60.092219)(81.79257961,59.23461655)(81.12365628,58.55282347)
\curveto(80.45900781,57.87531668)(79.62070142,57.53656371)(78.6087346,57.53656356)
\curveto(77.98268075,57.53656371)(77.38450305,57.67806812)(76.81419969,57.96107719)
\curveto(76.24817981,58.24408573)(75.81723458,58.65787891)(75.52136271,59.20245797)
\curveto(75.2254889,59.75132203)(75.07755247,60.41810793)(75.077553,61.20281767)
\moveto(76.92997613,61.1063373)
\curveto(76.92997375,60.42882796)(77.09077421,59.90997848)(77.41237799,59.54978731)
\curveto(77.73397604,59.18959243)(78.13061717,59.00949591)(78.60230257,59.00949723)
\curveto(79.07397986,59.00949591)(79.46847699,59.18959243)(79.78579513,59.54978731)
\curveto(80.10739081,59.90997848)(80.26819127,60.43311597)(80.26819698,61.11920135)
\curveto(80.26819127,61.78812783)(80.10739081,62.3026893)(79.78579513,62.6628873)
\curveto(79.46847699,63.02307535)(79.07397986,63.20317186)(78.60230257,63.20317738)
\curveto(78.13061717,63.20317186)(77.73397604,63.02307535)(77.41237799,62.6628873)
\curveto(77.09077421,62.3026893)(76.92997375,61.78383982)(76.92997613,61.1063373)
}
}
{
\newrgbcolor{curcolor}{0 0 0}
\pscustom[linestyle=none,fillstyle=solid,fillcolor=curcolor]
{
\newpath
\moveto(86.68092526,64.52174245)
\lineto(86.68092526,63.08096891)
\lineto(85.44597651,63.08096891)
\lineto(85.44597651,60.32806231)
\curveto(85.44597366,59.77061808)(85.4566937,59.44472915)(85.47813663,59.35039454)
\curveto(85.50386183,59.26034463)(85.55746198,59.18530441)(85.63893725,59.12527368)
\curveto(85.72469446,59.06524007)(85.82760675,59.03522399)(85.94767444,59.03522533)
\curveto(86.11490357,59.03522399)(86.35717626,59.09311215)(86.67449324,59.20889)
\lineto(86.82886183,57.8067086)
\curveto(86.40863241,57.62661197)(85.93266305,57.53656371)(85.40095233,57.53656356)
\curveto(85.07506061,57.53656371)(84.78133177,57.59016387)(84.51976494,57.69736418)
\curveto(84.25819428,57.80885249)(84.06523373,57.95035689)(83.94088271,58.12187781)
\curveto(83.82081703,58.29768588)(83.7372008,58.53352656)(83.69003375,58.82940054)
\curveto(83.65144055,59.039512)(83.6321445,59.46402521)(83.63214552,60.10294144)
\lineto(83.63214552,63.08096891)
\lineto(82.80241433,63.08096891)
\lineto(82.80241433,64.52174245)
\lineto(83.63214552,64.52174245)
\lineto(83.63214552,65.87889968)
\lineto(85.44597651,66.93375174)
\lineto(85.44597651,64.52174245)
\lineto(86.68092526,64.52174245)
}
}
{
\newrgbcolor{curcolor}{0 0 0}
\pscustom[linestyle=none,fillstyle=solid,fillcolor=curcolor]
{
\newpath
\moveto(87.52995321,61.20281767)
\curveto(87.52995268,61.80313587)(87.6778891,62.38416153)(87.97376292,62.94589639)
\curveto(88.26963479,63.50762073)(88.68771598,63.93642195)(89.22800775,64.23230134)
\curveto(89.77258307,64.52816764)(90.3793368,64.67610406)(91.04827075,64.67611105)
\curveto(92.08167765,64.67610406)(92.92856007,64.3394951)(93.58892053,63.66628316)
\curveto(94.24926783,62.99734728)(94.57944477,62.15046486)(94.57945235,61.12563338)
\curveto(94.57944477,60.092219)(94.24497982,59.23461655)(93.57605649,58.55282347)
\curveto(92.91140802,57.87531668)(92.07310163,57.53656371)(91.0611348,57.53656356)
\curveto(90.43508096,57.53656371)(89.83690326,57.67806812)(89.26659989,57.96107719)
\curveto(88.70058002,58.24408573)(88.26963479,58.65787891)(87.97376292,59.20245797)
\curveto(87.6778891,59.75132203)(87.52995268,60.41810793)(87.52995321,61.20281767)
\moveto(89.38237634,61.1063373)
\curveto(89.38237396,60.42882796)(89.54317442,59.90997848)(89.8647782,59.54978731)
\curveto(90.18637625,59.18959243)(90.58301738,59.00949591)(91.05470278,59.00949723)
\curveto(91.52638007,59.00949591)(91.92087719,59.18959243)(92.23819533,59.54978731)
\curveto(92.55979102,59.90997848)(92.72059147,60.43311597)(92.72059719,61.11920135)
\curveto(92.72059147,61.78812783)(92.55979102,62.3026893)(92.23819533,62.6628873)
\curveto(91.92087719,63.02307535)(91.52638007,63.20317186)(91.05470278,63.20317738)
\curveto(90.58301738,63.20317186)(90.18637625,63.02307535)(89.8647782,62.6628873)
\curveto(89.54317442,62.3026893)(89.38237396,61.78383982)(89.38237634,61.1063373)
}
}
{
\newrgbcolor{curcolor}{0 0 0}
\pscustom[linestyle=none,fillstyle=solid,fillcolor=curcolor]
{
\newpath
\moveto(95.36415896,59.63983566)
\lineto(97.17798994,59.91641272)
\curveto(97.25517204,59.56479349)(97.41168449,59.29679273)(97.64752775,59.11240963)
\curveto(97.88336583,58.93231169)(98.21354277,58.84226344)(98.63805956,58.84226459)
\curveto(99.10544931,58.84226344)(99.45706631,58.92802368)(99.69291162,59.09954558)
\curveto(99.85156344,59.21960851)(99.93089166,59.38040897)(99.93089654,59.58194743)
\curveto(99.93089166,59.71916193)(99.88801154,59.83279426)(99.80225604,59.92284475)
\curveto(99.71220304,60.00860276)(99.51066647,60.08793099)(99.19764572,60.16082966)
\curveto(97.73971742,60.48242811)(96.81565079,60.77615695)(96.42544304,61.04201706)
\curveto(95.88515214,61.41078276)(95.61500737,61.92320022)(95.61500792,62.57927097)
\curveto(95.61500737,63.17101177)(95.84870403,63.66842119)(96.31609862,64.07150072)
\curveto(96.7834907,64.47456749)(97.50816476,64.67610406)(98.49012299,64.67611105)
\curveto(99.42490622,64.67610406)(100.1195642,64.52387963)(100.57409902,64.21943729)
\curveto(101.02862279,63.91498189)(101.34164769,63.46474061)(101.51317463,62.86871209)
\lineto(99.80868807,62.55354287)
\curveto(99.73578711,62.81939477)(99.59642671,63.02307535)(99.39060646,63.16458523)
\curveto(99.18906555,63.30608416)(98.89962473,63.37683636)(98.52228312,63.37684204)
\curveto(98.04631029,63.37683636)(97.70541332,63.31037217)(97.49959118,63.17744928)
\curveto(97.36237234,63.08310752)(97.29376415,62.96089917)(97.29376639,62.81082387)
\curveto(97.29376415,62.68217838)(97.35379632,62.57283407)(97.47386308,62.4827906)
\curveto(97.63680513,62.36272147)(98.19853473,62.19334499)(99.15905357,61.97466065)
\curveto(100.12385221,61.75596774)(100.79707013,61.48796698)(101.17870934,61.17065755)
\curveto(101.5560483,60.84905315)(101.74472083,60.40095588)(101.74472752,59.82636438)
\curveto(101.74472083,59.20031246)(101.48315209,58.66216692)(100.9600205,58.21192616)
\curveto(100.43687711,57.76168436)(99.6628909,57.53656371)(98.63805956,57.53656356)
\curveto(97.70755733,57.53656371)(96.97001923,57.72523625)(96.42544304,58.10258174)
\curveto(95.88515214,58.4799264)(95.53139113,58.99234386)(95.36415896,59.63983566)
}
}
{
\newrgbcolor{curcolor}{0 1 0.25098041}
\pscustom[linewidth=2.63455725,linecolor=curcolor]
{
\newpath
\moveto(283.00129,45.89200856)
\lineto(367.30713,45.89200856)
\lineto(367.30713,98.58315856)
\lineto(310.66415,98.58315856)
\lineto(310.66415,109.12138856)
\lineto(283.00129,109.12138856)
\lineto(283.00129,45.89200856)
\closepath
}
}
{
\newrgbcolor{curcolor}{0 0 0}
\pscustom[linestyle=none,fillstyle=solid,fillcolor=curcolor]
{
\newpath
\moveto(298.21171251,81.57295427)
\lineto(298.21171251,82.65353443)
\lineto(304.45720855,85.29066458)
\lineto(304.45720855,84.13933215)
\lineto(299.50454948,82.10681232)
\lineto(304.45720855,80.05499642)
\lineto(304.45720855,78.90366399)
\lineto(298.21171251,81.57295427)
}
}
{
\newrgbcolor{curcolor}{0 0 0}
\pscustom[linestyle=none,fillstyle=solid,fillcolor=curcolor]
{
\newpath
\moveto(305.90441432,81.57295427)
\lineto(305.90441432,82.65353443)
\lineto(312.14991037,85.29066458)
\lineto(312.14991037,84.13933215)
\lineto(307.1972513,82.10681232)
\lineto(312.14991037,80.05499642)
\lineto(312.14991037,78.90366399)
\lineto(305.90441432,81.57295427)
}
}
{
\newrgbcolor{curcolor}{0 0 0}
\pscustom[linestyle=none,fillstyle=solid,fillcolor=curcolor]
{
\newpath
\moveto(318.20244587,78.29262164)
\curveto(317.77363932,77.92813976)(317.35984614,77.67085902)(316.96106509,77.52077867)
\curveto(316.56656388,77.37069817)(316.14205067,77.29565795)(315.68752419,77.2956578)
\curveto(314.93711924,77.29565795)(314.36038159,77.47789847)(313.95730953,77.8423799)
\curveto(313.5542353,78.21114856)(313.35269872,78.6806859)(313.3526992,79.25099333)
\curveto(313.35269872,79.58545648)(313.42773894,79.88990535)(313.57782006,80.16434084)
\curveto(313.7321878,80.44305892)(313.93158037,80.66603556)(314.17599837,80.83327142)
\curveto(314.42470178,81.00050051)(314.70342257,81.12699687)(315.01216159,81.21276088)
\curveto(315.2394241,81.27278929)(315.58246508,81.33067745)(316.04128555,81.38642555)
\curveto(316.97606905,81.49790993)(317.66429501,81.63083831)(318.1059655,81.78521108)
\curveto(318.11024828,81.9438632)(318.11239229,82.04463149)(318.11239752,82.08751625)
\curveto(318.11239229,82.55919295)(318.00304797,82.8915139)(317.78436426,83.08448008)
\curveto(317.48848651,83.3460432)(317.04896526,83.47682757)(316.46579918,83.47683359)
\curveto(315.92121804,83.47682757)(315.51814489,83.38034729)(315.25657853,83.18739248)
\curveto(314.99929541,82.99871421)(314.80847887,82.66210525)(314.68412832,82.17756459)
\lineto(313.55209197,82.33193319)
\curveto(313.65500358,82.81647369)(313.82438007,83.2066828)(314.06022192,83.50256169)
\curveto(314.29606141,83.8027165)(314.63695838,84.03212515)(315.08291386,84.19078834)
\curveto(315.52886492,84.35372607)(316.0455704,84.4351983)(316.63303183,84.43520528)
\curveto(317.21619773,84.4351983)(317.69002308,84.3665901)(318.0545093,84.22938049)
\curveto(318.41898516,84.09215732)(318.68698592,83.91849283)(318.85851239,83.70838649)
\curveto(319.0300269,83.50255564)(319.15009124,83.2409869)(319.21870578,82.92367947)
\curveto(319.25729155,82.72642543)(319.2765876,82.37052042)(319.276594,81.85596335)
\lineto(319.276594,80.31227741)
\curveto(319.2765876,79.23598348)(319.30017167,78.55418954)(319.34734628,78.26689354)
\curveto(319.39879595,77.98388391)(319.49742023,77.71159514)(319.64321942,77.45002639)
\lineto(318.43399876,77.45002639)
\curveto(318.31392886,77.69015508)(318.23674464,77.97101988)(318.20244587,78.29262164)
\moveto(318.1059655,80.87829559)
\curveto(317.68573507,80.70677167)(317.05539727,80.56097926)(316.21495022,80.44091791)
\curveto(315.73897752,80.37230672)(315.40236856,80.2951225)(315.20512233,80.20936502)
\curveto(315.00787144,80.12360201)(314.85564701,79.99710565)(314.74844857,79.82987555)
\curveto(314.64124639,79.66692871)(314.58764624,79.48468819)(314.58764795,79.28315345)
\curveto(314.58764624,78.97441474)(314.70342257,78.717134)(314.93497729,78.51131048)
\curveto(315.1708159,78.30548483)(315.51385688,78.20257254)(315.96410125,78.20257329)
\curveto(316.41005143,78.20257254)(316.80669257,78.29905281)(317.15402583,78.4920144)
\curveto(317.50135054,78.68926192)(317.75648727,78.95726269)(317.91943678,79.2960175)
\curveto(318.04378409,79.5575844)(318.10596027,79.9435055)(318.1059655,80.45378196)
\lineto(318.1059655,80.87829559)
}
}
{
\newrgbcolor{curcolor}{0 0 0}
\pscustom[linestyle=none,fillstyle=solid,fillcolor=curcolor]
{
\newpath
\moveto(321.07755984,74.83219231)
\lineto(321.07755984,84.28083669)
\lineto(322.1324119,84.28083669)
\lineto(322.1324119,83.39321727)
\curveto(322.38111469,83.74054032)(322.66197949,83.99996506)(322.97500715,84.17149227)
\curveto(323.28802927,84.34729405)(323.66751835,84.4351983)(324.11347553,84.43520528)
\curveto(324.69664129,84.4351983)(325.21120275,84.28511787)(325.65716147,83.98496355)
\curveto(326.1031093,83.68479616)(326.43971825,83.26028295)(326.66698936,82.71142265)
\curveto(326.89424755,82.16683983)(327.00787987,81.56866213)(327.00788667,80.91688774)
\curveto(327.00787987,80.21793828)(326.88138351,79.58760048)(326.62839721,79.02587246)
\curveto(326.37968608,78.4684293)(326.01520504,78.03962807)(325.534953,77.73946751)
\curveto(325.05897832,77.44359437)(324.55728089,77.29565795)(324.02985921,77.2956578)
\curveto(323.64393429,77.29565795)(323.2966053,77.37713019)(322.9878712,77.54007474)
\curveto(322.68341955,77.70301911)(322.43257084,77.9088437)(322.2353243,78.15754912)
\lineto(322.2353243,74.83219231)
\lineto(321.07755984,74.83219231)
\moveto(322.12597988,80.82683939)
\curveto(322.12597796,79.94779351)(322.30393047,79.29815966)(322.65983793,78.87793589)
\curveto(323.0157405,78.45770926)(323.44668573,78.24759667)(323.95267491,78.24759746)
\curveto(324.46723263,78.24759667)(324.90675389,78.46414128)(325.27123999,78.89723196)
\curveto(325.64000398,79.33460776)(325.8243885,80.00996969)(325.82439412,80.92331976)
\curveto(325.8243885,81.79378277)(325.64429199,82.44556063)(325.28410404,82.87865529)
\curveto(324.92819395,83.3117391)(324.50153673,83.52828371)(324.00413111,83.52828979)
\curveto(323.51100591,83.52828371)(323.07362866,83.29673105)(322.69199806,82.83363112)
\curveto(322.3146505,82.37480843)(322.12597796,81.70587852)(322.12597988,80.82683939)
}
}
{
\newrgbcolor{curcolor}{0 0 0}
\pscustom[linestyle=none,fillstyle=solid,fillcolor=curcolor]
{
\newpath
\moveto(328.41006793,74.83219231)
\lineto(328.41006793,84.28083669)
\lineto(329.46491999,84.28083669)
\lineto(329.46491999,83.39321727)
\curveto(329.71362278,83.74054032)(329.99448758,83.99996506)(330.30751523,84.17149227)
\curveto(330.62053736,84.34729405)(331.00002644,84.4351983)(331.44598362,84.43520528)
\curveto(332.02914937,84.4351983)(332.54371084,84.28511787)(332.98966956,83.98496355)
\curveto(333.43561738,83.68479616)(333.77222634,83.26028295)(333.99949745,82.71142265)
\curveto(334.22675564,82.16683983)(334.34038796,81.56866213)(334.34039476,80.91688774)
\curveto(334.34038796,80.21793828)(334.2138916,79.58760048)(333.9609053,79.02587246)
\curveto(333.71219417,78.4684293)(333.34771313,78.03962807)(332.86746109,77.73946751)
\curveto(332.39148641,77.44359437)(331.88978898,77.29565795)(331.36236729,77.2956578)
\curveto(330.97644237,77.29565795)(330.62911338,77.37713019)(330.32037928,77.54007474)
\curveto(330.01592764,77.70301911)(329.76507892,77.9088437)(329.56783239,78.15754912)
\lineto(329.56783239,74.83219231)
\lineto(328.41006793,74.83219231)
\moveto(329.45848797,80.82683939)
\curveto(329.45848605,79.94779351)(329.63643856,79.29815966)(329.99234602,78.87793589)
\curveto(330.34824858,78.45770926)(330.77919381,78.24759667)(331.285183,78.24759746)
\curveto(331.79974072,78.24759667)(332.23926197,78.46414128)(332.60374807,78.89723196)
\curveto(332.97251206,79.33460776)(333.15689659,80.00996969)(333.1569022,80.92331976)
\curveto(333.15689659,81.79378277)(332.97680008,82.44556063)(332.61661212,82.87865529)
\curveto(332.26070203,83.3117391)(331.83404482,83.52828371)(331.3366392,83.52828979)
\curveto(330.843514,83.52828371)(330.40613675,83.29673105)(330.02450614,82.83363112)
\curveto(329.64715859,82.37480843)(329.45848605,81.70587852)(329.45848797,80.82683939)
}
}
{
\newrgbcolor{curcolor}{0 0 0}
\pscustom[linestyle=none,fillstyle=solid,fillcolor=curcolor]
{
\newpath
\moveto(341.84013549,81.57295427)
\lineto(335.59463945,78.90366399)
\lineto(335.59463945,80.05499642)
\lineto(340.54086649,82.10681232)
\lineto(335.59463945,84.13933215)
\lineto(335.59463945,85.29066458)
\lineto(341.84013549,82.65353443)
\lineto(341.84013549,81.57295427)
}
}
{
\newrgbcolor{curcolor}{0 0 0}
\pscustom[linestyle=none,fillstyle=solid,fillcolor=curcolor]
{
\newpath
\moveto(349.53283874,81.57295427)
\lineto(343.28734269,78.90366399)
\lineto(343.28734269,80.05499642)
\lineto(348.23356974,82.10681232)
\lineto(343.28734269,84.13933215)
\lineto(343.28734269,85.29066458)
\lineto(349.53283874,82.65353443)
\lineto(349.53283874,81.57295427)
}
}
{
\newrgbcolor{curcolor}{0 0 0}
\pscustom[linestyle=none,fillstyle=solid,fillcolor=curcolor]
{
\newpath
\moveto(305.58411912,57.69093215)
\lineto(302.83121253,64.52174245)
\lineto(304.72865983,64.52174245)
\lineto(306.01506478,61.03558503)
\lineto(306.38812222,59.87138855)
\curveto(306.48674287,60.16725921)(306.54891905,60.36236377)(306.57465094,60.4567028)
\curveto(306.63467929,60.64966059)(306.69899948,60.84262114)(306.76761168,61.03558503)
\lineto(308.06688068,64.52174245)
\lineto(309.92573584,64.52174245)
\lineto(307.21142139,57.69093215)
\lineto(305.58411912,57.69093215)
}
}
{
\newrgbcolor{curcolor}{0 0 0}
\pscustom[linestyle=none,fillstyle=solid,fillcolor=curcolor]
{
\newpath
\moveto(311.06420421,65.44795402)
\lineto(311.06420421,67.12028046)
\lineto(312.87160316,67.12028046)
\lineto(312.87160316,65.44795402)
\lineto(311.06420421,65.44795402)
\moveto(311.06420421,57.69093215)
\lineto(311.06420421,64.52174245)
\lineto(312.87160316,64.52174245)
\lineto(312.87160316,57.69093215)
\lineto(311.06420421,57.69093215)
}
}
{
\newrgbcolor{curcolor}{0 0 0}
\pscustom[linestyle=none,fillstyle=solid,fillcolor=curcolor]
{
\newpath
\moveto(320.98238626,57.69093215)
\lineto(319.30362779,57.69093215)
\lineto(319.30362779,58.69432802)
\curveto(319.02490147,58.3041179)(318.69472453,58.01253307)(318.31309598,57.81957265)
\curveto(317.93574636,57.63089998)(317.55411328,57.53656371)(317.16819557,57.53656356)
\curveto(316.38348594,57.53656371)(315.71026802,57.85173261)(315.1485398,58.4820712)
\curveto(314.59109683,59.11669622)(314.31237604,60.00002674)(314.31237658,61.1320654)
\curveto(314.31237604,62.28982526)(314.58466481,63.16886777)(315.12924372,63.76919555)
\curveto(315.67381992,64.3737992)(316.36204588,64.67610406)(317.19392367,64.67611105)
\curveto(317.95718642,64.67610406)(318.61754031,64.35879116)(319.1749873,63.72417138)
\lineto(319.1749873,67.12028046)
\lineto(320.98238626,67.12028046)
\lineto(320.98238626,57.69093215)
\moveto(316.15836768,61.25427387)
\curveto(316.1583653,60.52530823)(316.25913359,59.99788273)(316.46067285,59.67199578)
\curveto(316.75225499,59.20031246)(317.15961615,58.96447178)(317.68275755,58.96447306)
\curveto(318.09869083,58.96447178)(318.45245184,59.14028029)(318.74404164,59.49189909)
\curveto(319.0356215,59.8478023)(319.18141391,60.37737181)(319.18141932,61.0806092)
\curveto(319.18141391,61.86531205)(319.03990951,62.42918566)(318.75690569,62.77223172)
\curveto(318.4738919,63.11955563)(318.11155486,63.29322012)(317.6698935,63.29322572)
\curveto(317.24108838,63.29322012)(316.88089536,63.12169963)(316.58931334,62.77866374)
\curveto(316.30201371,62.43990569)(316.1583653,61.93177624)(316.15836768,61.25427387)
}
}
{
\newrgbcolor{curcolor}{0 0 0}
\pscustom[linestyle=none,fillstyle=solid,fillcolor=curcolor]
{
\newpath
\moveto(326.72618491,59.86495652)
\lineto(328.52715185,59.56265136)
\curveto(328.29559248,58.90229561)(327.92896744,58.39845417)(327.42727561,58.05112554)
\curveto(326.92986059,57.7080842)(326.30595481,57.53656371)(325.55555641,57.53656356)
\curveto(324.36777329,57.53656371)(323.48873078,57.92462882)(322.91842625,58.70076004)
\curveto(322.46818388,59.3225208)(322.24306323,60.10722704)(322.24306365,61.05488111)
\curveto(322.24306323,62.18691297)(322.53893608,63.07238749)(323.13068307,63.71130733)
\curveto(323.72242745,64.35450314)(324.47068558,64.67610406)(325.37545971,64.67611105)
\curveto(326.39171506,64.67610406)(327.19357334,64.3394951)(327.78103697,63.66628316)
\curveto(328.36848869,62.99734728)(328.64935349,61.97036835)(328.62363222,60.5853433)
\lineto(324.09548678,60.5853433)
\curveto(324.10834855,60.04933888)(324.25414097,59.63125768)(324.53286447,59.33109847)
\curveto(324.81158255,59.03522399)(325.15891154,58.88728756)(325.57485248,58.88728876)
\curveto(325.85785754,58.88728756)(326.09584221,58.96447178)(326.28880723,59.11884165)
\curveto(326.48176331,59.27320866)(326.62755573,59.52191337)(326.72618491,59.86495652)
\moveto(326.82909731,61.69165156)
\curveto(326.81622827,62.21478505)(326.68115588,62.61142618)(326.42387975,62.88157614)
\curveto(326.16659442,63.15600373)(325.85356952,63.29322012)(325.48480413,63.29322572)
\curveto(325.09030335,63.29322012)(324.76441442,63.14957171)(324.50713637,62.86228006)
\curveto(324.24985295,62.57497807)(324.12335659,62.18476896)(324.12764691,61.69165156)
\lineto(326.82909731,61.69165156)
}
}
{
\newrgbcolor{curcolor}{0 0 0}
\pscustom[linestyle=none,fillstyle=solid,fillcolor=curcolor]
{
\newpath
\moveto(329.68491616,61.20281767)
\curveto(329.68491563,61.80313587)(329.83285206,62.38416153)(330.12872587,62.94589639)
\curveto(330.42459774,63.50762073)(330.84267893,63.93642195)(331.3829707,64.23230134)
\curveto(331.92754602,64.52816764)(332.53429975,64.67610406)(333.20323371,64.67611105)
\curveto(334.23664061,64.67610406)(335.08352302,64.3394951)(335.74388349,63.66628316)
\curveto(336.40423078,62.99734728)(336.73440772,62.15046486)(336.7344153,61.12563338)
\curveto(336.73440772,60.092219)(336.39994277,59.23461655)(335.73101944,58.55282347)
\curveto(335.06637097,57.87531668)(334.22806458,57.53656371)(333.21609776,57.53656356)
\curveto(332.59004391,57.53656371)(331.99186621,57.67806812)(331.42156285,57.96107719)
\curveto(330.85554297,58.24408573)(330.42459774,58.65787891)(330.12872587,59.20245797)
\curveto(329.83285206,59.75132203)(329.68491563,60.41810793)(329.68491616,61.20281767)
\moveto(331.53733929,61.1063373)
\curveto(331.53733691,60.42882796)(331.69813737,59.90997848)(332.01974115,59.54978731)
\curveto(332.3413392,59.18959243)(332.73798033,59.00949591)(333.20966573,59.00949723)
\curveto(333.68134302,59.00949591)(334.07584015,59.18959243)(334.39315829,59.54978731)
\curveto(334.71475397,59.90997848)(334.87555443,60.43311597)(334.87556014,61.11920135)
\curveto(334.87555443,61.78812783)(334.71475397,62.3026893)(334.39315829,62.6628873)
\curveto(334.07584015,63.02307535)(333.68134302,63.20317186)(333.20966573,63.20317738)
\curveto(332.73798033,63.20317186)(332.3413392,63.02307535)(332.01974115,62.6628873)
\curveto(331.69813737,62.3026893)(331.53733691,61.78383982)(331.53733929,61.1063373)
}
}
{
\newrgbcolor{curcolor}{0 0 0}
\pscustom[linestyle=none,fillstyle=solid,fillcolor=curcolor]
{
\newpath
\moveto(337.51912191,59.63983566)
\lineto(339.33295289,59.91641272)
\curveto(339.41013499,59.56479349)(339.56664744,59.29679273)(339.8024907,59.11240963)
\curveto(340.03832878,58.93231169)(340.36850572,58.84226344)(340.79302252,58.84226459)
\curveto(341.26041226,58.84226344)(341.61202927,58.92802368)(341.84787458,59.09954558)
\curveto(342.00652639,59.21960851)(342.08585462,59.38040897)(342.08585949,59.58194743)
\curveto(342.08585462,59.71916193)(342.04297449,59.83279426)(341.957219,59.92284475)
\curveto(341.86716599,60.00860276)(341.66562942,60.08793099)(341.35260867,60.16082966)
\curveto(339.89468037,60.48242811)(338.97061374,60.77615695)(338.580406,61.04201706)
\curveto(338.04011509,61.41078276)(337.76997032,61.92320022)(337.76997088,62.57927097)
\curveto(337.76997032,63.17101177)(338.00366698,63.66842119)(338.47106158,64.07150072)
\curveto(338.93845365,64.47456749)(339.66312771,64.67610406)(340.64508595,64.67611105)
\curveto(341.57986918,64.67610406)(342.27452715,64.52387963)(342.72906197,64.21943729)
\curveto(343.18358575,63.91498189)(343.49661064,63.46474061)(343.66813758,62.86871209)
\lineto(341.96365102,62.55354287)
\curveto(341.89075006,62.81939477)(341.75138966,63.02307535)(341.54556941,63.16458523)
\curveto(341.3440285,63.30608416)(341.05458768,63.37683636)(340.67724607,63.37684204)
\curveto(340.20127325,63.37683636)(339.86037627,63.31037217)(339.65455413,63.17744928)
\curveto(339.5173353,63.08310752)(339.4487271,62.96089917)(339.44872934,62.81082387)
\curveto(339.4487271,62.68217838)(339.50875927,62.57283407)(339.62882603,62.4827906)
\curveto(339.79176808,62.36272147)(340.35349768,62.19334499)(341.31401652,61.97466065)
\curveto(342.27881517,61.75596774)(342.95203309,61.48796698)(343.3336723,61.17065755)
\curveto(343.71101125,60.84905315)(343.89968379,60.40095588)(343.89969048,59.82636438)
\curveto(343.89968379,59.20031246)(343.63811504,58.66216692)(343.11498345,58.21192616)
\curveto(342.59184006,57.76168436)(341.81785385,57.53656371)(340.79302252,57.53656356)
\curveto(339.86252028,57.53656371)(339.12498218,57.72523625)(338.580406,58.10258174)
\curveto(338.04011509,58.4799264)(337.68635408,58.99234386)(337.51912191,59.63983566)
}
}
{
\newrgbcolor{curcolor}{0 1 0.25098041}
\pscustom[linewidth=2.63455725,linecolor=curcolor]
{
\newpath
\moveto(37.36788,284.70119456)
\lineto(121.673712,284.70119456)
\lineto(121.673712,337.39234456)
\lineto(66.348009,337.39234456)
\lineto(66.348009,347.93057456)
\lineto(37.36788,347.93057456)
\lineto(37.36788,284.70119456)
\closepath
}
}
{
\newrgbcolor{curcolor}{0 0 0}
\pscustom[linestyle=none,fillstyle=solid,fillcolor=curcolor]
{
\newpath
\moveto(52.57833311,320.38213631)
\lineto(52.57833311,321.46271647)
\lineto(58.82382915,324.09984662)
\lineto(58.82382915,322.94851419)
\lineto(53.87117009,320.91599436)
\lineto(58.82382915,318.86417846)
\lineto(58.82382915,317.71284603)
\lineto(52.57833311,320.38213631)
}
}
{
\newrgbcolor{curcolor}{0 0 0}
\pscustom[linestyle=none,fillstyle=solid,fillcolor=curcolor]
{
\newpath
\moveto(60.27103493,320.38213631)
\lineto(60.27103493,321.46271647)
\lineto(66.51653097,324.09984662)
\lineto(66.51653097,322.94851419)
\lineto(61.5638719,320.91599436)
\lineto(66.51653097,318.86417846)
\lineto(66.51653097,317.71284603)
\lineto(60.27103493,320.38213631)
}
}
{
\newrgbcolor{curcolor}{0 0 0}
\pscustom[linestyle=none,fillstyle=solid,fillcolor=curcolor]
{
\newpath
\moveto(72.56906647,317.10180368)
\curveto(72.14025993,316.7373218)(71.72646675,316.48004106)(71.32768569,316.32996071)
\curveto(70.93318449,316.17988021)(70.50867128,316.10483999)(70.05414479,316.10483984)
\curveto(69.30373984,316.10483999)(68.7270022,316.28708051)(68.32393013,316.65156194)
\curveto(67.9208559,317.0203306)(67.71931933,317.48986794)(67.7193198,318.06017537)
\curveto(67.71931933,318.39463852)(67.79435954,318.69908739)(67.94444067,318.97352288)
\curveto(68.09880841,319.25224096)(68.29820098,319.4752176)(68.54261897,319.64245346)
\curveto(68.79132238,319.80968255)(69.07004318,319.93617891)(69.37878219,320.02194292)
\curveto(69.6060447,320.08197133)(69.94908568,320.13985949)(70.40790615,320.19560759)
\curveto(71.34268965,320.30709197)(72.03091561,320.44002035)(72.4725861,320.59439312)
\curveto(72.47686888,320.75304524)(72.47901289,320.85381353)(72.47901813,320.89669829)
\curveto(72.47901289,321.36837499)(72.36966858,321.70069594)(72.15098486,321.89366213)
\curveto(71.85510711,322.15522524)(71.41558586,322.28600961)(70.83241979,322.28601564)
\curveto(70.28783865,322.28600961)(69.8847655,322.18952933)(69.62319913,321.99657452)
\curveto(69.36591602,321.80789625)(69.17509948,321.47128729)(69.05074893,320.98674663)
\lineto(67.91871257,321.14111523)
\curveto(68.02162419,321.62565573)(68.19100067,322.01586484)(68.42684253,322.31174374)
\curveto(68.66268202,322.61189854)(69.00357899,322.84130719)(69.44953446,322.99997038)
\curveto(69.89548553,323.16290811)(70.412191,323.24438034)(70.99965243,323.24438733)
\curveto(71.58281834,323.24438034)(72.05664369,323.17577215)(72.4211299,323.03856253)
\curveto(72.78560576,322.90133936)(73.05360653,322.72767487)(73.225133,322.51756853)
\curveto(73.39664751,322.31173768)(73.51671185,322.05016894)(73.58532639,321.73286151)
\curveto(73.62391215,321.53560747)(73.64320821,321.17970246)(73.64321461,320.6651454)
\lineto(73.64321461,319.12145945)
\curveto(73.64320821,318.04516552)(73.66679228,317.36337158)(73.71396688,317.07607558)
\curveto(73.76541656,316.79306596)(73.86404084,316.52077718)(74.00984002,316.25920843)
\lineto(72.80061936,316.25920843)
\curveto(72.68054947,316.49933712)(72.60336525,316.78020192)(72.56906647,317.10180368)
\moveto(72.4725861,319.68747763)
\curveto(72.05235568,319.51595372)(71.42201788,319.3701613)(70.58157082,319.25009995)
\curveto(70.10559813,319.18148876)(69.76898917,319.10430454)(69.57174293,319.01854706)
\curveto(69.37449204,318.93278405)(69.22226761,318.80628769)(69.11506918,318.6390576)
\curveto(69.007867,318.47611075)(68.95426685,318.29387023)(68.95426856,318.09233549)
\curveto(68.95426685,317.78359678)(69.07004318,317.52631605)(69.30159789,317.32049252)
\curveto(69.53743651,317.11466687)(69.88047749,317.01175458)(70.33072186,317.01175533)
\curveto(70.77667204,317.01175458)(71.17331317,317.10823485)(71.52064644,317.30119645)
\curveto(71.86797115,317.49844397)(72.12310788,317.76644473)(72.28605738,318.10519954)
\curveto(72.4104047,318.36676644)(72.47258087,318.75268754)(72.4725861,319.262964)
\lineto(72.4725861,319.68747763)
}
}
{
\newrgbcolor{curcolor}{0 0 0}
\pscustom[linestyle=none,fillstyle=solid,fillcolor=curcolor]
{
\newpath
\moveto(75.44418045,313.64137436)
\lineto(75.44418045,323.09001873)
\lineto(76.49903251,323.09001873)
\lineto(76.49903251,322.20239931)
\curveto(76.74773529,322.54972236)(77.02860009,322.8091471)(77.34162775,322.98067431)
\curveto(77.65464988,323.15647609)(78.03413896,323.24438034)(78.48009613,323.24438733)
\curveto(79.06326189,323.24438034)(79.57782336,323.09429991)(80.02378208,322.79414559)
\curveto(80.4697299,322.4939782)(80.80633886,322.06946499)(81.03360996,321.52060469)
\curveto(81.26086815,320.97602188)(81.37450048,320.37784417)(81.37450728,319.72606978)
\curveto(81.37450048,319.02712032)(81.24800412,318.39678253)(80.99501782,317.8350545)
\curveto(80.74630669,317.27761134)(80.38182565,316.84881011)(79.90157361,316.54864955)
\curveto(79.42559892,316.25277642)(78.92390149,316.10483999)(78.39647981,316.10483984)
\curveto(78.01055489,316.10483999)(77.6632259,316.18631223)(77.3544918,316.34925678)
\curveto(77.05004015,316.51220116)(76.79919144,316.71802574)(76.6019449,316.96673116)
\lineto(76.6019449,313.64137436)
\lineto(75.44418045,313.64137436)
\moveto(76.49260048,319.63602143)
\curveto(76.49259857,318.75697555)(76.67055107,318.1073417)(77.02645854,317.68711793)
\curveto(77.3823611,317.26689131)(77.81330633,317.05677871)(78.31929552,317.0567795)
\curveto(78.83385324,317.05677871)(79.27337449,317.27332332)(79.63786059,317.70641401)
\curveto(80.00662458,318.14378981)(80.19100911,318.81915173)(80.19101472,319.73250181)
\curveto(80.19100911,320.60296481)(80.01091259,321.25474267)(79.65072464,321.68783733)
\curveto(79.29481455,322.12092114)(78.86815734,322.33746576)(78.37075171,322.33747183)
\curveto(77.87762651,322.33746576)(77.44024927,322.1059131)(77.05861866,321.64281316)
\curveto(76.6812711,321.18399047)(76.49259857,320.51506056)(76.49260048,319.63602143)
}
}
{
\newrgbcolor{curcolor}{0 0 0}
\pscustom[linestyle=none,fillstyle=solid,fillcolor=curcolor]
{
\newpath
\moveto(82.77668853,313.64137436)
\lineto(82.77668853,323.09001873)
\lineto(83.83154059,323.09001873)
\lineto(83.83154059,322.20239931)
\curveto(84.08024338,322.54972236)(84.36110818,322.8091471)(84.67413584,322.98067431)
\curveto(84.98715796,323.15647609)(85.36664705,323.24438034)(85.81260422,323.24438733)
\curveto(86.39576998,323.24438034)(86.91033145,323.09429991)(87.35629016,322.79414559)
\curveto(87.80223799,322.4939782)(88.13884695,322.06946499)(88.36611805,321.52060469)
\curveto(88.59337624,320.97602188)(88.70700857,320.37784417)(88.70701536,319.72606978)
\curveto(88.70700857,319.02712032)(88.58051221,318.39678253)(88.3275259,317.8350545)
\curveto(88.07881478,317.27761134)(87.71433374,316.84881011)(87.23408169,316.54864955)
\curveto(86.75810701,316.25277642)(86.25640958,316.10483999)(85.7289879,316.10483984)
\curveto(85.34306298,316.10483999)(84.99573399,316.18631223)(84.68699989,316.34925678)
\curveto(84.38254824,316.51220116)(84.13169953,316.71802574)(83.93445299,316.96673116)
\lineto(83.93445299,313.64137436)
\lineto(82.77668853,313.64137436)
\moveto(83.82510857,319.63602143)
\curveto(83.82510665,318.75697555)(84.00305916,318.1073417)(84.35896663,317.68711793)
\curveto(84.71486919,317.26689131)(85.14581442,317.05677871)(85.6518036,317.0567795)
\curveto(86.16636133,317.05677871)(86.60588258,317.27332332)(86.97036868,317.70641401)
\curveto(87.33913267,318.14378981)(87.52351719,318.81915173)(87.52352281,319.73250181)
\curveto(87.52351719,320.60296481)(87.34342068,321.25474267)(86.98323273,321.68783733)
\curveto(86.62732264,322.12092114)(86.20066542,322.33746576)(85.7032598,322.33747183)
\curveto(85.2101346,322.33746576)(84.77275735,322.1059131)(84.39112675,321.64281316)
\curveto(84.01377919,321.18399047)(83.82510665,320.51506056)(83.82510857,319.63602143)
}
}
{
\newrgbcolor{curcolor}{0 0 0}
\pscustom[linestyle=none,fillstyle=solid,fillcolor=curcolor]
{
\newpath
\moveto(96.2067561,320.38213631)
\lineto(89.96126005,317.71284603)
\lineto(89.96126005,318.86417846)
\lineto(94.90748709,320.91599436)
\lineto(89.96126005,322.94851419)
\lineto(89.96126005,324.09984662)
\lineto(96.2067561,321.46271647)
\lineto(96.2067561,320.38213631)
}
}
{
\newrgbcolor{curcolor}{0 0 0}
\pscustom[linestyle=none,fillstyle=solid,fillcolor=curcolor]
{
\newpath
\moveto(103.89945934,320.38213631)
\lineto(97.6539633,317.71284603)
\lineto(97.6539633,318.86417846)
\lineto(102.60019034,320.91599436)
\lineto(97.6539633,322.94851419)
\lineto(97.6539633,324.09984662)
\lineto(103.89945934,321.46271647)
\lineto(103.89945934,320.38213631)
}
}
{
\newrgbcolor{curcolor}{0 0 0}
\pscustom[linestyle=none,fillstyle=solid,fillcolor=curcolor]
{
\newpath
\moveto(64.65896087,296.49999213)
\lineto(64.65896087,305.92934043)
\lineto(66.46635983,305.92934043)
\lineto(66.46635983,296.49999213)
\lineto(64.65896087,296.49999213)
}
}
{
\newrgbcolor{curcolor}{0 0 0}
\pscustom[linestyle=none,fillstyle=solid,fillcolor=curcolor]
{
\newpath
\moveto(68.31235104,304.25701399)
\lineto(68.31235104,305.92934043)
\lineto(70.11975,305.92934043)
\lineto(70.11975,304.25701399)
\lineto(68.31235104,304.25701399)
\moveto(68.31235104,296.49999213)
\lineto(68.31235104,303.33080242)
\lineto(70.11975,303.33080242)
\lineto(70.11975,296.49999213)
\lineto(68.31235104,296.49999213)
}
}
{
\newrgbcolor{curcolor}{0 0 0}
\pscustom[linestyle=none,fillstyle=solid,fillcolor=curcolor]
{
\newpath
\moveto(78.17907713,296.49999213)
\lineto(76.37167817,296.49999213)
\lineto(76.37167817,299.98614955)
\curveto(76.37167282,300.72368416)(76.33308071,301.19965352)(76.25590173,301.41405904)
\curveto(76.17871227,301.63274275)(76.05221591,301.80211924)(75.87641227,301.922189)
\curveto(75.70488692,302.04224792)(75.49691833,302.10228009)(75.25250587,302.10228569)
\curveto(74.93947674,302.10228009)(74.65861194,302.01651985)(74.40991062,301.8450047)
\curveto(74.16120252,301.67347887)(73.98968203,301.44621422)(73.89534864,301.16321008)
\curveto(73.80529751,300.88019661)(73.76027338,300.35705912)(73.76027612,299.59379604)
\lineto(73.76027612,296.49999213)
\lineto(71.95287716,296.49999213)
\lineto(71.95287716,303.33080242)
\lineto(73.63163563,303.33080242)
\lineto(73.63163563,302.32740656)
\curveto(74.22766671,303.09924293)(74.97806885,303.48516403)(75.88284429,303.48517102)
\curveto(76.28162457,303.48516403)(76.64610561,303.41226782)(76.9762885,303.26648218)
\curveto(77.30645949,303.12497101)(77.5551642,302.94273049)(77.72240337,302.71976007)
\curveto(77.89391716,302.49677722)(78.0118375,302.24378449)(78.07616474,301.96078115)
\curveto(78.14476588,301.67776688)(78.17906997,301.27254973)(78.17907713,300.74512847)
\lineto(78.17907713,296.49999213)
}
}
{
\newrgbcolor{curcolor}{0 0 0}
\pscustom[linestyle=none,fillstyle=solid,fillcolor=curcolor]
{
\newpath
\moveto(79.95431508,296.49999213)
\lineto(79.95431508,305.92934043)
\lineto(81.76171404,305.92934043)
\lineto(81.76171404,300.92522516)
\lineto(83.87785019,303.33080242)
\lineto(86.10333075,303.33080242)
\lineto(83.76850576,300.83517682)
\lineto(86.2705634,296.49999213)
\lineto(84.32165989,296.49999213)
\lineto(82.60430928,299.56806794)
\lineto(81.76171404,298.68688054)
\lineto(81.76171404,296.49999213)
\lineto(79.95431508,296.49999213)
}
}
{
\newrgbcolor{curcolor}{0 0 0}
\pscustom[linestyle=none,fillstyle=solid,fillcolor=curcolor]
{
\newpath
\moveto(86.71437296,298.44889563)
\lineto(88.52820395,298.72547269)
\curveto(88.60538604,298.37385347)(88.76189849,298.1058527)(88.99774175,297.9214696)
\curveto(89.23357983,297.74137166)(89.56375678,297.65132341)(89.98827357,297.65132456)
\curveto(90.45566332,297.65132341)(90.80728032,297.73708365)(91.04312563,297.90860555)
\curveto(91.20177744,298.02866848)(91.28110567,298.18946894)(91.28111054,298.39100741)
\curveto(91.28110567,298.52822191)(91.23822555,298.64185423)(91.15247005,298.73190472)
\curveto(91.06241705,298.81766273)(90.86088047,298.89699096)(90.54785972,298.96988963)
\curveto(89.08993142,299.29148808)(88.16586479,299.58521692)(87.77565705,299.85107703)
\curveto(87.23536614,300.21984273)(86.96522137,300.73226019)(86.96522193,301.38833094)
\curveto(86.96522137,301.98007174)(87.19891804,302.47748116)(87.66631263,302.88056069)
\curveto(88.1337047,303.28362746)(88.85837876,303.48516403)(89.840337,303.48517102)
\curveto(90.77512023,303.48516403)(91.46977821,303.3329396)(91.92431302,303.02849726)
\curveto(92.3788368,302.72404186)(92.69186169,302.27380058)(92.86338864,301.67777206)
\lineto(91.15890207,301.36260285)
\curveto(91.08600111,301.62845474)(90.94664072,301.83213532)(90.74082046,301.9736452)
\curveto(90.53927956,302.11514413)(90.24983873,302.18589633)(89.87249712,302.18590202)
\curveto(89.3965243,302.18589633)(89.05562733,302.11943214)(88.84980518,301.98650925)
\curveto(88.71258635,301.89216749)(88.64397815,301.76995914)(88.64398039,301.61988384)
\curveto(88.64397815,301.49123835)(88.70401032,301.38189404)(88.82407709,301.29185057)
\curveto(88.98701913,301.17178144)(89.54874873,301.00240496)(90.50926757,300.78372062)
\curveto(91.47406622,300.56502771)(92.14728414,300.29702695)(92.52892335,299.97971752)
\curveto(92.9062623,299.65811313)(93.09493484,299.21001585)(93.09494153,298.63542435)
\curveto(93.09493484,298.00937243)(92.83336609,297.47122689)(92.31023451,297.02098613)
\curveto(91.78709111,296.57074433)(91.01310491,296.34562369)(89.98827357,296.34562353)
\curveto(89.05777133,296.34562369)(88.32023323,296.53429622)(87.77565705,296.91164171)
\curveto(87.23536614,297.28898637)(86.88160513,297.80140383)(86.71437296,298.44889563)
}
}
{
\newrgbcolor{curcolor}{0 1 0.25098041}
\pscustom[linewidth=2.63455725,linecolor=curcolor]
{
\newpath
\moveto(283.00129,165.29658456)
\lineto(367.30712,165.29658456)
\lineto(367.30712,217.98774456)
\lineto(311.98142,217.98774456)
\lineto(311.98142,228.52596456)
\lineto(283.00129,228.52596456)
\lineto(283.00129,165.29658456)
\closepath
}
}
{
\newrgbcolor{curcolor}{0 0 0}
\pscustom[linestyle=none,fillstyle=solid,fillcolor=curcolor]
{
\newpath
\moveto(298.21174111,200.97753031)
\lineto(298.21174111,202.05811047)
\lineto(304.45723715,204.69524062)
\lineto(304.45723715,203.54390819)
\lineto(299.50457809,201.51138836)
\lineto(304.45723715,199.45957246)
\lineto(304.45723715,198.30824003)
\lineto(298.21174111,200.97753031)
}
}
{
\newrgbcolor{curcolor}{0 0 0}
\pscustom[linestyle=none,fillstyle=solid,fillcolor=curcolor]
{
\newpath
\moveto(305.90444293,200.97753031)
\lineto(305.90444293,202.05811047)
\lineto(312.14993897,204.69524062)
\lineto(312.14993897,203.54390819)
\lineto(307.1972799,201.51138836)
\lineto(312.14993897,199.45957246)
\lineto(312.14993897,198.30824003)
\lineto(305.90444293,200.97753031)
}
}
{
\newrgbcolor{curcolor}{0 0 0}
\pscustom[linestyle=none,fillstyle=solid,fillcolor=curcolor]
{
\newpath
\moveto(318.20247447,197.69719768)
\curveto(317.77366793,197.3327158)(317.35987475,197.07543506)(316.96109369,196.92535471)
\curveto(316.56659249,196.77527421)(316.14207928,196.70023399)(315.68755279,196.70023384)
\curveto(314.93714784,196.70023399)(314.3604102,196.88247451)(313.95733813,197.24695594)
\curveto(313.5542639,197.6157246)(313.35272733,198.08526194)(313.3527278,198.65556937)
\curveto(313.35272733,198.99003252)(313.42776754,199.29448139)(313.57784867,199.56891688)
\curveto(313.73221641,199.84763496)(313.93160898,200.0706116)(314.17602697,200.23784746)
\curveto(314.42473038,200.40507655)(314.70345118,200.53157291)(315.01219019,200.61733692)
\curveto(315.2394527,200.67736533)(315.58249368,200.73525349)(316.04131415,200.79100159)
\curveto(316.97609765,200.90248597)(317.66432361,201.03541435)(318.1059941,201.18978712)
\curveto(318.11027688,201.34843924)(318.11242089,201.44920753)(318.11242613,201.49209229)
\curveto(318.11242089,201.96376899)(318.00307658,202.29608994)(317.78439286,202.48905613)
\curveto(317.48851511,202.75061924)(317.04899386,202.88140361)(316.46582779,202.88140964)
\curveto(315.92124665,202.88140361)(315.5181735,202.78492333)(315.25660713,202.59196852)
\curveto(314.99932402,202.40329025)(314.80850748,202.06668129)(314.68415693,201.58214063)
\lineto(313.55212057,201.73650923)
\curveto(313.65503219,202.22104973)(313.82440867,202.61125884)(314.06025053,202.90713773)
\curveto(314.29609002,203.20729254)(314.63698699,203.43670119)(315.08294246,203.59536438)
\curveto(315.52889353,203.75830211)(316.045599,203.83977434)(316.63306043,203.83978133)
\curveto(317.21622634,203.83977434)(317.69005169,203.77116614)(318.0545379,203.63395653)
\curveto(318.41901376,203.49673336)(318.68701453,203.32306887)(318.858541,203.11296253)
\curveto(319.03005551,202.90713168)(319.15011985,202.64556294)(319.21873439,202.32825551)
\curveto(319.25732015,202.13100147)(319.27661621,201.77509646)(319.27662261,201.2605394)
\lineto(319.27662261,199.71685345)
\curveto(319.27661621,198.64055952)(319.30020028,197.95876558)(319.34737488,197.67146958)
\curveto(319.39882456,197.38845995)(319.49744884,197.11617118)(319.64324802,196.85460243)
\lineto(318.43402736,196.85460243)
\curveto(318.31395747,197.09473112)(318.23677325,197.37559592)(318.20247447,197.69719768)
\moveto(318.1059941,200.28287163)
\curveto(317.68576368,200.11134771)(317.05542588,199.9655553)(316.21497882,199.84549395)
\curveto(315.73900613,199.77688276)(315.40239717,199.69969854)(315.20515093,199.61394106)
\curveto(315.00790004,199.52817805)(314.85567561,199.40168169)(314.74847718,199.2344516)
\curveto(314.641275,199.07150475)(314.58767485,198.88926423)(314.58767656,198.68772949)
\curveto(314.58767485,198.37899078)(314.70345118,198.12171004)(314.93500589,197.91588652)
\curveto(315.17084451,197.71006087)(315.51388549,197.60714858)(315.96412986,197.60714933)
\curveto(316.41008004,197.60714858)(316.80672117,197.70362885)(317.15405444,197.89659045)
\curveto(317.50137915,198.09383797)(317.75651588,198.36183873)(317.91946538,198.70059354)
\curveto(318.0438127,198.96216044)(318.10598887,199.34808154)(318.1059941,199.858358)
\lineto(318.1059941,200.28287163)
}
}
{
\newrgbcolor{curcolor}{0 0 0}
\pscustom[linestyle=none,fillstyle=solid,fillcolor=curcolor]
{
\newpath
\moveto(321.07758845,194.23676836)
\lineto(321.07758845,203.68541273)
\lineto(322.13244051,203.68541273)
\lineto(322.13244051,202.79779331)
\curveto(322.38114329,203.14511636)(322.66200809,203.4045411)(322.97503575,203.57606831)
\curveto(323.28805788,203.75187009)(323.66754696,203.83977434)(324.11350413,203.83978133)
\curveto(324.69666989,203.83977434)(325.21123136,203.68969391)(325.65719008,203.38953959)
\curveto(326.1031379,203.0893722)(326.43974686,202.66485899)(326.66701796,202.11599869)
\curveto(326.89427615,201.57141588)(327.00790848,200.97323817)(327.00791528,200.32146378)
\curveto(327.00790848,199.62251432)(326.88141212,198.99217653)(326.62842582,198.4304485)
\curveto(326.37971469,197.87300534)(326.01523365,197.44420411)(325.53498161,197.14404355)
\curveto(325.05900692,196.84817042)(324.55730949,196.70023399)(324.02988781,196.70023384)
\curveto(323.64396289,196.70023399)(323.2966339,196.78170623)(322.9878998,196.94465078)
\curveto(322.68344815,197.10759515)(322.43259944,197.31341974)(322.2353529,197.56212516)
\lineto(322.2353529,194.23676836)
\lineto(321.07758845,194.23676836)
\moveto(322.12600848,200.23141543)
\curveto(322.12600657,199.35236955)(322.30395907,198.7027357)(322.65986654,198.28251193)
\curveto(323.0157691,197.86228531)(323.44671433,197.65217271)(323.95270352,197.6521735)
\curveto(324.46726124,197.65217271)(324.90678249,197.86871732)(325.27126859,198.30180801)
\curveto(325.64003258,198.7391838)(325.82441711,199.41454573)(325.82442272,200.32789581)
\curveto(325.82441711,201.19835881)(325.64432059,201.85013667)(325.28413264,202.28323133)
\curveto(324.92822255,202.71631514)(324.50156534,202.93285976)(324.00415971,202.93286583)
\curveto(323.51103451,202.93285976)(323.07365727,202.7013071)(322.69202666,202.23820716)
\curveto(322.3146791,201.77938447)(322.12600657,201.11045456)(322.12600848,200.23141543)
}
}
{
\newrgbcolor{curcolor}{0 0 0}
\pscustom[linestyle=none,fillstyle=solid,fillcolor=curcolor]
{
\newpath
\moveto(328.41009653,194.23676836)
\lineto(328.41009653,203.68541273)
\lineto(329.46494859,203.68541273)
\lineto(329.46494859,202.79779331)
\curveto(329.71365138,203.14511636)(329.99451618,203.4045411)(330.30754384,203.57606831)
\curveto(330.62056596,203.75187009)(331.00005505,203.83977434)(331.44601222,203.83978133)
\curveto(332.02917798,203.83977434)(332.54373945,203.68969391)(332.98969816,203.38953959)
\curveto(333.43564599,203.0893722)(333.77225495,202.66485899)(333.99952605,202.11599869)
\curveto(334.22678424,201.57141588)(334.34041657,200.97323817)(334.34042336,200.32146378)
\curveto(334.34041657,199.62251432)(334.21392021,198.99217653)(333.9609339,198.4304485)
\curveto(333.71222278,197.87300534)(333.34774174,197.44420411)(332.86748969,197.14404355)
\curveto(332.39151501,196.84817042)(331.88981758,196.70023399)(331.3623959,196.70023384)
\curveto(330.97647098,196.70023399)(330.62914199,196.78170623)(330.32040789,196.94465078)
\curveto(330.01595624,197.10759515)(329.76510753,197.31341974)(329.56786099,197.56212516)
\lineto(329.56786099,194.23676836)
\lineto(328.41009653,194.23676836)
\moveto(329.45851657,200.23141543)
\curveto(329.45851465,199.35236955)(329.63646716,198.7027357)(329.99237463,198.28251193)
\curveto(330.34827719,197.86228531)(330.77922242,197.65217271)(331.2852116,197.6521735)
\curveto(331.79976933,197.65217271)(332.23929058,197.86871732)(332.60377668,198.30180801)
\curveto(332.97254067,198.7391838)(333.15692519,199.41454573)(333.15693081,200.32789581)
\curveto(333.15692519,201.19835881)(332.97682868,201.85013667)(332.61664073,202.28323133)
\curveto(332.26073064,202.71631514)(331.83407342,202.93285976)(331.3366678,202.93286583)
\curveto(330.8435426,202.93285976)(330.40616535,202.7013071)(330.02453475,202.23820716)
\curveto(329.64718719,201.77938447)(329.45851465,201.11045456)(329.45851657,200.23141543)
}
}
{
\newrgbcolor{curcolor}{0 0 0}
\pscustom[linestyle=none,fillstyle=solid,fillcolor=curcolor]
{
\newpath
\moveto(341.8401641,200.97753031)
\lineto(335.59466805,198.30824003)
\lineto(335.59466805,199.45957246)
\lineto(340.54089509,201.51138836)
\lineto(335.59466805,203.54390819)
\lineto(335.59466805,204.69524062)
\lineto(341.8401641,202.05811047)
\lineto(341.8401641,200.97753031)
}
}
{
\newrgbcolor{curcolor}{0 0 0}
\pscustom[linestyle=none,fillstyle=solid,fillcolor=curcolor]
{
\newpath
\moveto(349.53286734,200.97753031)
\lineto(343.2873713,198.30824003)
\lineto(343.2873713,199.45957246)
\lineto(348.23359834,201.51138836)
\lineto(343.2873713,203.54390819)
\lineto(343.2873713,204.69524062)
\lineto(349.53286734,202.05811047)
\lineto(349.53286734,200.97753031)
}
}
{
\newrgbcolor{curcolor}{0 0 0}
\pscustom[linestyle=none,fillstyle=solid,fillcolor=curcolor]
{
\newpath
\moveto(296.32844693,183.92619642)
\lineto(297.33184279,183.92619642)
\lineto(297.33184279,184.4407584)
\curveto(297.33184164,185.0153447)(297.39187381,185.44414592)(297.51193949,185.72716336)
\curveto(297.6362905,186.01016353)(297.86141114,186.23957218)(298.18730209,186.41539001)
\curveto(298.51747701,186.5954772)(298.9334142,186.68552546)(299.43511489,186.68553505)
\curveto(299.9496731,186.68552546)(300.45351453,186.60834124)(300.94664071,186.45398215)
\lineto(300.70222377,185.1933053)
\curveto(300.41492242,185.2619054)(300.13834563,185.2962095)(299.87249257,185.2962177)
\curveto(299.61092013,185.2962095)(299.42224759,185.23403332)(299.3064744,185.10968898)
\curveto(299.19498295,184.98961662)(299.13923879,184.75591996)(299.13924175,184.40859828)
\lineto(299.13924175,183.92619642)
\lineto(300.48996695,183.92619642)
\lineto(300.48996695,182.50471895)
\lineto(299.13924175,182.50471895)
\lineto(299.13924175,177.09538613)
\lineto(297.33184279,177.09538613)
\lineto(297.33184279,182.50471895)
\lineto(296.32844693,182.50471895)
\lineto(296.32844693,183.92619642)
}
}
{
\newrgbcolor{curcolor}{0 0 0}
\pscustom[linestyle=none,fillstyle=solid,fillcolor=curcolor]
{
\newpath
\moveto(305.47478634,179.26941049)
\lineto(307.27575328,178.96710533)
\curveto(307.04419391,178.30674958)(306.67756887,177.80290814)(306.17587704,177.45557951)
\curveto(305.67846202,177.11253817)(305.05455624,176.94101769)(304.30415784,176.94101753)
\curveto(303.11637472,176.94101769)(302.23733222,177.32908279)(301.66702768,178.10521401)
\curveto(301.21678531,178.72697477)(300.99166467,179.51168101)(300.99166508,180.45933508)
\curveto(300.99166467,181.59136694)(301.28753751,182.47684146)(301.8792845,183.1157613)
\curveto(302.47102888,183.75895711)(303.21928701,184.08055803)(304.12406114,184.08056502)
\curveto(305.14031649,184.08055803)(305.94217477,183.74394907)(306.5296384,183.07073713)
\curveto(307.11709012,182.40180125)(307.39795492,181.37482232)(307.37223365,179.98979727)
\lineto(302.84408822,179.98979727)
\curveto(302.85694998,179.45379285)(303.0027424,179.03571165)(303.2814659,178.73555244)
\curveto(303.56018399,178.43967796)(303.90751298,178.29174153)(304.32345391,178.29174273)
\curveto(304.60645897,178.29174153)(304.84444365,178.36892575)(305.03740866,178.52329562)
\curveto(305.23036475,178.67766263)(305.37615716,178.92636734)(305.47478634,179.26941049)
\moveto(305.57769874,181.09610553)
\curveto(305.5648297,181.61923902)(305.42975731,182.01588015)(305.17248118,182.28603011)
\curveto(304.91519585,182.5604577)(304.60217096,182.69767409)(304.23340556,182.69767969)
\curveto(303.83890478,182.69767409)(303.51301585,182.55402568)(303.2557378,182.26673403)
\curveto(302.99845438,181.97943204)(302.87195802,181.58922293)(302.87624834,181.09610553)
\lineto(305.57769874,181.09610553)
}
}
{
\newrgbcolor{curcolor}{0 0 0}
\pscustom[linestyle=none,fillstyle=solid,fillcolor=curcolor]
{
\newpath
\moveto(312.80729491,179.26941049)
\lineto(314.60826184,178.96710533)
\curveto(314.37670248,178.30674958)(314.01007743,177.80290814)(313.50838561,177.45557951)
\curveto(313.01097059,177.11253817)(312.38706481,176.94101769)(311.6366664,176.94101753)
\curveto(310.44888329,176.94101769)(309.56984078,177.32908279)(308.99953625,178.10521401)
\curveto(308.54929387,178.72697477)(308.32417323,179.51168101)(308.32417365,180.45933508)
\curveto(308.32417323,181.59136694)(308.62004607,182.47684146)(309.21179307,183.1157613)
\curveto(309.80353745,183.75895711)(310.55179558,184.08055803)(311.45656971,184.08056502)
\curveto(312.47282505,184.08055803)(313.27468334,183.74394907)(313.86214697,183.07073713)
\curveto(314.44959869,182.40180125)(314.73046349,181.37482232)(314.70474221,179.98979727)
\lineto(310.17659678,179.98979727)
\curveto(310.18945855,179.45379285)(310.33525096,179.03571165)(310.61397446,178.73555244)
\curveto(310.89269255,178.43967796)(311.24002154,178.29174153)(311.65596247,178.29174273)
\curveto(311.93896753,178.29174153)(312.17695221,178.36892575)(312.36991722,178.52329562)
\curveto(312.56287331,178.67766263)(312.70866572,178.92636734)(312.80729491,179.26941049)
\moveto(312.9102073,181.09610553)
\curveto(312.89733826,181.61923902)(312.76226588,182.01588015)(312.50498974,182.28603011)
\curveto(312.24770441,182.5604577)(311.93467952,182.69767409)(311.56591413,182.69767969)
\curveto(311.17141334,182.69767409)(310.84552442,182.55402568)(310.58824636,182.26673403)
\curveto(310.33096295,181.97943204)(310.20446659,181.58922293)(310.2087569,181.09610553)
\lineto(312.9102073,181.09610553)
}
}
{
\newrgbcolor{curcolor}{0 0 0}
\pscustom[linestyle=none,fillstyle=solid,fillcolor=curcolor]
{
\newpath
\moveto(322.44889988,177.09538613)
\lineto(320.77014142,177.09538613)
\lineto(320.77014142,178.09878199)
\curveto(320.4914151,177.70857187)(320.16123815,177.41698704)(319.77960961,177.22402662)
\curveto(319.40225999,177.03535395)(319.0206269,176.94101769)(318.6347092,176.94101753)
\curveto(317.84999957,176.94101769)(317.17678165,177.25618658)(316.61505342,177.88652517)
\curveto(316.05761046,178.52115019)(315.77888967,179.40448071)(315.77889021,180.53651937)
\curveto(315.77888967,181.69427923)(316.05117844,182.57332174)(316.59575735,183.17364953)
\curveto(317.14033355,183.77825317)(317.82855951,184.08055803)(318.6604373,184.08056502)
\curveto(319.42370005,184.08055803)(320.08405393,183.76324513)(320.64150093,183.12862535)
\lineto(320.64150093,186.52473443)
\lineto(322.44889988,186.52473443)
\lineto(322.44889988,177.09538613)
\moveto(317.62488131,180.65872784)
\curveto(317.62487893,179.9297622)(317.72564721,179.4023367)(317.92718648,179.07644975)
\curveto(318.21876862,178.60476643)(318.62612978,178.36892575)(319.14927118,178.36892703)
\curveto(319.56520446,178.36892575)(319.91896546,178.54473426)(320.21055527,178.89635306)
\curveto(320.50213513,179.25225627)(320.64792754,179.78182578)(320.64793295,180.48506317)
\curveto(320.64792754,181.26976602)(320.50642314,181.83363963)(320.22341932,182.17668569)
\curveto(319.94040552,182.5240096)(319.57806849,182.69767409)(319.13640713,182.69767969)
\curveto(318.70760201,182.69767409)(318.34740899,182.5261536)(318.05582697,182.18311771)
\curveto(317.76852734,181.84435966)(317.62487893,181.33623021)(317.62488131,180.65872784)
}
}
{
\newrgbcolor{curcolor}{0 0 0}
\pscustom[linestyle=none,fillstyle=solid,fillcolor=curcolor]
{
\newpath
\moveto(324.15981806,177.09538613)
\lineto(324.15981806,186.52473443)
\lineto(325.96721702,186.52473443)
\lineto(325.96721702,183.12862535)
\curveto(326.52465593,183.76324513)(327.18500981,184.08055803)(327.94828065,184.08056502)
\curveto(328.78015036,184.08055803)(329.46837632,183.77825317)(330.01296059,183.17364953)
\curveto(330.55753142,182.57332174)(330.8298202,181.70928727)(330.82982774,180.58154355)
\curveto(330.8298202,179.41520074)(330.55109941,178.51686218)(329.99366452,177.88652517)
\curveto(329.44050424,177.25618658)(328.76728632,176.94101769)(327.97400874,176.94101753)
\curveto(327.58379495,176.94101769)(327.19787385,177.03749796)(326.81624429,177.23045865)
\curveto(326.43889569,177.42770707)(326.11300676,177.7171479)(325.83857652,178.09878199)
\lineto(325.83857652,177.09538613)
\lineto(324.15981806,177.09538613)
\moveto(325.95435297,180.65872784)
\curveto(325.95435031,179.95120226)(326.06583862,179.42806477)(326.28881826,179.0893138)
\curveto(326.60184015,178.60905444)(327.01777734,178.36892575)(327.53663106,178.36892703)
\curveto(327.93541195,178.36892575)(328.27416492,178.53830224)(328.55289097,178.87705698)
\curveto(328.83589452,179.22009618)(328.97739892,179.75824171)(328.97740461,180.4914952)
\curveto(328.97739892,181.27191003)(328.83589452,181.83363963)(328.55289097,182.17668569)
\curveto(328.26987691,182.5240096)(327.90753987,182.69767409)(327.46587879,182.69767969)
\curveto(327.03278538,182.69767409)(326.67259235,182.52829761)(326.38529863,182.18954974)
\curveto(326.09799872,181.85507969)(325.95435031,181.34480624)(325.95435297,180.65872784)
}
}
{
\newrgbcolor{curcolor}{0 0 0}
\pscustom[linestyle=none,fillstyle=solid,fillcolor=curcolor]
{
\newpath
\moveto(333.64062215,181.8422204)
\lineto(332.00045584,182.13809354)
\curveto(332.18483971,182.79844238)(332.50215261,183.28727577)(332.9523955,183.60459518)
\curveto(333.40263518,183.92190158)(334.07156508,184.08055803)(334.95918723,184.08056502)
\curveto(335.76532991,184.08055803)(336.36565162,183.98407776)(336.76015416,183.7911239)
\curveto(337.15464587,183.60244467)(337.43122266,183.36017198)(337.58988535,183.0643051)
\curveto(337.75282357,182.7727143)(337.83429581,182.23456877)(337.8343023,181.44986689)
\lineto(337.81500622,179.34016277)
\curveto(337.81499975,178.73983881)(337.84287183,178.29602955)(337.89862254,178.00873364)
\curveto(337.95864816,177.72572392)(338.06799247,177.42127505)(338.22665581,177.09538613)
\lineto(336.43855292,177.09538613)
\curveto(336.39137969,177.21545047)(336.33349153,177.39340297)(336.26488825,177.62924418)
\curveto(336.23486725,177.73644395)(336.21342719,177.80719615)(336.20056801,177.841501)
\curveto(335.89182627,177.5413394)(335.56164933,177.31621875)(335.21003619,177.1661384)
\curveto(334.85841532,177.0160579)(334.48321426,176.94101769)(334.08443186,176.94101753)
\curveto(333.38119511,176.94101769)(332.82589753,177.13183423)(332.41853745,177.51346773)
\curveto(332.01546322,177.8951004)(331.81392665,178.37750178)(331.81392712,178.96067331)
\curveto(331.81392665,179.34659254)(331.90611891,179.68963352)(332.09050418,179.98979727)
\curveto(332.27488796,180.29424324)(332.5321687,180.5257959)(332.86234715,180.68445594)
\curveto(333.19681059,180.84739682)(333.67706796,180.98890122)(334.3031207,181.10896958)
\curveto(335.14785615,181.26762202)(335.73316982,181.41555844)(336.05906346,181.55277929)
\lineto(336.05906346,181.73287598)
\curveto(336.05905875,182.08020033)(335.9732985,182.32676103)(335.80178247,182.47255883)
\curveto(335.63025752,182.62263388)(335.3065126,182.69767409)(334.83054673,182.69767969)
\curveto(334.50894233,182.69767409)(334.25809361,182.63335391)(334.07799983,182.50471895)
\curveto(333.89790059,182.38036119)(333.75210817,182.15952856)(333.64062215,181.8422204)
\moveto(336.05906346,180.37571875)
\curveto(335.82750609,180.29853125)(335.46088104,180.20633899)(334.95918723,180.09914169)
\curveto(334.45748618,179.99193838)(334.12945325,179.88688208)(333.97508744,179.78397248)
\curveto(333.73924414,179.61673731)(333.6213238,179.40448071)(333.62132608,179.14720202)
\curveto(333.6213238,178.89420725)(333.71566007,178.67551863)(333.90433517,178.4911355)
\curveto(334.09300514,178.30674958)(334.33313383,178.21455731)(334.62472194,178.21455843)
\curveto(334.95060759,178.21455731)(335.26148847,178.32175762)(335.55736553,178.53615967)
\curveto(335.77604994,178.6991027)(335.91969835,178.89849526)(335.98831119,179.13433797)
\curveto(336.03547468,179.28870438)(336.05905875,179.58243321)(336.05906346,180.01552537)
\lineto(336.05906346,180.37571875)
}
}
{
\newrgbcolor{curcolor}{0 0 0}
\pscustom[linestyle=none,fillstyle=solid,fillcolor=curcolor]
{
\newpath
\moveto(345.57845997,181.90654065)
\lineto(343.79678911,181.58493941)
\curveto(343.73675182,181.94083993)(343.59953543,182.2088407)(343.38513952,182.3889425)
\curveto(343.17502222,182.56903372)(342.90058943,182.65908198)(342.56184035,182.65908754)
\curveto(342.11159519,182.65908198)(341.75140216,182.50256953)(341.48126019,182.18954974)
\curveto(341.21540063,181.88080776)(341.08247225,181.36195828)(341.08247466,180.63299974)
\curveto(341.08247225,179.8225619)(341.21754464,179.25011227)(341.48769222,178.91564913)
\curveto(341.76212219,178.58118236)(342.12874723,178.41394988)(342.58756845,178.4139512)
\curveto(342.93060552,178.41394988)(343.21147032,178.51043016)(343.4301637,178.70339232)
\curveto(343.64884757,178.90063927)(343.80321601,179.23724823)(343.89326948,179.7132202)
\lineto(345.66850831,179.41091504)
\curveto(345.4841168,178.5961904)(345.13035579,177.98086065)(344.60722423,177.56492393)
\curveto(344.08408081,177.14898628)(343.38299081,176.94101769)(342.50395213,176.94101753)
\curveto(341.50484146,176.94101769)(340.70727118,177.25618658)(340.11123892,177.88652517)
\curveto(339.5194918,178.51686218)(339.22361896,179.38947266)(339.2236195,180.50435925)
\curveto(339.22361896,181.63210305)(339.5216358,182.50900155)(340.11767094,183.13505738)
\curveto(340.7137032,183.76538913)(341.5198495,184.08055803)(342.53611225,184.08056502)
\curveto(343.36798277,184.08055803)(344.02833665,183.90046152)(344.51717588,183.54027494)
\curveto(345.01029145,183.18436348)(345.36405245,182.63978593)(345.57845997,181.90654065)
}
}
{
\newrgbcolor{curcolor}{0 0 0}
\pscustom[linestyle=none,fillstyle=solid,fillcolor=curcolor]
{
\newpath
\moveto(346.89059288,177.09538613)
\lineto(346.89059288,186.52473443)
\lineto(348.69799184,186.52473443)
\lineto(348.69799184,181.52061916)
\lineto(350.81412798,183.92619642)
\lineto(353.03960855,183.92619642)
\lineto(350.70478356,181.43057081)
\lineto(353.20684119,177.09538613)
\lineto(351.25793769,177.09538613)
\lineto(349.54058708,180.16346194)
\lineto(348.69799184,179.28227454)
\lineto(348.69799184,177.09538613)
\lineto(346.89059288,177.09538613)
}
}
{
\newrgbcolor{curcolor}{0.50196081 0.50196081 0}
\pscustom[linewidth=2.63455725,linecolor=curcolor]
{
\newpath
\moveto(404.51,284.70119456)
\lineto(488.81584,284.70119456)
\lineto(488.81584,337.39234456)
\lineto(432.17286,337.39234456)
\lineto(432.17286,347.93057456)
\lineto(404.51,347.93057456)
\lineto(404.51,284.70119456)
\closepath
}
}
{
\newrgbcolor{curcolor}{0 0 0}
\pscustom[linestyle=none,fillstyle=solid,fillcolor=curcolor]
{
\newpath
\moveto(422.35500503,318.79343021)
\lineto(422.35500503,319.87401037)
\lineto(428.60050107,322.51114052)
\lineto(428.60050107,321.35980809)
\lineto(423.647842,319.32728826)
\lineto(428.60050107,317.27547236)
\lineto(428.60050107,316.12413993)
\lineto(422.35500503,318.79343021)
}
}
{
\newrgbcolor{curcolor}{0 0 0}
\pscustom[linestyle=none,fillstyle=solid,fillcolor=curcolor]
{
\newpath
\moveto(430.04770684,318.79343021)
\lineto(430.04770684,319.87401037)
\lineto(436.29320289,322.51114052)
\lineto(436.29320289,321.35980809)
\lineto(431.34054382,319.32728826)
\lineto(436.29320289,317.27547236)
\lineto(436.29320289,316.12413993)
\lineto(430.04770684,318.79343021)
}
}
{
\newrgbcolor{curcolor}{0 0 0}
\pscustom[linestyle=none,fillstyle=solid,fillcolor=curcolor]
{
\newpath
\moveto(442.36503446,314.67050233)
\lineto(442.36503446,315.6738982)
\curveto(441.8333156,314.90205499)(441.11078554,314.51613389)(440.19744212,314.51613374)
\curveto(439.79436579,314.51613389)(439.41702072,314.59331811)(439.06540576,314.74768663)
\curveto(438.71807472,314.90205499)(438.45864998,315.09501554)(438.28713076,315.32656886)
\curveto(438.11989702,315.56240887)(438.00197668,315.84970569)(437.9333694,316.18846018)
\curveto(437.88620035,316.41572331)(437.86261629,316.77591633)(437.86261713,317.26904034)
\lineto(437.86261713,321.50131263)
\lineto(439.02038159,321.50131263)
\lineto(439.02038159,317.71285005)
\curveto(439.02037959,317.10823728)(439.04396365,316.70087612)(439.09113386,316.49076534)
\curveto(439.164028,316.18631465)(439.31839643,315.94618597)(439.55423964,315.77037857)
\curveto(439.79007778,315.59885698)(440.08166261,315.51309673)(440.42899501,315.51309758)
\curveto(440.77632059,315.51309673)(441.10220952,315.60100098)(441.40666277,315.77681059)
\curveto(441.71110725,315.956906)(441.92550786,316.19917869)(442.04986525,316.50362939)
\curveto(442.17850059,316.81236444)(442.24282077,317.25831771)(442.24282599,317.84149054)
\lineto(442.24282599,321.50131263)
\lineto(443.40059045,321.50131263)
\lineto(443.40059045,314.67050233)
\lineto(442.36503446,314.67050233)
}
}
{
\newrgbcolor{curcolor}{0 0 0}
\pscustom[linestyle=none,fillstyle=solid,fillcolor=curcolor]
{
\newpath
\moveto(447.74863809,315.70605832)
\lineto(447.91587074,314.68336638)
\curveto(447.58997824,314.61475817)(447.29839341,314.58045408)(447.04111537,314.58045399)
\curveto(446.62088748,314.58045408)(446.29499855,314.64691827)(446.06344761,314.77984675)
\curveto(445.83189323,314.91277502)(445.66894877,315.08643952)(445.57461372,315.30084076)
\curveto(445.48027623,315.51952875)(445.4331081,315.97620205)(445.43310918,316.67086203)
\lineto(445.43310918,320.60082916)
\lineto(444.58408191,320.60082916)
\lineto(444.58408191,321.50131263)
\lineto(445.43310918,321.50131263)
\lineto(445.43310918,323.19293514)
\lineto(446.58444161,323.88759382)
\lineto(446.58444161,321.50131263)
\lineto(447.74863809,321.50131263)
\lineto(447.74863809,320.60082916)
\lineto(446.58444161,320.60082916)
\lineto(446.58444161,316.60654179)
\curveto(446.58443938,316.27636291)(446.60373543,316.0641063)(446.64232983,315.96977133)
\curveto(446.68520767,315.87543377)(446.75167186,315.80039355)(446.8417226,315.74465047)
\curveto(446.93605638,315.68890524)(447.06898476,315.66103316)(447.24050814,315.66103415)
\curveto(447.36914562,315.66103316)(447.5385221,315.6760412)(447.74863809,315.70605832)
}
}
{
\newrgbcolor{curcolor}{0 0 0}
\pscustom[linestyle=none,fillstyle=solid,fillcolor=curcolor]
{
\newpath
\moveto(448.88067527,322.76842151)
\lineto(448.88067527,324.09985063)
\lineto(450.03843973,324.09985063)
\lineto(450.03843973,322.76842151)
\lineto(448.88067527,322.76842151)
\moveto(448.88067527,314.67050233)
\lineto(448.88067527,321.50131263)
\lineto(450.03843973,321.50131263)
\lineto(450.03843973,314.67050233)
\lineto(448.88067527,314.67050233)
}
}
{
\newrgbcolor{curcolor}{0 0 0}
\pscustom[linestyle=none,fillstyle=solid,fillcolor=curcolor]
{
\newpath
\moveto(451.78151953,314.67050233)
\lineto(451.78151953,324.09985063)
\lineto(452.93928398,324.09985063)
\lineto(452.93928398,314.67050233)
\lineto(451.78151953,314.67050233)
}
}
{
\newrgbcolor{curcolor}{0 0 0}
\pscustom[linestyle=none,fillstyle=solid,fillcolor=curcolor]
{
\newpath
\moveto(460.83780957,318.79343021)
\lineto(454.59231353,316.12413993)
\lineto(454.59231353,317.27547236)
\lineto(459.53854057,319.32728826)
\lineto(454.59231353,321.35980809)
\lineto(454.59231353,322.51114052)
\lineto(460.83780957,319.87401037)
\lineto(460.83780957,318.79343021)
}
}
{
\newrgbcolor{curcolor}{0 0 0}
\pscustom[linestyle=none,fillstyle=solid,fillcolor=curcolor]
{
\newpath
\moveto(468.53051282,318.79343021)
\lineto(462.28501677,316.12413993)
\lineto(462.28501677,317.27547236)
\lineto(467.23124382,319.32728826)
\lineto(462.28501677,321.35980809)
\lineto(462.28501677,322.51114052)
\lineto(468.53051282,319.87401037)
\lineto(468.53051282,318.79343021)
}
}
{
\newrgbcolor{curcolor}{0 0 0}
\pscustom[linestyle=none,fillstyle=solid,fillcolor=curcolor]
{
\newpath
\moveto(420.63252482,299.72244055)
\lineto(418.85085396,299.40083931)
\curveto(418.79081667,299.75673983)(418.65360027,300.0247406)(418.43920437,300.2048424)
\curveto(418.22908706,300.38493362)(417.95465428,300.47498188)(417.6159052,300.47498744)
\curveto(417.16566003,300.47498188)(416.80546701,300.31846943)(416.53532504,300.00544964)
\curveto(416.26946548,299.69670766)(416.1365371,299.17785818)(416.13653951,298.44889964)
\curveto(416.1365371,297.6384618)(416.27160949,297.06601216)(416.54175707,296.73154903)
\curveto(416.81618704,296.39708226)(417.18281208,296.22984978)(417.6416333,296.2298511)
\curveto(417.98467037,296.22984978)(418.26553517,296.32633006)(418.48422855,296.51929221)
\curveto(418.70291242,296.71653917)(418.85728086,297.05314813)(418.94733433,297.5291201)
\lineto(420.72257316,297.22681494)
\curveto(420.53818165,296.4120903)(420.18442064,295.79676055)(419.66128908,295.38082383)
\curveto(419.13814566,294.96488618)(418.43705566,294.75691758)(417.55801698,294.75691743)
\curveto(416.55890631,294.75691758)(415.76133603,295.07208648)(415.16530377,295.70242507)
\curveto(414.57355665,296.33276208)(414.2776838,297.20537256)(414.27768435,298.32025915)
\curveto(414.2776838,299.44800295)(414.57570065,300.32490145)(415.17173579,300.95095728)
\curveto(415.76776805,301.58128903)(416.57391435,301.89645793)(417.5901771,301.89646492)
\curveto(418.42204761,301.89645793)(419.0824015,301.71636142)(419.57124073,301.35617484)
\curveto(420.0643563,301.00026338)(420.4181173,300.45568582)(420.63252482,299.72244055)
}
}
{
\newrgbcolor{curcolor}{0 0 0}
\pscustom[linestyle=none,fillstyle=solid,fillcolor=curcolor]
{
\newpath
\moveto(421.59089636,298.42317154)
\curveto(421.59089584,299.02348974)(421.73883226,299.6045154)(422.03470607,300.16625025)
\curveto(422.33057794,300.7279746)(422.74865914,301.15677582)(423.2889509,301.45265521)
\curveto(423.83352623,301.74852151)(424.44027996,301.89645793)(425.10921391,301.89646492)
\curveto(426.14262081,301.89645793)(426.98950322,301.55984897)(427.64986369,300.88663703)
\curveto(428.31021099,300.21770115)(428.64038793,299.37081873)(428.6403955,298.34598725)
\curveto(428.64038793,297.31257287)(428.30592297,296.45497042)(427.63699964,295.77317734)
\curveto(426.97235117,295.09567055)(426.13404478,294.75691758)(425.12207796,294.75691743)
\curveto(424.49602412,294.75691758)(423.89784641,294.89842199)(423.32754305,295.18143106)
\curveto(422.76152317,295.4644396)(422.33057794,295.87823278)(422.03470607,296.42281184)
\curveto(421.73883226,296.9716759)(421.59089584,297.6384618)(421.59089636,298.42317154)
\moveto(423.4433195,298.32669117)
\curveto(423.44331712,297.64918183)(423.60411757,297.13033235)(423.92572135,296.77014118)
\curveto(424.24731941,296.4099463)(424.64396054,296.22984978)(425.11564593,296.2298511)
\curveto(425.58732323,296.22984978)(425.98182035,296.4099463)(426.29913849,296.77014118)
\curveto(426.62073417,297.13033235)(426.78153463,297.65346984)(426.78154035,298.33955522)
\curveto(426.78153463,299.0084817)(426.62073417,299.52304317)(426.29913849,299.88324117)
\curveto(425.98182035,300.24342922)(425.58732323,300.42352573)(425.11564593,300.42353125)
\curveto(424.64396054,300.42352573)(424.24731941,300.24342922)(423.92572135,299.88324117)
\curveto(423.60411757,299.52304317)(423.44331712,299.00419369)(423.4433195,298.32669117)
}
}
{
\newrgbcolor{curcolor}{0 0 0}
\pscustom[linestyle=none,fillstyle=solid,fillcolor=curcolor]
{
\newpath
\moveto(429.92680005,301.74209632)
\lineto(431.59269446,301.74209632)
\lineto(431.59269446,300.80945273)
\curveto(432.18872568,301.5341209)(432.8983917,301.89645793)(433.72169466,301.89646492)
\curveto(434.1590673,301.89645793)(434.53855638,301.80640967)(434.86016304,301.62631988)
\curveto(435.18175821,301.44621665)(435.44547096,301.17392787)(435.65130208,300.80945273)
\curveto(435.9514564,301.17392787)(436.27520133,301.44621665)(436.62253782,301.62631988)
\curveto(436.96985931,301.80640967)(437.34077236,301.89645793)(437.73527811,301.89646492)
\curveto(438.23696692,301.89645793)(438.66148013,301.79354564)(439.00881901,301.58772773)
\curveto(439.35613811,301.38618448)(439.61556285,301.08816763)(439.78709401,300.69367629)
\curveto(439.91143569,300.40208567)(439.97361187,299.93040433)(439.97362272,299.27863084)
\lineto(439.97362272,294.91128602)
\lineto(438.16622377,294.91128602)
\lineto(438.16622377,298.81552505)
\curveto(438.16621472,299.49302708)(438.10403854,299.93040433)(437.97969505,300.12765811)
\curveto(437.81245371,300.38493362)(437.55517297,300.51357399)(437.20785208,300.51357959)
\curveto(436.95485126,300.51357399)(436.71686659,300.43638977)(436.49389733,300.2820267)
\curveto(436.27091331,300.12765289)(436.11011286,299.90038824)(436.01149547,299.60023208)
\curveto(435.91286429,299.30435454)(435.86355215,298.83481721)(435.8635589,298.19161865)
\lineto(435.8635589,294.91128602)
\lineto(434.05615994,294.91128602)
\lineto(434.05615994,298.65472444)
\curveto(434.056155,299.31936259)(434.02399491,299.74816381)(433.95967957,299.94112939)
\curveto(433.89535454,300.13408491)(433.79458626,300.27773332)(433.65737441,300.37207505)
\curveto(433.52444149,300.46640586)(433.34220097,300.51357399)(433.1106523,300.51357959)
\curveto(432.83192751,300.51357399)(432.5810788,300.43853378)(432.35810541,300.28845873)
\curveto(432.13512553,300.13837292)(431.97432507,299.9218283)(431.87570355,299.63882422)
\curveto(431.78136452,299.35581069)(431.73419639,298.88627335)(431.734199,298.2302108)
\lineto(431.734199,294.91128602)
\lineto(429.92680005,294.91128602)
\lineto(429.92680005,301.74209632)
}
}
{
\newrgbcolor{curcolor}{0 0 0}
\pscustom[linestyle=none,fillstyle=solid,fillcolor=curcolor]
{
\newpath
\moveto(441.63308552,301.74209632)
\lineto(443.29897994,301.74209632)
\lineto(443.29897994,300.80945273)
\curveto(443.89501116,301.5341209)(444.60467718,301.89645793)(445.42798013,301.89646492)
\curveto(445.86535277,301.89645793)(446.24484185,301.80640967)(446.56644851,301.62631988)
\curveto(446.88804369,301.44621665)(447.15175644,301.17392787)(447.35758756,300.80945273)
\curveto(447.65774188,301.17392787)(447.9814868,301.44621665)(448.3288233,301.62631988)
\curveto(448.67614478,301.80640967)(449.04705784,301.89645793)(449.44156358,301.89646492)
\curveto(449.94325239,301.89645793)(450.3677656,301.79354564)(450.71510449,301.58772773)
\curveto(451.06242358,301.38618448)(451.32184832,301.08816763)(451.49337948,300.69367629)
\curveto(451.61772117,300.40208567)(451.67989734,299.93040433)(451.6799082,299.27863084)
\lineto(451.6799082,294.91128602)
\lineto(449.87250924,294.91128602)
\lineto(449.87250924,298.81552505)
\curveto(449.87250019,299.49302708)(449.81032402,299.93040433)(449.68598052,300.12765811)
\curveto(449.51873918,300.38493362)(449.26145845,300.51357399)(448.91413755,300.51357959)
\curveto(448.66113674,300.51357399)(448.42315206,300.43638977)(448.2001828,300.2820267)
\curveto(447.97719879,300.12765289)(447.81639833,299.90038824)(447.71778095,299.60023208)
\curveto(447.61914977,299.30435454)(447.56983763,298.83481721)(447.56984438,298.19161865)
\lineto(447.56984438,294.91128602)
\lineto(445.76244542,294.91128602)
\lineto(445.76244542,298.65472444)
\curveto(445.76244048,299.31936259)(445.73028039,299.74816381)(445.66596505,299.94112939)
\curveto(445.60164002,300.13408491)(445.50087173,300.27773332)(445.36365988,300.37207505)
\curveto(445.23072696,300.46640586)(445.04848645,300.51357399)(444.81693778,300.51357959)
\curveto(444.53821299,300.51357399)(444.28736428,300.43853378)(444.06439088,300.28845873)
\curveto(443.84141101,300.13837292)(443.68061055,299.9218283)(443.58198903,299.63882422)
\curveto(443.48765,299.35581069)(443.44048186,298.88627335)(443.44048448,298.2302108)
\lineto(443.44048448,294.91128602)
\lineto(441.63308552,294.91128602)
\lineto(441.63308552,301.74209632)
}
}
{
\newrgbcolor{curcolor}{0 0 0}
\pscustom[linestyle=none,fillstyle=solid,fillcolor=curcolor]
{
\newpath
\moveto(457.43013875,297.08531039)
\lineto(459.23110568,296.78300523)
\curveto(458.99954632,296.12264948)(458.63292127,295.61880804)(458.13122945,295.27147941)
\curveto(457.63381443,294.92843807)(457.00990865,294.75691758)(456.25951024,294.75691743)
\curveto(455.07172713,294.75691758)(454.19268462,295.14498269)(453.62238009,295.92111391)
\curveto(453.17213771,296.54287467)(452.94701707,297.32758091)(452.94701749,298.27523497)
\curveto(452.94701707,299.40726684)(453.24288991,300.29274136)(453.83463691,300.9316612)
\curveto(454.42638129,301.57485701)(455.17463942,301.89645793)(456.07941355,301.89646492)
\curveto(457.09566889,301.89645793)(457.89752718,301.55984897)(458.48499081,300.88663703)
\curveto(459.07244253,300.21770115)(459.35330733,299.19072222)(459.32758605,297.80569717)
\lineto(454.79944062,297.80569717)
\curveto(454.81230239,297.26969275)(454.9580948,296.85161155)(455.2368183,296.55145234)
\curveto(455.51553639,296.25557786)(455.86286538,296.10764143)(456.27880632,296.10764263)
\curveto(456.56181137,296.10764143)(456.79979605,296.18482565)(456.99276106,296.33919552)
\curveto(457.18571715,296.49356253)(457.33150957,296.74226724)(457.43013875,297.08531039)
\moveto(457.53305114,298.91200543)
\curveto(457.5201821,299.43513892)(457.38510972,299.83178005)(457.12783358,300.10193001)
\curveto(456.87054825,300.3763576)(456.55752336,300.51357399)(456.18875797,300.51357959)
\curveto(455.79425718,300.51357399)(455.46836826,300.36992558)(455.2110902,300.08263393)
\curveto(454.95380679,299.79533194)(454.82731043,299.40512283)(454.83160074,298.91200543)
\lineto(457.53305114,298.91200543)
}
}
{
\newrgbcolor{curcolor}{0 0 0}
\pscustom[linestyle=none,fillstyle=solid,fillcolor=curcolor]
{
\newpath
\moveto(467.02028753,294.91128602)
\lineto(465.21288857,294.91128602)
\lineto(465.21288857,298.39744345)
\curveto(465.21288322,299.13497806)(465.17429111,299.61094742)(465.09711212,299.82535294)
\curveto(465.01992267,300.04403665)(464.89342631,300.21341313)(464.71762266,300.3334829)
\curveto(464.54609732,300.45354182)(464.33812872,300.51357399)(464.09371626,300.51357959)
\curveto(463.78068714,300.51357399)(463.49982233,300.42781375)(463.25112102,300.2562986)
\curveto(463.00241292,300.08477277)(462.83089243,299.85750812)(462.73655903,299.57450398)
\curveto(462.6465079,299.29149051)(462.60148377,298.76835302)(462.60148651,298.00508993)
\lineto(462.60148651,294.91128602)
\lineto(460.79408756,294.91128602)
\lineto(460.79408756,301.74209632)
\lineto(462.47284602,301.74209632)
\lineto(462.47284602,300.73870046)
\curveto(463.06887711,301.51053683)(463.81927925,301.89645793)(464.72405469,301.89646492)
\curveto(465.12283496,301.89645793)(465.487316,301.82356172)(465.8174989,301.67777607)
\curveto(466.14766988,301.5362649)(466.39637459,301.35402438)(466.56361377,301.13105397)
\curveto(466.73512755,300.90807111)(466.85304789,300.65507839)(466.91737513,300.37207505)
\curveto(466.98597627,300.08906078)(467.02028037,299.68384363)(467.02028753,299.15642237)
\lineto(467.02028753,294.91128602)
}
}
{
\newrgbcolor{curcolor}{0 0 0}
\pscustom[linestyle=none,fillstyle=solid,fillcolor=curcolor]
{
\newpath
\moveto(471.99224226,301.74209632)
\lineto(471.99224226,300.30132277)
\lineto(470.7572935,300.30132277)
\lineto(470.7572935,297.54841618)
\curveto(470.75729066,296.99097195)(470.76801069,296.66508302)(470.78945363,296.57074841)
\curveto(470.81517883,296.4806985)(470.86877898,296.40565828)(470.95025425,296.34562755)
\curveto(471.03601145,296.28559394)(471.13892375,296.25557786)(471.25899143,296.2555792)
\curveto(471.42622057,296.25557786)(471.66849326,296.31346602)(471.98581023,296.42924387)
\lineto(472.14017883,295.02706247)
\curveto(471.7199494,294.84696584)(471.24398005,294.75691758)(470.71226933,294.75691743)
\curveto(470.3863776,294.75691758)(470.09264877,294.81051774)(469.83108194,294.91771805)
\curveto(469.56951128,295.02920636)(469.37655073,295.17071076)(469.25219971,295.34223168)
\curveto(469.13213403,295.51803975)(469.04851779,295.75388043)(469.00135074,296.04975441)
\curveto(468.96275755,296.25986587)(468.94346149,296.68437908)(468.94346252,297.32329531)
\lineto(468.94346252,300.30132277)
\lineto(468.11373133,300.30132277)
\lineto(468.11373133,301.74209632)
\lineto(468.94346252,301.74209632)
\lineto(468.94346252,303.09925355)
\lineto(470.7572935,304.15410561)
\lineto(470.7572935,301.74209632)
\lineto(471.99224226,301.74209632)
}
}
{
\newrgbcolor{curcolor}{0 0 0}
\pscustom[linestyle=none,fillstyle=solid,fillcolor=curcolor]
{
\newpath
\moveto(472.62258136,296.86018953)
\lineto(474.43641235,297.13676659)
\curveto(474.51359444,296.78514736)(474.67010689,296.5171466)(474.90595015,296.3327635)
\curveto(475.14178823,296.15266556)(475.47196517,296.06261731)(475.89648197,296.06261846)
\curveto(476.36387172,296.06261731)(476.71548872,296.14837755)(476.95133403,296.31989945)
\curveto(477.10998584,296.43996238)(477.18931407,296.60076284)(477.18931894,296.8023013)
\curveto(477.18931407,296.9395158)(477.14643395,297.05314813)(477.06067845,297.14319862)
\curveto(476.97062544,297.22895663)(476.76908887,297.30828486)(476.45606812,297.38118353)
\curveto(474.99813982,297.70278198)(474.07407319,297.99651082)(473.68386545,298.26237093)
\curveto(473.14357454,298.63113663)(472.87342977,299.14355409)(472.87343033,299.79962484)
\curveto(472.87342977,300.39136564)(473.10712643,300.88877506)(473.57452103,301.29185459)
\curveto(474.0419131,301.69492136)(474.76658716,301.89645793)(475.7485454,301.89646492)
\curveto(476.68332863,301.89645793)(477.37798661,301.7442335)(477.83252142,301.43979116)
\curveto(478.2870452,301.13533576)(478.60007009,300.68509448)(478.77159703,300.08906596)
\lineto(477.06711047,299.77389674)
\curveto(476.99420951,300.03974864)(476.85484911,300.24342922)(476.64902886,300.3849391)
\curveto(476.44748795,300.52643803)(476.15804713,300.59719023)(475.78070552,300.59719591)
\curveto(475.3047327,300.59719023)(474.96383573,300.53072604)(474.75801358,300.39780315)
\curveto(474.62079475,300.30346139)(474.55218655,300.18125304)(474.55218879,300.03117773)
\curveto(474.55218655,299.90253225)(474.61221872,299.79318794)(474.73228548,299.70314447)
\curveto(474.89522753,299.58307534)(475.45695713,299.41369886)(476.41747597,299.19501452)
\curveto(477.38227462,298.97632161)(478.05549254,298.70832085)(478.43713175,298.39101142)
\curveto(478.8144707,298.06940702)(479.00314324,297.62130975)(479.00314993,297.04671825)
\curveto(479.00314324,296.42066633)(478.74157449,295.88252079)(478.21844291,295.43228003)
\curveto(477.69529951,294.98203823)(476.9213133,294.75691758)(475.89648197,294.75691743)
\curveto(474.96597973,294.75691758)(474.22844163,294.94559012)(473.68386545,295.32293561)
\curveto(473.14357454,295.70028027)(472.78981353,296.21269773)(472.62258136,296.86018953)
}
}
{
\newrgbcolor{curcolor}{0.50196081 0.50196081 0}
\pscustom[linewidth=2.63455725,linecolor=curcolor]
{
\newpath
\moveto(405.82729,165.29658456)
\lineto(490.13312,165.29658456)
\lineto(490.13312,217.98773456)
\lineto(433.49014,217.98773456)
\lineto(433.49014,228.52596456)
\lineto(405.82729,228.52596456)
\lineto(405.82729,165.29658456)
\closepath
}
}
{
\newrgbcolor{curcolor}{0 0 0}
\pscustom[linestyle=none,fillstyle=solid,fillcolor=curcolor]
{
\newpath
\moveto(423.67226423,199.38882201)
\lineto(423.67226423,200.46940217)
\lineto(429.91776028,203.10653232)
\lineto(429.91776028,201.95519989)
\lineto(424.96510121,199.92268006)
\lineto(429.91776028,197.87086416)
\lineto(429.91776028,196.71953173)
\lineto(423.67226423,199.38882201)
}
}
{
\newrgbcolor{curcolor}{0 0 0}
\pscustom[linestyle=none,fillstyle=solid,fillcolor=curcolor]
{
\newpath
\moveto(431.36496605,199.38882201)
\lineto(431.36496605,200.46940217)
\lineto(437.61046209,203.10653232)
\lineto(437.61046209,201.95519989)
\lineto(432.65780303,199.92268006)
\lineto(437.61046209,197.87086416)
\lineto(437.61046209,196.71953173)
\lineto(431.36496605,199.38882201)
}
}
{
\newrgbcolor{curcolor}{0 0 0}
\pscustom[linestyle=none,fillstyle=solid,fillcolor=curcolor]
{
\newpath
\moveto(443.68229367,195.26589413)
\lineto(443.68229367,196.26929)
\curveto(443.15057481,195.49744679)(442.42804475,195.11152569)(441.51470133,195.11152554)
\curveto(441.111625,195.11152569)(440.73427993,195.18870991)(440.38266497,195.34307843)
\curveto(440.03533393,195.49744679)(439.77590919,195.69040734)(439.60438997,195.92196066)
\curveto(439.43715623,196.15780067)(439.31923589,196.44509749)(439.25062861,196.78385198)
\curveto(439.20345956,197.01111511)(439.1798755,197.37130813)(439.17987634,197.86443214)
\lineto(439.17987634,202.09670443)
\lineto(440.3376408,202.09670443)
\lineto(440.3376408,198.30824185)
\curveto(440.33763879,197.70362908)(440.36122286,197.29626792)(440.40839307,197.08615714)
\curveto(440.4812872,196.78170645)(440.63565564,196.54157777)(440.87149885,196.36577037)
\curveto(441.10733699,196.19424878)(441.39892182,196.10848853)(441.74625422,196.10848938)
\curveto(442.0935798,196.10848853)(442.41946873,196.19639278)(442.72392198,196.37220239)
\curveto(443.02836646,196.5522978)(443.24276707,196.79457049)(443.36712446,197.09902119)
\curveto(443.49575979,197.40775624)(443.56007998,197.85370951)(443.5600852,198.43688234)
\lineto(443.5600852,202.09670443)
\lineto(444.71784966,202.09670443)
\lineto(444.71784966,195.26589413)
\lineto(443.68229367,195.26589413)
}
}
{
\newrgbcolor{curcolor}{0 0 0}
\pscustom[linestyle=none,fillstyle=solid,fillcolor=curcolor]
{
\newpath
\moveto(449.0658973,196.30145012)
\lineto(449.23312995,195.27875818)
\curveto(448.90723745,195.21014997)(448.61565262,195.17584588)(448.35837458,195.17584579)
\curveto(447.93814669,195.17584588)(447.61225776,195.24231007)(447.38070681,195.37523855)
\curveto(447.14915244,195.50816682)(446.98620798,195.68183132)(446.89187293,195.89623256)
\curveto(446.79753544,196.11492055)(446.75036731,196.57159385)(446.75036839,197.26625383)
\lineto(446.75036839,201.19622096)
\lineto(445.90134112,201.19622096)
\lineto(445.90134112,202.09670443)
\lineto(446.75036839,202.09670443)
\lineto(446.75036839,203.78832694)
\lineto(447.90170082,204.48298562)
\lineto(447.90170082,202.09670443)
\lineto(449.0658973,202.09670443)
\lineto(449.0658973,201.19622096)
\lineto(447.90170082,201.19622096)
\lineto(447.90170082,197.20193359)
\curveto(447.90169859,196.87175471)(447.92099464,196.6594981)(447.95958904,196.56516313)
\curveto(448.00246688,196.47082557)(448.06893106,196.39578535)(448.15898181,196.34004227)
\curveto(448.25331559,196.28429704)(448.38624397,196.25642496)(448.55776735,196.25642595)
\curveto(448.68640482,196.25642496)(448.85578131,196.271433)(449.0658973,196.30145012)
}
}
{
\newrgbcolor{curcolor}{0 0 0}
\pscustom[linestyle=none,fillstyle=solid,fillcolor=curcolor]
{
\newpath
\moveto(450.19793448,203.36381331)
\lineto(450.19793448,204.69524243)
\lineto(451.35569894,204.69524243)
\lineto(451.35569894,203.36381331)
\lineto(450.19793448,203.36381331)
\moveto(450.19793448,195.26589413)
\lineto(450.19793448,202.09670443)
\lineto(451.35569894,202.09670443)
\lineto(451.35569894,195.26589413)
\lineto(450.19793448,195.26589413)
}
}
{
\newrgbcolor{curcolor}{0 0 0}
\pscustom[linestyle=none,fillstyle=solid,fillcolor=curcolor]
{
\newpath
\moveto(453.09877874,195.26589413)
\lineto(453.09877874,204.69524243)
\lineto(454.25654319,204.69524243)
\lineto(454.25654319,195.26589413)
\lineto(453.09877874,195.26589413)
}
}
{
\newrgbcolor{curcolor}{0 0 0}
\pscustom[linestyle=none,fillstyle=solid,fillcolor=curcolor]
{
\newpath
\moveto(462.15506878,199.38882201)
\lineto(455.90957274,196.71953173)
\lineto(455.90957274,197.87086416)
\lineto(460.85579978,199.92268006)
\lineto(455.90957274,201.95519989)
\lineto(455.90957274,203.10653232)
\lineto(462.15506878,200.46940217)
\lineto(462.15506878,199.38882201)
}
}
{
\newrgbcolor{curcolor}{0 0 0}
\pscustom[linestyle=none,fillstyle=solid,fillcolor=curcolor]
{
\newpath
\moveto(469.84777203,199.38882201)
\lineto(463.60227598,196.71953173)
\lineto(463.60227598,197.87086416)
\lineto(468.54850303,199.92268006)
\lineto(463.60227598,201.95519989)
\lineto(463.60227598,203.10653232)
\lineto(469.84777203,200.46940217)
\lineto(469.84777203,199.38882201)
}
}
{
\newrgbcolor{curcolor}{0 0 0}
\pscustom[linestyle=none,fillstyle=solid,fillcolor=curcolor]
{
\newpath
\moveto(428.26218228,175.50667782)
\lineto(426.45478332,175.50667782)
\lineto(426.45478332,182.33748812)
\lineto(428.13354179,182.33748812)
\lineto(428.13354179,181.36625238)
\curveto(428.42083606,181.82506383)(428.67811679,182.12736869)(428.90538476,182.27316787)
\curveto(429.1369341,182.41895352)(429.39850284,182.49184973)(429.69009178,182.49185672)
\curveto(430.10173685,182.49184973)(430.49837798,182.37821741)(430.88001636,182.1509594)
\lineto(430.32043021,180.57511334)
\curveto(430.0159766,180.77235683)(429.7329678,180.87098111)(429.47140294,180.87098648)
\curveto(429.21840633,180.87098111)(429.00400572,180.80022891)(428.82820046,180.65872966)
\curveto(428.65238872,180.52150812)(428.51302832,180.2706594)(428.41011885,179.90618276)
\curveto(428.31149175,179.54169732)(428.26217961,178.77843115)(428.26218228,177.61638195)
\lineto(428.26218228,175.50667782)
}
}
{
\newrgbcolor{curcolor}{0 0 0}
\pscustom[linestyle=none,fillstyle=solid,fillcolor=curcolor]
{
\newpath
\moveto(433.00258446,180.2535121)
\lineto(431.36241815,180.54938524)
\curveto(431.54680202,181.20973408)(431.86411492,181.69856747)(432.31435781,182.01588688)
\curveto(432.76459749,182.33319328)(433.4335274,182.49184973)(434.32114954,182.49185672)
\curveto(435.12729222,182.49184973)(435.72761393,182.39536946)(436.12211647,182.2024156)
\curveto(436.51660818,182.01373637)(436.79318497,181.77146368)(436.95184767,181.4755968)
\curveto(437.11478589,181.184006)(437.19625812,180.64586047)(437.19626461,179.86115859)
\lineto(437.17696853,177.75145447)
\curveto(437.17696206,177.15113051)(437.20483414,176.70732125)(437.26058486,176.42002534)
\curveto(437.32061047,176.13701562)(437.42995478,175.83256675)(437.58861812,175.50667782)
\lineto(435.80051524,175.50667782)
\curveto(435.75334201,175.62674217)(435.69545384,175.80469467)(435.62685057,176.04053588)
\curveto(435.59682956,176.14773565)(435.5753895,176.21848785)(435.56253032,176.2527927)
\curveto(435.25378858,175.9526311)(434.92361164,175.72751045)(434.57199851,175.5774301)
\curveto(434.22037764,175.4273496)(433.84517657,175.35230938)(433.44639417,175.35230923)
\curveto(432.74315743,175.35230938)(432.18785985,175.54312593)(431.78049976,175.92475943)
\curveto(431.37742554,176.3063921)(431.17588896,176.78879348)(431.17588943,177.37196501)
\curveto(431.17588896,177.75788424)(431.26808122,178.10092522)(431.4524665,178.40108897)
\curveto(431.63685028,178.70553494)(431.89413101,178.9370876)(432.22430947,179.09574764)
\curveto(432.5587729,179.25868852)(433.03903027,179.40019292)(433.66508301,179.52026128)
\curveto(434.50981846,179.67891371)(435.09513213,179.82685014)(435.42102577,179.96407098)
\lineto(435.42102577,180.14416768)
\curveto(435.42102106,180.49149203)(435.33526082,180.73805273)(435.16374478,180.88385053)
\curveto(434.99221984,181.03392558)(434.66847492,181.10896579)(434.19250905,181.10897139)
\curveto(433.87090464,181.10896579)(433.62005593,181.04464561)(433.43996215,180.91601065)
\curveto(433.2598629,180.79165289)(433.11407049,180.57082026)(433.00258446,180.2535121)
\moveto(435.42102577,178.78701045)
\curveto(435.1894684,178.70982295)(434.82284336,178.61763069)(434.32114954,178.51043339)
\curveto(433.8194485,178.40323008)(433.49141556,178.29817378)(433.33704975,178.19526417)
\curveto(433.10120645,178.02802901)(432.98328611,177.8157724)(432.98328839,177.55849372)
\curveto(432.98328611,177.30549895)(433.07762238,177.08681033)(433.26629748,176.9024272)
\curveto(433.45496746,176.71804128)(433.69509614,176.62584901)(433.98668425,176.62585013)
\curveto(434.3125699,176.62584901)(434.62345079,176.73304932)(434.91932784,176.94745137)
\curveto(435.13801225,177.11039439)(435.28166066,177.30978696)(435.3502735,177.54562967)
\curveto(435.39743699,177.69999607)(435.42102106,177.99372491)(435.42102577,178.42681707)
\lineto(435.42102577,178.78701045)
}
}
{
\newrgbcolor{curcolor}{0 0 0}
\pscustom[linestyle=none,fillstyle=solid,fillcolor=curcolor]
{
\newpath
\moveto(442.11676389,182.33748812)
\lineto(442.11676389,180.89671457)
\lineto(440.88181513,180.89671457)
\lineto(440.88181513,178.14380798)
\curveto(440.88181229,177.58636375)(440.89253232,177.26047482)(440.91397526,177.16614021)
\curveto(440.93970045,177.0760903)(440.99330061,177.00105008)(441.07477588,176.94101935)
\curveto(441.16053308,176.88098574)(441.26344538,176.85096966)(441.38351306,176.850971)
\curveto(441.5507422,176.85096966)(441.79301489,176.90885782)(442.11033186,177.02463567)
\lineto(442.26470046,175.62245427)
\curveto(441.84447103,175.44235764)(441.36850168,175.35230938)(440.83679096,175.35230923)
\curveto(440.51089923,175.35230938)(440.2171704,175.40590954)(439.95560357,175.51310985)
\curveto(439.6940329,175.62459816)(439.50107236,175.76610256)(439.37672134,175.93762348)
\curveto(439.25665566,176.11343155)(439.17303942,176.34927223)(439.12587237,176.64514621)
\curveto(439.08727918,176.85525767)(439.06798312,177.27977088)(439.06798415,177.91868711)
\lineto(439.06798415,180.89671457)
\lineto(438.23825296,180.89671457)
\lineto(438.23825296,182.33748812)
\lineto(439.06798415,182.33748812)
\lineto(439.06798415,183.69464535)
\lineto(440.88181513,184.74949741)
\lineto(440.88181513,182.33748812)
\lineto(442.11676389,182.33748812)
}
}
{
\newrgbcolor{curcolor}{0 0 0}
\pscustom[linestyle=none,fillstyle=solid,fillcolor=curcolor]
{
\newpath
\moveto(443.38387344,183.26369969)
\lineto(443.38387344,184.93602613)
\lineto(445.1912724,184.93602613)
\lineto(445.1912724,183.26369969)
\lineto(443.38387344,183.26369969)
\moveto(443.38387344,175.50667782)
\lineto(443.38387344,182.33748812)
\lineto(445.1912724,182.33748812)
\lineto(445.1912724,175.50667782)
\lineto(443.38387344,175.50667782)
}
}
{
\newrgbcolor{curcolor}{0 0 0}
\pscustom[linestyle=none,fillstyle=solid,fillcolor=curcolor]
{
\newpath
\moveto(453.25059834,175.50667782)
\lineto(451.44319938,175.50667782)
\lineto(451.44319938,178.99283525)
\curveto(451.44319403,179.73036986)(451.40460192,180.20633922)(451.32742294,180.42074474)
\curveto(451.25023348,180.63942845)(451.12373712,180.80880493)(450.94793348,180.9288747)
\curveto(450.77640813,181.04893362)(450.56843954,181.10896579)(450.32402707,181.10897139)
\curveto(450.01099795,181.10896579)(449.73013315,181.02320555)(449.48143183,180.8516904)
\curveto(449.23272373,180.68016457)(449.06120324,180.45289992)(448.96686985,180.16989578)
\curveto(448.87681872,179.88688231)(448.83179459,179.36374482)(448.83179733,178.60048173)
\lineto(448.83179733,175.50667782)
\lineto(447.02439837,175.50667782)
\lineto(447.02439837,182.33748812)
\lineto(448.70315683,182.33748812)
\lineto(448.70315683,181.33409226)
\curveto(449.29918792,182.10592863)(450.04959006,182.49184973)(450.9543655,182.49185672)
\curveto(451.35314577,182.49184973)(451.71762681,182.41895352)(452.04780971,182.27316787)
\curveto(452.37798069,182.1316567)(452.6266854,181.94941618)(452.79392458,181.72644577)
\curveto(452.96543837,181.50346291)(453.0833587,181.25047019)(453.14768594,180.96746685)
\curveto(453.21628708,180.68445258)(453.25059118,180.27923543)(453.25059834,179.75181417)
\lineto(453.25059834,175.50667782)
}
}
{
\newrgbcolor{curcolor}{0 0 0}
\pscustom[linestyle=none,fillstyle=solid,fillcolor=curcolor]
{
\newpath
\moveto(454.92292437,175.05643609)
\lineto(456.98760432,174.80558713)
\curveto(457.02190557,174.56545914)(457.1012338,174.40037067)(457.22558923,174.31032122)
\curveto(457.39710664,174.18168205)(457.66725141,174.11736186)(458.03602435,174.11736048)
\curveto(458.50770181,174.11736186)(458.86146281,174.18811407)(459.09730844,174.32961729)
\curveto(459.25595994,174.42395474)(459.37602428,174.57617917)(459.45750183,174.78629105)
\curveto(459.51324067,174.9363722)(459.54111275,175.21294899)(459.54111815,175.61602225)
\lineto(459.54111815,176.61298608)
\curveto(459.00082321,175.87544688)(458.31902927,175.50667782)(457.49573427,175.50667782)
\curveto(456.57809631,175.50667782)(455.85127824,175.89474293)(455.31527788,176.67087431)
\curveto(454.89505151,177.28405889)(454.68493891,178.04732506)(454.68493945,178.96067512)
\curveto(454.68493891,180.10557093)(454.95937169,180.98032542)(455.50823862,181.58494122)
\curveto(456.06139083,182.18954487)(456.74747279,182.49184973)(457.56648655,182.49185672)
\curveto(458.41122153,182.49184973)(459.10802352,182.12093667)(459.65689459,181.37911643)
\lineto(459.65689459,182.33748812)
\lineto(461.34851711,182.33748812)
\lineto(461.34851711,176.20776852)
\curveto(461.3485099,175.40162152)(461.28204571,174.79915581)(461.14912434,174.40036957)
\curveto(461.01618896,174.00158553)(460.82966042,173.68856064)(460.58953818,173.46129395)
\curveto(460.34940305,173.23403135)(460.02780214,173.05607884)(459.62473447,172.92743589)
\curveto(459.22594385,172.79879811)(458.71995841,172.73447792)(458.10677663,172.73447515)
\curveto(456.94900936,172.73447792)(456.12785502,172.93387049)(455.64331114,173.33265345)
\curveto(455.15876426,173.72715275)(454.91649157,174.22885018)(454.91649234,174.83774725)
\curveto(454.91649157,174.89778009)(454.91863558,174.9706763)(454.92292437,175.05643609)
\moveto(456.53736258,179.06358752)
\curveto(456.53736019,178.3389099)(456.67672059,177.80719638)(456.95544419,177.46844538)
\curveto(457.23845019,177.13397846)(457.58577918,176.96674599)(457.99743221,176.96674745)
\curveto(458.43909361,176.96674599)(458.81215067,177.13826647)(459.11660451,177.48130943)
\curveto(459.42104841,177.82863644)(459.57327284,178.3410539)(459.57327827,179.01856334)
\curveto(459.57327284,179.72608185)(459.42748043,180.25136335)(459.13590059,180.59440941)
\curveto(458.84431077,180.9374453)(458.47554171,181.10896579)(458.02959233,181.10897139)
\curveto(457.59649921,181.10896579)(457.23845019,180.93958931)(456.95544419,180.60084144)
\curveto(456.67672059,180.26637139)(456.53736019,179.75395393)(456.53736258,179.06358752)
}
}
{
\newrgbcolor{curcolor}{0 0 0}
\pscustom[linestyle=none,fillstyle=solid,fillcolor=curcolor]
{
\newpath
\moveto(462.50628115,177.45558133)
\lineto(464.32011214,177.73215839)
\curveto(464.39729423,177.38053916)(464.55380668,177.1125384)(464.78964994,176.9281553)
\curveto(465.02548802,176.74805736)(465.35566496,176.65800911)(465.78018176,176.65801026)
\curveto(466.24757151,176.65800911)(466.59918851,176.74376935)(466.83503382,176.91529125)
\curveto(466.99368563,177.03535418)(467.07301386,177.19615464)(467.07301873,177.3976931)
\curveto(467.07301386,177.5349076)(467.03013374,177.64853993)(466.94437824,177.73859042)
\curveto(466.85432524,177.82434843)(466.65278866,177.90367666)(466.33976791,177.97657533)
\curveto(464.88183961,178.29817378)(463.95777298,178.59190262)(463.56756524,178.85776273)
\curveto(463.02727433,179.22652843)(462.75712956,179.73894589)(462.75713012,180.39501664)
\curveto(462.75712956,180.98675744)(462.99082623,181.48416686)(463.45822082,181.88724639)
\curveto(463.92561289,182.29031316)(464.65028695,182.49184973)(465.63224519,182.49185672)
\curveto(466.56702842,182.49184973)(467.2616864,182.3396253)(467.71622121,182.03518296)
\curveto(468.17074499,181.73072756)(468.48376988,181.28048628)(468.65529683,180.68445776)
\lineto(466.95081026,180.36928854)
\curveto(466.8779093,180.63514044)(466.73854891,180.83882102)(466.53272865,180.9803309)
\curveto(466.33118774,181.12182983)(466.04174692,181.19258203)(465.66440531,181.19258771)
\curveto(465.18843249,181.19258203)(464.84753552,181.12611784)(464.64171337,180.99319495)
\curveto(464.50449454,180.89885319)(464.43588634,180.77664484)(464.43588858,180.62656953)
\curveto(464.43588634,180.49792405)(464.49591851,180.38857974)(464.61598528,180.29853627)
\curveto(464.77892732,180.17846714)(465.34065692,180.00909066)(466.30117576,179.79040632)
\curveto(467.26597441,179.57171341)(467.93919233,179.30371265)(468.32083154,178.98640322)
\curveto(468.69817049,178.66479882)(468.88684303,178.21670155)(468.88684972,177.64211005)
\curveto(468.88684303,177.01605813)(468.62527428,176.47791259)(468.1021427,176.02767183)
\curveto(467.5789993,175.57743003)(466.8050131,175.35230938)(465.78018176,175.35230923)
\curveto(464.84967952,175.35230938)(464.11214142,175.54098192)(463.56756524,175.91832741)
\curveto(463.02727433,176.29567207)(462.67351332,176.80808953)(462.50628115,177.45558133)
}
}
{
\newrgbcolor{curcolor}{0 1 0.25098041}
\pscustom[linewidth=2.63455725,linecolor=curcolor]
{
\newpath
\moveto(405.82729,404.10576456)
\lineto(490.13312,404.10576456)
\lineto(490.13312,456.79691456)
\lineto(434.80742,456.79691456)
\lineto(434.80742,467.33514456)
\lineto(405.82729,467.33514456)
\lineto(405.82729,404.10576456)
\closepath
}
}
{
\newrgbcolor{curcolor}{0 0 0}
\pscustom[linestyle=none,fillstyle=solid,fillcolor=curcolor]
{
\newpath
\moveto(421.03774111,439.78671201)
\lineto(421.03774111,440.86729217)
\lineto(427.28323715,443.50442232)
\lineto(427.28323715,442.35308989)
\lineto(422.33057809,440.32057006)
\lineto(427.28323715,438.26875416)
\lineto(427.28323715,437.11742173)
\lineto(421.03774111,439.78671201)
}
}
{
\newrgbcolor{curcolor}{0 0 0}
\pscustom[linestyle=none,fillstyle=solid,fillcolor=curcolor]
{
\newpath
\moveto(428.73044293,439.78671201)
\lineto(428.73044293,440.86729217)
\lineto(434.97593897,443.50442232)
\lineto(434.97593897,442.35308989)
\lineto(430.0232799,440.32057006)
\lineto(434.97593897,438.26875416)
\lineto(434.97593897,437.11742173)
\lineto(428.73044293,439.78671201)
}
}
{
\newrgbcolor{curcolor}{0 0 0}
\pscustom[linestyle=none,fillstyle=solid,fillcolor=curcolor]
{
\newpath
\moveto(441.02847447,436.50637938)
\curveto(440.59966793,436.1418975)(440.18587475,435.88461676)(439.78709369,435.73453641)
\curveto(439.39259249,435.58445591)(438.96807928,435.50941569)(438.51355279,435.50941554)
\curveto(437.76314784,435.50941569)(437.1864102,435.69165621)(436.78333813,436.05613764)
\curveto(436.3802639,436.4249063)(436.17872733,436.89444364)(436.1787278,437.46475107)
\curveto(436.17872733,437.79921422)(436.25376754,438.10366309)(436.40384867,438.37809858)
\curveto(436.55821641,438.65681666)(436.75760898,438.8797933)(437.00202697,439.04702916)
\curveto(437.25073038,439.21425825)(437.52945118,439.34075461)(437.83819019,439.42651862)
\curveto(438.0654527,439.48654703)(438.40849368,439.54443519)(438.86731415,439.60018329)
\curveto(439.80209765,439.71166767)(440.49032361,439.84459605)(440.9319941,439.99896882)
\curveto(440.93627688,440.15762094)(440.93842089,440.25838923)(440.93842613,440.30127399)
\curveto(440.93842089,440.77295069)(440.82907658,441.10527164)(440.61039286,441.29823782)
\curveto(440.31451511,441.55980094)(439.87499386,441.69058531)(439.29182779,441.69059133)
\curveto(438.74724665,441.69058531)(438.3441735,441.59410503)(438.08260713,441.40115022)
\curveto(437.82532402,441.21247195)(437.63450748,440.87586299)(437.51015693,440.39132233)
\lineto(436.37812057,440.54569093)
\curveto(436.48103219,441.03023143)(436.65040867,441.42044054)(436.88625053,441.71631943)
\curveto(437.12209002,442.01647424)(437.46298699,442.24588289)(437.90894246,442.40454608)
\curveto(438.35489353,442.56748381)(438.871599,442.64895604)(439.45906043,442.64896302)
\curveto(440.04222634,442.64895604)(440.51605169,442.58034784)(440.8805379,442.44313823)
\curveto(441.24501376,442.30591506)(441.51301453,442.13225057)(441.684541,441.92214423)
\curveto(441.85605551,441.71631338)(441.97611985,441.45474464)(442.04473439,441.13743721)
\curveto(442.08332015,440.94018317)(442.10261621,440.58427816)(442.10262261,440.06972109)
\lineto(442.10262261,438.52603515)
\curveto(442.10261621,437.44974122)(442.12620028,436.76794728)(442.17337488,436.48065128)
\curveto(442.22482456,436.19764165)(442.32344884,435.92535288)(442.46924802,435.66378413)
\lineto(441.26002736,435.66378413)
\curveto(441.13995747,435.90391282)(441.06277325,436.18477762)(441.02847447,436.50637938)
\moveto(440.9319941,439.09205333)
\curveto(440.51176368,438.92052941)(439.88142588,438.774737)(439.04097882,438.65467565)
\curveto(438.56500613,438.58606446)(438.22839717,438.50888024)(438.03115093,438.42312276)
\curveto(437.83390004,438.33735975)(437.68167561,438.21086339)(437.57447718,438.04363329)
\curveto(437.467275,437.88068645)(437.41367485,437.69844593)(437.41367656,437.49691119)
\curveto(437.41367485,437.18817248)(437.52945118,436.93089174)(437.76100589,436.72506822)
\curveto(437.99684451,436.51924257)(438.33988549,436.41633028)(438.79012986,436.41633103)
\curveto(439.23608004,436.41633028)(439.63272117,436.51281055)(439.98005444,436.70577214)
\curveto(440.32737915,436.90301966)(440.58251588,437.17102043)(440.74546538,437.50977524)
\curveto(440.8698127,437.77134214)(440.93198887,438.15726324)(440.9319941,438.6675397)
\lineto(440.9319941,439.09205333)
}
}
{
\newrgbcolor{curcolor}{0 0 0}
\pscustom[linestyle=none,fillstyle=solid,fillcolor=curcolor]
{
\newpath
\moveto(443.90358845,433.04595005)
\lineto(443.90358845,442.49459443)
\lineto(444.95844051,442.49459443)
\lineto(444.95844051,441.60697501)
\curveto(445.20714329,441.95429806)(445.48800809,442.2137228)(445.80103575,442.38525001)
\curveto(446.11405788,442.56105179)(446.49354696,442.64895604)(446.93950413,442.64896302)
\curveto(447.52266989,442.64895604)(448.03723136,442.49887561)(448.48319008,442.19872129)
\curveto(448.9291379,441.8985539)(449.26574686,441.47404069)(449.49301796,440.92518039)
\curveto(449.72027615,440.38059757)(449.83390848,439.78241987)(449.83391528,439.13064548)
\curveto(449.83390848,438.43169602)(449.70741212,437.80135822)(449.45442582,437.2396302)
\curveto(449.20571469,436.68218704)(448.84123365,436.25338581)(448.36098161,435.95322525)
\curveto(447.88500692,435.65735211)(447.38330949,435.50941569)(446.85588781,435.50941554)
\curveto(446.46996289,435.50941569)(446.1226339,435.59088793)(445.8138998,435.75383248)
\curveto(445.50944815,435.91677685)(445.25859944,436.12260144)(445.0613529,436.37130686)
\lineto(445.0613529,433.04595005)
\lineto(443.90358845,433.04595005)
\moveto(444.95200848,439.04059713)
\curveto(444.95200657,438.16155125)(445.12995907,437.5119174)(445.48586654,437.09169363)
\curveto(445.8417691,436.671467)(446.27271433,436.46135441)(446.77870352,436.4613552)
\curveto(447.29326124,436.46135441)(447.73278249,436.67789902)(448.09726859,437.1109897)
\curveto(448.46603258,437.5483655)(448.65041711,438.22372743)(448.65042272,439.1370775)
\curveto(448.65041711,440.00754051)(448.47032059,440.65931837)(448.11013264,441.09241303)
\curveto(447.75422255,441.52549684)(447.32756534,441.74204145)(446.83015971,441.74204753)
\curveto(446.33703451,441.74204145)(445.89965727,441.51048879)(445.51802666,441.04738886)
\curveto(445.1406791,440.58856617)(444.95200657,439.91963626)(444.95200848,439.04059713)
}
}
{
\newrgbcolor{curcolor}{0 0 0}
\pscustom[linestyle=none,fillstyle=solid,fillcolor=curcolor]
{
\newpath
\moveto(451.23609653,433.04595005)
\lineto(451.23609653,442.49459443)
\lineto(452.29094859,442.49459443)
\lineto(452.29094859,441.60697501)
\curveto(452.53965138,441.95429806)(452.82051618,442.2137228)(453.13354384,442.38525001)
\curveto(453.44656596,442.56105179)(453.82605505,442.64895604)(454.27201222,442.64896302)
\curveto(454.85517798,442.64895604)(455.36973945,442.49887561)(455.81569816,442.19872129)
\curveto(456.26164599,441.8985539)(456.59825495,441.47404069)(456.82552605,440.92518039)
\curveto(457.05278424,440.38059757)(457.16641657,439.78241987)(457.16642336,439.13064548)
\curveto(457.16641657,438.43169602)(457.03992021,437.80135822)(456.7869339,437.2396302)
\curveto(456.53822278,436.68218704)(456.17374174,436.25338581)(455.69348969,435.95322525)
\curveto(455.21751501,435.65735211)(454.71581758,435.50941569)(454.1883959,435.50941554)
\curveto(453.80247098,435.50941569)(453.45514199,435.59088793)(453.14640789,435.75383248)
\curveto(452.84195624,435.91677685)(452.59110753,436.12260144)(452.39386099,436.37130686)
\lineto(452.39386099,433.04595005)
\lineto(451.23609653,433.04595005)
\moveto(452.28451657,439.04059713)
\curveto(452.28451465,438.16155125)(452.46246716,437.5119174)(452.81837463,437.09169363)
\curveto(453.17427719,436.671467)(453.60522242,436.46135441)(454.1112116,436.4613552)
\curveto(454.62576933,436.46135441)(455.06529058,436.67789902)(455.42977668,437.1109897)
\curveto(455.79854067,437.5483655)(455.98292519,438.22372743)(455.98293081,439.1370775)
\curveto(455.98292519,440.00754051)(455.80282868,440.65931837)(455.44264073,441.09241303)
\curveto(455.08673064,441.52549684)(454.66007342,441.74204145)(454.1626678,441.74204753)
\curveto(453.6695426,441.74204145)(453.23216535,441.51048879)(452.85053475,441.04738886)
\curveto(452.47318719,440.58856617)(452.28451465,439.91963626)(452.28451657,439.04059713)
}
}
{
\newrgbcolor{curcolor}{0 0 0}
\pscustom[linestyle=none,fillstyle=solid,fillcolor=curcolor]
{
\newpath
\moveto(464.6661641,439.78671201)
\lineto(458.42066805,437.11742173)
\lineto(458.42066805,438.26875416)
\lineto(463.36689509,440.32057006)
\lineto(458.42066805,442.35308989)
\lineto(458.42066805,443.50442232)
\lineto(464.6661641,440.86729217)
\lineto(464.6661641,439.78671201)
}
}
{
\newrgbcolor{curcolor}{0 0 0}
\pscustom[linestyle=none,fillstyle=solid,fillcolor=curcolor]
{
\newpath
\moveto(472.35886734,439.78671201)
\lineto(466.1133713,437.11742173)
\lineto(466.1133713,438.26875416)
\lineto(471.05959834,440.32057006)
\lineto(466.1133713,442.35308989)
\lineto(466.1133713,443.50442232)
\lineto(472.35886734,440.86729217)
\lineto(472.35886734,439.78671201)
}
}
{
\newrgbcolor{curcolor}{0 0 0}
\pscustom[linestyle=none,fillstyle=solid,fillcolor=curcolor]
{
\newpath
\moveto(438.88531694,422.73537812)
\lineto(438.88531694,421.29460457)
\lineto(437.65036818,421.29460457)
\lineto(437.65036818,418.54169798)
\curveto(437.65036534,417.98425375)(437.66108537,417.65836482)(437.6825283,417.56403021)
\curveto(437.7082535,417.4739803)(437.76185366,417.39894008)(437.84332892,417.33890935)
\curveto(437.92908613,417.27887574)(438.03199843,417.24885966)(438.15206611,417.248861)
\curveto(438.31929524,417.24885966)(438.56156793,417.30674782)(438.87888491,417.42252567)
\lineto(439.0332535,416.02034427)
\curveto(438.61302408,415.84024764)(438.13705472,415.75019938)(437.60534401,415.75019923)
\curveto(437.27945228,415.75019938)(436.98572344,415.80379954)(436.72415662,415.91099985)
\curveto(436.46258595,416.02248816)(436.2696254,416.16399256)(436.14527439,416.33551348)
\curveto(436.02520871,416.51132155)(435.94159247,416.74716223)(435.89442542,417.04303621)
\curveto(435.85583222,417.25314767)(435.83653617,417.67766088)(435.8365372,418.31657711)
\lineto(435.8365372,421.29460457)
\lineto(435.006806,421.29460457)
\lineto(435.006806,422.73537812)
\lineto(435.8365372,422.73537812)
\lineto(435.8365372,424.09253535)
\lineto(437.65036818,425.14738741)
\lineto(437.65036818,422.73537812)
\lineto(438.88531694,422.73537812)
}
}
{
\newrgbcolor{curcolor}{0 0 0}
\pscustom[linestyle=none,fillstyle=solid,fillcolor=curcolor]
{
\newpath
\moveto(441.50315121,420.6514021)
\lineto(439.8629849,420.94727524)
\curveto(440.04736877,421.60762408)(440.36468167,422.09645747)(440.81492456,422.41377688)
\curveto(441.26516424,422.73108328)(441.93409415,422.88973973)(442.82171629,422.88974672)
\curveto(443.62785897,422.88973973)(444.22818068,422.79325946)(444.62268322,422.6003056)
\curveto(445.01717493,422.41162637)(445.29375172,422.16935368)(445.45241442,421.8734868)
\curveto(445.61535264,421.581896)(445.69682487,421.04375047)(445.69683136,420.25904859)
\lineto(445.67753529,418.14934447)
\curveto(445.67752881,417.54902051)(445.70540089,417.10521125)(445.76115161,416.81791534)
\curveto(445.82117722,416.53490562)(445.93052154,416.23045675)(446.08918487,415.90456782)
\lineto(444.30108199,415.90456782)
\curveto(444.25390876,416.02463217)(444.19602059,416.20258467)(444.12741732,416.43842588)
\curveto(444.09739631,416.54562565)(444.07595625,416.61637785)(444.06309707,416.6506827)
\curveto(443.75435533,416.3505211)(443.42417839,416.12540045)(443.07256526,415.9753201)
\curveto(442.72094439,415.8252396)(442.34574332,415.75019938)(441.94696092,415.75019923)
\curveto(441.24372418,415.75019938)(440.6884266,415.94101593)(440.28106651,416.32264943)
\curveto(439.87799229,416.7042821)(439.67645571,417.18668348)(439.67645618,417.76985501)
\curveto(439.67645571,418.15577424)(439.76864798,418.49881522)(439.95303325,418.79897897)
\curveto(440.13741703,419.10342494)(440.39469776,419.3349776)(440.72487622,419.49363764)
\curveto(441.05933965,419.65657852)(441.53959702,419.79808292)(442.16564977,419.91815128)
\curveto(443.01038521,420.07680371)(443.59569888,420.22474014)(443.92159253,420.36196098)
\lineto(443.92159253,420.54205768)
\curveto(443.92158781,420.88938203)(443.83582757,421.13594273)(443.66431153,421.28174053)
\curveto(443.49278659,421.43181558)(443.16904167,421.50685579)(442.6930758,421.50686139)
\curveto(442.37147139,421.50685579)(442.12062268,421.44253561)(441.9405289,421.31390065)
\curveto(441.76042965,421.18954289)(441.61463724,420.96871026)(441.50315121,420.6514021)
\moveto(443.92159253,419.18490045)
\curveto(443.69003515,419.10771295)(443.32341011,419.01552069)(442.82171629,418.90832339)
\curveto(442.32001525,418.80112008)(441.99198231,418.69606378)(441.8376165,418.59315417)
\curveto(441.6017732,418.42591901)(441.48385286,418.2136624)(441.48385514,417.95638372)
\curveto(441.48385286,417.70338895)(441.57818913,417.48470033)(441.76686423,417.3003172)
\curveto(441.95553421,417.11593128)(442.19566289,417.02373901)(442.487251,417.02374013)
\curveto(442.81313665,417.02373901)(443.12401754,417.13093932)(443.41989459,417.34534137)
\curveto(443.638579,417.50828439)(443.78222741,417.70767696)(443.85084025,417.94351967)
\curveto(443.89800374,418.09788607)(443.92158781,418.39161491)(443.92159253,418.82470707)
\lineto(443.92159253,419.18490045)
}
}
{
\newrgbcolor{curcolor}{0 0 0}
\pscustom[linestyle=none,fillstyle=solid,fillcolor=curcolor]
{
\newpath
\moveto(447.31770193,415.45432609)
\lineto(449.38238188,415.20347713)
\curveto(449.41668314,414.96334914)(449.49601136,414.79826067)(449.6203668,414.70821122)
\curveto(449.79188421,414.57957205)(450.06202898,414.51525186)(450.43080192,414.51525048)
\curveto(450.90247937,414.51525186)(451.25624038,414.58600407)(451.49208601,414.72750729)
\curveto(451.6507375,414.82184474)(451.77080185,414.97406917)(451.85227939,415.18418105)
\curveto(451.90801824,415.3342622)(451.93589032,415.61083899)(451.93589571,416.01391225)
\lineto(451.93589571,417.01087608)
\curveto(451.39560078,416.27333688)(450.71380683,415.90456782)(449.89051184,415.90456782)
\curveto(448.97287387,415.90456782)(448.2460558,416.29263293)(447.71005545,417.06876431)
\curveto(447.28982908,417.68194889)(447.07971648,418.44521506)(447.07971702,419.35856512)
\curveto(447.07971648,420.50346093)(447.35414926,421.37821542)(447.90301619,421.98283122)
\curveto(448.4561684,422.58743487)(449.14225036,422.88973973)(449.96126411,422.88974672)
\curveto(450.8059991,422.88973973)(451.50280108,422.51882667)(452.05167216,421.77700643)
\lineto(452.05167216,422.73537812)
\lineto(453.74329467,422.73537812)
\lineto(453.74329467,416.60565852)
\curveto(453.74328747,415.79951152)(453.67682328,415.19704581)(453.5439019,414.79825957)
\curveto(453.41096652,414.39947553)(453.22443799,414.08645064)(452.98431575,413.85918395)
\curveto(452.74418062,413.63192135)(452.4225797,413.45396884)(452.01951204,413.32532589)
\curveto(451.62072142,413.19668811)(451.11473598,413.13236792)(450.50155419,413.13236515)
\curveto(449.34378693,413.13236792)(448.52263259,413.33176049)(448.03808871,413.73054345)
\curveto(447.55354183,414.12504275)(447.31126914,414.62674018)(447.31126991,415.23563725)
\curveto(447.31126914,415.29567009)(447.31341314,415.3685663)(447.31770193,415.45432609)
\moveto(448.93214015,419.46147752)
\curveto(448.93213776,418.7367999)(449.07149815,418.20508638)(449.35022176,417.86633538)
\curveto(449.63322776,417.53186846)(449.98055675,417.36463599)(450.39220977,417.36463745)
\curveto(450.83387118,417.36463599)(451.20692824,417.53615647)(451.51138208,417.87919943)
\curveto(451.81582598,418.22652644)(451.96805041,418.7389439)(451.96805584,419.41645334)
\curveto(451.96805041,420.12397185)(451.82225799,420.64925335)(451.53067815,420.99229941)
\curveto(451.23908833,421.3353353)(450.87031928,421.50685579)(450.42436989,421.50686139)
\curveto(449.99127678,421.50685579)(449.63322776,421.33747931)(449.35022176,420.99873144)
\curveto(449.07149815,420.66426139)(448.93213776,420.15184393)(448.93214015,419.46147752)
}
}
{
\newrgbcolor{curcolor}{0 0 0}
\pscustom[linestyle=none,fillstyle=solid,fillcolor=curcolor]
{
\newpath
\moveto(454.90105872,417.85347133)
\lineto(456.7148897,418.13004839)
\curveto(456.7920718,417.77842916)(456.94858425,417.5104284)(457.18442751,417.3260453)
\curveto(457.42026559,417.14594736)(457.75044253,417.05589911)(458.17495932,417.05590026)
\curveto(458.64234907,417.05589911)(458.99396607,417.14165935)(459.22981138,417.31318125)
\curveto(459.3884632,417.43324418)(459.46779142,417.59404464)(459.4677963,417.7955831)
\curveto(459.46779142,417.9327976)(459.4249113,418.04642993)(459.33915581,418.13648042)
\curveto(459.2491028,418.22223843)(459.04756623,418.30156666)(458.73454548,418.37446533)
\curveto(457.27661718,418.69606378)(456.35255055,418.98979262)(455.96234281,419.25565273)
\curveto(455.4220519,419.62441843)(455.15190713,420.13683589)(455.15190769,420.79290664)
\curveto(455.15190713,421.38464744)(455.38560379,421.88205686)(455.85299838,422.28513639)
\curveto(456.32039046,422.68820316)(457.04506452,422.88973973)(458.02702275,422.88974672)
\curveto(458.96180598,422.88973973)(459.65646396,422.7375153)(460.11099878,422.43307296)
\curveto(460.56552255,422.12861756)(460.87854745,421.67837628)(461.05007439,421.08234776)
\lineto(459.34558783,420.76717854)
\curveto(459.27268687,421.03303044)(459.13332647,421.23671102)(458.92750622,421.3782209)
\curveto(458.72596531,421.51971983)(458.43652449,421.59047203)(458.05918288,421.59047771)
\curveto(457.58321005,421.59047203)(457.24231308,421.52400784)(457.03649094,421.39108495)
\curveto(456.89927211,421.29674319)(456.83066391,421.17453484)(456.83066615,421.02445953)
\curveto(456.83066391,420.89581405)(456.89069608,420.78646974)(457.01076284,420.69642627)
\curveto(457.17370489,420.57635714)(457.73543449,420.40698066)(458.69595333,420.18829632)
\curveto(459.66075197,419.96960341)(460.33396989,419.70160265)(460.7156091,419.38429322)
\curveto(461.09294806,419.06268882)(461.28162059,418.61459155)(461.28162728,418.04000005)
\curveto(461.28162059,417.41394813)(461.02005185,416.87580259)(460.49692026,416.42556183)
\curveto(459.97377687,415.97532003)(459.19979066,415.75019938)(458.17495932,415.75019923)
\curveto(457.24445709,415.75019938)(456.50691899,415.93887192)(455.96234281,416.31621741)
\curveto(455.4220519,416.69356207)(455.06829089,417.20597953)(454.90105872,417.85347133)
}
}
{
\newrgbcolor{curcolor}{0 1 0.25098041}
\pscustom[linewidth=2.63455725,linecolor=curcolor]
{
\newpath
\moveto(527.33596,523.51032456)
\lineto(620.86274,523.51032456)
\lineto(620.86274,576.20147456)
\lineto(558.95065,576.20147456)
\lineto(558.95065,586.73970456)
\lineto(527.33596,586.73970456)
\lineto(527.33596,523.51032456)
\closepath
}
}
{
\newrgbcolor{curcolor}{0 0 0}
\pscustom[linestyle=none,fillstyle=solid,fillcolor=curcolor]
{
\newpath
\moveto(547.81558052,559.19127622)
\lineto(547.81558052,560.27185638)
\lineto(554.06107657,562.90898654)
\lineto(554.06107657,561.7576541)
\lineto(549.1084175,559.72513428)
\lineto(554.06107657,557.67331838)
\lineto(554.06107657,556.52198595)
\lineto(547.81558052,559.19127622)
}
}
{
\newrgbcolor{curcolor}{0 0 0}
\pscustom[linestyle=none,fillstyle=solid,fillcolor=curcolor]
{
\newpath
\moveto(555.50828234,559.19127622)
\lineto(555.50828234,560.27185638)
\lineto(561.75377838,562.90898654)
\lineto(561.75377838,561.7576541)
\lineto(556.80111932,559.72513428)
\lineto(561.75377838,557.67331838)
\lineto(561.75377838,556.52198595)
\lineto(555.50828234,559.19127622)
}
}
{
\newrgbcolor{curcolor}{0 0 0}
\pscustom[linestyle=none,fillstyle=solid,fillcolor=curcolor]
{
\newpath
\moveto(567.80631389,555.9109436)
\curveto(567.37750734,555.54646171)(566.96371416,555.28918098)(566.56493311,555.13910062)
\curveto(566.1704319,554.98902013)(565.74591869,554.91397991)(565.2913922,554.91397976)
\curveto(564.54098725,554.91397991)(563.96424961,555.09622043)(563.56117754,555.46070186)
\curveto(563.15810331,555.82947052)(562.95656674,556.29900786)(562.95656722,556.86931528)
\curveto(562.95656674,557.20377844)(563.03160695,557.5082273)(563.18168808,557.7826628)
\curveto(563.33605582,558.06138088)(563.53544839,558.28435752)(563.77986639,558.45159338)
\curveto(564.02856979,558.61882247)(564.30729059,558.74531883)(564.6160296,558.83108284)
\curveto(564.84329212,558.89111125)(565.18633309,558.94899941)(565.64515357,559.00474751)
\curveto(566.57993707,559.11623189)(567.26816303,559.24916027)(567.70983351,559.40353304)
\curveto(567.7141163,559.56218516)(567.7162603,559.66295345)(567.71626554,559.7058382)
\curveto(567.7162603,560.17751491)(567.60691599,560.50983586)(567.38823228,560.70280204)
\curveto(567.09235453,560.96436515)(566.65283327,561.09514953)(566.0696672,561.09515555)
\curveto(565.52508606,561.09514953)(565.12201291,560.99866925)(564.86044655,560.80571444)
\curveto(564.60316343,560.61703616)(564.41234689,560.2804272)(564.28799634,559.79588655)
\lineto(563.15595998,559.95025515)
\curveto(563.2588716,560.43479564)(563.42824808,560.82500476)(563.66408994,561.12088365)
\curveto(563.89992943,561.42103846)(564.2408264,561.65044711)(564.68678188,561.8091103)
\curveto(565.13273294,561.97204803)(565.64943841,562.05352026)(566.23689984,562.05352724)
\curveto(566.82006575,562.05352026)(567.2938911,561.98491206)(567.65837732,561.84770245)
\curveto(568.02285318,561.71047928)(568.29085394,561.53681479)(568.46238041,561.32670845)
\curveto(568.63389492,561.1208776)(568.75395926,560.85930885)(568.8225738,560.54200142)
\curveto(568.86115957,560.34474739)(568.88045562,559.98884237)(568.88046202,559.47428531)
\lineto(568.88046202,557.93059937)
\curveto(568.88045562,556.85430544)(568.90403969,556.1725115)(568.95121429,555.8852155)
\curveto(569.00266397,555.60220587)(569.10128825,555.3299171)(569.24708743,555.06834835)
\lineto(568.03786678,555.06834835)
\curveto(567.91779688,555.30847704)(567.84061266,555.58934184)(567.80631389,555.9109436)
\moveto(567.70983351,558.49661755)
\curveto(567.28960309,558.32509363)(566.65926529,558.17930122)(565.81881823,558.05923987)
\curveto(565.34284554,557.99062868)(565.00623658,557.91344446)(564.80899035,557.82768697)
\curveto(564.61173946,557.74192397)(564.45951502,557.61542761)(564.35231659,557.44819751)
\curveto(564.24511441,557.28525067)(564.19151426,557.10301015)(564.19151597,556.90147541)
\curveto(564.19151426,556.5927367)(564.30729059,556.33545596)(564.53884531,556.12963244)
\curveto(564.77468392,555.92380679)(565.1177249,555.8208945)(565.56796927,555.82089525)
\curveto(566.01391945,555.8208945)(566.41056058,555.91737477)(566.75789385,556.11033636)
\curveto(567.10521856,556.30758388)(567.36035529,556.57558465)(567.5233048,556.91433946)
\curveto(567.64765211,557.17590636)(567.70982829,557.56182746)(567.70983351,558.07210392)
\lineto(567.70983351,558.49661755)
}
}
{
\newrgbcolor{curcolor}{0 0 0}
\pscustom[linestyle=none,fillstyle=solid,fillcolor=curcolor]
{
\newpath
\moveto(570.68142786,552.45051427)
\lineto(570.68142786,561.89915865)
\lineto(571.73627992,561.89915865)
\lineto(571.73627992,561.01153923)
\curveto(571.98498271,561.35886228)(572.26584751,561.61828702)(572.57887516,561.78981423)
\curveto(572.89189729,561.96561601)(573.27138637,562.05352026)(573.71734355,562.05352724)
\curveto(574.3005093,562.05352026)(574.81507077,561.90343983)(575.26102949,561.60328551)
\curveto(575.70697731,561.30311812)(576.04358627,560.87860491)(576.27085738,560.32974461)
\curveto(576.49811557,559.78516179)(576.61174789,559.18698409)(576.61175469,558.5352097)
\curveto(576.61174789,557.83626024)(576.48525153,557.20592244)(576.23226523,556.64419442)
\curveto(575.9835541,556.08675125)(575.61907306,555.65795003)(575.13882102,555.35778947)
\curveto(574.66284634,555.06191633)(574.16114891,554.91397991)(573.63372723,554.91397976)
\curveto(573.2478023,554.91397991)(572.90047332,554.99545214)(572.59173921,555.1583967)
\curveto(572.28728757,555.32134107)(572.03643885,555.52716566)(571.83919232,555.77587108)
\lineto(571.83919232,552.45051427)
\lineto(570.68142786,552.45051427)
\moveto(571.7298479,558.44516135)
\curveto(571.72984598,557.56611547)(571.90779849,556.91648162)(572.26370595,556.49625785)
\curveto(572.61960851,556.07603122)(573.05055374,555.86591862)(573.55654293,555.86591942)
\curveto(574.07110065,555.86591862)(574.5106219,556.08246324)(574.875108,556.51555392)
\curveto(575.24387199,556.95292972)(575.42825652,557.62829165)(575.42826213,558.54164172)
\curveto(575.42825652,559.41210473)(575.24816001,560.06388259)(574.88797205,560.49697725)
\curveto(574.53206196,560.93006106)(574.10540475,561.14660567)(573.60799913,561.14661175)
\curveto(573.11487393,561.14660567)(572.67749668,560.91505301)(572.29586607,560.45195308)
\curveto(571.91851852,559.99313039)(571.72984598,559.32420048)(571.7298479,558.44516135)
}
}
{
\newrgbcolor{curcolor}{0 0 0}
\pscustom[linestyle=none,fillstyle=solid,fillcolor=curcolor]
{
\newpath
\moveto(578.01393595,552.45051427)
\lineto(578.01393595,561.89915865)
\lineto(579.06878801,561.89915865)
\lineto(579.06878801,561.01153923)
\curveto(579.31749079,561.35886228)(579.59835559,561.61828702)(579.91138325,561.78981423)
\curveto(580.22440538,561.96561601)(580.60389446,562.05352026)(581.04985163,562.05352724)
\curveto(581.63301739,562.05352026)(582.14757886,561.90343983)(582.59353758,561.60328551)
\curveto(583.0394854,561.30311812)(583.37609436,560.87860491)(583.60336546,560.32974461)
\curveto(583.83062365,559.78516179)(583.94425598,559.18698409)(583.94426278,558.5352097)
\curveto(583.94425598,557.83626024)(583.81775962,557.20592244)(583.56477332,556.64419442)
\curveto(583.31606219,556.08675125)(582.95158115,555.65795003)(582.47132911,555.35778947)
\curveto(581.99535442,555.06191633)(581.49365699,554.91397991)(580.96623531,554.91397976)
\curveto(580.58031039,554.91397991)(580.2329814,554.99545214)(579.9242473,555.1583967)
\curveto(579.61979565,555.32134107)(579.36894694,555.52716566)(579.1717004,555.77587108)
\lineto(579.1717004,552.45051427)
\lineto(578.01393595,552.45051427)
\moveto(579.06235598,558.44516135)
\curveto(579.06235407,557.56611547)(579.24030657,556.91648162)(579.59621404,556.49625785)
\curveto(579.9521166,556.07603122)(580.38306183,555.86591862)(580.88905102,555.86591942)
\curveto(581.40360874,555.86591862)(581.84312999,556.08246324)(582.20761609,556.51555392)
\curveto(582.57638008,556.95292972)(582.76076461,557.62829165)(582.76077022,558.54164172)
\curveto(582.76076461,559.41210473)(582.58066809,560.06388259)(582.22048014,560.49697725)
\curveto(581.86457005,560.93006106)(581.43791284,561.14660567)(580.94050721,561.14661175)
\curveto(580.44738201,561.14660567)(580.01000477,560.91505301)(579.62837416,560.45195308)
\curveto(579.2510266,559.99313039)(579.06235407,559.32420048)(579.06235598,558.44516135)
}
}
{
\newrgbcolor{curcolor}{0 0 0}
\pscustom[linestyle=none,fillstyle=solid,fillcolor=curcolor]
{
\newpath
\moveto(591.44400351,559.19127622)
\lineto(585.19850746,556.52198595)
\lineto(585.19850746,557.67331838)
\lineto(590.14473451,559.72513428)
\lineto(585.19850746,561.7576541)
\lineto(585.19850746,562.90898654)
\lineto(591.44400351,560.27185638)
\lineto(591.44400351,559.19127622)
}
}
{
\newrgbcolor{curcolor}{0 0 0}
\pscustom[linestyle=none,fillstyle=solid,fillcolor=curcolor]
{
\newpath
\moveto(599.13670676,559.19127622)
\lineto(592.89121071,556.52198595)
\lineto(592.89121071,557.67331838)
\lineto(597.83743775,559.72513428)
\lineto(592.89121071,561.7576541)
\lineto(592.89121071,562.90898654)
\lineto(599.13670676,560.27185638)
\lineto(599.13670676,559.19127622)
}
}
{
\newrgbcolor{curcolor}{0 0 0}
\pscustom[linestyle=none,fillstyle=solid,fillcolor=curcolor]
{
\newpath
\moveto(540.82390476,540.12028656)
\lineto(539.0422339,539.79868533)
\curveto(538.98219661,540.15458585)(538.84498022,540.42258662)(538.63058431,540.60268842)
\curveto(538.42046701,540.78277964)(538.14603422,540.8728279)(537.80728514,540.87283346)
\curveto(537.35703998,540.8728279)(536.99684695,540.71631545)(536.72670498,540.40329565)
\curveto(536.46084542,540.09455368)(536.32791704,539.5757042)(536.32791945,538.84674566)
\curveto(536.32791704,538.03630781)(536.46298943,537.46385818)(536.73313701,537.12939505)
\curveto(537.00756698,536.79492828)(537.37419202,536.6276958)(537.83301324,536.62769712)
\curveto(538.17605031,536.6276958)(538.45691511,536.72417608)(538.67560849,536.91713823)
\curveto(538.89429236,537.11438519)(539.0486608,537.45099415)(539.13871427,537.92696612)
\lineto(540.9139531,537.62466096)
\curveto(540.72956159,536.80993632)(540.37580058,536.19460657)(539.85266902,535.77866985)
\curveto(539.3295256,535.3627322)(538.6284356,535.1547636)(537.74939692,535.15476345)
\curveto(536.75028625,535.1547636)(535.95271597,535.4699325)(535.35668371,536.10027109)
\curveto(534.76493659,536.73060809)(534.46906375,537.60321858)(534.46906429,538.71810517)
\curveto(534.46906375,539.84584897)(534.76708059,540.72274747)(535.36311573,541.34880329)
\curveto(535.95914799,541.97913505)(536.76529429,542.29430395)(537.78155704,542.29431093)
\curveto(538.61342756,542.29430395)(539.27378144,542.11420744)(539.76262067,541.75402085)
\curveto(540.25573624,541.3981094)(540.60949724,540.85353184)(540.82390476,540.12028656)
}
}
{
\newrgbcolor{curcolor}{0 0 0}
\pscustom[linestyle=none,fillstyle=solid,fillcolor=curcolor]
{
\newpath
\moveto(541.78227631,538.82101756)
\curveto(541.78227578,539.42133576)(541.9302122,540.00236142)(542.22608601,540.56409627)
\curveto(542.52195789,541.12582062)(542.94003908,541.55462184)(543.48033084,541.85050123)
\curveto(544.02490617,542.14636753)(544.6316599,542.29430395)(545.30059385,542.29431093)
\curveto(546.33400075,542.29430395)(547.18088316,541.95769499)(547.84124363,541.28448305)
\curveto(548.50159093,540.61554717)(548.83176787,539.76866475)(548.83177544,538.74383327)
\curveto(548.83176787,537.71041889)(548.49730291,536.85281644)(547.82837958,536.17102336)
\curveto(547.16373111,535.49351657)(546.32542472,535.1547636)(545.3134579,535.15476345)
\curveto(544.68740406,535.1547636)(544.08922635,535.29626801)(543.51892299,535.57927708)
\curveto(542.95290311,535.86228562)(542.52195789,536.2760788)(542.22608601,536.82065786)
\curveto(541.9302122,537.36952191)(541.78227578,538.03630781)(541.78227631,538.82101756)
\moveto(543.63469944,538.72453719)
\curveto(543.63469706,538.04702785)(543.79549752,537.52817837)(544.11710129,537.1679872)
\curveto(544.43869935,536.80779231)(544.83534048,536.6276958)(545.30702587,536.62769712)
\curveto(545.77870317,536.6276958)(546.17320029,536.80779231)(546.49051843,537.1679872)
\curveto(546.81211411,537.52817837)(546.97291457,538.05131586)(546.97292029,538.73740124)
\curveto(546.97291457,539.40632772)(546.81211411,539.92088919)(546.49051843,540.28108718)
\curveto(546.17320029,540.64127524)(545.77870317,540.82137175)(545.30702587,540.82137726)
\curveto(544.83534048,540.82137175)(544.43869935,540.64127524)(544.11710129,540.28108718)
\curveto(543.79549752,539.92088919)(543.63469706,539.40203971)(543.63469944,538.72453719)
}
}
{
\newrgbcolor{curcolor}{0 0 0}
\pscustom[linestyle=none,fillstyle=solid,fillcolor=curcolor]
{
\newpath
\moveto(550.11817999,542.13994234)
\lineto(551.7840744,542.13994234)
\lineto(551.7840744,541.20729875)
\curveto(552.38010562,541.93196692)(553.08977164,542.29430395)(553.9130746,542.29431093)
\curveto(554.35044724,542.29430395)(554.72993632,542.20425569)(555.05154298,542.02416589)
\curveto(555.37313815,541.84406267)(555.6368509,541.57177389)(555.84268203,541.20729875)
\curveto(556.14283635,541.57177389)(556.46658127,541.84406267)(556.81391776,542.02416589)
\curveto(557.16123925,542.20425569)(557.5321523,542.29430395)(557.92665805,542.29431093)
\curveto(558.42834686,542.29430395)(558.85286007,542.19139166)(559.20019895,541.98557375)
\curveto(559.54751805,541.7840305)(559.80694279,541.48601365)(559.97847395,541.0915223)
\curveto(560.10281563,540.79993169)(560.16499181,540.32825035)(560.16500267,539.67647686)
\lineto(560.16500267,535.30913204)
\lineto(558.35760371,535.30913204)
\lineto(558.35760371,539.21337107)
\curveto(558.35759466,539.8908731)(558.29541848,540.32825035)(558.17107499,540.52550412)
\curveto(558.00383365,540.78277964)(557.74655292,540.91142001)(557.39923202,540.91142561)
\curveto(557.14623121,540.91142001)(556.90824653,540.83423579)(556.68527727,540.67987272)
\curveto(556.46229326,540.52549891)(556.3014928,540.29823426)(556.20287541,539.99807809)
\curveto(556.10424424,539.70220056)(556.0549321,539.23266322)(556.05493884,538.58946467)
\lineto(556.05493884,535.30913204)
\lineto(554.24753988,535.30913204)
\lineto(554.24753988,539.05257045)
\curveto(554.24753494,539.71720861)(554.21537485,540.14600983)(554.15105951,540.33897541)
\curveto(554.08673449,540.53193093)(553.9859662,540.67557934)(553.84875435,540.76992107)
\curveto(553.71582143,540.86425187)(553.53358091,540.91142001)(553.30203224,540.91142561)
\curveto(553.02330746,540.91142001)(552.77245874,540.83637979)(552.54948535,540.68630474)
\curveto(552.32650547,540.53621894)(552.16570501,540.31967432)(552.06708349,540.03667024)
\curveto(551.97274446,539.75365671)(551.92557633,539.28411937)(551.92557895,538.62805682)
\lineto(551.92557895,535.30913204)
\lineto(550.11817999,535.30913204)
\lineto(550.11817999,542.13994234)
}
}
{
\newrgbcolor{curcolor}{0 0 0}
\pscustom[linestyle=none,fillstyle=solid,fillcolor=curcolor]
{
\newpath
\moveto(561.82446546,542.13994234)
\lineto(563.49035988,542.13994234)
\lineto(563.49035988,541.20729875)
\curveto(564.0863911,541.93196692)(564.79605712,542.29430395)(565.61936007,542.29431093)
\curveto(566.05673271,542.29430395)(566.4362218,542.20425569)(566.75782846,542.02416589)
\curveto(567.07942363,541.84406267)(567.34313638,541.57177389)(567.5489675,541.20729875)
\curveto(567.84912182,541.57177389)(568.17286674,541.84406267)(568.52020324,542.02416589)
\curveto(568.86752472,542.20425569)(569.23843778,542.29430395)(569.63294352,542.29431093)
\curveto(570.13463234,542.29430395)(570.55914555,542.19139166)(570.90648443,541.98557375)
\curveto(571.25380353,541.7840305)(571.51322826,541.48601365)(571.68475942,541.0915223)
\curveto(571.80910111,540.79993169)(571.87127728,540.32825035)(571.87128814,539.67647686)
\lineto(571.87128814,535.30913204)
\lineto(570.06388918,535.30913204)
\lineto(570.06388918,539.21337107)
\curveto(570.06388013,539.8908731)(570.00170396,540.32825035)(569.87736047,540.52550412)
\curveto(569.71011913,540.78277964)(569.45283839,540.91142001)(569.10551749,540.91142561)
\curveto(568.85251668,540.91142001)(568.614532,540.83423579)(568.39156275,540.67987272)
\curveto(568.16857873,540.52549891)(568.00777827,540.29823426)(567.90916089,539.99807809)
\curveto(567.81052971,539.70220056)(567.76121757,539.23266322)(567.76122432,538.58946467)
\lineto(567.76122432,535.30913204)
\lineto(565.95382536,535.30913204)
\lineto(565.95382536,539.05257045)
\curveto(565.95382042,539.71720861)(565.92166033,540.14600983)(565.85734499,540.33897541)
\curveto(565.79301996,540.53193093)(565.69225168,540.67557934)(565.55503983,540.76992107)
\curveto(565.42210691,540.86425187)(565.23986639,540.91142001)(565.00831772,540.91142561)
\curveto(564.72959293,540.91142001)(564.47874422,540.83637979)(564.25577082,540.68630474)
\curveto(564.03279095,540.53621894)(563.87199049,540.31967432)(563.77336897,540.03667024)
\curveto(563.67902994,539.75365671)(563.6318618,539.28411937)(563.63186442,538.62805682)
\lineto(563.63186442,535.30913204)
\lineto(561.82446546,535.30913204)
\lineto(561.82446546,542.13994234)
}
}
{
\newrgbcolor{curcolor}{0 0 0}
\pscustom[linestyle=none,fillstyle=solid,fillcolor=curcolor]
{
\newpath
\moveto(578.16180877,535.30913204)
\lineto(578.16180877,536.33182398)
\curveto(577.91309862,535.96734192)(577.58506568,535.6800451)(577.17770898,535.46993266)
\curveto(576.77463137,535.2598199)(576.34797416,535.1547636)(575.89773605,535.15476345)
\curveto(575.43891557,535.1547636)(575.02726639,535.25553189)(574.6627873,535.45706861)
\curveto(574.29830432,535.65860504)(574.03459157,535.94161385)(573.87164825,536.30609588)
\curveto(573.70870264,536.67057592)(573.6272304,537.17441736)(573.62723131,537.8176217)
\lineto(573.62723131,542.13994234)
\lineto(575.43463027,542.13994234)
\lineto(575.43463027,539.00111426)
\curveto(575.43462756,538.04059583)(575.46678765,537.45099415)(575.53111064,537.23230745)
\curveto(575.59971603,537.01790491)(575.72192437,536.84638442)(575.89773605,536.71774547)
\curveto(576.07354138,536.5933917)(576.29651801,536.53121553)(576.56666663,536.53121675)
\curveto(576.87539966,536.53121553)(577.15197645,536.61483176)(577.39639782,536.78206571)
\curveto(577.64080984,536.95358473)(577.80804232,537.16369733)(577.89809575,537.41240414)
\curveto(577.98813883,537.66539476)(578.03316296,538.28072451)(578.03316827,539.25839525)
\lineto(578.03316827,542.13994234)
\lineto(579.84056723,542.13994234)
\lineto(579.84056723,535.30913204)
\lineto(578.16180877,535.30913204)
}
}
{
\newrgbcolor{curcolor}{0 0 0}
\pscustom[linestyle=none,fillstyle=solid,fillcolor=curcolor]
{
\newpath
\moveto(587.93205397,535.30913204)
\lineto(586.12465501,535.30913204)
\lineto(586.12465501,538.79528946)
\curveto(586.12464966,539.53282408)(586.08605755,540.00879344)(586.00887857,540.22319896)
\curveto(585.93168911,540.44188267)(585.80519275,540.61125915)(585.62938911,540.73132892)
\curveto(585.45786376,540.85138784)(585.24989517,540.91142001)(585.00548271,540.91142561)
\curveto(584.69245358,540.91142001)(584.41158878,540.82565976)(584.16288746,540.65414462)
\curveto(583.91417936,540.48261879)(583.74265887,540.25535414)(583.64832548,539.97235)
\curveto(583.55827435,539.68933653)(583.51325022,539.16619903)(583.51325296,538.40293595)
\lineto(583.51325296,535.30913204)
\lineto(581.705854,535.30913204)
\lineto(581.705854,542.13994234)
\lineto(583.38461247,542.13994234)
\lineto(583.38461247,541.13654648)
\curveto(583.98064355,541.90838285)(584.73104569,542.29430395)(585.63582113,542.29431093)
\curveto(586.03460141,542.29430395)(586.39908245,542.22140774)(586.72926534,542.07562209)
\curveto(587.05943633,541.93411092)(587.30814104,541.7518704)(587.47538021,541.52889999)
\curveto(587.646894,541.30591713)(587.76481434,541.05292441)(587.82914158,540.76992107)
\curveto(587.89774272,540.4869068)(587.93204681,540.08168964)(587.93205397,539.55426839)
\lineto(587.93205397,535.30913204)
}
}
{
\newrgbcolor{curcolor}{0 0 0}
\pscustom[linestyle=none,fillstyle=solid,fillcolor=curcolor]
{
\newpath
\moveto(589.77161264,543.06615391)
\lineto(589.77161264,544.73848034)
\lineto(591.5790116,544.73848034)
\lineto(591.5790116,543.06615391)
\lineto(589.77161264,543.06615391)
\moveto(589.77161264,535.30913204)
\lineto(589.77161264,542.13994234)
\lineto(591.5790116,542.13994234)
\lineto(591.5790116,535.30913204)
\lineto(589.77161264,535.30913204)
}
}
{
\newrgbcolor{curcolor}{0 0 0}
\pscustom[linestyle=none,fillstyle=solid,fillcolor=curcolor]
{
\newpath
\moveto(596.55739768,542.13994234)
\lineto(596.55739768,540.69916879)
\lineto(595.32244893,540.69916879)
\lineto(595.32244893,537.9462622)
\curveto(595.32244608,537.38881797)(595.33316611,537.06292904)(595.35460905,536.96859443)
\curveto(595.38033425,536.87854452)(595.4339344,536.8035043)(595.51540967,536.74347356)
\curveto(595.60116688,536.68343996)(595.70407917,536.65342387)(595.82414686,536.65342522)
\curveto(595.99137599,536.65342387)(596.23364868,536.71131204)(596.55096566,536.82708989)
\lineto(596.70533425,535.42490849)
\curveto(596.28510483,535.24481186)(595.80913547,535.1547636)(595.27742475,535.15476345)
\curveto(594.95153303,535.1547636)(594.65780419,535.20836376)(594.39623736,535.31556407)
\curveto(594.1346667,535.42705238)(593.94170615,535.56855678)(593.81735513,535.7400777)
\curveto(593.69728945,535.91588577)(593.61367321,536.15172644)(593.56650617,536.44760043)
\curveto(593.52791297,536.65771189)(593.50861691,537.0822251)(593.50861794,537.72114133)
\lineto(593.50861794,540.69916879)
\lineto(592.67888675,540.69916879)
\lineto(592.67888675,542.13994234)
\lineto(593.50861794,542.13994234)
\lineto(593.50861794,543.49709957)
\lineto(595.32244893,544.55195163)
\lineto(595.32244893,542.13994234)
\lineto(596.55739768,542.13994234)
}
}
{
\newrgbcolor{curcolor}{0 0 0}
\pscustom[linestyle=none,fillstyle=solid,fillcolor=curcolor]
{
\newpath
\moveto(597.82450724,543.06615391)
\lineto(597.82450724,544.73848034)
\lineto(599.6319062,544.73848034)
\lineto(599.6319062,543.06615391)
\lineto(597.82450724,543.06615391)
\moveto(597.82450724,535.30913204)
\lineto(597.82450724,542.13994234)
\lineto(599.6319062,542.13994234)
\lineto(599.6319062,535.30913204)
\lineto(597.82450724,535.30913204)
}
}
{
\newrgbcolor{curcolor}{0 0 0}
\pscustom[linestyle=none,fillstyle=solid,fillcolor=curcolor]
{
\newpath
\moveto(605.43359144,537.48315641)
\lineto(607.23455838,537.18085125)
\curveto(607.00299901,536.5204955)(606.63637397,536.01665406)(606.13468214,535.66932543)
\curveto(605.63726712,535.32628409)(605.01336134,535.1547636)(604.26296294,535.15476345)
\curveto(603.07517982,535.1547636)(602.19613732,535.54282871)(601.62583278,536.31895993)
\curveto(601.17559041,536.94072069)(600.95046977,537.72542693)(600.95047018,538.67308099)
\curveto(600.95046977,539.80511286)(601.24634261,540.69058738)(601.8380896,541.32950722)
\curveto(602.42983398,541.97270303)(603.17809211,542.29430395)(604.08286624,542.29431093)
\curveto(605.09912159,542.29430395)(605.90097987,541.95769499)(606.4884435,541.28448305)
\curveto(607.07589522,540.61554717)(607.35676002,539.58856824)(607.33103875,538.20354319)
\lineto(602.80289332,538.20354319)
\curveto(602.81575508,537.66753876)(602.9615475,537.24945757)(603.240271,536.94929836)
\curveto(603.51898909,536.65342387)(603.86631808,536.50548745)(604.28225901,536.50548865)
\curveto(604.56526407,536.50548745)(604.80324875,536.58267167)(604.99621376,536.73704154)
\curveto(605.18916985,536.89140855)(605.33496226,537.14011326)(605.43359144,537.48315641)
\moveto(605.53650384,539.30985144)
\curveto(605.5236348,539.83298494)(605.38856241,540.22962607)(605.13128628,540.49977603)
\curveto(604.87400095,540.77420362)(604.56097605,540.91142001)(604.19221066,540.91142561)
\curveto(603.79770988,540.91142001)(603.47182095,540.7677716)(603.2145429,540.48047995)
\curveto(602.95725948,540.19317796)(602.83076312,539.80296885)(602.83505344,539.30985144)
\lineto(605.53650384,539.30985144)
}
}
{
\newrgbcolor{curcolor}{0 0 0}
\pscustom[linestyle=none,fillstyle=solid,fillcolor=curcolor]
{
\newpath
\moveto(608.17363766,537.25803555)
\lineto(609.98746865,537.53461261)
\curveto(610.06465074,537.18299338)(610.22116319,536.91499262)(610.45700646,536.73060952)
\curveto(610.69284454,536.55051158)(611.02302148,536.46046332)(611.44753827,536.46046448)
\curveto(611.91492802,536.46046332)(612.26654502,536.54622357)(612.50239033,536.71774547)
\curveto(612.66104214,536.8378084)(612.74037037,536.99860886)(612.74037525,537.20014732)
\curveto(612.74037037,537.33736182)(612.69749025,537.45099415)(612.61173475,537.54104464)
\curveto(612.52168175,537.62680265)(612.32014517,537.70613087)(612.00712442,537.77902955)
\curveto(610.54919613,538.100628)(609.62512949,538.39435684)(609.23492175,538.66021694)
\curveto(608.69463084,539.02898264)(608.42448607,539.5414001)(608.42448663,540.19747086)
\curveto(608.42448607,540.78921166)(608.65818274,541.28662108)(609.12557733,541.68970061)
\curveto(609.5929694,542.09276738)(610.31764347,542.29430395)(611.2996017,542.29431093)
\curveto(612.23438493,542.29430395)(612.92904291,542.14207952)(613.38357772,541.83763718)
\curveto(613.8381015,541.53318178)(614.15112639,541.0829405)(614.32265334,540.48691198)
\lineto(612.61816678,540.17174276)
\curveto(612.54526581,540.43759466)(612.40590542,540.64127524)(612.20008517,540.78278512)
\curveto(611.99854426,540.92428405)(611.70910343,540.99503625)(611.33176182,540.99504193)
\curveto(610.855789,540.99503625)(610.51489203,540.92857206)(610.30906989,540.79564916)
\curveto(610.17185105,540.70130741)(610.10324285,540.57909906)(610.10324509,540.42902375)
\curveto(610.10324285,540.30037827)(610.16327503,540.19103396)(610.28334179,540.10099049)
\curveto(610.44628383,539.98092136)(611.00801343,539.81154487)(611.96853227,539.59286053)
\curveto(612.93333092,539.37416763)(613.60654884,539.10616686)(613.98818805,538.78885744)
\curveto(614.365527,538.46725304)(614.55419954,538.01915577)(614.55420623,537.44456426)
\curveto(614.55419954,536.81851234)(614.29263079,536.28036681)(613.76949921,535.83012605)
\curveto(613.24635581,535.37988424)(612.47236961,535.1547636)(611.44753827,535.15476345)
\curveto(610.51703603,535.1547636)(609.77949793,535.34343614)(609.23492175,535.72078163)
\curveto(608.69463084,536.09812629)(608.34086983,536.61054375)(608.17363766,537.25803555)
}
}
{
\newrgbcolor{curcolor}{0 1 0.25098041}
\pscustom[linewidth=2.63455725,linecolor=curcolor]
{
\newpath
\moveto(648.84463,523.51032456)
\lineto(733.15046,523.51032456)
\lineto(733.15046,576.20147456)
\lineto(676.50748,576.20147456)
\lineto(676.50748,586.73970456)
\lineto(648.84463,586.73970456)
\lineto(648.84463,523.51032456)
\closepath
}
}
{
\newrgbcolor{curcolor}{0 0 0}
\pscustom[linestyle=none,fillstyle=solid,fillcolor=curcolor]
{
\newpath
\moveto(664.05504773,559.19127445)
\lineto(664.05504773,560.27185461)
\lineto(670.30054377,562.90898476)
\lineto(670.30054377,561.75765233)
\lineto(665.34788471,559.7251325)
\lineto(670.30054377,557.6733166)
\lineto(670.30054377,556.52198417)
\lineto(664.05504773,559.19127445)
}
}
{
\newrgbcolor{curcolor}{0 0 0}
\pscustom[linestyle=none,fillstyle=solid,fillcolor=curcolor]
{
\newpath
\moveto(671.74774955,559.19127445)
\lineto(671.74774955,560.27185461)
\lineto(677.99324559,562.90898476)
\lineto(677.99324559,561.75765233)
\lineto(673.04058652,559.7251325)
\lineto(677.99324559,557.6733166)
\lineto(677.99324559,556.52198417)
\lineto(671.74774955,559.19127445)
}
}
{
\newrgbcolor{curcolor}{0 0 0}
\pscustom[linestyle=none,fillstyle=solid,fillcolor=curcolor]
{
\newpath
\moveto(684.04578109,555.91094182)
\curveto(683.61697455,555.54645994)(683.20318137,555.2891792)(682.80440031,555.13909885)
\curveto(682.40989911,554.98901835)(681.9853859,554.91397813)(681.53085941,554.91397798)
\curveto(680.78045446,554.91397813)(680.20371682,555.09621865)(679.80064475,555.46070008)
\curveto(679.39757052,555.82946874)(679.19603395,556.29900608)(679.19603442,556.86931351)
\curveto(679.19603395,557.20377666)(679.27107416,557.50822553)(679.42115529,557.78266102)
\curveto(679.57552303,558.0613791)(679.7749156,558.28435574)(680.01933359,558.4515916)
\curveto(680.268037,558.61882069)(680.5467578,558.74531705)(680.85549681,558.83108106)
\curveto(681.08275932,558.89110947)(681.4258003,558.94899763)(681.88462077,559.00474573)
\curveto(682.81940427,559.11623011)(683.50763023,559.24915849)(683.94930072,559.40353126)
\curveto(683.9535835,559.56218338)(683.95572751,559.66295167)(683.95573275,559.70583643)
\curveto(683.95572751,560.17751313)(683.8463832,560.50983408)(683.62769948,560.70280027)
\curveto(683.33182173,560.96436338)(682.89230048,561.09514775)(682.30913441,561.09515378)
\curveto(681.76455327,561.09514775)(681.36148012,560.99866747)(681.09991375,560.80571266)
\curveto(680.84263064,560.61703439)(680.6518141,560.28042543)(680.52746355,559.79588477)
\lineto(679.39542719,559.95025337)
\curveto(679.49833881,560.43479387)(679.66771529,560.82500298)(679.90355715,561.12088188)
\curveto(680.13939663,561.42103668)(680.48029361,561.65044533)(680.92624908,561.80910852)
\curveto(681.37220015,561.97204625)(681.88890562,562.05351848)(682.47636705,562.05352547)
\curveto(683.05953296,562.05351848)(683.53335831,561.98491029)(683.89784452,561.84770067)
\curveto(684.26232038,561.7104775)(684.53032115,561.53681301)(684.70184762,561.32670667)
\curveto(684.87336213,561.12087582)(684.99342647,560.85930708)(685.06204101,560.54199965)
\curveto(685.10062677,560.34474561)(685.11992283,559.9888406)(685.11992923,559.47428354)
\lineto(685.11992923,557.93059759)
\curveto(685.11992283,556.85430366)(685.1435069,556.17250972)(685.1906815,555.88521372)
\curveto(685.24213118,555.6022041)(685.34075546,555.32991532)(685.48655464,555.06834657)
\lineto(684.27733398,555.06834657)
\curveto(684.15726409,555.30847526)(684.08007987,555.58934006)(684.04578109,555.91094182)
\moveto(683.94930072,558.49661577)
\curveto(683.52907029,558.32509186)(682.8987325,558.17929944)(682.05828544,558.05923809)
\curveto(681.58231275,557.9906269)(681.24570379,557.91344268)(681.04845755,557.8276852)
\curveto(680.85120666,557.74192219)(680.69898223,557.61542583)(680.5917838,557.44819574)
\curveto(680.48458162,557.28524889)(680.43098147,557.10300837)(680.43098318,556.90147363)
\curveto(680.43098147,556.59273492)(680.5467578,556.33545419)(680.77831251,556.12963066)
\curveto(681.01415113,555.92380501)(681.35719211,555.82089272)(681.80743648,555.82089347)
\curveto(682.25338666,555.82089272)(682.65002779,555.91737299)(682.99736106,556.11033459)
\curveto(683.34468577,556.30758211)(683.5998225,556.57558287)(683.762772,556.91433768)
\curveto(683.88711932,557.17590458)(683.94929549,557.56182568)(683.94930072,558.07210214)
\lineto(683.94930072,558.49661577)
}
}
{
\newrgbcolor{curcolor}{0 0 0}
\pscustom[linestyle=none,fillstyle=solid,fillcolor=curcolor]
{
\newpath
\moveto(686.92089507,552.4505125)
\lineto(686.92089507,561.89915687)
\lineto(687.97574713,561.89915687)
\lineto(687.97574713,561.01153745)
\curveto(688.22444991,561.3588605)(688.50531471,561.61828524)(688.81834237,561.78981245)
\curveto(689.1313645,561.96561423)(689.51085358,562.05351848)(689.95681075,562.05352547)
\curveto(690.53997651,562.05351848)(691.05453798,561.90343805)(691.5004967,561.60328373)
\curveto(691.94644452,561.30311634)(692.28305348,560.87860313)(692.51032458,560.32974283)
\curveto(692.73758277,559.78516002)(692.8512151,559.18698231)(692.8512219,558.53520792)
\curveto(692.8512151,557.83625846)(692.72471874,557.20592067)(692.47173244,556.64419264)
\curveto(692.22302131,556.08674948)(691.85854027,555.65794825)(691.37828823,555.35778769)
\curveto(690.90231354,555.06191456)(690.40061611,554.91397813)(689.87319443,554.91397798)
\curveto(689.48726951,554.91397813)(689.13994052,554.99545037)(688.83120642,555.15839492)
\curveto(688.52675477,555.3213393)(688.27590606,555.52716388)(688.07865952,555.7758693)
\lineto(688.07865952,552.4505125)
\lineto(686.92089507,552.4505125)
\moveto(687.9693151,558.44515957)
\curveto(687.96931319,557.56611369)(688.14726569,556.91647984)(688.50317316,556.49625607)
\curveto(688.85907572,556.07602945)(689.29002095,555.86591685)(689.79601014,555.86591764)
\curveto(690.31056786,555.86591685)(690.75008911,556.08246146)(691.11457521,556.51555215)
\curveto(691.4833392,556.95292794)(691.66772373,557.62828987)(691.66772934,558.54163995)
\curveto(691.66772373,559.41210295)(691.48762721,560.06388081)(691.12743926,560.49697547)
\curveto(690.77152917,560.93005928)(690.34487196,561.1466039)(689.84746633,561.14660997)
\curveto(689.35434113,561.1466039)(688.91696389,560.91505124)(688.53533328,560.4519513)
\curveto(688.15798572,559.99312861)(687.96931319,559.3241987)(687.9693151,558.44515957)
}
}
{
\newrgbcolor{curcolor}{0 0 0}
\pscustom[linestyle=none,fillstyle=solid,fillcolor=curcolor]
{
\newpath
\moveto(694.25340315,552.4505125)
\lineto(694.25340315,561.89915687)
\lineto(695.30825521,561.89915687)
\lineto(695.30825521,561.01153745)
\curveto(695.556958,561.3588605)(695.8378228,561.61828524)(696.15085046,561.78981245)
\curveto(696.46387258,561.96561423)(696.84336167,562.05351848)(697.28931884,562.05352547)
\curveto(697.8724846,562.05351848)(698.38704607,561.90343805)(698.83300478,561.60328373)
\curveto(699.27895261,561.30311634)(699.61556157,560.87860313)(699.84283267,560.32974283)
\curveto(700.07009086,559.78516002)(700.18372319,559.18698231)(700.18372998,558.53520792)
\curveto(700.18372319,557.83625846)(700.05722683,557.20592067)(699.80424052,556.64419264)
\curveto(699.5555294,556.08674948)(699.19104836,555.65794825)(698.71079631,555.35778769)
\curveto(698.23482163,555.06191456)(697.7331242,554.91397813)(697.20570252,554.91397798)
\curveto(696.8197776,554.91397813)(696.47244861,554.99545037)(696.16371451,555.15839492)
\curveto(695.85926286,555.3213393)(695.60841415,555.52716388)(695.41116761,555.7758693)
\lineto(695.41116761,552.4505125)
\lineto(694.25340315,552.4505125)
\moveto(695.30182319,558.44515957)
\curveto(695.30182127,557.56611369)(695.47977378,556.91647984)(695.83568125,556.49625607)
\curveto(696.19158381,556.07602945)(696.62252904,555.86591685)(697.12851822,555.86591764)
\curveto(697.64307595,555.86591685)(698.0825972,556.08246146)(698.4470833,556.51555215)
\curveto(698.81584729,556.95292794)(699.00023181,557.62828987)(699.00023743,558.54163995)
\curveto(699.00023181,559.41210295)(698.8201353,560.06388081)(698.45994735,560.49697547)
\curveto(698.10403726,560.93005928)(697.67738004,561.1466039)(697.17997442,561.14660997)
\curveto(696.68684922,561.1466039)(696.24947197,560.91505124)(695.86784137,560.4519513)
\curveto(695.49049381,559.99312861)(695.30182127,559.3241987)(695.30182319,558.44515957)
}
}
{
\newrgbcolor{curcolor}{0 0 0}
\pscustom[linestyle=none,fillstyle=solid,fillcolor=curcolor]
{
\newpath
\moveto(707.68347072,559.19127445)
\lineto(701.43797467,556.52198417)
\lineto(701.43797467,557.6733166)
\lineto(706.38420171,559.7251325)
\lineto(701.43797467,561.75765233)
\lineto(701.43797467,562.90898476)
\lineto(707.68347072,560.27185461)
\lineto(707.68347072,559.19127445)
}
}
{
\newrgbcolor{curcolor}{0 0 0}
\pscustom[linestyle=none,fillstyle=solid,fillcolor=curcolor]
{
\newpath
\moveto(715.37617396,559.19127445)
\lineto(709.13067792,556.52198417)
\lineto(709.13067792,557.6733166)
\lineto(714.07690496,559.7251325)
\lineto(709.13067792,561.75765233)
\lineto(709.13067792,562.90898476)
\lineto(715.37617396,560.27185461)
\lineto(715.37617396,559.19127445)
}
}
{
\newrgbcolor{curcolor}{0 0 0}
\pscustom[linestyle=none,fillstyle=solid,fillcolor=curcolor]
{
\newpath
\moveto(664.11293595,534.85894957)
\lineto(666.1776159,534.6081006)
\curveto(666.21191716,534.36797262)(666.29124538,534.20288415)(666.41560082,534.1128347)
\curveto(666.58711823,533.98419552)(666.857263,533.91987534)(667.22603594,533.91987395)
\curveto(667.69771339,533.91987534)(668.0514744,533.99062754)(668.28732002,534.13213077)
\curveto(668.44597152,534.22646822)(668.56603587,534.37869265)(668.64751341,534.58880453)
\curveto(668.70325226,534.73888568)(668.73112434,535.01546246)(668.73112973,535.41853572)
\lineto(668.73112973,536.41549956)
\curveto(668.1908348,535.67796035)(667.50904085,535.3091913)(666.68574586,535.3091913)
\curveto(665.76810789,535.3091913)(665.04128982,535.69725641)(664.50528946,536.47338778)
\curveto(664.0850631,537.08657237)(663.8749505,537.84983854)(663.87495104,538.7631886)
\curveto(663.8749505,539.90808441)(664.14938328,540.7828389)(664.69825021,541.3874547)
\curveto(665.25140242,541.99205835)(665.93748437,542.29436321)(666.75649813,542.29437019)
\curveto(667.60123312,542.29436321)(668.2980351,541.92345015)(668.84690618,541.18162991)
\lineto(668.84690618,542.1400016)
\lineto(670.53852869,542.1400016)
\lineto(670.53852869,536.010282)
\curveto(670.53852149,535.204135)(670.4720573,534.60166928)(670.33913592,534.20288304)
\curveto(670.20620054,533.80409901)(670.01967201,533.49107412)(669.77954977,533.26380743)
\curveto(669.53941464,533.03654482)(669.21781372,532.85859232)(668.81474605,532.72994937)
\curveto(668.41595544,532.60131158)(667.90997,532.5369914)(667.29678821,532.53698863)
\curveto(666.13902095,532.5369914)(665.31786661,532.73638397)(664.83332273,533.13516693)
\curveto(664.34877585,533.52966623)(664.10650316,534.03136366)(664.10650393,534.64026073)
\curveto(664.10650316,534.70029357)(664.10864716,534.77318977)(664.11293595,534.85894957)
\moveto(665.72737417,538.86610099)
\curveto(665.72737178,538.14142337)(665.86673217,537.60970986)(666.14545578,537.27095885)
\curveto(666.42846177,536.93649194)(666.77579076,536.76925946)(667.18744379,536.76926092)
\curveto(667.6291052,536.76925946)(668.00216226,536.94077995)(668.3066161,537.2838229)
\curveto(668.61105999,537.63114992)(668.76328443,538.14356738)(668.76328986,538.82107682)
\curveto(668.76328443,539.52859533)(668.61749201,540.05387682)(668.32591217,540.39692289)
\curveto(668.03432235,540.73995878)(667.6655533,540.91147927)(667.21960391,540.91148487)
\curveto(666.78651079,540.91147927)(666.42846177,540.74210278)(666.14545578,540.40335491)
\curveto(665.86673217,540.06888486)(665.72737178,539.5564674)(665.72737417,538.86610099)
}
}
{
\newrgbcolor{curcolor}{0 0 0}
\pscustom[linestyle=none,fillstyle=solid,fillcolor=curcolor]
{
\newpath
\moveto(676.28875842,537.48321567)
\lineto(678.08972535,537.18091051)
\curveto(677.85816599,536.52055475)(677.49154094,536.01671332)(676.98984912,535.66938469)
\curveto(676.4924341,535.32634335)(675.86852832,535.15482286)(675.11812991,535.15482271)
\curveto(673.9303468,535.15482286)(673.05130429,535.54288797)(672.48099976,536.31901919)
\curveto(672.03075738,536.94077995)(671.80563674,537.72548619)(671.80563716,538.67314025)
\curveto(671.80563674,539.80517211)(672.10150958,540.69064664)(672.69325658,541.32956648)
\curveto(673.28500096,541.97276229)(674.03325909,542.29436321)(674.93803322,542.29437019)
\curveto(675.95428856,542.29436321)(676.75614685,541.95775425)(677.34361048,541.2845423)
\curveto(677.9310622,540.61560642)(678.211927,539.5886275)(678.18620572,538.20360244)
\lineto(673.65806029,538.20360244)
\curveto(673.67092206,537.66759802)(673.81671447,537.24951683)(674.09543797,536.94935762)
\curveto(674.37415606,536.65348313)(674.72148505,536.50554671)(675.13742599,536.50554791)
\curveto(675.42043104,536.50554671)(675.65841572,536.58273093)(675.85138073,536.7371008)
\curveto(676.04433682,536.89146781)(676.19012924,537.14017252)(676.28875842,537.48321567)
\moveto(676.39167081,539.3099107)
\curveto(676.37880177,539.83304419)(676.24372939,540.22968532)(675.98645325,540.49983528)
\curveto(675.72916792,540.77426288)(675.41614303,540.91147927)(675.04737764,540.91148487)
\curveto(674.65287685,540.91147927)(674.32698793,540.76783086)(674.06970988,540.48053921)
\curveto(673.81242646,540.19323722)(673.6859301,539.80302811)(673.69022041,539.3099107)
\lineto(676.39167081,539.3099107)
}
}
{
\newrgbcolor{curcolor}{0 0 0}
\pscustom[linestyle=none,fillstyle=solid,fillcolor=curcolor]
{
\newpath
\moveto(679.02880083,537.2580948)
\lineto(680.84263181,537.53467187)
\curveto(680.91981391,537.18305264)(681.07632635,536.91505188)(681.31216962,536.73066877)
\curveto(681.5480077,536.55057084)(681.87818464,536.46052258)(682.30270143,536.46052373)
\curveto(682.77009118,536.46052258)(683.12170818,536.54628283)(683.35755349,536.71780472)
\curveto(683.5162053,536.83786766)(683.59553353,536.99866812)(683.59553841,537.20020658)
\curveto(683.59553353,537.33742108)(683.55265341,537.4510534)(683.46689791,537.54110389)
\curveto(683.37684491,537.62686191)(683.17530833,537.70619013)(682.86228758,537.77908881)
\curveto(681.40435929,538.10068726)(680.48029265,538.39441609)(680.09008491,538.6602762)
\curveto(679.549794,539.0290419)(679.27964923,539.54145936)(679.27964979,540.19753012)
\curveto(679.27964923,540.78927092)(679.5133459,541.28668034)(679.98074049,541.68975986)
\curveto(680.44813256,542.09282663)(681.17280663,542.29436321)(682.15476486,542.29437019)
\curveto(683.08954809,542.29436321)(683.78420607,542.14213877)(684.23874088,541.83769643)
\curveto(684.69326466,541.53324104)(685.00628955,541.08299976)(685.1778165,540.48697123)
\lineto(683.47332994,540.17180202)
\curveto(683.40042897,540.43765392)(683.26106858,540.6413345)(683.05524833,540.78284437)
\curveto(682.85370742,540.9243433)(682.56426659,540.9950955)(682.18692498,540.99510119)
\curveto(681.71095216,540.9950955)(681.37005519,540.92863132)(681.16423305,540.79570842)
\curveto(681.02701421,540.70136667)(680.95840602,540.57915832)(680.95840825,540.42908301)
\curveto(680.95840602,540.30043752)(681.01843819,540.19109321)(681.13850495,540.10104975)
\curveto(681.30144699,539.98098061)(681.86317659,539.81160413)(682.82369543,539.59291979)
\curveto(683.78849408,539.37422689)(684.461712,539.10622612)(684.84335121,538.7889167)
\curveto(685.22069016,538.4673123)(685.4093627,538.01921502)(685.40936939,537.44462352)
\curveto(685.4093627,536.8185716)(685.14779395,536.28042607)(684.62466237,535.83018531)
\curveto(684.10151897,535.3799435)(683.32753277,535.15482286)(682.30270143,535.15482271)
\curveto(681.37219919,535.15482286)(680.63466109,535.3434954)(680.09008491,535.72084089)
\curveto(679.549794,536.09818555)(679.19603299,536.61060301)(679.02880083,537.2580948)
}
}
{
\newrgbcolor{curcolor}{0 0 0}
\pscustom[linestyle=none,fillstyle=solid,fillcolor=curcolor]
{
\newpath
\moveto(690.13047542,542.1400016)
\lineto(690.13047542,540.69922805)
\lineto(688.89552667,540.69922805)
\lineto(688.89552667,537.94632145)
\curveto(688.89552383,537.38887723)(688.90624386,537.0629883)(688.92768679,536.96865369)
\curveto(688.95341199,536.87860377)(689.00701214,536.80356356)(689.08848741,536.74353282)
\curveto(689.17424462,536.68349922)(689.27715691,536.65348313)(689.3972246,536.65348448)
\curveto(689.56445373,536.65348313)(689.80672642,536.7113713)(690.1240434,536.82714914)
\lineto(690.27841199,535.42496775)
\curveto(689.85818257,535.24487112)(689.38221321,535.15482286)(688.8505025,535.15482271)
\curveto(688.52461077,535.15482286)(688.23088193,535.20842301)(687.9693151,535.31562333)
\curveto(687.70774444,535.42711164)(687.51478389,535.56861604)(687.39043287,535.74013696)
\curveto(687.27036719,535.91594503)(687.18675096,536.1517857)(687.13958391,536.44765968)
\curveto(687.10099071,536.65777114)(687.08169466,537.08228435)(687.08169569,537.72120059)
\lineto(687.08169569,540.69922805)
\lineto(686.25196449,540.69922805)
\lineto(686.25196449,542.1400016)
\lineto(687.08169569,542.1400016)
\lineto(687.08169569,543.49715882)
\lineto(688.89552667,544.55201088)
\lineto(688.89552667,542.1400016)
\lineto(690.13047542,542.1400016)
}
}
{
\newrgbcolor{curcolor}{0 0 0}
\pscustom[linestyle=none,fillstyle=solid,fillcolor=curcolor]
{
\newpath
\moveto(691.39758498,543.06621316)
\lineto(691.39758498,544.7385396)
\lineto(693.20498394,544.7385396)
\lineto(693.20498394,543.06621316)
\lineto(691.39758498,543.06621316)
\moveto(691.39758498,535.3091913)
\lineto(691.39758498,542.1400016)
\lineto(693.20498394,542.1400016)
\lineto(693.20498394,535.3091913)
\lineto(691.39758498,535.3091913)
}
}
{
\newrgbcolor{curcolor}{0 0 0}
\pscustom[linestyle=none,fillstyle=solid,fillcolor=curcolor]
{
\newpath
\moveto(694.63289235,538.82107682)
\curveto(694.63289182,539.42139502)(694.78082824,540.00242068)(695.07670205,540.56415553)
\curveto(695.37257393,541.12587988)(695.79065512,541.5546811)(696.33094688,541.85056048)
\curveto(696.87552221,542.14642679)(697.48227594,542.29436321)(698.15120989,542.29437019)
\curveto(699.18461679,542.29436321)(700.0314992,541.95775425)(700.69185967,541.2845423)
\curveto(701.35220697,540.61560642)(701.68238391,539.76872401)(701.68239149,538.74389252)
\curveto(701.68238391,537.71047814)(701.34791896,536.8528757)(700.67899562,536.17108262)
\curveto(700.01434716,535.49357583)(699.17604077,535.15482286)(698.16407394,535.15482271)
\curveto(697.5380201,535.15482286)(696.93984239,535.29632726)(696.36953903,535.57933634)
\curveto(695.80351916,535.86234488)(695.37257393,536.27613806)(695.07670205,536.82071712)
\curveto(694.78082824,537.36958117)(694.63289182,538.03636707)(694.63289235,538.82107682)
\moveto(696.48531548,538.72459645)
\curveto(696.4853131,538.0470871)(696.64611356,537.52823762)(696.96771734,537.16804646)
\curveto(697.28931539,536.80785157)(697.68595652,536.62775506)(698.15764192,536.62775638)
\curveto(698.62931921,536.62775506)(699.02381633,536.80785157)(699.34113447,537.16804646)
\curveto(699.66273015,537.52823762)(699.82353061,538.05137512)(699.82353633,538.7374605)
\curveto(699.82353061,539.40638698)(699.66273015,539.92094844)(699.34113447,540.28114644)
\curveto(699.02381633,540.6413345)(698.62931921,540.82143101)(698.15764192,540.82143652)
\curveto(697.68595652,540.82143101)(697.28931539,540.6413345)(696.96771734,540.28114644)
\curveto(696.64611356,539.92094844)(696.4853131,539.40209896)(696.48531548,538.72459645)
}
}
{
\newrgbcolor{curcolor}{0 0 0}
\pscustom[linestyle=none,fillstyle=solid,fillcolor=curcolor]
{
\newpath
\moveto(709.31720447,535.3091913)
\lineto(707.50980551,535.3091913)
\lineto(707.50980551,538.79534872)
\curveto(707.50980016,539.53288334)(707.47120805,540.00885269)(707.39402906,540.22325822)
\curveto(707.31683961,540.44194193)(707.19034325,540.61131841)(707.0145396,540.73138818)
\curveto(706.84301426,540.8514471)(706.63504567,540.91147927)(706.3906332,540.91148487)
\curveto(706.07760408,540.91147927)(705.79673928,540.82571902)(705.54803796,540.65420388)
\curveto(705.29932986,540.48267804)(705.12780937,540.2554134)(705.03347598,539.97240925)
\curveto(704.94342485,539.68939578)(704.89840072,539.16625829)(704.89840346,538.40299521)
\lineto(704.89840346,535.3091913)
\lineto(703.0910045,535.3091913)
\lineto(703.0910045,542.1400016)
\lineto(704.76976296,542.1400016)
\lineto(704.76976296,541.13660574)
\curveto(705.36579405,541.90844211)(706.11619619,542.29436321)(707.02097163,542.29437019)
\curveto(707.4197519,542.29436321)(707.78423294,542.221467)(708.11441584,542.07568135)
\curveto(708.44458682,541.93417018)(708.69329153,541.75192966)(708.86053071,541.52895925)
\curveto(709.0320445,541.30597639)(709.14996483,541.05298367)(709.21429207,540.76998032)
\curveto(709.28289321,540.48696606)(709.31719731,540.0817489)(709.31720447,539.55432764)
\lineto(709.31720447,535.3091913)
}
}
{
\newrgbcolor{curcolor}{0 0 0}
\pscustom[linestyle=none,fillstyle=solid,fillcolor=curcolor]
{
\newpath
\moveto(710.51999269,537.2580948)
\lineto(712.33382367,537.53467187)
\curveto(712.41100577,537.18305264)(712.56751822,536.91505188)(712.80336148,536.73066877)
\curveto(713.03919956,536.55057084)(713.3693765,536.46052258)(713.79389329,536.46052373)
\curveto(714.26128304,536.46052258)(714.61290004,536.54628283)(714.84874535,536.71780472)
\curveto(715.00739717,536.83786766)(715.08672539,536.99866812)(715.08673027,537.20020658)
\curveto(715.08672539,537.33742108)(715.04384527,537.4510534)(714.95808977,537.54110389)
\curveto(714.86803677,537.62686191)(714.6665002,537.70619013)(714.35347945,537.77908881)
\curveto(712.89555115,538.10068726)(711.97148452,538.39441609)(711.58127677,538.6602762)
\curveto(711.04098587,539.0290419)(710.7708411,539.54145936)(710.77084165,540.19753012)
\curveto(710.7708411,540.78927092)(711.00453776,541.28668034)(711.47193235,541.68975986)
\curveto(711.93932443,542.09282663)(712.66399849,542.29436321)(713.64595672,542.29437019)
\curveto(714.58073995,542.29436321)(715.27539793,542.14213877)(715.72993275,541.83769643)
\curveto(716.18445652,541.53324104)(716.49748142,541.08299976)(716.66900836,540.48697123)
\lineto(714.9645218,540.17180202)
\curveto(714.89162084,540.43765392)(714.75226044,540.6413345)(714.54644019,540.78284437)
\curveto(714.34489928,540.9243433)(714.05545846,540.9950955)(713.67811685,540.99510119)
\curveto(713.20214402,540.9950955)(712.86124705,540.92863132)(712.65542491,540.79570842)
\curveto(712.51820607,540.70136667)(712.44959788,540.57915832)(712.44960012,540.42908301)
\curveto(712.44959788,540.30043752)(712.50963005,540.19109321)(712.62969681,540.10104975)
\curveto(712.79263886,539.98098061)(713.35436846,539.81160413)(714.3148873,539.59291979)
\curveto(715.27968594,539.37422689)(715.95290386,539.10622612)(716.33454307,538.7889167)
\curveto(716.71188203,538.4673123)(716.90055456,538.01921502)(716.90056125,537.44462352)
\curveto(716.90055456,536.8185716)(716.63898582,536.28042607)(716.11585423,535.83018531)
\curveto(715.59271084,535.3799435)(714.81872463,535.15482286)(713.79389329,535.15482271)
\curveto(712.86339106,535.15482286)(712.12585296,535.3434954)(711.58127677,535.72084089)
\curveto(711.04098587,536.09818555)(710.68722486,536.61060301)(710.51999269,537.2580948)
}
}
{
\newrgbcolor{curcolor}{0 1 0.25098041}
\pscustom[linewidth=2.63455725,linecolor=curcolor]
{
\newpath
\moveto(648.84463,404.10576456)
\lineto(733.15046,404.10576456)
\lineto(733.15046,456.79691456)
\lineto(676.50748,456.79691456)
\lineto(676.50748,467.33514456)
\lineto(648.84463,467.33514456)
\lineto(648.84463,404.10576456)
\closepath
}
}
{
\newrgbcolor{curcolor}{0 0 0}
\pscustom[linestyle=none,fillstyle=solid,fillcolor=curcolor]
{
\newpath
\moveto(664.05504773,439.78671201)
\lineto(664.05504773,440.86729217)
\lineto(670.30054377,443.50442232)
\lineto(670.30054377,442.35308989)
\lineto(665.34788471,440.32057006)
\lineto(670.30054377,438.26875416)
\lineto(670.30054377,437.11742173)
\lineto(664.05504773,439.78671201)
}
}
{
\newrgbcolor{curcolor}{0 0 0}
\pscustom[linestyle=none,fillstyle=solid,fillcolor=curcolor]
{
\newpath
\moveto(671.74774955,439.78671201)
\lineto(671.74774955,440.86729217)
\lineto(677.99324559,443.50442232)
\lineto(677.99324559,442.35308989)
\lineto(673.04058652,440.32057006)
\lineto(677.99324559,438.26875416)
\lineto(677.99324559,437.11742173)
\lineto(671.74774955,439.78671201)
}
}
{
\newrgbcolor{curcolor}{0 0 0}
\pscustom[linestyle=none,fillstyle=solid,fillcolor=curcolor]
{
\newpath
\moveto(684.04578109,436.50637938)
\curveto(683.61697455,436.1418975)(683.20318137,435.88461676)(682.80440031,435.73453641)
\curveto(682.40989911,435.58445591)(681.9853859,435.50941569)(681.53085941,435.50941554)
\curveto(680.78045446,435.50941569)(680.20371682,435.69165621)(679.80064475,436.05613764)
\curveto(679.39757052,436.4249063)(679.19603395,436.89444364)(679.19603442,437.46475107)
\curveto(679.19603395,437.79921422)(679.27107416,438.10366309)(679.42115529,438.37809858)
\curveto(679.57552303,438.65681666)(679.7749156,438.8797933)(680.01933359,439.04702916)
\curveto(680.268037,439.21425825)(680.5467578,439.34075461)(680.85549681,439.42651862)
\curveto(681.08275932,439.48654703)(681.4258003,439.54443519)(681.88462077,439.60018329)
\curveto(682.81940427,439.71166767)(683.50763023,439.84459605)(683.94930072,439.99896882)
\curveto(683.9535835,440.15762094)(683.95572751,440.25838923)(683.95573275,440.30127399)
\curveto(683.95572751,440.77295069)(683.8463832,441.10527164)(683.62769948,441.29823782)
\curveto(683.33182173,441.55980094)(682.89230048,441.69058531)(682.30913441,441.69059133)
\curveto(681.76455327,441.69058531)(681.36148012,441.59410503)(681.09991375,441.40115022)
\curveto(680.84263064,441.21247195)(680.6518141,440.87586299)(680.52746355,440.39132233)
\lineto(679.39542719,440.54569093)
\curveto(679.49833881,441.03023143)(679.66771529,441.42044054)(679.90355715,441.71631943)
\curveto(680.13939663,442.01647424)(680.48029361,442.24588289)(680.92624908,442.40454608)
\curveto(681.37220015,442.56748381)(681.88890562,442.64895604)(682.47636705,442.64896302)
\curveto(683.05953296,442.64895604)(683.53335831,442.58034784)(683.89784452,442.44313823)
\curveto(684.26232038,442.30591506)(684.53032115,442.13225057)(684.70184762,441.92214423)
\curveto(684.87336213,441.71631338)(684.99342647,441.45474464)(685.06204101,441.13743721)
\curveto(685.10062677,440.94018317)(685.11992283,440.58427816)(685.11992923,440.06972109)
\lineto(685.11992923,438.52603515)
\curveto(685.11992283,437.44974122)(685.1435069,436.76794728)(685.1906815,436.48065128)
\curveto(685.24213118,436.19764165)(685.34075546,435.92535288)(685.48655464,435.66378413)
\lineto(684.27733398,435.66378413)
\curveto(684.15726409,435.90391282)(684.08007987,436.18477762)(684.04578109,436.50637938)
\moveto(683.94930072,439.09205333)
\curveto(683.52907029,438.92052941)(682.8987325,438.774737)(682.05828544,438.65467565)
\curveto(681.58231275,438.58606446)(681.24570379,438.50888024)(681.04845755,438.42312276)
\curveto(680.85120666,438.33735975)(680.69898223,438.21086339)(680.5917838,438.04363329)
\curveto(680.48458162,437.88068645)(680.43098147,437.69844593)(680.43098318,437.49691119)
\curveto(680.43098147,437.18817248)(680.5467578,436.93089174)(680.77831251,436.72506822)
\curveto(681.01415113,436.51924257)(681.35719211,436.41633028)(681.80743648,436.41633103)
\curveto(682.25338666,436.41633028)(682.65002779,436.51281055)(682.99736106,436.70577214)
\curveto(683.34468577,436.90301966)(683.5998225,437.17102043)(683.762772,437.50977524)
\curveto(683.88711932,437.77134214)(683.94929549,438.15726324)(683.94930072,438.6675397)
\lineto(683.94930072,439.09205333)
}
}
{
\newrgbcolor{curcolor}{0 0 0}
\pscustom[linestyle=none,fillstyle=solid,fillcolor=curcolor]
{
\newpath
\moveto(686.92089507,433.04595005)
\lineto(686.92089507,442.49459443)
\lineto(687.97574713,442.49459443)
\lineto(687.97574713,441.60697501)
\curveto(688.22444991,441.95429806)(688.50531471,442.2137228)(688.81834237,442.38525001)
\curveto(689.1313645,442.56105179)(689.51085358,442.64895604)(689.95681075,442.64896302)
\curveto(690.53997651,442.64895604)(691.05453798,442.49887561)(691.5004967,442.19872129)
\curveto(691.94644452,441.8985539)(692.28305348,441.47404069)(692.51032458,440.92518039)
\curveto(692.73758277,440.38059757)(692.8512151,439.78241987)(692.8512219,439.13064548)
\curveto(692.8512151,438.43169602)(692.72471874,437.80135822)(692.47173244,437.2396302)
\curveto(692.22302131,436.68218704)(691.85854027,436.25338581)(691.37828823,435.95322525)
\curveto(690.90231354,435.65735211)(690.40061611,435.50941569)(689.87319443,435.50941554)
\curveto(689.48726951,435.50941569)(689.13994052,435.59088793)(688.83120642,435.75383248)
\curveto(688.52675477,435.91677685)(688.27590606,436.12260144)(688.07865952,436.37130686)
\lineto(688.07865952,433.04595005)
\lineto(686.92089507,433.04595005)
\moveto(687.9693151,439.04059713)
\curveto(687.96931319,438.16155125)(688.14726569,437.5119174)(688.50317316,437.09169363)
\curveto(688.85907572,436.671467)(689.29002095,436.46135441)(689.79601014,436.4613552)
\curveto(690.31056786,436.46135441)(690.75008911,436.67789902)(691.11457521,437.1109897)
\curveto(691.4833392,437.5483655)(691.66772373,438.22372743)(691.66772934,439.1370775)
\curveto(691.66772373,440.00754051)(691.48762721,440.65931837)(691.12743926,441.09241303)
\curveto(690.77152917,441.52549684)(690.34487196,441.74204145)(689.84746633,441.74204753)
\curveto(689.35434113,441.74204145)(688.91696389,441.51048879)(688.53533328,441.04738886)
\curveto(688.15798572,440.58856617)(687.96931319,439.91963626)(687.9693151,439.04059713)
}
}
{
\newrgbcolor{curcolor}{0 0 0}
\pscustom[linestyle=none,fillstyle=solid,fillcolor=curcolor]
{
\newpath
\moveto(694.25340315,433.04595005)
\lineto(694.25340315,442.49459443)
\lineto(695.30825521,442.49459443)
\lineto(695.30825521,441.60697501)
\curveto(695.556958,441.95429806)(695.8378228,442.2137228)(696.15085046,442.38525001)
\curveto(696.46387258,442.56105179)(696.84336167,442.64895604)(697.28931884,442.64896302)
\curveto(697.8724846,442.64895604)(698.38704607,442.49887561)(698.83300478,442.19872129)
\curveto(699.27895261,441.8985539)(699.61556157,441.47404069)(699.84283267,440.92518039)
\curveto(700.07009086,440.38059757)(700.18372319,439.78241987)(700.18372998,439.13064548)
\curveto(700.18372319,438.43169602)(700.05722683,437.80135822)(699.80424052,437.2396302)
\curveto(699.5555294,436.68218704)(699.19104836,436.25338581)(698.71079631,435.95322525)
\curveto(698.23482163,435.65735211)(697.7331242,435.50941569)(697.20570252,435.50941554)
\curveto(696.8197776,435.50941569)(696.47244861,435.59088793)(696.16371451,435.75383248)
\curveto(695.85926286,435.91677685)(695.60841415,436.12260144)(695.41116761,436.37130686)
\lineto(695.41116761,433.04595005)
\lineto(694.25340315,433.04595005)
\moveto(695.30182319,439.04059713)
\curveto(695.30182127,438.16155125)(695.47977378,437.5119174)(695.83568125,437.09169363)
\curveto(696.19158381,436.671467)(696.62252904,436.46135441)(697.12851822,436.4613552)
\curveto(697.64307595,436.46135441)(698.0825972,436.67789902)(698.4470833,437.1109897)
\curveto(698.81584729,437.5483655)(699.00023181,438.22372743)(699.00023743,439.1370775)
\curveto(699.00023181,440.00754051)(698.8201353,440.65931837)(698.45994735,441.09241303)
\curveto(698.10403726,441.52549684)(697.67738004,441.74204145)(697.17997442,441.74204753)
\curveto(696.68684922,441.74204145)(696.24947197,441.51048879)(695.86784137,441.04738886)
\curveto(695.49049381,440.58856617)(695.30182127,439.91963626)(695.30182319,439.04059713)
}
}
{
\newrgbcolor{curcolor}{0 0 0}
\pscustom[linestyle=none,fillstyle=solid,fillcolor=curcolor]
{
\newpath
\moveto(707.68347072,439.78671201)
\lineto(701.43797467,437.11742173)
\lineto(701.43797467,438.26875416)
\lineto(706.38420171,440.32057006)
\lineto(701.43797467,442.35308989)
\lineto(701.43797467,443.50442232)
\lineto(707.68347072,440.86729217)
\lineto(707.68347072,439.78671201)
}
}
{
\newrgbcolor{curcolor}{0 0 0}
\pscustom[linestyle=none,fillstyle=solid,fillcolor=curcolor]
{
\newpath
\moveto(715.37617396,439.78671201)
\lineto(709.13067792,437.11742173)
\lineto(709.13067792,438.26875416)
\lineto(714.07690496,440.32057006)
\lineto(709.13067792,442.35308989)
\lineto(709.13067792,443.50442232)
\lineto(715.37617396,440.86729217)
\lineto(715.37617396,439.78671201)
}
}
{
\newrgbcolor{curcolor}{0 0 0}
\pscustom[linestyle=none,fillstyle=solid,fillcolor=curcolor]
{
\newpath
\moveto(663.64339815,417.85347133)
\lineto(665.45722913,418.13004839)
\curveto(665.53441123,417.77842916)(665.69092367,417.5104284)(665.92676694,417.3260453)
\curveto(666.16260502,417.14594736)(666.49278196,417.05589911)(666.91729875,417.05590026)
\curveto(667.3846885,417.05589911)(667.7363055,417.14165935)(667.97215081,417.31318125)
\curveto(668.13080263,417.43324418)(668.21013085,417.59404464)(668.21013573,417.7955831)
\curveto(668.21013085,417.9327976)(668.16725073,418.04642993)(668.08149523,418.13648042)
\curveto(667.99144223,418.22223843)(667.78990565,418.30156666)(667.4768849,418.37446533)
\curveto(666.01895661,418.69606378)(665.09488997,418.98979262)(664.70468223,419.25565273)
\curveto(664.16439132,419.62441843)(663.89424655,420.13683589)(663.89424711,420.79290664)
\curveto(663.89424655,421.38464744)(664.12794322,421.88205686)(664.59533781,422.28513639)
\curveto(665.06272988,422.68820316)(665.78740395,422.88973973)(666.76936218,422.88974672)
\curveto(667.70414541,422.88973973)(668.39880339,422.7375153)(668.8533382,422.43307296)
\curveto(669.30786198,422.12861756)(669.62088687,421.67837628)(669.79241382,421.08234776)
\lineto(668.08792726,420.76717854)
\curveto(668.0150263,421.03303044)(667.8756659,421.23671102)(667.66984565,421.3782209)
\curveto(667.46830474,421.51971983)(667.17886391,421.59047203)(666.8015223,421.59047771)
\curveto(666.32554948,421.59047203)(665.98465251,421.52400784)(665.77883037,421.39108495)
\curveto(665.64161153,421.29674319)(665.57300334,421.17453484)(665.57300557,421.02445953)
\curveto(665.57300334,420.89581405)(665.63303551,420.78646974)(665.75310227,420.69642627)
\curveto(665.91604431,420.57635714)(666.47777391,420.40698066)(667.43829276,420.18829632)
\curveto(668.4030914,419.96960341)(669.07630932,419.70160265)(669.45794853,419.38429322)
\curveto(669.83528748,419.06268882)(670.02396002,418.61459155)(670.02396671,418.04000005)
\curveto(670.02396002,417.41394813)(669.76239127,416.87580259)(669.23925969,416.42556183)
\curveto(668.71611629,415.97532003)(667.94213009,415.75019938)(666.91729875,415.75019923)
\curveto(665.98679652,415.75019938)(665.24925841,415.93887192)(664.70468223,416.31621741)
\curveto(664.16439132,416.69356207)(663.81063031,417.20597953)(663.64339815,417.85347133)
}
}
{
\newrgbcolor{curcolor}{0 0 0}
\pscustom[linestyle=none,fillstyle=solid,fillcolor=curcolor]
{
\newpath
\moveto(676.10866199,415.90456782)
\lineto(676.10866199,416.92725976)
\curveto(675.85995184,416.5627777)(675.53191891,416.27548088)(675.1245622,416.06536844)
\curveto(674.7214846,415.85525568)(674.29482738,415.75019938)(673.84458928,415.75019923)
\curveto(673.38576879,415.75019938)(672.97411962,415.85096767)(672.60964052,416.05250439)
\curveto(672.24515754,416.25404082)(671.98144479,416.53704963)(671.81850148,416.90153166)
\curveto(671.65555586,417.2660117)(671.57408363,417.76985314)(671.57408454,418.41305748)
\lineto(671.57408454,422.73537812)
\lineto(673.38148349,422.73537812)
\lineto(673.38148349,419.59655004)
\curveto(673.38148078,418.63603161)(673.41364087,418.04642993)(673.47796387,417.82774323)
\curveto(673.54656925,417.61334069)(673.6687776,417.44182021)(673.84458928,417.31318125)
\curveto(674.0203946,417.18882748)(674.24337124,417.12665131)(674.51351985,417.12665253)
\curveto(674.82225289,417.12665131)(675.09882967,417.21026755)(675.34325105,417.37750149)
\curveto(675.58766307,417.54902051)(675.75489554,417.75913311)(675.84494898,418.00783992)
\curveto(675.93499206,418.26083054)(675.98001618,418.87616029)(675.9800215,419.85383103)
\lineto(675.9800215,422.73537812)
\lineto(677.78742046,422.73537812)
\lineto(677.78742046,415.90456782)
\lineto(676.10866199,415.90456782)
}
}
{
\newrgbcolor{curcolor}{0 0 0}
\pscustom[linestyle=none,fillstyle=solid,fillcolor=curcolor]
{
\newpath
\moveto(679.58838698,415.90456782)
\lineto(679.58838698,425.33391613)
\lineto(681.39578594,425.33391613)
\lineto(681.39578594,421.93780705)
\curveto(681.95322485,422.57242683)(682.61357873,422.88973973)(683.37684956,422.88974672)
\curveto(684.20871928,422.88973973)(684.89694524,422.58743487)(685.44152951,421.98283122)
\curveto(685.98610034,421.38250344)(686.25838912,420.51846897)(686.25839666,419.39072525)
\curveto(686.25838912,418.22438244)(685.97966832,417.32604388)(685.42223344,416.69570687)
\curveto(684.86907316,416.06536828)(684.19585524,415.75019938)(683.40257766,415.75019923)
\curveto(683.01236387,415.75019938)(682.62644277,415.84667966)(682.24481321,416.03964034)
\curveto(681.86746461,416.23688877)(681.54157568,416.5263296)(681.26714544,416.90796369)
\lineto(681.26714544,415.90456782)
\lineto(679.58838698,415.90456782)
\moveto(681.38292189,419.46790954)
\curveto(681.38291923,418.76038396)(681.49440754,418.23724647)(681.71738718,417.8984955)
\curveto(682.03040907,417.41823614)(682.44634626,417.17810745)(682.96519998,417.17810873)
\curveto(683.36398087,417.17810745)(683.70273384,417.34748394)(683.98145989,417.68623868)
\curveto(684.26446344,418.02927788)(684.40596784,418.56742341)(684.40597353,419.3006769)
\curveto(684.40596784,420.08109173)(684.26446344,420.64282133)(683.98145989,420.98586739)
\curveto(683.69844582,421.33319129)(683.33610879,421.50685579)(682.89444771,421.50686139)
\curveto(682.4613543,421.50685579)(682.10116127,421.33747931)(681.81386755,420.99873144)
\curveto(681.52656763,420.66426139)(681.38291923,420.15398793)(681.38292189,419.46790954)
}
}
{
\newrgbcolor{curcolor}{0 0 0}
\pscustom[linestyle=none,fillstyle=solid,fillcolor=curcolor]
{
\newpath
\moveto(687.67987372,423.66158969)
\lineto(687.67987372,425.33391613)
\lineto(689.48727268,425.33391613)
\lineto(689.48727268,423.66158969)
\lineto(687.67987372,423.66158969)
\moveto(689.48727268,422.73537812)
\lineto(689.48727268,416.11682464)
\curveto(689.48726996,415.24635795)(689.4293818,414.6331722)(689.31360801,414.27726556)
\curveto(689.20211715,413.91707416)(688.98342853,413.63620936)(688.65754148,413.43467032)
\curveto(688.33593868,413.23313621)(687.92428951,413.13236792)(687.42259273,413.13236515)
\curveto(687.24249557,413.13236792)(687.04739101,413.14951997)(686.83727848,413.18382135)
\curveto(686.63145383,413.21384016)(686.40847719,413.26100829)(686.1683479,413.32532589)
\lineto(686.48351711,414.86901184)
\curveto(686.56927765,414.85186082)(686.65074988,414.83899679)(686.72793406,414.83041969)
\curveto(686.80083031,414.81755673)(686.8694385,414.81112471)(686.93375885,414.81112361)
\curveto(687.11814321,414.81112471)(687.26822364,414.85186082)(687.38400058,414.93333209)
\curveto(687.50406431,415.01051728)(687.58339254,415.10485355)(687.6219855,415.21634117)
\curveto(687.66057676,415.32783018)(687.67987281,415.66229513)(687.67987372,416.21973704)
\lineto(687.67987372,422.73537812)
\lineto(689.48727268,422.73537812)
}
}
{
\newrgbcolor{curcolor}{0 0 0}
\pscustom[linestyle=none,fillstyle=solid,fillcolor=curcolor]
{
\newpath
\moveto(695.32755198,418.07859219)
\lineto(697.12851892,417.77628703)
\curveto(696.89695955,417.11593128)(696.53033451,416.61208984)(696.02864268,416.26476121)
\curveto(695.53122766,415.92171987)(694.90732188,415.75019938)(694.15692348,415.75019923)
\curveto(692.96914036,415.75019938)(692.09009785,416.13826449)(691.51979332,416.91439571)
\curveto(691.06955095,417.53615647)(690.8444303,418.32086271)(690.84443072,419.26851677)
\curveto(690.8444303,420.40054864)(691.14030315,421.28602316)(691.73205014,421.924943)
\curveto(692.32379452,422.56813881)(693.07205265,422.88973973)(693.97682678,422.88974672)
\curveto(694.99308213,422.88973973)(695.79494041,422.55313077)(696.38240404,421.87991883)
\curveto(696.96985576,421.21098295)(697.25072056,420.18400402)(697.22499929,418.79897897)
\lineto(692.69685385,418.79897897)
\curveto(692.70971562,418.26297455)(692.85550804,417.84489335)(693.13423154,417.54473414)
\curveto(693.41294962,417.24885966)(693.76027861,417.10092323)(694.17621955,417.10092443)
\curveto(694.45922461,417.10092323)(694.69720928,417.17810745)(694.8901743,417.33247732)
\curveto(695.08313038,417.48684433)(695.2289228,417.73554904)(695.32755198,418.07859219)
\moveto(695.43046438,419.90528723)
\curveto(695.41759534,420.42842072)(695.28252295,420.82506185)(695.02524682,421.09521181)
\curveto(694.76796149,421.3696394)(694.45493659,421.50685579)(694.0861712,421.50686139)
\curveto(693.69167042,421.50685579)(693.36578149,421.36320738)(693.10850344,421.07591573)
\curveto(692.85122002,420.78861374)(692.72472366,420.39840463)(692.72901398,419.90528723)
\lineto(695.43046438,419.90528723)
}
}
{
\newrgbcolor{curcolor}{0 0 0}
\pscustom[linestyle=none,fillstyle=solid,fillcolor=curcolor]
{
\newpath
\moveto(704.66041786,420.71572235)
\lineto(702.878747,420.39412111)
\curveto(702.81870971,420.75002163)(702.68149332,421.0180224)(702.46709742,421.1981242)
\curveto(702.25698011,421.37821542)(701.98254733,421.46826368)(701.64379825,421.46826924)
\curveto(701.19355308,421.46826368)(700.83336005,421.31175123)(700.56321809,420.99873144)
\curveto(700.29735853,420.68998946)(700.16443015,420.17113998)(700.16443255,419.44218144)
\curveto(700.16443015,418.6317436)(700.29950253,418.05929396)(700.56965011,417.72483083)
\curveto(700.84408009,417.39036406)(701.21070513,417.22313158)(701.66952635,417.2231329)
\curveto(702.01256342,417.22313158)(702.29342822,417.31961186)(702.51212159,417.51257401)
\curveto(702.73080546,417.70982097)(702.8851739,418.04642993)(702.97522737,418.5224019)
\lineto(704.75046621,418.22009674)
\curveto(704.56607469,417.4053721)(704.21231368,416.79004235)(703.68918212,416.37410563)
\curveto(703.1660387,415.95816798)(702.4649487,415.75019938)(701.58591003,415.75019923)
\curveto(700.58679935,415.75019938)(699.78922908,416.06536828)(699.19319681,416.69570687)
\curveto(698.60144969,417.32604388)(698.30557685,418.19865436)(698.3055774,419.31354095)
\curveto(698.30557685,420.44128475)(698.6035937,421.31818325)(699.19962884,421.94423908)
\curveto(699.7956611,422.57457083)(700.60180739,422.88973973)(701.61807015,422.88974672)
\curveto(702.44994066,422.88973973)(703.11029454,422.70964322)(703.59913378,422.34945664)
\curveto(704.09224934,421.99354518)(704.44601035,421.44896762)(704.66041786,420.71572235)
}
}
{
\newrgbcolor{curcolor}{0 0 0}
\pscustom[linestyle=none,fillstyle=solid,fillcolor=curcolor]
{
\newpath
\moveto(709.16926708,422.73537812)
\lineto(709.16926708,421.29460457)
\lineto(707.93431832,421.29460457)
\lineto(707.93431832,418.54169798)
\curveto(707.93431548,417.98425375)(707.94503551,417.65836482)(707.96647845,417.56403021)
\curveto(707.99220365,417.4739803)(708.0458038,417.39894008)(708.12727907,417.33890935)
\curveto(708.21303628,417.27887574)(708.31594857,417.24885966)(708.43601626,417.248861)
\curveto(708.60324539,417.24885966)(708.84551808,417.30674782)(709.16283505,417.42252567)
\lineto(709.31720365,416.02034427)
\curveto(708.89697423,415.84024764)(708.42100487,415.75019938)(707.88929415,415.75019923)
\curveto(707.56340242,415.75019938)(707.26967359,415.80379954)(707.00810676,415.91099985)
\curveto(706.7465361,416.02248816)(706.55357555,416.16399256)(706.42922453,416.33551348)
\curveto(706.30915885,416.51132155)(706.22554261,416.74716223)(706.17837556,417.04303621)
\curveto(706.13978237,417.25314767)(706.12048631,417.67766088)(706.12048734,418.31657711)
\lineto(706.12048734,421.29460457)
\lineto(705.29075615,421.29460457)
\lineto(705.29075615,422.73537812)
\lineto(706.12048734,422.73537812)
\lineto(706.12048734,424.09253535)
\lineto(707.93431832,425.14738741)
\lineto(707.93431832,422.73537812)
\lineto(709.16926708,422.73537812)
}
}
{
\newrgbcolor{curcolor}{0 0 0}
\pscustom[linestyle=none,fillstyle=solid,fillcolor=curcolor]
{
\newpath
\moveto(709.79960618,417.85347133)
\lineto(711.61343717,418.13004839)
\curveto(711.69061926,417.77842916)(711.84713171,417.5104284)(712.08297497,417.3260453)
\curveto(712.31881305,417.14594736)(712.64899,417.05589911)(713.07350679,417.05590026)
\curveto(713.54089654,417.05589911)(713.89251354,417.14165935)(714.12835885,417.31318125)
\curveto(714.28701066,417.43324418)(714.36633889,417.59404464)(714.36634376,417.7955831)
\curveto(714.36633889,417.9327976)(714.32345877,418.04642993)(714.23770327,418.13648042)
\curveto(714.14765027,418.22223843)(713.94611369,418.30156666)(713.63309294,418.37446533)
\curveto(712.17516465,418.69606378)(711.25109801,418.98979262)(710.86089027,419.25565273)
\curveto(710.32059936,419.62441843)(710.05045459,420.13683589)(710.05045515,420.79290664)
\curveto(710.05045459,421.38464744)(710.28415126,421.88205686)(710.75154585,422.28513639)
\curveto(711.21893792,422.68820316)(711.94361199,422.88973973)(712.92557022,422.88974672)
\curveto(713.86035345,422.88973973)(714.55501143,422.7375153)(715.00954624,422.43307296)
\curveto(715.46407002,422.12861756)(715.77709491,421.67837628)(715.94862186,421.08234776)
\lineto(714.24413529,420.76717854)
\curveto(714.17123433,421.03303044)(714.03187394,421.23671102)(713.82605368,421.3782209)
\curveto(713.62451278,421.51971983)(713.33507195,421.59047203)(712.95773034,421.59047771)
\curveto(712.48175752,421.59047203)(712.14086055,421.52400784)(711.9350384,421.39108495)
\curveto(711.79781957,421.29674319)(711.72921137,421.17453484)(711.72921361,421.02445953)
\curveto(711.72921137,420.89581405)(711.78924355,420.78646974)(711.90931031,420.69642627)
\curveto(712.07225235,420.57635714)(712.63398195,420.40698066)(713.59450079,420.18829632)
\curveto(714.55929944,419.96960341)(715.23251736,419.70160265)(715.61415657,419.38429322)
\curveto(715.99149552,419.06268882)(716.18016806,418.61459155)(716.18017475,418.04000005)
\curveto(716.18016806,417.41394813)(715.91859931,416.87580259)(715.39546773,416.42556183)
\curveto(714.87232433,415.97532003)(714.09833813,415.75019938)(713.07350679,415.75019923)
\curveto(712.14300455,415.75019938)(711.40546645,415.93887192)(710.86089027,416.31621741)
\curveto(710.32059936,416.69356207)(709.96683835,417.20597953)(709.79960618,417.85347133)
}
}
{
\newrgbcolor{curcolor}{0 1 0.25098041}
\pscustom[linewidth=2.63455725,linecolor=curcolor]
{
\newpath
\moveto(770.35329,284.70119456)
\lineto(854.65912,284.70119456)
\lineto(854.65912,337.39235456)
\lineto(799.33342,337.39235456)
\lineto(799.33342,347.93057456)
\lineto(770.35329,347.93057456)
\lineto(770.35329,284.70119456)
\closepath
}
}
{
\newrgbcolor{curcolor}{0 0 0}
\pscustom[linestyle=none,fillstyle=solid,fillcolor=curcolor]
{
\newpath
\moveto(785.56374048,320.38215031)
\lineto(785.56374048,321.46273047)
\lineto(791.80923652,324.09986062)
\lineto(791.80923652,322.94852819)
\lineto(786.85657745,320.91600836)
\lineto(791.80923652,318.86419246)
\lineto(791.80923652,317.71286003)
\lineto(785.56374048,320.38215031)
}
}
{
\newrgbcolor{curcolor}{0 0 0}
\pscustom[linestyle=none,fillstyle=solid,fillcolor=curcolor]
{
\newpath
\moveto(793.25644229,320.38215031)
\lineto(793.25644229,321.46273047)
\lineto(799.50193834,324.09986062)
\lineto(799.50193834,322.94852819)
\lineto(794.54927927,320.91600836)
\lineto(799.50193834,318.86419246)
\lineto(799.50193834,317.71286003)
\lineto(793.25644229,320.38215031)
}
}
{
\newrgbcolor{curcolor}{0 0 0}
\pscustom[linestyle=none,fillstyle=solid,fillcolor=curcolor]
{
\newpath
\moveto(805.55447384,317.10181768)
\curveto(805.12566729,316.7373358)(804.71187411,316.48005506)(804.31309306,316.32997471)
\curveto(803.91859185,316.17989421)(803.49407864,316.10485399)(803.03955216,316.10485384)
\curveto(802.28914721,316.10485399)(801.71240956,316.28709451)(801.3093375,316.65157594)
\curveto(800.90626327,317.0203446)(800.70472669,317.48988194)(800.70472717,318.06018937)
\curveto(800.70472669,318.39465252)(800.77976691,318.69910139)(800.92984804,318.97353688)
\curveto(801.08421577,319.25225496)(801.28360834,319.4752316)(801.52802634,319.64246746)
\curveto(801.77672975,319.80969655)(802.05545054,319.93619291)(802.36418956,320.02195692)
\curveto(802.59145207,320.08198533)(802.93449305,320.13987349)(803.39331352,320.19562159)
\curveto(804.32809702,320.30710597)(805.01632298,320.44003435)(805.45799347,320.59440712)
\curveto(805.46227625,320.75305924)(805.46442026,320.85382753)(805.46442549,320.89671229)
\curveto(805.46442026,321.36838899)(805.35507595,321.70070994)(805.13639223,321.89367613)
\curveto(804.84051448,322.15523924)(804.40099323,322.28602361)(803.81782715,322.28602964)
\curveto(803.27324601,322.28602361)(802.87017286,322.18954333)(802.6086065,321.99658852)
\curveto(802.35132339,321.80791025)(802.16050684,321.47130129)(802.03615629,320.98676063)
\lineto(800.90411994,321.14112923)
\curveto(801.00703155,321.62566973)(801.17640804,322.01587884)(801.41224989,322.31175773)
\curveto(801.64808938,322.61191254)(801.98898635,322.84132119)(802.43494183,322.99998438)
\curveto(802.88089289,323.16292211)(803.39759837,323.24439434)(803.9850598,323.24440133)
\curveto(804.5682257,323.24439434)(805.04205105,323.17578614)(805.40653727,323.03857653)
\curveto(805.77101313,322.90135336)(806.03901389,322.72768887)(806.21054036,322.51758253)
\curveto(806.38205487,322.31175168)(806.50211921,322.05018294)(806.57073375,321.73287551)
\curveto(806.60931952,321.53562147)(806.62861557,321.17971646)(806.62862197,320.6651594)
\lineto(806.62862197,319.12147345)
\curveto(806.62861557,318.04517952)(806.65219964,317.36338558)(806.69937425,317.07608958)
\curveto(806.75082392,316.79307995)(806.8494482,316.52079118)(806.99524739,316.25922243)
\lineto(805.78602673,316.25922243)
\curveto(805.66595683,316.49935112)(805.58877261,316.78021592)(805.55447384,317.10181768)
\moveto(805.45799347,319.68749163)
\curveto(805.03776304,319.51596771)(804.40742524,319.3701753)(803.56697819,319.25011395)
\curveto(803.09100549,319.18150276)(802.75439653,319.10431854)(802.5571503,319.01856106)
\curveto(802.35989941,318.93279805)(802.20767498,318.80630169)(802.10047654,318.6390716)
\curveto(801.99327436,318.47612475)(801.93967421,318.29388423)(801.93967592,318.09234949)
\curveto(801.93967421,317.78361078)(802.05545054,317.52633004)(802.28700526,317.32050652)
\curveto(802.52284387,317.11468087)(802.86588485,317.01176858)(803.31612922,317.01176933)
\curveto(803.76207941,317.01176858)(804.15872054,317.10824885)(804.5060538,317.30121045)
\curveto(804.85337852,317.49845797)(805.10851524,317.76645873)(805.27146475,318.10521354)
\curveto(805.39581206,318.36678044)(805.45798824,318.75270154)(805.45799347,319.262978)
\lineto(805.45799347,319.68749163)
}
}
{
\newrgbcolor{curcolor}{0 0 0}
\pscustom[linestyle=none,fillstyle=solid,fillcolor=curcolor]
{
\newpath
\moveto(808.42958781,313.64138836)
\lineto(808.42958781,323.09003273)
\lineto(809.48443987,323.09003273)
\lineto(809.48443987,322.20241331)
\curveto(809.73314266,322.54973636)(810.01400746,322.8091611)(810.32703512,322.98068831)
\curveto(810.64005724,323.15649009)(811.01954633,323.24439434)(811.4655035,323.24440133)
\curveto(812.04866926,323.24439434)(812.56323072,323.09431391)(813.00918944,322.79415959)
\curveto(813.45513727,322.4939922)(813.79174623,322.06947899)(814.01901733,321.52061869)
\curveto(814.24627552,320.97603588)(814.35990784,320.37785817)(814.35991464,319.72608378)
\curveto(814.35990784,319.02713432)(814.23341148,318.39679653)(813.98042518,317.8350685)
\curveto(813.73171405,317.27762534)(813.36723302,316.84882411)(812.88698097,316.54866355)
\curveto(812.41100629,316.25279042)(811.90930886,316.10485399)(811.38188718,316.10485384)
\curveto(810.99596226,316.10485399)(810.64863327,316.18632623)(810.33989917,316.34927078)
\curveto(810.03544752,316.51221515)(809.78459881,316.71803974)(809.58735227,316.96674516)
\lineto(809.58735227,313.64138836)
\lineto(808.42958781,313.64138836)
\moveto(809.47800785,319.63603543)
\curveto(809.47800593,318.75698955)(809.65595844,318.1073557)(810.0118659,317.68713193)
\curveto(810.36776847,317.26690531)(810.7987137,317.05679271)(811.30470288,317.0567935)
\curveto(811.8192606,317.05679271)(812.25878186,317.27333732)(812.62326796,317.70642801)
\curveto(812.99203195,318.1438038)(813.17641647,318.81916573)(813.17642209,319.73251581)
\curveto(813.17641647,320.60297881)(812.99631996,321.25475667)(812.63613201,321.68785133)
\curveto(812.28022192,322.12093514)(811.8535647,322.33747976)(811.35615908,322.33748583)
\curveto(810.86303388,322.33747976)(810.42565663,322.1059271)(810.04402603,321.64282716)
\curveto(809.66667847,321.18400447)(809.47800593,320.51507456)(809.47800785,319.63603543)
}
}
{
\newrgbcolor{curcolor}{0 0 0}
\pscustom[linestyle=none,fillstyle=solid,fillcolor=curcolor]
{
\newpath
\moveto(815.7620959,313.64138836)
\lineto(815.7620959,323.09003273)
\lineto(816.81694796,323.09003273)
\lineto(816.81694796,322.20241331)
\curveto(817.06565075,322.54973636)(817.34651555,322.8091611)(817.6595432,322.98068831)
\curveto(817.97256533,323.15649009)(818.35205441,323.24439434)(818.79801159,323.24440133)
\curveto(819.38117735,323.24439434)(819.89573881,323.09431391)(820.34169753,322.79415959)
\curveto(820.78764535,322.4939922)(821.12425431,322.06947899)(821.35152542,321.52061869)
\curveto(821.57878361,320.97603588)(821.69241593,320.37785817)(821.69242273,319.72608378)
\curveto(821.69241593,319.02713432)(821.56591957,318.39679653)(821.31293327,317.8350685)
\curveto(821.06422214,317.27762534)(820.6997411,316.84882411)(820.21948906,316.54866355)
\curveto(819.74351438,316.25279042)(819.24181695,316.10485399)(818.71439527,316.10485384)
\curveto(818.32847035,316.10485399)(817.98114136,316.18632623)(817.67240725,316.34927078)
\curveto(817.36795561,316.51221515)(817.11710689,316.71803974)(816.91986036,316.96674516)
\lineto(816.91986036,313.64138836)
\lineto(815.7620959,313.64138836)
\moveto(816.81051594,319.63603543)
\curveto(816.81051402,318.75698955)(816.98846653,318.1073557)(817.34437399,317.68713193)
\curveto(817.70027655,317.26690531)(818.13122178,317.05679271)(818.63721097,317.0567935)
\curveto(819.15176869,317.05679271)(819.59128994,317.27333732)(819.95577604,317.70642801)
\curveto(820.32454003,318.1438038)(820.50892456,318.81916573)(820.50893017,319.73251581)
\curveto(820.50892456,320.60297881)(820.32882805,321.25475667)(819.96864009,321.68785133)
\curveto(819.61273001,322.12093514)(819.18607279,322.33747976)(818.68866717,322.33748583)
\curveto(818.19554197,322.33747976)(817.75816472,322.1059271)(817.37653412,321.64282716)
\curveto(816.99918656,321.18400447)(816.81051402,320.51507456)(816.81051594,319.63603543)
}
}
{
\newrgbcolor{curcolor}{0 0 0}
\pscustom[linestyle=none,fillstyle=solid,fillcolor=curcolor]
{
\newpath
\moveto(829.19216346,320.38215031)
\lineto(822.94666742,317.71286003)
\lineto(822.94666742,318.86419246)
\lineto(827.89289446,320.91600836)
\lineto(822.94666742,322.94852819)
\lineto(822.94666742,324.09986062)
\lineto(829.19216346,321.46273047)
\lineto(829.19216346,320.38215031)
}
}
{
\newrgbcolor{curcolor}{0 0 0}
\pscustom[linestyle=none,fillstyle=solid,fillcolor=curcolor]
{
\newpath
\moveto(836.88486671,320.38215031)
\lineto(830.63937066,317.71286003)
\lineto(830.63937066,318.86419246)
\lineto(835.58559771,320.91600836)
\lineto(830.63937066,322.94852819)
\lineto(830.63937066,324.09986062)
\lineto(836.88486671,321.46273047)
\lineto(836.88486671,320.38215031)
}
}
{
\newrgbcolor{curcolor}{0 0 0}
\pscustom[linestyle=none,fillstyle=solid,fillcolor=curcolor]
{
\newpath
\moveto(796.36059484,301.2468404)
\lineto(794.72042853,301.54271354)
\curveto(794.9048124,302.20306238)(795.2221253,302.69189577)(795.67236819,303.00921518)
\curveto(796.12260787,303.32652158)(796.79153778,303.48517803)(797.67915992,303.48518502)
\curveto(798.4853026,303.48517803)(799.08562431,303.38869776)(799.48012685,303.1957439)
\curveto(799.87461856,303.00706467)(800.15119535,302.76479198)(800.30985805,302.4689251)
\curveto(800.47279627,302.1773343)(800.5542685,301.63918877)(800.55427499,300.85448689)
\lineto(800.53497891,298.74478277)
\curveto(800.53497244,298.14445881)(800.56284452,297.70064955)(800.61859524,297.41335364)
\curveto(800.67862085,297.13034392)(800.78796516,296.82589505)(800.9466285,296.50000613)
\lineto(799.15852562,296.50000613)
\curveto(799.11135239,296.62007047)(799.05346422,296.79802297)(798.98486095,297.03386418)
\curveto(798.95483994,297.14106395)(798.93339988,297.21181615)(798.9205407,297.246121)
\curveto(798.61179896,296.9459594)(798.28162202,296.72083875)(797.93000889,296.5707584)
\curveto(797.57838802,296.4206779)(797.20318695,296.34563769)(796.80440455,296.34563753)
\curveto(796.10116781,296.34563769)(795.54587023,296.53645423)(795.13851014,296.91808773)
\curveto(794.73543592,297.2997204)(794.53389934,297.78212178)(794.53389981,298.36529331)
\curveto(794.53389934,298.75121254)(794.6260916,299.09425352)(794.81047688,299.39441727)
\curveto(794.99486066,299.69886324)(795.25214139,299.9304159)(795.58231985,300.08907594)
\curveto(795.91678328,300.25201682)(796.39704065,300.39352122)(797.02309339,300.51358958)
\curveto(797.86782884,300.67224202)(798.45314251,300.82017844)(798.77903615,300.95739929)
\lineto(798.77903615,301.13749598)
\curveto(798.77903144,301.48482033)(798.6932712,301.73138103)(798.52175516,301.87717883)
\curveto(798.35023022,302.02725388)(798.0264853,302.10229409)(797.55051942,302.10229969)
\curveto(797.22891502,302.10229409)(796.97806631,302.03797391)(796.79797253,301.90933895)
\curveto(796.61787328,301.78498119)(796.47208087,301.56414856)(796.36059484,301.2468404)
\moveto(798.77903615,299.78033875)
\curveto(798.54747878,299.70315125)(798.18085374,299.61095899)(797.67915992,299.50376169)
\curveto(797.17745888,299.39655838)(796.84942594,299.29150208)(796.69506013,299.18859248)
\curveto(796.45921683,299.02135731)(796.34129649,298.80910071)(796.34129877,298.55182202)
\curveto(796.34129649,298.29882725)(796.43563276,298.08013863)(796.62430786,297.8957555)
\curveto(796.81297784,297.71136958)(797.05310652,297.61917731)(797.34469463,297.61917843)
\curveto(797.67058028,297.61917731)(797.98146117,297.72637762)(798.27733822,297.94077967)
\curveto(798.49602263,298.1037227)(798.63967104,298.30311526)(798.70828388,298.53895797)
\curveto(798.75544737,298.69332438)(798.77903144,298.98705321)(798.77903615,299.42014537)
\lineto(798.77903615,299.78033875)
}
}
{
\newrgbcolor{curcolor}{0 0 0}
\pscustom[linestyle=none,fillstyle=solid,fillcolor=curcolor]
{
\newpath
\moveto(804.07259239,296.50000613)
\lineto(802.26519343,296.50000613)
\lineto(802.26519343,303.33081642)
\lineto(803.9439519,303.33081642)
\lineto(803.9439519,302.35958068)
\curveto(804.23124617,302.81839213)(804.4885269,303.12069699)(804.71579487,303.26649617)
\curveto(804.94734421,303.41228182)(805.20891295,303.48517803)(805.50050189,303.48518502)
\curveto(805.91214696,303.48517803)(806.30878809,303.37154571)(806.69042647,303.1442877)
\lineto(806.13084032,301.56844164)
\curveto(805.82638671,301.76568513)(805.54337791,301.86430941)(805.28181305,301.86431478)
\curveto(805.02881644,301.86430941)(804.81441583,301.79355721)(804.63861057,301.65205796)
\curveto(804.46279883,301.51483642)(804.32343843,301.2639877)(804.22052896,300.89951106)
\curveto(804.12190186,300.53502562)(804.07258972,299.77175945)(804.07259239,298.60971025)
\lineto(804.07259239,296.50000613)
}
}
{
\newrgbcolor{curcolor}{0 0 0}
\pscustom[linestyle=none,fillstyle=solid,fillcolor=curcolor]
{
\newpath
\moveto(811.41796508,298.67403049)
\lineto(813.21893201,298.37172533)
\curveto(812.98737265,297.71136958)(812.62074761,297.20752814)(812.11905578,296.86019951)
\curveto(811.62164076,296.51715817)(810.99773498,296.34563769)(810.24733657,296.34563753)
\curveto(809.05955346,296.34563769)(808.18051095,296.73370279)(807.61020642,297.50983401)
\curveto(807.15996404,298.13159477)(806.9348434,298.91630101)(806.93484382,299.86395508)
\curveto(806.9348434,300.99598694)(807.23071625,301.88146146)(807.82246324,302.5203813)
\curveto(808.41420762,303.16357711)(809.16246575,303.48517803)(810.06723988,303.48518502)
\curveto(811.08349523,303.48517803)(811.88535351,303.14856907)(812.47281714,302.47535713)
\curveto(813.06026886,301.80642125)(813.34113366,300.77944232)(813.31541238,299.39441727)
\lineto(808.78726695,299.39441727)
\curveto(808.80012872,298.85841285)(808.94592113,298.44033165)(809.22464464,298.14017244)
\curveto(809.50336272,297.84429796)(809.85069171,297.69636153)(810.26663265,297.69636273)
\curveto(810.5496377,297.69636153)(810.78762238,297.77354575)(810.9805874,297.92791562)
\curveto(811.17354348,298.08228263)(811.3193359,298.33098734)(811.41796508,298.67403049)
\moveto(811.52087748,300.50072553)
\curveto(811.50800844,301.02385902)(811.37293605,301.42050015)(811.11565992,301.69065011)
\curveto(810.85837458,301.9650777)(810.54534969,302.10229409)(810.1765843,302.10229969)
\curveto(809.78208352,302.10229409)(809.45619459,301.95864568)(809.19891654,301.67135403)
\curveto(808.94163312,301.38405204)(808.81513676,300.99384293)(808.81942708,300.50072553)
\lineto(811.52087748,300.50072553)
}
}
{
\newrgbcolor{curcolor}{0 0 0}
\pscustom[linestyle=none,fillstyle=solid,fillcolor=curcolor]
{
\newpath
\moveto(816.14550314,301.2468404)
\lineto(814.50533682,301.54271354)
\curveto(814.68972069,302.20306238)(815.0070336,302.69189577)(815.45727649,303.00921518)
\curveto(815.90751616,303.32652158)(816.57644607,303.48517803)(817.46406821,303.48518502)
\curveto(818.2702109,303.48517803)(818.87053261,303.38869776)(819.26503515,303.1957439)
\curveto(819.65952686,303.00706467)(819.93610364,302.76479198)(820.09476634,302.4689251)
\curveto(820.25770456,302.1773343)(820.33917679,301.63918877)(820.33918328,300.85448689)
\lineto(820.31988721,298.74478277)
\curveto(820.31988074,298.14445881)(820.34775282,297.70064955)(820.40350353,297.41335364)
\curveto(820.46352915,297.13034392)(820.57287346,296.82589505)(820.73153679,296.50000613)
\lineto(818.94343391,296.50000613)
\curveto(818.89626068,296.62007047)(818.83837252,296.79802297)(818.76976924,297.03386418)
\curveto(818.73974824,297.14106395)(818.71830817,297.21181615)(818.70544899,297.246121)
\curveto(818.39670726,296.9459594)(818.06653032,296.72083875)(817.71491718,296.5707584)
\curveto(817.36329631,296.4206779)(816.98809524,296.34563769)(816.58931285,296.34563753)
\curveto(815.8860761,296.34563769)(815.33077852,296.53645423)(814.92341843,296.91808773)
\curveto(814.52034421,297.2997204)(814.31880764,297.78212178)(814.31880811,298.36529331)
\curveto(814.31880764,298.75121254)(814.4109999,299.09425352)(814.59538517,299.39441727)
\curveto(814.77976895,299.69886324)(815.03704968,299.9304159)(815.36722814,300.08907594)
\curveto(815.70169158,300.25201682)(816.18194895,300.39352122)(816.80800169,300.51358958)
\curveto(817.65273714,300.67224202)(818.23805081,300.82017844)(818.56394445,300.95739929)
\lineto(818.56394445,301.13749598)
\curveto(818.56393973,301.48482033)(818.47817949,301.73138103)(818.30666346,301.87717883)
\curveto(818.13513851,302.02725388)(817.81139359,302.10229409)(817.33542772,302.10229969)
\curveto(817.01382332,302.10229409)(816.7629746,302.03797391)(816.58288082,301.90933895)
\curveto(816.40278158,301.78498119)(816.25698916,301.56414856)(816.14550314,301.2468404)
\moveto(818.56394445,299.78033875)
\curveto(818.33238707,299.70315125)(817.96576203,299.61095899)(817.46406821,299.50376169)
\curveto(816.96236717,299.39655838)(816.63433424,299.29150208)(816.47996843,299.18859248)
\curveto(816.24412512,299.02135731)(816.12620479,298.80910071)(816.12620706,298.55182202)
\curveto(816.12620479,298.29882725)(816.22054106,298.08013863)(816.40921615,297.8957555)
\curveto(816.59788613,297.71136958)(816.83801482,297.61917731)(817.12960293,297.61917843)
\curveto(817.45548858,297.61917731)(817.76636946,297.72637762)(818.06224652,297.94077967)
\curveto(818.28093093,298.1037227)(818.42457934,298.30311526)(818.49319218,298.53895797)
\curveto(818.54035567,298.69332438)(818.56393973,298.98705321)(818.56394445,299.42014537)
\lineto(818.56394445,299.78033875)
}
}
{
\newrgbcolor{curcolor}{0 0 0}
\pscustom[linestyle=none,fillstyle=solid,fillcolor=curcolor]
{
\newpath
\moveto(821.49051557,298.44890963)
\lineto(823.30434656,298.72548669)
\curveto(823.38152865,298.37386747)(823.5380411,298.1058667)(823.77388436,297.9214836)
\curveto(824.00972244,297.74138566)(824.33989939,297.65133741)(824.76441618,297.65133856)
\curveto(825.23180593,297.65133741)(825.58342293,297.73709765)(825.81926824,297.90861955)
\curveto(825.97792005,298.02868248)(826.05724828,298.18948294)(826.05725316,298.39102141)
\curveto(826.05724828,298.52823591)(826.01436816,298.64186823)(825.92861266,298.73191872)
\curveto(825.83855966,298.81767673)(825.63702308,298.89700496)(825.32400233,298.96990363)
\curveto(823.86607404,299.29150208)(822.9420074,299.58523092)(822.55179966,299.85109103)
\curveto(822.01150875,300.21985673)(821.74136398,300.73227419)(821.74136454,301.38834494)
\curveto(821.74136398,301.98008574)(821.97506065,302.47749516)(822.44245524,302.88057469)
\curveto(822.90984731,303.28364146)(823.63452138,303.48517803)(824.61647961,303.48518502)
\curveto(825.55126284,303.48517803)(826.24592082,303.3329536)(826.70045563,303.02851126)
\curveto(827.15497941,302.72405586)(827.4680043,302.27381458)(827.63953125,301.67778606)
\lineto(825.93504468,301.36261685)
\curveto(825.86214372,301.62846874)(825.72278333,301.83214932)(825.51696308,301.9736592)
\curveto(825.31542217,302.11515813)(825.02598134,302.18591033)(824.64863973,302.18591601)
\curveto(824.17266691,302.18591033)(823.83176994,302.11944614)(823.6259478,301.98652325)
\curveto(823.48872896,301.89218149)(823.42012076,301.76997314)(823.420123,301.61989784)
\curveto(823.42012076,301.49125235)(823.48015294,301.38190804)(823.6002197,301.29186457)
\curveto(823.76316174,301.17179544)(824.32489134,301.00241896)(825.28541018,300.78373462)
\curveto(826.25020883,300.56504171)(826.92342675,300.29704095)(827.30506596,299.97973152)
\curveto(827.68240491,299.65812713)(827.87107745,299.21002985)(827.87108414,298.63543835)
\curveto(827.87107745,298.00938643)(827.6095087,297.47124089)(827.08637712,297.02100013)
\curveto(826.56323372,296.57075833)(825.78924752,296.34563769)(824.76441618,296.34563753)
\curveto(823.83391394,296.34563769)(823.09637584,296.53431022)(822.55179966,296.91165571)
\curveto(822.01150875,297.28900037)(821.65774774,297.80141783)(821.49051557,298.44890963)
}
}
{
\newrgbcolor{curcolor}{0 1 0.25098041}
\pscustom[linewidth=2.63455725,linecolor=curcolor]
{
\newpath
\moveto(648.84463,284.70119456)
\lineto(733.15046,284.70119456)
\lineto(733.15046,337.39235456)
\lineto(676.50748,337.39235456)
\lineto(676.50748,347.93057456)
\lineto(648.84463,347.93057456)
\lineto(648.84463,284.70119456)
\closepath
}
}
{
\newrgbcolor{curcolor}{0 0 0}
\pscustom[linestyle=none,fillstyle=solid,fillcolor=curcolor]
{
\newpath
\moveto(664.05504373,320.38215031)
\lineto(664.05504373,321.46273047)
\lineto(670.30053977,324.09986062)
\lineto(670.30053977,322.94852819)
\lineto(665.34788071,320.91600836)
\lineto(670.30053977,318.86419246)
\lineto(670.30053977,317.71286003)
\lineto(664.05504373,320.38215031)
}
}
{
\newrgbcolor{curcolor}{0 0 0}
\pscustom[linestyle=none,fillstyle=solid,fillcolor=curcolor]
{
\newpath
\moveto(671.74774555,320.38215031)
\lineto(671.74774555,321.46273047)
\lineto(677.99324159,324.09986062)
\lineto(677.99324159,322.94852819)
\lineto(673.04058252,320.91600836)
\lineto(677.99324159,318.86419246)
\lineto(677.99324159,317.71286003)
\lineto(671.74774555,320.38215031)
}
}
{
\newrgbcolor{curcolor}{0 0 0}
\pscustom[linestyle=none,fillstyle=solid,fillcolor=curcolor]
{
\newpath
\moveto(684.04577709,317.10181768)
\curveto(683.61697055,316.7373358)(683.20317737,316.48005506)(682.80439631,316.32997471)
\curveto(682.40989511,316.17989421)(681.9853819,316.10485399)(681.53085541,316.10485384)
\curveto(680.78045046,316.10485399)(680.20371282,316.28709451)(679.80064075,316.65157594)
\curveto(679.39756652,317.0203446)(679.19602995,317.48988194)(679.19603042,318.06018937)
\curveto(679.19602995,318.39465252)(679.27107016,318.69910139)(679.42115129,318.97353688)
\curveto(679.57551903,319.25225496)(679.7749116,319.4752316)(680.01932959,319.64246746)
\curveto(680.268033,319.80969655)(680.5467538,319.93619291)(680.85549281,320.02195692)
\curveto(681.08275532,320.08198533)(681.4257963,320.13987349)(681.88461677,320.19562159)
\curveto(682.81940027,320.30710597)(683.50762623,320.44003435)(683.94929672,320.59440712)
\curveto(683.9535795,320.75305924)(683.95572351,320.85382753)(683.95572875,320.89671229)
\curveto(683.95572351,321.36838899)(683.8463792,321.70070994)(683.62769548,321.89367613)
\curveto(683.33181773,322.15523924)(682.89229648,322.28602361)(682.30913041,322.28602964)
\curveto(681.76454927,322.28602361)(681.36147612,322.18954333)(681.09990975,321.99658852)
\curveto(680.84262664,321.80791025)(680.6518101,321.47130129)(680.52745955,320.98676063)
\lineto(679.39542319,321.14112923)
\curveto(679.49833481,321.62566973)(679.66771129,322.01587884)(679.90355315,322.31175773)
\curveto(680.13939263,322.61191254)(680.48028961,322.84132119)(680.92624508,322.99998438)
\curveto(681.37219615,323.16292211)(681.88890162,323.24439434)(682.47636305,323.24440133)
\curveto(683.05952896,323.24439434)(683.53335431,323.17578614)(683.89784052,323.03857653)
\curveto(684.26231638,322.90135336)(684.53031715,322.72768887)(684.70184362,322.51758253)
\curveto(684.87335813,322.31175168)(684.99342247,322.05018294)(685.06203701,321.73287551)
\curveto(685.10062277,321.53562147)(685.11991883,321.17971646)(685.11992523,320.6651594)
\lineto(685.11992523,319.12147345)
\curveto(685.11991883,318.04517952)(685.1435029,317.36338558)(685.1906775,317.07608958)
\curveto(685.24212718,316.79307995)(685.34075146,316.52079118)(685.48655064,316.25922243)
\lineto(684.27732998,316.25922243)
\curveto(684.15726009,316.49935112)(684.08007587,316.78021592)(684.04577709,317.10181768)
\moveto(683.94929672,319.68749163)
\curveto(683.52906629,319.51596771)(682.8987285,319.3701753)(682.05828144,319.25011395)
\curveto(681.58230875,319.18150276)(681.24569979,319.10431854)(681.04845355,319.01856106)
\curveto(680.85120266,318.93279805)(680.69897823,318.80630169)(680.5917798,318.6390716)
\curveto(680.48457762,318.47612475)(680.43097747,318.29388423)(680.43097918,318.09234949)
\curveto(680.43097747,317.78361078)(680.5467538,317.52633004)(680.77830851,317.32050652)
\curveto(681.01414713,317.11468087)(681.35718811,317.01176858)(681.80743248,317.01176933)
\curveto(682.25338266,317.01176858)(682.65002379,317.10824885)(682.99735706,317.30121045)
\curveto(683.34468177,317.49845797)(683.5998185,317.76645873)(683.762768,318.10521354)
\curveto(683.88711532,318.36678044)(683.94929149,318.75270154)(683.94929672,319.262978)
\lineto(683.94929672,319.68749163)
}
}
{
\newrgbcolor{curcolor}{0 0 0}
\pscustom[linestyle=none,fillstyle=solid,fillcolor=curcolor]
{
\newpath
\moveto(686.92089107,313.64138836)
\lineto(686.92089107,323.09003273)
\lineto(687.97574313,323.09003273)
\lineto(687.97574313,322.20241331)
\curveto(688.22444591,322.54973636)(688.50531071,322.8091611)(688.81833837,322.98068831)
\curveto(689.1313605,323.15649009)(689.51084958,323.24439434)(689.95680675,323.24440133)
\curveto(690.53997251,323.24439434)(691.05453398,323.09431391)(691.5004927,322.79415959)
\curveto(691.94644052,322.4939922)(692.28304948,322.06947899)(692.51032058,321.52061869)
\curveto(692.73757877,320.97603588)(692.8512111,320.37785817)(692.8512179,319.72608378)
\curveto(692.8512111,319.02713432)(692.72471474,318.39679653)(692.47172844,317.8350685)
\curveto(692.22301731,317.27762534)(691.85853627,316.84882411)(691.37828423,316.54866355)
\curveto(690.90230954,316.25279042)(690.40061211,316.10485399)(689.87319043,316.10485384)
\curveto(689.48726551,316.10485399)(689.13993652,316.18632623)(688.83120242,316.34927078)
\curveto(688.52675077,316.51221515)(688.27590206,316.71803974)(688.07865552,316.96674516)
\lineto(688.07865552,313.64138836)
\lineto(686.92089107,313.64138836)
\moveto(687.9693111,319.63603543)
\curveto(687.96930919,318.75698955)(688.14726169,318.1073557)(688.50316916,317.68713193)
\curveto(688.85907172,317.26690531)(689.29001695,317.05679271)(689.79600614,317.0567935)
\curveto(690.31056386,317.05679271)(690.75008511,317.27333732)(691.11457121,317.70642801)
\curveto(691.4833352,318.1438038)(691.66771973,318.81916573)(691.66772534,319.73251581)
\curveto(691.66771973,320.60297881)(691.48762321,321.25475667)(691.12743526,321.68785133)
\curveto(690.77152517,322.12093514)(690.34486796,322.33747976)(689.84746233,322.33748583)
\curveto(689.35433713,322.33747976)(688.91695989,322.1059271)(688.53532928,321.64282716)
\curveto(688.15798172,321.18400447)(687.96930919,320.51507456)(687.9693111,319.63603543)
}
}
{
\newrgbcolor{curcolor}{0 0 0}
\pscustom[linestyle=none,fillstyle=solid,fillcolor=curcolor]
{
\newpath
\moveto(694.25339915,313.64138836)
\lineto(694.25339915,323.09003273)
\lineto(695.30825121,323.09003273)
\lineto(695.30825121,322.20241331)
\curveto(695.556954,322.54973636)(695.8378188,322.8091611)(696.15084646,322.98068831)
\curveto(696.46386858,323.15649009)(696.84335767,323.24439434)(697.28931484,323.24440133)
\curveto(697.8724806,323.24439434)(698.38704207,323.09431391)(698.83300078,322.79415959)
\curveto(699.27894861,322.4939922)(699.61555757,322.06947899)(699.84282867,321.52061869)
\curveto(700.07008686,320.97603588)(700.18371919,320.37785817)(700.18372598,319.72608378)
\curveto(700.18371919,319.02713432)(700.05722283,318.39679653)(699.80423652,317.8350685)
\curveto(699.5555254,317.27762534)(699.19104436,316.84882411)(698.71079231,316.54866355)
\curveto(698.23481763,316.25279042)(697.7331202,316.10485399)(697.20569852,316.10485384)
\curveto(696.8197736,316.10485399)(696.47244461,316.18632623)(696.16371051,316.34927078)
\curveto(695.85925886,316.51221515)(695.60841015,316.71803974)(695.41116361,316.96674516)
\lineto(695.41116361,313.64138836)
\lineto(694.25339915,313.64138836)
\moveto(695.30181919,319.63603543)
\curveto(695.30181727,318.75698955)(695.47976978,318.1073557)(695.83567725,317.68713193)
\curveto(696.19157981,317.26690531)(696.62252504,317.05679271)(697.12851422,317.0567935)
\curveto(697.64307195,317.05679271)(698.0825932,317.27333732)(698.4470793,317.70642801)
\curveto(698.81584329,318.1438038)(699.00022781,318.81916573)(699.00023343,319.73251581)
\curveto(699.00022781,320.60297881)(698.8201313,321.25475667)(698.45994335,321.68785133)
\curveto(698.10403326,322.12093514)(697.67737604,322.33747976)(697.17997042,322.33748583)
\curveto(696.68684522,322.33747976)(696.24946797,322.1059271)(695.86783737,321.64282716)
\curveto(695.49048981,321.18400447)(695.30181727,320.51507456)(695.30181919,319.63603543)
}
}
{
\newrgbcolor{curcolor}{0 0 0}
\pscustom[linestyle=none,fillstyle=solid,fillcolor=curcolor]
{
\newpath
\moveto(707.68346672,320.38215031)
\lineto(701.43797067,317.71286003)
\lineto(701.43797067,318.86419246)
\lineto(706.38419771,320.91600836)
\lineto(701.43797067,322.94852819)
\lineto(701.43797067,324.09986062)
\lineto(707.68346672,321.46273047)
\lineto(707.68346672,320.38215031)
}
}
{
\newrgbcolor{curcolor}{0 0 0}
\pscustom[linestyle=none,fillstyle=solid,fillcolor=curcolor]
{
\newpath
\moveto(715.37616996,320.38215031)
\lineto(709.13067392,317.71286003)
\lineto(709.13067392,318.86419246)
\lineto(714.07690096,320.91600836)
\lineto(709.13067392,322.94852819)
\lineto(709.13067392,324.09986062)
\lineto(715.37616996,321.46273047)
\lineto(715.37616996,320.38215031)
}
}
{
\newrgbcolor{curcolor}{0 0 0}
\pscustom[linestyle=none,fillstyle=solid,fillcolor=curcolor]
{
\newpath
\moveto(669.38203596,296.04976439)
\lineto(671.44671591,295.79891543)
\curveto(671.48101716,295.55878744)(671.56034539,295.39369897)(671.68470082,295.30364952)
\curveto(671.85621823,295.17501035)(672.126363,295.11069017)(672.49513594,295.11068878)
\curveto(672.96681339,295.11069017)(673.3205744,295.18144237)(673.55642003,295.32294559)
\curveto(673.71507153,295.41728304)(673.83513587,295.56950747)(673.91661341,295.77961935)
\curveto(673.97235226,295.9297005)(674.00022434,296.20627729)(674.00022974,296.60935055)
\lineto(674.00022974,297.60631438)
\curveto(673.4599348,296.86877518)(672.77814086,296.50000613)(671.95484586,296.50000613)
\curveto(671.0372079,296.50000613)(670.31038982,296.88807123)(669.77438947,297.66420261)
\curveto(669.3541631,298.27738719)(669.1440505,299.04065337)(669.14405104,299.95400342)
\curveto(669.1440505,301.09889923)(669.41848328,301.97365372)(669.96735021,302.57826953)
\curveto(670.52050242,303.18287317)(671.20658438,303.48517803)(672.02559813,303.48518502)
\curveto(672.87033312,303.48517803)(673.56713511,303.11426497)(674.11600618,302.37244473)
\lineto(674.11600618,303.33081642)
\lineto(675.80762869,303.33081642)
\lineto(675.80762869,297.20109682)
\curveto(675.80762149,296.39494983)(675.7411573,295.79248411)(675.60823593,295.39369787)
\curveto(675.47530054,294.99491384)(675.28877201,294.68188894)(675.04864977,294.45462225)
\curveto(674.80851464,294.22735965)(674.48691373,294.04940714)(674.08384606,293.9207642)
\curveto(673.68505544,293.79212641)(673.17907,293.72780622)(672.56588821,293.72780345)
\curveto(671.40812095,293.72780622)(670.58696661,293.92719879)(670.10242273,294.32598176)
\curveto(669.61787585,294.72048105)(669.37560316,295.22217848)(669.37560393,295.83107555)
\curveto(669.37560316,295.89110839)(669.37774717,295.9640046)(669.38203596,296.04976439)
\moveto(670.99647417,300.05691582)
\curveto(670.99647178,299.3322382)(671.13583218,298.80052468)(671.41455578,298.46177368)
\curveto(671.69756178,298.12730676)(672.04489077,297.96007429)(672.45654379,297.96007575)
\curveto(672.8982052,297.96007429)(673.27126226,298.13159477)(673.5757161,298.47463773)
\curveto(673.88016,298.82196474)(674.03238443,299.3343822)(674.03238986,300.01189165)
\curveto(674.03238443,300.71941015)(673.88659202,301.24469165)(673.59501218,301.58773771)
\curveto(673.30342235,301.9307736)(672.9346533,302.10229409)(672.48870392,302.10229969)
\curveto(672.0556108,302.10229409)(671.69756178,301.93291761)(671.41455578,301.59416974)
\curveto(671.13583218,301.25969969)(670.99647178,300.74728223)(670.99647417,300.05691582)
}
}
{
\newrgbcolor{curcolor}{0 0 0}
\pscustom[linestyle=none,fillstyle=solid,fillcolor=curcolor]
{
\newpath
\moveto(679.33237785,296.50000613)
\lineto(677.5249789,296.50000613)
\lineto(677.5249789,303.33081642)
\lineto(679.20373736,303.33081642)
\lineto(679.20373736,302.35958068)
\curveto(679.49103163,302.81839213)(679.74831236,303.12069699)(679.97558033,303.26649617)
\curveto(680.20712967,303.41228182)(680.46869842,303.48517803)(680.76028735,303.48518502)
\curveto(681.17193242,303.48517803)(681.56857355,303.37154571)(681.95021193,303.1442877)
\lineto(681.39062578,301.56844164)
\curveto(681.08617218,301.76568513)(680.80316337,301.86430941)(680.54159851,301.86431478)
\curveto(680.2886019,301.86430941)(680.07420129,301.79355721)(679.89839603,301.65205796)
\curveto(679.72258429,301.51483642)(679.58322389,301.2639877)(679.48031442,300.89951106)
\curveto(679.38168732,300.53502562)(679.33237518,299.77175945)(679.33237785,298.60971025)
\lineto(679.33237785,296.50000613)
}
}
{
\newrgbcolor{curcolor}{0 0 0}
\pscustom[linestyle=none,fillstyle=solid,fillcolor=curcolor]
{
\newpath
\moveto(682.3039737,300.01189165)
\curveto(682.30397318,300.61220984)(682.4519096,301.1932355)(682.74778341,301.75497036)
\curveto(683.04365528,302.3166947)(683.46173648,302.74549592)(684.00202824,303.04137531)
\curveto(684.54660357,303.33724161)(685.1533573,303.48517803)(685.82229125,303.48518502)
\curveto(686.85569815,303.48517803)(687.70258056,303.14856907)(688.36294103,302.47535713)
\curveto(689.02328833,301.80642125)(689.35346527,300.95953883)(689.35347284,299.93470735)
\curveto(689.35346527,298.90129297)(689.01900031,298.04369052)(688.35007698,297.36189744)
\curveto(687.68542851,296.68439065)(686.84712212,296.34563769)(685.8351553,296.34563753)
\curveto(685.20910146,296.34563769)(684.61092375,296.48714209)(684.04062039,296.77015117)
\curveto(683.47460051,297.0531597)(683.04365528,297.46695288)(682.74778341,298.01153194)
\curveto(682.4519096,298.560396)(682.30397318,299.2271819)(682.3039737,300.01189165)
\moveto(684.15639684,299.91541127)
\curveto(684.15639446,299.23790193)(684.31719491,298.71905245)(684.63879869,298.35886128)
\curveto(684.96039675,297.9986664)(685.35703788,297.81856988)(685.82872327,297.8185712)
\curveto(686.30040057,297.81856988)(686.69489769,297.9986664)(687.01221583,298.35886128)
\curveto(687.33381151,298.71905245)(687.49461197,299.24218994)(687.49461769,299.92827532)
\curveto(687.49461197,300.5972018)(687.33381151,301.11176327)(687.01221583,301.47196127)
\curveto(686.69489769,301.83214932)(686.30040057,302.01224583)(685.82872327,302.01225135)
\curveto(685.35703788,302.01224583)(684.96039675,301.83214932)(684.63879869,301.47196127)
\curveto(684.31719491,301.11176327)(684.15639446,300.59291379)(684.15639684,299.91541127)
}
}
{
\newrgbcolor{curcolor}{0 0 0}
\pscustom[linestyle=none,fillstyle=solid,fillcolor=curcolor]
{
\newpath
\moveto(695.27093521,296.50000613)
\lineto(695.27093521,297.52269806)
\curveto(695.02222506,297.158216)(694.69419213,296.87091918)(694.28683543,296.66080674)
\curveto(693.88375782,296.45069398)(693.4571006,296.34563769)(693.0068625,296.34563753)
\curveto(692.54804201,296.34563769)(692.13639284,296.44640597)(691.77191374,296.64794269)
\curveto(691.40743076,296.84947912)(691.14371801,297.13248793)(690.9807747,297.49696996)
\curveto(690.81782908,297.86145)(690.73635685,298.36529144)(690.73635776,299.00849578)
\lineto(690.73635776,303.33081642)
\lineto(692.54375672,303.33081642)
\lineto(692.54375672,300.19198834)
\curveto(692.543754,299.23146991)(692.57591409,298.64186823)(692.64023709,298.42318153)
\curveto(692.70884247,298.20877899)(692.83105082,298.03725851)(693.0068625,297.90861955)
\curveto(693.18266782,297.78426578)(693.40564446,297.72208961)(693.67579307,297.72209083)
\curveto(693.98452611,297.72208961)(694.2611029,297.80570585)(694.50552427,297.9729398)
\curveto(694.74993629,298.14445881)(694.91716876,298.35457141)(695.0072222,298.60327822)
\curveto(695.09726528,298.85626884)(695.14228941,299.47159859)(695.14229472,300.44926933)
\lineto(695.14229472,303.33081642)
\lineto(696.94969368,303.33081642)
\lineto(696.94969368,296.50000613)
\lineto(695.27093521,296.50000613)
}
}
{
\newrgbcolor{curcolor}{0 0 0}
\pscustom[linestyle=none,fillstyle=solid,fillcolor=curcolor]
{
\newpath
\moveto(698.7763883,303.33081642)
\lineto(700.46157879,303.33081642)
\lineto(700.46157879,302.32742056)
\curveto(700.68026483,302.67045571)(700.97613767,302.9491765)(701.3491982,303.16358378)
\curveto(701.7222518,303.37797773)(702.13604498,303.48517803)(702.59057898,303.48518502)
\curveto(703.38385654,303.48517803)(704.05707445,303.17429715)(704.61023476,302.55254143)
\curveto(705.16338161,301.9307736)(705.4399584,301.06459513)(705.43996595,299.95400342)
\curveto(705.4399584,298.81338872)(705.1612376,297.92577019)(704.60380273,297.29114517)
\curveto(704.04635442,296.66080658)(703.3709925,296.34563769)(702.57771493,296.34563753)
\curveto(702.20036516,296.34563769)(701.85732419,296.4206779)(701.54859097,296.5707584)
\curveto(701.24413844,296.72083875)(700.92253752,296.97811949)(700.58378726,297.34260137)
\lineto(700.58378726,293.90146812)
\lineto(698.7763883,293.90146812)
\lineto(698.7763883,303.33081642)
\moveto(700.56449118,300.03118772)
\curveto(700.5644885,299.26363)(700.71671294,298.69546838)(701.02116494,298.32670116)
\curveto(701.32561067,297.96221829)(701.69652373,297.77997777)(702.13390523,297.77997905)
\curveto(702.55412617,297.77997777)(702.90359917,297.94721025)(703.18232526,298.28167698)
\curveto(703.46104076,298.62042817)(703.60040115,299.17358174)(703.60040687,299.94113937)
\curveto(703.60040115,300.65723397)(703.45675274,301.18894749)(703.16946121,301.53628151)
\curveto(702.88215911,301.88360547)(702.52625409,302.05726996)(702.1017451,302.05727552)
\curveto(701.66007562,302.05726996)(701.29345058,301.88574947)(701.00186887,301.54271354)
\curveto(700.71028092,301.20395553)(700.5644885,300.70011409)(700.56449118,300.03118772)
}
}
{
\newrgbcolor{curcolor}{0 0 0}
\pscustom[linestyle=none,fillstyle=solid,fillcolor=curcolor]
{
\newpath
\moveto(706.24396864,298.44890963)
\lineto(708.05779962,298.72548669)
\curveto(708.13498172,298.37386747)(708.29149417,298.1058667)(708.52733743,297.9214836)
\curveto(708.76317551,297.74138566)(709.09335245,297.65133741)(709.51786924,297.65133856)
\curveto(709.98525899,297.65133741)(710.33687599,297.73709765)(710.5727213,297.90861955)
\curveto(710.73137312,298.02868248)(710.81070134,298.18948294)(710.81070622,298.39102141)
\curveto(710.81070134,298.52823591)(710.76782122,298.64186823)(710.68206572,298.73191872)
\curveto(710.59201272,298.81767673)(710.39047615,298.89700496)(710.0774554,298.96990363)
\curveto(708.6195271,299.29150208)(707.69546047,299.58523092)(707.30525272,299.85109103)
\curveto(706.76496182,300.21985673)(706.49481705,300.73227419)(706.4948176,301.38834494)
\curveto(706.49481705,301.98008574)(706.72851371,302.47749516)(707.1959083,302.88057469)
\curveto(707.66330038,303.28364146)(708.38797444,303.48517803)(709.36993267,303.48518502)
\curveto(710.3047159,303.48517803)(710.99937388,303.3329536)(711.4539087,303.02851126)
\curveto(711.90843247,302.72405586)(712.22145736,302.27381458)(712.39298431,301.67778606)
\lineto(710.68849775,301.36261685)
\curveto(710.61559679,301.62846874)(710.47623639,301.83214932)(710.27041614,301.9736592)
\curveto(710.06887523,302.11515813)(709.77943441,302.18591033)(709.4020928,302.18591601)
\curveto(708.92611997,302.18591033)(708.585223,302.11944614)(708.37940086,301.98652325)
\curveto(708.24218202,301.89218149)(708.17357383,301.76997314)(708.17357607,301.61989784)
\curveto(708.17357383,301.49125235)(708.233606,301.38190804)(708.35367276,301.29186457)
\curveto(708.51661481,301.17179544)(709.07834441,301.00241896)(710.03886325,300.78373462)
\curveto(711.00366189,300.56504171)(711.67687981,300.29704095)(712.05851902,299.97973152)
\curveto(712.43585798,299.65812713)(712.62453051,299.21002985)(712.6245372,298.63543835)
\curveto(712.62453051,298.00938643)(712.36296177,297.47124089)(711.83983018,297.02100013)
\curveto(711.31668679,296.57075833)(710.54270058,296.34563769)(709.51786924,296.34563753)
\curveto(708.58736701,296.34563769)(707.84982891,296.53431022)(707.30525272,296.91165571)
\curveto(706.76496182,297.28900037)(706.41120081,297.80141783)(706.24396864,298.44890963)
}
}
{
\newrgbcolor{curcolor}{0 1 0.25098041}
\pscustom[linewidth=2.63455725,linecolor=curcolor]
{
\newpath
\moveto(770.35329,165.29658456)
\lineto(854.65912,165.29658456)
\lineto(854.65912,217.98773456)
\lineto(799.33342,217.98773456)
\lineto(799.33342,228.52596456)
\lineto(770.35329,228.52596456)
\lineto(770.35329,165.29658456)
\closepath
}
}
{
\newrgbcolor{curcolor}{0 0 0}
\pscustom[linestyle=none,fillstyle=solid,fillcolor=curcolor]
{
\newpath
\moveto(785.56374248,200.97753661)
\lineto(785.56374248,202.05811677)
\lineto(791.80923852,204.69524692)
\lineto(791.80923852,203.54391449)
\lineto(786.85657945,201.51139466)
\lineto(791.80923852,199.45957876)
\lineto(791.80923852,198.30824633)
\lineto(785.56374248,200.97753661)
}
}
{
\newrgbcolor{curcolor}{0 0 0}
\pscustom[linestyle=none,fillstyle=solid,fillcolor=curcolor]
{
\newpath
\moveto(793.25644429,200.97753661)
\lineto(793.25644429,202.05811677)
\lineto(799.50194034,204.69524692)
\lineto(799.50194034,203.54391449)
\lineto(794.54928127,201.51139466)
\lineto(799.50194034,199.45957876)
\lineto(799.50194034,198.30824633)
\lineto(793.25644429,200.97753661)
}
}
{
\newrgbcolor{curcolor}{0 0 0}
\pscustom[linestyle=none,fillstyle=solid,fillcolor=curcolor]
{
\newpath
\moveto(805.55447584,197.69720398)
\curveto(805.12566929,197.3327221)(804.71187611,197.07544136)(804.31309506,196.92536101)
\curveto(803.91859385,196.77528051)(803.49408064,196.70024029)(803.03955416,196.70024014)
\curveto(802.28914921,196.70024029)(801.71241156,196.88248081)(801.3093395,197.24696224)
\curveto(800.90626527,197.6157309)(800.70472869,198.08526824)(800.70472917,198.65557567)
\curveto(800.70472869,198.99003882)(800.77976891,199.29448769)(800.92985004,199.56892318)
\curveto(801.08421777,199.84764126)(801.28361034,200.0706179)(801.52802834,200.23785376)
\curveto(801.77673175,200.40508285)(802.05545254,200.53157921)(802.36419156,200.61734322)
\curveto(802.59145407,200.67737163)(802.93449505,200.73525979)(803.39331552,200.79100789)
\curveto(804.32809902,200.90249227)(805.01632498,201.03542065)(805.45799547,201.18979342)
\curveto(805.46227825,201.34844554)(805.46442226,201.44921383)(805.46442749,201.49209859)
\curveto(805.46442226,201.96377529)(805.35507795,202.29609624)(805.13639423,202.48906243)
\curveto(804.84051648,202.75062554)(804.40099523,202.88140991)(803.81782915,202.88141594)
\curveto(803.27324801,202.88140991)(802.87017486,202.78492963)(802.6086085,202.59197482)
\curveto(802.35132539,202.40329655)(802.16050884,202.06668759)(802.03615829,201.58214693)
\lineto(800.90412194,201.73651553)
\curveto(801.00703355,202.22105603)(801.17641004,202.61126514)(801.41225189,202.90714404)
\curveto(801.64809138,203.20729884)(801.98898835,203.43670749)(802.43494383,203.59537068)
\curveto(802.88089489,203.75830841)(803.39760037,203.83978064)(803.9850618,203.83978763)
\curveto(804.5682277,203.83978064)(805.04205305,203.77117245)(805.40653927,203.63396283)
\curveto(805.77101513,203.49673966)(806.03901589,203.32307517)(806.21054236,203.11296883)
\curveto(806.38205687,202.90713798)(806.50212121,202.64556924)(806.57073575,202.32826181)
\curveto(806.60932152,202.13100777)(806.62861757,201.77510276)(806.62862397,201.2605457)
\lineto(806.62862397,199.71685975)
\curveto(806.62861757,198.64056582)(806.65220164,197.95877188)(806.69937625,197.67147588)
\curveto(806.75082592,197.38846626)(806.8494502,197.11617748)(806.99524939,196.85460873)
\lineto(805.78602873,196.85460873)
\curveto(805.66595883,197.09473742)(805.58877461,197.37560222)(805.55447584,197.69720398)
\moveto(805.45799547,200.28287793)
\curveto(805.03776504,200.11135402)(804.40742724,199.9655616)(803.56698019,199.84550025)
\curveto(803.09100749,199.77688906)(802.75439853,199.69970484)(802.5571523,199.61394736)
\curveto(802.35990141,199.52818435)(802.20767698,199.40168799)(802.10047854,199.2344579)
\curveto(801.99327636,199.07151105)(801.93967621,198.88927053)(801.93967792,198.68773579)
\curveto(801.93967621,198.37899708)(802.05545254,198.12171635)(802.28700726,197.91589282)
\curveto(802.52284587,197.71006717)(802.86588685,197.60715488)(803.31613122,197.60715563)
\curveto(803.76208141,197.60715488)(804.15872254,197.70363515)(804.5060558,197.89659675)
\curveto(804.85338052,198.09384427)(805.10851724,198.36184503)(805.27146675,198.70059984)
\curveto(805.39581406,198.96216674)(805.45799024,199.34808784)(805.45799547,199.8583643)
\lineto(805.45799547,200.28287793)
}
}
{
\newrgbcolor{curcolor}{0 0 0}
\pscustom[linestyle=none,fillstyle=solid,fillcolor=curcolor]
{
\newpath
\moveto(808.42958981,194.23677466)
\lineto(808.42958981,203.68541903)
\lineto(809.48444187,203.68541903)
\lineto(809.48444187,202.79779961)
\curveto(809.73314466,203.14512266)(810.01400946,203.4045474)(810.32703712,203.57607461)
\curveto(810.64005924,203.75187639)(811.01954833,203.83978064)(811.4655055,203.83978763)
\curveto(812.04867126,203.83978064)(812.56323272,203.68970021)(813.00919144,203.38954589)
\curveto(813.45513927,203.0893785)(813.79174823,202.66486529)(814.01901933,202.11600499)
\curveto(814.24627752,201.57142218)(814.35990984,200.97324447)(814.35991664,200.32147008)
\curveto(814.35990984,199.62252062)(814.23341348,198.99218283)(813.98042718,198.4304548)
\curveto(813.73171605,197.87301164)(813.36723502,197.44421041)(812.88698297,197.14404985)
\curveto(812.41100829,196.84817672)(811.90931086,196.70024029)(811.38188918,196.70024014)
\curveto(810.99596426,196.70024029)(810.64863527,196.78171253)(810.33990117,196.94465708)
\curveto(810.03544952,197.10760146)(809.78460081,197.31342604)(809.58735427,197.56213146)
\lineto(809.58735427,194.23677466)
\lineto(808.42958981,194.23677466)
\moveto(809.47800985,200.23142173)
\curveto(809.47800793,199.35237585)(809.65596044,198.702742)(810.0118679,198.28251823)
\curveto(810.36777047,197.86229161)(810.7987157,197.65217901)(811.30470488,197.6521798)
\curveto(811.8192626,197.65217901)(812.25878386,197.86872362)(812.62326996,198.30181431)
\curveto(812.99203395,198.73919011)(813.17641847,199.41455203)(813.17642409,200.32790211)
\curveto(813.17641847,201.19836511)(812.99632196,201.85014297)(812.63613401,202.28323763)
\curveto(812.28022392,202.71632144)(811.8535667,202.93286606)(811.35616108,202.93287213)
\curveto(810.86303588,202.93286606)(810.42565863,202.7013134)(810.04402803,202.23821346)
\curveto(809.66668047,201.77939077)(809.47800793,201.11046086)(809.47800985,200.23142173)
}
}
{
\newrgbcolor{curcolor}{0 0 0}
\pscustom[linestyle=none,fillstyle=solid,fillcolor=curcolor]
{
\newpath
\moveto(815.7620979,194.23677466)
\lineto(815.7620979,203.68541903)
\lineto(816.81694996,203.68541903)
\lineto(816.81694996,202.79779961)
\curveto(817.06565275,203.14512266)(817.34651755,203.4045474)(817.6595452,203.57607461)
\curveto(817.97256733,203.75187639)(818.35205641,203.83978064)(818.79801359,203.83978763)
\curveto(819.38117935,203.83978064)(819.89574081,203.68970021)(820.34169953,203.38954589)
\curveto(820.78764735,203.0893785)(821.12425631,202.66486529)(821.35152742,202.11600499)
\curveto(821.57878561,201.57142218)(821.69241793,200.97324447)(821.69242473,200.32147008)
\curveto(821.69241793,199.62252062)(821.56592157,198.99218283)(821.31293527,198.4304548)
\curveto(821.06422414,197.87301164)(820.6997431,197.44421041)(820.21949106,197.14404985)
\curveto(819.74351638,196.84817672)(819.24181895,196.70024029)(818.71439727,196.70024014)
\curveto(818.32847235,196.70024029)(817.98114336,196.78171253)(817.67240925,196.94465708)
\curveto(817.36795761,197.10760146)(817.11710889,197.31342604)(816.91986236,197.56213146)
\lineto(816.91986236,194.23677466)
\lineto(815.7620979,194.23677466)
\moveto(816.81051794,200.23142173)
\curveto(816.81051602,199.35237585)(816.98846853,198.702742)(817.34437599,198.28251823)
\curveto(817.70027855,197.86229161)(818.13122378,197.65217901)(818.63721297,197.6521798)
\curveto(819.15177069,197.65217901)(819.59129194,197.86872362)(819.95577804,198.30181431)
\curveto(820.32454203,198.73919011)(820.50892656,199.41455203)(820.50893217,200.32790211)
\curveto(820.50892656,201.19836511)(820.32883005,201.85014297)(819.96864209,202.28323763)
\curveto(819.61273201,202.71632144)(819.18607479,202.93286606)(818.68866917,202.93287213)
\curveto(818.19554397,202.93286606)(817.75816672,202.7013134)(817.37653612,202.23821346)
\curveto(816.99918856,201.77939077)(816.81051602,201.11046086)(816.81051794,200.23142173)
}
}
{
\newrgbcolor{curcolor}{0 0 0}
\pscustom[linestyle=none,fillstyle=solid,fillcolor=curcolor]
{
\newpath
\moveto(829.19216546,200.97753661)
\lineto(822.94666942,198.30824633)
\lineto(822.94666942,199.45957876)
\lineto(827.89289646,201.51139466)
\lineto(822.94666942,203.54391449)
\lineto(822.94666942,204.69524692)
\lineto(829.19216546,202.05811677)
\lineto(829.19216546,200.97753661)
}
}
{
\newrgbcolor{curcolor}{0 0 0}
\pscustom[linestyle=none,fillstyle=solid,fillcolor=curcolor]
{
\newpath
\moveto(836.88486871,200.97753661)
\lineto(830.63937266,198.30824633)
\lineto(830.63937266,199.45957876)
\lineto(835.58559971,201.51139466)
\lineto(830.63937266,203.54391449)
\lineto(830.63937266,204.69524692)
\lineto(836.88486871,202.05811677)
\lineto(836.88486871,200.97753661)
}
}
{
\newrgbcolor{curcolor}{0 0 0}
\pscustom[linestyle=none,fillstyle=solid,fillcolor=curcolor]
{
\newpath
\moveto(796.82488489,183.92620272)
\lineto(796.82488489,182.48542918)
\lineto(795.58993614,182.48542918)
\lineto(795.58993614,179.73252258)
\curveto(795.58993329,179.17507835)(795.60065332,178.84918942)(795.62209626,178.75485481)
\curveto(795.64782146,178.6648049)(795.70142161,178.58976468)(795.78289688,178.52973395)
\curveto(795.86865409,178.46970034)(795.97156638,178.43968426)(796.09163407,178.4396856)
\curveto(796.2588632,178.43968426)(796.50113589,178.49757242)(796.81845287,178.61335027)
\lineto(796.97282146,177.21116887)
\curveto(796.55259204,177.03107224)(796.07662268,176.94102399)(795.54491196,176.94102383)
\curveto(795.21902024,176.94102399)(794.9252914,176.99462414)(794.66372457,177.10182445)
\curveto(794.40215391,177.21331276)(794.20919336,177.35481717)(794.08484234,177.52633808)
\curveto(793.96477666,177.70214615)(793.88116042,177.93798683)(793.83399338,178.23386081)
\curveto(793.79540018,178.44397227)(793.77610412,178.86848548)(793.77610515,179.50740171)
\lineto(793.77610515,182.48542918)
\lineto(792.94637396,182.48542918)
\lineto(792.94637396,183.92620272)
\lineto(793.77610515,183.92620272)
\lineto(793.77610515,185.28335995)
\lineto(795.58993614,186.33821201)
\lineto(795.58993614,183.92620272)
\lineto(796.82488489,183.92620272)
}
}
{
\newrgbcolor{curcolor}{0 0 0}
\pscustom[linestyle=none,fillstyle=solid,fillcolor=curcolor]
{
\newpath
\moveto(802.0476892,179.2694168)
\lineto(803.84865613,178.96711163)
\curveto(803.61709677,178.30675588)(803.25047172,177.80291444)(802.7487799,177.45558581)
\curveto(802.25136488,177.11254447)(801.6274591,176.94102399)(800.87706069,176.94102383)
\curveto(799.68927758,176.94102399)(798.81023507,177.32908909)(798.23993054,178.10522031)
\curveto(797.78968816,178.72698108)(797.56456752,179.51168731)(797.56456794,180.45934138)
\curveto(797.56456752,181.59137324)(797.86044036,182.47684776)(798.45218736,183.1157676)
\curveto(799.04393174,183.75896342)(799.79218987,184.08056433)(800.696964,184.08057132)
\curveto(801.71321934,184.08056433)(802.51507763,183.74395537)(803.10254126,183.07074343)
\curveto(803.68999298,182.40180755)(803.97085778,181.37482862)(803.9451365,179.98980357)
\lineto(799.41699107,179.98980357)
\curveto(799.42985284,179.45379915)(799.57564525,179.03571796)(799.85436875,178.73555874)
\curveto(800.13308684,178.43968426)(800.48041583,178.29174784)(800.89635677,178.29174903)
\curveto(801.17936182,178.29174784)(801.4173465,178.36893206)(801.61031151,178.52330192)
\curveto(801.8032676,178.67766893)(801.94906002,178.92637364)(802.0476892,179.2694168)
\moveto(802.15060159,181.09611183)
\curveto(802.13773255,181.61924532)(802.00266017,182.01588645)(801.74538403,182.28603641)
\curveto(801.4880987,182.560464)(801.17507381,182.69768039)(800.80630842,182.69768599)
\curveto(800.41180764,182.69768039)(800.08591871,182.55403198)(799.82864066,182.26674033)
\curveto(799.57135724,181.97943834)(799.44486088,181.58922923)(799.44915119,181.09611183)
\lineto(802.15060159,181.09611183)
}
}
{
\newrgbcolor{curcolor}{0 0 0}
\pscustom[linestyle=none,fillstyle=solid,fillcolor=curcolor]
{
\newpath
\moveto(806.77522773,181.8422267)
\lineto(805.13506142,182.13809984)
\curveto(805.31944529,182.79844868)(805.63675819,183.28728207)(806.08700108,183.60460148)
\curveto(806.53724076,183.92190788)(807.20617067,184.08056433)(808.09379281,184.08057132)
\curveto(808.89993549,184.08056433)(809.5002572,183.98408406)(809.89475974,183.7911302)
\curveto(810.28925145,183.60245097)(810.56582824,183.36017828)(810.72449094,183.0643114)
\curveto(810.88742916,182.77272061)(810.96890139,182.23457507)(810.96890788,181.44987319)
\lineto(810.9496118,179.34016907)
\curveto(810.94960533,178.73984511)(810.97747741,178.29603585)(811.03322813,178.00873994)
\curveto(811.09325374,177.72573022)(811.20259805,177.42128135)(811.36126139,177.09539243)
\lineto(809.57315851,177.09539243)
\curveto(809.52598528,177.21545677)(809.46809711,177.39340928)(809.39949384,177.62925048)
\curveto(809.36947283,177.73645025)(809.34803277,177.80720245)(809.33517359,177.8415073)
\curveto(809.02643185,177.5413457)(808.69625491,177.31622506)(808.34464178,177.1661447)
\curveto(807.99302091,177.0160642)(807.61781984,176.94102399)(807.21903744,176.94102383)
\curveto(806.5158007,176.94102399)(805.96050312,177.13184053)(805.55314303,177.51347404)
\curveto(805.15006881,177.8951067)(804.94853223,178.37750808)(804.9485327,178.96067961)
\curveto(804.94853223,179.34659884)(805.04072449,179.68963982)(805.22510977,179.98980357)
\curveto(805.40949355,180.29424954)(805.66677428,180.5258022)(805.99695274,180.68446224)
\curveto(806.33141617,180.84740312)(806.81167354,180.98890752)(807.43772628,181.10897588)
\curveto(808.28246173,181.26762832)(808.8677754,181.41556474)(809.19366904,181.55278559)
\lineto(809.19366904,181.73288228)
\curveto(809.19366433,182.08020663)(809.10790409,182.32676733)(808.93638805,182.47256513)
\curveto(808.76486311,182.62264018)(808.44111819,182.69768039)(807.96515231,182.69768599)
\curveto(807.64354791,182.69768039)(807.3926992,182.63336021)(807.21260542,182.50472525)
\curveto(807.03250617,182.38036749)(806.88671376,182.15953486)(806.77522773,181.8422267)
\moveto(809.19366904,180.37572505)
\curveto(808.96211167,180.29853755)(808.59548663,180.20634529)(808.09379281,180.09914799)
\curveto(807.59209177,179.99194468)(807.26405883,179.88688838)(807.10969302,179.78397878)
\curveto(806.87384972,179.61674361)(806.75592938,179.40448701)(806.75593166,179.14720832)
\curveto(806.75592938,178.89421355)(806.85026565,178.67552493)(807.03894075,178.4911418)
\curveto(807.22761073,178.30675588)(807.46773941,178.21456362)(807.75932752,178.21456473)
\curveto(808.08521317,178.21456362)(808.39609406,178.32176392)(808.69197111,178.53616597)
\curveto(808.91065552,178.699109)(809.05430393,178.89850156)(809.12291677,179.13434428)
\curveto(809.17008026,179.28871068)(809.19366433,179.58243951)(809.19366904,180.01553167)
\lineto(809.19366904,180.37572505)
}
}
{
\newrgbcolor{curcolor}{0 0 0}
\pscustom[linestyle=none,fillstyle=solid,fillcolor=curcolor]
{
\newpath
\moveto(812.6219381,183.92620272)
\lineto(814.28783251,183.92620272)
\lineto(814.28783251,182.99355913)
\curveto(814.88386374,183.7182273)(815.59352976,184.08056433)(816.41683271,184.08057132)
\curveto(816.85420535,184.08056433)(817.23369443,183.99051608)(817.55530109,183.81042628)
\curveto(817.87689627,183.63032305)(818.14060902,183.35803427)(818.34644014,182.99355913)
\curveto(818.64659446,183.35803427)(818.97033938,183.63032305)(819.31767588,183.81042628)
\curveto(819.66499736,183.99051608)(820.03591042,184.08056433)(820.43041616,184.08057132)
\curveto(820.93210497,184.08056433)(821.35661818,183.97765204)(821.70395706,183.77183413)
\curveto(822.05127616,183.57029088)(822.3107009,183.27227403)(822.48223206,182.87778269)
\curveto(822.60657374,182.58619207)(822.66874992,182.11451073)(822.66876078,181.46273724)
\lineto(822.66876078,177.09539243)
\lineto(820.86136182,177.09539243)
\lineto(820.86136182,180.99963146)
\curveto(820.86135277,181.67713348)(820.79917659,182.11451073)(820.6748331,182.31176451)
\curveto(820.50759176,182.56904002)(820.25031103,182.69768039)(819.90299013,182.69768599)
\curveto(819.64998932,182.69768039)(819.41200464,182.62049617)(819.18903538,182.4661331)
\curveto(818.96605137,182.31175929)(818.80525091,182.08449464)(818.70663353,181.78433848)
\curveto(818.60800235,181.48846095)(818.55869021,181.01892361)(818.55869696,180.37572505)
\lineto(818.55869696,177.09539243)
\lineto(816.751298,177.09539243)
\lineto(816.751298,180.83883084)
\curveto(816.75129306,181.50346899)(816.71913297,181.93227021)(816.65481763,182.12523579)
\curveto(816.5904926,182.31819131)(816.48972431,182.46183972)(816.35251246,182.55618145)
\curveto(816.21957954,182.65051226)(816.03733902,182.69768039)(815.80579036,182.69768599)
\curveto(815.52706557,182.69768039)(815.27621685,182.62264018)(815.05324346,182.47256513)
\curveto(814.83026358,182.32247932)(814.66946313,182.1059347)(814.5708416,181.82293063)
\curveto(814.47650258,181.53991709)(814.42933444,181.07037975)(814.42933706,180.4143172)
\lineto(814.42933706,177.09539243)
\lineto(812.6219381,177.09539243)
\lineto(812.6219381,183.92620272)
}
}
{
\newrgbcolor{curcolor}{0 0 0}
\pscustom[linestyle=none,fillstyle=solid,fillcolor=curcolor]
{
\newpath
\moveto(823.82652565,179.04429593)
\lineto(825.64035663,179.32087299)
\curveto(825.71753873,178.96925377)(825.87405117,178.701253)(826.10989444,178.5168699)
\curveto(826.34573252,178.33677196)(826.67590946,178.24672371)(827.10042625,178.24672486)
\curveto(827.567816,178.24672371)(827.919433,178.33248395)(828.15527831,178.50400585)
\curveto(828.31393013,178.62406878)(828.39325835,178.78486924)(828.39326323,178.98640771)
\curveto(828.39325835,179.12362221)(828.35037823,179.23725453)(828.26462273,179.32730502)
\curveto(828.17456973,179.41306303)(827.97303315,179.49239126)(827.6600124,179.56528993)
\curveto(826.20208411,179.88688838)(825.27801747,180.18061722)(824.88780973,180.44647733)
\curveto(824.34751882,180.81524303)(824.07737405,181.32766049)(824.07737461,181.98373124)
\curveto(824.07737405,182.57547204)(824.31107072,183.07288146)(824.77846531,183.47596099)
\curveto(825.24585738,183.87902776)(825.97053145,184.08056433)(826.95248968,184.08057132)
\curveto(827.88727291,184.08056433)(828.58193089,183.9283399)(829.0364657,183.62389756)
\curveto(829.49098948,183.31944216)(829.80401437,182.86920088)(829.97554132,182.27317236)
\lineto(828.27105476,181.95800315)
\curveto(828.1981538,182.22385504)(828.0587934,182.42753562)(827.85297315,182.5690455)
\curveto(827.65143224,182.71054443)(827.36199141,182.78129663)(826.9846498,182.78130232)
\curveto(826.50867698,182.78129663)(826.16778001,182.71483244)(825.96195787,182.58190955)
\curveto(825.82473903,182.48756779)(825.75613084,182.36535944)(825.75613307,182.21528414)
\curveto(825.75613084,182.08663865)(825.81616301,181.97729434)(825.93622977,181.88725087)
\curveto(826.09917181,181.76718174)(826.66090141,181.59780526)(827.62142026,181.37912092)
\curveto(828.5862189,181.16042801)(829.25943682,180.89242725)(829.64107603,180.57511782)
\curveto(830.01841498,180.25351343)(830.20708752,179.80541615)(830.20709421,179.23082465)
\curveto(830.20708752,178.60477273)(829.94551878,178.06662719)(829.42238719,177.61638643)
\curveto(828.89924379,177.16614463)(828.12525759,176.94102399)(827.10042625,176.94102383)
\curveto(826.16992402,176.94102399)(825.43238591,177.12969652)(824.88780973,177.50704201)
\curveto(824.34751882,177.88438667)(823.99375781,178.39680413)(823.82652565,179.04429593)
}
}
{
\newrgbcolor{curcolor}{0 1 0.25098041}
\pscustom[linewidth=2.63455725,linecolor=curcolor]
{
\newpath
\moveto(527.33596,284.70119456)
\lineto(611.64179,284.70119456)
\lineto(611.64179,337.39235456)
\lineto(554.99881,337.39235456)
\lineto(554.99881,347.93057456)
\lineto(527.33596,347.93057456)
\lineto(527.33596,284.70119456)
\closepath
}
}
{
\newrgbcolor{curcolor}{0 0 0}
\pscustom[linestyle=none,fillstyle=solid,fillcolor=curcolor]
{
\newpath
\moveto(542.54641375,320.38215031)
\lineto(542.54641375,321.46273047)
\lineto(548.79190979,324.09986062)
\lineto(548.79190979,322.94852819)
\lineto(543.83925073,320.91600836)
\lineto(548.79190979,318.86419246)
\lineto(548.79190979,317.71286003)
\lineto(542.54641375,320.38215031)
}
}
{
\newrgbcolor{curcolor}{0 0 0}
\pscustom[linestyle=none,fillstyle=solid,fillcolor=curcolor]
{
\newpath
\moveto(550.23911557,320.38215031)
\lineto(550.23911557,321.46273047)
\lineto(556.48461161,324.09986062)
\lineto(556.48461161,322.94852819)
\lineto(551.53195254,320.91600836)
\lineto(556.48461161,318.86419246)
\lineto(556.48461161,317.71286003)
\lineto(550.23911557,320.38215031)
}
}
{
\newrgbcolor{curcolor}{0 0 0}
\pscustom[linestyle=none,fillstyle=solid,fillcolor=curcolor]
{
\newpath
\moveto(562.53714711,317.10181768)
\curveto(562.10834056,316.7373358)(561.69454739,316.48005506)(561.29576633,316.32997471)
\curveto(560.90126512,316.17989421)(560.47675192,316.10485399)(560.02222543,316.10485384)
\curveto(559.27182048,316.10485399)(558.69508284,316.28709451)(558.29201077,316.65157594)
\curveto(557.88893654,317.0203446)(557.68739997,317.48988194)(557.68740044,318.06018937)
\curveto(557.68739997,318.39465252)(557.76244018,318.69910139)(557.91252131,318.97353688)
\curveto(558.06688905,319.25225496)(558.26628162,319.4752316)(558.51069961,319.64246746)
\curveto(558.75940302,319.80969655)(559.03812382,319.93619291)(559.34686283,320.02195692)
\curveto(559.57412534,320.08198533)(559.91716632,320.13987349)(560.37598679,320.19562159)
\curveto(561.31077029,320.30710597)(561.99899625,320.44003435)(562.44066674,320.59440712)
\curveto(562.44494952,320.75305924)(562.44709353,320.85382753)(562.44709877,320.89671229)
\curveto(562.44709353,321.36838899)(562.33774922,321.70070994)(562.1190655,321.89367613)
\curveto(561.82318775,322.15523924)(561.3836665,322.28602361)(560.80050043,322.28602964)
\curveto(560.25591929,322.28602361)(559.85284614,322.18954333)(559.59127977,321.99658852)
\curveto(559.33399666,321.80791025)(559.14318011,321.47130129)(559.01882957,320.98676063)
\lineto(557.88679321,321.14112923)
\curveto(557.98970483,321.62566973)(558.15908131,322.01587884)(558.39492317,322.31175773)
\curveto(558.63076265,322.61191254)(558.97165963,322.84132119)(559.4176151,322.99998438)
\curveto(559.86356617,323.16292211)(560.38027164,323.24439434)(560.96773307,323.24440133)
\curveto(561.55089898,323.24439434)(562.02472433,323.17578614)(562.38921054,323.03857653)
\curveto(562.7536864,322.90135336)(563.02168717,322.72768887)(563.19321364,322.51758253)
\curveto(563.36472815,322.31175168)(563.48479249,322.05018294)(563.55340703,321.73287551)
\curveto(563.59199279,321.53562147)(563.61128885,321.17971646)(563.61129525,320.6651594)
\lineto(563.61129525,319.12147345)
\curveto(563.61128885,318.04517952)(563.63487292,317.36338558)(563.68204752,317.07608958)
\curveto(563.7334972,316.79307995)(563.83212148,316.52079118)(563.97792066,316.25922243)
\lineto(562.7687,316.25922243)
\curveto(562.6486301,316.49935112)(562.57144588,316.78021592)(562.53714711,317.10181768)
\moveto(562.44066674,319.68749163)
\curveto(562.02043631,319.51596771)(561.39009852,319.3701753)(560.54965146,319.25011395)
\curveto(560.07367877,319.18150276)(559.73706981,319.10431854)(559.53982357,319.01856106)
\curveto(559.34257268,318.93279805)(559.19034825,318.80630169)(559.08314982,318.6390716)
\curveto(558.97594764,318.47612475)(558.92234749,318.29388423)(558.9223492,318.09234949)
\curveto(558.92234749,317.78361078)(559.03812382,317.52633004)(559.26967853,317.32050652)
\curveto(559.50551715,317.11468087)(559.84855812,317.01176858)(560.2988025,317.01176933)
\curveto(560.74475268,317.01176858)(561.14139381,317.10824885)(561.48872708,317.30121045)
\curveto(561.83605179,317.49845797)(562.09118852,317.76645873)(562.25413802,318.10521354)
\curveto(562.37848533,318.36678044)(562.44066151,318.75270154)(562.44066674,319.262978)
\lineto(562.44066674,319.68749163)
}
}
{
\newrgbcolor{curcolor}{0 0 0}
\pscustom[linestyle=none,fillstyle=solid,fillcolor=curcolor]
{
\newpath
\moveto(565.41226109,313.64138836)
\lineto(565.41226109,323.09003273)
\lineto(566.46711315,323.09003273)
\lineto(566.46711315,322.20241331)
\curveto(566.71581593,322.54973636)(566.99668073,322.8091611)(567.30970839,322.98068831)
\curveto(567.62273052,323.15649009)(568.0022196,323.24439434)(568.44817677,323.24440133)
\curveto(569.03134253,323.24439434)(569.545904,323.09431391)(569.99186272,322.79415959)
\curveto(570.43781054,322.4939922)(570.7744195,322.06947899)(571.0016906,321.52061869)
\curveto(571.22894879,320.97603588)(571.34258112,320.37785817)(571.34258792,319.72608378)
\curveto(571.34258112,319.02713432)(571.21608476,318.39679653)(570.96309846,317.8350685)
\curveto(570.71438733,317.27762534)(570.34990629,316.84882411)(569.86965425,316.54866355)
\curveto(569.39367956,316.25279042)(568.89198213,316.10485399)(568.36456045,316.10485384)
\curveto(567.97863553,316.10485399)(567.63130654,316.18632623)(567.32257244,316.34927078)
\curveto(567.01812079,316.51221515)(566.76727208,316.71803974)(566.57002554,316.96674516)
\lineto(566.57002554,313.64138836)
\lineto(565.41226109,313.64138836)
\moveto(566.46068112,319.63603543)
\curveto(566.46067921,318.75698955)(566.63863171,318.1073557)(566.99453918,317.68713193)
\curveto(567.35044174,317.26690531)(567.78138697,317.05679271)(568.28737615,317.0567935)
\curveto(568.80193388,317.05679271)(569.24145513,317.27333732)(569.60594123,317.70642801)
\curveto(569.97470522,318.1438038)(570.15908975,318.81916573)(570.15909536,319.73251581)
\curveto(570.15908975,320.60297881)(569.97899323,321.25475667)(569.61880528,321.68785133)
\curveto(569.26289519,322.12093514)(568.83623798,322.33747976)(568.33883235,322.33748583)
\curveto(567.84570715,322.33747976)(567.40832991,322.1059271)(567.0266993,321.64282716)
\curveto(566.64935174,321.18400447)(566.46067921,320.51507456)(566.46068112,319.63603543)
}
}
{
\newrgbcolor{curcolor}{0 0 0}
\pscustom[linestyle=none,fillstyle=solid,fillcolor=curcolor]
{
\newpath
\moveto(572.74476917,313.64138836)
\lineto(572.74476917,323.09003273)
\lineto(573.79962123,323.09003273)
\lineto(573.79962123,322.20241331)
\curveto(574.04832402,322.54973636)(574.32918882,322.8091611)(574.64221648,322.98068831)
\curveto(574.9552386,323.15649009)(575.33472769,323.24439434)(575.78068486,323.24440133)
\curveto(576.36385062,323.24439434)(576.87841208,323.09431391)(577.3243708,322.79415959)
\curveto(577.77031863,322.4939922)(578.10692759,322.06947899)(578.33419869,321.52061869)
\curveto(578.56145688,320.97603588)(578.67508921,320.37785817)(578.675096,319.72608378)
\curveto(578.67508921,319.02713432)(578.54859284,318.39679653)(578.29560654,317.8350685)
\curveto(578.04689541,317.27762534)(577.68241438,316.84882411)(577.20216233,316.54866355)
\curveto(576.72618765,316.25279042)(576.22449022,316.10485399)(575.69706854,316.10485384)
\curveto(575.31114362,316.10485399)(574.96381463,316.18632623)(574.65508053,316.34927078)
\curveto(574.35062888,316.51221515)(574.09978017,316.71803974)(573.90253363,316.96674516)
\lineto(573.90253363,313.64138836)
\lineto(572.74476917,313.64138836)
\moveto(573.79318921,319.63603543)
\curveto(573.79318729,318.75698955)(573.9711398,318.1073557)(574.32704726,317.68713193)
\curveto(574.68294983,317.26690531)(575.11389506,317.05679271)(575.61988424,317.0567935)
\curveto(576.13444196,317.05679271)(576.57396322,317.27333732)(576.93844932,317.70642801)
\curveto(577.30721331,318.1438038)(577.49159783,318.81916573)(577.49160345,319.73251581)
\curveto(577.49159783,320.60297881)(577.31150132,321.25475667)(576.95131337,321.68785133)
\curveto(576.59540328,322.12093514)(576.16874606,322.33747976)(575.67134044,322.33748583)
\curveto(575.17821524,322.33747976)(574.74083799,322.1059271)(574.35920739,321.64282716)
\curveto(573.98185983,321.18400447)(573.79318729,320.51507456)(573.79318921,319.63603543)
}
}
{
\newrgbcolor{curcolor}{0 0 0}
\pscustom[linestyle=none,fillstyle=solid,fillcolor=curcolor]
{
\newpath
\moveto(586.17483673,320.38215031)
\lineto(579.92934069,317.71286003)
\lineto(579.92934069,318.86419246)
\lineto(584.87556773,320.91600836)
\lineto(579.92934069,322.94852819)
\lineto(579.92934069,324.09986062)
\lineto(586.17483673,321.46273047)
\lineto(586.17483673,320.38215031)
}
}
{
\newrgbcolor{curcolor}{0 0 0}
\pscustom[linestyle=none,fillstyle=solid,fillcolor=curcolor]
{
\newpath
\moveto(593.86753998,320.38215031)
\lineto(587.62204394,317.71286003)
\lineto(587.62204394,318.86419246)
\lineto(592.56827098,320.91600836)
\lineto(587.62204394,322.94852819)
\lineto(587.62204394,324.09986062)
\lineto(593.86753998,321.46273047)
\lineto(593.86753998,320.38215031)
}
}
{
\newrgbcolor{curcolor}{0 0 0}
\pscustom[linestyle=none,fillstyle=solid,fillcolor=curcolor]
{
\newpath
\moveto(555.31401533,301.31116065)
\lineto(553.53234447,300.98955941)
\curveto(553.47230718,301.34545993)(553.33509079,301.6134607)(553.12069488,301.7935625)
\curveto(552.91057758,301.97365372)(552.63614479,302.06370198)(552.29739571,302.06370754)
\curveto(551.84715055,302.06370198)(551.48695752,301.90718953)(551.21681555,301.59416974)
\curveto(550.95095599,301.28542776)(550.81802761,300.76657828)(550.81803002,300.03761974)
\curveto(550.81802761,299.2271819)(550.9531,298.65473227)(551.22324758,298.32026913)
\curveto(551.49767755,297.98580236)(551.86430259,297.81856988)(552.32312381,297.8185712)
\curveto(552.66616088,297.81856988)(552.94702568,297.91505016)(553.16571906,298.10801232)
\curveto(553.38440293,298.30525927)(553.53877137,298.64186823)(553.62882484,299.1178402)
\lineto(555.40406367,298.81553504)
\curveto(555.21967216,298.0008104)(554.86591115,297.38548065)(554.34277959,296.96954393)
\curveto(553.81963617,296.55360628)(553.11854617,296.34563769)(552.23950749,296.34563753)
\curveto(551.24039682,296.34563769)(550.44282654,296.66080658)(549.84679428,297.29114517)
\curveto(549.25504716,297.92148218)(548.95917432,298.79409266)(548.95917486,299.90897925)
\curveto(548.95917432,301.03672305)(549.25719117,301.91362155)(549.8532263,302.53967738)
\curveto(550.44925856,303.17000913)(551.25540486,303.48517803)(552.27166762,303.48518502)
\curveto(553.10353813,303.48517803)(553.76389201,303.30508152)(554.25273124,302.94489494)
\curveto(554.74584681,302.58898348)(555.09960781,302.04440593)(555.31401533,301.31116065)
}
}
{
\newrgbcolor{curcolor}{0 0 0}
\pscustom[linestyle=none,fillstyle=solid,fillcolor=curcolor]
{
\newpath
\moveto(558.04119369,301.2468404)
\lineto(556.40102737,301.54271354)
\curveto(556.58541124,302.20306238)(556.90272414,302.69189577)(557.35296704,303.00921518)
\curveto(557.80320671,303.32652158)(558.47213662,303.48517803)(559.35975876,303.48518502)
\curveto(560.16590144,303.48517803)(560.76622315,303.38869776)(561.16072569,303.1957439)
\curveto(561.5552174,303.00706467)(561.83179419,302.76479198)(561.99045689,302.4689251)
\curveto(562.15339511,302.1773343)(562.23486734,301.63918877)(562.23487383,300.85448689)
\lineto(562.21557776,298.74478277)
\curveto(562.21557128,298.14445881)(562.24344336,297.70064955)(562.29919408,297.41335364)
\curveto(562.35921969,297.13034392)(562.46856401,296.82589505)(562.62722734,296.50000613)
\lineto(560.83912446,296.50000613)
\curveto(560.79195123,296.62007047)(560.73406306,296.79802297)(560.66545979,297.03386418)
\curveto(560.63543878,297.14106395)(560.61399872,297.21181615)(560.60113954,297.246121)
\curveto(560.2923978,296.9459594)(559.96222086,296.72083875)(559.61060773,296.5707584)
\curveto(559.25898686,296.4206779)(558.88378579,296.34563769)(558.48500339,296.34563753)
\curveto(557.78176665,296.34563769)(557.22646907,296.53645423)(556.81910898,296.91808773)
\curveto(556.41603476,297.2997204)(556.21449818,297.78212178)(556.21449865,298.36529331)
\curveto(556.21449818,298.75121254)(556.30669045,299.09425352)(556.49107572,299.39441727)
\curveto(556.6754595,299.69886324)(556.93274023,299.9304159)(557.26291869,300.08907594)
\curveto(557.59738212,300.25201682)(558.07763949,300.39352122)(558.70369224,300.51358958)
\curveto(559.54842768,300.67224202)(560.13374135,300.82017844)(560.459635,300.95739929)
\lineto(560.459635,301.13749598)
\curveto(560.45963028,301.48482033)(560.37387004,301.73138103)(560.20235401,301.87717883)
\curveto(560.03082906,302.02725388)(559.70708414,302.10229409)(559.23111827,302.10229969)
\curveto(558.90951386,302.10229409)(558.65866515,302.03797391)(558.47857137,301.90933895)
\curveto(558.29847212,301.78498119)(558.15267971,301.56414856)(558.04119369,301.2468404)
\moveto(560.459635,299.78033875)
\curveto(560.22807762,299.70315125)(559.86145258,299.61095899)(559.35975876,299.50376169)
\curveto(558.85805772,299.39655838)(558.53002478,299.29150208)(558.37565897,299.18859248)
\curveto(558.13981567,299.02135731)(558.02189533,298.80910071)(558.02189761,298.55182202)
\curveto(558.02189533,298.29882725)(558.1162316,298.08013863)(558.3049067,297.8957555)
\curveto(558.49357668,297.71136958)(558.73370536,297.61917731)(559.02529347,297.61917843)
\curveto(559.35117912,297.61917731)(559.66206001,297.72637762)(559.95793706,297.94077967)
\curveto(560.17662147,298.1037227)(560.32026988,298.30311526)(560.38888272,298.53895797)
\curveto(560.43604621,298.69332438)(560.45963028,298.98705321)(560.459635,299.42014537)
\lineto(560.459635,299.78033875)
}
}
{
\newrgbcolor{curcolor}{0 0 0}
\pscustom[linestyle=none,fillstyle=solid,fillcolor=curcolor]
{
\newpath
\moveto(565.75319123,296.50000613)
\lineto(563.94579228,296.50000613)
\lineto(563.94579228,303.33081642)
\lineto(565.62455074,303.33081642)
\lineto(565.62455074,302.35958068)
\curveto(565.91184501,302.81839213)(566.16912574,303.12069699)(566.39639371,303.26649617)
\curveto(566.62794305,303.41228182)(566.8895118,303.48517803)(567.18110073,303.48518502)
\curveto(567.5927458,303.48517803)(567.98938693,303.37154571)(568.37102531,303.1442877)
\lineto(567.81143916,301.56844164)
\curveto(567.50698556,301.76568513)(567.22397675,301.86430941)(566.96241189,301.86431478)
\curveto(566.70941528,301.86430941)(566.49501467,301.79355721)(566.31920941,301.65205796)
\curveto(566.14339767,301.51483642)(566.00403727,301.2639877)(565.9011278,300.89951106)
\curveto(565.8025007,300.53502562)(565.75318856,299.77175945)(565.75319123,298.60971025)
\lineto(565.75319123,296.50000613)
}
}
{
\newrgbcolor{curcolor}{0 0 0}
\pscustom[linestyle=none,fillstyle=solid,fillcolor=curcolor]
{
\newpath
\moveto(573.09856392,298.67403049)
\lineto(574.89953085,298.37172533)
\curveto(574.66797149,297.71136958)(574.30134645,297.20752814)(573.79965462,296.86019951)
\curveto(573.3022396,296.51715817)(572.67833382,296.34563769)(571.92793541,296.34563753)
\curveto(570.7401523,296.34563769)(569.86110979,296.73370279)(569.29080526,297.50983401)
\curveto(568.84056289,298.13159477)(568.61544224,298.91630101)(568.61544266,299.86395508)
\curveto(568.61544224,300.99598694)(568.91131509,301.88146146)(569.50306208,302.5203813)
\curveto(570.09480646,303.16357711)(570.84306459,303.48517803)(571.74783872,303.48518502)
\curveto(572.76409407,303.48517803)(573.56595235,303.14856907)(574.15341598,302.47535713)
\curveto(574.7408677,301.80642125)(575.0217325,300.77944232)(574.99601123,299.39441727)
\lineto(570.46786579,299.39441727)
\curveto(570.48072756,298.85841285)(570.62651998,298.44033165)(570.90524348,298.14017244)
\curveto(571.18396156,297.84429796)(571.53129055,297.69636153)(571.94723149,297.69636273)
\curveto(572.23023655,297.69636153)(572.46822122,297.77354575)(572.66118624,297.92791562)
\curveto(572.85414232,298.08228263)(572.99993474,298.33098734)(573.09856392,298.67403049)
\moveto(573.20147632,300.50072553)
\curveto(573.18860728,301.02385902)(573.05353489,301.42050015)(572.79625876,301.69065011)
\curveto(572.53897343,301.9650777)(572.22594853,302.10229409)(571.85718314,302.10229969)
\curveto(571.46268236,302.10229409)(571.13679343,301.95864568)(570.87951538,301.67135403)
\curveto(570.62223196,301.38405204)(570.4957356,300.99384293)(570.50002592,300.50072553)
\lineto(573.20147632,300.50072553)
}
}
{
\newrgbcolor{curcolor}{0 0 0}
\pscustom[linestyle=none,fillstyle=solid,fillcolor=curcolor]
{
\newpath
\moveto(580.43107201,298.67403049)
\lineto(582.23203894,298.37172533)
\curveto(582.00047958,297.71136958)(581.63385453,297.20752814)(581.13216271,296.86019951)
\curveto(580.63474769,296.51715817)(580.01084191,296.34563769)(579.2604435,296.34563753)
\curveto(578.07266039,296.34563769)(577.19361788,296.73370279)(576.62331335,297.50983401)
\curveto(576.17307097,298.13159477)(575.94795033,298.91630101)(575.94795075,299.86395508)
\curveto(575.94795033,300.99598694)(576.24382317,301.88146146)(576.83557017,302.5203813)
\curveto(577.42731455,303.16357711)(578.17557268,303.48517803)(579.08034681,303.48518502)
\curveto(580.09660215,303.48517803)(580.89846044,303.14856907)(581.48592407,302.47535713)
\curveto(582.07337579,301.80642125)(582.35424059,300.77944232)(582.32851931,299.39441727)
\lineto(577.80037388,299.39441727)
\curveto(577.81323565,298.85841285)(577.95902806,298.44033165)(578.23775156,298.14017244)
\curveto(578.51646965,297.84429796)(578.86379864,297.69636153)(579.27973958,297.69636273)
\curveto(579.56274463,297.69636153)(579.80072931,297.77354575)(579.99369432,297.92791562)
\curveto(580.18665041,298.08228263)(580.33244283,298.33098734)(580.43107201,298.67403049)
\moveto(580.5339844,300.50072553)
\curveto(580.52111536,301.02385902)(580.38604298,301.42050015)(580.12876684,301.69065011)
\curveto(579.87148151,301.9650777)(579.55845662,302.10229409)(579.18969123,302.10229969)
\curveto(578.79519045,302.10229409)(578.46930152,301.95864568)(578.21202347,301.67135403)
\curveto(577.95474005,301.38405204)(577.82824369,300.99384293)(577.832534,300.50072553)
\lineto(580.5339844,300.50072553)
}
}
{
\newrgbcolor{curcolor}{0 0 0}
\pscustom[linestyle=none,fillstyle=solid,fillcolor=curcolor]
{
\newpath
\moveto(585.53809953,296.50000613)
\lineto(583.73070057,296.50000613)
\lineto(583.73070057,303.33081642)
\lineto(585.40945903,303.33081642)
\lineto(585.40945903,302.35958068)
\curveto(585.6967533,302.81839213)(585.95403404,303.12069699)(586.181302,303.26649617)
\curveto(586.41285134,303.41228182)(586.67442009,303.48517803)(586.96600902,303.48518502)
\curveto(587.37765409,303.48517803)(587.77429522,303.37154571)(588.15593361,303.1442877)
\lineto(587.59634745,301.56844164)
\curveto(587.29189385,301.76568513)(587.00888504,301.86430941)(586.74732018,301.86431478)
\curveto(586.49432358,301.86430941)(586.27992297,301.79355721)(586.10411771,301.65205796)
\curveto(585.92830596,301.51483642)(585.78894557,301.2639877)(585.6860361,300.89951106)
\curveto(585.58740899,300.53502562)(585.53809685,299.77175945)(585.53809953,298.60971025)
\lineto(585.53809953,296.50000613)
}
}
{
\newrgbcolor{curcolor}{0.50196081 0.50196081 0}
\pscustom[linewidth=2.63455725,linecolor=curcolor]
{
\newpath
\moveto(527.33596,165.29658456)
\lineto(611.64179,165.29658456)
\lineto(611.64179,217.98773456)
\lineto(554.99881,217.98773456)
\lineto(554.99881,228.52596456)
\lineto(527.33596,228.52596456)
\lineto(527.33596,165.29658456)
\closepath
}
}
{
\newrgbcolor{curcolor}{0 0 0}
\pscustom[linestyle=none,fillstyle=solid,fillcolor=curcolor]
{
\newpath
\moveto(545.18093623,199.38882861)
\lineto(545.18093623,200.46940877)
\lineto(551.42643228,203.10653892)
\lineto(551.42643228,201.95520649)
\lineto(546.47377321,199.92268666)
\lineto(551.42643228,197.87087076)
\lineto(551.42643228,196.71953833)
\lineto(545.18093623,199.38882861)
}
}
{
\newrgbcolor{curcolor}{0 0 0}
\pscustom[linestyle=none,fillstyle=solid,fillcolor=curcolor]
{
\newpath
\moveto(552.87363805,199.38882861)
\lineto(552.87363805,200.46940877)
\lineto(559.11913409,203.10653892)
\lineto(559.11913409,201.95520649)
\lineto(554.16647503,199.92268666)
\lineto(559.11913409,197.87087076)
\lineto(559.11913409,196.71953833)
\lineto(552.87363805,199.38882861)
}
}
{
\newrgbcolor{curcolor}{0 0 0}
\pscustom[linestyle=none,fillstyle=solid,fillcolor=curcolor]
{
\newpath
\moveto(565.19096567,195.26590073)
\lineto(565.19096567,196.2692966)
\curveto(564.65924681,195.49745339)(563.93671675,195.11153229)(563.02337333,195.11153214)
\curveto(562.620297,195.11153229)(562.24295193,195.18871651)(561.89133697,195.34308503)
\curveto(561.54400593,195.49745339)(561.28458119,195.69041394)(561.11306197,195.92196726)
\curveto(560.94582823,196.15780728)(560.82790789,196.44510409)(560.75930061,196.78385858)
\curveto(560.71213156,197.01112171)(560.6885475,197.37131473)(560.68854834,197.86443874)
\lineto(560.68854834,202.09671103)
\lineto(561.8463128,202.09671103)
\lineto(561.8463128,198.30824845)
\curveto(561.84631079,197.70363568)(561.86989486,197.29627452)(561.91706507,197.08616374)
\curveto(561.9899592,196.78171305)(562.14432764,196.54158437)(562.38017085,196.36577697)
\curveto(562.61600899,196.19425538)(562.90759382,196.10849514)(563.25492622,196.10849598)
\curveto(563.6022518,196.10849514)(563.92814073,196.19639939)(564.23259398,196.37220899)
\curveto(564.53703846,196.5523044)(564.75143907,196.79457709)(564.87579646,197.09902779)
\curveto(565.00443179,197.40776284)(565.06875198,197.85371611)(565.0687572,198.43688894)
\lineto(565.0687572,202.09671103)
\lineto(566.22652166,202.09671103)
\lineto(566.22652166,195.26590073)
\lineto(565.19096567,195.26590073)
}
}
{
\newrgbcolor{curcolor}{0 0 0}
\pscustom[linestyle=none,fillstyle=solid,fillcolor=curcolor]
{
\newpath
\moveto(570.5745693,196.30145672)
\lineto(570.74180195,195.27876478)
\curveto(570.41590945,195.21015658)(570.12432462,195.17585248)(569.86704658,195.17585239)
\curveto(569.44681869,195.17585248)(569.12092976,195.24231667)(568.88937881,195.37524516)
\curveto(568.65782444,195.50817342)(568.49487998,195.68183792)(568.40054493,195.89623916)
\curveto(568.30620744,196.11492715)(568.25903931,196.57160046)(568.25904039,197.26626044)
\lineto(568.25904039,201.19622756)
\lineto(567.41001312,201.19622756)
\lineto(567.41001312,202.09671103)
\lineto(568.25904039,202.09671103)
\lineto(568.25904039,203.78833354)
\lineto(569.41037282,204.48299222)
\lineto(569.41037282,202.09671103)
\lineto(570.5745693,202.09671103)
\lineto(570.5745693,201.19622756)
\lineto(569.41037282,201.19622756)
\lineto(569.41037282,197.20194019)
\curveto(569.41037059,196.87176131)(569.42966664,196.65950471)(569.46826104,196.56516974)
\curveto(569.51113888,196.47083217)(569.57760306,196.39579195)(569.66765381,196.34004887)
\curveto(569.76198759,196.28430364)(569.89491597,196.25643156)(570.06643935,196.25643255)
\curveto(570.19507682,196.25643156)(570.36445331,196.2714396)(570.5745693,196.30145672)
}
}
{
\newrgbcolor{curcolor}{0 0 0}
\pscustom[linestyle=none,fillstyle=solid,fillcolor=curcolor]
{
\newpath
\moveto(571.70660648,203.36381991)
\lineto(571.70660648,204.69524904)
\lineto(572.86437094,204.69524904)
\lineto(572.86437094,203.36381991)
\lineto(571.70660648,203.36381991)
\moveto(571.70660648,195.26590073)
\lineto(571.70660648,202.09671103)
\lineto(572.86437094,202.09671103)
\lineto(572.86437094,195.26590073)
\lineto(571.70660648,195.26590073)
}
}
{
\newrgbcolor{curcolor}{0 0 0}
\pscustom[linestyle=none,fillstyle=solid,fillcolor=curcolor]
{
\newpath
\moveto(574.60745074,195.26590073)
\lineto(574.60745074,204.69524904)
\lineto(575.76521519,204.69524904)
\lineto(575.76521519,195.26590073)
\lineto(574.60745074,195.26590073)
}
}
{
\newrgbcolor{curcolor}{0 0 0}
\pscustom[linestyle=none,fillstyle=solid,fillcolor=curcolor]
{
\newpath
\moveto(583.66374078,199.38882861)
\lineto(577.41824474,196.71953833)
\lineto(577.41824474,197.87087076)
\lineto(582.36447178,199.92268666)
\lineto(577.41824474,201.95520649)
\lineto(577.41824474,203.10653892)
\lineto(583.66374078,200.46940877)
\lineto(583.66374078,199.38882861)
}
}
{
\newrgbcolor{curcolor}{0 0 0}
\pscustom[linestyle=none,fillstyle=solid,fillcolor=curcolor]
{
\newpath
\moveto(591.35644403,199.38882861)
\lineto(585.11094798,196.71953833)
\lineto(585.11094798,197.87087076)
\lineto(590.05717503,199.92268666)
\lineto(585.11094798,201.95520649)
\lineto(585.11094798,203.10653892)
\lineto(591.35644403,200.46940877)
\lineto(591.35644403,199.38882861)
}
}
{
\newrgbcolor{curcolor}{0 0 0}
\pscustom[linestyle=none,fillstyle=solid,fillcolor=curcolor]
{
\newpath
\moveto(538.65252075,175.05644269)
\lineto(540.71720069,174.80559373)
\curveto(540.75150195,174.56546574)(540.83083018,174.40037727)(540.95518561,174.31032782)
\curveto(541.12670302,174.18168865)(541.39684779,174.11736847)(541.76562073,174.11736708)
\curveto(542.23729818,174.11736847)(542.59105919,174.18812067)(542.82690482,174.32962389)
\curveto(542.98555632,174.42396134)(543.10562066,174.57618577)(543.1870982,174.78629765)
\curveto(543.24283705,174.9363788)(543.27070913,175.21295559)(543.27071453,175.61602885)
\lineto(543.27071453,176.61299268)
\curveto(542.73041959,175.87545348)(542.04862565,175.50668443)(541.22533065,175.50668443)
\curveto(540.30769268,175.50668443)(539.58087461,175.89474953)(539.04487426,176.67088091)
\curveto(538.62464789,177.28406549)(538.41453529,178.04733167)(538.41453583,178.96068172)
\curveto(538.41453529,180.10557753)(538.68896807,180.98033202)(539.237835,181.58494783)
\curveto(539.79098721,182.18955147)(540.47706917,182.49185633)(541.29608292,182.49186332)
\curveto(542.14081791,182.49185633)(542.83761989,182.12094328)(543.38649097,181.37912303)
\lineto(543.38649097,182.33749472)
\lineto(545.07811348,182.33749472)
\lineto(545.07811348,176.20777512)
\curveto(545.07810628,175.40162813)(545.01164209,174.79916241)(544.87872072,174.40037617)
\curveto(544.74578533,174.00159214)(544.5592568,173.68856724)(544.31913456,173.46130055)
\curveto(544.07899943,173.23403795)(543.75739852,173.05608544)(543.35433085,172.9274425)
\curveto(542.95554023,172.79880471)(542.44955479,172.73448453)(541.836373,172.73448175)
\curveto(540.67860574,172.73448453)(539.8574514,172.93387709)(539.37290752,173.33266006)
\curveto(538.88836064,173.72715935)(538.64608795,174.22885678)(538.64608872,174.83775385)
\curveto(538.64608795,174.89778669)(538.64823196,174.9706829)(538.65252075,175.05644269)
\moveto(540.26695896,179.06359412)
\curveto(540.26695657,178.3389165)(540.40631697,177.80720298)(540.68504057,177.46845198)
\curveto(540.96804657,177.13398506)(541.31537556,176.96675259)(541.72702858,176.96675405)
\curveto(542.16868999,176.96675259)(542.54174705,177.13827308)(542.84620089,177.48131603)
\curveto(543.15064479,177.82864304)(543.30286922,178.3410605)(543.30287465,179.01856995)
\curveto(543.30286922,179.72608845)(543.15707681,180.25136995)(542.86549697,180.59441601)
\curveto(542.57390714,180.9374519)(542.20513809,181.10897239)(541.75918871,181.10897799)
\curveto(541.32609559,181.10897239)(540.96804657,180.93959591)(540.68504057,180.60084804)
\curveto(540.40631697,180.26637799)(540.26695657,179.75396053)(540.26695896,179.06359412)
}
}
{
\newrgbcolor{curcolor}{0 0 0}
\pscustom[linestyle=none,fillstyle=solid,fillcolor=curcolor]
{
\newpath
\moveto(548.60286264,175.50668443)
\lineto(546.79546369,175.50668443)
\lineto(546.79546369,182.33749472)
\lineto(548.47422215,182.33749472)
\lineto(548.47422215,181.36625898)
\curveto(548.76151642,181.82507043)(549.01879715,182.12737529)(549.24606512,182.27317448)
\curveto(549.47761446,182.41896012)(549.73918321,182.49185633)(550.03077214,182.49186332)
\curveto(550.44241721,182.49185633)(550.83905834,182.37822401)(551.22069672,182.150966)
\lineto(550.66111057,180.57511994)
\curveto(550.35665697,180.77236343)(550.07364816,180.87098771)(549.8120833,180.87099308)
\curveto(549.55908669,180.87098771)(549.34468608,180.80023551)(549.16888082,180.65873626)
\curveto(548.99306908,180.52151472)(548.85370868,180.270666)(548.75079921,179.90618936)
\curveto(548.65217211,179.54170392)(548.60285997,178.77843775)(548.60286264,177.61638855)
\lineto(548.60286264,175.50668443)
}
}
{
\newrgbcolor{curcolor}{0 0 0}
\pscustom[linestyle=none,fillstyle=solid,fillcolor=curcolor]
{
\newpath
\moveto(551.57445849,179.01856995)
\curveto(551.57445797,179.61888814)(551.72239439,180.1999138)(552.0182682,180.76164866)
\curveto(552.31414007,181.323373)(552.73222127,181.75217422)(553.27251303,182.04805361)
\curveto(553.81708836,182.34391991)(554.42384209,182.49185633)(555.09277604,182.49186332)
\curveto(556.12618294,182.49185633)(556.97306535,182.15524737)(557.63342582,181.48203543)
\curveto(558.29377311,180.81309955)(558.62395006,179.96621713)(558.62395763,178.94138565)
\curveto(558.62395006,177.90797127)(558.2894851,177.05036882)(557.62056177,176.36857574)
\curveto(556.9559133,175.69106895)(556.11760691,175.35231599)(555.10564009,175.35231583)
\curveto(554.47958624,175.35231599)(553.88140854,175.49382039)(553.31110518,175.77682947)
\curveto(552.7450853,176.059838)(552.31414007,176.47363118)(552.0182682,177.01821024)
\curveto(551.72239439,177.5670743)(551.57445797,178.2338602)(551.57445849,179.01856995)
\moveto(553.42688162,178.92208957)
\curveto(553.42687924,178.24458023)(553.5876797,177.72573075)(553.90928348,177.36553958)
\curveto(554.23088154,177.0053447)(554.62752267,176.82524818)(555.09920806,176.8252495)
\curveto(555.57088535,176.82524818)(555.96538248,177.0053447)(556.28270062,177.36553958)
\curveto(556.6042963,177.72573075)(556.76509676,178.24886824)(556.76510248,178.93495362)
\curveto(556.76509676,179.6038801)(556.6042963,180.11844157)(556.28270062,180.47863957)
\curveto(555.96538248,180.83882762)(555.57088535,181.01892413)(555.09920806,181.01892965)
\curveto(554.62752267,181.01892413)(554.23088154,180.83882762)(553.90928348,180.47863957)
\curveto(553.5876797,180.11844157)(553.42687924,179.59959209)(553.42688162,178.92208957)
}
}
{
\newrgbcolor{curcolor}{0 0 0}
\pscustom[linestyle=none,fillstyle=solid,fillcolor=curcolor]
{
\newpath
\moveto(564.54142,175.50668443)
\lineto(564.54142,176.52937636)
\curveto(564.29270985,176.1648943)(563.96467692,175.87759748)(563.55732022,175.66748504)
\curveto(563.15424261,175.45737229)(562.72758539,175.35231599)(562.27734729,175.35231583)
\curveto(561.8185268,175.35231599)(561.40687763,175.45308427)(561.04239853,175.654621)
\curveto(560.67791555,175.85615742)(560.4142028,176.13916623)(560.25125949,176.50364826)
\curveto(560.08831387,176.86812831)(560.00684164,177.37196974)(560.00684255,178.01517408)
\lineto(560.00684255,182.33749472)
\lineto(561.8142415,182.33749472)
\lineto(561.8142415,179.19866664)
\curveto(561.81423879,178.23814821)(561.84639888,177.64854653)(561.91072188,177.42985983)
\curveto(561.97932726,177.2154573)(562.10153561,177.04393681)(562.27734729,176.91529785)
\curveto(562.45315261,176.79094409)(562.67612925,176.72876791)(562.94627786,176.72876913)
\curveto(563.2550109,176.72876791)(563.53158768,176.81238415)(563.77600906,176.9796181)
\curveto(564.02042108,177.15113711)(564.18765355,177.36124971)(564.27770699,177.60995652)
\curveto(564.36775007,177.86294714)(564.4127742,178.47827689)(564.41277951,179.45594763)
\lineto(564.41277951,182.33749472)
\lineto(566.22017847,182.33749472)
\lineto(566.22017847,175.50668443)
\lineto(564.54142,175.50668443)
}
}
{
\newrgbcolor{curcolor}{0 0 0}
\pscustom[linestyle=none,fillstyle=solid,fillcolor=curcolor]
{
\newpath
\moveto(568.04687309,182.33749472)
\lineto(569.73206358,182.33749472)
\lineto(569.73206358,181.33409886)
\curveto(569.95074962,181.67713401)(570.24662246,181.9558548)(570.61968299,182.17026208)
\curveto(570.99273659,182.38465603)(571.40652977,182.49185633)(571.86106377,182.49186332)
\curveto(572.65434133,182.49185633)(573.32755924,182.18097545)(573.88071955,181.55921973)
\curveto(574.4338664,180.9374519)(574.71044318,180.07127343)(574.71045074,178.96068172)
\curveto(574.71044318,177.82006702)(574.43172239,176.93244849)(573.87428752,176.29782347)
\curveto(573.31683921,175.66748488)(572.64147729,175.35231599)(571.84819972,175.35231583)
\curveto(571.47084995,175.35231599)(571.12780898,175.4273562)(570.81907576,175.5774367)
\curveto(570.51462323,175.72751706)(570.19302231,175.98479779)(569.85427205,176.34927967)
\lineto(569.85427205,172.90814642)
\lineto(568.04687309,172.90814642)
\lineto(568.04687309,182.33749472)
\moveto(569.83497597,179.03786602)
\curveto(569.83497329,178.2703083)(569.98719772,177.70214668)(570.29164973,177.33337946)
\curveto(570.59609546,176.96889659)(570.96700852,176.78665607)(571.40439001,176.78665735)
\curveto(571.82461096,176.78665607)(572.17408396,176.95388855)(572.45281005,177.28835528)
\curveto(572.73152555,177.62710647)(572.87088594,178.18026004)(572.87089166,178.94781767)
\curveto(572.87088594,179.66391227)(572.72723753,180.19562579)(572.439946,180.54295981)
\curveto(572.1526439,180.89028377)(571.79673888,181.06394826)(571.37222989,181.06395382)
\curveto(570.93056041,181.06394826)(570.56393537,180.89242777)(570.27235366,180.54939184)
\curveto(569.98076571,180.21063383)(569.83497329,179.7067924)(569.83497597,179.03786602)
}
}
{
\newrgbcolor{curcolor}{0 0 0}
\pscustom[linestyle=none,fillstyle=solid,fillcolor=curcolor]
{
\newpath
\moveto(575.51445343,177.45558793)
\lineto(577.32828441,177.73216499)
\curveto(577.40546651,177.38054577)(577.56197895,177.112545)(577.79782222,176.9281619)
\curveto(578.0336603,176.74806396)(578.36383724,176.65801571)(578.78835403,176.65801686)
\curveto(579.25574378,176.65801571)(579.60736078,176.74377595)(579.84320609,176.91529785)
\curveto(580.00185791,177.03536078)(580.08118613,177.19616124)(580.08119101,177.39769971)
\curveto(580.08118613,177.53491421)(580.03830601,177.64854653)(579.95255051,177.73859702)
\curveto(579.86249751,177.82435503)(579.66096094,177.90368326)(579.34794019,177.97658193)
\curveto(577.89001189,178.29818038)(576.96594526,178.59190922)(576.57573751,178.85776933)
\curveto(576.0354466,179.22653503)(575.76530183,179.73895249)(575.76530239,180.39502324)
\curveto(575.76530183,180.98676404)(575.9989985,181.48417346)(576.46639309,181.88725299)
\curveto(576.93378516,182.29031976)(577.65845923,182.49185633)(578.64041746,182.49186332)
\curveto(579.57520069,182.49185633)(580.26985867,182.3396319)(580.72439349,182.03518956)
\curveto(581.17891726,181.73073416)(581.49194215,181.28049288)(581.6634691,180.68446436)
\lineto(579.95898254,180.36929515)
\curveto(579.88608158,180.63514704)(579.74672118,180.83882762)(579.54090093,180.9803375)
\curveto(579.33936002,181.12183643)(579.04991919,181.19258863)(578.67257759,181.19259432)
\curveto(578.19660476,181.19258863)(577.85570779,181.12612444)(577.64988565,180.99320155)
\curveto(577.51266681,180.89885979)(577.44405862,180.77665144)(577.44406086,180.62657614)
\curveto(577.44405862,180.49793065)(577.50409079,180.38858634)(577.62415755,180.29854287)
\curveto(577.7870996,180.17847374)(578.3488292,180.00909726)(579.30934804,179.79041292)
\curveto(580.27414668,179.57172001)(580.9473646,179.30371925)(581.32900381,178.98640982)
\curveto(581.70634277,178.66480543)(581.8950153,178.21670815)(581.89502199,177.64211665)
\curveto(581.8950153,177.01606473)(581.63344656,176.47791919)(581.11031497,176.02767843)
\curveto(580.58717158,175.57743663)(579.81318537,175.35231599)(578.78835403,175.35231583)
\curveto(577.8578518,175.35231599)(577.1203137,175.54098852)(576.57573751,175.91833401)
\curveto(576.0354466,176.29567867)(575.6816856,176.80809613)(575.51445343,177.45558793)
}
}
{
\newrgbcolor{curcolor}{0 0 0}
\pscustom[linestyle=none,fillstyle=solid,fillcolor=curcolor]
{
\newpath
\moveto(587.4394272,177.6807088)
\lineto(589.24039413,177.37840363)
\curveto(589.00883477,176.71804788)(588.64220972,176.21420644)(588.14051789,175.86687781)
\curveto(587.64310287,175.52383647)(587.0191971,175.35231599)(586.26879869,175.35231583)
\curveto(585.08101557,175.35231599)(584.20197307,175.74038109)(583.63166854,176.51651231)
\curveto(583.18142616,177.13827308)(582.95630552,177.92297931)(582.95630594,178.87063338)
\curveto(582.95630552,180.00266524)(583.25217836,180.88813976)(583.84392535,181.5270596)
\curveto(584.43566973,182.17025542)(585.18392787,182.49185633)(586.088702,182.49186332)
\curveto(587.10495734,182.49185633)(587.90681563,182.15524737)(588.49427926,181.48203543)
\curveto(589.08173097,180.81309955)(589.36259577,179.78612062)(589.3368745,178.40109557)
\lineto(584.80872907,178.40109557)
\curveto(584.82159083,177.86509115)(584.96738325,177.44700996)(585.24610675,177.14685074)
\curveto(585.52482484,176.85097626)(585.87215383,176.70303984)(586.28809476,176.70304103)
\curveto(586.57109982,176.70303984)(586.8090845,176.78022406)(587.00204951,176.93459392)
\curveto(587.1950056,177.08896093)(587.34079801,177.33766564)(587.4394272,177.6807088)
\moveto(587.54233959,179.50740383)
\curveto(587.52947055,180.03053732)(587.39439817,180.42717845)(587.13712203,180.69732841)
\curveto(586.8798367,180.971756)(586.56681181,181.10897239)(586.19804642,181.10897799)
\curveto(585.80354563,181.10897239)(585.4776567,180.96532398)(585.22037865,180.67803233)
\curveto(584.96309524,180.39073034)(584.83659888,180.00052123)(584.84088919,179.50740383)
\lineto(587.54233959,179.50740383)
}
}
{
\newrgbcolor{curcolor}{0 0 0}
\pscustom[linestyle=none,fillstyle=solid,fillcolor=curcolor]
{
\newpath
\moveto(593.94863611,182.33749472)
\lineto(593.94863611,180.89672118)
\lineto(592.71368736,180.89672118)
\lineto(592.71368736,178.14381458)
\curveto(592.71368452,177.58637035)(592.72440455,177.26048142)(592.74584748,177.16614681)
\curveto(592.77157268,177.0760969)(592.82517283,177.00105668)(592.9066481,176.94102595)
\curveto(592.99240531,176.88099234)(593.0953176,176.85097626)(593.21538529,176.8509776)
\curveto(593.38261442,176.85097626)(593.62488711,176.90886442)(593.94220409,177.02464227)
\lineto(594.09657268,175.62246087)
\curveto(593.67634326,175.44236424)(593.2003739,175.35231599)(592.66866319,175.35231583)
\curveto(592.34277146,175.35231599)(592.04904262,175.40591614)(591.78747579,175.51311645)
\curveto(591.52590513,175.62460476)(591.33294458,175.76610917)(591.20859356,175.93763008)
\curveto(591.08852788,176.11343815)(591.00491165,176.34927883)(590.9577446,176.64515281)
\curveto(590.9191514,176.85526427)(590.89985535,177.27977748)(590.89985638,177.91869371)
\lineto(590.89985638,180.89672118)
\lineto(590.07012518,180.89672118)
\lineto(590.07012518,182.33749472)
\lineto(590.89985638,182.33749472)
\lineto(590.89985638,183.69465195)
\lineto(592.71368736,184.74950401)
\lineto(592.71368736,182.33749472)
\lineto(593.94863611,182.33749472)
}
}
{
\newrgbcolor{curcolor}{0 0 0}
\pscustom[linestyle=none,fillstyle=solid,fillcolor=curcolor]
{
\newpath
\moveto(594.57897522,177.45558793)
\lineto(596.3928062,177.73216499)
\curveto(596.4699883,177.38054577)(596.62650074,177.112545)(596.86234401,176.9281619)
\curveto(597.09818209,176.74806396)(597.42835903,176.65801571)(597.85287582,176.65801686)
\curveto(598.32026557,176.65801571)(598.67188257,176.74377595)(598.90772788,176.91529785)
\curveto(599.0663797,177.03536078)(599.14570792,177.19616124)(599.1457128,177.39769971)
\curveto(599.14570792,177.53491421)(599.1028278,177.64854653)(599.0170723,177.73859702)
\curveto(598.9270193,177.82435503)(598.72548273,177.90368326)(598.41246198,177.97658193)
\curveto(596.95453368,178.29818038)(596.03046705,178.59190922)(595.6402593,178.85776933)
\curveto(595.09996839,179.22653503)(594.82982362,179.73895249)(594.82982418,180.39502324)
\curveto(594.82982362,180.98676404)(595.06352029,181.48417346)(595.53091488,181.88725299)
\curveto(595.99830695,182.29031976)(596.72298102,182.49185633)(597.70493925,182.49186332)
\curveto(598.63972248,182.49185633)(599.33438046,182.3396319)(599.78891527,182.03518956)
\curveto(600.24343905,181.73073416)(600.55646394,181.28049288)(600.72799089,180.68446436)
\lineto(599.02350433,180.36929515)
\curveto(598.95060337,180.63514704)(598.81124297,180.83882762)(598.60542272,180.9803375)
\curveto(598.40388181,181.12183643)(598.11444098,181.19258863)(597.73709938,181.19259432)
\curveto(597.26112655,181.19258863)(596.92022958,181.12612444)(596.71440744,180.99320155)
\curveto(596.5771886,180.89885979)(596.50858041,180.77665144)(596.50858265,180.62657614)
\curveto(596.50858041,180.49793065)(596.56861258,180.38858634)(596.68867934,180.29854287)
\curveto(596.85162139,180.17847374)(597.41335099,180.00909726)(598.37386983,179.79041292)
\curveto(599.33866847,179.57172001)(600.01188639,179.30371925)(600.3935256,178.98640982)
\curveto(600.77086455,178.66480543)(600.95953709,178.21670815)(600.95954378,177.64211665)
\curveto(600.95953709,177.01606473)(600.69796835,176.47791919)(600.17483676,176.02767843)
\curveto(599.65169337,175.57743663)(598.87770716,175.35231599)(597.85287582,175.35231583)
\curveto(596.92237359,175.35231599)(596.18483549,175.54098852)(595.6402593,175.91833401)
\curveto(595.09996839,176.29567867)(594.74620739,176.80809613)(594.57897522,177.45558793)
}
}
{
\newrgbcolor{curcolor}{0 1 0.25098041}
\pscustom[linewidth=2.63455725,linecolor=curcolor]
{
\newpath
\moveto(648.84463,165.29658456)
\lineto(735.78502,165.29658456)
\lineto(735.78502,216.67045456)
\lineto(677.82477,216.67045456)
\lineto(677.82477,228.52596456)
\lineto(648.84463,228.52596456)
\lineto(648.84463,165.29658456)
\closepath
}
}
{
\newrgbcolor{curcolor}{0 0 0}
\pscustom[linestyle=none,fillstyle=solid,fillcolor=curcolor]
{
\newpath
\moveto(665.37230847,200.97754113)
\lineto(665.37230847,202.05812129)
\lineto(671.61780452,204.69525144)
\lineto(671.61780452,203.54391901)
\lineto(666.66514545,201.51139918)
\lineto(671.61780452,199.45958328)
\lineto(671.61780452,198.30825085)
\lineto(665.37230847,200.97754113)
}
}
{
\newrgbcolor{curcolor}{0 0 0}
\pscustom[linestyle=none,fillstyle=solid,fillcolor=curcolor]
{
\newpath
\moveto(673.06501029,200.97754113)
\lineto(673.06501029,202.05812129)
\lineto(679.31050633,204.69525144)
\lineto(679.31050633,203.54391901)
\lineto(674.35784727,201.51139918)
\lineto(679.31050633,199.45958328)
\lineto(679.31050633,198.30825085)
\lineto(673.06501029,200.97754113)
}
}
{
\newrgbcolor{curcolor}{0 0 0}
\pscustom[linestyle=none,fillstyle=solid,fillcolor=curcolor]
{
\newpath
\moveto(685.36304184,197.6972085)
\curveto(684.93423529,197.33272662)(684.52044211,197.07544588)(684.12166106,196.92536553)
\curveto(683.72715985,196.77528503)(683.30264664,196.70024481)(682.84812015,196.70024466)
\curveto(682.0977152,196.70024481)(681.52097756,196.88248533)(681.11790549,197.24696676)
\curveto(680.71483126,197.61573542)(680.51329469,198.08527276)(680.51329516,198.65558019)
\curveto(680.51329469,198.99004334)(680.5883349,199.29449221)(680.73841603,199.5689277)
\curveto(680.89278377,199.84764578)(681.09217634,200.07062242)(681.33659433,200.23785828)
\curveto(681.58529774,200.40508737)(681.86401854,200.53158373)(682.17275755,200.61734774)
\curveto(682.40002007,200.67737615)(682.74306104,200.73526431)(683.20188152,200.79101241)
\curveto(684.13666501,200.90249679)(684.82489098,201.03542517)(685.26656146,201.18979794)
\curveto(685.27084425,201.34845006)(685.27298825,201.44921835)(685.27299349,201.49210311)
\curveto(685.27298825,201.96377981)(685.16364394,202.29610076)(684.94496023,202.48906695)
\curveto(684.64908247,202.75063006)(684.20956122,202.88141443)(683.62639515,202.88142046)
\curveto(683.08181401,202.88141443)(682.67874086,202.78493415)(682.41717449,202.59197934)
\curveto(682.15989138,202.40330107)(681.96907484,202.06669211)(681.84472429,201.58215145)
\lineto(680.71268793,201.73652005)
\curveto(680.81559955,202.22106055)(680.98497603,202.61126966)(681.22081789,202.90714855)
\curveto(681.45665738,203.20730336)(681.79755435,203.43671201)(682.24350983,203.5953752)
\curveto(682.68946089,203.75831293)(683.20616636,203.83978516)(683.79362779,203.83979215)
\curveto(684.3767937,203.83978516)(684.85061905,203.77117696)(685.21510527,203.63396735)
\curveto(685.57958113,203.49674418)(685.84758189,203.32307969)(686.01910836,203.11297335)
\curveto(686.19062287,202.9071425)(686.31068721,202.64557376)(686.37930175,202.32826633)
\curveto(686.41788752,202.13101229)(686.43718357,201.77510728)(686.43718997,201.26055022)
\lineto(686.43718997,199.71686427)
\curveto(686.43718357,198.64057034)(686.46076764,197.9587764)(686.50794224,197.6714804)
\curveto(686.55939192,197.38847078)(686.6580162,197.116182)(686.80381538,196.85461325)
\lineto(685.59459473,196.85461325)
\curveto(685.47452483,197.09474194)(685.39734061,197.37560674)(685.36304184,197.6972085)
\moveto(685.26656146,200.28288245)
\curveto(684.84633104,200.11135854)(684.21599324,199.96556612)(683.37554618,199.84550477)
\curveto(682.89957349,199.77689358)(682.56296453,199.69970936)(682.3657183,199.61395188)
\curveto(682.16846741,199.52818887)(682.01624297,199.40169251)(681.90904454,199.23446242)
\curveto(681.80184236,199.07151557)(681.74824221,198.88927505)(681.74824392,198.68774031)
\curveto(681.74824221,198.3790016)(681.86401854,198.12172086)(682.09557326,197.91589734)
\curveto(682.33141187,197.71007169)(682.67445285,197.6071594)(683.12469722,197.60716015)
\curveto(683.5706474,197.6071594)(683.96728853,197.70363967)(684.3146218,197.89660127)
\curveto(684.66194651,198.09384879)(684.91708324,198.36184955)(685.08003275,198.70060436)
\curveto(685.20438006,198.96217126)(685.26655623,199.34809236)(685.26656146,199.85836882)
\lineto(685.26656146,200.28288245)
}
}
{
\newrgbcolor{curcolor}{0 0 0}
\pscustom[linestyle=none,fillstyle=solid,fillcolor=curcolor]
{
\newpath
\moveto(688.23815581,194.23677918)
\lineto(688.23815581,203.68542355)
\lineto(689.29300787,203.68542355)
\lineto(689.29300787,202.79780413)
\curveto(689.54171066,203.14512718)(689.82257546,203.40455192)(690.13560311,203.57607913)
\curveto(690.44862524,203.75188091)(690.82811432,203.83978516)(691.2740715,203.83979215)
\curveto(691.85723725,203.83978516)(692.37179872,203.68970473)(692.81775744,203.38955041)
\curveto(693.26370526,203.08938302)(693.60031422,202.66486981)(693.82758533,202.11600951)
\curveto(694.05484352,201.5714267)(694.16847584,200.97324899)(694.16848264,200.3214746)
\curveto(694.16847584,199.62252514)(694.04197948,198.99218735)(693.78899318,198.43045932)
\curveto(693.54028205,197.87301616)(693.17580101,197.44421493)(692.69554897,197.14405437)
\curveto(692.21957429,196.84818124)(691.71787686,196.70024481)(691.19045517,196.70024466)
\curveto(690.80453025,196.70024481)(690.45720126,196.78171705)(690.14846716,196.9446616)
\curveto(689.84401552,197.10760597)(689.5931668,197.31343056)(689.39592027,197.56213598)
\lineto(689.39592027,194.23677918)
\lineto(688.23815581,194.23677918)
\moveto(689.28657584,200.23142625)
\curveto(689.28657393,199.35238037)(689.46452644,198.70274652)(689.8204339,198.28252275)
\curveto(690.17633646,197.86229613)(690.60728169,197.65218353)(691.11327088,197.65218432)
\curveto(691.6278286,197.65218353)(692.06734985,197.86872814)(692.43183595,198.30181883)
\curveto(692.80059994,198.73919462)(692.98498447,199.41455655)(692.98499008,200.32790663)
\curveto(692.98498447,201.19836963)(692.80488795,201.85014749)(692.4447,202.28324215)
\curveto(692.08878991,202.71632596)(691.6621327,202.93287058)(691.16472708,202.93287665)
\curveto(690.67160188,202.93287058)(690.23422463,202.70131792)(689.85259402,202.23821798)
\curveto(689.47524647,201.77939529)(689.28657393,201.11046538)(689.28657584,200.23142625)
}
}
{
\newrgbcolor{curcolor}{0 0 0}
\pscustom[linestyle=none,fillstyle=solid,fillcolor=curcolor]
{
\newpath
\moveto(695.5706639,194.23677918)
\lineto(695.5706639,203.68542355)
\lineto(696.62551596,203.68542355)
\lineto(696.62551596,202.79780413)
\curveto(696.87421874,203.14512718)(697.15508354,203.40455192)(697.4681112,203.57607913)
\curveto(697.78113333,203.75188091)(698.16062241,203.83978516)(698.60657958,203.83979215)
\curveto(699.18974534,203.83978516)(699.70430681,203.68970473)(700.15026553,203.38955041)
\curveto(700.59621335,203.08938302)(700.93282231,202.66486981)(701.16009341,202.11600951)
\curveto(701.3873516,201.5714267)(701.50098393,200.97324899)(701.50099073,200.3214746)
\curveto(701.50098393,199.62252514)(701.37448757,198.99218735)(701.12150127,198.43045932)
\curveto(700.87279014,197.87301616)(700.5083091,197.44421493)(700.02805706,197.14405437)
\curveto(699.55208237,196.84818124)(699.05038494,196.70024481)(698.52296326,196.70024466)
\curveto(698.13703834,196.70024481)(697.78970935,196.78171705)(697.48097525,196.9446616)
\curveto(697.1765236,197.10760597)(696.92567489,197.31343056)(696.72842835,197.56213598)
\lineto(696.72842835,194.23677918)
\lineto(695.5706639,194.23677918)
\moveto(696.61908393,200.23142625)
\curveto(696.61908202,199.35238037)(696.79703452,198.70274652)(697.15294199,198.28252275)
\curveto(697.50884455,197.86229613)(697.93978978,197.65218353)(698.44577896,197.65218432)
\curveto(698.96033669,197.65218353)(699.39985794,197.86872814)(699.76434404,198.30181883)
\curveto(700.13310803,198.73919462)(700.31749256,199.41455655)(700.31749817,200.32790663)
\curveto(700.31749256,201.19836963)(700.13739604,201.85014749)(699.77720809,202.28324215)
\curveto(699.421298,202.71632596)(698.99464079,202.93287058)(698.49723516,202.93287665)
\curveto(698.00410996,202.93287058)(697.56673272,202.70131792)(697.18510211,202.23821798)
\curveto(696.80775455,201.77939529)(696.61908202,201.11046538)(696.61908393,200.23142625)
}
}
{
\newrgbcolor{curcolor}{0 0 0}
\pscustom[linestyle=none,fillstyle=solid,fillcolor=curcolor]
{
\newpath
\moveto(709.00073146,200.97754113)
\lineto(702.75523541,198.30825085)
\lineto(702.75523541,199.45958328)
\lineto(707.70146246,201.51139918)
\lineto(702.75523541,203.54391901)
\lineto(702.75523541,204.69525144)
\lineto(709.00073146,202.05812129)
\lineto(709.00073146,200.97754113)
}
}
{
\newrgbcolor{curcolor}{0 0 0}
\pscustom[linestyle=none,fillstyle=solid,fillcolor=curcolor]
{
\newpath
\moveto(716.6934347,200.97754113)
\lineto(710.44793866,198.30825085)
\lineto(710.44793866,199.45958328)
\lineto(715.3941657,201.51139918)
\lineto(710.44793866,203.54391901)
\lineto(710.44793866,204.69525144)
\lineto(716.6934347,202.05812129)
\lineto(716.6934347,200.97754113)
}
}
{
\newrgbcolor{curcolor}{0 0 0}
\pscustom[linestyle=none,fillstyle=solid,fillcolor=curcolor]
{
\newpath
\moveto(659.0149776,179.26942131)
\lineto(660.81594453,178.96711615)
\curveto(660.58438517,178.3067604)(660.21776012,177.80291896)(659.71606829,177.45559033)
\curveto(659.21865327,177.11254899)(658.5947475,176.94102851)(657.84434909,176.94102835)
\curveto(656.65656597,176.94102851)(655.77752347,177.32909361)(655.20721894,178.10522483)
\curveto(654.75697656,178.7269856)(654.53185592,179.51169183)(654.53185634,180.4593459)
\curveto(654.53185592,181.59137776)(654.82772876,182.47685228)(655.41947575,183.11577212)
\curveto(656.01122013,183.75896794)(656.75947827,184.08056885)(657.6642524,184.08057584)
\curveto(658.68050774,184.08056885)(659.48236603,183.74395989)(660.06982966,183.07074795)
\curveto(660.65728137,182.40181207)(660.93814617,181.37483314)(660.9124249,179.98980809)
\lineto(656.38427947,179.98980809)
\curveto(656.39714123,179.45380367)(656.54293365,179.03572247)(656.82165715,178.73556326)
\curveto(657.10037524,178.43968878)(657.44770423,178.29175235)(657.86364516,178.29175355)
\curveto(658.14665022,178.29175235)(658.3846349,178.36893657)(658.57759991,178.52330644)
\curveto(658.770556,178.67767345)(658.91634841,178.92637816)(659.0149776,179.26942131)
\moveto(659.11788999,181.09611635)
\curveto(659.10502095,181.61924984)(658.96994857,182.01589097)(658.71267243,182.28604093)
\curveto(658.4553871,182.56046852)(658.14236221,182.69768491)(657.77359682,182.69769051)
\curveto(657.37909603,182.69768491)(657.0532071,182.5540365)(656.79592905,182.26674485)
\curveto(656.53864564,181.97944286)(656.41214928,181.58923375)(656.41643959,181.09611635)
\lineto(659.11788999,181.09611635)
}
}
{
\newrgbcolor{curcolor}{0 0 0}
\pscustom[linestyle=none,fillstyle=solid,fillcolor=curcolor]
{
\newpath
\moveto(664.26994168,177.09539695)
\lineto(661.51703509,183.92620724)
\lineto(663.41448239,183.92620724)
\lineto(664.70088734,180.44004982)
\lineto(665.07394478,179.27585334)
\curveto(665.17256543,179.571724)(665.23474161,179.76682856)(665.2604735,179.86116759)
\curveto(665.32050185,180.05412538)(665.38482204,180.24708593)(665.45343424,180.44004982)
\lineto(666.75270324,183.92620724)
\lineto(668.6115584,183.92620724)
\lineto(665.89724395,177.09539695)
\lineto(664.26994168,177.09539695)
}
}
{
\newrgbcolor{curcolor}{0 0 0}
\pscustom[linestyle=none,fillstyle=solid,fillcolor=curcolor]
{
\newpath
\moveto(671.10075197,181.84223122)
\lineto(669.46058565,182.13810436)
\curveto(669.64496952,182.7984532)(669.96228243,183.28728659)(670.41252532,183.604606)
\curveto(670.86276499,183.9219124)(671.5316949,184.08056885)(672.41931704,184.08057584)
\curveto(673.22545973,184.08056885)(673.82578144,183.98408858)(674.22028398,183.79113472)
\curveto(674.61477568,183.60245549)(674.89135247,183.3601828)(675.05001517,183.06431592)
\curveto(675.21295339,182.77272512)(675.29442562,182.23457959)(675.29443211,181.44987771)
\lineto(675.27513604,179.34017359)
\curveto(675.27512957,178.73984963)(675.30300165,178.29604037)(675.35875236,178.00874446)
\curveto(675.41877798,177.72573474)(675.52812229,177.42128587)(675.68678562,177.09539695)
\lineto(673.89868274,177.09539695)
\curveto(673.85150951,177.21546129)(673.79362134,177.39341379)(673.72501807,177.629255)
\curveto(673.69499706,177.73645477)(673.673557,177.80720697)(673.66069782,177.84151182)
\curveto(673.35195609,177.54135022)(673.02177914,177.31622957)(672.67016601,177.16614922)
\curveto(672.31854514,177.01606872)(671.94334407,176.94102851)(671.54456168,176.94102835)
\curveto(670.84132493,176.94102851)(670.28602735,177.13184505)(669.87866726,177.51347855)
\curveto(669.47559304,177.89511122)(669.27405646,178.3775126)(669.27405693,178.96068413)
\curveto(669.27405646,179.34660336)(669.36624873,179.68964434)(669.550634,179.98980809)
\curveto(669.73501778,180.29425406)(669.99229851,180.52580672)(670.32247697,180.68446676)
\curveto(670.65694041,180.84740764)(671.13719777,180.98891204)(671.76325052,181.1089804)
\curveto(672.60798597,181.26763284)(673.19329963,181.41556926)(673.51919328,181.55279011)
\lineto(673.51919328,181.7328868)
\curveto(673.51918856,182.08021115)(673.43342832,182.32677185)(673.26191229,182.47256965)
\curveto(673.09038734,182.6226447)(672.76664242,182.69768491)(672.29067655,182.69769051)
\curveto(671.96907214,182.69768491)(671.71822343,182.63336473)(671.53812965,182.50472977)
\curveto(671.3580304,182.38037201)(671.21223799,182.15953938)(671.10075197,181.84223122)
\moveto(673.51919328,180.37572957)
\curveto(673.2876359,180.29854207)(672.92101086,180.20634981)(672.41931704,180.09915251)
\curveto(671.917616,179.9919492)(671.58958306,179.8868929)(671.43521725,179.7839833)
\curveto(671.19937395,179.61674813)(671.08145362,179.40449153)(671.08145589,179.14721284)
\curveto(671.08145362,178.89421807)(671.17578988,178.67552945)(671.36446498,178.49114632)
\curveto(671.55313496,178.3067604)(671.79326364,178.21456813)(672.08485176,178.21456925)
\curveto(672.4107374,178.21456813)(672.72161829,178.32176844)(673.01749535,178.53617049)
\curveto(673.23617976,178.69911352)(673.37982817,178.89850608)(673.448441,179.13434879)
\curveto(673.4956045,179.2887152)(673.51918856,179.58244403)(673.51919328,180.01553619)
\lineto(673.51919328,180.37572957)
}
}
{
\newrgbcolor{curcolor}{0 0 0}
\pscustom[linestyle=none,fillstyle=solid,fillcolor=curcolor]
{
\newpath
\moveto(677.08253485,177.09539695)
\lineto(677.08253485,186.52474525)
\lineto(678.88993381,186.52474525)
\lineto(678.88993381,177.09539695)
\lineto(677.08253485,177.09539695)
}
}
{
\newrgbcolor{curcolor}{0 0 0}
\pscustom[linestyle=none,fillstyle=solid,fillcolor=curcolor]
{
\newpath
\moveto(685.23191105,177.09539695)
\lineto(685.23191105,178.11808888)
\curveto(684.9832009,177.75360682)(684.65516796,177.46631)(684.24781126,177.25619756)
\curveto(683.84473365,177.0460848)(683.41807644,176.94102851)(682.96783833,176.94102835)
\curveto(682.50901785,176.94102851)(682.09736867,177.04179679)(681.73288958,177.24333351)
\curveto(681.3684066,177.44486994)(681.10469384,177.72787875)(680.94175053,178.09236078)
\curveto(680.77880491,178.45684083)(680.69733268,178.96068226)(680.69733359,179.6038866)
\lineto(680.69733359,183.92620724)
\lineto(682.50473255,183.92620724)
\lineto(682.50473255,180.78737916)
\curveto(682.50472983,179.82686073)(682.53688993,179.23725905)(682.60121292,179.01857235)
\curveto(682.6698183,178.80416981)(682.79202665,178.63264933)(682.96783833,178.50401037)
\curveto(683.14364365,178.37965661)(683.36662029,178.31748043)(683.63676891,178.31748165)
\curveto(683.94550194,178.31748043)(684.22207873,178.40109667)(684.4665001,178.56833062)
\curveto(684.71091212,178.73984963)(684.8781446,178.94996223)(684.96819803,179.19866904)
\curveto(685.05824111,179.45165966)(685.10326524,180.06698941)(685.10327055,181.04466015)
\lineto(685.10327055,183.92620724)
\lineto(686.91066951,183.92620724)
\lineto(686.91066951,177.09539695)
\lineto(685.23191105,177.09539695)
}
}
{
\newrgbcolor{curcolor}{0 0 0}
\pscustom[linestyle=none,fillstyle=solid,fillcolor=curcolor]
{
\newpath
\moveto(690.13954553,181.84223122)
\lineto(688.49937922,182.13810436)
\curveto(688.68376309,182.7984532)(689.00107599,183.28728659)(689.45131888,183.604606)
\curveto(689.90155856,183.9219124)(690.57048846,184.08056885)(691.45811061,184.08057584)
\curveto(692.26425329,184.08056885)(692.864575,183.98408858)(693.25907754,183.79113472)
\curveto(693.65356925,183.60245549)(693.93014604,183.3601828)(694.08880873,183.06431592)
\curveto(694.25174695,182.77272512)(694.33321919,182.23457959)(694.33322568,181.44987771)
\lineto(694.3139296,179.34017359)
\curveto(694.31392313,178.73984963)(694.34179521,178.29604037)(694.39754592,178.00874446)
\curveto(694.45757154,177.72573474)(694.56691585,177.42128587)(694.72557919,177.09539695)
\lineto(692.9374763,177.09539695)
\curveto(692.89030307,177.21546129)(692.83241491,177.39341379)(692.76381163,177.629255)
\curveto(692.73379063,177.73645477)(692.71235057,177.80720697)(692.69949139,177.84151182)
\curveto(692.39074965,177.54135022)(692.06057271,177.31622957)(691.70895957,177.16614922)
\curveto(691.3573387,177.01606872)(690.98213764,176.94102851)(690.58335524,176.94102835)
\curveto(689.88011849,176.94102851)(689.32482091,177.13184505)(688.91746083,177.51347855)
\curveto(688.5143866,177.89511122)(688.31285003,178.3775126)(688.3128505,178.96068413)
\curveto(688.31285003,179.34660336)(688.40504229,179.68964434)(688.58942756,179.98980809)
\curveto(688.77381134,180.29425406)(689.03109208,180.52580672)(689.36127053,180.68446676)
\curveto(689.69573397,180.84740764)(690.17599134,180.98891204)(690.80204408,181.1089804)
\curveto(691.64677953,181.26763284)(692.2320932,181.41556926)(692.55798684,181.55279011)
\lineto(692.55798684,181.7328868)
\curveto(692.55798213,182.08021115)(692.47222188,182.32677185)(692.30070585,182.47256965)
\curveto(692.1291809,182.6226447)(691.80543598,182.69768491)(691.32947011,182.69769051)
\curveto(691.00786571,182.69768491)(690.75701699,182.63336473)(690.57692321,182.50472977)
\curveto(690.39682397,182.38037201)(690.25103155,182.15953938)(690.13954553,181.84223122)
\moveto(692.55798684,180.37572957)
\curveto(692.32642947,180.29854207)(691.95980442,180.20634981)(691.45811061,180.09915251)
\curveto(690.95640956,179.9919492)(690.62837663,179.8868929)(690.47401082,179.7839833)
\curveto(690.23816752,179.61674813)(690.12024718,179.40449153)(690.12024946,179.14721284)
\curveto(690.12024718,178.89421807)(690.21458345,178.67552945)(690.40325855,178.49114632)
\curveto(690.59192852,178.3067604)(690.83205721,178.21456813)(691.12364532,178.21456925)
\curveto(691.44953097,178.21456813)(691.76041185,178.32176844)(692.05628891,178.53617049)
\curveto(692.27497332,178.69911352)(692.41862173,178.89850608)(692.48723457,179.13434879)
\curveto(692.53439806,179.2887152)(692.55798213,179.58244403)(692.55798684,180.01553619)
\lineto(692.55798684,180.37572957)
}
}
{
\newrgbcolor{curcolor}{0 0 0}
\pscustom[linestyle=none,fillstyle=solid,fillcolor=curcolor]
{
\newpath
\moveto(699.25372448,183.92620724)
\lineto(699.25372448,182.4854337)
\lineto(698.01877572,182.4854337)
\lineto(698.01877572,179.7325271)
\curveto(698.01877288,179.17508287)(698.02949291,178.84919394)(698.05093585,178.75485933)
\curveto(698.07666104,178.66480942)(698.1302612,178.5897692)(698.21173647,178.52973847)
\curveto(698.29749367,178.46970486)(698.40040597,178.43968878)(698.52047365,178.43969012)
\curveto(698.68770279,178.43968878)(698.92997548,178.49757694)(699.24729245,178.61335479)
\lineto(699.40166105,177.21117339)
\curveto(698.98143162,177.03107676)(698.50546227,176.94102851)(697.97375155,176.94102835)
\curveto(697.64785982,176.94102851)(697.35413099,176.99462866)(697.09256416,177.10182897)
\curveto(696.83099349,177.21331728)(696.63803294,177.35482168)(696.51368193,177.5263426)
\curveto(696.39361625,177.70215067)(696.31000001,177.93799135)(696.26283296,178.23386533)
\curveto(696.22423977,178.44397679)(696.20494371,178.86849)(696.20494474,179.50740623)
\lineto(696.20494474,182.4854337)
\lineto(695.37521355,182.4854337)
\lineto(695.37521355,183.92620724)
\lineto(696.20494474,183.92620724)
\lineto(696.20494474,185.28336447)
\lineto(698.01877572,186.33821653)
\lineto(698.01877572,183.92620724)
\lineto(699.25372448,183.92620724)
}
}
{
\newrgbcolor{curcolor}{0 0 0}
\pscustom[linestyle=none,fillstyle=solid,fillcolor=curcolor]
{
\newpath
\moveto(700.52083403,184.85241881)
\lineto(700.52083403,186.52474525)
\lineto(702.32823299,186.52474525)
\lineto(702.32823299,184.85241881)
\lineto(700.52083403,184.85241881)
\moveto(700.52083403,177.09539695)
\lineto(700.52083403,183.92620724)
\lineto(702.32823299,183.92620724)
\lineto(702.32823299,177.09539695)
\lineto(700.52083403,177.09539695)
}
}
{
\newrgbcolor{curcolor}{0 0 0}
\pscustom[linestyle=none,fillstyle=solid,fillcolor=curcolor]
{
\newpath
\moveto(703.7561414,180.60728247)
\curveto(703.75614087,181.20760066)(703.90407729,181.78862632)(704.19995111,182.35036118)
\curveto(704.49582298,182.91208552)(704.91390417,183.34088674)(705.45419594,183.63676613)
\curveto(705.99877126,183.93263243)(706.60552499,184.08056885)(707.27445895,184.08057584)
\curveto(708.30786584,184.08056885)(709.15474826,183.74395989)(709.81510873,183.07074795)
\curveto(710.47545602,182.40181207)(710.80563296,181.55492965)(710.80564054,180.53009817)
\curveto(710.80563296,179.49668379)(710.47116801,178.63908134)(709.80224468,177.95728826)
\curveto(709.13759621,177.27978147)(708.29928982,176.94102851)(707.28732299,176.94102835)
\curveto(706.66126915,176.94102851)(706.06309145,177.08253291)(705.49278809,177.36554199)
\curveto(704.92676821,177.64855052)(704.49582298,178.0623437)(704.19995111,178.60692276)
\curveto(703.90407729,179.15578682)(703.75614087,179.82257272)(703.7561414,180.60728247)
\moveto(705.60856453,180.51080209)
\curveto(705.60856215,179.83329275)(705.76936261,179.31444327)(706.09096639,178.9542521)
\curveto(706.41256444,178.59405722)(706.80920557,178.4139607)(707.28089097,178.41396202)
\curveto(707.75256826,178.4139607)(708.14706539,178.59405722)(708.46438353,178.9542521)
\curveto(708.78597921,179.31444327)(708.94677967,179.83758076)(708.94678538,180.52366614)
\curveto(708.94677967,181.19259262)(708.78597921,181.70715409)(708.46438353,182.06735209)
\curveto(708.14706539,182.42754014)(707.75256826,182.60763665)(707.28089097,182.60764217)
\curveto(706.80920557,182.60763665)(706.41256444,182.42754014)(706.09096639,182.06735209)
\curveto(705.76936261,181.70715409)(705.60856215,181.18830461)(705.60856453,180.51080209)
}
}
{
\newrgbcolor{curcolor}{0 0 0}
\pscustom[linestyle=none,fillstyle=solid,fillcolor=curcolor]
{
\newpath
\moveto(718.44045352,177.09539695)
\lineto(716.63305456,177.09539695)
\lineto(716.63305456,180.58155437)
\curveto(716.63304921,181.31908898)(716.5944571,181.79505834)(716.51727812,182.00946386)
\curveto(716.44008866,182.22814757)(716.3135923,182.39752406)(716.13778866,182.51759382)
\curveto(715.96626331,182.63765274)(715.75829472,182.69768491)(715.51388226,182.69769051)
\curveto(715.20085313,182.69768491)(714.91998833,182.61192467)(714.67128701,182.44040952)
\curveto(714.42257891,182.26888369)(714.25105842,182.04161904)(714.15672503,181.7586149)
\curveto(714.0666739,181.47560143)(714.02164977,180.95246394)(714.02165251,180.18920086)
\lineto(714.02165251,177.09539695)
\lineto(712.21425355,177.09539695)
\lineto(712.21425355,183.92620724)
\lineto(713.89301202,183.92620724)
\lineto(713.89301202,182.92281138)
\curveto(714.4890431,183.69464775)(715.23944524,184.08056885)(716.14422068,184.08057584)
\curveto(716.54300096,184.08056885)(716.90748199,184.00767264)(717.23766489,183.86188699)
\curveto(717.56783588,183.72037583)(717.81654059,183.53813531)(717.98377976,183.31516489)
\curveto(718.15529355,183.09218203)(718.27321389,182.83918931)(718.33754113,182.55618597)
\curveto(718.40614227,182.2731717)(718.44044636,181.86795455)(718.44045352,181.34053329)
\lineto(718.44045352,177.09539695)
}
}
{
\newrgbcolor{curcolor}{0 0 0}
\pscustom[linestyle=none,fillstyle=solid,fillcolor=curcolor]
{
\newpath
\moveto(719.64324556,179.04430045)
\lineto(721.45707654,179.32087751)
\curveto(721.53425864,178.96925829)(721.69077108,178.70125752)(721.92661435,178.51687442)
\curveto(722.16245243,178.33677648)(722.49262937,178.24672823)(722.91714616,178.24672938)
\curveto(723.38453591,178.24672823)(723.73615291,178.33248847)(723.97199822,178.50401037)
\curveto(724.13065004,178.6240733)(724.20997826,178.78487376)(724.20998314,178.98641223)
\curveto(724.20997826,179.12362673)(724.16709814,179.23725905)(724.08134264,179.32730954)
\curveto(723.99128964,179.41306755)(723.78975307,179.49239578)(723.47673232,179.56529445)
\curveto(722.01880402,179.8868929)(721.09473739,180.18062174)(720.70452964,180.44648185)
\curveto(720.16423873,180.81524755)(719.89409396,181.32766501)(719.89409452,181.98373576)
\curveto(719.89409396,182.57547656)(720.12779063,183.07288598)(720.59518522,183.47596551)
\curveto(721.06257729,183.87903228)(721.78725136,184.08056885)(722.76920959,184.08057584)
\curveto(723.70399282,184.08056885)(724.3986508,183.92834442)(724.85318561,183.62390208)
\curveto(725.30770939,183.31944668)(725.62073428,182.8692054)(725.79226123,182.27317688)
\lineto(724.08777467,181.95800767)
\curveto(724.01487371,182.22385956)(723.87551331,182.42754014)(723.66969306,182.56905002)
\curveto(723.46815215,182.71054895)(723.17871132,182.78130115)(722.80136972,182.78130683)
\curveto(722.32539689,182.78130115)(721.98449992,182.71483696)(721.77867778,182.58191407)
\curveto(721.64145894,182.48757231)(721.57285075,182.36536396)(721.57285299,182.21528866)
\curveto(721.57285075,182.08664317)(721.63288292,181.97729886)(721.75294968,181.88725539)
\curveto(721.91589173,181.76718626)(722.47762133,181.59780978)(723.43814017,181.37912544)
\curveto(724.40293881,181.16043253)(725.07615673,180.89243177)(725.45779594,180.57512234)
\curveto(725.83513489,180.25351795)(726.02380743,179.80542067)(726.02381412,179.23082917)
\curveto(726.02380743,178.60477725)(725.76223869,178.06663171)(725.2391071,177.61639095)
\curveto(724.71596371,177.16614915)(723.9419775,176.94102851)(722.91714616,176.94102835)
\curveto(721.98664393,176.94102851)(721.24910583,177.12970104)(720.70452964,177.50704653)
\curveto(720.16423873,177.88439119)(719.81047773,178.39680865)(719.64324556,179.04430045)
}
}
{
\newrgbcolor{curcolor}{0 1 0.25098041}
\pscustom[linewidth=2.63455725,linecolor=curcolor]
{
\newpath
\moveto(527.33596,45.89200856)
\lineto(616.91091,45.89200856)
\lineto(616.91091,98.58315856)
\lineto(557.63337,98.58315856)
\lineto(557.63337,109.12138856)
\lineto(527.33596,109.12138856)
\lineto(527.33596,45.89200856)
\closepath
}
}
{
\newrgbcolor{curcolor}{0 0 0}
\pscustom[linestyle=none,fillstyle=solid,fillcolor=curcolor]
{
\newpath
\moveto(545.18093423,81.57296018)
\lineto(545.18093423,82.65354034)
\lineto(551.42643028,85.2906705)
\lineto(551.42643028,84.13933807)
\lineto(546.47377121,82.10681824)
\lineto(551.42643028,80.05500234)
\lineto(551.42643028,78.90366991)
\lineto(545.18093423,81.57296018)
}
}
{
\newrgbcolor{curcolor}{0 0 0}
\pscustom[linestyle=none,fillstyle=solid,fillcolor=curcolor]
{
\newpath
\moveto(552.87363605,81.57296018)
\lineto(552.87363605,82.65354034)
\lineto(559.11913209,85.2906705)
\lineto(559.11913209,84.13933807)
\lineto(554.16647303,82.10681824)
\lineto(559.11913209,80.05500234)
\lineto(559.11913209,78.90366991)
\lineto(552.87363605,81.57296018)
}
}
{
\newrgbcolor{curcolor}{0 0 0}
\pscustom[linestyle=none,fillstyle=solid,fillcolor=curcolor]
{
\newpath
\moveto(565.1716676,78.29262756)
\curveto(564.74286105,77.92814568)(564.32906787,77.67086494)(563.93028682,77.52078458)
\curveto(563.53578561,77.37070409)(563.1112724,77.29566387)(562.65674592,77.29566372)
\curveto(561.90634097,77.29566387)(561.32960332,77.47790439)(560.92653125,77.84238582)
\curveto(560.52345702,78.21115448)(560.32192045,78.68069182)(560.32192093,79.25099925)
\curveto(560.32192045,79.5854624)(560.39696066,79.88991127)(560.54704179,80.16434676)
\curveto(560.70140953,80.44306484)(560.9008021,80.66604148)(561.1452201,80.83327734)
\curveto(561.39392351,81.00050643)(561.6726443,81.12700279)(561.98138332,81.2127668)
\curveto(562.20864583,81.27279521)(562.5516868,81.33068337)(563.01050728,81.38643147)
\curveto(563.94529078,81.49791585)(564.63351674,81.63084423)(565.07518723,81.785217)
\curveto(565.07947001,81.94386912)(565.08161401,82.04463741)(565.08161925,82.08752217)
\curveto(565.08161401,82.55919887)(564.9722697,82.89151982)(564.75358599,83.084486)
\curveto(564.45770824,83.34604912)(564.01818698,83.47683349)(563.43502091,83.47683951)
\curveto(562.89043977,83.47683349)(562.48736662,83.38035321)(562.22580026,83.1873984)
\curveto(561.96851714,82.99872013)(561.7777006,82.66211117)(561.65335005,82.17757051)
\lineto(560.52131369,82.33193911)
\curveto(560.62422531,82.81647961)(560.79360179,83.20668872)(561.02944365,83.50256761)
\curveto(561.26528314,83.80272242)(561.60618011,84.03213107)(562.05213559,84.19079426)
\curveto(562.49808665,84.35373199)(563.01479212,84.43520422)(563.60225356,84.4352112)
\curveto(564.18541946,84.43520422)(564.65924481,84.36659602)(565.02373103,84.22938641)
\curveto(565.38820689,84.09216324)(565.65620765,83.91849875)(565.82773412,83.70839241)
\curveto(565.99924863,83.50256156)(566.11931297,83.24099282)(566.18792751,82.92368538)
\curveto(566.22651328,82.72643135)(566.24580933,82.37052633)(566.24581573,81.85596927)
\lineto(566.24581573,80.31228333)
\curveto(566.24580933,79.2359894)(566.2693934,78.55419546)(566.316568,78.26689946)
\curveto(566.36801768,77.98388983)(566.46664196,77.71160106)(566.61244114,77.45003231)
\lineto(565.40322049,77.45003231)
\curveto(565.28315059,77.690161)(565.20596637,77.9710258)(565.1716676,78.29262756)
\moveto(565.07518723,80.87830151)
\curveto(564.6549568,80.70677759)(564.024619,80.56098518)(563.18417195,80.44092383)
\curveto(562.70819925,80.37231264)(562.37159029,80.29512842)(562.17434406,80.20937094)
\curveto(561.97709317,80.12360793)(561.82486873,79.99711157)(561.7176703,79.82988147)
\curveto(561.61046812,79.66693463)(561.55686797,79.48469411)(561.55686968,79.28315937)
\curveto(561.55686797,78.97442066)(561.6726443,78.71713992)(561.90419902,78.5113164)
\curveto(562.14003763,78.30549075)(562.48307861,78.20257846)(562.93332298,78.20257921)
\curveto(563.37927316,78.20257846)(563.77591429,78.29905873)(564.12324756,78.49202032)
\curveto(564.47057227,78.68926784)(564.725709,78.95726861)(564.88865851,79.29602342)
\curveto(565.01300582,79.55759032)(565.075182,79.94351142)(565.07518723,80.45378788)
\lineto(565.07518723,80.87830151)
}
}
{
\newrgbcolor{curcolor}{0 0 0}
\pscustom[linestyle=none,fillstyle=solid,fillcolor=curcolor]
{
\newpath
\moveto(568.04678157,74.83219823)
\lineto(568.04678157,84.28084261)
\lineto(569.10163363,84.28084261)
\lineto(569.10163363,83.39322319)
\curveto(569.35033642,83.74054624)(569.63120122,83.99997098)(569.94422888,84.17149819)
\curveto(570.257251,84.34729997)(570.63674008,84.43520422)(571.08269726,84.4352112)
\curveto(571.66586302,84.43520422)(572.18042448,84.28512379)(572.6263832,83.98496947)
\curveto(573.07233102,83.68480208)(573.40893998,83.26028887)(573.63621109,82.71142857)
\curveto(573.86346928,82.16684575)(573.9771016,81.56866805)(573.9771084,80.91689366)
\curveto(573.9771016,80.2179442)(573.85060524,79.5876064)(573.59761894,79.02587838)
\curveto(573.34890781,78.46843521)(572.98442677,78.03963399)(572.50417473,77.73947343)
\curveto(572.02820005,77.44360029)(571.52650262,77.29566387)(570.99908094,77.29566372)
\curveto(570.61315602,77.29566387)(570.26582703,77.3771361)(569.95709292,77.54008066)
\curveto(569.65264128,77.70302503)(569.40179256,77.90884962)(569.20454603,78.15755504)
\lineto(569.20454603,74.83219823)
\lineto(568.04678157,74.83219823)
\moveto(569.09520161,80.82684531)
\curveto(569.09519969,79.94779943)(569.2731522,79.29816558)(569.62905966,78.87794181)
\curveto(569.98496223,78.45771518)(570.41590745,78.24760259)(570.92189664,78.24760338)
\curveto(571.43645436,78.24760259)(571.87597561,78.4641472)(572.24046172,78.89723788)
\curveto(572.6092257,79.33461368)(572.79361023,80.00997561)(572.79361584,80.92332568)
\curveto(572.79361023,81.79378869)(572.61351372,82.44556655)(572.25332576,82.87866121)
\curveto(571.89741568,83.31174502)(571.47075846,83.52828963)(570.97335284,83.52829571)
\curveto(570.48022764,83.52828963)(570.04285039,83.29673697)(569.66121979,82.83363704)
\curveto(569.28387223,82.37481435)(569.09519969,81.70588444)(569.09520161,80.82684531)
}
}
{
\newrgbcolor{curcolor}{0 0 0}
\pscustom[linestyle=none,fillstyle=solid,fillcolor=curcolor]
{
\newpath
\moveto(575.37928966,74.83219823)
\lineto(575.37928966,84.28084261)
\lineto(576.43414172,84.28084261)
\lineto(576.43414172,83.39322319)
\curveto(576.6828445,83.74054624)(576.9637093,83.99997098)(577.27673696,84.17149819)
\curveto(577.58975909,84.34729997)(577.96924817,84.43520422)(578.41520535,84.4352112)
\curveto(578.9983711,84.43520422)(579.51293257,84.28512379)(579.95889129,83.98496947)
\curveto(580.40483911,83.68480208)(580.74144807,83.26028887)(580.96871918,82.71142857)
\curveto(581.19597737,82.16684575)(581.30960969,81.56866805)(581.30961649,80.91689366)
\curveto(581.30960969,80.2179442)(581.18311333,79.5876064)(580.93012703,79.02587838)
\curveto(580.6814159,78.46843521)(580.31693486,78.03963399)(579.83668282,77.73947343)
\curveto(579.36070814,77.44360029)(578.85901071,77.29566387)(578.33158902,77.29566372)
\curveto(577.9456641,77.29566387)(577.59833511,77.3771361)(577.28960101,77.54008066)
\curveto(576.98514937,77.70302503)(576.73430065,77.90884962)(576.53705411,78.15755504)
\lineto(576.53705411,74.83219823)
\lineto(575.37928966,74.83219823)
\moveto(576.42770969,80.82684531)
\curveto(576.42770778,79.94779943)(576.60566028,79.29816558)(576.96156775,78.87794181)
\curveto(577.31747031,78.45771518)(577.74841554,78.24760259)(578.25440473,78.24760338)
\curveto(578.76896245,78.24760259)(579.2084837,78.4641472)(579.5729698,78.89723788)
\curveto(579.94173379,79.33461368)(580.12611832,80.00997561)(580.12612393,80.92332568)
\curveto(580.12611832,81.79378869)(579.9460218,82.44556655)(579.58583385,82.87866121)
\curveto(579.22992376,83.31174502)(578.80326655,83.52828963)(578.30586092,83.52829571)
\curveto(577.81273572,83.52828963)(577.37535848,83.29673697)(576.99372787,82.83363704)
\curveto(576.61638031,82.37481435)(576.42770778,81.70588444)(576.42770969,80.82684531)
}
}
{
\newrgbcolor{curcolor}{0 0 0}
\pscustom[linestyle=none,fillstyle=solid,fillcolor=curcolor]
{
\newpath
\moveto(588.80935722,81.57296018)
\lineto(582.56386118,78.90366991)
\lineto(582.56386118,80.05500234)
\lineto(587.51008822,82.10681824)
\lineto(582.56386118,84.13933807)
\lineto(582.56386118,85.2906705)
\lineto(588.80935722,82.65354034)
\lineto(588.80935722,81.57296018)
}
}
{
\newrgbcolor{curcolor}{0 0 0}
\pscustom[linestyle=none,fillstyle=solid,fillcolor=curcolor]
{
\newpath
\moveto(596.50206047,81.57296018)
\lineto(590.25656442,78.90366991)
\lineto(590.25656442,80.05500234)
\lineto(595.20279146,82.10681824)
\lineto(590.25656442,84.13933807)
\lineto(590.25656442,85.2906705)
\lineto(596.50206047,82.65354034)
\lineto(596.50206047,81.57296018)
}
}
{
\newrgbcolor{curcolor}{0 0 0}
\pscustom[linestyle=none,fillstyle=solid,fillcolor=curcolor]
{
\newpath
\moveto(539.50664128,62.50190949)
\lineto(537.72497042,62.18030825)
\curveto(537.66493313,62.53620878)(537.52771674,62.80420954)(537.31332084,62.98431135)
\curveto(537.10320353,63.16440257)(536.82877075,63.25445082)(536.49002167,63.25445639)
\curveto(536.0397765,63.25445082)(535.67958347,63.09793838)(535.40944151,62.78491858)
\curveto(535.14358195,62.47617661)(535.01065357,61.95732713)(535.01065597,61.22836859)
\curveto(535.01065357,60.41793074)(535.14572595,59.84548111)(535.41587353,59.51101798)
\curveto(535.6903035,59.1765512)(536.05692855,59.00931873)(536.51574977,59.00932004)
\curveto(536.85878683,59.00931873)(537.13965163,59.105799)(537.35834501,59.29876116)
\curveto(537.57702888,59.49600811)(537.73139732,59.83261707)(537.82145079,60.30858905)
\lineto(539.59668963,60.00628388)
\curveto(539.41229811,59.19155925)(539.0585371,58.57622949)(538.53540554,58.16029278)
\curveto(538.01226212,57.74435512)(537.31117212,57.53638653)(536.43213344,57.53638637)
\curveto(535.43302277,57.53638653)(534.6354525,57.85155543)(534.03942023,58.48189401)
\curveto(533.44767311,59.11223102)(533.15180027,59.98484151)(533.15180082,61.09972809)
\curveto(533.15180027,62.2274719)(533.44981712,63.1043704)(534.04585226,63.73042622)
\curveto(534.64188452,64.36075798)(535.44803081,64.67592688)(536.46429357,64.67593386)
\curveto(537.29616408,64.67592688)(537.95651796,64.49583036)(538.44535719,64.13564378)
\curveto(538.93847276,63.77973232)(539.29223377,63.23515477)(539.50664128,62.50190949)
}
}
{
\newrgbcolor{curcolor}{0 0 0}
\pscustom[linestyle=none,fillstyle=solid,fillcolor=curcolor]
{
\newpath
\moveto(542.23381964,62.43758924)
\lineto(540.59365332,62.73346238)
\curveto(540.77803719,63.39381122)(541.0953501,63.88264461)(541.54559299,64.19996403)
\curveto(541.99583266,64.51727042)(542.66476257,64.67592688)(543.55238471,64.67593386)
\curveto(544.3585274,64.67592688)(544.95884911,64.5794466)(545.35335165,64.38649275)
\curveto(545.74784336,64.19781351)(546.02442014,63.95554082)(546.18308284,63.65967395)
\curveto(546.34602106,63.36808315)(546.42749329,62.82993761)(546.42749978,62.04523573)
\lineto(546.40820371,59.93553161)
\curveto(546.40819724,59.33520766)(546.43606932,58.89139839)(546.49182003,58.60410248)
\curveto(546.55184565,58.32109277)(546.66118996,58.0166439)(546.81985329,57.69075497)
\lineto(545.03175041,57.69075497)
\curveto(544.98457718,57.81081931)(544.92668902,57.98877182)(544.85808574,58.22461302)
\curveto(544.82806474,58.3318128)(544.80662467,58.402565)(544.79376549,58.43686984)
\curveto(544.48502376,58.13670824)(544.15484682,57.9115876)(543.80323368,57.76150724)
\curveto(543.45161281,57.61142674)(543.07641174,57.53638653)(542.67762935,57.53638637)
\curveto(541.9743926,57.53638653)(541.41909502,57.72720307)(541.01173493,58.10883658)
\curveto(540.60866071,58.49046925)(540.40712414,58.97287062)(540.40712461,59.55604215)
\curveto(540.40712414,59.94196138)(540.4993164,60.28500236)(540.68370167,60.58516611)
\curveto(540.86808545,60.88961208)(541.12536618,61.12116474)(541.45554464,61.27982479)
\curveto(541.79000808,61.44276566)(542.27026545,61.58427006)(542.89631819,61.70433842)
\curveto(543.74105364,61.86299086)(544.32636731,62.01092728)(544.65226095,62.14814813)
\lineto(544.65226095,62.32824482)
\curveto(544.65225623,62.67556917)(544.56649599,62.92212988)(544.39497996,63.06792767)
\curveto(544.22345501,63.21800272)(543.89971009,63.29304293)(543.42374422,63.29304854)
\curveto(543.10213982,63.29304293)(542.8512911,63.22872275)(542.67119732,63.10008779)
\curveto(542.49109808,62.97573003)(542.34530566,62.7548974)(542.23381964,62.43758924)
\moveto(544.65226095,60.9710876)
\curveto(544.42070357,60.8939001)(544.05407853,60.80170783)(543.55238471,60.69451053)
\curveto(543.05068367,60.58730722)(542.72265073,60.48225092)(542.56828493,60.37934132)
\curveto(542.33244162,60.21210615)(542.21452129,59.99984955)(542.21452356,59.74257087)
\curveto(542.21452129,59.4895761)(542.30885756,59.27088747)(542.49753265,59.08650434)
\curveto(542.68620263,58.90211842)(542.92633132,58.80992616)(543.21791943,58.80992728)
\curveto(543.54380508,58.80992616)(543.85468596,58.91712646)(544.15056302,59.13152852)
\curveto(544.36924743,59.29447154)(544.51289584,59.49386411)(544.58150868,59.72970682)
\curveto(544.62867217,59.88407322)(544.65225623,60.17780206)(544.65226095,60.61089421)
\lineto(544.65226095,60.9710876)
}
}
{
\newrgbcolor{curcolor}{0 0 0}
\pscustom[linestyle=none,fillstyle=solid,fillcolor=curcolor]
{
\newpath
\moveto(548.21560253,57.69075497)
\lineto(548.21560253,67.12010327)
\lineto(550.02300148,67.12010327)
\lineto(550.02300148,57.69075497)
\lineto(548.21560253,57.69075497)
}
}
{
\newrgbcolor{curcolor}{0 0 0}
\pscustom[linestyle=none,fillstyle=solid,fillcolor=curcolor]
{
\newpath
\moveto(551.86899341,65.44777683)
\lineto(551.86899341,67.12010327)
\lineto(553.67639237,67.12010327)
\lineto(553.67639237,65.44777683)
\lineto(551.86899341,65.44777683)
\moveto(551.86899341,57.69075497)
\lineto(551.86899341,64.52156527)
\lineto(553.67639237,64.52156527)
\lineto(553.67639237,57.69075497)
\lineto(551.86899341,57.69075497)
}
}
{
\newrgbcolor{curcolor}{0 0 0}
\pscustom[linestyle=none,fillstyle=solid,fillcolor=curcolor]
{
\newpath
\moveto(554.73124334,64.52156527)
\lineto(555.7346392,64.52156527)
\lineto(555.7346392,65.03612725)
\curveto(555.73463805,65.61071354)(555.79467022,66.03951476)(555.9147359,66.3225322)
\curveto(556.03908691,66.60553237)(556.26420756,66.83494103)(556.5900985,67.01075885)
\curveto(556.92027342,67.19084604)(557.33621061,67.2808943)(557.8379113,67.28090389)
\curveto(558.35246951,67.2808943)(558.85631094,67.20371008)(559.34943712,67.049351)
\lineto(559.10502018,65.78867414)
\curveto(558.81771883,65.85727424)(558.54114204,65.89157834)(558.27528898,65.89158654)
\curveto(558.01371654,65.89157834)(557.825044,65.82940216)(557.70927081,65.70505782)
\curveto(557.59777936,65.58498547)(557.5420352,65.3512888)(557.54203816,65.00396712)
\lineto(557.54203816,64.52156527)
\lineto(558.89276336,64.52156527)
\lineto(558.89276336,63.10008779)
\lineto(557.54203816,63.10008779)
\lineto(557.54203816,57.69075497)
\lineto(555.7346392,57.69075497)
\lineto(555.7346392,63.10008779)
\lineto(554.73124334,63.10008779)
\lineto(554.73124334,64.52156527)
}
}
{
\newrgbcolor{curcolor}{0 0 0}
\pscustom[linestyle=none,fillstyle=solid,fillcolor=curcolor]
{
\newpath
\moveto(559.921888,65.44777683)
\lineto(559.921888,67.12010327)
\lineto(561.72928696,67.12010327)
\lineto(561.72928696,65.44777683)
\lineto(559.921888,65.44777683)
\moveto(559.921888,57.69075497)
\lineto(559.921888,64.52156527)
\lineto(561.72928696,64.52156527)
\lineto(561.72928696,57.69075497)
\lineto(559.921888,57.69075497)
}
}
{
\newrgbcolor{curcolor}{0 0 0}
\pscustom[linestyle=none,fillstyle=solid,fillcolor=curcolor]
{
\newpath
\moveto(569.53133382,62.50190949)
\lineto(567.74966296,62.18030825)
\curveto(567.68962567,62.53620878)(567.55240927,62.80420954)(567.33801337,62.98431135)
\curveto(567.12789607,63.16440257)(566.85346328,63.25445082)(566.5147142,63.25445639)
\curveto(566.06446903,63.25445082)(565.70427601,63.09793838)(565.43413404,62.78491858)
\curveto(565.16827448,62.47617661)(565.0353461,61.95732713)(565.03534851,61.22836859)
\curveto(565.0353461,60.41793074)(565.17041849,59.84548111)(565.44056607,59.51101798)
\curveto(565.71499604,59.1765512)(566.08162108,59.00931873)(566.5404423,59.00932004)
\curveto(566.88347937,59.00931873)(567.16434417,59.105799)(567.38303755,59.29876116)
\curveto(567.60172142,59.49600811)(567.75608986,59.83261707)(567.84614333,60.30858905)
\lineto(569.62138216,60.00628388)
\curveto(569.43699065,59.19155925)(569.08322964,58.57622949)(568.56009808,58.16029278)
\curveto(568.03695466,57.74435512)(567.33586466,57.53638653)(566.45682598,57.53638637)
\curveto(565.45771531,57.53638653)(564.66014503,57.85155543)(564.06411277,58.48189401)
\curveto(563.47236565,59.11223102)(563.1764928,59.98484151)(563.17649335,61.09972809)
\curveto(563.1764928,62.2274719)(563.47450965,63.1043704)(564.07054479,63.73042622)
\curveto(564.66657705,64.36075798)(565.47272335,64.67592688)(566.4889861,64.67593386)
\curveto(567.32085662,64.67592688)(567.9812105,64.49583036)(568.47004973,64.13564378)
\curveto(568.9631653,63.77973232)(569.3169263,63.23515477)(569.53133382,62.50190949)
}
}
{
\newrgbcolor{curcolor}{0 0 0}
\pscustom[linestyle=none,fillstyle=solid,fillcolor=curcolor]
{
\newpath
\moveto(572.25851217,62.43758924)
\lineto(570.61834586,62.73346238)
\curveto(570.80272973,63.39381122)(571.12004263,63.88264461)(571.57028552,64.19996403)
\curveto(572.0205252,64.51727042)(572.68945511,64.67592688)(573.57707725,64.67593386)
\curveto(574.38321993,64.67592688)(574.98354164,64.5794466)(575.37804418,64.38649275)
\curveto(575.77253589,64.19781351)(576.04911268,63.95554082)(576.20777538,63.65967395)
\curveto(576.3707136,63.36808315)(576.45218583,62.82993761)(576.45219232,62.04523573)
\lineto(576.43289624,59.93553161)
\curveto(576.43288977,59.33520766)(576.46076185,58.89139839)(576.51651257,58.60410248)
\curveto(576.57653818,58.32109277)(576.68588249,58.0166439)(576.84454583,57.69075497)
\lineto(575.05644295,57.69075497)
\curveto(575.00926972,57.81081931)(574.95138155,57.98877182)(574.88277828,58.22461302)
\curveto(574.85275727,58.3318128)(574.83131721,58.402565)(574.81845803,58.43686984)
\curveto(574.50971629,58.13670824)(574.17953935,57.9115876)(573.82792622,57.76150724)
\curveto(573.47630535,57.61142674)(573.10110428,57.53638653)(572.70232188,57.53638637)
\curveto(571.99908514,57.53638653)(571.44378756,57.72720307)(571.03642747,58.10883658)
\curveto(570.63335325,58.49046925)(570.43181667,58.97287062)(570.43181714,59.55604215)
\curveto(570.43181667,59.94196138)(570.52400893,60.28500236)(570.70839421,60.58516611)
\curveto(570.89277799,60.88961208)(571.15005872,61.12116474)(571.48023718,61.27982479)
\curveto(571.81470061,61.44276566)(572.29495798,61.58427006)(572.92101072,61.70433842)
\curveto(573.76574617,61.86299086)(574.35105984,62.01092728)(574.67695348,62.14814813)
\lineto(574.67695348,62.32824482)
\curveto(574.67694877,62.67556917)(574.59118852,62.92212988)(574.41967249,63.06792767)
\curveto(574.24814755,63.21800272)(573.92440262,63.29304293)(573.44843675,63.29304854)
\curveto(573.12683235,63.29304293)(572.87598364,63.22872275)(572.69588986,63.10008779)
\curveto(572.51579061,62.97573003)(572.3699982,62.7548974)(572.25851217,62.43758924)
\moveto(574.67695348,60.9710876)
\curveto(574.44539611,60.8939001)(574.07877106,60.80170783)(573.57707725,60.69451053)
\curveto(573.07537621,60.58730722)(572.74734327,60.48225092)(572.59297746,60.37934132)
\curveto(572.35713416,60.21210615)(572.23921382,59.99984955)(572.2392161,59.74257087)
\curveto(572.23921382,59.4895761)(572.33355009,59.27088747)(572.52222519,59.08650434)
\curveto(572.71089517,58.90211842)(572.95102385,58.80992616)(573.24261196,58.80992728)
\curveto(573.56849761,58.80992616)(573.8793785,58.91712646)(574.17525555,59.13152852)
\curveto(574.39393996,59.29447154)(574.53758837,59.49386411)(574.60620121,59.72970682)
\curveto(574.6533647,59.88407322)(574.67694877,60.17780206)(574.67695348,60.61089421)
\lineto(574.67695348,60.9710876)
}
}
{
\newrgbcolor{curcolor}{0 0 0}
\pscustom[linestyle=none,fillstyle=solid,fillcolor=curcolor]
{
\newpath
\moveto(581.37269112,64.52156527)
\lineto(581.37269112,63.08079172)
\lineto(580.13774237,63.08079172)
\lineto(580.13774237,60.32788512)
\curveto(580.13773952,59.7704409)(580.14845955,59.44455197)(580.16990249,59.35021736)
\curveto(580.19562769,59.26016744)(580.24922784,59.18512723)(580.33070311,59.12509649)
\curveto(580.41646032,59.06506289)(580.51937261,59.0350468)(580.6394403,59.03504814)
\curveto(580.80666943,59.0350468)(581.04894212,59.09293496)(581.3662591,59.20871281)
\lineto(581.52062769,57.80653141)
\curveto(581.10039827,57.62643479)(580.62442891,57.53638653)(580.09271819,57.53638637)
\curveto(579.76682647,57.53638653)(579.47309763,57.58998668)(579.2115308,57.69718699)
\curveto(578.94996014,57.8086753)(578.75699959,57.95017971)(578.63264857,58.12170063)
\curveto(578.51258289,58.2975087)(578.42896665,58.53334937)(578.38179961,58.82922335)
\curveto(578.34320641,59.03933481)(578.32391035,59.46384802)(578.32391138,60.10276425)
\lineto(578.32391138,63.08079172)
\lineto(577.49418019,63.08079172)
\lineto(577.49418019,64.52156527)
\lineto(578.32391138,64.52156527)
\lineto(578.32391138,65.87872249)
\lineto(580.13774237,66.93357455)
\lineto(580.13774237,64.52156527)
\lineto(581.37269112,64.52156527)
}
}
{
\newrgbcolor{curcolor}{0 0 0}
\pscustom[linestyle=none,fillstyle=solid,fillcolor=curcolor]
{
\newpath
\moveto(582.63980068,65.44777683)
\lineto(582.63980068,67.12010327)
\lineto(584.44719963,67.12010327)
\lineto(584.44719963,65.44777683)
\lineto(582.63980068,65.44777683)
\moveto(582.63980068,57.69075497)
\lineto(582.63980068,64.52156527)
\lineto(584.44719963,64.52156527)
\lineto(584.44719963,57.69075497)
\lineto(582.63980068,57.69075497)
}
}
{
\newrgbcolor{curcolor}{0 0 0}
\pscustom[linestyle=none,fillstyle=solid,fillcolor=curcolor]
{
\newpath
\moveto(585.87510804,61.20264049)
\curveto(585.87510752,61.80295869)(586.02304394,62.38398434)(586.31891775,62.9457192)
\curveto(586.61478962,63.50744355)(587.03287082,63.93624477)(587.57316258,64.23212415)
\curveto(588.11773791,64.52799045)(588.72449164,64.67592688)(589.39342559,64.67593386)
\curveto(590.42683249,64.67592688)(591.2737149,64.33931792)(591.93407537,63.66610597)
\curveto(592.59442266,62.99717009)(592.92459961,62.15028768)(592.92460718,61.12545619)
\curveto(592.92459961,60.09204181)(592.59013465,59.23443937)(591.92121132,58.55264629)
\curveto(591.25656285,57.87513949)(590.41825646,57.53638653)(589.40628964,57.53638637)
\curveto(588.7802358,57.53638653)(588.18205809,57.67789093)(587.61175473,57.96090001)
\curveto(587.04573485,58.24390855)(586.61478962,58.65770172)(586.31891775,59.20228079)
\curveto(586.02304394,59.75114484)(585.87510752,60.41793074)(585.87510804,61.20264049)
\moveto(587.72753117,61.10616012)
\curveto(587.7275288,60.42865077)(587.88832925,59.90980129)(588.20993303,59.54961012)
\curveto(588.53153109,59.18941524)(588.92817222,59.00931873)(589.39985761,59.00932004)
\curveto(589.87153491,59.00931873)(590.26603203,59.18941524)(590.58335017,59.54961012)
\curveto(590.90494585,59.90980129)(591.06574631,60.43293878)(591.06575203,61.11902417)
\curveto(591.06574631,61.78795064)(590.90494585,62.30251211)(590.58335017,62.66271011)
\curveto(590.26603203,63.02289816)(589.87153491,63.20299468)(589.39985761,63.20300019)
\curveto(588.92817222,63.20299468)(588.53153109,63.02289816)(588.20993303,62.66271011)
\curveto(587.88832925,62.30251211)(587.7275288,61.78366263)(587.72753117,61.10616012)
}
}
{
\newrgbcolor{curcolor}{0 0 0}
\pscustom[linestyle=none,fillstyle=solid,fillcolor=curcolor]
{
\newpath
\moveto(600.55942017,57.69075497)
\lineto(598.75202121,57.69075497)
\lineto(598.75202121,61.17691239)
\curveto(598.75201586,61.91444701)(598.71342375,62.39041636)(598.63624476,62.60482189)
\curveto(598.55905531,62.8235056)(598.43255895,62.99288208)(598.2567553,63.11295184)
\curveto(598.08522996,63.23301076)(597.87726136,63.29304293)(597.6328489,63.29304854)
\curveto(597.31981977,63.29304293)(597.03895497,63.20728269)(596.79025365,63.03576755)
\curveto(596.54154556,62.86424171)(596.37002507,62.63697706)(596.27569167,62.35397292)
\curveto(596.18564054,62.07095945)(596.14061641,61.54782196)(596.14061915,60.78455888)
\lineto(596.14061915,57.69075497)
\lineto(594.3332202,57.69075497)
\lineto(594.3332202,64.52156527)
\lineto(596.01197866,64.52156527)
\lineto(596.01197866,63.5181694)
\curveto(596.60800975,64.29000578)(597.35841188,64.67592688)(598.26318733,64.67593386)
\curveto(598.6619676,64.67592688)(599.02644864,64.60303067)(599.35663153,64.45724502)
\curveto(599.68680252,64.31573385)(599.93550723,64.13349333)(600.10274641,63.91052291)
\curveto(600.27426019,63.68754006)(600.39218053,63.43454734)(600.45650777,63.15154399)
\curveto(600.52510891,62.86852972)(600.55941301,62.46331257)(600.55942017,61.93589131)
\lineto(600.55942017,57.69075497)
}
}
{
\newrgbcolor{curcolor}{0 0 0}
\pscustom[linestyle=none,fillstyle=solid,fillcolor=curcolor]
{
\newpath
\moveto(601.76220457,59.63965847)
\lineto(603.57603555,59.91623554)
\curveto(603.65321765,59.56461631)(603.8097301,59.29661555)(604.04557336,59.11223244)
\curveto(604.28141144,58.93213451)(604.61158838,58.84208625)(605.03610518,58.8420874)
\curveto(605.50349492,58.84208625)(605.85511193,58.92784649)(606.09095724,59.09936839)
\curveto(606.24960905,59.21943133)(606.32893728,59.38023178)(606.32894215,59.58177025)
\curveto(606.32893728,59.71898475)(606.28605715,59.83261707)(606.20030166,59.92266756)
\curveto(606.11024865,60.00842557)(605.90871208,60.0877538)(605.59569133,60.16065248)
\curveto(604.13776303,60.48225092)(603.2136964,60.77597976)(602.82348866,61.04183987)
\curveto(602.28319775,61.41060557)(602.01305298,61.92302303)(602.01305354,62.57909379)
\curveto(602.01305298,63.17083459)(602.24674964,63.668244)(602.71414424,64.07132353)
\curveto(603.18153631,64.4743903)(603.90621037,64.67592688)(604.88816861,64.67593386)
\curveto(605.82295183,64.67592688)(606.51760981,64.52370244)(606.97214463,64.2192601)
\curveto(607.42666841,63.91480471)(607.7396933,63.46456342)(607.91122024,62.8685349)
\lineto(606.20673368,62.55336569)
\curveto(606.13383272,62.81921758)(605.99447232,63.02289816)(605.78865207,63.16440804)
\curveto(605.58711116,63.30590697)(605.29767034,63.37665917)(604.92032873,63.37666486)
\curveto(604.44435591,63.37665917)(604.10345893,63.31019498)(603.89763679,63.17727209)
\curveto(603.76041796,63.08293034)(603.69180976,62.96072199)(603.691812,62.81064668)
\curveto(603.69180976,62.68200119)(603.75184193,62.57265688)(603.87190869,62.48261342)
\curveto(604.03485074,62.36254428)(604.59658034,62.1931678)(605.55709918,61.97448346)
\curveto(606.52189783,61.75579055)(607.19511575,61.48778979)(607.57675496,61.17048037)
\curveto(607.95409391,60.84887597)(608.14276645,60.40077869)(608.14277314,59.82618719)
\curveto(608.14276645,59.20013527)(607.8811977,58.66198974)(607.35806611,58.21174897)
\curveto(606.83492272,57.76150717)(606.06093651,57.53638653)(605.03610518,57.53638637)
\curveto(604.10560294,57.53638653)(603.36806484,57.72505907)(602.82348866,58.10240455)
\curveto(602.28319775,58.47974922)(601.92943674,58.99216668)(601.76220457,59.63965847)
}
}
{
\newrgbcolor{curcolor}{0.7019608 0.7019608 0.7019608}
\pscustom[linewidth=3,linecolor=curcolor]
{
\newpath
\moveto(492.95444,879.33778656)
\lineto(643.46717,828.32508656)
}
}
{
\newrgbcolor{curcolor}{0.7019608 0.7019608 0.7019608}
\pscustom[linewidth=3,linecolor=curcolor]
{
\newpath
\moveto(446.9925,879.84286656)
\lineto(446.9925,699.53063656)
}
}
{
\newrgbcolor{curcolor}{0.7019608 0.7019608 0.7019608}
\pscustom[linewidth=3,linecolor=curcolor]
{
\newpath
\moveto(198.95109,878.71015656)
\lineto(80.128556,818.96848656)
}
}
{
\newrgbcolor{curcolor}{0.7019608 0.7019608 0.7019608}
\pscustom[linewidth=3,linecolor=curcolor]
{
\newpath
\moveto(78.286822,757.75708656)
\lineto(78.286822,700.09699656)
}
}
{
\newrgbcolor{curcolor}{0.7019608 0.7019608 0.7019608}
\pscustom[linewidth=3,linecolor=curcolor]
{
\newpath
\moveto(445.6988,638.58972656)
\lineto(328.09036,580.24597656)
}
}
{
\newrgbcolor{curcolor}{0.7019608 0.7019608 0.7019608}
\pscustom[linewidth=3,linecolor=curcolor]
{
\newpath
\moveto(76.771593,637.75814656)
\lineto(76.771593,580.83771656)
}
}
{
\newrgbcolor{curcolor}{0.7019608 0.7019608 0.7019608}
\pscustom[linewidth=3,linecolor=curcolor]
{
\newpath
\moveto(78.286822,518.65205656)
\lineto(78.286822,462.05812656)
}
}
{
\newrgbcolor{curcolor}{0.7019608 0.7019608 0.7019608}
\pscustom[linewidth=3,linecolor=curcolor]
{
\newpath
\moveto(325.46527,520.19792656)
\lineto(203.54574,460.12448656)
}
}
{
\newrgbcolor{curcolor}{0.7019608 0.7019608 0.7019608}
\pscustom[linewidth=3,linecolor=curcolor]
{
\newpath
\moveto(403.05087,469.72093656)
\lineto(325.64793,519.97602656)
\lineto(402.04071,351.02800656)
}
}
{
\newrgbcolor{curcolor}{0.7019608 0.7019608 0.7019608}
\pscustom[linewidth=3,linecolor=curcolor]
{
\newpath
\moveto(325.39539,520.22855656)
\lineto(402.69373,231.02906656)
}
}
{
\newrgbcolor{curcolor}{0.7019608 0.7019608 0.7019608}
\pscustom[linewidth=3,linecolor=curcolor]
{
\newpath
\moveto(79.358251,344.34997656)
\lineto(200.06885,399.57397656)
\lineto(94.72846,223.78468656)
}
}
{
\newrgbcolor{curcolor}{0.7019608 0.7019608 0.7019608}
\pscustom[linewidth=3,linecolor=curcolor]
{
\newpath
\moveto(201.02036,223.24371656)
\lineto(200.52929,399.59061656)
\lineto(327.63519,224.24248656)
}
}
{
\newrgbcolor{curcolor}{0.7019608 0.7019608 0.7019608}
\pscustom[linewidth=3,linecolor=curcolor]
{
\newpath
\moveto(279.81226,111.11677656)
\lineto(201.02036,161.62440656)
\lineto(80.307127,102.02540656)
}
}
{
\newrgbcolor{curcolor}{0.7019608 0.7019608 0.7019608}
\pscustom[linewidth=3,linecolor=curcolor]
{
\newpath
\moveto(140.3034668,957.7323494)
\lineto(511.2449646,957.7323494)
\lineto(511.2449646,866.60313797)
\lineto(140.3034668,866.60313797)
\closepath
}
}
{
\newrgbcolor{curcolor}{0.7019608 0.7019608 0.7019608}
\pscustom[linewidth=3,linecolor=curcolor]
{
\newpath
\moveto(125.25892,640.94178656)
\lineto(278.29702,588.41385656)
}
}
{
\newrgbcolor{curcolor}{0.7019608 0.7019608 0.7019608}
\pscustom[linewidth=3,linecolor=curcolor]
{
\newpath
\moveto(646.42857,588.21428656)
\lineto(445.71429,638.57142656)
\lineto(524.28572,588.92857656)
}
}
{
\newrgbcolor{curcolor}{0.7019608 0.7019608 0.7019608}
\pscustom[linewidth=3,linecolor=curcolor]
{
\newpath
\moveto(688.41896,520.22855656)
\lineto(688.41896,462.64986656)
}
}
{
\newrgbcolor{curcolor}{0.7019608 0.7019608 0.7019608}
\pscustom[linewidth=3,linecolor=curcolor]
{
\newpath
\moveto(688.41896,342.44171656)
\lineto(688.41896,399.51533656)
\lineto(766.70578,350.01785656)
}
}
{
\newrgbcolor{curcolor}{0.7019608 0.7019608 0.7019608}
\pscustom[linewidth=3,linecolor=curcolor]
{
\newpath
\moveto(688.41896,400.02040656)
\lineto(570.23111,340.92648656)
}
}
{
\newrgbcolor{curcolor}{0.7019608 0.7019608 0.7019608}
\pscustom[linewidth=3,linecolor=curcolor]
{
\newpath
\moveto(687.91388,220.71832656)
\lineto(687.91388,279.81225656)
\lineto(766.20071,231.83000656)
}
}
{
\newrgbcolor{curcolor}{0.7019608 0.7019608 0.7019608}
\pscustom[linewidth=3,linecolor=curcolor]
{
\newpath
\moveto(687.40881,279.81225656)
\lineto(566.1905,222.73863656)
}
}
{
\newrgbcolor{curcolor}{0.7019608 0.7019608 0.7019608}
\pscustom[linewidth=3,linecolor=curcolor]
{
\newpath
\moveto(685.3885,161.11932656)
\lineto(569.72604,103.03555656)
}
}
\end{pspicture}

\caption{Módulos a ser desarrollados, y sus grados de dependencia.}
\label{paquetes}
\end{figure}

Se ha definido cuatro tipos de módulos en el sistema, estos son:

\begin{description}
\item [base] Módulos que pertenecen al núcleo del sistema (estos están
representados con color rojo, en la parte superior de diagrama).
\item [middle] Módulos para creación de espacios y recursos (representados con
color morado en el diagrama).
\item [app] Módulos para administración de recursos, perfiles, y otros
(representados con verde en la parte inferior del diagrama).
\item [util] Módulos que agregan funcionalidad a otros módulos (representados
con color café en el diagrama).
\end{description}

Las modularidad establecida en el sistema representa un modelo básico de
separación y en ningún caso podría considerarse totalmente refinado (eso escapa
del alcance de los objetivos del sistema).

\section{Grupos y privilegios}

Adicionalmente a la implementación de una lógica modular para el sistema, es
deseable el manejo dinámico de permisos, de forma que un administrador pueda
definir una conjunto de funcionalidades disponibles para ciertas categorías de
usuarios.

Para esto se han establecido dos módulos:

\begin{description}
\item [Roles] Módulo para la administración de grupos de usuarios, este debe
contemplar todas las operaciones CRUD, mencionadas para un recurso, además de la
manipulación de grupos de permisos para un rol común.
\item [Privileges] Módulo para la administración de credenciales en el sistema,
se debe contemplar la creación dinámica de permisos definidos por algún módulo.
\end{description}

\section{Gestión de contenido}

Para finalizar el conjunto de requisitos que se han establecido desarrollar
en el sistema, se plantea la necesitad de creación y administración de
diversas plantillas web, además de poder definirse pequeños utilitarios
alrededor de una cierta página.

Para esto se ha definido la creación de un modulo:

\begin{description}
\item [Templates] Módulo encargado de la reenderización de contenido, además de
la administración de funcionalidad adicional a una página (widgets), y la
definición de regiones para el sistema (debe cumplir las funciones mas básicas
de un sistema de administración de contenido CMS).
\end{description}

\section{Planificación}

Una vez definidas las funcionalidades a ser desarrolladas, y las herramientas
con las que se cuenta para tal desarrollo, en esta sección se ha de definir la
planificación que se ha determinado seguir.

\subsection{Iteraciones}

Para comenzar se ha determinado realizar el desarrollo del proyecto en
iteraciones, estas están detalladas en el cuadro \ref{iteraciones}.

\begin{table}
\centering
\begin{tabular}{|c|l|p{8.0cm}|}
\hline
Iteración & Módulo & Descripción \\
\hline

\multirow{4}{*}{1} &
Usuarios (USERS) &
\multirow{4}{8cm}{Análisis, diseño, e implementación de las funciones para el
manejo de usuarios, además de la creación de datos de prueba, e implementación
de la lógica de autenticación.} \\
 &  & \\
 &  & \\
 &  & \\
\hline

\multirow{4}{*}{2} &
Paquetes (PACKAGES) &
\multirow{4}{8cm}{Análisis, diseño, implementación, evaluaciones de las
funciones que proveen modularidad, manejo de credenciales, y manejo de roles de
usuarios en el sistema.} \\
 & Privilegios (PRIVILEGES) & \\
 & Roles (ROLES) & \\
 &  & \\
\hline

\multirow{3}{*}{3} &
Rutas (ROUTES) &
\multirow{3}{8cm}{Análisis, diseño, e implementación de las funciones para el
manejo de peticiones HTTP, y gestión de contenido.} \\
 & Plantillas (TEMPLATES) & \\
 & & \\
\hline

\multirow{5}{*}{4} &
Espacios (SPACES) &
\multirow{5}{8cm}{Análisis, diseño, e implementación de las funciones de
administración de espacios virtuales, además de la creación de las funciones
generales para la adición de recursos, y funciones utilitarias.} \\
 & Áreas (AREAS) & \\
 & Gestiones (GESTIONS) & \\
 & & \\
 & & \\
\hline

\multirow{4}{*}{5} &
Materias (SUBJECTS) &
\multirow{4}{8cm}{Análisis, diseño, e implementación de los espacios virtuales
formales, de acuerdo a la estructura que se aplica en el contexto de
implantación (UMSS).} \\
 & Grupos (GROUPS) & \\
 & Equipos (TEAMS) & \\
 & & \\
\hline

\multirow{4}{*}{6} &
Comunidades (COMMUNITIES) &
\multirow{4}{8cm}{Análisis, diseño, e implementación de los espacios virtuales
informales, de acuerdo a las estructuras clásicas que pueden verse en Internet.}
\\
 & Carreras (CARRERS) & \\
 & & \\
 & & \\
\hline

\multirow{3}{*}{7} &
Evaluaciones (EVALUATIONS) &
\multirow{3}{8cm}{Análisis, diseño, e implementación de las funciones para la
evaluación y calificaciones de los estudiantes, por parte de los docentes.} \\
 & Calificaciones (CALIFICATIONS) & \\
 & Conjuntos (GROUPSETS) & \\
\hline

\multirow{3}{*}{8} &
Contactos (CONTACTS) &
\multirow{3}{8cm}{Análisis, diseño, e implementación de la lógica de red social,
es decir, la gestión de contactos entre usuarios.} \\
 & Invitaciones (INVITATIONS) & \\
 & & \\
\hline

\multirow{7}{*}{9} &
Recursos (RESOURCES) &
\multirow{7}{8cm}{Análisis, diseño, e implementación de la gestión de recursos
básicos, además de la adición de estos a espacios virtuales determinados, y la
generalización de estos para brindar la posibilidad de extender su funcionalidad
para la posterior implementación de paquetes utilitarios.} \\
 & Notas (NOTES) & \\
 & Enlaces (LINKS) & \\
 & Sugerencias (FEEDBACK) & \\
 & & \\
 & & \\
 & & \\
\hline

\multirow{3}{*}{10} &
Archivos (FILES) &
\multirow{3}{8cm}{Análisis, diseño, e implementación de los recursos básicos
extendidos, es decir, aquellos que requieren manipular archivos adjuntos.} \\
 & Imágenes (PHOTOS) & \\
 & Vídeos (VIDEOS) & \\
\hline

\multirow{4}{*}{11} &
Etiquetas (TAGS) &
\multirow{4}{8cm}{Análisis, diseño, e implementación de las funciones
utilitarias para valoraciones sobre los recursos, además de la implementación
del sistema de reputación.} \\
 & Comentarios (COMMENTS) & \\
 & Valoraciones (VALORATIONS) & \\
 & & \\
\hline
\end{tabular}
\caption{Definición de iteraciones para el proyecto.}
\label{iteraciones}
\end{table}

