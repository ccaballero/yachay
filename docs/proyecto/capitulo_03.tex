\chapter{Metodología de desarrollo}

En este capítulo, se desarrollan los aspectos necesarios para la definición del 
proceso de desarrollo, primeramente se hará referencia a las cuestiones 
relacionadas con la metodología de desarrollo, posteriormente se verán los
documentos propios que se han definido para las diferentes etapas; luego se
tratarán las etapas de la planificación ha tomarse en cuenta.

\section{Modelo iterativo}

Considerando el contexto de desarrollo (el contexto esta descrito mas adelante
en este capitulo), se ha visto conveniente seguir un modelo de desarrollo que
sea iterativo e incremental\footnote{Para una definición exacta puede
consultarse: https://es.wikipedia.org/wiki/Desarrollo\_iterativo\_y\_creciente}.

La idea central es que, en cada una de esas iteraciones, se construye una parte
pequeña del sistema. Para esa parte del sistema, se realiza todo el proceso:
análisis, diseño, programación y pruebas. Se acaba la iteración con un
prototipo funcional, que incluya todas las partes del sistema construidas hasta
el momento. Los aspectos del sistema con más riesgo (por ejemplo, la
arquitectura) se definen y construyen en las primeras iteraciones.

Las ventajas de este tipo de modelo son las siguientes:

\begin{description}
\item [Flexibilidad] Los requerimientos no quedan totalmente fijados hasta el
final del proyecto de desarrollo. Por ello, se pueden realizar cambios de forma
flexible. Por una parte, el conocimiento que se adquiere en una iteración sirve
para plantear de forma más realista los requerimientos de la siguiente. Por
otra parte, este conocimiento nos puede hacer reformar partes del sistema
construidas en iteraciones anteriores. En una palabra, todos los documentos del
sistema (requerimientos, diseño y código) no son rígidos sino que pueden
cambiarse durante todo el proceso de desarrollo. (Típicamente suelen ser
modificados en mayor medida en las primeras iteraciones y en menor medida en
las últimas).
\item [Mitigación de riesgos] Como las pruebas se hacen desde el principio del
proyecto, puede determinarse la viabilidad o eficiencia de las decisiones de
diseño. Además, los elementos con más riesgo se tratan en las primeras
iteraciones, con lo cual se puede implementar una mitigación de riesgos más 
temprana y exitosa.
\item [Retroalimentación] Como hay prototipos desde el mismo comienzo del
proyecto, estos pueden examinarse, y revalorizarse. También existe una rápida
retroalimentación de lo que funciona y lo que no, ya que las pruebas se 
realizan desde el comienzo mismo del proyecto y no se debe esperar al final
para hacer las modificaciones necesarias.
\end{description}

\section{Requerimientos funcionales}

A partir de la solución y la estrategia planteada, se pasa a establecer el
conjunto de funciones que el sistema debe tener, para esto se ha optado por
crear un conjunto de categorias, de forma que agrupen las tareas que son comunes
a un objetivo especifico.

\subsection{Espacios virtuales}

Para la correcta navegación sobre el sistema se han establecido, varios tipos de
espacios agrupadores de recursos.

\begin{itemize}
\item Gestiones
\item Carreras
\item Areas
\item Comunidades
\item Materias
\item Grupos
\item Equipos
\end{itemize}

Cada uno de ellos posee la funcionalidad caracteristica de un recurso
administrable, es decir, posee las siguientes operaciones\footnote{Para una
definición exacta puede consultarse:
https://es.wikipedia.org/wiki/Create,\_read,\_update\_and\_delete}:

\begin{itemize}
\item Crear un nuevo elemento (CREATE).
\item Visualizar el elemento a detalle (READ).
\item Editar las caracteristicas del elemento (UPDATE).
\item Eliminación del elemento (DELETE).
\end{itemize}

A su vez se han establecido un conjunto de tareas por lote para facilitar la
correcta manipulación de amplios volumenes de información. Estas tareas son:

\begin{itemize}
\item Importación de datos desde un archivo CSV.
\item Exportación de datos hacia un archivo CSV.
\item Habilitación/Inhabilitación de elementos, ya sea individualmente o en
      grupos de elementos.
\end{itemize}

\subsection{Intercambio de recursos}

Cada espacio virtual debe poseer la capacidad de contener información en
distintos formatos, y para diversos propositos. El objetivo principal es poder
compartir piezas de información entre usuarios del sistema.

Para este proposito, se han definido piezas atomicas de información básica,
estas son:

\begin{itemize}
\item Notas
\item Archivos
\item Imagenes
\item Videos
\item Eventos
\item Enlaces
\end{itemize}

Todos estos tipos de recursos poseen tambien caracteristicas de espacio virtual,
es decir, que cada una de ellas posee operaciones CRUD, ademas de
funcionalidades para el fomento a la participación.

\subsection{Instancias multiples}

Se considera la creación de feeds de
sindicación\footnote{Definición disponible en:
http://es.wikipedia.org/wiki/Redifusión\_web}.

\subsection{Canales de comunicación}

Para la mejora de los canales de comunicación se ha definido el manejo de otros
tipos de espacios-recursos, estos son:

\begin{itemize}
\item Usuarios
\item Roles
\item Contactos
\end{itemize}

Estos recursos, son los componentes propios de un sitio web, ademas de darle las
caracteristicas de una red social propiamente dicha.

\subsection{Fomento a la participación}

La parte mas fundamental del sistema, y el factor clave para el exito de toda
red social, son las funcionalidades que propicien la cultura de participación.

Estos elementos, que estan inspirados en la tendencia de los sitios considerados
dentro de la web 2.0\footnote{Definición disponible en:
http://es.wikipedia.org/wiki/Web\_2.0}, han sido considerados como base para el
establecimiento y definición siguiente.

Para tal proposito se han definido los siguientes elementos:

\begin{itemize}
\item Comentarios
\item Valoraciones
\item Etiquetado
\item Sistema de reputación\footnote{Si bien vamos a ahondar en este concepto,
pueden verse los detalles introductorios en:
http://en.wikipedia.org/wiki/Reputation\_system}.
\end{itemize}

Estos elementos deben estar disponibles para cualquiera de los recursos
intercambiables definidos anteriormente.

\section{Requerimientos no funcionales}

\section{Estandarización}
\subsection{Estándares de análisis}
\subsection{Estándares de diseño}
\subsection{Estándares de codificación}
\subsection{Estándares de pruebas}

\section{Planificación}
\subsection{Iteraciones}

