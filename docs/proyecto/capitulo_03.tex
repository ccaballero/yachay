\chapter{Metodología de desarrollo}

En este capítulo, se desarrollan los aspectos necesarios para la definición del 
proceso de desarrollo, primeramente se hará referencia a las cuestiones 
relacionadas a la metodología de desarrollo, posteriormente se verán los
documentos propios que se han definido para las diferentes etapas; luego se
tratarán las etapas de la planificación ha tomarse en cuenta.

\section{Modelo iterativo}

Considerando el contexto de desarrollo, se ha visto conveniente seguir un modelo
de desarrollo que sea iterativo e incremental\footnote{Para una definición exacta
puede consultarse: https://es.wikipedia.org/wiki/Desarrollo\_iterativo\_y\_creciente}.

La idea central es que, en cada una de esas iteraciones, se construye una parte
pequeña del sistema. Para esa parte del sistema, se realiza todo el proceso:
análisis, diseño, programación y pruebas. Se acaba la iteración con un
ejecutable que incluye todas las partes del sistema construidas hasta el
momento. Los aspectos del sistema con más riesgo (por ejemplo, la arquitectura) 
se construyen en las primeras iteraciones.

Las ventajas de este tipo de modelo son las siguientes:
\begin{description}
\item [Flexibilidad] Los requerimientos no quedan totalmente fijados hasta el
final del proyecto de desarrollo. Por ello, se pueden realizar cambios de forma
flexible. Por una parte, el conocimiento que se adquiere en una iteración sirve
para plantear de forma más realista los requerimientos de la siguiente. Por
otra parte, este conocimiento nos puede hacer reformar partes del sistema
construidas en iteraciones anteriores. En una palabra, todos los documentos del
sistema (requerimientos, diseño y código) no son rígidos sino que pueden
cambiarse durante todo el proceso de desarrollo. (Típicamente suelen ser
modificados en mayor medida en las primeras iteraciones y en menor medida en
las últimas).
\item [Mitigación de riesgos] Como las pruebas se hacen desde el principio del
proyecto, puede determinarse la viabilidad o eficiencia de las decisiones de
diseño. Además, los elementos con más riesgo se tratan en las primeras
iteraciones, con lo cual se puede implementar una mitigación de riesgos más 
temprana y exitosa.
\item [Retroalimentación] Como hay prototipos desde el mismo comienzo del
proyecto, estos pueden examinarse, y revalorizarse. También existe una rápida
retroalimentación de lo que funciona y lo que no, ya que las pruebas se 
realizan desde el comienzo mismo del proyecto y no se debe esperar al final
para hacer las modificaciones necesarias.
\end{description}

\section{Estandarización}
\subsection{Estándares de análisis}
\subsection{Estándares de diseño}
\subsection{Estándares de codificación}
\subsection{Estándares de pruebas}

\section{Requerimientos funcionales}
\section{Requerimientos no funcionales}

\section{Planificación}
\subsection{Iteraciones}
