\chapter{El modelo base}

Como parte del análisis requerido para la desarrollo del sistema, en este
capítulo describiremos las características y limitaciones del sistema Yeah!,
antecesor del cuál se basaron las primeras ideas e intenciones del presente
trabajo.

\section{Funciones que moldean el sistema}
\begin{description}
\item [Gestión de usuarios] Componente encargado de la gestión de usuarios y roles de usuarios.
\item [Gestión de espacios virtuales] Componente encargado de la gestión de espacios de materias, espacios de discusión y espacios privados.
\item [Gestión de equipos de trabajo] Componente encargado de la gestión de equipos de trabajo entre los usuarios del sistema.
\item [Sub-sistema de administración del sistema] Componente encargado de facilitar la administración del sistema.
\item [Sub-sistema de administración de módulos] Componente encargado de la gestión de los módulos del sistema.
\end{description}

\section{Módulos}
El proyecto tenía desarrollados varios módulos, cuyas funciones iban desde la manipulación de menús, hasta el manejo de
comunidades. Una de las máximas desventajas del modelo base fue su alto acoplamiento, que hizó a los módulos
tremendamente interdependientes, haciendo de todo el sistema algo complejo de mantener.

En el cuadro~\ref{modulos_base} se detallan los módulos que ya estaban construidos con anterioridad y la función que
desempeñan.

\begin{table}
\begin{tabular}{l|l}
Módulo & Descripción \\
\hline
areas & Administración de áreas temáticas. \\
califications & Manejo de notas en los cursos. \\
careers & Administración de carreras. \\
comments & Manejo de comentarios. \\
communities & Administración de comunidades. \\
evaluations & Manejo de las formas de evaluación en los cursos. \\
events & Manejo de recursos tipo evento. \\
feedback & Manejo de mensajes de retroalimentación. \\
files & Manejo de recursos tipo archivo. \\
friends & Administración de la red de contactos. \\
frontpage & Manejo de pagina principal. \\
gestions & Administración de periodos académicos. \\
groups & Administración de grupos de estudio. \\
groupsets & Utilitario de manejo de múltiples grupos. \\
invitations & Manejo de invitaciones de registro. \\
links & Manejo de recursos tipo enlace. \\
login & Autentificación de usuarios. \\
menus & Manejo de regiones en plantilla. \\
packages & Administración de módulos. \\
notes & Manejo de recursos tipo nota. \\
pages & Administración de paginas. \\
paginator & Utilidad de paginación de recursos. \\
photos & Manejo de recursos tipo fotografía. \\
privileges & Administración de privilegios del sistema. \\
profile & Utilidad de configuración de información de usuario. \\
ratings & Manejo de rankings sobre recursos. \\
regions & Manejo genérico de regiones. \\
resources & Administración genérica de recursos. \\
roles & Administración de grupos de privilegios. \\
settings & Utilidad de configuración de usuario. \\
subjects & Administración de materias. \\
tags & Manejo de etiquetas sobre recursos. \\
teams & Administración de equipos de trabajo. \\
templates & Manejo de plantillas. \\
toolbar & Manejo de la region toolbar en plantilla. \\
users & Administración de usuarios. \\
valorations & Manejo de valoraciones en los usuarios. \\
videos & Manejo de recursos tipo video. \\
widgets & Manejo de widgets en plantilla. \\
\end{tabular}
\caption{Módulos existentes en el modelo base.}
\label{modulos_base}
\end{table}

\section{Críticas al modelo establecido}
Aunque se han discutido ya los problemas concernientes a la ya casi caduca estructuración del sistema, y el problema del
alto acoplamiento de los módulos; existían otros problemas menores que requerían plantearse.

\subsection{Sistema de plantillas}
Se tenían creadas dos plantillas, una de ellas, construida en html, y la otra, en xhtml con hojas de estilo (css2). Si
bien era posible crear plantillas que aprovechen toda la potencia de html, css, y javascript, resultaba este, un proceso
altamente costoso, debido al uso de paginas muy personalizadas, lo que conducía al casi imposible uso de la tecnología
ajax, haciendo de toda la arquitectura de plantillas implementada, cuestión de imperioso reemplazo.

\subsection{Instalación}
El sistema no poseía un método cómodo o claramente establecido para la instalación, lo que complicaba el uso de esta
herramienta a los administradores.

\subsection{Conectitividad}
Una de las características más llamativas para el sistema, es la conexión con otros sistemas, su bien las redes sociales
populares tienden hacia la centralización, el objetivo prioritario en este sistema será la descentralización de la
información, una cuestión muy análoga a lo referente a las tecnologías p2p.

\subsection{Modelos muy teoricos}
El problema final encontrado era la falta de roce del sistema con la vida real, lo que resultó un modelo de soluciones a
un campo de visión muy estrecho, siendo prioridad para este nuevo recorrido escuchar a los usuarios en todo su espectro
y categorización.
