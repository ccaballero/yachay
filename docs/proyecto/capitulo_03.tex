\chapter{El sistema}

En lo que respecta al sistema creado, antes de describir paso a paso, lo referente a sus funciones, primeramente 
describiremos aspectos referentes a las decisiones tomadas respecto a su instalación, tareas de mantenimiento y puesta en
marcha.

\section{Instalación}
Cabe recalcar que toda la hermenéutica descrita, se restringe al sistema operativo linux, no siendo probado o
corregido para ningun otro sistema operativo.

\subsection{Archivo de hosts}
Edicion del archivo de hosts (/etc/hosts) para la creacion del dominio y subdominios, dependiendo de los paquetes que se
tengan instalados, estos podrian ser entre otros los que siguen:

\small
\begin{verbatim}
127.0.0.1    yachay.local         -> Requerido
127.0.0.1    www.yachay.local     -> Requerido
127.0.0.1    xml.yachay.local     -> Requerido
127.0.0.1    dtd.yachay.local     -> Requerido

127.0.0.1    json.yachay.local    -> Requerido por el paquete routes
127.0.0.1    rss.yachay.local     -> Requerido por el paquete axon
127.0.0.1    meta.yachay.local    -> Requerido por el paquete axon
\end{verbatim}

En este caso \emph{yachay.local} es el dominio sobre el que estamos instalando.

\subsection{Host virtual}
Hecho esto, se pasa a configurar un host virtual para el servidor de aplicaciones web, en este caso Apache (y el unico en
el que se hicieron todas las pruebas):

\small
\begin{verbatim}
<VirtualHost *:80>
    ServerName yachay.local
    ServerAlias *.yachay.local
    DocumentRoot /var/www/Yachay/public

    SetEnv APPLICATION_ENV "production"

    LogLevel info
    ErrorLog "/var/www/Yachay/logs/error.log"
    CustomLog "/var/www/Yachay/logs/user-agent.log" "%{User-agent}i"
    CustomLog "/var/www/Yachay/logs/referer.log" "%{Referer}i"
    CustomLog "/var/www/Yachay/logs/resume.log" "%v %m %U%q"

    <Directory /home/carlos/Yachay/public>
        DirectoryIndex index.php
        AllowOverride All
        Order allow,deny
        Allow from all
    </Directory>
</VirtualHost>
\end{verbatim}

En este caso \emph{/var/www/Yachay} es el sitio donde esta instalado.

\subsection{Logs}
Ahora nos aseguramos que los archivos de log, posean los permisos suficientes y que existan. En caso de no existir, debe
reiniciar su servidor apache, que es el encargado de crearlos y escribir en ellos.

Como puede apreciarse en la configuración del host virtual; se definen 4 tipos de ficheros de log:
\begin{description}
\item [error.log] Fichero para los posibles errores de la aplicación.
\item [user-agent.log] Fichero que captura la variable \emph{User-Agent} del navegador cliente, este fichero es ampliamente discutido en el capitulo acerca de los sistemas de control.
\item [referer.log] Fichero que captura la variable \emph{Referer} del navegador cliente, tambien es muy util para tareas
de definición y utilización de indicadores.
\item [resume.log] Fichero de registro de acceso.
\end{description}

\subsection{Script instalador}
Para la instalación se dispone de un script de ejecución por consola, que se encargará del sistema base.

\small
\begin{verbatim}
$> cd shell
$> php installer.php --env=production --check
checking db resource............................................[OK]
checking the database adapter...................................[OK]
test the database connection....................................[OK]
check was completed successfully
$> php installer.php --env=production --install
installing the base package.....................................[OK]
install was completed successfully
\end{verbatim}

Ejecutamos la opcion check para la verificacion de las condiciones, es decir, el estado de los recursos vitales, como
por ejemplo la conexion hacia la base de datos.
Si todo resultó exitoso podremos ejecutar el install, y se crearán las tablas iniciales en la base de datos.

Las opciones y parametros de este y otros scripts se discutirán mas adelante en su totalidad.

\subsection{Archivo .htaccess}
Como parte de la definición de la arquitectura diseñada. se detalla tambien el archivo .htaccess utilizado, para manejar 
varios formatos de respuesta de la peticion, y asegurar la calidad de las url, se define como se muestra a continuacion:

\begin{verbatim}
RewriteEngine on
RewriteCond %{SCRIPT_FILENAME} !-f
RewriteCond %{HTTP_HOST} ^([a-z]*)\.yachay\.local$ [NC]
RewriteRule ^(.*)$ index.php?format=%1 [L,QSA]
RewriteCond %{SCRIPT_FILENAME} !-f
RewriteCond %{HTTP_HOST} ^yachay\.local$ [NC]
RewriteRule ^(.*)$ index.php?format=www [L,QSA]
\end{verbatim}

Pueden verse dos reglas; la primera para cualquier subdominio que los paquetes requieran; la segunda para el dominio
principal, de modo tal que en cualquier caso se envie por peticion GET la variable `format` que define el tipo de
respuesta que se espera.

\section{Linea de comandos}
Para reducir cualquier brecha de accesibilidad impuesta por la amplia variedad tecnologica en la que vivimos, se ha
considerado crear un conjunto de programas en linea de comandos, para la administracion eficiente del sistema en su
totalidad.

Todos los comandos creados, se encontrarán en /shell; estos podrán ser definidos por los paquetes, ademas de existir
algunos plenamente establecidos en tiempo de instalacion.

\subsection{installer.php}
Por defecto los unicos paquetes que se distribuyen con la aplicacion son los esenciales (packages, routes, privileges),
los demas paquetes pueden obtenerse desde internet o desde archivos especificos en el disco, pudiendo instalarse
cualquier paquete en cualquier momento, siempre y cuando se consideren las dependencias establecidas en cada uno.

Para eso se creó el script \emph{installer.php}, que se encarga de verificar e instalar nuevos paquetes, asi como
realizar otras tareas de mantenimiento y reparación, descritas en su totalidad en el capítulo 4.

Las opciones del script son:

\begin{description}
\item [--check] Ejecuta una verificacion de condiciones necesarias para la instalación.
\item [--install] Instala los paquetes 'base' segun las configuraciones establecidas.
\end{description}

\section{Sobre modulos, controladores y acciones}
Todo los recursos (no importando de que tipo, es decir, en su mayor grado de abstraccion) poseen en esencia dos
controladores iniciales:

\begin{description}
\item [index (list)] Que presenta una lista de los recursos disponibles.
\item [manager (admin)] Que presenta el conjunto de funciones disponibles sobre los recursos.
\end{description}

Estas funciones sobre los recursos, dependiendo del paquete, pueden ser:

\begin{description}
\item [new (put)] Funcion de creacion de un nuevo recurso.
\item [view (get)] Funcion de presentacion de la informacion de un recurso.
\item [edit (post)] Funcion de edicion del recurso.
\item [delete (delete)] Funcion de eliminacion del recurso.
\item [lock] Funcion de bloqueo o deshabilitacion del recurso.
\item [unlock] Funcion de desbloqueo o habilitacion del recurso.
\end{description}

\section{Paquetes construidos}
Si bien, ya existian una gran cantidad de paquetes (antes denominados modulos), contemplando varios aspectos, como la
modularidad, el acoplamiento, entre otros; fueron readecuados a los nuevos requerimientos, muchos fueron eliminados,
otros nuevos se construyeron, o fusionaron. En el cuadro~\ref{modulos_actuales} se detallan los paquetes de los que se
dispone en la actualidad y la función que desempeñan.

\begin{table}
\begin{tabular}{l|l}
Paquete & Descripción \\
\hline
packages & Administración de paquetes del sistema. \\
privileges & Administración de privilegios del sistema. \\
routes & Administración de direcciónes (URIs) del sistema. \\
i18n & Administración y soporte para la utilización de varios idiomas. \\
templates & Administración de plantillas basado en xslt 1.0. \\
\hline
cron & Administración de las funciones automaticas. \\
mailer & Administración de la cola de envio de correo electronico. \\
roles & Administración de conjuntos de privilegios. \\
users & Administración de usuarios. \\
search & Motor de busqueda sobre la información del sistema. \\
axon & Administración de conexiones remotas con pares. \\
spaces & Administración de espacios genericos del sistema. \\
resources & Administración de recursos genericos del sistema. \\
\hline
contacts & Administración de la red de contactos. \\
communities & Administración de comunidades. \\
gestions & Administración de periodos académicos. \\
subjects & Administración de materias. \\
careers & Administración de carreras. \\
areas & Administración de áreas temáticas. \\
groups & Administración de grupos de estudio. \\
teams & Administración de equipos de trabajo. \\
evaluations & Manejo de las formas de evaluación en los cursos. \\
califications & Manejo de notas en los cursos. \\
tasks & Manejo de tareas en los cursos. \\
notes & Manejo de recursos tipo nota. \\
feedback & Manejo de mensajes de retroalimentación. \\
events & Manejo de recursos tipo evento. \\
links & Manejo de recursos tipo enlace. \\
files & Manejo de recursos tipo archivo. \\
photos & Manejo de recursos tipo fotografía. \\
videos & Manejo de recursos tipo video. \\
tags & Manejo de etiquetas sobre recursos. \\
valorations & Manejo de valoraciones en los usuarios. \\
stats & Recopilador de información en general. \\
\hline
sets & Manejo de grupos de contacto. \\
invitations & Manejo de invitaciones de registro. \\
comments & Manejo de comentarios. \\
ratings & Manejo de rankings sobre recursos. \\
groupsets & Utilitario de manejo de múltiples grupos. \\
panels & Sistema de monitoreo y alarma sobre indicadores estadisticos. \\
awards & Sistema de manejo de reputaciones.
\end{tabular}
\caption{Paquetes construidos durante el proyecto.}
\label{modulos_actuales}
\end{table}

El objetivo de los siguientes capitulos tratan exclusivamente de la explicación detallada de estos paquetes, detallando
sus especificaciones y justificación de existencia, enfatizando en como estan relacionados con los objetivos del
proyecto.

Adicionalmente, en el cuadro~\ref{modulos_eliminados} se detallan los que antes eran modulos y ahora no estan presentes,
tambien se justifica su eliminación.

\begin{table}
\begin{tabular}{l|l}
Modulo & Detalle \\
\hline
frontpage & Sus funciones fueron incluidas en el paquete \emph{spaces}. \\
login & Sus funciones fueron incluidas en el paquete \emph{users}. \\
menus & Sus funciones fueron incluidas en el paquete \emph{templates}. \\
paginator & Sus funciones fueron incluidas en el paquete \emph{spaces}. \\
profile & Sus funciones fueron incluidas en el paquete \emph{users}. \\
regions & Sus funciones fueron incluidas en el paquete \emph{templates}. \\
settings & Sus funciones fueron incluidas en el paquete \emph{users}. \\
toolbar & Sus funciones fueron incluidas en el paquete \emph{templates}. \\
widgets & Sus funciones fueron incluidas en el paquete \emph{templates}. \\
\end{tabular}
\caption{Modulos que no llegaron a ser paquetes.}
\label{modulos_eliminados}
\end{table}
