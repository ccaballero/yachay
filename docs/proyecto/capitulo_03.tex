\chapter{Metodología de desarrollo}

En este capítulo, se desarrollan los aspectos necesarios para la definición del 
proceso de desarrollo, primeramente se hará referencia a las cuestiones 
relacionadas con la metodología de desarrollo, posteriormente se detallarán los
requisitos del sistema, introduciendo los conceptos claves que se utilizan en el
producto de software; para terminar con una descripción de las etapas
establecidas según la planificación del proyecto.

\section{Modelo iterativo}

Considerando el contexto de desarrollo (el contexto esta descrito mas adelante
en este capitulo), se ha visto conveniente seguir un modelo de desarrollo que
sea iterativo e incremental\footnote{Para una definición exacta puede
consultarse:
https://es.wikipedia.org/wiki/Desarrollo\_iterativo\_y\_creciente}, estos dos
conceptos se definen a continuación\cite{Cockburn}:

\begin{description}
\item [Desarrollo Incremental] Es una estrategia de desarrollo en la cual se
desarrollan diversas partes del sistema en diferentes etapas, y estas son
integradas a medida que son completadas.
\item [Desarrollo Iterativo] Es una estrategia de desarrollo en la cual se
reserva una parte del tiempo para la revision y mejora de partes del sistema.
\end{description}

La idea central es que, en cada una de esas iteraciones, se construye una parte
pequeña del sistema. Para esa parte del sistema, se realiza todo el proceso:
análisis, diseño, programación y pruebas. Se acaba la iteración con un
prototipo funcional, que incluya todas las partes del sistema construidas hasta
el momento. Los aspectos del sistema con más riesgo (por ejemplo, la
arquitectura) se definen y construyen en las primeras iteraciones.

Las ventajas de este modelo son las siguientes:

\begin{description}
\item [Flexibilidad] Los requisitos funcionales no quedan totalmente fijados
hasta el final del proyecto de desarrollo. Por ello, se pueden realizar cambios
de forma flexible. Por una parte, el conocimiento que se adquiere en una
iteración sirve para plantear de forma más realista los requisitos de la
siguiente. Por otra parte, este conocimiento nos puede hacer reformar partes del
sistema construidas en iteraciones anteriores. En una palabra, todos los
documentos del sistema (requisitos funcionales, diseño de datos y código fuente)
son flexibles y pueden cambiar durante todo el proceso de desarrollo
(Típicamente suelen ser modificados en mayor medida en las primeras iteraciones
y en menor medida en las últimas).
\item [Mitigación de riesgos] Como las pruebas se hacen desde el principio del
proyecto, puede determinarse la viabilidad o eficiencia de las decisiones de
diseño. Además, los elementos con más riesgo se tratan en las primeras
iteraciones, con lo cual se puede implementar una mitigación de riesgos más 
temprana y exitosa.
\item [Retroalimentación] Como hay prototipos desde el mismo comienzo del
proyecto, estos pueden examinarse, y revalorizarse. También existe una rápida
retroalimentación de lo que funciona y lo que no, ya que las pruebas se 
realizan desde el comienzo mismo del proyecto y no se debe esperar al final
para hacer las modificaciones necesarias.
\end{description}

\section{Requisitos funcionales}

Un requisito funcional define una función del sistema de software o sus
componentes\footnote{Para una definición exacta puede consultarse:
https://es.wikipedia.org/wiki/Requisito\_funcional}, y estos detallan
completamente las capacidades que un sistema posee; en nuestro caso, estos han
sido clasificados según el objetivo especifico del proyecto al cual estos
contribuyen.

\subsection{Espacios virtuales}

Uno de los puntos fundamentales en la construcción del sistema, es el control y
manejo organizado de los espacios disponibles del sistema, estos espacios
constituyen los lugares, de intercambio, producción, y discusión de los recursos
que posea el sistema.

Estos recursos además pueden clasificarse según su temporalidad, es decir si
poseen alguna forma de caducidad, o si no poseen tal cualidad. Estos son:

\begin{description}
\item [Temporales] Es un espacio temporal todo aquel que depende de la gestión
en la que uno se encuentre; tanto su acceso, como su visibilidad están
delimitadas por la gestión que este presentándose, estos espacios son lo que se
construyeron inicialmente, ejemplos de estos son: materias, grupos, etc.
\item [Atemporales] Es un espacio atemporal aquel que no esta englobado en una
gestión determinada, su acceso y visibilidad es independiente, ejemplos de
estos son: comunidades, áreas, etc.
\end{description}

Los espacios virtuales que se han planificado construir son los siguientes:

\begin{description}
\item [Gestiones] Una gestión representa la división básica de periodos
académicos, estos trazan un marco de referencia temporal (es decir, su valor de
caducidad) para muchos de los espacios restantes.
\item [Materias] Una materia es el espacio que concentra todos los recursos de
una materia, (esta a su vez concentra a otros sub-espacios). Este espacio es a
su vez un sub-espacio de algún espacio de gestión.
\item [Grupos] Los grupos son espacios de separación de una materia, esta está
basada en el sistema utilizado en el dominio de implementación del sistema
(UMSS).
\item [Equipos] Los equipos son espacios opcionales de creación, que pueden
utilizarse para dividir aún mas un grupo de estudio, según el método que el
docente pretenda utilizar.
\item [Carreras] Las carreras representan una concentración de materias que a su
vez están agrupadas según gestiones especificas.
\item [Áreas] Un área es otra forma de agrupación de materias, que carecen de
una cualidad temporal, es decir, que no poseen caducidad.
\item [Comunidades] Una comunidad es una forma de espacio virtual independiente
de toda gestión (lo que implica que no tiene caducidad), y la intención es poder
agrupar a los usuarios según un interés en particular.
\end{description}

En la figura (\ref{espacios}) pueden apreciarse los espacios virtuales que se
construirán, remarcando su característica temporal. Como se verá posteriormente
la clasificación de los espacios según su temporalidad es imprescindible para
corregir los inconvenientes creados por la formalidad que poseen algunos
espacios, brindando espacios que poseen un carácter mas libre.

\begin{figure}
\centering
%LaTeX with PSTricks extensions
%%Creator: inkscape 0.48.4
%%Please note this file requires PSTricks extensions
\psset{xunit=.5pt,yunit=.5pt,runit=.5pt}
\begin{pspicture}(531.49603271,425.19683838)
{
\newrgbcolor{curcolor}{0 0 0}
\pscustom[linestyle=none,fillstyle=solid,fillcolor=curcolor]
{
\newpath
\moveto(50.2411504,357.87891102)
\lineto(52.0411504,357.87891102)
\lineto(52.0411504,369.15891102)
\lineto(42.6511504,369.15891102)
\lineto(42.6511504,366.75891102)
\lineto(49.4911504,366.75891102)
\curveto(49.67115022,362.85891492)(47.03114599,359.76891102)(42.6211504,359.76891102)
\curveto(37.85115517,359.76891102)(35.2711504,363.84891549)(35.2711504,368.31891102)
\curveto(35.2711504,372.90890643)(37.43115559,377.40891102)(42.6211504,377.40891102)
\curveto(45.80114722,377.40891102)(48.41115097,375.93890778)(48.9811504,372.69891102)
\lineto(51.8311504,372.69891102)
\curveto(51.02115121,377.70890601)(47.30114572,379.80891102)(42.6211504,379.80891102)
\curveto(35.84115718,379.80891102)(32.4211504,374.40890478)(32.4211504,368.16891102)
\curveto(32.4211504,362.5889166)(36.23115679,357.36891102)(42.6211504,357.36891102)
\curveto(45.14114788,357.36891102)(47.84115205,358.29891327)(49.4911504,360.54891102)
\lineto(50.2411504,357.87891102)
}
}
{
\newrgbcolor{curcolor}{0 0 0}
\pscustom[linestyle=none,fillstyle=solid,fillcolor=curcolor]
{
\newpath
\moveto(66.55411915,362.79891102)
\curveto(66.1041196,360.78891303)(64.63411705,359.76891102)(62.53411915,359.76891102)
\curveto(59.14412254,359.76891102)(57.61411924,362.16891372)(57.70411915,364.86891102)
\lineto(69.31411915,364.86891102)
\curveto(69.464119,368.61890727)(67.78411366,373.74891102)(62.29411915,373.74891102)
\curveto(58.06412338,373.74891102)(55.00411915,370.32890637)(55.00411915,365.67891102)
\curveto(55.154119,360.93891576)(57.4941241,357.51891102)(62.44411915,357.51891102)
\curveto(65.92411567,357.51891102)(68.38411984,359.37891444)(69.07411915,362.79891102)
\lineto(66.55411915,362.79891102)
\moveto(57.70411915,367.11891102)
\curveto(57.88411897,369.48890865)(59.47412182,371.49891102)(62.14411915,371.49891102)
\curveto(64.66411663,371.49891102)(66.49411927,369.54890859)(66.61411915,367.11891102)
\lineto(57.70411915,367.11891102)
}
}
{
\newrgbcolor{curcolor}{0 0 0}
\pscustom[linestyle=none,fillstyle=solid,fillcolor=curcolor]
{
\newpath
\moveto(70.9674004,362.76891102)
\curveto(71.11740025,358.92891486)(74.05740388,357.51891102)(77.5374004,357.51891102)
\curveto(80.68739725,357.51891102)(84.1374004,358.71891471)(84.1374004,362.40891102)
\curveto(84.1374004,365.40890802)(81.61739785,366.24891159)(79.0674004,366.81891102)
\curveto(76.69740277,367.38891045)(73.9974004,367.68891285)(73.9974004,369.51891102)
\curveto(73.9974004,371.07890946)(75.76740193,371.49891102)(77.2974004,371.49891102)
\curveto(78.97739872,371.49891102)(80.71740058,370.86890904)(80.8974004,368.88891102)
\lineto(83.4474004,368.88891102)
\curveto(83.23740061,372.66890724)(80.50739698,373.74891102)(77.0874004,373.74891102)
\curveto(74.3874031,373.74891102)(71.2974004,372.4589079)(71.2974004,369.33891102)
\curveto(71.2974004,366.36891399)(73.84740292,365.52891045)(76.3674004,364.95891102)
\curveto(78.91739785,364.38891159)(81.4374004,364.05890904)(81.4374004,362.07891102)
\curveto(81.4374004,360.12891297)(79.27739881,359.76891102)(77.6874004,359.76891102)
\curveto(75.5874025,359.76891102)(73.60740031,360.4889133)(73.5174004,362.76891102)
\lineto(70.9674004,362.76891102)
}
}
{
\newrgbcolor{curcolor}{0 0 0}
\pscustom[linestyle=none,fillstyle=solid,fillcolor=curcolor]
{
\newpath
\moveto(90.4974004,378.03891102)
\lineto(87.9474004,378.03891102)
\lineto(87.9474004,373.38891102)
\lineto(85.3074004,373.38891102)
\lineto(85.3074004,371.13891102)
\lineto(87.9474004,371.13891102)
\lineto(87.9474004,361.26891102)
\curveto(87.9474004,358.41891387)(88.99740304,357.87891102)(91.6374004,357.87891102)
\lineto(93.5874004,357.87891102)
\lineto(93.5874004,360.12891102)
\lineto(92.4174004,360.12891102)
\curveto(90.82740199,360.12891102)(90.4974004,360.33891219)(90.4974004,361.50891102)
\lineto(90.4974004,371.13891102)
\lineto(93.5874004,371.13891102)
\lineto(93.5874004,373.38891102)
\lineto(90.4974004,373.38891102)
\lineto(90.4974004,378.03891102)
}
}
{
\newrgbcolor{curcolor}{0 0 0}
\pscustom[linestyle=none,fillstyle=solid,fillcolor=curcolor]
{
\newpath
\moveto(96.54099415,357.87891102)
\lineto(99.09099415,357.87891102)
\lineto(99.09099415,373.38891102)
\lineto(96.54099415,373.38891102)
\lineto(96.54099415,357.87891102)
\moveto(99.09099415,379.29891102)
\lineto(96.54099415,379.29891102)
\lineto(96.54099415,376.17891102)
\lineto(99.09099415,376.17891102)
\lineto(99.09099415,379.29891102)
}
}
{
\newrgbcolor{curcolor}{0 0 0}
\pscustom[linestyle=none,fillstyle=solid,fillcolor=curcolor]
{
\newpath
\moveto(102.23068165,365.61891102)
\curveto(102.23068165,361.08891555)(104.84068657,357.51891102)(109.76068165,357.51891102)
\curveto(114.68067673,357.51891102)(117.29068165,361.08891555)(117.29068165,365.61891102)
\curveto(117.29068165,370.17890646)(114.68067673,373.74891102)(109.76068165,373.74891102)
\curveto(104.84068657,373.74891102)(102.23068165,370.17890646)(102.23068165,365.61891102)
\moveto(104.93068165,365.61891102)
\curveto(104.93068165,369.39890724)(107.09068432,371.49891102)(109.76068165,371.49891102)
\curveto(112.43067898,371.49891102)(114.59068165,369.39890724)(114.59068165,365.61891102)
\curveto(114.59068165,361.86891477)(112.43067898,359.76891102)(109.76068165,359.76891102)
\curveto(107.09068432,359.76891102)(104.93068165,361.86891477)(104.93068165,365.61891102)
\moveto(107.93068165,375.51891102)
\lineto(109.85068165,375.51891102)
\lineto(113.78068165,379.80891102)
\lineto(110.51068165,379.80891102)
\lineto(107.93068165,375.51891102)
}
}
{
\newrgbcolor{curcolor}{0 0 0}
\pscustom[linestyle=none,fillstyle=solid,fillcolor=curcolor]
{
\newpath
\moveto(120.29724415,357.87891102)
\lineto(122.84724415,357.87891102)
\lineto(122.84724415,366.63891102)
\curveto(122.84724415,369.42890823)(124.34724724,371.49891102)(127.43724415,371.49891102)
\curveto(129.3872422,371.49891102)(130.58724415,370.26890913)(130.58724415,368.37891102)
\lineto(130.58724415,357.87891102)
\lineto(133.13724415,357.87891102)
\lineto(133.13724415,368.07891102)
\curveto(133.13724415,371.40890769)(131.87724007,373.74891102)(127.79724415,373.74891102)
\curveto(125.57724637,373.74891102)(123.83724307,372.8489091)(122.75724415,370.92891102)
\lineto(122.69724415,370.92891102)
\lineto(122.69724415,373.38891102)
\lineto(120.29724415,373.38891102)
\lineto(120.29724415,357.87891102)
}
}
{
\newrgbcolor{curcolor}{0 0 0}
\pscustom[linestyle=none,fillstyle=solid,fillcolor=curcolor]
{
\newpath
\moveto(82.93620258,305.22198963)
\lineto(85.63620258,305.22198963)
\lineto(85.63620258,323.04198963)
\lineto(85.69620258,323.04198963)
\lineto(92.38620258,305.22198963)
\lineto(94.81620258,305.22198963)
\lineto(101.50620258,323.04198963)
\lineto(101.56620258,323.04198963)
\lineto(101.56620258,305.22198963)
\lineto(104.26620258,305.22198963)
\lineto(104.26620258,326.64198963)
\lineto(100.36620258,326.64198963)
\lineto(93.58620258,308.64198963)
\lineto(86.83620258,326.64198963)
\lineto(82.93620258,326.64198963)
\lineto(82.93620258,305.22198963)
}
}
{
\newrgbcolor{curcolor}{0 0 0}
\pscustom[linestyle=none,fillstyle=solid,fillcolor=curcolor]
{
\newpath
\moveto(118.18901508,310.62198963)
\curveto(118.18901508,309.21199104)(116.80901172,307.11198963)(113.44901508,307.11198963)
\curveto(111.88901664,307.11198963)(110.44901508,307.71199131)(110.44901508,309.39198963)
\curveto(110.44901508,311.28198774)(111.88901676,311.88198993)(113.56901508,312.18198963)
\curveto(115.27901337,312.48198933)(117.19901607,312.51199035)(118.18901508,313.23198963)
\lineto(118.18901508,310.62198963)
\moveto(122.32901508,307.26198963)
\curveto(121.99901541,307.14198975)(121.75901487,307.11198963)(121.54901508,307.11198963)
\curveto(120.73901589,307.11198963)(120.73901508,307.65199083)(120.73901508,308.85198963)
\lineto(120.73901508,316.83198963)
\curveto(120.73901508,320.461986)(117.70901229,321.09198963)(114.91901508,321.09198963)
\curveto(111.46901853,321.09198963)(108.49901493,319.74198579)(108.34901508,315.90198963)
\lineto(110.89901508,315.90198963)
\curveto(111.01901496,318.18198735)(112.60901724,318.84198963)(114.76901508,318.84198963)
\curveto(116.38901346,318.84198963)(118.21901508,318.48198741)(118.21901508,316.26198963)
\curveto(118.21901508,314.34199155)(115.81901226,314.52198909)(112.99901508,313.98198963)
\curveto(110.35901772,313.47199014)(107.74901508,312.72198612)(107.74901508,309.21198963)
\curveto(107.74901508,306.12199272)(110.0590179,304.86198963)(112.87901508,304.86198963)
\curveto(115.03901292,304.86198963)(116.92901649,305.61199128)(118.33901508,307.26198963)
\curveto(118.33901508,305.58199131)(119.1790164,304.86198963)(120.49901508,304.86198963)
\curveto(121.30901427,304.86198963)(121.87901553,305.0119899)(122.32901508,305.28198963)
\lineto(122.32901508,307.26198963)
}
}
{
\newrgbcolor{curcolor}{0 0 0}
\pscustom[linestyle=none,fillstyle=solid,fillcolor=curcolor]
{
\newpath
\moveto(128.24229633,325.38198963)
\lineto(125.69229633,325.38198963)
\lineto(125.69229633,320.73198963)
\lineto(123.05229633,320.73198963)
\lineto(123.05229633,318.48198963)
\lineto(125.69229633,318.48198963)
\lineto(125.69229633,308.61198963)
\curveto(125.69229633,305.76199248)(126.74229897,305.22198963)(129.38229633,305.22198963)
\lineto(131.33229633,305.22198963)
\lineto(131.33229633,307.47198963)
\lineto(130.16229633,307.47198963)
\curveto(128.57229792,307.47198963)(128.24229633,307.6819908)(128.24229633,308.85198963)
\lineto(128.24229633,318.48198963)
\lineto(131.33229633,318.48198963)
\lineto(131.33229633,320.73198963)
\lineto(128.24229633,320.73198963)
\lineto(128.24229633,325.38198963)
}
}
{
\newrgbcolor{curcolor}{0 0 0}
\pscustom[linestyle=none,fillstyle=solid,fillcolor=curcolor]
{
\newpath
\moveto(144.84589008,310.14198963)
\curveto(144.39589053,308.13199164)(142.92588798,307.11198963)(140.82589008,307.11198963)
\curveto(137.43589347,307.11198963)(135.90589017,309.51199233)(135.99589008,312.21198963)
\lineto(147.60589008,312.21198963)
\curveto(147.75588993,315.96198588)(146.07588459,321.09198963)(140.58589008,321.09198963)
\curveto(136.35589431,321.09198963)(133.29589008,317.67198498)(133.29589008,313.02198963)
\curveto(133.44588993,308.28199437)(135.78589503,304.86198963)(140.73589008,304.86198963)
\curveto(144.2158866,304.86198963)(146.67589077,306.72199305)(147.36589008,310.14198963)
\lineto(144.84589008,310.14198963)
\moveto(135.99589008,314.46198963)
\curveto(136.1758899,316.83198726)(137.76589275,318.84198963)(140.43589008,318.84198963)
\curveto(142.95588756,318.84198963)(144.7858902,316.8919872)(144.90589008,314.46198963)
\lineto(135.99589008,314.46198963)
}
}
{
\newrgbcolor{curcolor}{0 0 0}
\pscustom[linestyle=none,fillstyle=solid,fillcolor=curcolor]
{
\newpath
\moveto(150.15917133,305.22198963)
\lineto(152.70917133,305.22198963)
\lineto(152.70917133,312.12198963)
\curveto(152.70917133,316.0519857)(154.20917544,318.39198963)(158.31917133,318.39198963)
\lineto(158.31917133,321.09198963)
\curveto(155.55917409,321.18198954)(153.8491701,319.95198714)(152.61917133,317.46198963)
\lineto(152.55917133,317.46198963)
\lineto(152.55917133,320.73198963)
\lineto(150.15917133,320.73198963)
\lineto(150.15917133,305.22198963)
}
}
{
\newrgbcolor{curcolor}{0 0 0}
\pscustom[linestyle=none,fillstyle=solid,fillcolor=curcolor]
{
\newpath
\moveto(160.41870258,305.22198963)
\lineto(162.96870258,305.22198963)
\lineto(162.96870258,320.73198963)
\lineto(160.41870258,320.73198963)
\lineto(160.41870258,305.22198963)
\moveto(162.96870258,326.64198963)
\lineto(160.41870258,326.64198963)
\lineto(160.41870258,323.52198963)
\lineto(162.96870258,323.52198963)
\lineto(162.96870258,326.64198963)
}
}
{
\newrgbcolor{curcolor}{0 0 0}
\pscustom[linestyle=none,fillstyle=solid,fillcolor=curcolor]
{
\newpath
\moveto(176.54839008,310.62198963)
\curveto(176.54839008,309.21199104)(175.16838672,307.11198963)(171.80839008,307.11198963)
\curveto(170.24839164,307.11198963)(168.80839008,307.71199131)(168.80839008,309.39198963)
\curveto(168.80839008,311.28198774)(170.24839176,311.88198993)(171.92839008,312.18198963)
\curveto(173.63838837,312.48198933)(175.55839107,312.51199035)(176.54839008,313.23198963)
\lineto(176.54839008,310.62198963)
\moveto(180.68839008,307.26198963)
\curveto(180.35839041,307.14198975)(180.11838987,307.11198963)(179.90839008,307.11198963)
\curveto(179.09839089,307.11198963)(179.09839008,307.65199083)(179.09839008,308.85198963)
\lineto(179.09839008,316.83198963)
\curveto(179.09839008,320.461986)(176.06838729,321.09198963)(173.27839008,321.09198963)
\curveto(169.82839353,321.09198963)(166.85838993,319.74198579)(166.70839008,315.90198963)
\lineto(169.25839008,315.90198963)
\curveto(169.37838996,318.18198735)(170.96839224,318.84198963)(173.12839008,318.84198963)
\curveto(174.74838846,318.84198963)(176.57839008,318.48198741)(176.57839008,316.26198963)
\curveto(176.57839008,314.34199155)(174.17838726,314.52198909)(171.35839008,313.98198963)
\curveto(168.71839272,313.47199014)(166.10839008,312.72198612)(166.10839008,309.21198963)
\curveto(166.10839008,306.12199272)(168.4183929,304.86198963)(171.23839008,304.86198963)
\curveto(173.39838792,304.86198963)(175.28839149,305.61199128)(176.69839008,307.26198963)
\curveto(176.69839008,305.58199131)(177.5383914,304.86198963)(178.85839008,304.86198963)
\curveto(179.66838927,304.86198963)(180.23839053,305.0119899)(180.68839008,305.28198963)
\lineto(180.68839008,307.26198963)
}
}
{
\newrgbcolor{curcolor}{0 0 0}
\pscustom[linestyle=none,fillstyle=solid,fillcolor=curcolor]
{
\newpath
\moveto(148.30735065,252.20502186)
\lineto(150.10735065,252.20502186)
\lineto(150.10735065,263.48502186)
\lineto(140.71735065,263.48502186)
\lineto(140.71735065,261.08502186)
\lineto(147.55735065,261.08502186)
\curveto(147.73735047,257.18502576)(145.09734624,254.09502186)(140.68735065,254.09502186)
\curveto(135.91735542,254.09502186)(133.33735065,258.17502633)(133.33735065,262.64502186)
\curveto(133.33735065,267.23501727)(135.49735584,271.73502186)(140.68735065,271.73502186)
\curveto(143.86734747,271.73502186)(146.47735122,270.26501862)(147.04735065,267.02502186)
\lineto(149.89735065,267.02502186)
\curveto(149.08735146,272.03501685)(145.36734597,274.13502186)(140.68735065,274.13502186)
\curveto(133.90735743,274.13502186)(130.48735065,268.73501562)(130.48735065,262.49502186)
\curveto(130.48735065,256.91502744)(134.29735704,251.69502186)(140.68735065,251.69502186)
\curveto(143.20734813,251.69502186)(145.9073523,252.62502411)(147.55735065,254.87502186)
\lineto(148.30735065,252.20502186)
}
}
{
\newrgbcolor{curcolor}{0 0 0}
\pscustom[linestyle=none,fillstyle=solid,fillcolor=curcolor]
{
\newpath
\moveto(153.8203194,252.20502186)
\lineto(156.3703194,252.20502186)
\lineto(156.3703194,259.10502186)
\curveto(156.3703194,263.03501793)(157.87032351,265.37502186)(161.9803194,265.37502186)
\lineto(161.9803194,268.07502186)
\curveto(159.22032216,268.16502177)(157.51031817,266.93501937)(156.2803194,264.44502186)
\lineto(156.2203194,264.44502186)
\lineto(156.2203194,267.71502186)
\lineto(153.8203194,267.71502186)
\lineto(153.8203194,252.20502186)
}
}
{
\newrgbcolor{curcolor}{0 0 0}
\pscustom[linestyle=none,fillstyle=solid,fillcolor=curcolor]
{
\newpath
\moveto(176.76985065,267.71502186)
\lineto(174.21985065,267.71502186)
\lineto(174.21985065,258.95502186)
\curveto(174.21985065,256.16502465)(172.71984756,254.09502186)(169.62985065,254.09502186)
\curveto(167.6798526,254.09502186)(166.47985065,255.32502375)(166.47985065,257.21502186)
\lineto(166.47985065,267.71502186)
\lineto(163.92985065,267.71502186)
\lineto(163.92985065,257.51502186)
\curveto(163.92985065,254.18502519)(165.18985473,251.84502186)(169.26985065,251.84502186)
\curveto(171.48984843,251.84502186)(173.22985173,252.74502378)(174.30985065,254.66502186)
\lineto(174.36985065,254.66502186)
\lineto(174.36985065,252.20502186)
\lineto(176.76985065,252.20502186)
\lineto(176.76985065,267.71502186)
}
}
{
\newrgbcolor{curcolor}{0 0 0}
\pscustom[linestyle=none,fillstyle=solid,fillcolor=curcolor]
{
\newpath
\moveto(192.7190694,260.09502186)
\curveto(192.7190694,257.06502489)(191.54906592,254.09502186)(188.0690694,254.09502186)
\curveto(184.55907291,254.09502186)(183.1790694,256.91502492)(183.1790694,259.97502186)
\curveto(183.1790694,262.88501895)(184.49907282,265.82502186)(187.9190694,265.82502186)
\curveto(191.2190661,265.82502186)(192.7190694,263.00501895)(192.7190694,260.09502186)
\moveto(180.7190694,246.26502186)
\lineto(183.2690694,246.26502186)
\lineto(183.2690694,254.27502186)
\lineto(183.3290694,254.27502186)
\curveto(184.46906826,252.44502369)(186.74907099,251.84502186)(188.3390694,251.84502186)
\curveto(193.07906466,251.84502186)(195.4190694,255.53502624)(195.4190694,259.91502186)
\curveto(195.4190694,264.29501748)(193.04906463,268.07502186)(188.2790694,268.07502186)
\curveto(186.14907153,268.07502186)(184.16906856,267.32502015)(183.3290694,265.61502186)
\lineto(183.2690694,265.61502186)
\lineto(183.2690694,267.71502186)
\lineto(180.7190694,267.71502186)
\lineto(180.7190694,246.26502186)
}
}
{
\newrgbcolor{curcolor}{0 0 0}
\pscustom[linestyle=none,fillstyle=solid,fillcolor=curcolor]
{
\newpath
\moveto(197.6015694,259.94502186)
\curveto(197.6015694,255.41502639)(200.21157432,251.84502186)(205.1315694,251.84502186)
\curveto(210.05156448,251.84502186)(212.6615694,255.41502639)(212.6615694,259.94502186)
\curveto(212.6615694,264.5050173)(210.05156448,268.07502186)(205.1315694,268.07502186)
\curveto(200.21157432,268.07502186)(197.6015694,264.5050173)(197.6015694,259.94502186)
\moveto(200.3015694,259.94502186)
\curveto(200.3015694,263.72501808)(202.46157207,265.82502186)(205.1315694,265.82502186)
\curveto(207.80156673,265.82502186)(209.9615694,263.72501808)(209.9615694,259.94502186)
\curveto(209.9615694,256.19502561)(207.80156673,254.09502186)(205.1315694,254.09502186)
\curveto(202.46157207,254.09502186)(200.3015694,256.19502561)(200.3015694,259.94502186)
}
}
{
\newrgbcolor{curcolor}{0 0 0}
\pscustom[linestyle=none,fillstyle=solid,fillcolor=curcolor]
{
\newpath
\moveto(162.46053791,194.11804676)
\lineto(177.34053791,194.11804676)
\lineto(177.34053791,196.51804676)
\lineto(165.31053791,196.51804676)
\lineto(165.31053791,203.92804676)
\lineto(176.44053791,203.92804676)
\lineto(176.44053791,206.32804676)
\lineto(165.31053791,206.32804676)
\lineto(165.31053791,213.13804676)
\lineto(177.25053791,213.13804676)
\lineto(177.25053791,215.53804676)
\lineto(162.46053791,215.53804676)
\lineto(162.46053791,194.11804676)
}
}
{
\newrgbcolor{curcolor}{0 0 0}
\pscustom[linestyle=none,fillstyle=solid,fillcolor=curcolor]
{
\newpath
\moveto(194.24038166,209.62804676)
\lineto(191.69038166,209.62804676)
\lineto(191.69038166,207.55804676)
\lineto(191.63038166,207.55804676)
\curveto(190.4903828,209.38804493)(188.21038007,209.98804676)(186.62038166,209.98804676)
\curveto(181.8803864,209.98804676)(179.54038166,206.29804238)(179.54038166,201.91804676)
\curveto(179.54038166,197.53805114)(181.91038643,193.75804676)(186.68038166,193.75804676)
\curveto(188.81037953,193.75804676)(190.7903825,194.50804847)(191.63038166,196.21804676)
\lineto(191.69038166,196.21804676)
\lineto(191.69038166,188.17804676)
\lineto(194.24038166,188.17804676)
\lineto(194.24038166,209.62804676)
\moveto(182.24038166,201.73804676)
\curveto(182.24038166,204.76804373)(183.41038514,207.73804676)(186.89038166,207.73804676)
\curveto(190.40037815,207.73804676)(191.78038166,204.9180437)(191.78038166,201.85804676)
\curveto(191.78038166,198.94804967)(190.46037824,196.00804676)(187.04038166,196.00804676)
\curveto(183.74038496,196.00804676)(182.24038166,198.82804967)(182.24038166,201.73804676)
}
}
{
\newrgbcolor{curcolor}{0 0 0}
\pscustom[linestyle=none,fillstyle=solid,fillcolor=curcolor]
{
\newpath
\moveto(211.03288166,209.62804676)
\lineto(208.48288166,209.62804676)
\lineto(208.48288166,200.86804676)
\curveto(208.48288166,198.07804955)(206.98287857,196.00804676)(203.89288166,196.00804676)
\curveto(201.94288361,196.00804676)(200.74288166,197.23804865)(200.74288166,199.12804676)
\lineto(200.74288166,209.62804676)
\lineto(198.19288166,209.62804676)
\lineto(198.19288166,199.42804676)
\curveto(198.19288166,196.09805009)(199.45288574,193.75804676)(203.53288166,193.75804676)
\curveto(205.75287944,193.75804676)(207.49288274,194.65804868)(208.57288166,196.57804676)
\lineto(208.63288166,196.57804676)
\lineto(208.63288166,194.11804676)
\lineto(211.03288166,194.11804676)
\lineto(211.03288166,209.62804676)
}
}
{
\newrgbcolor{curcolor}{0 0 0}
\pscustom[linestyle=none,fillstyle=solid,fillcolor=curcolor]
{
\newpath
\moveto(215.04210041,194.11804676)
\lineto(217.59210041,194.11804676)
\lineto(217.59210041,209.62804676)
\lineto(215.04210041,209.62804676)
\lineto(215.04210041,194.11804676)
\moveto(217.59210041,215.53804676)
\lineto(215.04210041,215.53804676)
\lineto(215.04210041,212.41804676)
\lineto(217.59210041,212.41804676)
\lineto(217.59210041,215.53804676)
}
}
{
\newrgbcolor{curcolor}{0 0 0}
\pscustom[linestyle=none,fillstyle=solid,fillcolor=curcolor]
{
\newpath
\moveto(233.66178791,202.00804676)
\curveto(233.66178791,198.97804979)(232.49178443,196.00804676)(229.01178791,196.00804676)
\curveto(225.50179142,196.00804676)(224.12178791,198.82804982)(224.12178791,201.88804676)
\curveto(224.12178791,204.79804385)(225.44179133,207.73804676)(228.86178791,207.73804676)
\curveto(232.16178461,207.73804676)(233.66178791,204.91804385)(233.66178791,202.00804676)
\moveto(221.66178791,188.17804676)
\lineto(224.21178791,188.17804676)
\lineto(224.21178791,196.18804676)
\lineto(224.27178791,196.18804676)
\curveto(225.41178677,194.35804859)(227.6917895,193.75804676)(229.28178791,193.75804676)
\curveto(234.02178317,193.75804676)(236.36178791,197.44805114)(236.36178791,201.82804676)
\curveto(236.36178791,206.20804238)(233.99178314,209.98804676)(229.22178791,209.98804676)
\curveto(227.09179004,209.98804676)(225.11178707,209.23804505)(224.27178791,207.52804676)
\lineto(224.21178791,207.52804676)
\lineto(224.21178791,209.62804676)
\lineto(221.66178791,209.62804676)
\lineto(221.66178791,188.17804676)
}
}
{
\newrgbcolor{curcolor}{0 0 0}
\pscustom[linestyle=none,fillstyle=solid,fillcolor=curcolor]
{
\newpath
\moveto(238.54428791,201.85804676)
\curveto(238.54428791,197.32805129)(241.15429283,193.75804676)(246.07428791,193.75804676)
\curveto(250.99428299,193.75804676)(253.60428791,197.32805129)(253.60428791,201.85804676)
\curveto(253.60428791,206.4180422)(250.99428299,209.98804676)(246.07428791,209.98804676)
\curveto(241.15429283,209.98804676)(238.54428791,206.4180422)(238.54428791,201.85804676)
\moveto(241.24428791,201.85804676)
\curveto(241.24428791,205.63804298)(243.40429058,207.73804676)(246.07428791,207.73804676)
\curveto(248.74428524,207.73804676)(250.90428791,205.63804298)(250.90428791,201.85804676)
\curveto(250.90428791,198.10805051)(248.74428524,196.00804676)(246.07428791,196.00804676)
\curveto(243.40429058,196.00804676)(241.24428791,198.10805051)(241.24428791,201.85804676)
}
}
{
\newrgbcolor{curcolor}{0 0 0}
\pscustom[linewidth=2.65748024,linecolor=curcolor]
{
\newpath
\moveto(149.79512897,227.43689442)
\lineto(266.26969257,227.43689442)
\lineto(266.26969257,176.27921009)
\lineto(149.79512897,176.27921009)
\closepath
}
}
{
\newrgbcolor{curcolor}{0 0 0}
\pscustom[linewidth=2.65748024,linecolor=curcolor]
{
\newpath
\moveto(121.62025562,284.20666981)
\lineto(276.46554676,284.20666981)
\lineto(276.46554676,164.87876606)
\lineto(121.62025562,164.87876606)
\closepath
}
}
{
\newrgbcolor{curcolor}{0 0 0}
\pscustom[linewidth=2.65748024,linecolor=curcolor]
{
\newpath
\moveto(71.49931446,337.21338368)
\lineto(289.23117176,337.21338368)
\lineto(289.23117176,152.97859669)
\lineto(71.49931446,152.97859669)
\closepath
}
}
{
\newrgbcolor{curcolor}{0 0 0}
\pscustom[linewidth=2.65748024,linecolor=curcolor]
{
\newpath
\moveto(21.24184147,392.00745106)
\lineto(301.73921696,392.00745106)
\lineto(301.73921696,139.45129872)
\lineto(21.24184147,139.45129872)
\closepath
}
}
{
\newrgbcolor{curcolor}{0 0 0}
\pscustom[linewidth=2.48031497,linecolor=curcolor,linestyle=dashed,dash=2.48031496 9.92125984]
{
\newpath
\moveto(318.73283,412.24300838)
\lineto(318.73283,22.04274838)
}
}
{
\newrgbcolor{curcolor}{0 0 0}
\pscustom[linestyle=none,fillstyle=solid,fillcolor=curcolor]
{
\newpath
\moveto(76.25002533,61.54572387)
\lineto(86.17002533,61.54572387)
\lineto(86.17002533,63.14572387)
\lineto(78.15002533,63.14572387)
\lineto(78.15002533,68.08572387)
\lineto(85.57002533,68.08572387)
\lineto(85.57002533,69.68572387)
\lineto(78.15002533,69.68572387)
\lineto(78.15002533,74.22572387)
\lineto(86.11002533,74.22572387)
\lineto(86.11002533,75.82572387)
\lineto(76.25002533,75.82572387)
\lineto(76.25002533,61.54572387)
}
}
{
\newrgbcolor{curcolor}{0 0 0}
\pscustom[linestyle=none,fillstyle=solid,fillcolor=curcolor]
{
\newpath
\moveto(87.53658783,64.80572387)
\curveto(87.63658773,62.24572643)(89.59659015,61.30572387)(91.91658783,61.30572387)
\curveto(94.01658573,61.30572387)(96.31658783,62.10572633)(96.31658783,64.56572387)
\curveto(96.31658783,66.56572187)(94.63658613,67.12572425)(92.93658783,67.50572387)
\curveto(91.35658941,67.88572349)(89.55658783,68.08572509)(89.55658783,69.30572387)
\curveto(89.55658783,70.34572283)(90.73658885,70.62572387)(91.75658783,70.62572387)
\curveto(92.87658671,70.62572387)(94.03658795,70.20572255)(94.15658783,68.88572387)
\lineto(95.85658783,68.88572387)
\curveto(95.71658797,71.40572135)(93.89658555,72.12572387)(91.61658783,72.12572387)
\curveto(89.81658963,72.12572387)(87.75658783,71.26572179)(87.75658783,69.18572387)
\curveto(87.75658783,67.20572585)(89.45658951,66.64572349)(91.13658783,66.26572387)
\curveto(92.83658613,65.88572425)(94.51658783,65.66572255)(94.51658783,64.34572387)
\curveto(94.51658783,63.04572517)(93.07658677,62.80572387)(92.01658783,62.80572387)
\curveto(90.61658923,62.80572387)(89.29658777,63.28572539)(89.23658783,64.80572387)
\lineto(87.53658783,64.80572387)
}
}
{
\newrgbcolor{curcolor}{0 0 0}
\pscustom[linestyle=none,fillstyle=solid,fillcolor=curcolor]
{
\newpath
\moveto(106.25658783,66.80572387)
\curveto(106.25658783,64.78572589)(105.47658551,62.80572387)(103.15658783,62.80572387)
\curveto(100.81659017,62.80572387)(99.89658783,64.68572591)(99.89658783,66.72572387)
\curveto(99.89658783,68.66572193)(100.77659011,70.62572387)(103.05658783,70.62572387)
\curveto(105.25658563,70.62572387)(106.25658783,68.74572193)(106.25658783,66.80572387)
\moveto(98.25658783,57.58572387)
\lineto(99.95658783,57.58572387)
\lineto(99.95658783,62.92572387)
\lineto(99.99658783,62.92572387)
\curveto(100.75658707,61.70572509)(102.27658889,61.30572387)(103.33658783,61.30572387)
\curveto(106.49658467,61.30572387)(108.05658783,63.76572679)(108.05658783,66.68572387)
\curveto(108.05658783,69.60572095)(106.47658465,72.12572387)(103.29658783,72.12572387)
\curveto(101.87658925,72.12572387)(100.55658727,71.62572273)(99.99658783,70.48572387)
\lineto(99.95658783,70.48572387)
\lineto(99.95658783,71.88572387)
\lineto(98.25658783,71.88572387)
\lineto(98.25658783,57.58572387)
}
}
{
\newrgbcolor{curcolor}{0 0 0}
\pscustom[linestyle=none,fillstyle=solid,fillcolor=curcolor]
{
\newpath
\moveto(116.47158783,65.14572387)
\curveto(116.47158783,64.20572481)(115.55158559,62.80572387)(113.31158783,62.80572387)
\curveto(112.27158887,62.80572387)(111.31158783,63.20572499)(111.31158783,64.32572387)
\curveto(111.31158783,65.58572261)(112.27158895,65.98572407)(113.39158783,66.18572387)
\curveto(114.53158669,66.38572367)(115.81158849,66.40572435)(116.47158783,66.88572387)
\lineto(116.47158783,65.14572387)
\moveto(119.23158783,62.90572387)
\curveto(119.01158805,62.82572395)(118.85158769,62.80572387)(118.71158783,62.80572387)
\curveto(118.17158837,62.80572387)(118.17158783,63.16572467)(118.17158783,63.96572387)
\lineto(118.17158783,69.28572387)
\curveto(118.17158783,71.70572145)(116.15158597,72.12572387)(114.29158783,72.12572387)
\curveto(111.99159013,72.12572387)(110.01158773,71.22572131)(109.91158783,68.66572387)
\lineto(111.61158783,68.66572387)
\curveto(111.69158775,70.18572235)(112.75158927,70.62572387)(114.19158783,70.62572387)
\curveto(115.27158675,70.62572387)(116.49158783,70.38572239)(116.49158783,68.90572387)
\curveto(116.49158783,67.62572515)(114.89158595,67.74572351)(113.01158783,67.38572387)
\curveto(111.25158959,67.04572421)(109.51158783,66.54572153)(109.51158783,64.20572387)
\curveto(109.51158783,62.14572593)(111.05158971,61.30572387)(112.93158783,61.30572387)
\curveto(114.37158639,61.30572387)(115.63158877,61.80572497)(116.57158783,62.90572387)
\curveto(116.57158783,61.78572499)(117.13158871,61.30572387)(118.01158783,61.30572387)
\curveto(118.55158729,61.30572387)(118.93158813,61.40572405)(119.23158783,61.58572387)
\lineto(119.23158783,62.90572387)
}
}
{
\newrgbcolor{curcolor}{0 0 0}
\pscustom[linestyle=none,fillstyle=solid,fillcolor=curcolor]
{
\newpath
\moveto(129.59377533,68.56572387)
\curveto(129.35377557,71.02572141)(127.47377299,72.12572387)(125.13377533,72.12572387)
\curveto(121.85377861,72.12572387)(120.25377533,69.68572077)(120.25377533,66.58572387)
\curveto(120.25377533,63.50572695)(121.93377849,61.30572387)(125.09377533,61.30572387)
\curveto(127.69377273,61.30572387)(129.27377571,62.80572639)(129.65377533,65.32572387)
\lineto(127.91377533,65.32572387)
\curveto(127.69377555,63.76572543)(126.71377369,62.80572387)(125.07377533,62.80572387)
\curveto(122.91377749,62.80572387)(122.05377533,64.68572577)(122.05377533,66.58572387)
\curveto(122.05377533,68.68572177)(122.81377779,70.62572387)(125.27377533,70.62572387)
\curveto(126.67377393,70.62572387)(127.57377559,69.86572257)(127.83377533,68.56572387)
\lineto(129.59377533,68.56572387)
}
}
{
\newrgbcolor{curcolor}{0 0 0}
\pscustom[linestyle=none,fillstyle=solid,fillcolor=curcolor]
{
\newpath
\moveto(131.65596283,61.54572387)
\lineto(133.35596283,61.54572387)
\lineto(133.35596283,71.88572387)
\lineto(131.65596283,71.88572387)
\lineto(131.65596283,61.54572387)
\moveto(133.35596283,75.82572387)
\lineto(131.65596283,75.82572387)
\lineto(131.65596283,73.74572387)
\lineto(133.35596283,73.74572387)
\lineto(133.35596283,75.82572387)
}
}
{
\newrgbcolor{curcolor}{0 0 0}
\pscustom[linestyle=none,fillstyle=solid,fillcolor=curcolor]
{
\newpath
\moveto(135.44908783,66.70572387)
\curveto(135.44908783,63.68572689)(137.18909111,61.30572387)(140.46908783,61.30572387)
\curveto(143.74908455,61.30572387)(145.48908783,63.68572689)(145.48908783,66.70572387)
\curveto(145.48908783,69.74572083)(143.74908455,72.12572387)(140.46908783,72.12572387)
\curveto(137.18909111,72.12572387)(135.44908783,69.74572083)(135.44908783,66.70572387)
\moveto(137.24908783,66.70572387)
\curveto(137.24908783,69.22572135)(138.68908961,70.62572387)(140.46908783,70.62572387)
\curveto(142.24908605,70.62572387)(143.68908783,69.22572135)(143.68908783,66.70572387)
\curveto(143.68908783,64.20572637)(142.24908605,62.80572387)(140.46908783,62.80572387)
\curveto(138.68908961,62.80572387)(137.24908783,64.20572637)(137.24908783,66.70572387)
}
}
{
\newrgbcolor{curcolor}{0 0 0}
\pscustom[linestyle=none,fillstyle=solid,fillcolor=curcolor]
{
\newpath
\moveto(146.83346283,64.80572387)
\curveto(146.93346273,62.24572643)(148.89346515,61.30572387)(151.21346283,61.30572387)
\curveto(153.31346073,61.30572387)(155.61346283,62.10572633)(155.61346283,64.56572387)
\curveto(155.61346283,66.56572187)(153.93346113,67.12572425)(152.23346283,67.50572387)
\curveto(150.65346441,67.88572349)(148.85346283,68.08572509)(148.85346283,69.30572387)
\curveto(148.85346283,70.34572283)(150.03346385,70.62572387)(151.05346283,70.62572387)
\curveto(152.17346171,70.62572387)(153.33346295,70.20572255)(153.45346283,68.88572387)
\lineto(155.15346283,68.88572387)
\curveto(155.01346297,71.40572135)(153.19346055,72.12572387)(150.91346283,72.12572387)
\curveto(149.11346463,72.12572387)(147.05346283,71.26572179)(147.05346283,69.18572387)
\curveto(147.05346283,67.20572585)(148.75346451,66.64572349)(150.43346283,66.26572387)
\curveto(152.13346113,65.88572425)(153.81346283,65.66572255)(153.81346283,64.34572387)
\curveto(153.81346283,63.04572517)(152.37346177,62.80572387)(151.31346283,62.80572387)
\curveto(149.91346423,62.80572387)(148.59346277,63.28572539)(148.53346283,64.80572387)
\lineto(146.83346283,64.80572387)
}
}
{
\newrgbcolor{curcolor}{0 0 0}
\pscustom[linestyle=none,fillstyle=solid,fillcolor=curcolor]
{
\newpath
\moveto(78.33002533,49.98572387)
\lineto(76.63002533,49.98572387)
\lineto(76.63002533,46.88572387)
\lineto(74.87002533,46.88572387)
\lineto(74.87002533,45.38572387)
\lineto(76.63002533,45.38572387)
\lineto(76.63002533,38.80572387)
\curveto(76.63002533,36.90572577)(77.33002709,36.54572387)(79.09002533,36.54572387)
\lineto(80.39002533,36.54572387)
\lineto(80.39002533,38.04572387)
\lineto(79.61002533,38.04572387)
\curveto(78.55002639,38.04572387)(78.33002533,38.18572465)(78.33002533,38.96572387)
\lineto(78.33002533,45.38572387)
\lineto(80.39002533,45.38572387)
\lineto(80.39002533,46.88572387)
\lineto(78.33002533,46.88572387)
\lineto(78.33002533,49.98572387)
}
}
{
\newrgbcolor{curcolor}{0 0 0}
\pscustom[linestyle=none,fillstyle=solid,fillcolor=curcolor]
{
\newpath
\moveto(89.39908783,39.82572387)
\curveto(89.09908813,38.48572521)(88.11908643,37.80572387)(86.71908783,37.80572387)
\curveto(84.45909009,37.80572387)(83.43908789,39.40572567)(83.49908783,41.20572387)
\lineto(91.23908783,41.20572387)
\curveto(91.33908773,43.70572137)(90.21908417,47.12572387)(86.55908783,47.12572387)
\curveto(83.73909065,47.12572387)(81.69908783,44.84572077)(81.69908783,41.74572387)
\curveto(81.79908773,38.58572703)(83.35909113,36.30572387)(86.65908783,36.30572387)
\curveto(88.97908551,36.30572387)(90.61908829,37.54572615)(91.07908783,39.82572387)
\lineto(89.39908783,39.82572387)
\moveto(83.49908783,42.70572387)
\curveto(83.61908771,44.28572229)(84.67908961,45.62572387)(86.45908783,45.62572387)
\curveto(88.13908615,45.62572387)(89.35908791,44.32572225)(89.43908783,42.70572387)
\lineto(83.49908783,42.70572387)
}
}
{
\newrgbcolor{curcolor}{0 0 0}
\pscustom[linestyle=none,fillstyle=solid,fillcolor=curcolor]
{
\newpath
\moveto(93.00127533,36.54572387)
\lineto(94.70127533,36.54572387)
\lineto(94.70127533,42.98572387)
\curveto(94.70127533,43.76572309)(95.44127735,45.62572387)(97.46127533,45.62572387)
\curveto(98.98127381,45.62572387)(99.40127533,44.66572253)(99.40127533,43.32572387)
\lineto(99.40127533,36.54572387)
\lineto(101.10127533,36.54572387)
\lineto(101.10127533,42.98572387)
\curveto(101.10127533,44.58572227)(102.16127697,45.62572387)(103.80127533,45.62572387)
\curveto(105.46127367,45.62572387)(105.80127533,44.60572259)(105.80127533,43.32572387)
\lineto(105.80127533,36.54572387)
\lineto(107.50127533,36.54572387)
\lineto(107.50127533,44.12572387)
\curveto(107.50127533,46.26572173)(106.12127327,47.12572387)(104.06127533,47.12572387)
\curveto(102.74127665,47.12572387)(101.52127463,46.46572277)(100.82127533,45.36572387)
\curveto(100.40127575,46.62572261)(99.24127407,47.12572387)(97.98127533,47.12572387)
\curveto(96.56127675,47.12572387)(95.40127457,46.52572271)(94.64127533,45.36572387)
\lineto(94.60127533,45.36572387)
\lineto(94.60127533,46.88572387)
\lineto(93.00127533,46.88572387)
\lineto(93.00127533,36.54572387)
}
}
{
\newrgbcolor{curcolor}{0 0 0}
\pscustom[linestyle=none,fillstyle=solid,fillcolor=curcolor]
{
\newpath
\moveto(118.13158783,41.80572387)
\curveto(118.13158783,39.78572589)(117.35158551,37.80572387)(115.03158783,37.80572387)
\curveto(112.69159017,37.80572387)(111.77158783,39.68572591)(111.77158783,41.72572387)
\curveto(111.77158783,43.66572193)(112.65159011,45.62572387)(114.93158783,45.62572387)
\curveto(117.13158563,45.62572387)(118.13158783,43.74572193)(118.13158783,41.80572387)
\moveto(110.13158783,32.58572387)
\lineto(111.83158783,32.58572387)
\lineto(111.83158783,37.92572387)
\lineto(111.87158783,37.92572387)
\curveto(112.63158707,36.70572509)(114.15158889,36.30572387)(115.21158783,36.30572387)
\curveto(118.37158467,36.30572387)(119.93158783,38.76572679)(119.93158783,41.68572387)
\curveto(119.93158783,44.60572095)(118.35158465,47.12572387)(115.17158783,47.12572387)
\curveto(113.75158925,47.12572387)(112.43158727,46.62572273)(111.87158783,45.48572387)
\lineto(111.83158783,45.48572387)
\lineto(111.83158783,46.88572387)
\lineto(110.13158783,46.88572387)
\lineto(110.13158783,32.58572387)
}
}
{
\newrgbcolor{curcolor}{0 0 0}
\pscustom[linestyle=none,fillstyle=solid,fillcolor=curcolor]
{
\newpath
\moveto(121.38658783,41.70572387)
\curveto(121.38658783,38.68572689)(123.12659111,36.30572387)(126.40658783,36.30572387)
\curveto(129.68658455,36.30572387)(131.42658783,38.68572689)(131.42658783,41.70572387)
\curveto(131.42658783,44.74572083)(129.68658455,47.12572387)(126.40658783,47.12572387)
\curveto(123.12659111,47.12572387)(121.38658783,44.74572083)(121.38658783,41.70572387)
\moveto(123.18658783,41.70572387)
\curveto(123.18658783,44.22572135)(124.62658961,45.62572387)(126.40658783,45.62572387)
\curveto(128.18658605,45.62572387)(129.62658783,44.22572135)(129.62658783,41.70572387)
\curveto(129.62658783,39.20572637)(128.18658605,37.80572387)(126.40658783,37.80572387)
\curveto(124.62658961,37.80572387)(123.18658783,39.20572637)(123.18658783,41.70572387)
}
}
{
\newrgbcolor{curcolor}{0 0 0}
\pscustom[linestyle=none,fillstyle=solid,fillcolor=curcolor]
{
\newpath
\moveto(133.37096283,36.54572387)
\lineto(135.07096283,36.54572387)
\lineto(135.07096283,41.14572387)
\curveto(135.07096283,43.76572125)(136.07096557,45.32572387)(138.81096283,45.32572387)
\lineto(138.81096283,47.12572387)
\curveto(136.97096467,47.18572381)(135.83096201,46.36572221)(135.01096283,44.70572387)
\lineto(134.97096283,44.70572387)
\lineto(134.97096283,46.88572387)
\lineto(133.37096283,46.88572387)
\lineto(133.37096283,36.54572387)
}
}
{
\newrgbcolor{curcolor}{0 0 0}
\pscustom[linestyle=none,fillstyle=solid,fillcolor=curcolor]
{
\newpath
\moveto(146.51065033,40.14572387)
\curveto(146.51065033,39.20572481)(145.59064809,37.80572387)(143.35065033,37.80572387)
\curveto(142.31065137,37.80572387)(141.35065033,38.20572499)(141.35065033,39.32572387)
\curveto(141.35065033,40.58572261)(142.31065145,40.98572407)(143.43065033,41.18572387)
\curveto(144.57064919,41.38572367)(145.85065099,41.40572435)(146.51065033,41.88572387)
\lineto(146.51065033,40.14572387)
\moveto(149.27065033,37.90572387)
\curveto(149.05065055,37.82572395)(148.89065019,37.80572387)(148.75065033,37.80572387)
\curveto(148.21065087,37.80572387)(148.21065033,38.16572467)(148.21065033,38.96572387)
\lineto(148.21065033,44.28572387)
\curveto(148.21065033,46.70572145)(146.19064847,47.12572387)(144.33065033,47.12572387)
\curveto(142.03065263,47.12572387)(140.05065023,46.22572131)(139.95065033,43.66572387)
\lineto(141.65065033,43.66572387)
\curveto(141.73065025,45.18572235)(142.79065177,45.62572387)(144.23065033,45.62572387)
\curveto(145.31064925,45.62572387)(146.53065033,45.38572239)(146.53065033,43.90572387)
\curveto(146.53065033,42.62572515)(144.93064845,42.74572351)(143.05065033,42.38572387)
\curveto(141.29065209,42.04572421)(139.55065033,41.54572153)(139.55065033,39.20572387)
\curveto(139.55065033,37.14572593)(141.09065221,36.30572387)(142.97065033,36.30572387)
\curveto(144.41064889,36.30572387)(145.67065127,36.80572497)(146.61065033,37.90572387)
\curveto(146.61065033,36.78572499)(147.17065121,36.30572387)(148.05065033,36.30572387)
\curveto(148.59064979,36.30572387)(148.97065063,36.40572405)(149.27065033,36.58572387)
\lineto(149.27065033,37.90572387)
}
}
{
\newrgbcolor{curcolor}{0 0 0}
\pscustom[linestyle=none,fillstyle=solid,fillcolor=curcolor]
{
\newpath
\moveto(150.95283783,36.54572387)
\lineto(152.65283783,36.54572387)
\lineto(152.65283783,50.82572387)
\lineto(150.95283783,50.82572387)
\lineto(150.95283783,36.54572387)
}
}
{
\newrgbcolor{curcolor}{0 0 0}
\pscustom[linestyle=none,fillstyle=solid,fillcolor=curcolor]
{
\newpath
\moveto(162.44596283,39.82572387)
\curveto(162.14596313,38.48572521)(161.16596143,37.80572387)(159.76596283,37.80572387)
\curveto(157.50596509,37.80572387)(156.48596289,39.40572567)(156.54596283,41.20572387)
\lineto(164.28596283,41.20572387)
\curveto(164.38596273,43.70572137)(163.26595917,47.12572387)(159.60596283,47.12572387)
\curveto(156.78596565,47.12572387)(154.74596283,44.84572077)(154.74596283,41.74572387)
\curveto(154.84596273,38.58572703)(156.40596613,36.30572387)(159.70596283,36.30572387)
\curveto(162.02596051,36.30572387)(163.66596329,37.54572615)(164.12596283,39.82572387)
\lineto(162.44596283,39.82572387)
\moveto(156.54596283,42.70572387)
\curveto(156.66596271,44.28572229)(157.72596461,45.62572387)(159.50596283,45.62572387)
\curveto(161.18596115,45.62572387)(162.40596291,44.32572225)(162.48596283,42.70572387)
\lineto(156.54596283,42.70572387)
}
}
{
\newrgbcolor{curcolor}{0 0 0}
\pscustom[linestyle=none,fillstyle=solid,fillcolor=curcolor]
{
\newpath
\moveto(165.38815033,39.80572387)
\curveto(165.48815023,37.24572643)(167.44815265,36.30572387)(169.76815033,36.30572387)
\curveto(171.86814823,36.30572387)(174.16815033,37.10572633)(174.16815033,39.56572387)
\curveto(174.16815033,41.56572187)(172.48814863,42.12572425)(170.78815033,42.50572387)
\curveto(169.20815191,42.88572349)(167.40815033,43.08572509)(167.40815033,44.30572387)
\curveto(167.40815033,45.34572283)(168.58815135,45.62572387)(169.60815033,45.62572387)
\curveto(170.72814921,45.62572387)(171.88815045,45.20572255)(172.00815033,43.88572387)
\lineto(173.70815033,43.88572387)
\curveto(173.56815047,46.40572135)(171.74814805,47.12572387)(169.46815033,47.12572387)
\curveto(167.66815213,47.12572387)(165.60815033,46.26572179)(165.60815033,44.18572387)
\curveto(165.60815033,42.20572585)(167.30815201,41.64572349)(168.98815033,41.26572387)
\curveto(170.68814863,40.88572425)(172.36815033,40.66572255)(172.36815033,39.34572387)
\curveto(172.36815033,38.04572517)(170.92814927,37.80572387)(169.86815033,37.80572387)
\curveto(168.46815173,37.80572387)(167.14815027,38.28572539)(167.08815033,39.80572387)
\lineto(165.38815033,39.80572387)
}
}
{
\newrgbcolor{curcolor}{0 0 0}
\pscustom[linestyle=none,fillstyle=solid,fillcolor=curcolor]
{
\newpath
\moveto(382.96151978,61.54572387)
\lineto(392.88151978,61.54572387)
\lineto(392.88151978,63.14572387)
\lineto(384.86151978,63.14572387)
\lineto(384.86151978,68.08572387)
\lineto(392.28151978,68.08572387)
\lineto(392.28151978,69.68572387)
\lineto(384.86151978,69.68572387)
\lineto(384.86151978,74.22572387)
\lineto(392.82151978,74.22572387)
\lineto(392.82151978,75.82572387)
\lineto(382.96151978,75.82572387)
\lineto(382.96151978,61.54572387)
}
}
{
\newrgbcolor{curcolor}{0 0 0}
\pscustom[linestyle=none,fillstyle=solid,fillcolor=curcolor]
{
\newpath
\moveto(394.24808228,64.80572387)
\curveto(394.34808218,62.24572643)(396.3080846,61.30572387)(398.62808228,61.30572387)
\curveto(400.72808018,61.30572387)(403.02808228,62.10572633)(403.02808228,64.56572387)
\curveto(403.02808228,66.56572187)(401.34808058,67.12572425)(399.64808228,67.50572387)
\curveto(398.06808386,67.88572349)(396.26808228,68.08572509)(396.26808228,69.30572387)
\curveto(396.26808228,70.34572283)(397.4480833,70.62572387)(398.46808228,70.62572387)
\curveto(399.58808116,70.62572387)(400.7480824,70.20572255)(400.86808228,68.88572387)
\lineto(402.56808228,68.88572387)
\curveto(402.42808242,71.40572135)(400.60808,72.12572387)(398.32808228,72.12572387)
\curveto(396.52808408,72.12572387)(394.46808228,71.26572179)(394.46808228,69.18572387)
\curveto(394.46808228,67.20572585)(396.16808396,66.64572349)(397.84808228,66.26572387)
\curveto(399.54808058,65.88572425)(401.22808228,65.66572255)(401.22808228,64.34572387)
\curveto(401.22808228,63.04572517)(399.78808122,62.80572387)(398.72808228,62.80572387)
\curveto(397.32808368,62.80572387)(396.00808222,63.28572539)(395.94808228,64.80572387)
\lineto(394.24808228,64.80572387)
}
}
{
\newrgbcolor{curcolor}{0 0 0}
\pscustom[linestyle=none,fillstyle=solid,fillcolor=curcolor]
{
\newpath
\moveto(412.96808228,66.80572387)
\curveto(412.96808228,64.78572589)(412.18807996,62.80572387)(409.86808228,62.80572387)
\curveto(407.52808462,62.80572387)(406.60808228,64.68572591)(406.60808228,66.72572387)
\curveto(406.60808228,68.66572193)(407.48808456,70.62572387)(409.76808228,70.62572387)
\curveto(411.96808008,70.62572387)(412.96808228,68.74572193)(412.96808228,66.80572387)
\moveto(404.96808228,57.58572387)
\lineto(406.66808228,57.58572387)
\lineto(406.66808228,62.92572387)
\lineto(406.70808228,62.92572387)
\curveto(407.46808152,61.70572509)(408.98808334,61.30572387)(410.04808228,61.30572387)
\curveto(413.20807912,61.30572387)(414.76808228,63.76572679)(414.76808228,66.68572387)
\curveto(414.76808228,69.60572095)(413.1880791,72.12572387)(410.00808228,72.12572387)
\curveto(408.5880837,72.12572387)(407.26808172,71.62572273)(406.70808228,70.48572387)
\lineto(406.66808228,70.48572387)
\lineto(406.66808228,71.88572387)
\lineto(404.96808228,71.88572387)
\lineto(404.96808228,57.58572387)
}
}
{
\newrgbcolor{curcolor}{0 0 0}
\pscustom[linestyle=none,fillstyle=solid,fillcolor=curcolor]
{
\newpath
\moveto(423.18308228,65.14572387)
\curveto(423.18308228,64.20572481)(422.26308004,62.80572387)(420.02308228,62.80572387)
\curveto(418.98308332,62.80572387)(418.02308228,63.20572499)(418.02308228,64.32572387)
\curveto(418.02308228,65.58572261)(418.9830834,65.98572407)(420.10308228,66.18572387)
\curveto(421.24308114,66.38572367)(422.52308294,66.40572435)(423.18308228,66.88572387)
\lineto(423.18308228,65.14572387)
\moveto(425.94308228,62.90572387)
\curveto(425.7230825,62.82572395)(425.56308214,62.80572387)(425.42308228,62.80572387)
\curveto(424.88308282,62.80572387)(424.88308228,63.16572467)(424.88308228,63.96572387)
\lineto(424.88308228,69.28572387)
\curveto(424.88308228,71.70572145)(422.86308042,72.12572387)(421.00308228,72.12572387)
\curveto(418.70308458,72.12572387)(416.72308218,71.22572131)(416.62308228,68.66572387)
\lineto(418.32308228,68.66572387)
\curveto(418.4030822,70.18572235)(419.46308372,70.62572387)(420.90308228,70.62572387)
\curveto(421.9830812,70.62572387)(423.20308228,70.38572239)(423.20308228,68.90572387)
\curveto(423.20308228,67.62572515)(421.6030804,67.74572351)(419.72308228,67.38572387)
\curveto(417.96308404,67.04572421)(416.22308228,66.54572153)(416.22308228,64.20572387)
\curveto(416.22308228,62.14572593)(417.76308416,61.30572387)(419.64308228,61.30572387)
\curveto(421.08308084,61.30572387)(422.34308322,61.80572497)(423.28308228,62.90572387)
\curveto(423.28308228,61.78572499)(423.84308316,61.30572387)(424.72308228,61.30572387)
\curveto(425.26308174,61.30572387)(425.64308258,61.40572405)(425.94308228,61.58572387)
\lineto(425.94308228,62.90572387)
}
}
{
\newrgbcolor{curcolor}{0 0 0}
\pscustom[linestyle=none,fillstyle=solid,fillcolor=curcolor]
{
\newpath
\moveto(436.30526978,68.56572387)
\curveto(436.06527002,71.02572141)(434.18526744,72.12572387)(431.84526978,72.12572387)
\curveto(428.56527306,72.12572387)(426.96526978,69.68572077)(426.96526978,66.58572387)
\curveto(426.96526978,63.50572695)(428.64527294,61.30572387)(431.80526978,61.30572387)
\curveto(434.40526718,61.30572387)(435.98527016,62.80572639)(436.36526978,65.32572387)
\lineto(434.62526978,65.32572387)
\curveto(434.40527,63.76572543)(433.42526814,62.80572387)(431.78526978,62.80572387)
\curveto(429.62527194,62.80572387)(428.76526978,64.68572577)(428.76526978,66.58572387)
\curveto(428.76526978,68.68572177)(429.52527224,70.62572387)(431.98526978,70.62572387)
\curveto(433.38526838,70.62572387)(434.28527004,69.86572257)(434.54526978,68.56572387)
\lineto(436.30526978,68.56572387)
}
}
{
\newrgbcolor{curcolor}{0 0 0}
\pscustom[linestyle=none,fillstyle=solid,fillcolor=curcolor]
{
\newpath
\moveto(438.36745728,61.54572387)
\lineto(440.06745728,61.54572387)
\lineto(440.06745728,71.88572387)
\lineto(438.36745728,71.88572387)
\lineto(438.36745728,61.54572387)
\moveto(440.06745728,75.82572387)
\lineto(438.36745728,75.82572387)
\lineto(438.36745728,73.74572387)
\lineto(440.06745728,73.74572387)
\lineto(440.06745728,75.82572387)
}
}
{
\newrgbcolor{curcolor}{0 0 0}
\pscustom[linestyle=none,fillstyle=solid,fillcolor=curcolor]
{
\newpath
\moveto(442.16058228,66.70572387)
\curveto(442.16058228,63.68572689)(443.90058556,61.30572387)(447.18058228,61.30572387)
\curveto(450.460579,61.30572387)(452.20058228,63.68572689)(452.20058228,66.70572387)
\curveto(452.20058228,69.74572083)(450.460579,72.12572387)(447.18058228,72.12572387)
\curveto(443.90058556,72.12572387)(442.16058228,69.74572083)(442.16058228,66.70572387)
\moveto(443.96058228,66.70572387)
\curveto(443.96058228,69.22572135)(445.40058406,70.62572387)(447.18058228,70.62572387)
\curveto(448.9605805,70.62572387)(450.40058228,69.22572135)(450.40058228,66.70572387)
\curveto(450.40058228,64.20572637)(448.9605805,62.80572387)(447.18058228,62.80572387)
\curveto(445.40058406,62.80572387)(443.96058228,64.20572637)(443.96058228,66.70572387)
}
}
{
\newrgbcolor{curcolor}{0 0 0}
\pscustom[linestyle=none,fillstyle=solid,fillcolor=curcolor]
{
\newpath
\moveto(453.54495728,64.80572387)
\curveto(453.64495718,62.24572643)(455.6049596,61.30572387)(457.92495728,61.30572387)
\curveto(460.02495518,61.30572387)(462.32495728,62.10572633)(462.32495728,64.56572387)
\curveto(462.32495728,66.56572187)(460.64495558,67.12572425)(458.94495728,67.50572387)
\curveto(457.36495886,67.88572349)(455.56495728,68.08572509)(455.56495728,69.30572387)
\curveto(455.56495728,70.34572283)(456.7449583,70.62572387)(457.76495728,70.62572387)
\curveto(458.88495616,70.62572387)(460.0449574,70.20572255)(460.16495728,68.88572387)
\lineto(461.86495728,68.88572387)
\curveto(461.72495742,71.40572135)(459.904955,72.12572387)(457.62495728,72.12572387)
\curveto(455.82495908,72.12572387)(453.76495728,71.26572179)(453.76495728,69.18572387)
\curveto(453.76495728,67.20572585)(455.46495896,66.64572349)(457.14495728,66.26572387)
\curveto(458.84495558,65.88572425)(460.52495728,65.66572255)(460.52495728,64.34572387)
\curveto(460.52495728,63.04572517)(459.08495622,62.80572387)(458.02495728,62.80572387)
\curveto(456.62495868,62.80572387)(455.30495722,63.28572539)(455.24495728,64.80572387)
\lineto(453.54495728,64.80572387)
}
}
{
\newrgbcolor{curcolor}{0 0 0}
\pscustom[linestyle=none,fillstyle=solid,fillcolor=curcolor]
{
\newpath
\moveto(389.08151978,40.14572387)
\curveto(389.08151978,39.20572481)(388.16151754,37.80572387)(385.92151978,37.80572387)
\curveto(384.88152082,37.80572387)(383.92151978,38.20572499)(383.92151978,39.32572387)
\curveto(383.92151978,40.58572261)(384.8815209,40.98572407)(386.00151978,41.18572387)
\curveto(387.14151864,41.38572367)(388.42152044,41.40572435)(389.08151978,41.88572387)
\lineto(389.08151978,40.14572387)
\moveto(391.84151978,37.90572387)
\curveto(391.62152,37.82572395)(391.46151964,37.80572387)(391.32151978,37.80572387)
\curveto(390.78152032,37.80572387)(390.78151978,38.16572467)(390.78151978,38.96572387)
\lineto(390.78151978,44.28572387)
\curveto(390.78151978,46.70572145)(388.76151792,47.12572387)(386.90151978,47.12572387)
\curveto(384.60152208,47.12572387)(382.62151968,46.22572131)(382.52151978,43.66572387)
\lineto(384.22151978,43.66572387)
\curveto(384.3015197,45.18572235)(385.36152122,45.62572387)(386.80151978,45.62572387)
\curveto(387.8815187,45.62572387)(389.10151978,45.38572239)(389.10151978,43.90572387)
\curveto(389.10151978,42.62572515)(387.5015179,42.74572351)(385.62151978,42.38572387)
\curveto(383.86152154,42.04572421)(382.12151978,41.54572153)(382.12151978,39.20572387)
\curveto(382.12151978,37.14572593)(383.66152166,36.30572387)(385.54151978,36.30572387)
\curveto(386.98151834,36.30572387)(388.24152072,36.80572497)(389.18151978,37.90572387)
\curveto(389.18151978,36.78572499)(389.74152066,36.30572387)(390.62151978,36.30572387)
\curveto(391.16151924,36.30572387)(391.54152008,36.40572405)(391.84151978,36.58572387)
\lineto(391.84151978,37.90572387)
}
}
{
\newrgbcolor{curcolor}{0 0 0}
\pscustom[linestyle=none,fillstyle=solid,fillcolor=curcolor]
{
\newpath
\moveto(395.78370728,49.98572387)
\lineto(394.08370728,49.98572387)
\lineto(394.08370728,46.88572387)
\lineto(392.32370728,46.88572387)
\lineto(392.32370728,45.38572387)
\lineto(394.08370728,45.38572387)
\lineto(394.08370728,38.80572387)
\curveto(394.08370728,36.90572577)(394.78370904,36.54572387)(396.54370728,36.54572387)
\lineto(397.84370728,36.54572387)
\lineto(397.84370728,38.04572387)
\lineto(397.06370728,38.04572387)
\curveto(396.00370834,38.04572387)(395.78370728,38.18572465)(395.78370728,38.96572387)
\lineto(395.78370728,45.38572387)
\lineto(397.84370728,45.38572387)
\lineto(397.84370728,46.88572387)
\lineto(395.78370728,46.88572387)
\lineto(395.78370728,49.98572387)
}
}
{
\newrgbcolor{curcolor}{0 0 0}
\pscustom[linestyle=none,fillstyle=solid,fillcolor=curcolor]
{
\newpath
\moveto(406.85276978,39.82572387)
\curveto(406.55277008,38.48572521)(405.57276838,37.80572387)(404.17276978,37.80572387)
\curveto(401.91277204,37.80572387)(400.89276984,39.40572567)(400.95276978,41.20572387)
\lineto(408.69276978,41.20572387)
\curveto(408.79276968,43.70572137)(407.67276612,47.12572387)(404.01276978,47.12572387)
\curveto(401.1927726,47.12572387)(399.15276978,44.84572077)(399.15276978,41.74572387)
\curveto(399.25276968,38.58572703)(400.81277308,36.30572387)(404.11276978,36.30572387)
\curveto(406.43276746,36.30572387)(408.07277024,37.54572615)(408.53276978,39.82572387)
\lineto(406.85276978,39.82572387)
\moveto(400.95276978,42.70572387)
\curveto(401.07276966,44.28572229)(402.13277156,45.62572387)(403.91276978,45.62572387)
\curveto(405.5927681,45.62572387)(406.81276986,44.32572225)(406.89276978,42.70572387)
\lineto(400.95276978,42.70572387)
}
}
{
\newrgbcolor{curcolor}{0 0 0}
\pscustom[linestyle=none,fillstyle=solid,fillcolor=curcolor]
{
\newpath
\moveto(410.45495728,36.54572387)
\lineto(412.15495728,36.54572387)
\lineto(412.15495728,42.98572387)
\curveto(412.15495728,43.76572309)(412.8949593,45.62572387)(414.91495728,45.62572387)
\curveto(416.43495576,45.62572387)(416.85495728,44.66572253)(416.85495728,43.32572387)
\lineto(416.85495728,36.54572387)
\lineto(418.55495728,36.54572387)
\lineto(418.55495728,42.98572387)
\curveto(418.55495728,44.58572227)(419.61495892,45.62572387)(421.25495728,45.62572387)
\curveto(422.91495562,45.62572387)(423.25495728,44.60572259)(423.25495728,43.32572387)
\lineto(423.25495728,36.54572387)
\lineto(424.95495728,36.54572387)
\lineto(424.95495728,44.12572387)
\curveto(424.95495728,46.26572173)(423.57495522,47.12572387)(421.51495728,47.12572387)
\curveto(420.1949586,47.12572387)(418.97495658,46.46572277)(418.27495728,45.36572387)
\curveto(417.8549577,46.62572261)(416.69495602,47.12572387)(415.43495728,47.12572387)
\curveto(414.0149587,47.12572387)(412.85495652,46.52572271)(412.09495728,45.36572387)
\lineto(412.05495728,45.36572387)
\lineto(412.05495728,46.88572387)
\lineto(410.45495728,46.88572387)
\lineto(410.45495728,36.54572387)
}
}
{
\newrgbcolor{curcolor}{0 0 0}
\pscustom[linestyle=none,fillstyle=solid,fillcolor=curcolor]
{
\newpath
\moveto(435.58526978,41.80572387)
\curveto(435.58526978,39.78572589)(434.80526746,37.80572387)(432.48526978,37.80572387)
\curveto(430.14527212,37.80572387)(429.22526978,39.68572591)(429.22526978,41.72572387)
\curveto(429.22526978,43.66572193)(430.10527206,45.62572387)(432.38526978,45.62572387)
\curveto(434.58526758,45.62572387)(435.58526978,43.74572193)(435.58526978,41.80572387)
\moveto(427.58526978,32.58572387)
\lineto(429.28526978,32.58572387)
\lineto(429.28526978,37.92572387)
\lineto(429.32526978,37.92572387)
\curveto(430.08526902,36.70572509)(431.60527084,36.30572387)(432.66526978,36.30572387)
\curveto(435.82526662,36.30572387)(437.38526978,38.76572679)(437.38526978,41.68572387)
\curveto(437.38526978,44.60572095)(435.8052666,47.12572387)(432.62526978,47.12572387)
\curveto(431.2052712,47.12572387)(429.88526922,46.62572273)(429.32526978,45.48572387)
\lineto(429.28526978,45.48572387)
\lineto(429.28526978,46.88572387)
\lineto(427.58526978,46.88572387)
\lineto(427.58526978,32.58572387)
}
}
{
\newrgbcolor{curcolor}{0 0 0}
\pscustom[linestyle=none,fillstyle=solid,fillcolor=curcolor]
{
\newpath
\moveto(438.84026978,41.70572387)
\curveto(438.84026978,38.68572689)(440.58027306,36.30572387)(443.86026978,36.30572387)
\curveto(447.1402665,36.30572387)(448.88026978,38.68572689)(448.88026978,41.70572387)
\curveto(448.88026978,44.74572083)(447.1402665,47.12572387)(443.86026978,47.12572387)
\curveto(440.58027306,47.12572387)(438.84026978,44.74572083)(438.84026978,41.70572387)
\moveto(440.64026978,41.70572387)
\curveto(440.64026978,44.22572135)(442.08027156,45.62572387)(443.86026978,45.62572387)
\curveto(445.640268,45.62572387)(447.08026978,44.22572135)(447.08026978,41.70572387)
\curveto(447.08026978,39.20572637)(445.640268,37.80572387)(443.86026978,37.80572387)
\curveto(442.08027156,37.80572387)(440.64026978,39.20572637)(440.64026978,41.70572387)
}
}
{
\newrgbcolor{curcolor}{0 0 0}
\pscustom[linestyle=none,fillstyle=solid,fillcolor=curcolor]
{
\newpath
\moveto(450.82464478,36.54572387)
\lineto(452.52464478,36.54572387)
\lineto(452.52464478,41.14572387)
\curveto(452.52464478,43.76572125)(453.52464752,45.32572387)(456.26464478,45.32572387)
\lineto(456.26464478,47.12572387)
\curveto(454.42464662,47.18572381)(453.28464396,46.36572221)(452.46464478,44.70572387)
\lineto(452.42464478,44.70572387)
\lineto(452.42464478,46.88572387)
\lineto(450.82464478,46.88572387)
\lineto(450.82464478,36.54572387)
}
}
{
\newrgbcolor{curcolor}{0 0 0}
\pscustom[linestyle=none,fillstyle=solid,fillcolor=curcolor]
{
\newpath
\moveto(463.96433228,40.14572387)
\curveto(463.96433228,39.20572481)(463.04433004,37.80572387)(460.80433228,37.80572387)
\curveto(459.76433332,37.80572387)(458.80433228,38.20572499)(458.80433228,39.32572387)
\curveto(458.80433228,40.58572261)(459.7643334,40.98572407)(460.88433228,41.18572387)
\curveto(462.02433114,41.38572367)(463.30433294,41.40572435)(463.96433228,41.88572387)
\lineto(463.96433228,40.14572387)
\moveto(466.72433228,37.90572387)
\curveto(466.5043325,37.82572395)(466.34433214,37.80572387)(466.20433228,37.80572387)
\curveto(465.66433282,37.80572387)(465.66433228,38.16572467)(465.66433228,38.96572387)
\lineto(465.66433228,44.28572387)
\curveto(465.66433228,46.70572145)(463.64433042,47.12572387)(461.78433228,47.12572387)
\curveto(459.48433458,47.12572387)(457.50433218,46.22572131)(457.40433228,43.66572387)
\lineto(459.10433228,43.66572387)
\curveto(459.1843322,45.18572235)(460.24433372,45.62572387)(461.68433228,45.62572387)
\curveto(462.7643312,45.62572387)(463.98433228,45.38572239)(463.98433228,43.90572387)
\curveto(463.98433228,42.62572515)(462.3843304,42.74572351)(460.50433228,42.38572387)
\curveto(458.74433404,42.04572421)(457.00433228,41.54572153)(457.00433228,39.20572387)
\curveto(457.00433228,37.14572593)(458.54433416,36.30572387)(460.42433228,36.30572387)
\curveto(461.86433084,36.30572387)(463.12433322,36.80572497)(464.06433228,37.90572387)
\curveto(464.06433228,36.78572499)(464.62433316,36.30572387)(465.50433228,36.30572387)
\curveto(466.04433174,36.30572387)(466.42433258,36.40572405)(466.72433228,36.58572387)
\lineto(466.72433228,37.90572387)
}
}
{
\newrgbcolor{curcolor}{0 0 0}
\pscustom[linestyle=none,fillstyle=solid,fillcolor=curcolor]
{
\newpath
\moveto(468.40651978,36.54572387)
\lineto(470.10651978,36.54572387)
\lineto(470.10651978,50.82572387)
\lineto(468.40651978,50.82572387)
\lineto(468.40651978,36.54572387)
}
}
{
\newrgbcolor{curcolor}{0 0 0}
\pscustom[linestyle=none,fillstyle=solid,fillcolor=curcolor]
{
\newpath
\moveto(479.89964478,39.82572387)
\curveto(479.59964508,38.48572521)(478.61964338,37.80572387)(477.21964478,37.80572387)
\curveto(474.95964704,37.80572387)(473.93964484,39.40572567)(473.99964478,41.20572387)
\lineto(481.73964478,41.20572387)
\curveto(481.83964468,43.70572137)(480.71964112,47.12572387)(477.05964478,47.12572387)
\curveto(474.2396476,47.12572387)(472.19964478,44.84572077)(472.19964478,41.74572387)
\curveto(472.29964468,38.58572703)(473.85964808,36.30572387)(477.15964478,36.30572387)
\curveto(479.47964246,36.30572387)(481.11964524,37.54572615)(481.57964478,39.82572387)
\lineto(479.89964478,39.82572387)
\moveto(473.99964478,42.70572387)
\curveto(474.11964466,44.28572229)(475.17964656,45.62572387)(476.95964478,45.62572387)
\curveto(478.6396431,45.62572387)(479.85964486,44.32572225)(479.93964478,42.70572387)
\lineto(473.99964478,42.70572387)
}
}
{
\newrgbcolor{curcolor}{0 0 0}
\pscustom[linestyle=none,fillstyle=solid,fillcolor=curcolor]
{
\newpath
\moveto(482.84183228,39.80572387)
\curveto(482.94183218,37.24572643)(484.9018346,36.30572387)(487.22183228,36.30572387)
\curveto(489.32183018,36.30572387)(491.62183228,37.10572633)(491.62183228,39.56572387)
\curveto(491.62183228,41.56572187)(489.94183058,42.12572425)(488.24183228,42.50572387)
\curveto(486.66183386,42.88572349)(484.86183228,43.08572509)(484.86183228,44.30572387)
\curveto(484.86183228,45.34572283)(486.0418333,45.62572387)(487.06183228,45.62572387)
\curveto(488.18183116,45.62572387)(489.3418324,45.20572255)(489.46183228,43.88572387)
\lineto(491.16183228,43.88572387)
\curveto(491.02183242,46.40572135)(489.20183,47.12572387)(486.92183228,47.12572387)
\curveto(485.12183408,47.12572387)(483.06183228,46.26572179)(483.06183228,44.18572387)
\curveto(483.06183228,42.20572585)(484.76183396,41.64572349)(486.44183228,41.26572387)
\curveto(488.14183058,40.88572425)(489.82183228,40.66572255)(489.82183228,39.34572387)
\curveto(489.82183228,38.04572517)(488.38183122,37.80572387)(487.32183228,37.80572387)
\curveto(485.92183368,37.80572387)(484.60183222,38.28572539)(484.54183228,39.80572387)
\lineto(482.84183228,39.80572387)
}
}
{
\newrgbcolor{curcolor}{0 0 0}
\pscustom[linewidth=2.48031497,linecolor=curcolor,linestyle=dashed,dash=2.48031496 9.92125984]
{
\newpath
\moveto(505.93645,100.84821838)
\lineto(30.368497,100.84821838)
}
}
{
\newrgbcolor{curcolor}{0 0 0}
\pscustom[linestyle=none,fillstyle=solid,fillcolor=curcolor]
{
\newpath
\moveto(428.79632602,370.6886327)
\curveto(428.13632668,375.24862814)(424.44632155,377.6486327)(419.97632602,377.6486327)
\curveto(413.37633262,377.6486327)(409.83632602,372.57862649)(409.83632602,366.3686327)
\curveto(409.83632602,360.12863894)(413.07633268,355.2086327)(419.73632602,355.2086327)
\curveto(425.13632062,355.2086327)(428.46632656,358.44863804)(429.00632602,363.7886327)
\lineto(426.15632602,363.7886327)
\curveto(425.88632629,360.24863624)(423.72632227,357.6086327)(419.97632602,357.6086327)
\curveto(414.84633115,357.6086327)(412.68632602,361.68863759)(412.68632602,366.5786327)
\curveto(412.68632602,371.04862823)(414.84633112,375.2486327)(419.94632602,375.2486327)
\curveto(422.91632305,375.2486327)(425.34632662,373.71862967)(425.94632602,370.6886327)
\lineto(428.79632602,370.6886327)
}
}
{
\newrgbcolor{curcolor}{0 0 0}
\pscustom[linestyle=none,fillstyle=solid,fillcolor=curcolor]
{
\newpath
\moveto(441.74601352,361.1186327)
\curveto(441.74601352,359.70863411)(440.36601016,357.6086327)(437.00601352,357.6086327)
\curveto(435.44601508,357.6086327)(434.00601352,358.20863438)(434.00601352,359.8886327)
\curveto(434.00601352,361.77863081)(435.4460152,362.378633)(437.12601352,362.6786327)
\curveto(438.83601181,362.9786324)(440.75601451,363.00863342)(441.74601352,363.7286327)
\lineto(441.74601352,361.1186327)
\moveto(445.88601352,357.7586327)
\curveto(445.55601385,357.63863282)(445.31601331,357.6086327)(445.10601352,357.6086327)
\curveto(444.29601433,357.6086327)(444.29601352,358.1486339)(444.29601352,359.3486327)
\lineto(444.29601352,367.3286327)
\curveto(444.29601352,370.95862907)(441.26601073,371.5886327)(438.47601352,371.5886327)
\curveto(435.02601697,371.5886327)(432.05601337,370.23862886)(431.90601352,366.3986327)
\lineto(434.45601352,366.3986327)
\curveto(434.5760134,368.67863042)(436.16601568,369.3386327)(438.32601352,369.3386327)
\curveto(439.9460119,369.3386327)(441.77601352,368.97863048)(441.77601352,366.7586327)
\curveto(441.77601352,364.83863462)(439.3760107,365.01863216)(436.55601352,364.4786327)
\curveto(433.91601616,363.96863321)(431.30601352,363.21862919)(431.30601352,359.7086327)
\curveto(431.30601352,356.61863579)(433.61601634,355.3586327)(436.43601352,355.3586327)
\curveto(438.59601136,355.3586327)(440.48601493,356.10863435)(441.89601352,357.7586327)
\curveto(441.89601352,356.07863438)(442.73601484,355.3586327)(444.05601352,355.3586327)
\curveto(444.86601271,355.3586327)(445.43601397,355.50863297)(445.88601352,355.7786327)
\lineto(445.88601352,357.7586327)
}
}
{
\newrgbcolor{curcolor}{0 0 0}
\pscustom[linestyle=none,fillstyle=solid,fillcolor=curcolor]
{
\newpath
\moveto(448.16929477,355.7186327)
\lineto(450.71929477,355.7186327)
\lineto(450.71929477,362.6186327)
\curveto(450.71929477,366.54862877)(452.21929888,368.8886327)(456.32929477,368.8886327)
\lineto(456.32929477,371.5886327)
\curveto(453.56929753,371.67863261)(451.85929354,370.44863021)(450.62929477,367.9586327)
\lineto(450.56929477,367.9586327)
\lineto(450.56929477,371.2286327)
\lineto(448.16929477,371.2286327)
\lineto(448.16929477,355.7186327)
}
}
{
\newrgbcolor{curcolor}{0 0 0}
\pscustom[linestyle=none,fillstyle=solid,fillcolor=curcolor]
{
\newpath
\moveto(458.18882602,355.7186327)
\lineto(460.73882602,355.7186327)
\lineto(460.73882602,362.6186327)
\curveto(460.73882602,366.54862877)(462.23883013,368.8886327)(466.34882602,368.8886327)
\lineto(466.34882602,371.5886327)
\curveto(463.58882878,371.67863261)(461.87882479,370.44863021)(460.64882602,367.9586327)
\lineto(460.58882602,367.9586327)
\lineto(460.58882602,371.2286327)
\lineto(458.18882602,371.2286327)
\lineto(458.18882602,355.7186327)
}
}
{
\newrgbcolor{curcolor}{0 0 0}
\pscustom[linestyle=none,fillstyle=solid,fillcolor=curcolor]
{
\newpath
\moveto(479.00835727,360.6386327)
\curveto(478.55835772,358.62863471)(477.08835517,357.6086327)(474.98835727,357.6086327)
\curveto(471.59836066,357.6086327)(470.06835736,360.0086354)(470.15835727,362.7086327)
\lineto(481.76835727,362.7086327)
\curveto(481.91835712,366.45862895)(480.23835178,371.5886327)(474.74835727,371.5886327)
\curveto(470.5183615,371.5886327)(467.45835727,368.16862805)(467.45835727,363.5186327)
\curveto(467.60835712,358.77863744)(469.94836222,355.3586327)(474.89835727,355.3586327)
\curveto(478.37835379,355.3586327)(480.83835796,357.21863612)(481.52835727,360.6386327)
\lineto(479.00835727,360.6386327)
\moveto(470.15835727,364.9586327)
\curveto(470.33835709,367.32863033)(471.92835994,369.3386327)(474.59835727,369.3386327)
\curveto(477.11835475,369.3386327)(478.94835739,367.38863027)(479.06835727,364.9586327)
\lineto(470.15835727,364.9586327)
}
}
{
\newrgbcolor{curcolor}{0 0 0}
\pscustom[linestyle=none,fillstyle=solid,fillcolor=curcolor]
{
\newpath
\moveto(484.32163852,355.7186327)
\lineto(486.87163852,355.7186327)
\lineto(486.87163852,362.6186327)
\curveto(486.87163852,366.54862877)(488.37164263,368.8886327)(492.48163852,368.8886327)
\lineto(492.48163852,371.5886327)
\curveto(489.72164128,371.67863261)(488.01163729,370.44863021)(486.78163852,367.9586327)
\lineto(486.72163852,367.9586327)
\lineto(486.72163852,371.2286327)
\lineto(484.32163852,371.2286327)
\lineto(484.32163852,355.7186327)
}
}
{
\newrgbcolor{curcolor}{0 0 0}
\pscustom[linestyle=none,fillstyle=solid,fillcolor=curcolor]
{
\newpath
\moveto(504.03116977,361.1186327)
\curveto(504.03116977,359.70863411)(502.65116641,357.6086327)(499.29116977,357.6086327)
\curveto(497.73117133,357.6086327)(496.29116977,358.20863438)(496.29116977,359.8886327)
\curveto(496.29116977,361.77863081)(497.73117145,362.378633)(499.41116977,362.6786327)
\curveto(501.12116806,362.9786324)(503.04117076,363.00863342)(504.03116977,363.7286327)
\lineto(504.03116977,361.1186327)
\moveto(508.17116977,357.7586327)
\curveto(507.8411701,357.63863282)(507.60116956,357.6086327)(507.39116977,357.6086327)
\curveto(506.58117058,357.6086327)(506.58116977,358.1486339)(506.58116977,359.3486327)
\lineto(506.58116977,367.3286327)
\curveto(506.58116977,370.95862907)(503.55116698,371.5886327)(500.76116977,371.5886327)
\curveto(497.31117322,371.5886327)(494.34116962,370.23862886)(494.19116977,366.3986327)
\lineto(496.74116977,366.3986327)
\curveto(496.86116965,368.67863042)(498.45117193,369.3386327)(500.61116977,369.3386327)
\curveto(502.23116815,369.3386327)(504.06116977,368.97863048)(504.06116977,366.7586327)
\curveto(504.06116977,364.83863462)(501.66116695,365.01863216)(498.84116977,364.4786327)
\curveto(496.20117241,363.96863321)(493.59116977,363.21862919)(493.59116977,359.7086327)
\curveto(493.59116977,356.61863579)(495.90117259,355.3586327)(498.72116977,355.3586327)
\curveto(500.88116761,355.3586327)(502.77117118,356.10863435)(504.18116977,357.7586327)
\curveto(504.18116977,356.07863438)(505.02117109,355.3586327)(506.34116977,355.3586327)
\curveto(507.15116896,355.3586327)(507.72117022,355.50863297)(508.17116977,355.7786327)
\lineto(508.17116977,357.7586327)
}
}
{
\newrgbcolor{curcolor}{0 0 0}
\pscustom[linewidth=2.65748024,linecolor=curcolor]
{
\newpath
\moveto(400.76647983,392.00745106)
\lineto(517.24104343,392.00745106)
\lineto(517.24104343,340.84976673)
\lineto(400.76647983,340.84976673)
\closepath
}
}
{
\newrgbcolor{curcolor}{0 0 0}
\pscustom[linestyle=none,fillstyle=solid,fillcolor=curcolor]
{
\newpath
\moveto(444.86088428,279.27615453)
\lineto(447.80088428,279.27615453)
\lineto(450.20088428,285.72615453)
\lineto(459.26088428,285.72615453)
\lineto(461.60088428,279.27615453)
\lineto(464.75088428,279.27615453)
\lineto(456.38088428,300.69615452)
\lineto(453.23088428,300.69615452)
\lineto(444.86088428,279.27615453)
\moveto(454.73088428,298.11615453)
\lineto(454.79088428,298.11615453)
\lineto(458.36088428,288.12615453)
\lineto(451.10088428,288.12615453)
\lineto(454.73088428,298.11615453)
\moveto(452.96088428,302.22615453)
\lineto(454.88088428,302.22615453)
\lineto(458.81088428,306.51615453)
\lineto(455.54088428,306.51615453)
\lineto(452.96088428,302.22615453)
}
}
{
\newrgbcolor{curcolor}{0 0 0}
\pscustom[linestyle=none,fillstyle=solid,fillcolor=curcolor]
{
\newpath
\moveto(466.35400928,279.27615453)
\lineto(468.90400928,279.27615453)
\lineto(468.90400928,286.17615453)
\curveto(468.90400928,290.1061506)(470.40401339,292.44615452)(474.51400928,292.44615452)
\lineto(474.51400928,295.14615453)
\curveto(471.75401204,295.23615444)(470.04400805,294.00615204)(468.81400928,291.51615453)
\lineto(468.75400928,291.51615453)
\lineto(468.75400928,294.78615453)
\lineto(466.35400928,294.78615453)
\lineto(466.35400928,279.27615453)
}
}
{
\newrgbcolor{curcolor}{0 0 0}
\pscustom[linestyle=none,fillstyle=solid,fillcolor=curcolor]
{
\newpath
\moveto(487.17354053,284.19615452)
\curveto(486.72354098,282.18615654)(485.25353843,281.16615453)(483.15354053,281.16615453)
\curveto(479.76354392,281.16615453)(478.23354062,283.56615722)(478.32354053,286.26615453)
\lineto(489.93354053,286.26615453)
\curveto(490.08354038,290.01615078)(488.40353504,295.14615453)(482.91354053,295.14615453)
\curveto(478.68354476,295.14615453)(475.62354053,291.72614988)(475.62354053,287.07615452)
\curveto(475.77354038,282.33615926)(478.11354548,278.91615453)(483.06354053,278.91615453)
\curveto(486.54353705,278.91615453)(489.00354122,280.77615795)(489.69354053,284.19615452)
\lineto(487.17354053,284.19615452)
\moveto(478.32354053,288.51615453)
\curveto(478.50354035,290.88615215)(480.0935432,292.89615453)(482.76354053,292.89615453)
\curveto(485.28353801,292.89615453)(487.11354065,290.9461521)(487.23354053,288.51615453)
\lineto(478.32354053,288.51615453)
}
}
{
\newrgbcolor{curcolor}{0 0 0}
\pscustom[linestyle=none,fillstyle=solid,fillcolor=curcolor]
{
\newpath
\moveto(502.17682178,284.67615453)
\curveto(502.17682178,283.26615594)(500.79681842,281.16615453)(497.43682178,281.16615453)
\curveto(495.87682334,281.16615453)(494.43682178,281.76615621)(494.43682178,283.44615452)
\curveto(494.43682178,285.33615264)(495.87682346,285.93615483)(497.55682178,286.23615453)
\curveto(499.26682007,286.53615423)(501.18682277,286.56615525)(502.17682178,287.28615453)
\lineto(502.17682178,284.67615453)
\moveto(506.31682178,281.31615453)
\curveto(505.98682211,281.19615465)(505.74682157,281.16615453)(505.53682178,281.16615453)
\curveto(504.72682259,281.16615453)(504.72682178,281.70615573)(504.72682178,282.90615453)
\lineto(504.72682178,290.88615453)
\curveto(504.72682178,294.5161509)(501.69681899,295.14615453)(498.90682178,295.14615453)
\curveto(495.45682523,295.14615453)(492.48682163,293.79615069)(492.33682178,289.95615452)
\lineto(494.88682178,289.95615452)
\curveto(495.00682166,292.23615225)(496.59682394,292.89615453)(498.75682178,292.89615453)
\curveto(500.37682016,292.89615453)(502.20682178,292.53615231)(502.20682178,290.31615453)
\curveto(502.20682178,288.39615645)(499.80681896,288.57615399)(496.98682178,288.03615453)
\curveto(494.34682442,287.52615503)(491.73682178,286.77615101)(491.73682178,283.26615453)
\curveto(491.73682178,280.17615762)(494.0468246,278.91615453)(496.86682178,278.91615453)
\curveto(499.02681962,278.91615453)(500.91682319,279.66615618)(502.32682178,281.31615453)
\curveto(502.32682178,279.63615621)(503.1668231,278.91615453)(504.48682178,278.91615453)
\curveto(505.29682097,278.91615453)(505.86682223,279.06615479)(506.31682178,279.33615452)
\lineto(506.31682178,281.31615453)
}
}
{
\newrgbcolor{curcolor}{0 0 0}
\pscustom[linewidth=2.65748024,linecolor=curcolor]
{
\newpath
\moveto(433.93666797,318.49563695)
\lineto(517.24104267,318.49563695)
\lineto(517.24104267,266.93671132)
\lineto(433.93666797,266.93671132)
\closepath
}
}
{
\newrgbcolor{curcolor}{0 0 0}
\pscustom[linestyle=none,fillstyle=solid,fillcolor=curcolor]
{
\newpath
\moveto(373.17144775,223.54569213)
\curveto(372.51144841,228.10568757)(368.82144328,230.50569213)(364.35144775,230.50569213)
\curveto(357.75145435,230.50569213)(354.21144775,225.43568592)(354.21144775,219.22569213)
\curveto(354.21144775,212.98569837)(357.45145441,208.06569213)(364.11144775,208.06569213)
\curveto(369.51144235,208.06569213)(372.84144829,211.30569747)(373.38144775,216.64569213)
\lineto(370.53144775,216.64569213)
\curveto(370.26144802,213.10569567)(368.101444,210.46569213)(364.35144775,210.46569213)
\curveto(359.22145288,210.46569213)(357.06144775,214.54569702)(357.06144775,219.43569213)
\curveto(357.06144775,223.90568766)(359.22145285,228.10569213)(364.32144775,228.10569213)
\curveto(367.29144478,228.10569213)(369.72144835,226.5756891)(370.32144775,223.54569213)
\lineto(373.17144775,223.54569213)
}
}
{
\newrgbcolor{curcolor}{0 0 0}
\pscustom[linestyle=none,fillstyle=solid,fillcolor=curcolor]
{
\newpath
\moveto(375.68113525,216.31569213)
\curveto(375.68113525,211.78569666)(378.29114017,208.21569213)(383.21113525,208.21569213)
\curveto(388.13113033,208.21569213)(390.74113525,211.78569666)(390.74113525,216.31569213)
\curveto(390.74113525,220.87568757)(388.13113033,224.44569213)(383.21113525,224.44569213)
\curveto(378.29114017,224.44569213)(375.68113525,220.87568757)(375.68113525,216.31569213)
\moveto(378.38113525,216.31569213)
\curveto(378.38113525,220.09568835)(380.54113792,222.19569213)(383.21113525,222.19569213)
\curveto(385.88113258,222.19569213)(388.04113525,220.09568835)(388.04113525,216.31569213)
\curveto(388.04113525,212.56569588)(385.88113258,210.46569213)(383.21113525,210.46569213)
\curveto(380.54113792,210.46569213)(378.38113525,212.56569588)(378.38113525,216.31569213)
}
}
{
\newrgbcolor{curcolor}{0 0 0}
\pscustom[linestyle=none,fillstyle=solid,fillcolor=curcolor]
{
\newpath
\moveto(393.74769775,208.57569213)
\lineto(396.29769775,208.57569213)
\lineto(396.29769775,218.23569213)
\curveto(396.29769775,219.40569096)(397.40770078,222.19569213)(400.43769775,222.19569213)
\curveto(402.71769547,222.19569213)(403.34769775,220.75569012)(403.34769775,218.74569213)
\lineto(403.34769775,208.57569213)
\lineto(405.89769775,208.57569213)
\lineto(405.89769775,218.23569213)
\curveto(405.89769775,220.63568973)(407.48770021,222.19569213)(409.94769775,222.19569213)
\curveto(412.43769526,222.19569213)(412.94769775,220.66569021)(412.94769775,218.74569213)
\lineto(412.94769775,208.57569213)
\lineto(415.49769775,208.57569213)
\lineto(415.49769775,219.94569213)
\curveto(415.49769775,223.15568892)(413.42769466,224.44569213)(410.33769775,224.44569213)
\curveto(408.35769973,224.44569213)(406.5276967,223.45569048)(405.47769775,221.80569213)
\curveto(404.84769838,223.69569024)(403.10769586,224.44569213)(401.21769775,224.44569213)
\curveto(399.08769988,224.44569213)(397.34769661,223.54569039)(396.20769775,221.80569213)
\lineto(396.14769775,221.80569213)
\lineto(396.14769775,224.08569213)
\lineto(393.74769775,224.08569213)
\lineto(393.74769775,208.57569213)
}
}
{
\newrgbcolor{curcolor}{0 0 0}
\pscustom[linestyle=none,fillstyle=solid,fillcolor=curcolor]
{
\newpath
\moveto(432.1931665,224.08569213)
\lineto(429.6431665,224.08569213)
\lineto(429.6431665,215.32569213)
\curveto(429.6431665,212.53569492)(428.14316341,210.46569213)(425.0531665,210.46569213)
\curveto(423.10316845,210.46569213)(421.9031665,211.69569402)(421.9031665,213.58569213)
\lineto(421.9031665,224.08569213)
\lineto(419.3531665,224.08569213)
\lineto(419.3531665,213.88569213)
\curveto(419.3531665,210.55569546)(420.61317058,208.21569213)(424.6931665,208.21569213)
\curveto(426.91316428,208.21569213)(428.65316758,209.11569405)(429.7331665,211.03569213)
\lineto(429.7931665,211.03569213)
\lineto(429.7931665,208.57569213)
\lineto(432.1931665,208.57569213)
\lineto(432.1931665,224.08569213)
}
}
{
\newrgbcolor{curcolor}{0 0 0}
\pscustom[linestyle=none,fillstyle=solid,fillcolor=curcolor]
{
\newpath
\moveto(436.05238525,208.57569213)
\lineto(438.60238525,208.57569213)
\lineto(438.60238525,217.33569213)
\curveto(438.60238525,220.12568934)(440.10238834,222.19569213)(443.19238525,222.19569213)
\curveto(445.1423833,222.19569213)(446.34238525,220.96569024)(446.34238525,219.07569213)
\lineto(446.34238525,208.57569213)
\lineto(448.89238525,208.57569213)
\lineto(448.89238525,218.77569213)
\curveto(448.89238525,222.1056888)(447.63238117,224.44569213)(443.55238525,224.44569213)
\curveto(441.33238747,224.44569213)(439.59238417,223.54569021)(438.51238525,221.62569213)
\lineto(438.45238525,221.62569213)
\lineto(438.45238525,224.08569213)
\lineto(436.05238525,224.08569213)
\lineto(436.05238525,208.57569213)
}
}
{
\newrgbcolor{curcolor}{0 0 0}
\pscustom[linestyle=none,fillstyle=solid,fillcolor=curcolor]
{
\newpath
\moveto(452.901604,208.57569213)
\lineto(455.451604,208.57569213)
\lineto(455.451604,224.08569213)
\lineto(452.901604,224.08569213)
\lineto(452.901604,208.57569213)
\moveto(455.451604,229.99569213)
\lineto(452.901604,229.99569213)
\lineto(452.901604,226.87569213)
\lineto(455.451604,226.87569213)
\lineto(455.451604,229.99569213)
}
}
{
\newrgbcolor{curcolor}{0 0 0}
\pscustom[linestyle=none,fillstyle=solid,fillcolor=curcolor]
{
\newpath
\moveto(473.2912915,229.99569213)
\lineto(470.7412915,229.99569213)
\lineto(470.7412915,222.01569213)
\lineto(470.6812915,222.01569213)
\curveto(469.54129264,223.8456903)(467.26128991,224.44569213)(465.6712915,224.44569213)
\curveto(460.93129624,224.44569213)(458.5912915,220.75568775)(458.5912915,216.37569213)
\curveto(458.5912915,211.99569651)(460.96129627,208.21569213)(465.7312915,208.21569213)
\curveto(467.86128937,208.21569213)(469.84129234,208.96569384)(470.6812915,210.67569213)
\lineto(470.7412915,210.67569213)
\lineto(470.7412915,208.57569213)
\lineto(473.2912915,208.57569213)
\lineto(473.2912915,229.99569213)
\moveto(461.2912915,216.19569213)
\curveto(461.2912915,219.2256891)(462.46129498,222.19569213)(465.9412915,222.19569213)
\curveto(469.45128799,222.19569213)(470.8312915,219.37568907)(470.8312915,216.31569213)
\curveto(470.8312915,213.40569504)(469.51128808,210.46569213)(466.0912915,210.46569213)
\curveto(462.7912948,210.46569213)(461.2912915,213.28569504)(461.2912915,216.19569213)
}
}
{
\newrgbcolor{curcolor}{0 0 0}
\pscustom[linestyle=none,fillstyle=solid,fillcolor=curcolor]
{
\newpath
\moveto(486.8437915,213.97569213)
\curveto(486.8437915,212.56569354)(485.46378814,210.46569213)(482.1037915,210.46569213)
\curveto(480.54379306,210.46569213)(479.1037915,211.06569381)(479.1037915,212.74569213)
\curveto(479.1037915,214.63569024)(480.54379318,215.23569243)(482.2237915,215.53569213)
\curveto(483.93378979,215.83569183)(485.85379249,215.86569285)(486.8437915,216.58569213)
\lineto(486.8437915,213.97569213)
\moveto(490.9837915,210.61569213)
\curveto(490.65379183,210.49569225)(490.41379129,210.46569213)(490.2037915,210.46569213)
\curveto(489.39379231,210.46569213)(489.3937915,211.00569333)(489.3937915,212.20569213)
\lineto(489.3937915,220.18569213)
\curveto(489.3937915,223.8156885)(486.36378871,224.44569213)(483.5737915,224.44569213)
\curveto(480.12379495,224.44569213)(477.15379135,223.09568829)(477.0037915,219.25569213)
\lineto(479.5537915,219.25569213)
\curveto(479.67379138,221.53568985)(481.26379366,222.19569213)(483.4237915,222.19569213)
\curveto(485.04378988,222.19569213)(486.8737915,221.83568991)(486.8737915,219.61569213)
\curveto(486.8737915,217.69569405)(484.47378868,217.87569159)(481.6537915,217.33569213)
\curveto(479.01379414,216.82569264)(476.4037915,216.07568862)(476.4037915,212.56569213)
\curveto(476.4037915,209.47569522)(478.71379432,208.21569213)(481.5337915,208.21569213)
\curveto(483.69378934,208.21569213)(485.58379291,208.96569378)(486.9937915,210.61569213)
\curveto(486.9937915,208.93569381)(487.83379282,208.21569213)(489.1537915,208.21569213)
\curveto(489.96379069,208.21569213)(490.53379195,208.3656924)(490.9837915,208.63569213)
\lineto(490.9837915,210.61569213)
}
}
{
\newrgbcolor{curcolor}{0 0 0}
\pscustom[linestyle=none,fillstyle=solid,fillcolor=curcolor]
{
\newpath
\moveto(507.21707275,229.99569213)
\lineto(504.66707275,229.99569213)
\lineto(504.66707275,222.01569213)
\lineto(504.60707275,222.01569213)
\curveto(503.46707389,223.8456903)(501.18707116,224.44569213)(499.59707275,224.44569213)
\curveto(494.85707749,224.44569213)(492.51707275,220.75568775)(492.51707275,216.37569213)
\curveto(492.51707275,211.99569651)(494.88707752,208.21569213)(499.65707275,208.21569213)
\curveto(501.78707062,208.21569213)(503.76707359,208.96569384)(504.60707275,210.67569213)
\lineto(504.66707275,210.67569213)
\lineto(504.66707275,208.57569213)
\lineto(507.21707275,208.57569213)
\lineto(507.21707275,229.99569213)
\moveto(495.21707275,216.19569213)
\curveto(495.21707275,219.2256891)(496.38707623,222.19569213)(499.86707275,222.19569213)
\curveto(503.37706924,222.19569213)(504.75707275,219.37568907)(504.75707275,216.31569213)
\curveto(504.75707275,213.40569504)(503.43706933,210.46569213)(500.01707275,210.46569213)
\curveto(496.71707605,210.46569213)(495.21707275,213.28569504)(495.21707275,216.19569213)
}
}
{
\newrgbcolor{curcolor}{0 0 0}
\pscustom[linewidth=2.65748024,linecolor=curcolor]
{
\newpath
\moveto(344.1875,244.5825891)
\lineto(517.24104309,244.5825891)
\lineto(517.24104309,193.98877425)
\lineto(344.1875,193.98877425)
\closepath
}
}
{
\newrgbcolor{curcolor}{0 0 0}
\pscustom[linewidth=2.65748024,linecolor=curcolor]
{
\newpath
\moveto(375,366.42857138)
\lineto(170.71429,366.42857138)
}
}
{
\newrgbcolor{curcolor}{0 0 0}
\pscustom[linestyle=none,fillstyle=solid,fillcolor=curcolor]
{
\newpath
\moveto(184.61597488,359.99682843)
\lineto(167.19479306,366.40303843)
\lineto(184.61597584,372.80924702)
\curveto(181.83279933,369.02702404)(181.84883606,363.85228657)(184.61597488,359.99682843)
\closepath
}
}
{
\newrgbcolor{curcolor}{0 0 0}
\pscustom[linewidth=2.65748024,linecolor=curcolor]
{
\newpath
\moveto(412.85714,305.71427838)
\lineto(220.71429,305.71427838)
}
}
{
\newrgbcolor{curcolor}{0 0 0}
\pscustom[linestyle=none,fillstyle=solid,fillcolor=curcolor]
{
\newpath
\moveto(234.61597488,299.28253543)
\lineto(217.19479306,305.68874543)
\lineto(234.61597584,312.09495402)
\curveto(231.83279933,308.31273104)(231.84883606,303.13799357)(234.61597488,299.28253543)
\closepath
}
}
\end{pspicture}

\caption{Espacios virtuales que se han de construir y su clasificación en el
sistema.}
\label{espacios}
\end{figure}

Cada uno de ellos posee la funcionalidad característica de un recurso
administrable, es decir, posee las siguientes operaciones\footnote{Para una
definición exacta puede consultarse:
https://es.wikipedia.org/wiki/Create,\_read,\_update\_and\_delete}:

\begin{itemize}
\item Crear un nuevo elemento (CREATE).
\item Visualizar el elemento a detalle (READ).
\item Editar las características del elemento (UPDATE).
\item Eliminación del elemento (DELETE).
\end{itemize}

A su vez se han establecido un conjunto de tareas por lote para facilitar la
correcta manipulación de amplios volúmenes de información. Estas tareas son:

\begin{itemize}
\item Importación de datos desde un archivo CSV.
\item Exportación de datos hacia un archivo CSV.
\item Habilitación/Inhabilitación de elementos, ya sea individualmente o en
      grupos de elementos.
\end{itemize}

\subsection{Intercambio de recursos}

Cada espacio virtual debe poseer la capacidad de contener información en
distintos formatos, y para diversos propósitos. El objetivo principal es poder
compartir piezas de información entre usuarios del sistema.

Para este propósito, se han definido piezas atómicas de información básica,
estas son:

\begin{description}
\item [Notas] Son piezas de texto que no poseen formato, y representan la unidad
de información mas básica que ha de construirse.
\item [Archivos] Los archivos representan recursos que los usuarios suben al
sistema, y no esta contemplado ninguna tarea adicional, a parte de alojarlos y
brindar la capacidad de ser descargados por otros usuarios.
\item [Imágenes] Una imagen es la única pieza provista para representación
visual en el sistema. Esta adicionalmente a ser subida por un usuario debe
poder ser visualizada y descargada por parte de los demás usuarios.
\item [Vídeos] Inicialmente los vídeos representan archivos en el formato flv,
que puedan ser reproducibles en un player de adobe flash.
\item [Eventos] Los eventos son piezas que demarcan la iniciación y duración de
una actividad, estas pueden ser creadas por algún usuario y visualizados por
otros usuarios, según el espacio virtual en el que se encuentre.
\item [Enlaces] Inicialmente se contempla únicamente la publicación de enlaces,
sin análisis del lugar a donde conducen, a la larga se plantea la posibilidad de
analizar el recurso destino y poder reenderizarse la información según tal
inspección (es deseable, pero no esta contemplado en los alcances de este
proyecto).
\end{description}

Todos estos tipos de recursos poseen también características de espacio virtual,
es decir, que cada una de ellas posee operaciones CRUD, además de
funcionalidades para el fomento a la participación.

\subsection{Canales de comunicación}

Para la mejora de los canales de comunicación se ha definido el manejo de otros
tipos de espacios-recursos adicionales que poseen diferentes propósitos
utilitarios, estos son:

\begin{description}
\item [Usuarios] Para incrementar la afinidad de los usuarios hacia el sistema
que conforma este proyecto, se han definido la construcción de espacios propios
para cada usuario. De modo que este pueda controlar los recursos que produzca y
que sean realmente suyos.
\item [Roles] Un rol define el tipo de participación que puede poseer un usuario
en el sistema, inicialmente se han creado un conjunto de roles, que esta acordes
a la lógica del contexto de implantación (UMSS):
    \begin{itemize}
    \item Administrador
    \item Desarrollador
    \item Moderador
    \item Docente
    \item Auxiliar
    \item Estudiante
    \item Invitado
    \end{itemize}
\item [Contactos] La característica mas propia de una red social esta basada en
la creación de vínculos entre usuarios del sistema, para esto se ha creado una
cadena de contactos, estos vínculos pueden ser de tres tipos (estos están
basados en la forma que son manejados por la red social twitter) (figura
\ref{contactos}):
\begin{description}
\item [Follower] Representa una relación uní-direccional, de un usuario que
ve los recursos que produce otro usuario.
\item [Following] Representa una relación uní-direccional, de un usuario que
produce los recursos que otros usuarios pueden ver.
\item [Friend] Representa una relación bi-direccional, entre dos usuarios,
que comparten los recursos que producen.
\end{description}
Las relaciones de tipo \emph{friend}, se consideran relaciones fuertes, mientras
que las otras dos clases de relaciones, son consideradas relaciones debiles.
\end{description}

\begin{figure}
\centering
%LaTeX with PSTricks extensions
%%Creator: inkscape 0.48.4
%%Please note this file requires PSTricks extensions
\psset{xunit=.5pt,yunit=.5pt,runit=.5pt}
\begin{pspicture}(531.49603271,248.03149414)
{
\newrgbcolor{curcolor}{0 0 0}
\pscustom[linestyle=none,fillstyle=solid,fillcolor=curcolor]
{
\newpath
\moveto(48.30160622,53.29359723)
\lineto(45.45160622,53.29359723)
\lineto(45.45160622,39.61359723)
\curveto(45.45160622,35.83360101)(43.47160262,33.76359723)(39.87160622,33.76359723)
\curveto(36.09161,33.76359723)(33.93160622,35.83360101)(33.93160622,39.61359723)
\lineto(33.93160622,53.29359723)
\lineto(31.08160622,53.29359723)
\lineto(31.08160622,39.61359723)
\curveto(31.08160622,33.91360293)(34.35161174,31.36359723)(39.87160622,31.36359723)
\curveto(45.21160088,31.36359723)(48.30160622,34.21360263)(48.30160622,39.61359723)
\lineto(48.30160622,53.29359723)
}
}
{
\newrgbcolor{curcolor}{0 0 0}
\pscustom[linestyle=none,fillstyle=solid,fillcolor=curcolor]
{
\newpath
\moveto(51.47129372,36.76359723)
\curveto(51.62129357,32.92360107)(54.5612972,31.51359723)(58.04129372,31.51359723)
\curveto(61.19129057,31.51359723)(64.64129372,32.71360092)(64.64129372,36.40359723)
\curveto(64.64129372,39.40359423)(62.12129117,40.2435978)(59.57129372,40.81359723)
\curveto(57.20129609,41.38359666)(54.50129372,41.68359906)(54.50129372,43.51359723)
\curveto(54.50129372,45.07359567)(56.27129525,45.49359723)(57.80129372,45.49359723)
\curveto(59.48129204,45.49359723)(61.2212939,44.86359525)(61.40129372,42.88359723)
\lineto(63.95129372,42.88359723)
\curveto(63.74129393,46.66359345)(61.0112903,47.74359723)(57.59129372,47.74359723)
\curveto(54.89129642,47.74359723)(51.80129372,46.45359411)(51.80129372,43.33359723)
\curveto(51.80129372,40.3636002)(54.35129624,39.52359666)(56.87129372,38.95359723)
\curveto(59.42129117,38.3835978)(61.94129372,38.05359525)(61.94129372,36.07359723)
\curveto(61.94129372,34.12359918)(59.78129213,33.76359723)(58.19129372,33.76359723)
\curveto(56.09129582,33.76359723)(54.11129363,34.48359951)(54.02129372,36.76359723)
\lineto(51.47129372,36.76359723)
}
}
{
\newrgbcolor{curcolor}{0 0 0}
\pscustom[linestyle=none,fillstyle=solid,fillcolor=curcolor]
{
\newpath
\moveto(80.30129372,47.38359723)
\lineto(77.75129372,47.38359723)
\lineto(77.75129372,38.62359723)
\curveto(77.75129372,35.83360002)(76.25129063,33.76359723)(73.16129372,33.76359723)
\curveto(71.21129567,33.76359723)(70.01129372,34.99359912)(70.01129372,36.88359723)
\lineto(70.01129372,47.38359723)
\lineto(67.46129372,47.38359723)
\lineto(67.46129372,37.18359723)
\curveto(67.46129372,33.85360056)(68.7212978,31.51359723)(72.80129372,31.51359723)
\curveto(75.0212915,31.51359723)(76.7612948,32.41359915)(77.84129372,34.33359723)
\lineto(77.90129372,34.33359723)
\lineto(77.90129372,31.87359723)
\lineto(80.30129372,31.87359723)
\lineto(80.30129372,47.38359723)
}
}
{
\newrgbcolor{curcolor}{0 0 0}
\pscustom[linestyle=none,fillstyle=solid,fillcolor=curcolor]
{
\newpath
\moveto(93.76051247,37.27359723)
\curveto(93.76051247,35.86359864)(92.38050911,33.76359723)(89.02051247,33.76359723)
\curveto(87.46051403,33.76359723)(86.02051247,34.36359891)(86.02051247,36.04359723)
\curveto(86.02051247,37.93359534)(87.46051415,38.53359753)(89.14051247,38.83359723)
\curveto(90.85051076,39.13359693)(92.77051346,39.16359795)(93.76051247,39.88359723)
\lineto(93.76051247,37.27359723)
\moveto(97.90051247,33.91359723)
\curveto(97.5705128,33.79359735)(97.33051226,33.76359723)(97.12051247,33.76359723)
\curveto(96.31051328,33.76359723)(96.31051247,34.30359843)(96.31051247,35.50359723)
\lineto(96.31051247,43.48359723)
\curveto(96.31051247,47.1135936)(93.28050968,47.74359723)(90.49051247,47.74359723)
\curveto(87.04051592,47.74359723)(84.07051232,46.39359339)(83.92051247,42.55359723)
\lineto(86.47051247,42.55359723)
\curveto(86.59051235,44.83359495)(88.18051463,45.49359723)(90.34051247,45.49359723)
\curveto(91.96051085,45.49359723)(93.79051247,45.13359501)(93.79051247,42.91359723)
\curveto(93.79051247,40.99359915)(91.39050965,41.17359669)(88.57051247,40.63359723)
\curveto(85.93051511,40.12359774)(83.32051247,39.37359372)(83.32051247,35.86359723)
\curveto(83.32051247,32.77360032)(85.63051529,31.51359723)(88.45051247,31.51359723)
\curveto(90.61051031,31.51359723)(92.50051388,32.26359888)(93.91051247,33.91359723)
\curveto(93.91051247,32.23359891)(94.75051379,31.51359723)(96.07051247,31.51359723)
\curveto(96.88051166,31.51359723)(97.45051292,31.6635975)(97.90051247,31.93359723)
\lineto(97.90051247,33.91359723)
}
}
{
\newrgbcolor{curcolor}{0 0 0}
\pscustom[linestyle=none,fillstyle=solid,fillcolor=curcolor]
{
\newpath
\moveto(100.18379372,31.87359723)
\lineto(102.73379372,31.87359723)
\lineto(102.73379372,38.77359723)
\curveto(102.73379372,42.7035933)(104.23379783,45.04359723)(108.34379372,45.04359723)
\lineto(108.34379372,47.74359723)
\curveto(105.58379648,47.83359714)(103.87379249,46.60359474)(102.64379372,44.11359723)
\lineto(102.58379372,44.11359723)
\lineto(102.58379372,47.38359723)
\lineto(100.18379372,47.38359723)
\lineto(100.18379372,31.87359723)
}
}
{
\newrgbcolor{curcolor}{0 0 0}
\pscustom[linestyle=none,fillstyle=solid,fillcolor=curcolor]
{
\newpath
\moveto(110.44332497,31.87359723)
\lineto(112.99332497,31.87359723)
\lineto(112.99332497,47.38359723)
\lineto(110.44332497,47.38359723)
\lineto(110.44332497,31.87359723)
\moveto(112.99332497,53.29359723)
\lineto(110.44332497,53.29359723)
\lineto(110.44332497,50.17359723)
\lineto(112.99332497,50.17359723)
\lineto(112.99332497,53.29359723)
}
}
{
\newrgbcolor{curcolor}{0 0 0}
\pscustom[linestyle=none,fillstyle=solid,fillcolor=curcolor]
{
\newpath
\moveto(116.13301247,39.61359723)
\curveto(116.13301247,35.08360176)(118.74301739,31.51359723)(123.66301247,31.51359723)
\curveto(128.58300755,31.51359723)(131.19301247,35.08360176)(131.19301247,39.61359723)
\curveto(131.19301247,44.17359267)(128.58300755,47.74359723)(123.66301247,47.74359723)
\curveto(118.74301739,47.74359723)(116.13301247,44.17359267)(116.13301247,39.61359723)
\moveto(118.83301247,39.61359723)
\curveto(118.83301247,43.39359345)(120.99301514,45.49359723)(123.66301247,45.49359723)
\curveto(126.3330098,45.49359723)(128.49301247,43.39359345)(128.49301247,39.61359723)
\curveto(128.49301247,35.86360098)(126.3330098,33.76359723)(123.66301247,33.76359723)
\curveto(120.99301514,33.76359723)(118.83301247,35.86360098)(118.83301247,39.61359723)
}
}
{
\newrgbcolor{curcolor}{0 0 0}
\pscustom[linestyle=none,fillstyle=solid,fillcolor=curcolor]
{
}
}
{
\newrgbcolor{curcolor}{0 0 0}
\pscustom[linestyle=none,fillstyle=solid,fillcolor=curcolor]
{
\newpath
\moveto(151.27988747,53.14359723)
\lineto(149.32988747,53.14359723)
\curveto(148.75988804,49.93360044)(146.11988456,49.15359723)(143.20988747,49.15359723)
\lineto(143.20988747,47.11359723)
\lineto(148.72988747,47.11359723)
\lineto(148.72988747,31.87359723)
\lineto(151.27988747,31.87359723)
\lineto(151.27988747,53.14359723)
}
}
{
\newrgbcolor{curcolor}{0 0 0}
\pscustom[linewidth=2.65748024,linecolor=curcolor]
{
\newpath
\moveto(91.1807452,180.25557914)
\lineto(91.1807452,115.64372914)
}
}
{
\newrgbcolor{curcolor}{1 1 1}
\pscustom[linestyle=none,fillstyle=solid,fillcolor=curcolor]
{
\newpath
\moveto(114.0378901,192.12487159)
\curveto(114.0378901,179.30397606)(103.64450073,168.91058669)(90.8236052,168.91058669)
\curveto(78.00270967,168.91058669)(67.6093203,179.30397606)(67.6093203,192.12487159)
\curveto(67.6093203,204.94576711)(78.00270967,215.33915648)(90.8236052,215.33915648)
\curveto(103.63037851,215.33915648)(114.01790889,204.96779736)(114.03786192,192.16103959)
}
}
{
\newrgbcolor{curcolor}{0 0 0}
\pscustom[linewidth=2.65748024,linecolor=curcolor]
{
\newpath
\moveto(114.0378901,192.12487159)
\curveto(114.0378901,179.30397606)(103.64450073,168.91058669)(90.8236052,168.91058669)
\curveto(78.00270967,168.91058669)(67.6093203,179.30397606)(67.6093203,192.12487159)
\curveto(67.6093203,204.94576711)(78.00270967,215.33915648)(90.8236052,215.33915648)
\curveto(103.63037851,215.33915648)(114.01790889,204.96779736)(114.03786192,192.16103959)
}
}
{
\newrgbcolor{curcolor}{0 0 0}
\pscustom[linewidth=2.65748024,linecolor=curcolor]
{
\newpath
\moveto(70.0480442,79.26156914)
\lineto(91.3286852,117.51609914)
\lineto(111.4153452,79.26156914)
}
}
{
\newrgbcolor{curcolor}{0 0 0}
\pscustom[linewidth=2.65748024,linecolor=curcolor]
{
\newpath
\moveto(58.6807482,155.08045914)
\lineto(123.6807452,155.08045914)
}
}
{
\newrgbcolor{curcolor}{0 0 0}
\pscustom[linestyle=none,fillstyle=solid,fillcolor=curcolor]
{
\newpath
\moveto(397.43614522,53.29359723)
\lineto(394.58614522,53.29359723)
\lineto(394.58614522,39.61359723)
\curveto(394.58614522,35.83360101)(392.60614162,33.76359723)(389.00614522,33.76359723)
\curveto(385.226149,33.76359723)(383.06614522,35.83360101)(383.06614522,39.61359723)
\lineto(383.06614522,53.29359723)
\lineto(380.21614522,53.29359723)
\lineto(380.21614522,39.61359723)
\curveto(380.21614522,33.91360293)(383.48615074,31.36359723)(389.00614522,31.36359723)
\curveto(394.34613988,31.36359723)(397.43614522,34.21360263)(397.43614522,39.61359723)
\lineto(397.43614522,53.29359723)
}
}
{
\newrgbcolor{curcolor}{0 0 0}
\pscustom[linestyle=none,fillstyle=solid,fillcolor=curcolor]
{
\newpath
\moveto(400.60583272,36.76359723)
\curveto(400.75583257,32.92360107)(403.6958362,31.51359723)(407.17583272,31.51359723)
\curveto(410.32582957,31.51359723)(413.77583272,32.71360092)(413.77583272,36.40359723)
\curveto(413.77583272,39.40359423)(411.25583017,40.2435978)(408.70583272,40.81359723)
\curveto(406.33583509,41.38359666)(403.63583272,41.68359906)(403.63583272,43.51359723)
\curveto(403.63583272,45.07359567)(405.40583425,45.49359723)(406.93583272,45.49359723)
\curveto(408.61583104,45.49359723)(410.3558329,44.86359525)(410.53583272,42.88359723)
\lineto(413.08583272,42.88359723)
\curveto(412.87583293,46.66359345)(410.1458293,47.74359723)(406.72583272,47.74359723)
\curveto(404.02583542,47.74359723)(400.93583272,46.45359411)(400.93583272,43.33359723)
\curveto(400.93583272,40.3636002)(403.48583524,39.52359666)(406.00583272,38.95359723)
\curveto(408.55583017,38.3835978)(411.07583272,38.05359525)(411.07583272,36.07359723)
\curveto(411.07583272,34.12359918)(408.91583113,33.76359723)(407.32583272,33.76359723)
\curveto(405.22583482,33.76359723)(403.24583263,34.48359951)(403.15583272,36.76359723)
\lineto(400.60583272,36.76359723)
}
}
{
\newrgbcolor{curcolor}{0 0 0}
\pscustom[linestyle=none,fillstyle=solid,fillcolor=curcolor]
{
\newpath
\moveto(429.43583272,47.38359723)
\lineto(426.88583272,47.38359723)
\lineto(426.88583272,38.62359723)
\curveto(426.88583272,35.83360002)(425.38582963,33.76359723)(422.29583272,33.76359723)
\curveto(420.34583467,33.76359723)(419.14583272,34.99359912)(419.14583272,36.88359723)
\lineto(419.14583272,47.38359723)
\lineto(416.59583272,47.38359723)
\lineto(416.59583272,37.18359723)
\curveto(416.59583272,33.85360056)(417.8558368,31.51359723)(421.93583272,31.51359723)
\curveto(424.1558305,31.51359723)(425.8958338,32.41359915)(426.97583272,34.33359723)
\lineto(427.03583272,34.33359723)
\lineto(427.03583272,31.87359723)
\lineto(429.43583272,31.87359723)
\lineto(429.43583272,47.38359723)
}
}
{
\newrgbcolor{curcolor}{0 0 0}
\pscustom[linestyle=none,fillstyle=solid,fillcolor=curcolor]
{
\newpath
\moveto(442.89505147,37.27359723)
\curveto(442.89505147,35.86359864)(441.51504811,33.76359723)(438.15505147,33.76359723)
\curveto(436.59505303,33.76359723)(435.15505147,34.36359891)(435.15505147,36.04359723)
\curveto(435.15505147,37.93359534)(436.59505315,38.53359753)(438.27505147,38.83359723)
\curveto(439.98504976,39.13359693)(441.90505246,39.16359795)(442.89505147,39.88359723)
\lineto(442.89505147,37.27359723)
\moveto(447.03505147,33.91359723)
\curveto(446.7050518,33.79359735)(446.46505126,33.76359723)(446.25505147,33.76359723)
\curveto(445.44505228,33.76359723)(445.44505147,34.30359843)(445.44505147,35.50359723)
\lineto(445.44505147,43.48359723)
\curveto(445.44505147,47.1135936)(442.41504868,47.74359723)(439.62505147,47.74359723)
\curveto(436.17505492,47.74359723)(433.20505132,46.39359339)(433.05505147,42.55359723)
\lineto(435.60505147,42.55359723)
\curveto(435.72505135,44.83359495)(437.31505363,45.49359723)(439.47505147,45.49359723)
\curveto(441.09504985,45.49359723)(442.92505147,45.13359501)(442.92505147,42.91359723)
\curveto(442.92505147,40.99359915)(440.52504865,41.17359669)(437.70505147,40.63359723)
\curveto(435.06505411,40.12359774)(432.45505147,39.37359372)(432.45505147,35.86359723)
\curveto(432.45505147,32.77360032)(434.76505429,31.51359723)(437.58505147,31.51359723)
\curveto(439.74504931,31.51359723)(441.63505288,32.26359888)(443.04505147,33.91359723)
\curveto(443.04505147,32.23359891)(443.88505279,31.51359723)(445.20505147,31.51359723)
\curveto(446.01505066,31.51359723)(446.58505192,31.6635975)(447.03505147,31.93359723)
\lineto(447.03505147,33.91359723)
}
}
{
\newrgbcolor{curcolor}{0 0 0}
\pscustom[linestyle=none,fillstyle=solid,fillcolor=curcolor]
{
\newpath
\moveto(449.31833272,31.87359723)
\lineto(451.86833272,31.87359723)
\lineto(451.86833272,38.77359723)
\curveto(451.86833272,42.7035933)(453.36833683,45.04359723)(457.47833272,45.04359723)
\lineto(457.47833272,47.74359723)
\curveto(454.71833548,47.83359714)(453.00833149,46.60359474)(451.77833272,44.11359723)
\lineto(451.71833272,44.11359723)
\lineto(451.71833272,47.38359723)
\lineto(449.31833272,47.38359723)
\lineto(449.31833272,31.87359723)
}
}
{
\newrgbcolor{curcolor}{0 0 0}
\pscustom[linestyle=none,fillstyle=solid,fillcolor=curcolor]
{
\newpath
\moveto(459.57786397,31.87359723)
\lineto(462.12786397,31.87359723)
\lineto(462.12786397,47.38359723)
\lineto(459.57786397,47.38359723)
\lineto(459.57786397,31.87359723)
\moveto(462.12786397,53.29359723)
\lineto(459.57786397,53.29359723)
\lineto(459.57786397,50.17359723)
\lineto(462.12786397,50.17359723)
\lineto(462.12786397,53.29359723)
}
}
{
\newrgbcolor{curcolor}{0 0 0}
\pscustom[linestyle=none,fillstyle=solid,fillcolor=curcolor]
{
\newpath
\moveto(465.26755147,39.61359723)
\curveto(465.26755147,35.08360176)(467.87755639,31.51359723)(472.79755147,31.51359723)
\curveto(477.71754655,31.51359723)(480.32755147,35.08360176)(480.32755147,39.61359723)
\curveto(480.32755147,44.17359267)(477.71754655,47.74359723)(472.79755147,47.74359723)
\curveto(467.87755639,47.74359723)(465.26755147,44.17359267)(465.26755147,39.61359723)
\moveto(467.96755147,39.61359723)
\curveto(467.96755147,43.39359345)(470.12755414,45.49359723)(472.79755147,45.49359723)
\curveto(475.4675488,45.49359723)(477.62755147,43.39359345)(477.62755147,39.61359723)
\curveto(477.62755147,35.86360098)(475.4675488,33.76359723)(472.79755147,33.76359723)
\curveto(470.12755414,33.76359723)(467.96755147,35.86360098)(467.96755147,39.61359723)
}
}
{
\newrgbcolor{curcolor}{0 0 0}
\pscustom[linestyle=none,fillstyle=solid,fillcolor=curcolor]
{
}
}
{
\newrgbcolor{curcolor}{0 0 0}
\pscustom[linestyle=none,fillstyle=solid,fillcolor=curcolor]
{
\newpath
\moveto(493.60442647,45.61359723)
\curveto(493.51442656,48.16359468)(494.77442944,50.89359723)(497.74442647,50.89359723)
\curveto(499.99442422,50.89359723)(501.85442647,49.36359492)(501.85442647,47.05359723)
\curveto(501.85442647,44.11360017)(500.02442287,42.79359501)(496.42442647,40.57359723)
\curveto(493.42442947,38.71359909)(490.87442605,36.91359219)(490.45442647,31.87359723)
\lineto(504.34442647,31.87359723)
\lineto(504.34442647,34.12359723)
\lineto(493.42442647,34.12359723)
\curveto(493.93442596,36.76359459)(496.72442914,38.11359885)(499.39442647,39.73359723)
\curveto(502.03442383,41.38359558)(504.55442647,43.27360098)(504.55442647,47.02359723)
\curveto(504.55442647,50.98359327)(501.61442275,53.14359723)(497.89442647,53.14359723)
\curveto(493.39443097,53.14359723)(490.84442668,49.93359291)(491.05442647,45.61359723)
\lineto(493.60442647,45.61359723)
}
}
{
\newrgbcolor{curcolor}{0 0 0}
\pscustom[linewidth=2.65748024,linecolor=curcolor]
{
\newpath
\moveto(440.3152842,180.25557914)
\lineto(440.3152842,115.64372914)
}
}
{
\newrgbcolor{curcolor}{1 1 1}
\pscustom[linestyle=none,fillstyle=solid,fillcolor=curcolor]
{
\newpath
\moveto(463.1724291,192.12487159)
\curveto(463.1724291,179.30397606)(452.77903973,168.91058669)(439.9581442,168.91058669)
\curveto(427.13724867,168.91058669)(416.7438593,179.30397606)(416.7438593,192.12487159)
\curveto(416.7438593,204.94576711)(427.13724867,215.33915648)(439.9581442,215.33915648)
\curveto(452.76491751,215.33915648)(463.15244789,204.96779736)(463.17240092,192.16103959)
}
}
{
\newrgbcolor{curcolor}{0 0 0}
\pscustom[linewidth=2.65748024,linecolor=curcolor]
{
\newpath
\moveto(463.1724291,192.12487159)
\curveto(463.1724291,179.30397606)(452.77903973,168.91058669)(439.9581442,168.91058669)
\curveto(427.13724867,168.91058669)(416.7438593,179.30397606)(416.7438593,192.12487159)
\curveto(416.7438593,204.94576711)(427.13724867,215.33915648)(439.9581442,215.33915648)
\curveto(452.76491751,215.33915648)(463.15244789,204.96779736)(463.17240092,192.16103959)
}
}
{
\newrgbcolor{curcolor}{0 0 0}
\pscustom[linewidth=2.65748024,linecolor=curcolor]
{
\newpath
\moveto(419.1825832,79.26156914)
\lineto(440.4632242,117.51609914)
\lineto(460.5498842,79.26156914)
}
}
{
\newrgbcolor{curcolor}{0 0 0}
\pscustom[linewidth=2.65748024,linecolor=curcolor]
{
\newpath
\moveto(407.8152872,155.08045914)
\lineto(472.8152842,155.08045914)
}
}
{
\newrgbcolor{curcolor}{1 1 1}
\pscustom[linestyle=none,fillstyle=solid,fillcolor=curcolor]
{
\newpath
\moveto(158.9749,187.85983414)
\lineto(372.52114,187.85983414)
}
}
{
\newrgbcolor{curcolor}{0 0 0}
\pscustom[linewidth=2.65748024,linecolor=curcolor]
{
\newpath
\moveto(158.9749,187.85983414)
\lineto(372.52114,187.85983414)
}
}
{
\newrgbcolor{curcolor}{0 0 0}
\pscustom[linestyle=none,fillstyle=solid,fillcolor=curcolor]
{
\newpath
\moveto(172.87658488,181.42809119)
\lineto(155.45540306,187.83430119)
\lineto(172.87658584,194.24050978)
\curveto(170.09340933,190.4582868)(170.10944606,185.28354933)(172.87658488,181.42809119)
\closepath
}
}
{
\newrgbcolor{curcolor}{1 1 1}
\pscustom[linestyle=none,fillstyle=solid,fillcolor=curcolor]
{
\newpath
\moveto(374.28891,141.16527414)
\lineto(157.20713,141.16527414)
}
}
{
\newrgbcolor{curcolor}{0 0 0}
\pscustom[linewidth=2.65748024,linecolor=curcolor]
{
\newpath
\moveto(374.28891,141.16527414)
\lineto(157.20713,141.16527414)
}
}
{
\newrgbcolor{curcolor}{0 0 0}
\pscustom[linestyle=none,fillstyle=solid,fillcolor=curcolor]
{
\newpath
\moveto(360.38722512,147.59701709)
\lineto(377.80840694,141.19080709)
\lineto(360.38722416,134.7845985)
\curveto(363.17040067,138.56682148)(363.15436394,143.74155895)(360.38722512,147.59701709)
\closepath
}
}
{
\newrgbcolor{curcolor}{1 1 1}
\pscustom[linestyle=none,fillstyle=solid,fillcolor=curcolor]
{
\newpath
\moveto(157.20712,94.52176414)
\lineto(374.28891,94.52176414)
}
}
{
\newrgbcolor{curcolor}{0 0 0}
\pscustom[linewidth=2.65748024,linecolor=curcolor]
{
\newpath
\moveto(157.20712,94.52176414)
\lineto(374.28891,94.52176414)
}
}
{
\newrgbcolor{curcolor}{0 0 0}
\pscustom[linestyle=none,fillstyle=solid,fillcolor=curcolor]
{
\newpath
\moveto(171.10880488,88.09002119)
\lineto(153.68762306,94.49623119)
\lineto(171.10880584,100.90243978)
\curveto(168.32562933,97.1202168)(168.34166606,91.94547933)(171.10880488,88.09002119)
\closepath
}
}
{
\newrgbcolor{curcolor}{0 0 0}
\pscustom[linestyle=none,fillstyle=solid,fillcolor=curcolor]
{
\newpath
\moveto(360.38722512,100.95350709)
\lineto(377.80840694,94.54729709)
\lineto(360.38722416,88.1410885)
\curveto(363.17040067,91.92331148)(363.15436394,97.09804895)(360.38722512,100.95350709)
\closepath
}
}
{
\newrgbcolor{curcolor}{0 0 0}
\pscustom[linestyle=none,fillstyle=solid,fillcolor=curcolor]
{
\newpath
\moveto(228.40670776,206.9548334)
\lineto(228.40670776,194.93090762)
\lineto(229.3588562,194.93090762)
\lineto(229.3588562,206.9548334)
\lineto(232.03219604,206.9548334)
\lineto(232.03219604,207.78491152)
\lineto(229.3588562,207.78491152)
\lineto(229.3588562,210.36059512)
\curveto(229.35885295,211.0767248)(229.55009625,211.55686755)(229.93258667,211.8010248)
\curveto(230.26624137,212.0370103)(230.70162505,212.15501149)(231.23873901,212.15502871)
\curveto(231.62122048,212.15501149)(231.99556907,212.11432142)(232.36178589,212.0329584)
\lineto(232.36178589,212.85082949)
\curveto(231.99556907,212.94032971)(231.62122048,212.98508878)(231.23873901,212.98510684)
\curveto(230.44120864,212.98508878)(229.78202961,212.78977648)(229.26119995,212.39916934)
\curveto(228.69153591,211.98411322)(228.40670547,211.3290032)(228.40670776,210.4338373)
\lineto(228.40670776,207.78491152)
\lineto(226.11178589,207.78491152)
\lineto(226.11178589,206.9548334)
\lineto(228.40670776,206.9548334)
}
}
{
\newrgbcolor{curcolor}{0 0 0}
\pscustom[linestyle=none,fillstyle=solid,fillcolor=curcolor]
{
\newpath
\moveto(232.97213745,201.35180605)
\curveto(232.97213665,199.48819473)(233.48483145,197.91348927)(234.51022339,196.62768496)
\curveto(235.52747263,195.33373925)(236.9801079,194.6664222)(238.86813354,194.62573184)
\curveto(240.78055983,194.6664222)(242.2372641,195.33373925)(243.23825073,196.62768496)
\curveto(244.25549125,197.91348927)(244.76411704,199.48819473)(244.76412964,201.35180605)
\curveto(244.76411704,203.23981858)(244.25549125,204.82266205)(243.23825073,206.10034121)
\curveto(242.2372641,207.38613605)(240.78055983,208.04531508)(238.86813354,208.07788027)
\curveto(236.9801079,208.04531508)(235.52747263,207.38613605)(234.51022339,206.10034121)
\curveto(233.48483145,204.82266205)(232.97213665,203.23981858)(232.97213745,201.35180605)
\lineto(232.97213745,201.35180605)
\moveto(233.91207886,201.35180605)
\curveto(233.91207711,202.95498813)(234.34746079,204.33031228)(235.2182312,205.47778262)
\curveto(236.04830544,206.62523186)(237.26493834,207.21930679)(238.86813354,207.26000918)
\curveto(240.5038674,207.21930679)(241.73677632,206.62523186)(242.56686401,205.47778262)
\curveto(243.39693091,204.33031228)(243.81196956,202.95498813)(243.8119812,201.35180605)
\curveto(243.81196956,199.77302517)(243.39693091,198.41397705)(242.56686401,197.27465762)
\curveto(241.73677632,196.09464343)(240.5038674,195.48836149)(238.86813354,195.45580996)
\curveto(237.26493834,195.48836149)(236.04830544,196.09464343)(235.2182312,197.27465762)
\curveto(234.34746079,198.41397705)(233.91207711,199.77302517)(233.91207886,201.35180605)
\lineto(233.91207886,201.35180605)
}
}
{
\newrgbcolor{curcolor}{0 0 0}
\pscustom[linestyle=none,fillstyle=solid,fillcolor=curcolor]
{
\newpath
\moveto(247.21774292,212.7775873)
\lineto(247.21774292,194.93090762)
\lineto(248.16989136,194.93090762)
\lineto(248.16989136,212.7775873)
\lineto(247.21774292,212.7775873)
}
}
{
\newrgbcolor{curcolor}{0 0 0}
\pscustom[linestyle=none,fillstyle=solid,fillcolor=curcolor]
{
\newpath
\moveto(251.41696167,212.7775873)
\lineto(251.41696167,194.93090762)
\lineto(252.36911011,194.93090762)
\lineto(252.36911011,212.7775873)
\lineto(251.41696167,212.7775873)
}
}
{
\newrgbcolor{curcolor}{0 0 0}
\pscustom[linestyle=none,fillstyle=solid,fillcolor=curcolor]
{
\newpath
\moveto(254.79830933,201.35180605)
\curveto(254.79830852,199.48819473)(255.31100332,197.91348927)(256.33639526,196.62768496)
\curveto(257.35364451,195.33373925)(258.80627977,194.6664222)(260.69430542,194.62573184)
\curveto(262.6067317,194.6664222)(264.06343597,195.33373925)(265.06442261,196.62768496)
\curveto(266.08166312,197.91348927)(266.59028892,199.48819473)(266.59030151,201.35180605)
\curveto(266.59028892,203.23981858)(266.08166312,204.82266205)(265.06442261,206.10034121)
\curveto(264.06343597,207.38613605)(262.6067317,208.04531508)(260.69430542,208.07788027)
\curveto(258.80627977,208.04531508)(257.35364451,207.38613605)(256.33639526,206.10034121)
\curveto(255.31100332,204.82266205)(254.79830852,203.23981858)(254.79830933,201.35180605)
\lineto(254.79830933,201.35180605)
\moveto(255.73825073,201.35180605)
\curveto(255.73824899,202.95498813)(256.17363267,204.33031228)(257.04440308,205.47778262)
\curveto(257.87447732,206.62523186)(259.09111022,207.21930679)(260.69430542,207.26000918)
\curveto(262.33003927,207.21930679)(263.56294819,206.62523186)(264.39303589,205.47778262)
\curveto(265.22310278,204.33031228)(265.63814143,202.95498813)(265.63815308,201.35180605)
\curveto(265.63814143,199.77302517)(265.22310278,198.41397705)(264.39303589,197.27465762)
\curveto(263.56294819,196.09464343)(262.33003927,195.48836149)(260.69430542,195.45580996)
\curveto(259.09111022,195.48836149)(257.87447732,196.09464343)(257.04440308,197.27465762)
\curveto(256.17363267,198.41397705)(255.73824899,199.77302517)(255.73825073,201.35180605)
\lineto(255.73825073,201.35180605)
}
}
{
\newrgbcolor{curcolor}{0 0 0}
\pscustom[linestyle=none,fillstyle=solid,fillcolor=curcolor]
{
\newpath
\moveto(268.39694214,207.78491152)
\lineto(267.37155151,207.78491152)
\lineto(271.44869995,194.93090762)
\lineto(272.64498901,194.93090762)
\lineto(276.19723511,206.57641543)
\lineto(276.24606323,206.57641543)
\lineto(279.77389526,194.93090762)
\lineto(280.94577026,194.93090762)
\lineto(285.07174683,207.78491152)
\lineto(284.02194214,207.78491152)
\lineto(280.42086792,196.02954043)
\lineto(280.37203979,196.02954043)
\lineto(276.84420776,207.78491152)
\lineto(275.57467651,207.78491152)
\lineto(272.07125854,196.02954043)
\lineto(272.02243042,196.02954043)
\lineto(268.39694214,207.78491152)
}
}
{
\newrgbcolor{curcolor}{0 0 0}
\pscustom[linestyle=none,fillstyle=solid,fillcolor=curcolor]
{
\newpath
\moveto(297.19332886,201.22973574)
\curveto(297.29911088,203.15030044)(296.89221024,204.75755795)(295.97262573,206.05151309)
\curveto(295.03674335,207.36986002)(293.55969404,208.04531508)(291.54147339,208.07788027)
\curveto(289.62903391,208.04531508)(288.19674368,207.33730797)(287.24459839,205.95385684)
\curveto(286.29244871,204.58665968)(285.83672,202.97940217)(285.87741089,201.13207949)
\curveto(285.86113403,199.31729646)(286.33313877,197.79141908)(287.29342651,196.55444277)
\curveto(288.24557175,195.30932521)(289.66158596,194.6664222)(291.54147339,194.62573184)
\curveto(294.67460178,194.6664222)(296.52599967,196.21671362)(297.09567261,199.27661074)
\lineto(296.14352417,199.27661074)
\curveto(295.59826622,196.76196047)(294.06425083,195.48836149)(291.54147339,195.45580996)
\curveto(289.96269243,195.47208546)(288.78268059,196.06209138)(288.00143433,197.22582949)
\curveto(287.19576812,198.34887295)(286.8010745,199.68350703)(286.81735229,201.22973574)
\lineto(297.19332886,201.22973574)
\moveto(286.81735229,202.05981387)
\curveto(286.93942072,203.38630281)(287.39921843,204.57445266)(288.19674683,205.62426699)
\curveto(288.99426892,206.67405994)(290.10917666,207.21930679)(291.54147339,207.26000918)
\curveto(293.03072322,207.21930679)(294.18225201,206.69440497)(294.99606323,205.68530215)
\curveto(295.79357853,204.66803981)(296.20861718,203.45954492)(296.24118042,202.05981387)
\lineto(286.81735229,202.05981387)
}
}
{
\newrgbcolor{curcolor}{0 0 0}
\pscustom[linestyle=none,fillstyle=solid,fillcolor=curcolor]
{
\newpath
\moveto(300.37936401,207.78491152)
\lineto(299.43942261,207.78491152)
\lineto(299.43942261,194.93090762)
\lineto(300.37936401,194.93090762)
\lineto(300.37936401,201.94995059)
\curveto(300.43632776,202.90209105)(300.57874298,203.64671921)(300.80661011,204.1838373)
\curveto(301.02633368,204.71279888)(301.36813021,205.20514865)(301.83200073,205.66088809)
\curveto(302.43420988,206.22240023)(303.05269884,206.59674882)(303.68746948,206.78393496)
\curveto(304.3059528,206.94668336)(304.87154468,206.98737343)(305.38424683,206.90600527)
\lineto(305.38424683,207.85815371)
\curveto(304.20422764,207.89069283)(303.16256201,207.63027643)(302.25924683,207.07690371)
\curveto(301.33964717,206.51536869)(300.74150323,205.79922357)(300.46481323,204.92846621)
\lineto(300.37936401,204.92846621)
\lineto(300.37936401,207.78491152)
}
}
{
\newrgbcolor{curcolor}{0 0 0}
\pscustom[linestyle=none,fillstyle=solid,fillcolor=curcolor]
{
\newpath
\moveto(223.9694519,160.60363711)
\lineto(223.9694519,148.57971133)
\lineto(224.92160034,148.57971133)
\lineto(224.92160034,160.60363711)
\lineto(227.59494019,160.60363711)
\lineto(227.59494019,161.43371523)
\lineto(224.92160034,161.43371523)
\lineto(224.92160034,164.00939883)
\curveto(224.92159709,164.72552852)(225.11284039,165.20567126)(225.49533081,165.44982852)
\curveto(225.82898551,165.68581401)(226.26436919,165.8038152)(226.80148315,165.80383242)
\curveto(227.18396462,165.8038152)(227.55831321,165.76312513)(227.92453003,165.68176211)
\lineto(227.92453003,166.4996332)
\curveto(227.55831321,166.58913342)(227.18396462,166.63389249)(226.80148315,166.63391055)
\curveto(226.00395278,166.63389249)(225.34477375,166.43858019)(224.82394409,166.04797305)
\curveto(224.25428005,165.63291693)(223.96944961,164.97780691)(223.9694519,164.08264102)
\lineto(223.9694519,161.43371523)
\lineto(221.67453003,161.43371523)
\lineto(221.67453003,160.60363711)
\lineto(223.9694519,160.60363711)
}
}
{
\newrgbcolor{curcolor}{0 0 0}
\pscustom[linestyle=none,fillstyle=solid,fillcolor=curcolor]
{
\newpath
\moveto(228.53488159,155.00060977)
\curveto(228.53488079,153.13699844)(229.04757559,151.56229298)(230.07296753,150.27648867)
\curveto(231.09021677,148.98254296)(232.54285204,148.31522592)(234.43087769,148.27453555)
\curveto(236.34330397,148.31522592)(237.80000824,148.98254296)(238.80099487,150.27648867)
\curveto(239.81823539,151.56229298)(240.32686118,153.13699844)(240.32687378,155.00060977)
\curveto(240.32686118,156.88862229)(239.81823539,158.47146576)(238.80099487,159.74914492)
\curveto(237.80000824,161.03493976)(236.34330397,161.69411879)(234.43087769,161.72668398)
\curveto(232.54285204,161.69411879)(231.09021677,161.03493976)(230.07296753,159.74914492)
\curveto(229.04757559,158.47146576)(228.53488079,156.88862229)(228.53488159,155.00060977)
\lineto(228.53488159,155.00060977)
\moveto(229.474823,155.00060977)
\curveto(229.47482125,156.60379185)(229.91020493,157.97911599)(230.78097534,159.12658633)
\curveto(231.61104958,160.27403557)(232.82768248,160.8681105)(234.43087769,160.90881289)
\curveto(236.06661154,160.8681105)(237.29952046,160.27403557)(238.12960815,159.12658633)
\curveto(238.95967505,157.97911599)(239.3747137,156.60379185)(239.37472534,155.00060977)
\curveto(239.3747137,153.42182888)(238.95967505,152.06278076)(238.12960815,150.92346133)
\curveto(237.29952046,149.74344714)(236.06661154,149.1371652)(234.43087769,149.10461367)
\curveto(232.82768248,149.1371652)(231.61104958,149.74344714)(230.78097534,150.92346133)
\curveto(229.91020493,152.06278076)(229.47482125,153.42182888)(229.474823,155.00060977)
\lineto(229.474823,155.00060977)
}
}
{
\newrgbcolor{curcolor}{0 0 0}
\pscustom[linestyle=none,fillstyle=solid,fillcolor=curcolor]
{
\newpath
\moveto(242.78048706,166.42639102)
\lineto(242.78048706,148.57971133)
\lineto(243.7326355,148.57971133)
\lineto(243.7326355,166.42639102)
\lineto(242.78048706,166.42639102)
}
}
{
\newrgbcolor{curcolor}{0 0 0}
\pscustom[linestyle=none,fillstyle=solid,fillcolor=curcolor]
{
\newpath
\moveto(246.97970581,166.42639102)
\lineto(246.97970581,148.57971133)
\lineto(247.93185425,148.57971133)
\lineto(247.93185425,166.42639102)
\lineto(246.97970581,166.42639102)
}
}
{
\newrgbcolor{curcolor}{0 0 0}
\pscustom[linestyle=none,fillstyle=solid,fillcolor=curcolor]
{
\newpath
\moveto(250.36105347,155.00060977)
\curveto(250.36105266,153.13699844)(250.87374746,151.56229298)(251.8991394,150.27648867)
\curveto(252.91638865,148.98254296)(254.36902391,148.31522592)(256.25704956,148.27453555)
\curveto(258.16947584,148.31522592)(259.62618011,148.98254296)(260.62716675,150.27648867)
\curveto(261.64440726,151.56229298)(262.15303306,153.13699844)(262.15304565,155.00060977)
\curveto(262.15303306,156.88862229)(261.64440726,158.47146576)(260.62716675,159.74914492)
\curveto(259.62618011,161.03493976)(258.16947584,161.69411879)(256.25704956,161.72668398)
\curveto(254.36902391,161.69411879)(252.91638865,161.03493976)(251.8991394,159.74914492)
\curveto(250.87374746,158.47146576)(250.36105266,156.88862229)(250.36105347,155.00060977)
\lineto(250.36105347,155.00060977)
\moveto(251.30099487,155.00060977)
\curveto(251.30099313,156.60379185)(251.73637681,157.97911599)(252.60714722,159.12658633)
\curveto(253.43722146,160.27403557)(254.65385436,160.8681105)(256.25704956,160.90881289)
\curveto(257.89278341,160.8681105)(259.12569233,160.27403557)(259.95578003,159.12658633)
\curveto(260.78584692,157.97911599)(261.20088557,156.60379185)(261.20089722,155.00060977)
\curveto(261.20088557,153.42182888)(260.78584692,152.06278076)(259.95578003,150.92346133)
\curveto(259.12569233,149.74344714)(257.89278341,149.1371652)(256.25704956,149.10461367)
\curveto(254.65385436,149.1371652)(253.43722146,149.74344714)(252.60714722,150.92346133)
\curveto(251.73637681,152.06278076)(251.30099313,153.42182888)(251.30099487,155.00060977)
\lineto(251.30099487,155.00060977)
}
}
{
\newrgbcolor{curcolor}{0 0 0}
\pscustom[linestyle=none,fillstyle=solid,fillcolor=curcolor]
{
\newpath
\moveto(263.95968628,161.43371523)
\lineto(262.93429565,161.43371523)
\lineto(267.01144409,148.57971133)
\lineto(268.20773315,148.57971133)
\lineto(271.75997925,160.22521914)
\lineto(271.80880737,160.22521914)
\lineto(275.3366394,148.57971133)
\lineto(276.5085144,148.57971133)
\lineto(280.63449097,161.43371523)
\lineto(279.58468628,161.43371523)
\lineto(275.98361206,149.67834414)
\lineto(275.93478394,149.67834414)
\lineto(272.4069519,161.43371523)
\lineto(271.13742065,161.43371523)
\lineto(267.63400269,149.67834414)
\lineto(267.58517456,149.67834414)
\lineto(263.95968628,161.43371523)
}
}
{
\newrgbcolor{curcolor}{0 0 0}
\pscustom[linestyle=none,fillstyle=solid,fillcolor=curcolor]
{
\newpath
\moveto(283.1857605,148.57971133)
\lineto(283.1857605,161.43371523)
\lineto(282.23361206,161.43371523)
\lineto(282.23361206,148.57971133)
\lineto(283.1857605,148.57971133)
\moveto(283.1857605,163.85070742)
\lineto(283.1857605,166.42639102)
\lineto(282.23361206,166.42639102)
\lineto(282.23361206,163.85070742)
\lineto(283.1857605,163.85070742)
}
}
{
\newrgbcolor{curcolor}{0 0 0}
\pscustom[linestyle=none,fillstyle=solid,fillcolor=curcolor]
{
\newpath
\moveto(286.21310425,148.57971133)
\lineto(287.15304565,148.57971133)
\lineto(287.15304565,155.52551211)
\curveto(287.15304331,157.12869366)(287.57215096,158.42263768)(288.41036987,159.40734805)
\curveto(289.22416754,160.39203676)(290.34721329,160.89252454)(291.7795105,160.90881289)
\curveto(292.65027089,160.89252454)(293.34200197,160.7419713)(293.85470581,160.45715273)
\curveto(294.35925355,160.17231041)(294.71732611,159.79796183)(294.92892456,159.33410586)
\curveto(295.16491681,158.87836639)(295.31547004,158.41856868)(295.38058472,157.95471133)
\curveto(295.4131262,157.4745592)(295.42940222,157.05952056)(295.42941284,156.70959414)
\lineto(295.42941284,148.57971133)
\lineto(296.38156128,148.57971133)
\lineto(296.38156128,156.52648867)
\curveto(296.38154971,156.94151937)(296.36527368,157.44200715)(296.33273315,158.02795352)
\curveto(296.26761753,158.59760496)(296.09265026,159.16319684)(295.80783081,159.72473086)
\curveto(295.53926539,160.2943806)(295.08760569,160.77045435)(294.45285034,161.15295352)
\curveto(293.82621372,161.51915151)(292.94323934,161.71039481)(291.80392456,161.72668398)
\curveto(290.78666598,161.72667084)(289.85486353,161.47032344)(289.0085144,160.95764102)
\curveto(288.16215688,160.4205198)(287.56401295,159.65961561)(287.21408081,158.67492617)
\lineto(287.15304565,158.67492617)
\lineto(287.15304565,161.43371523)
\lineto(286.21310425,161.43371523)
\lineto(286.21310425,148.57971133)
}
}
{
\newrgbcolor{curcolor}{0 0 0}
\pscustom[linestyle=none,fillstyle=solid,fillcolor=curcolor]
{
\newpath
\moveto(304.16964722,160.90881289)
\curveto(305.75655332,160.8681105)(306.93249615,160.31472563)(307.69747925,159.24865664)
\curveto(308.46244254,158.18256631)(308.84492914,156.90082931)(308.84494019,155.4034418)
\curveto(308.84492914,153.90604064)(308.46244254,152.63244165)(307.69747925,151.58264102)
\curveto(306.93249615,150.46773027)(305.75655332,149.89806938)(304.16964722,149.87365664)
\curveto(302.67224651,149.89806938)(301.53292473,150.45145425)(300.75167847,151.53381289)
\curveto(299.96228828,152.58361357)(299.56759467,153.87348859)(299.56759644,155.4034418)
\curveto(299.56759467,156.83572521)(299.95415027,158.10118618)(300.7272644,159.19982852)
\curveto(301.47595864,160.29844961)(302.62341843,160.8681105)(304.16964722,160.90881289)
\lineto(304.16964722,160.90881289)
\moveto(308.86935425,161.43371523)
\lineto(308.86935425,158.74816836)
\lineto(308.82052612,158.74816836)
\curveto(308.45430453,159.73285773)(307.84802258,160.47748589)(307.00167847,160.98205508)
\curveto(306.14717793,161.47846145)(305.20316846,161.72667084)(304.16964722,161.72668398)
\curveto(302.37114004,161.69411879)(301.00395391,161.07156082)(300.06808472,159.8590082)
\curveto(299.10779695,158.65457104)(298.6276542,157.16938373)(298.62765503,155.4034418)
\curveto(298.6276542,153.58865814)(299.08338291,152.0871948)(299.99484253,150.89904727)
\curveto(300.91443577,149.68648105)(302.30603594,149.0720611)(304.16964722,149.05578555)
\curveto(306.35062825,149.0720611)(307.90091967,150.04862262)(308.82052612,151.98547305)
\lineto(308.86935425,151.98547305)
\lineto(308.86935425,149.03137148)
\curveto(308.86934318,148.81164469)(308.85306715,148.46984816)(308.82052612,148.00598086)
\curveto(308.755411,147.52583868)(308.58044373,147.00907488)(308.29562378,146.45568789)
\curveto(308.01078284,145.9348572)(307.55098512,145.49133551)(306.91622925,145.12512148)
\curveto(306.28959315,144.72636232)(305.41475679,144.51884299)(304.29171753,144.50256289)
\curveto(303.20935535,144.50256697)(302.29382892,144.75484536)(301.5451355,145.25939883)
\curveto(300.78015856,145.75582092)(300.33256786,146.56148418)(300.20236206,147.67639102)
\lineto(299.25021362,147.67639102)
\curveto(299.34786832,146.22782566)(299.86463213,145.19429805)(300.80050659,144.57580508)
\curveto(301.71196101,143.98173416)(302.86755881,143.68469669)(304.26730347,143.6846918)
\curveto(305.65075916,143.68469669)(306.72904584,143.91663005)(307.50216675,144.38049258)
\curveto(308.26713024,144.82808748)(308.81644609,145.38554135)(309.15011597,146.05285586)
\curveto(309.48376313,146.72017543)(309.68314445,147.35494042)(309.7482605,147.95715273)
\curveto(309.79707662,148.5593663)(309.82149066,148.99068097)(309.82150269,149.25109805)
\lineto(309.82150269,161.43371523)
\lineto(308.86935425,161.43371523)
}
}
{
\newrgbcolor{curcolor}{0 0 0}
\pscustom[linestyle=none,fillstyle=solid,fillcolor=curcolor]
{
\newpath
\moveto(240.32687378,111.81329287)
\lineto(240.32687378,99.78936709)
\lineto(241.27902222,99.78936709)
\lineto(241.27902222,111.81329287)
\lineto(243.95236206,111.81329287)
\lineto(243.95236206,112.643371)
\lineto(241.27902222,112.643371)
\lineto(241.27902222,115.21905459)
\curveto(241.27901897,115.93518428)(241.47026227,116.41532703)(241.85275269,116.65948428)
\curveto(242.18640739,116.89546978)(242.62179106,117.01347096)(243.15890503,117.01348818)
\curveto(243.5413865,117.01347096)(243.91573508,116.9727809)(244.2819519,116.89141787)
\lineto(244.2819519,117.70928896)
\curveto(243.91573508,117.79878918)(243.5413865,117.84354825)(243.15890503,117.84356631)
\curveto(242.36137466,117.84354825)(241.70219563,117.64823595)(241.18136597,117.25762881)
\curveto(240.61170193,116.84257269)(240.32687148,116.18746267)(240.32687378,115.29229678)
\lineto(240.32687378,112.643371)
\lineto(238.0319519,112.643371)
\lineto(238.0319519,111.81329287)
\lineto(240.32687378,111.81329287)
}
}
{
\newrgbcolor{curcolor}{0 0 0}
\pscustom[linestyle=none,fillstyle=solid,fillcolor=curcolor]
{
\newpath
\moveto(246.4303894,112.643371)
\lineto(245.490448,112.643371)
\lineto(245.490448,99.78936709)
\lineto(246.4303894,99.78936709)
\lineto(246.4303894,106.80841006)
\curveto(246.48735315,107.76055052)(246.62976837,108.50517869)(246.8576355,109.04229678)
\curveto(247.07735907,109.57125835)(247.4191556,110.06360812)(247.88302612,110.51934756)
\curveto(248.48523527,111.0808597)(249.10372423,111.45520829)(249.73849487,111.64239443)
\curveto(250.35697819,111.80514283)(250.92257007,111.8458329)(251.43527222,111.76446475)
\lineto(251.43527222,112.71661318)
\curveto(250.25525303,112.74915231)(249.2135874,112.4887359)(248.31027222,111.93536318)
\curveto(247.39067256,111.37382816)(246.79252863,110.65768304)(246.51583862,109.78692568)
\lineto(246.4303894,109.78692568)
\lineto(246.4303894,112.643371)
}
}
{
\newrgbcolor{curcolor}{0 0 0}
\pscustom[linestyle=none,fillstyle=solid,fillcolor=curcolor]
{
\newpath
\moveto(253.59591675,99.78936709)
\lineto(253.59591675,112.643371)
\lineto(252.64376831,112.643371)
\lineto(252.64376831,99.78936709)
\lineto(253.59591675,99.78936709)
\moveto(253.59591675,115.06036318)
\lineto(253.59591675,117.63604678)
\lineto(252.64376831,117.63604678)
\lineto(252.64376831,115.06036318)
\lineto(253.59591675,115.06036318)
}
}
{
\newrgbcolor{curcolor}{0 0 0}
\pscustom[linestyle=none,fillstyle=solid,fillcolor=curcolor]
{
\newpath
\moveto(267.365448,106.08819521)
\curveto(267.47123002,108.00875991)(267.06432938,109.61601742)(266.14474487,110.90997256)
\curveto(265.20886249,112.22831949)(263.73181318,112.90377455)(261.71359253,112.93633975)
\curveto(259.80115305,112.90377455)(258.36886282,112.19576744)(257.41671753,110.81231631)
\curveto(256.46456785,109.44511915)(256.00883914,107.83786165)(256.04953003,105.99053896)
\curveto(256.03325317,104.17575593)(256.50525791,102.64987855)(257.46554565,101.41290225)
\curveto(258.41769089,100.16778468)(259.8337051,99.52488168)(261.71359253,99.48419131)
\curveto(264.84672092,99.52488168)(266.69811881,101.0751731)(267.26779175,104.13507021)
\lineto(266.31564331,104.13507021)
\curveto(265.77038536,101.62041995)(264.23636997,100.34682096)(261.71359253,100.31426943)
\curveto(260.13481157,100.33054493)(258.95479973,100.92055085)(258.17355347,102.08428896)
\curveto(257.36788726,103.20733242)(256.97319364,104.5419665)(256.98947144,106.08819521)
\lineto(267.365448,106.08819521)
\moveto(256.98947144,106.91827334)
\curveto(257.11153986,108.24476228)(257.57133757,109.43291213)(258.36886597,110.48272646)
\curveto(259.16638806,111.53251941)(260.2812958,112.07776626)(261.71359253,112.11846865)
\curveto(263.20284236,112.07776626)(264.35437115,111.55286444)(265.16818237,110.54376162)
\curveto(265.96569767,109.52649928)(266.38073632,108.31800439)(266.41329956,106.91827334)
\lineto(256.98947144,106.91827334)
}
}
{
\newrgbcolor{curcolor}{0 0 0}
\pscustom[linestyle=none,fillstyle=solid,fillcolor=curcolor]
{
\newpath
\moveto(269.61154175,99.78936709)
\lineto(270.55148315,99.78936709)
\lineto(270.55148315,106.73516787)
\curveto(270.55148081,108.33834943)(270.97058846,109.63229344)(271.80880737,110.61700381)
\curveto(272.62260504,111.60169252)(273.74565079,112.1021803)(275.177948,112.11846865)
\curveto(276.04870839,112.1021803)(276.74043947,111.95162706)(277.25314331,111.6668085)
\curveto(277.75769105,111.38196617)(278.11576361,111.00761759)(278.32736206,110.54376162)
\curveto(278.56335431,110.08802216)(278.71390754,109.62822444)(278.77902222,109.16436709)
\curveto(278.8115637,108.68421497)(278.82783972,108.26917632)(278.82785034,107.9192499)
\lineto(278.82785034,99.78936709)
\lineto(279.77999878,99.78936709)
\lineto(279.77999878,107.73614443)
\curveto(279.77998721,108.15117513)(279.76371118,108.65166292)(279.73117065,109.23760928)
\curveto(279.66605503,109.80726072)(279.49108776,110.3728526)(279.20626831,110.93438662)
\curveto(278.93770289,111.50403636)(278.48604319,111.98011011)(277.85128784,112.36260928)
\curveto(277.22465122,112.72880728)(276.34167684,112.92005057)(275.20236206,112.93633975)
\curveto(274.18510348,112.9363266)(273.25330103,112.6799792)(272.4069519,112.16729678)
\curveto(271.56059438,111.63017556)(270.96245045,110.86927137)(270.61251831,109.88458193)
\lineto(270.55148315,109.88458193)
\lineto(270.55148315,112.643371)
\lineto(269.61154175,112.643371)
\lineto(269.61154175,99.78936709)
}
}
{
\newrgbcolor{curcolor}{0 0 0}
\pscustom[linestyle=none,fillstyle=solid,fillcolor=curcolor]
{
\newpath
\moveto(287.75119019,100.31426943)
\curveto(286.09916705,100.34682096)(284.89067217,100.95310291)(284.1257019,102.13311709)
\curveto(283.33631174,103.27243652)(282.94161813,104.62334663)(282.94161987,106.18585146)
\curveto(282.9253421,107.80530959)(283.30375969,109.18877175)(284.07687378,110.33624209)
\curveto(284.82556807,111.48369133)(286.03813196,112.07776626)(287.71456909,112.11846865)
\curveto(289.36657915,112.07776626)(290.58321205,111.46741531)(291.36447144,110.28741396)
\curveto(292.12129645,109.12366765)(292.49971404,107.75648152)(292.49972534,106.18585146)
\curveto(292.49971404,104.65589868)(292.11315844,103.31719559)(291.34005737,102.16973818)
\curveto(290.57507404,100.96530992)(289.37878617,100.34682096)(287.75119019,100.31426943)
\lineto(287.75119019,100.31426943)
\moveto(292.5241394,99.78936709)
\lineto(293.46408081,99.78936709)
\lineto(293.46408081,117.63604678)
\lineto(292.5241394,117.63604678)
\lineto(292.5241394,109.76251162)
\lineto(292.47531128,109.76251162)
\curveto(292.32067776,110.28333446)(292.08467539,110.74313218)(291.76730347,111.14190615)
\curveto(291.4499104,111.54065742)(291.07556182,111.87431594)(290.64425659,112.14288271)
\curveto(289.7816178,112.67184119)(288.80505627,112.9363266)(287.71456909,112.93633975)
\curveto(285.80212959,112.92005057)(284.36983936,112.27714757)(283.41769409,111.00762881)
\curveto(282.4736824,109.74622562)(282.00167766,108.13896812)(282.00167847,106.18585146)
\curveto(281.97726362,104.35479221)(282.42485432,102.79636278)(283.3444519,101.5105585)
\curveto(284.23149314,100.20033673)(285.63123132,99.52488168)(287.54367065,99.48419131)
\curveto(289.81416985,99.48419161)(291.45804841,100.5014432)(292.47531128,102.53594912)
\lineto(292.5241394,102.53594912)
\lineto(292.5241394,99.78936709)
}
}
\end{pspicture}

\caption{Tipos de relacionamientos entre usuarios.}
\label{contactos}
\end{figure}

Estos recursos, son los componentes propios de un sitio web, además de darle las
características de una red social propiamente dicha.

\subsection{Fomento a la participación}

Una parte fundamental del sistema, y el factor clave para el éxito de toda red
social, son sus políticas que propician la cultura de participación.

Estos elementos, que están inspirados en la tendencia de los sitios considerados
dentro de la web 2.0\footnote{Definición disponible en:
http://es.wikipedia.org/wiki/Web\_2.0}, han sido considerados como base para el
establecimiento de las definiciones siguientes:

\begin{description}
\item [Comentarios] Los comentarios crean el espacio de debate entre usuarios,
estos se encuentran en cada tipo de recursos que es provisto por el sistema.
\item [Valoraciones] Una valoración es un voto a favor o en contra de un recurso
determinado, y define la calidad misma del recurso.
\item [Etiquetado] Las etiquetas\footnote{Puede verse la definición extendida
en: http://es.wikipedia.org/wiki/Etiqueta\_(metadato)} son palabras clave que
son asignadas a un recurso, estas son de tipo informal (es decir, definidas por
el creador del recurso), y sirven como un medio alternativo de clasificación de
los recursos conocido como folcsonomías\footnote{Puede verse la definición
extendida en http://es.wikipedia.org/wiki/Folcsonomía}.
\item [Sistema de reputación] Un sistema de reputación\footnote{Si bien vamos a
ahondar en este concepto, pueden verse los detalles introductorios en:
http://en.wikipedia.org/wiki/Reputation\_system} define un conjunto de políticas
de fomento a la interacción o participación de los usuarios, entre los múltiples
métodos que pueden encontrarse en los sitios web actuales, se han definido
cuatro indicadores a ser tomados en cuenta:
\begin{description}
    \item [Actividad] El indicador de actividad, se basa en el numero de
    recursos que un usuario ha creado en el sistema.
    \item [Participación] El indicador de participación mide el numero de
    comentarios creados por el usuarios en los recursos del sistema.
    \item [Popularidad] El indicador de popularidad mide el apoyo de los
    usuarios hacia la calidad de los recursos que crea el usuario, es media a
    partir de las valoraciones realizadas en el recurso.
    \item [Sociabilidad] El indicador de sociabilidad mide el numero de
    conexiones (contactos) que posee un usuario.
    \end{description}
\end{description}

Estos elementos deben estar disponibles para cualquiera de los recursos
intercambiables definidos anteriormente.

\section{Requisitos no funcionales}

Un requisito no funcional es aquel que especifica criterios que pueden usarse
para juzgar la operación de un sistema en lugar de sus comportamientos
específicos\footnote{Para una definición exacta puede consultarse:
http://es.wikipedia.org/wiki/Requisito\_no\_funcional}. Para la construcción de
esta aplicación web, se han definido un conjunto de herramientas, estas se
describen en esta sección.

\subsection{Contexto de despliegue}

Se ha de construir el sistema, considerando a disposición del desarrollo un
servidor que se ejecute en un sistema operativo GNU/Linux, esto implica varias
consideraciones; como la disposición de un amplio conjunto de
herramientas disponibles para tareas de automatización y scripting.

\subsection{Servidor web}

Se ha determinado que el sitio web, puede ser ejecutado en el servidor web mas
popular que existe, además de en el segundo, estos son:

\begin{itemize}
\item Apache Web Server 2
\item Nginx\footnote{Pueden apreciarse sus características en:
http://es.wikipedia.org/wiki/Nginx}
\end{itemize}

Se crearan las pruebas necesarias para asegurar el correcto funcionamiento del
sistema en estos dos servidores HTTP.

\subsection{Base de datos}

Se ha determinado usar el DBMS mas popular: MySQL, además de probar el sistema
en la versión alternativa de este: MariaDB.

\subsection{Lenguaje de programación}

Para la creación del sistema, se ha optado por el uso del lenguaje PHP 5, con la
utilización de las librerías, estándares, y conceptos  que componen el marco
de trabajo denominado Zend Framework.

Para la implementación con Zend Framework, se ha establecido un conjunto de
módulos a ser desarrollados, estos se detallan en la sección siguiente.

\section{Diseño de paquetes}

Para la construcción del sistema se han determinado un conjunto de módulos ha
ser desarrollados, estos se detallan en la figura \ref{paquetes}.

\begin{figure}
\centering
%LaTeX with PSTricks extensions
%%Creator: inkscape 0.48.4
%%Please note this file requires PSTricks extensions
\psset{xunit=.5pt,yunit=.5pt,runit=.5pt}
\begin{pspicture}(779.52752686,921.25982666)
{
\newrgbcolor{curcolor}{0.75294119 0.75294119 0.75294119}
\pscustom[linewidth=2.32802935,linecolor=curcolor,linestyle=dashed,dash=3 2]
{
\newpath
\moveto(205.84945068,789.46405088)
\lineto(354.84332457,848.82880027)
}
}
{
\newrgbcolor{curcolor}{0.75294119 0.75294119 0.75294119}
\pscustom[linewidth=2.32802935,linecolor=curcolor]
{
\newpath
\moveto(342.03916659,839.51668199)
\lineto(354.84332457,848.82880027)
\lineto(338.54711757,847.66478083)
}
}
{
\newrgbcolor{curcolor}{0.75294119 0.75294119 0.75294119}
\pscustom[linewidth=2.32802935,linecolor=curcolor,linestyle=dashed,dash=3 2]
{
\newpath
\moveto(394.41982727,798.77616916)
\lineto(394.41982727,838.35267186)
}
}
{
\newrgbcolor{curcolor}{0.75294119 0.75294119 0.75294119}
\pscustom[linewidth=2.32802935,linecolor=curcolor]
{
\newpath
\moveto(399.07588641,823.22047499)
\lineto(394.41982727,838.35267186)
\lineto(389.76376813,823.22047499)
}
}
{
\newrgbcolor{curcolor}{0.75294119 0.75294119 0.75294119}
\pscustom[linewidth=2.32802935,linecolor=curcolor,linestyle=dashed,dash=3 2]
{
\newpath
\moveto(169.76498768,707.98302522)
\lineto(169.76498768,747.55952792)
}
}
{
\newrgbcolor{curcolor}{0.75294119 0.75294119 0.75294119}
\pscustom[linewidth=2.32802935,linecolor=curcolor]
{
\newpath
\moveto(174.42104682,731.26332093)
\lineto(169.76498768,747.55952792)
\lineto(165.10892854,731.26332093)
}
}
{
\newrgbcolor{curcolor}{0.75294119 0.75294119 0.75294119}
\pscustom[linewidth=2.32802935,linecolor=curcolor,linestyle=dashed,dash=3 2]
{
\newpath
\moveto(351.35128487,699.83492638)
\lineto(209.34149038,758.03565633)
}
}
{
\newrgbcolor{curcolor}{0.75294119 0.75294119 0.75294119}
\pscustom[linewidth=2.32802935,linecolor=curcolor]
{
\newpath
\moveto(225.63769737,756.87163689)
\lineto(209.34149038,758.03565633)
\lineto(222.14564836,748.72353805)
}
}
{
\newrgbcolor{curcolor}{0.75294119 0.75294119 0.75294119}
\pscustom[linewidth=2.32802935,linecolor=curcolor,linestyle=dashed,dash=3 2]
{
\newpath
\moveto(168.60097755,617.18988128)
\lineto(168.60097755,655.60236454)
}
}
{
\newrgbcolor{curcolor}{0.75294119 0.75294119 0.75294119}
\pscustom[linewidth=2.32802935,linecolor=curcolor]
{
\newpath
\moveto(173.25703669,640.47017699)
\lineto(168.60097755,655.60236454)
\lineto(163.94491841,640.47017699)
}
}
{
\newrgbcolor{curcolor}{0.75294119 0.75294119 0.75294119}
\pscustom[linewidth=2.32802935,linecolor=curcolor,linestyle=dashed,dash=3 2]
{
\newpath
\moveto(167.43695811,526.39673734)
\lineto(167.43695811,564.8092206)
}
}
{
\newrgbcolor{curcolor}{0.75294119 0.75294119 0.75294119}
\pscustom[linewidth=2.32802935,linecolor=curcolor]
{
\newpath
\moveto(172.09301725,549.67703305)
\lineto(167.43695811,564.8092206)
\lineto(162.78090828,549.67703305)
}
}
{
\newrgbcolor{curcolor}{0.75294119 0.75294119 0.75294119}
\pscustom[linewidth=2.32802935,linecolor=curcolor,linestyle=dashed,dash=3 2]
{
\newpath
\moveto(165.10892854,434.43957396)
\lineto(166.27294798,474.01607666)
}
}
{
\newrgbcolor{curcolor}{0.75294119 0.75294119 0.75294119}
\pscustom[linewidth=2.32802935,linecolor=curcolor]
{
\newpath
\moveto(169.76498768,457.71986967)
\lineto(166.27294798,474.01607666)
\lineto(160.45287871,458.88388911)
}
}
{
\newrgbcolor{curcolor}{0.75294119 0.75294119 0.75294119}
\pscustom[linewidth=2.32802935,linecolor=curcolor,linestyle=dashed,dash=3 2]
{
\newpath
\moveto(392.0917977,525.2327179)
\lineto(392.0917977,655.60236454)
}
}
{
\newrgbcolor{curcolor}{0.75294119 0.75294119 0.75294119}
\pscustom[linewidth=2.32802935,linecolor=curcolor]
{
\newpath
\moveto(396.74785684,640.47017699)
\lineto(392.0917977,655.60236454)
\lineto(387.43573856,640.47017699)
}
}
{
\newrgbcolor{curcolor}{0.75294119 0.75294119 0.75294119}
\pscustom[linewidth=2.32802935,linecolor=curcolor,linestyle=dashed,dash=3 2]
{
\newpath
\moveto(457.27661636,443.75169224)
\lineto(422.3561728,475.18008679)
}
}
{
\newrgbcolor{curcolor}{0.75294119 0.75294119 0.75294119}
\pscustom[linewidth=2.32802935,linecolor=curcolor]
{
\newpath
\moveto(437.48836967,468.19599808)
\lineto(422.3561728,475.18008679)
\lineto(430.50428096,461.21191868)
}
}
{
\newrgbcolor{curcolor}{0.75294119 0.75294119 0.75294119}
\pscustom[linewidth=2.32802935,linecolor=curcolor,linestyle=dashed,dash=3 2]
{
\newpath
\moveto(312.9387923,434.43957396)
\lineto(360.66339384,475.18008679)
}
}
{
\newrgbcolor{curcolor}{0.75294119 0.75294119 0.75294119}
\pscustom[linewidth=2.32802935,linecolor=curcolor]
{
\newpath
\moveto(352.515295,461.21191868)
\lineto(360.66339384,475.18008679)
\lineto(345.53120629,468.19599808)
}
}
{
\newrgbcolor{curcolor}{0.75294119 0.75294119 0.75294119}
\pscustom[linewidth=2.32802935,linecolor=curcolor,linestyle=dashed,dash=3 2]
{
\newpath
\moveto(580.66217429,425.12745568)
\lineto(432.83231053,484.49220507)
}
}
{
\newrgbcolor{curcolor}{0.75294119 0.75294119 0.75294119}
\pscustom[linewidth=2.32802935,linecolor=curcolor]
{
\newpath
\moveto(449.12851752,483.32819495)
\lineto(432.83231053,484.49220507)
\lineto(445.63646851,474.01607666)
}
}
{
\newrgbcolor{curcolor}{0.75294119 0.75294119 0.75294119}
\pscustom[linewidth=2.32802935,linecolor=curcolor,linestyle=dashed,dash=3 2]
{
\newpath
\moveto(94.10404061,333.17030161)
\lineto(237.27784523,390.20702144)
}
}
{
\newrgbcolor{curcolor}{0.75294119 0.75294119 0.75294119}
\pscustom[linewidth=2.32802935,linecolor=curcolor]
{
\newpath
\moveto(224.47367793,380.89490315)
\lineto(237.27784523,390.20702144)
\lineto(220.98163823,389.04300199)
}
}
{
\newrgbcolor{curcolor}{0.75294119 0.75294119 0.75294119}
\pscustom[linewidth=2.32802935,linecolor=curcolor,linestyle=dashed,dash=3 2]
{
\newpath
\moveto(197.70134253,343.64643002)
\lineto(248.91799308,384.38694285)
}
}
{
\newrgbcolor{curcolor}{0.75294119 0.75294119 0.75294119}
\pscustom[linewidth=2.32802935,linecolor=curcolor]
{
\newpath
\moveto(239.6058748,370.41876543)
\lineto(248.91799308,384.38694285)
\lineto(233.78579621,377.40285414)
}
}
{
\newrgbcolor{curcolor}{0.75294119 0.75294119 0.75294119}
\pscustom[linewidth=2.32802935,linecolor=curcolor,linestyle=dashed,dash=3 2]
{
\newpath
\moveto(279.18236818,343.64643002)
\lineto(279.18236818,382.05891328)
}
}
{
\newrgbcolor{curcolor}{0.75294119 0.75294119 0.75294119}
\pscustom[linewidth=2.32802935,linecolor=curcolor]
{
\newpath
\moveto(283.83842732,366.92672573)
\lineto(279.18236818,382.05891328)
\lineto(274.52630904,366.92672573)
}
}
{
\newrgbcolor{curcolor}{0.75294119 0.75294119 0.75294119}
\pscustom[linewidth=2.32802935,linecolor=curcolor,linestyle=dashed,dash=3 2]
{
\newpath
\moveto(351.35128487,350.63051874)
\lineto(310.61076273,384.38694285)
}
}
{
\newrgbcolor{curcolor}{0.75294119 0.75294119 0.75294119}
\pscustom[linewidth=2.32802935,linecolor=curcolor]
{
\newpath
\moveto(325.74295959,377.40285414)
\lineto(310.61076273,384.38694285)
\lineto(319.92288101,370.41876543)
}
}
{
\newrgbcolor{curcolor}{0.75294119 0.75294119 0.75294119}
\pscustom[linewidth=2.32802935,linecolor=curcolor,linestyle=dashed,dash=3 2]
{
\newpath
\moveto(463.09668563,333.17030161)
\lineto(321.08690045,392.53505101)
}
}
{
\newrgbcolor{curcolor}{0.75294119 0.75294119 0.75294119}
\pscustom[linewidth=2.32802935,linecolor=curcolor]
{
\newpath
\moveto(337.38310745,390.20702144)
\lineto(321.08690045,392.53505101)
\lineto(333.89105843,382.05891328)
}
}
{
\newrgbcolor{curcolor}{0.75294119 0.75294119 0.75294119}
\pscustom[linewidth=2.32802935,linecolor=curcolor,linestyle=dashed,dash=3 2]
{
\newpath
\moveto(84.79192046,252.85328609)
\lineto(134.84455344,292.42978879)
}
}
{
\newrgbcolor{curcolor}{0.75294119 0.75294119 0.75294119}
\pscustom[linewidth=2.32802935,linecolor=curcolor]
{
\newpath
\moveto(125.53243515,279.62562149)
\lineto(134.84455344,292.42978879)
\lineto(119.71235657,286.6097102)
}
}
{
\newrgbcolor{curcolor}{0.75294119 0.75294119 0.75294119}
\pscustom[linewidth=2.32802935,linecolor=curcolor,linestyle=dashed,dash=3 2]
{
\newpath
\moveto(73.15177354,160.89613202)
\lineto(152.30477056,292.42978879)
}
}
{
\newrgbcolor{curcolor}{0.75294119 0.75294119 0.75294119}
\pscustom[linewidth=2.32802935,linecolor=curcolor]
{
\newpath
\moveto(147.64871142,276.13358179)
\lineto(152.30477056,292.42978879)
\lineto(139.50061258,280.78964093)
}
}
{
\newrgbcolor{curcolor}{0.75294119 0.75294119 0.75294119}
\pscustom[linewidth=2.32802935,linecolor=curcolor,linestyle=dashed,dash=3 2]
{
\newpath
\moveto(167.43695811,160.89613202)
\lineto(167.43695811,291.26576934)
}
}
{
\newrgbcolor{curcolor}{0.75294119 0.75294119 0.75294119}
\pscustom[linewidth=2.32802935,linecolor=curcolor]
{
\newpath
\moveto(172.09301725,276.13358179)
\lineto(167.43695811,291.26576934)
\lineto(162.78090828,276.13358179)
}
}
{
\newrgbcolor{curcolor}{0.75294119 0.75294119 0.75294119}
\pscustom[linewidth=2.32802935,linecolor=curcolor,linestyle=dashed,dash=3 2]
{
\newpath
\moveto(240.76988492,261.00139424)
\lineto(203.52142111,292.42978879)
}
}
{
\newrgbcolor{curcolor}{0.75294119 0.75294119 0.75294119}
\pscustom[linewidth=2.32802935,linecolor=curcolor]
{
\newpath
\moveto(217.48958922,286.6097102)
\lineto(203.52142111,292.42978879)
\lineto(211.66951995,279.62562149)
}
}
{
\newrgbcolor{curcolor}{0.75294119 0.75294119 0.75294119}
\pscustom[linewidth=2.32802935,linecolor=curcolor,linestyle=dashed,dash=3 2]
{
\newpath
\moveto(88.28396482,70.10298808)
\lineto(136.00856356,110.84350091)
}
}
{
\newrgbcolor{curcolor}{0.75294119 0.75294119 0.75294119}
\pscustom[linewidth=2.32802935,linecolor=curcolor]
{
\newpath
\moveto(126.69644528,96.87532349)
\lineto(136.00856356,110.84350091)
\lineto(120.87637601,103.8594122)
}
}
{
\newrgbcolor{curcolor}{0.75294119 0.75294119 0.75294119}
\pscustom[linewidth=2.32802935,linecolor=curcolor,linestyle=dashed,dash=3 2]
{
\newpath
\moveto(240.76988492,74.75903791)
\lineto(196.5373324,110.84350091)
}
}
{
\newrgbcolor{curcolor}{0.75294119 0.75294119 0.75294119}
\pscustom[linewidth=2.32802935,linecolor=curcolor]
{
\newpath
\moveto(211.66951995,103.8594122)
\lineto(196.5373324,110.84350091)
\lineto(205.84945068,96.87532349)
}
}
{
\newrgbcolor{curcolor}{0.75294119 0.75294119 0.75294119}
\pscustom[linewidth=2.32802935,linecolor=curcolor,linestyle=dashed,dash=3 2]
{
\newpath
\moveto(389.76376813,252.85328609)
\lineto(389.76376813,291.26576934)
}
}
{
\newrgbcolor{curcolor}{0.75294119 0.75294119 0.75294119}
\pscustom[linewidth=2.32802935,linecolor=curcolor]
{
\newpath
\moveto(394.41982727,276.13358179)
\lineto(389.76376813,291.26576934)
\lineto(385.10770899,276.13358179)
}
}
{
\newrgbcolor{curcolor}{0.75294119 0.75294119 0.75294119}
\pscustom[linewidth=2.32802935,linecolor=curcolor,linestyle=dashed,dash=3 2]
{
\newpath
\moveto(619.07465755,343.64643002)
\lineto(619.07465755,382.05891328)
}
}
{
\newrgbcolor{curcolor}{0.75294119 0.75294119 0.75294119}
\pscustom[linewidth=2.32802935,linecolor=curcolor]
{
\newpath
\moveto(623.73071669,366.92672573)
\lineto(619.07465755,382.05891328)
\lineto(614.4185984,366.92672573)
}
}
{
\newrgbcolor{curcolor}{0.75294119 0.75294119 0.75294119}
\pscustom[linewidth=2.32802935,linecolor=curcolor,linestyle=dashed,dash=3 2]
{
\newpath
\moveto(619.07465755,252.85328609)
\lineto(619.07465755,291.26576934)
}
}
{
\newrgbcolor{curcolor}{0.75294119 0.75294119 0.75294119}
\pscustom[linewidth=2.32802935,linecolor=curcolor]
{
\newpath
\moveto(623.73071669,276.13358179)
\lineto(619.07465755,291.26576934)
\lineto(614.4185984,276.13358179)
}
}
{
\newrgbcolor{curcolor}{0.75294119 0.75294119 0.75294119}
\pscustom[linewidth=2.32802935,linecolor=curcolor,linestyle=dashed,dash=3 2]
{
\newpath
\moveto(532.93757275,252.85328609)
\lineto(582.99020386,292.42978879)
}
}
{
\newrgbcolor{curcolor}{0.75294119 0.75294119 0.75294119}
\pscustom[linewidth=2.32802935,linecolor=curcolor]
{
\newpath
\moveto(573.67808558,279.62562149)
\lineto(582.99020386,292.42978879)
\lineto(567.85800699,286.6097102)
}
}
{
\newrgbcolor{curcolor}{0.75294119 0.75294119 0.75294119}
\pscustom[linewidth=2.32802935,linecolor=curcolor,linestyle=dashed,dash=3 2]
{
\newpath
\moveto(685.42349565,262.16540437)
\lineto(651.66706222,292.42978879)
}
}
{
\newrgbcolor{curcolor}{0.75294119 0.75294119 0.75294119}
\pscustom[linewidth=2.32802935,linecolor=curcolor]
{
\newpath
\moveto(665.63523964,285.44570007)
\lineto(651.66706222,292.42978879)
\lineto(659.81517037,278.46161136)
}
}
{
\newrgbcolor{curcolor}{0.75294119 0.75294119 0.75294119}
\pscustom[linewidth=2.32802935,linecolor=curcolor,linestyle=dashed,dash=3 2]
{
\newpath
\moveto(620.23866767,160.89613202)
\lineto(620.23866767,200.4726254)
}
}
{
\newrgbcolor{curcolor}{0.75294119 0.75294119 0.75294119}
\pscustom[linewidth=2.32802935,linecolor=curcolor]
{
\newpath
\moveto(624.89472681,185.34043785)
\lineto(620.23866767,200.4726254)
\lineto(615.58260853,185.34043785)
}
}
{
\newrgbcolor{curcolor}{0.75294119 0.75294119 0.75294119}
\pscustom[linewidth=2.32802935,linecolor=curcolor,linestyle=dashed,dash=3 2]
{
\newpath
\moveto(534.10158288,160.89613202)
\lineto(584.15421399,201.63664485)
}
}
{
\newrgbcolor{curcolor}{0.75294119 0.75294119 0.75294119}
\pscustom[linewidth=2.32802935,linecolor=curcolor]
{
\newpath
\moveto(576.00611515,188.83247755)
\lineto(584.15421399,201.63664485)
\lineto(570.18603656,195.81656626)
}
}
{
\newrgbcolor{curcolor}{0.75294119 0.75294119 0.75294119}
\pscustom[linewidth=2.32802935,linecolor=curcolor,linestyle=dashed,dash=3 2]
{
\newpath
\moveto(685.42349565,169.04423086)
\lineto(650.50305209,201.63664485)
}
}
{
\newrgbcolor{curcolor}{0.75294119 0.75294119 0.75294119}
\pscustom[linewidth=2.32802935,linecolor=curcolor]
{
\newpath
\moveto(664.47122951,194.65255614)
\lineto(650.50305209,201.63664485)
\lineto(658.65115093,187.66846742)
}
}
{
\newrgbcolor{curcolor}{0.75294119 0.75294119 0.75294119}
\pscustom[linewidth=2.32802935,linecolor=curcolor,linestyle=dashed,dash=3 2]
{
\newpath
\moveto(536.42961245,70.10298808)
\lineto(587.646263,110.84350091)
}
}
{
\newrgbcolor{curcolor}{0.75294119 0.75294119 0.75294119}
\pscustom[linewidth=2.32802935,linecolor=curcolor]
{
\newpath
\moveto(578.33414472,96.87532349)
\lineto(587.646263,110.84350091)
\lineto(572.51406613,105.02342232)
}
}
{
\newrgbcolor{curcolor}{1 0.50196081 0.50196081}
\pscustom[linewidth=2.32802935,linecolor=curcolor]
{
\newpath
\moveto(358.33536427,843.008731)
\lineto(432.83231053,843.008731)
\lineto(432.83231053,889.56931403)
\lineto(383.94368954,889.56931403)
\lineto(383.94368954,898.88143138)
\lineto(358.33536427,898.88143138)
\lineto(358.33536427,843.008731)
\closepath
}
}
{
\newrgbcolor{curcolor}{0 0 0}
\pscustom[linestyle=none,fillstyle=solid,fillcolor=curcolor]
{
\newpath
\moveto(369.44807238,870.52141388)
\lineto(369.44807238,871.47626967)
\lineto(374.96691148,873.80657248)
\lineto(374.96691148,872.78919637)
\lineto(370.59048913,870.99315811)
\lineto(374.96691148,869.18006885)
\lineto(374.96691148,868.16269274)
\lineto(369.44807238,870.52141388)
}
}
{
\newrgbcolor{curcolor}{0 0 0}
\pscustom[linestyle=none,fillstyle=solid,fillcolor=curcolor]
{
\newpath
\moveto(376.24573637,870.52141388)
\lineto(376.24573637,871.47626967)
\lineto(381.76457548,873.80657248)
\lineto(381.76457548,872.78919637)
\lineto(377.38815312,870.99315811)
\lineto(381.76457548,869.18006885)
\lineto(381.76457548,868.16269274)
\lineto(376.24573637,870.52141388)
}
}
{
\newrgbcolor{curcolor}{0 0 0}
\pscustom[linestyle=none,fillstyle=solid,fillcolor=curcolor]
{
\newpath
\moveto(384.11761313,866.87818436)
\lineto(383.16844101,866.87818436)
\lineto(383.16844101,875.21043783)
\lineto(384.19150078,875.21043783)
\lineto(384.19150078,872.23788083)
\curveto(384.62345713,872.7797177)(385.17477212,873.05063881)(385.84544741,873.05064498)
\curveto(386.21677641,873.05063881)(386.56726875,872.97485668)(386.89692551,872.82329837)
\curveto(387.23036239,872.67551727)(387.50317805,872.46522186)(387.71537333,872.19241151)
\curveto(387.93134709,871.92337963)(388.09996232,871.59751648)(388.22121955,871.21482106)
\curveto(388.34246514,870.83211697)(388.40309084,870.42289347)(388.40309684,869.98714933)
\curveto(388.40309084,868.95272016)(388.14732616,868.15321869)(387.63580201,867.58864253)
\curveto(387.12426741,867.02406496)(386.51043216,866.74177653)(385.79429442,866.74177639)
\curveto(385.08193902,866.74177653)(384.52304581,867.03922139)(384.11761313,867.63411186)
\lineto(384.11761313,866.87818436)
\moveto(384.1062458,869.94168001)
\curveto(384.1062441,869.21795761)(384.20476087,868.69506092)(384.4017964,868.37298836)
\curveto(384.72386845,867.84630106)(385.1596157,867.58295816)(385.70903944,867.58295887)
\curveto(386.1561507,867.58295816)(386.54263956,867.77620259)(386.86850718,868.16269274)
\curveto(387.19436588,868.55296942)(387.35729745,869.13270271)(387.35730241,869.90189435)
\curveto(387.35729745,870.69002548)(387.20004954,871.27165332)(386.88555818,871.64677963)
\curveto(386.57484697,872.0218964)(386.19783087,872.20945717)(385.75450876,872.2094625)
\curveto(385.30739085,872.20945717)(384.92090199,872.01431819)(384.59504102,871.62404497)
\curveto(384.26917568,871.23755136)(384.1062441,870.6767636)(384.1062458,869.94168001)
}
}
{
\newrgbcolor{curcolor}{0 0 0}
\pscustom[linestyle=none,fillstyle=solid,fillcolor=curcolor]
{
\newpath
\moveto(393.59228275,867.62274453)
\curveto(393.2133674,867.30066973)(392.84771863,867.07332334)(392.49533533,866.94070468)
\curveto(392.14673393,866.80808589)(391.77161239,866.74177653)(391.36996958,866.74177639)
\curveto(390.70687347,866.74177653)(390.19723864,866.90281355)(389.84106359,867.22488795)
\curveto(389.48488663,867.55075076)(389.30679862,867.96565792)(389.30679904,868.46961067)
\curveto(389.30679862,868.76515939)(389.37310799,869.03418595)(389.50572733,869.27669116)
\curveto(389.64213455,869.52298068)(389.818328,869.72001422)(390.03430821,869.86779236)
\curveto(390.25407524,870.01556452)(390.50036716,870.12734316)(390.77318472,870.20312862)
\curveto(390.97400547,870.25617278)(391.27713399,870.30732572)(391.68257118,870.35658758)
\curveto(392.5085936,870.45510087)(393.11674519,870.57256317)(393.50702777,870.70897484)
\curveto(393.51081226,870.84916795)(393.51270681,870.93821195)(393.51271144,870.97610711)
\curveto(393.51270681,871.39290473)(393.4160846,871.68656048)(393.2228445,871.85707525)
\curveto(392.96139182,872.08820577)(392.57300841,872.20377351)(392.0576931,872.20377884)
\curveto(391.5764734,872.20377351)(391.22029739,872.11851862)(390.989164,871.9480139)
\curveto(390.76181551,871.78128814)(390.59320027,871.48384328)(390.48331778,871.05567843)
\lineto(389.48299267,871.1920864)
\curveto(389.57393063,871.62025112)(389.72360034,871.9650598)(389.93200224,872.2265135)
\curveto(390.14040205,872.4917456)(390.44163601,872.6944628)(390.83570504,872.8346657)
\curveto(391.22977016,872.97864579)(391.68635749,873.05063881)(392.2054684,873.05064498)
\curveto(392.72078356,873.05063881)(393.13947982,872.99001311)(393.46155845,872.86876769)
\curveto(393.78362792,872.74751029)(394.02044708,872.59405148)(394.17201662,872.40839079)
\curveto(394.3235756,872.22650815)(394.42967058,871.99537266)(394.49030189,871.71498361)
\curveto(394.52439824,871.54067988)(394.54144922,871.22618404)(394.54145488,870.77149516)
\lineto(394.54145488,869.40741546)
\curveto(394.54144922,868.45634721)(394.56228931,867.85387928)(394.60397519,867.60000987)
\curveto(394.64943875,867.34992812)(394.7365882,867.10931986)(394.8654238,866.87818436)
\lineto(393.79689471,866.87818436)
\curveto(393.69079482,867.09037432)(393.6225909,867.3385608)(393.59228275,867.62274453)
\moveto(393.50702777,869.90757802)
\curveto(393.13569072,869.75601073)(392.57869207,869.62718111)(391.83603015,869.52108877)
\curveto(391.41543638,869.46046042)(391.11799152,869.39225651)(390.94369468,869.31647682)
\curveto(390.76939372,869.24069225)(390.63488044,869.12891361)(390.54015444,868.98114056)
\curveto(390.44542512,868.83715241)(390.39806129,868.67611538)(390.3980628,868.498029)
\curveto(390.39806129,868.22521171)(390.50036716,867.99786532)(390.70498073,867.81598915)
\curveto(390.91337977,867.6341111)(391.21650829,867.54317255)(391.6143672,867.54317321)
\curveto(392.00843154,867.54317255)(392.35892389,867.62842744)(392.66584529,867.79893815)
\curveto(392.97275914,867.97323613)(393.19821097,868.21005529)(393.34220148,868.50939633)
\curveto(393.45208111,868.74053019)(393.50702315,869.08154978)(393.50702777,869.5324561)
\lineto(393.50702777,869.90757802)
}
}
{
\newrgbcolor{curcolor}{0 0 0}
\pscustom[linestyle=none,fillstyle=solid,fillcolor=curcolor]
{
\newpath
\moveto(395.72365757,868.67990629)
\lineto(396.73535001,868.83904892)
\curveto(396.79218524,868.43361257)(396.94943316,868.12290584)(397.20709424,867.9069278)
\curveto(397.46854075,867.6909477)(397.83229497,867.58295816)(398.298358,867.58295887)
\curveto(398.76820427,867.58295816)(399.11680206,867.67768583)(399.34415243,867.86714214)
\curveto(399.57149484,868.06038558)(399.68516804,868.28583742)(399.68517236,868.54349832)
\curveto(399.68516804,868.77463215)(399.58475671,868.95650926)(399.38393809,869.0891302)
\curveto(399.24373713,869.18006654)(398.89513934,869.29563429)(398.33814366,869.43583379)
\curveto(397.5878976,869.62528656)(397.06689546,869.78821813)(396.77513567,869.92462901)
\curveto(396.48716217,870.06482291)(396.26739399,870.25617278)(396.11583049,870.49867922)
\curveto(395.96805458,870.74496752)(395.89416701,871.01588863)(395.89416754,871.31144337)
\curveto(395.89416701,871.5804655)(395.95479271,871.82865197)(396.07604483,872.05600354)
\curveto(396.20108463,872.28713386)(396.36969987,872.47848373)(396.58189105,872.63005374)
\curveto(396.7410323,872.74751029)(396.95701137,872.84602706)(397.2298289,872.92560434)
\curveto(397.50643181,873.00895864)(397.80198212,873.05063881)(398.11648071,873.05064498)
\curveto(398.59011626,873.05063881)(399.00502342,872.98243489)(399.36120343,872.84603303)
\curveto(399.72116455,872.70961923)(399.986402,872.52395301)(400.15691658,872.28903382)
\curveto(400.32742158,872.05789291)(400.44488388,871.74718618)(400.50930384,871.35691269)
\lineto(399.50897873,871.22050472)
\curveto(399.46350531,871.53120711)(399.33088658,871.77370993)(399.11112215,871.9480139)
\curveto(398.89513934,872.12230772)(398.58822171,872.20945717)(398.19036836,872.2094625)
\curveto(397.72051633,872.20945717)(397.38518041,872.13178049)(397.18435958,871.97643222)
\curveto(396.98353512,871.82107376)(396.8831238,871.63919665)(396.88312531,871.43080034)
\curveto(396.8831238,871.29817706)(396.92480397,871.17882021)(397.00816595,871.07272942)
\curveto(397.09152465,870.96284114)(397.22224883,870.87190259)(397.40033887,870.79991348)
\curveto(397.50264271,870.7620185)(397.80387667,870.67486905)(398.30404166,870.53846488)
\curveto(399.02775806,870.34521679)(399.53170922,870.18607431)(399.81589666,870.06103698)
\curveto(400.1038643,869.93978239)(400.32931614,869.76169439)(400.49225284,869.52677244)
\curveto(400.65517929,869.29184519)(400.73664508,869.00008399)(400.73665045,868.65148796)
\curveto(400.73664508,868.31046661)(400.63623376,867.98839256)(400.43541619,867.68526485)
\curveto(400.23837758,867.38592463)(399.95230004,867.15289458)(399.57718271,866.986174)
\curveto(399.20205696,866.82324232)(398.77767704,866.74177653)(398.30404166,866.74177639)
\curveto(397.51969369,866.74177653)(396.92101486,866.90470811)(396.5080034,867.23057161)
\curveto(396.09877876,867.55643442)(395.83733041,868.0395455)(395.72365757,868.67990629)
}
}
{
\newrgbcolor{curcolor}{0 0 0}
\pscustom[linestyle=none,fillstyle=solid,fillcolor=curcolor]
{
\newpath
\moveto(406.08497871,868.82199793)
\lineto(407.14214047,868.69127362)
\curveto(406.97541383,868.07364745)(406.66660165,867.59432548)(406.21570301,867.25330628)
\curveto(405.76479431,866.91228632)(405.18885013,866.74177653)(404.48786873,866.74177639)
\curveto(403.60500362,866.74177653)(402.90401892,867.01269764)(402.38491253,867.55454054)
\curveto(401.86959285,868.1001712)(401.61193361,868.86367615)(401.61193404,869.8450577)
\curveto(401.61193361,870.86053527)(401.87338196,871.64866941)(402.39627986,872.2094625)
\curveto(402.91917535,872.77024493)(403.5974254,873.05063881)(404.43103208,873.05064498)
\curveto(405.23810851,873.05063881)(405.89741304,872.77592859)(406.40894763,872.2265135)
\curveto(406.92047178,871.67708771)(407.17623647,870.90410999)(407.17624246,869.90757802)
\curveto(407.17623647,869.84694928)(407.17434192,869.75601073)(407.1705588,869.63476208)
\lineto(402.6690958,869.63476208)
\curveto(402.70698538,868.97166569)(402.89454615,868.46392542)(403.23177868,868.11153975)
\curveto(403.56900711,867.75915162)(403.98959793,867.58295816)(404.49355239,867.58295887)
\curveto(404.86867063,867.58295816)(405.18885013,867.68147493)(405.45409185,867.87850947)
\curveto(405.71932503,868.07554201)(405.92962044,868.39003784)(406.08497871,868.82199793)
\moveto(402.72593246,870.47594456)
\lineto(406.09634604,870.47594456)
\curveto(406.05087185,870.98368123)(405.92204223,871.36448643)(405.70985679,871.6183613)
\curveto(405.38398911,872.01242364)(404.96150374,872.20945717)(404.44239941,872.2094625)
\curveto(403.97254695,872.20945717)(403.57658532,872.05220925)(403.25451334,871.73771828)
\curveto(402.93622632,871.42321758)(402.76003287,871.00262676)(402.72593246,870.47594456)
}
}
{
\newrgbcolor{curcolor}{0 0 0}
\pscustom[linestyle=none,fillstyle=solid,fillcolor=curcolor]
{
\newpath
\moveto(413.82044939,870.52141388)
\lineto(408.30161028,868.16269274)
\lineto(408.30161028,869.18006885)
\lineto(412.67234897,870.99315811)
\lineto(408.30161028,872.78919637)
\lineto(408.30161028,873.80657248)
\lineto(413.82044939,871.47626967)
\lineto(413.82044939,870.52141388)
}
}
{
\newrgbcolor{curcolor}{0 0 0}
\pscustom[linestyle=none,fillstyle=solid,fillcolor=curcolor]
{
\newpath
\moveto(420.61811205,870.52141388)
\lineto(415.09927294,868.16269274)
\lineto(415.09927294,869.18006885)
\lineto(419.47001164,870.99315811)
\lineto(415.09927294,872.78919637)
\lineto(415.09927294,873.80657248)
\lineto(420.61811205,871.47626967)
\lineto(420.61811205,870.52141388)
}
}
{
\newrgbcolor{curcolor}{0 0 0}
\pscustom[linestyle=none,fillstyle=solid,fillcolor=curcolor]
{
\newpath
\moveto(368.43751223,855.45402135)
\lineto(369.92663257,855.45402135)
\lineto(369.92663257,854.56736955)
\curveto(370.11987472,854.87049292)(370.38132306,855.11678484)(370.71097839,855.30624605)
\curveto(371.04062759,855.49569549)(371.40627637,855.59042315)(371.80792581,855.59042932)
\curveto(372.50890635,855.59042315)(373.10379607,855.31571293)(373.59259675,854.76629784)
\curveto(374.08138554,854.21687205)(374.32578291,853.45147254)(374.32578958,852.47009702)
\curveto(374.32578291,851.46219164)(374.07949098,850.6778466)(373.58691308,850.11705954)
\curveto(373.0943233,849.56006019)(372.49753903,849.28156087)(371.79655848,849.28156073)
\curveto(371.46311296,849.28156087)(371.15998444,849.34787023)(370.88717202,849.48048902)
\curveto(370.61814222,849.61310768)(370.33395923,849.84045407)(370.03462221,850.16252887)
\lineto(370.03462221,847.12176788)
\lineto(368.43751223,847.12176788)
\lineto(368.43751223,855.45402135)
\moveto(370.01757121,852.538301)
\curveto(370.01756884,851.86004782)(370.15208212,851.35799121)(370.42111145,851.03212967)
\curveto(370.69013524,850.71005401)(371.01789295,850.54901698)(371.40438557,850.54901811)
\curveto(371.77571425,850.54901698)(372.08452642,850.69679213)(372.33082303,850.99234401)
\curveto(372.57711027,851.29168185)(372.70025623,851.78047659)(372.70026128,852.45872969)
\curveto(372.70025623,853.09150743)(372.57332116,853.56135663)(372.3194557,853.86827871)
\curveto(372.06558089,854.17519188)(371.75108505,854.32865069)(371.37596724,854.3286556)
\curveto(370.98568555,854.32865069)(370.66171694,854.17708643)(370.40406046,853.87396237)
\curveto(370.14639846,853.5746185)(370.01756884,853.12939849)(370.01757121,852.538301)
}
}
{
\newrgbcolor{curcolor}{0 0 0}
\pscustom[linestyle=none,fillstyle=solid,fillcolor=curcolor]
{
\newpath
\moveto(376.79250043,853.61251376)
\lineto(375.34316575,853.87396237)
\curveto(375.50609675,854.45748031)(375.78649063,854.88943845)(376.18434823,855.16983808)
\curveto(376.58220299,855.45022621)(377.1733036,855.59042315)(377.95765183,855.59042932)
\curveto(378.67000066,855.59042315)(379.20047556,855.50516825)(379.54907814,855.33466438)
\curveto(379.89767115,855.16793778)(380.14206852,854.95385326)(380.28227098,854.69241019)
\curveto(380.42625151,854.43474567)(380.49824453,853.95921281)(380.49825027,853.26581017)
\lineto(380.48119927,851.40156792)
\curveto(380.48119355,850.87109103)(380.50582274,850.47891851)(380.55508692,850.22504919)
\curveto(380.60812862,849.97496735)(380.70475083,849.70594079)(380.84495385,849.4179687)
\lineto(379.26489487,849.4179687)
\curveto(379.2232102,849.52406368)(379.17205726,849.6813116)(379.11143591,849.88971293)
\curveto(379.08490781,849.98444012)(379.06596228,850.04696037)(379.05459925,850.07727389)
\curveto(378.7817793,849.81203577)(378.4900181,849.61310768)(378.17931478,849.48048902)
\curveto(377.86860464,849.34787023)(377.53705782,849.28156087)(377.18467334,849.28156073)
\curveto(376.56325745,849.28156087)(376.07256817,849.4501761)(375.712604,849.78740695)
\curveto(375.35642704,850.12463706)(375.17833904,850.55091154)(375.17833945,851.06623166)
\curveto(375.17833904,851.4072496)(375.25980483,851.71037812)(375.42273706,851.97561813)
\curveto(375.58566798,852.24464213)(375.81301437,852.44925388)(376.10477691,852.58945399)
\curveto(376.40032588,852.73343687)(376.8247058,852.85847738)(377.37791796,852.96457591)
\curveto(378.12436932,853.1047693)(378.64158236,853.23549347)(378.92955862,853.35674882)
\lineto(378.92955862,853.51589145)
\curveto(378.92955445,853.82280498)(378.85377232,854.0406786)(378.702212,854.16951297)
\curveto(378.5506438,854.30212695)(378.26456626,854.36843631)(377.84397852,854.36844126)
\curveto(377.55979246,854.36843631)(377.33812973,854.31159971)(377.17898967,854.1979313)
\curveto(377.01984478,854.08804243)(376.89101516,853.89290345)(376.79250043,853.61251376)
\moveto(378.92955862,852.31663805)
\curveto(378.7249427,852.24843124)(378.4009741,852.16696545)(377.95765183,852.07224044)
\curveto(377.51432318,851.97751012)(377.22445653,851.88467702)(377.08805103,851.79374084)
\curveto(376.87964785,851.64596331)(376.77544742,851.45840254)(376.77544943,851.23105796)
\curveto(376.77544742,851.00749887)(376.85880776,850.81425444)(377.02553071,850.65132409)
\curveto(377.19224913,850.48839128)(377.40443909,850.40692549)(377.66210123,850.40692648)
\curveto(377.95007042,850.40692549)(378.22478064,850.50165315)(378.48623271,850.69110975)
\curveto(378.67947342,850.83509452)(378.80640849,851.01128797)(378.8670383,851.21969063)
\curveto(378.90871436,851.35609666)(378.92955445,851.61565046)(378.92955862,851.99835279)
\lineto(378.92955862,852.31663805)
}
}
{
\newrgbcolor{curcolor}{0 0 0}
\pscustom[linestyle=none,fillstyle=solid,fillcolor=curcolor]
{
\newpath
\moveto(387.34138281,853.66935042)
\lineto(385.7670075,853.38516715)
\curveto(385.71395548,853.69965902)(385.59270407,853.93647817)(385.40325291,854.09562532)
\curveto(385.21758253,854.25476312)(384.97507972,854.33433435)(384.67574374,854.33433927)
\curveto(384.27788413,854.33433435)(383.95959918,854.19603197)(383.72088795,853.91943169)
\curveto(383.48596087,853.64661153)(383.36849857,853.18812964)(383.3685007,852.54398467)
\curveto(383.36849857,851.82784042)(383.48785543,851.3219947)(383.72657162,851.02644601)
\curveto(383.96907195,850.73089409)(384.29304055,850.58311894)(384.6984784,850.58312011)
\curveto(385.00160346,850.58311894)(385.24978994,850.66837384)(385.44303857,850.83888505)
\curveto(385.6362788,851.01318253)(385.77268663,851.31062738)(385.85226248,851.73122052)
\lineto(387.42095413,851.46408824)
\curveto(387.25801637,850.74415597)(386.94541509,850.20041919)(386.48314934,849.83287627)
\curveto(386.02087311,849.46533253)(385.4013542,849.28156087)(384.62459075,849.28156073)
\curveto(383.74172556,849.28156087)(383.03695176,849.56006019)(382.51026722,850.11705954)
\curveto(381.98736926,850.6740575)(381.72592091,851.44514066)(381.7259214,852.43031136)
\curveto(381.72592091,853.42684335)(381.98926381,854.20171563)(382.51595089,854.75493051)
\curveto(383.04263542,855.31192382)(383.75498743,855.59042315)(384.65300908,855.59042932)
\curveto(385.38809232,855.59042315)(385.97161472,855.43128068)(386.40357802,855.11300143)
\curveto(386.8393201,854.7984999)(387.15192139,854.31728337)(387.34138281,853.66935042)
}
}
{
\newrgbcolor{curcolor}{0 0 0}
\pscustom[linestyle=none,fillstyle=solid,fillcolor=curcolor]
{
\newpath
\moveto(388.50085085,849.4179687)
\lineto(388.50085085,857.75022217)
\lineto(390.09796082,857.75022217)
\lineto(390.09796082,853.32833049)
\lineto(391.96788674,855.45402135)
\lineto(393.93443497,855.45402135)
\lineto(391.87126443,853.24875918)
\lineto(394.08221027,849.4179687)
\lineto(392.36005965,849.4179687)
\lineto(390.84252099,852.12907709)
\lineto(390.09796082,851.35041493)
\lineto(390.09796082,849.4179687)
\lineto(388.50085085,849.4179687)
}
}
{
\newrgbcolor{curcolor}{0 0 0}
\pscustom[linestyle=none,fillstyle=solid,fillcolor=curcolor]
{
\newpath
\moveto(396.23063608,853.61251376)
\lineto(394.78130141,853.87396237)
\curveto(394.94423241,854.45748031)(395.22462629,854.88943845)(395.62248389,855.16983808)
\curveto(396.02033865,855.45022621)(396.61143926,855.59042315)(397.39578749,855.59042932)
\curveto(398.10813631,855.59042315)(398.63861122,855.50516825)(398.9872138,855.33466438)
\curveto(399.33580681,855.16793778)(399.58020418,854.95385326)(399.72040664,854.69241019)
\curveto(399.86438717,854.43474567)(399.93638019,853.95921281)(399.93638592,853.26581017)
\lineto(399.91933493,851.40156792)
\curveto(399.91932921,850.87109103)(399.9439584,850.47891851)(399.99322258,850.22504919)
\curveto(400.04626428,849.97496735)(400.14288649,849.70594079)(400.28308951,849.4179687)
\lineto(398.70303053,849.4179687)
\curveto(398.66134586,849.52406368)(398.61019292,849.6813116)(398.54957157,849.88971293)
\curveto(398.52304347,849.98444012)(398.50409794,850.04696037)(398.49273491,850.07727389)
\curveto(398.21991496,849.81203577)(397.92815376,849.61310768)(397.61745044,849.48048902)
\curveto(397.30674029,849.34787023)(396.97519348,849.28156087)(396.622809,849.28156073)
\curveto(396.00139311,849.28156087)(395.51070382,849.4501761)(395.15073966,849.78740695)
\curveto(394.7945627,850.12463706)(394.6164747,850.55091154)(394.61647511,851.06623166)
\curveto(394.6164747,851.4072496)(394.69794049,851.71037812)(394.86087272,851.97561813)
\curveto(395.02380364,852.24464213)(395.25115003,852.44925388)(395.54291257,852.58945399)
\curveto(395.83846153,852.73343687)(396.26284146,852.85847738)(396.81605362,852.96457591)
\curveto(397.56250498,853.1047693)(398.07971802,853.23549347)(398.36769427,853.35674882)
\lineto(398.36769427,853.51589145)
\curveto(398.36769011,853.82280498)(398.29190798,854.0406786)(398.14034766,854.16951297)
\curveto(397.98877946,854.30212695)(397.70270192,854.36843631)(397.28211418,854.36844126)
\curveto(396.99792812,854.36843631)(396.77626539,854.31159971)(396.61712533,854.1979313)
\curveto(396.45798044,854.08804243)(396.32915082,853.89290345)(396.23063608,853.61251376)
\moveto(398.36769427,852.31663805)
\curveto(398.16307836,852.24843124)(397.83910975,852.16696545)(397.39578749,852.07224044)
\curveto(396.95245884,851.97751012)(396.66259219,851.88467702)(396.52618669,851.79374084)
\curveto(396.3177835,851.64596331)(396.21358308,851.45840254)(396.21358509,851.23105796)
\curveto(396.21358308,851.00749887)(396.29694342,850.81425444)(396.46366637,850.65132409)
\curveto(396.63038479,850.48839128)(396.84257475,850.40692549)(397.10023689,850.40692648)
\curveto(397.38820608,850.40692549)(397.6629163,850.50165315)(397.92436837,850.69110975)
\curveto(398.11760908,850.83509452)(398.24454415,851.01128797)(398.30517395,851.21969063)
\curveto(398.34685002,851.35609666)(398.36769011,851.61565046)(398.36769427,851.99835279)
\lineto(398.36769427,852.31663805)
}
}
{
\newrgbcolor{curcolor}{0 0 0}
\pscustom[linestyle=none,fillstyle=solid,fillcolor=curcolor]
{
\newpath
\moveto(401.3686699,849.02011212)
\lineto(403.19312649,848.79844917)
\curveto(403.22343683,848.58625983)(403.2935353,848.44037923)(403.40342211,848.36080693)
\curveto(403.55498365,848.2471348)(403.79369736,848.1902982)(404.11956395,848.19029697)
\curveto(404.53636223,848.1902982)(404.84896351,848.25281846)(405.05736874,848.37785793)
\curveto(405.19756131,848.46121931)(405.30365629,848.59573259)(405.37565401,848.78139817)
\curveto(405.42490769,848.91401754)(405.44953689,849.15841491)(405.44954166,849.51459101)
\lineto(405.44954166,850.39555915)
\curveto(404.97210947,849.74383186)(404.36964154,849.4179687)(403.64213606,849.4179687)
\curveto(402.83126431,849.4179687)(402.18901076,849.76088283)(401.71537349,850.44671214)
\curveto(401.34404002,850.98855333)(401.1583738,851.66301429)(401.15837428,852.47009702)
\curveto(401.1583738,853.48178539)(401.40087662,854.25476312)(401.88588345,854.7890325)
\curveto(402.37467698,855.32329114)(402.98093402,855.59042315)(403.70465638,855.59042932)
\curveto(404.45110733,855.59042315)(405.06683713,855.26266544)(405.55184763,854.60715521)
\lineto(405.55184763,855.45402135)
\lineto(407.04665163,855.45402135)
\lineto(407.04665163,850.03748823)
\curveto(407.04664527,849.32513559)(406.98791412,848.79276613)(406.87045801,848.44037825)
\curveto(406.75298951,848.08799233)(406.58816338,847.81138755)(406.37597912,847.6105631)
\curveto(406.16378346,847.40974227)(405.87960047,847.25249435)(405.52342931,847.13881887)
\curveto(405.17103756,847.02514796)(404.723923,846.96831136)(404.18208427,846.96830891)
\curveto(403.15902202,846.96831136)(402.43340813,847.14450481)(402.00524042,847.49688979)
\curveto(401.57707007,847.84548951)(401.36298555,848.28881497)(401.36298623,848.8268675)
\curveto(401.36298555,848.87991558)(401.36488011,848.94433039)(401.3686699,849.02011212)
\moveto(402.79526991,852.56103566)
\curveto(402.7952678,851.92067353)(402.91841376,851.45082432)(403.16470817,851.15148665)
\curveto(403.41478671,850.85593461)(403.72170433,850.70815945)(404.08546196,850.70816074)
\curveto(404.47573652,850.70815945)(404.80538879,850.85972371)(405.07441974,851.16285398)
\curveto(405.34344191,851.46976986)(405.47795519,851.92256808)(405.47795998,852.52125001)
\curveto(405.47795519,853.14644947)(405.34912557,853.61061502)(405.09147074,853.91374803)
\curveto(404.83380708,854.21687205)(404.50794393,854.36843631)(404.11388029,854.36844126)
\curveto(403.7311771,854.36843631)(403.41478671,854.2187666)(403.16470817,853.91943169)
\curveto(402.91841376,853.62387689)(402.7952678,853.17107866)(402.79526991,852.56103566)
}
}
{
\newrgbcolor{curcolor}{0 0 0}
\pscustom[linestyle=none,fillstyle=solid,fillcolor=curcolor]
{
\newpath
\moveto(412.12784767,851.3390476)
\lineto(413.71927398,851.07191533)
\curveto(413.51465631,850.48839128)(413.1906877,850.04317127)(412.7473672,849.73625396)
\curveto(412.3078259,849.43312512)(411.7565109,849.28156087)(411.09342057,849.28156073)
\curveto(410.04383478,849.28156087)(409.26706795,849.624475)(408.76311775,850.31030417)
\curveto(408.36526061,850.85972371)(408.16633252,851.5531302)(408.16633289,852.3905257)
\curveto(408.16633252,853.39084684)(408.42778086,854.17329733)(408.95067871,854.73787951)
\curveto(409.47357425,855.30624016)(410.13477333,855.59042315)(410.93427794,855.59042932)
\curveto(411.83229303,855.59042315)(412.54085594,855.29297829)(413.05996879,854.69809385)
\curveto(413.57907112,854.10698796)(413.82725759,853.19949696)(413.80452896,851.97561813)
\lineto(409.80322852,851.97561813)
\curveto(409.81459383,851.50197726)(409.94342345,851.13253938)(410.18971777,850.86730338)
\curveto(410.4360073,850.60585358)(410.74292492,850.47512941)(411.11047156,850.47513046)
\curveto(411.36054928,850.47512941)(411.57084469,850.54333332)(411.74135842,850.67974242)
\curveto(411.91186427,850.81614899)(412.04069389,851.03591716)(412.12784767,851.3390476)
\moveto(412.21878632,852.95320858)
\curveto(412.20741457,853.41547603)(412.08805772,853.76596838)(411.8607154,854.00468668)
\curveto(411.63336494,854.2471849)(411.35676017,854.36843631)(411.03090025,854.36844126)
\curveto(410.68229922,854.36843631)(410.39432713,854.24150124)(410.16698311,853.98763568)
\curveto(409.93963435,853.73376098)(409.82785571,853.38895229)(409.83164685,852.95320858)
\lineto(412.21878632,852.95320858)
}
}
{
\newrgbcolor{curcolor}{0 0 0}
\pscustom[linestyle=none,fillstyle=solid,fillcolor=curcolor]
{
\newpath
\moveto(414.54908765,851.14011931)
\lineto(416.15188129,851.38451693)
\curveto(416.22008333,851.07380823)(416.35838572,850.83698907)(416.56678886,850.67405875)
\curveto(416.77518743,850.51491502)(417.06694863,850.43534379)(417.44207333,850.43534481)
\curveto(417.85508277,850.43534379)(418.16578951,850.51112592)(418.37419446,850.66269142)
\curveto(418.5143873,850.76878516)(418.58448577,850.91087665)(418.58449008,851.08896633)
\curveto(418.58448577,851.21021606)(418.54659471,851.31062738)(418.47081677,851.39020059)
\curveto(418.39124134,851.46598075)(418.21315334,851.53607922)(417.93655222,851.60049621)
\curveto(416.64825236,851.88467702)(415.83169992,852.14423081)(415.48689244,852.37915837)
\curveto(415.00946381,852.70501857)(414.7707501,853.15781679)(414.7707506,853.7375544)
\curveto(414.7707501,854.26044678)(414.97725641,854.69998313)(415.39027013,855.05616477)
\curveto(415.80328162,855.41233514)(416.44364061,855.59042315)(417.31134903,855.59042932)
\curveto(418.13737121,855.59042315)(418.75120646,855.45590987)(419.15285662,855.18688908)
\curveto(419.55449703,854.91785675)(419.8311018,854.52000057)(419.98267177,853.99331934)
\lineto(418.47650044,853.71481974)
\curveto(418.41208143,853.94974004)(418.28893547,854.1297226)(418.10706219,854.25476795)
\curveto(417.92897035,854.37980363)(417.67320566,854.44232389)(417.33976736,854.44232891)
\curveto(416.91917347,854.44232389)(416.61793951,854.38359274)(416.43606456,854.26613528)
\curveto(416.31481099,854.18277009)(416.25418529,854.07478056)(416.25418727,853.94216636)
\curveto(416.25418529,853.82848864)(416.30723278,853.73186642)(416.4133299,853.65229942)
\curveto(416.55731381,853.5462002)(417.05368675,853.3965305)(417.90245023,853.20328985)
\curveto(418.75499556,853.01004164)(419.34988528,852.77322248)(419.68712117,852.49283168)
\curveto(420.02055713,852.20864562)(420.18727781,851.81268399)(420.18728372,851.30494561)
\curveto(420.18727781,850.75173418)(419.95614232,850.27620132)(419.49387654,849.8783456)
\curveto(419.03160034,849.48048896)(418.34766662,849.28156087)(417.44207333,849.28156073)
\curveto(416.61983406,849.28156087)(415.96810775,849.44828155)(415.48689244,849.78172328)
\curveto(415.00946381,850.11516429)(414.69686253,850.56796251)(414.54908765,851.14011931)
}
}
{
\newrgbcolor{curcolor}{1 0.50196081 0.50196081}
\pscustom[linewidth=2.32802935,linecolor=curcolor]
{
\newpath
\moveto(130.18849429,752.21558706)
\lineto(204.68543124,752.21558706)
\lineto(204.68543124,798.77616916)
\lineto(154.63280013,798.77616916)
\lineto(154.63280013,808.08828744)
\lineto(130.18849429,808.08828744)
\lineto(130.18849429,752.21558706)
\closepath
}
}
{
\newrgbcolor{curcolor}{0 0 0}
\pscustom[linestyle=none,fillstyle=solid,fillcolor=curcolor]
{
\newpath
\moveto(141.30119281,779.72827823)
\lineto(141.30119281,780.68313401)
\lineto(146.82003191,783.01343683)
\lineto(146.82003191,781.99606072)
\lineto(142.44360955,780.20002245)
\lineto(146.82003191,778.38693319)
\lineto(146.82003191,777.36955709)
\lineto(141.30119281,779.72827823)
}
}
{
\newrgbcolor{curcolor}{0 0 0}
\pscustom[linestyle=none,fillstyle=solid,fillcolor=curcolor]
{
\newpath
\moveto(148.0988568,779.72827823)
\lineto(148.0988568,780.68313401)
\lineto(153.6176959,783.01343683)
\lineto(153.6176959,781.99606072)
\lineto(149.24127354,780.20002245)
\lineto(153.6176959,778.38693319)
\lineto(153.6176959,777.36955709)
\lineto(148.0988568,779.72827823)
}
}
{
\newrgbcolor{curcolor}{0 0 0}
\pscustom[linestyle=none,fillstyle=solid,fillcolor=curcolor]
{
\newpath
\moveto(155.97073355,776.08504871)
\lineto(155.02156143,776.08504871)
\lineto(155.02156143,784.41730218)
\lineto(156.0446212,784.41730218)
\lineto(156.0446212,781.44474518)
\curveto(156.47657756,781.98658204)(157.02789255,782.25750316)(157.69856783,782.25750933)
\curveto(158.06989683,782.25750316)(158.42038918,782.18172103)(158.75004593,782.03016271)
\curveto(159.08348281,781.88238161)(159.35629848,781.67208621)(159.56849375,781.39927585)
\curveto(159.78446751,781.13024398)(159.95308275,780.80438082)(160.07433997,780.4216854)
\curveto(160.19558556,780.03898131)(160.25621127,779.62975781)(160.25621726,779.19401368)
\curveto(160.25621127,778.1595845)(160.00044658,777.36008304)(159.48892243,776.79550688)
\curveto(158.97738783,776.23092931)(158.36355258,775.94864087)(157.64741484,775.94864074)
\curveto(156.93505944,775.94864087)(156.37616623,776.24608573)(155.97073355,776.8409762)
\lineto(155.97073355,776.08504871)
\moveto(155.95936622,779.14854436)
\curveto(155.95936452,778.42482195)(156.05788129,777.90192526)(156.25491682,777.57985271)
\curveto(156.57698888,777.05316541)(157.01273612,776.78982251)(157.56215986,776.78982322)
\curveto(158.00927113,776.78982251)(158.39575999,776.98306694)(158.72162761,777.36955709)
\curveto(159.0474863,777.75983377)(159.21041788,778.33956706)(159.21042283,779.1087587)
\curveto(159.21041788,779.89688982)(159.05316996,780.47851767)(158.7386786,780.85364397)
\curveto(158.42796739,781.22876075)(158.0509513,781.41632152)(157.60762919,781.41632685)
\curveto(157.16051128,781.41632152)(156.77402241,781.22118253)(156.44816145,780.83090931)
\curveto(156.1222961,780.44441571)(155.95936452,779.88362795)(155.95936622,779.14854436)
}
}
{
\newrgbcolor{curcolor}{0 0 0}
\pscustom[linestyle=none,fillstyle=solid,fillcolor=curcolor]
{
\newpath
\moveto(165.44540318,776.82960887)
\curveto(165.06648782,776.50753408)(164.70083905,776.28018769)(164.34845576,776.14756903)
\curveto(163.99985435,776.01495024)(163.62473281,775.94864087)(163.22309001,775.94864074)
\curveto(162.55999389,775.94864087)(162.05035907,776.1096779)(161.69418401,776.4317523)
\curveto(161.33800705,776.75761511)(161.15991905,777.17252226)(161.15991947,777.67647502)
\curveto(161.15991905,777.97202373)(161.22622841,778.24105029)(161.35884776,778.4835555)
\curveto(161.49525497,778.72984503)(161.67144842,778.92687856)(161.88742864,779.07465671)
\curveto(162.10719567,779.22242887)(162.35348759,779.33420751)(162.62630514,779.40999296)
\curveto(162.8271259,779.46303713)(163.13025441,779.51419007)(163.5356916,779.56345193)
\curveto(164.36171402,779.66196522)(164.96986561,779.77942752)(165.3601482,779.91583918)
\curveto(165.36393268,780.05603229)(165.36582724,780.1450763)(165.36583186,780.18297146)
\curveto(165.36582724,780.59976907)(165.26920502,780.89342482)(165.07596493,781.06393959)
\curveto(164.81451224,781.29507011)(164.42612883,781.41063786)(163.91081352,781.41064318)
\curveto(163.42959383,781.41063786)(163.07341782,781.32538296)(162.84228443,781.15487824)
\curveto(162.61493593,780.98815249)(162.4463207,780.69070763)(162.3364382,780.26254277)
\lineto(161.33611309,780.39895074)
\curveto(161.42705105,780.82711546)(161.57672076,781.17192415)(161.78512266,781.43337785)
\curveto(161.99352247,781.69860995)(162.29475644,781.90132715)(162.68882546,782.04153004)
\curveto(163.08289058,782.18551013)(163.53947791,782.25750316)(164.05858882,782.25750933)
\curveto(164.57390398,782.25750316)(164.99260025,782.19687745)(165.31467887,782.07563204)
\curveto(165.63674835,781.95437464)(165.8735675,781.80091583)(166.02513705,781.61525514)
\curveto(166.17669602,781.4333725)(166.282791,781.202237)(166.34342231,780.92184796)
\curveto(166.37751867,780.74754423)(166.39456964,780.43304839)(166.3945753,779.9783595)
\lineto(166.3945753,778.61427981)
\curveto(166.39456964,777.66321155)(166.41540973,777.06074362)(166.45709562,776.80687421)
\curveto(166.50255918,776.55679246)(166.58970863,776.3161842)(166.71854423,776.08504871)
\lineto(165.65001513,776.08504871)
\curveto(165.54391524,776.29723867)(165.47571132,776.54542514)(165.44540318,776.82960887)
\moveto(165.3601482,779.11444236)
\curveto(164.98881114,778.96287507)(164.43181249,778.83404545)(163.68915057,778.72795312)
\curveto(163.2685568,778.66732477)(162.97111194,778.59912085)(162.7968151,778.52334116)
\curveto(162.62251415,778.44755659)(162.48800087,778.33577795)(162.39327486,778.1880049)
\curveto(162.29854554,778.04401675)(162.25118171,777.88297973)(162.25118322,777.70489334)
\curveto(162.25118171,777.43207606)(162.35348759,777.20472967)(162.55810116,777.0228535)
\curveto(162.76650019,776.84097545)(163.06962871,776.75003689)(163.46748762,776.75003756)
\curveto(163.86155196,776.75003689)(164.21204431,776.83529179)(164.51896572,777.0058025)
\curveto(164.82587956,777.18010048)(165.0513314,777.41691963)(165.1953219,777.71626068)
\curveto(165.30520153,777.94739454)(165.36014358,778.28841412)(165.3601482,778.73932045)
\lineto(165.3601482,779.11444236)
}
}
{
\newrgbcolor{curcolor}{0 0 0}
\pscustom[linestyle=none,fillstyle=solid,fillcolor=curcolor]
{
\newpath
\moveto(167.576778,777.88677064)
\lineto(168.58847044,778.04591327)
\curveto(168.64530567,777.64047691)(168.80255358,777.32977018)(169.06021467,777.11379214)
\curveto(169.32166117,776.89781204)(169.68541539,776.78982251)(170.15147842,776.78982322)
\curveto(170.62132469,776.78982251)(170.96992249,776.88455017)(171.19727286,777.07400649)
\curveto(171.42461527,777.26724993)(171.53828846,777.49270176)(171.53829278,777.75036267)
\curveto(171.53828846,777.9814965)(171.43787714,778.16337361)(171.23705851,778.29599455)
\curveto(171.09685756,778.38693089)(170.74825976,778.50249864)(170.19126408,778.64269814)
\curveto(169.44101803,778.8321509)(168.92001589,778.99508248)(168.6282561,779.13149336)
\curveto(168.34028259,779.27168725)(168.12051442,779.46303713)(167.96895091,779.70554356)
\curveto(167.82117501,779.95183186)(167.74728743,780.22275298)(167.74728796,780.51830772)
\curveto(167.74728743,780.78732984)(167.80791313,781.03551632)(167.92916525,781.26286788)
\curveto(168.05420506,781.4939982)(168.22282029,781.68534808)(168.43501147,781.83691809)
\curveto(168.59415273,781.95437464)(168.8101318,782.05289141)(169.08294933,782.13246869)
\curveto(169.35955224,782.21582298)(169.65510254,782.25750316)(169.96960113,782.25750933)
\curveto(170.44323669,782.25750316)(170.85814385,782.18929924)(171.21432385,782.05289737)
\curveto(171.57428497,781.91648357)(171.83952243,781.73081736)(172.01003701,781.49589816)
\curveto(172.18054201,781.26475726)(172.29800431,780.95405053)(172.36242426,780.56377704)
\lineto(171.36209915,780.42736907)
\curveto(171.31662573,780.73807146)(171.184007,780.98057427)(170.96424257,781.15487824)
\curveto(170.74825976,781.32917207)(170.44134214,781.41632152)(170.04348878,781.41632685)
\curveto(169.57363675,781.41632152)(169.23830083,781.33864484)(169.03748,781.18329657)
\curveto(168.83665554,781.0279381)(168.73624422,780.84606099)(168.73624574,780.63766469)
\curveto(168.73624422,780.50504141)(168.77792439,780.38568456)(168.86128638,780.27959377)
\curveto(168.94464508,780.16970549)(169.07536925,780.07876693)(169.25345929,780.00677783)
\curveto(169.35576313,779.96888284)(169.65699709,779.8817334)(170.15716209,779.74532922)
\curveto(170.88087849,779.55208113)(171.38482965,779.39293866)(171.66901708,779.26790133)
\curveto(171.95698473,779.14664674)(172.18243656,778.96855873)(172.34537327,778.73363678)
\curveto(172.50829972,778.49870953)(172.58976551,778.20694833)(172.58977088,777.85835231)
\curveto(172.58976551,777.51733095)(172.48935419,777.1952569)(172.28853661,776.89212919)
\curveto(172.09149801,776.59278897)(171.80542047,776.35975893)(171.43030314,776.19303835)
\curveto(171.05517738,776.03010666)(170.63079746,775.94864087)(170.15716209,775.94864074)
\curveto(169.37281411,775.94864087)(168.77413529,776.11157245)(168.36112382,776.43743596)
\curveto(167.95189918,776.76329877)(167.69045083,777.24640984)(167.576778,777.88677064)
}
}
{
\newrgbcolor{curcolor}{0 0 0}
\pscustom[linestyle=none,fillstyle=solid,fillcolor=curcolor]
{
\newpath
\moveto(177.93809913,778.02886227)
\lineto(178.99526089,777.89813797)
\curveto(178.82853425,777.2805118)(178.51972207,776.80118983)(178.06882343,776.46017062)
\curveto(177.61791473,776.11915066)(177.04197055,775.94864087)(176.34098915,775.94864074)
\curveto(175.45812404,775.94864087)(174.75713934,776.21956199)(174.23803296,776.76140489)
\curveto(173.72271328,777.30703554)(173.46505404,778.0705405)(173.46505446,779.05192204)
\curveto(173.46505404,780.06739961)(173.72650238,780.85553376)(174.24940029,781.41632685)
\curveto(174.77229577,781.97710928)(175.45054583,782.25750316)(176.2841525,782.25750933)
\curveto(177.09122893,782.25750316)(177.75053346,781.98279294)(178.26206806,781.43337785)
\curveto(178.77359221,780.88395206)(179.0293569,780.11097434)(179.02936289,779.11444236)
\curveto(179.0293569,779.05381363)(179.02746234,778.96287507)(179.02367922,778.84162642)
\lineto(174.52221623,778.84162642)
\curveto(174.56010581,778.17853003)(174.74766658,777.67078977)(175.0848991,777.3184041)
\curveto(175.42212753,776.96601596)(175.84271835,776.78982251)(176.34667282,776.78982322)
\curveto(176.72179105,776.78982251)(177.04197055,776.88833928)(177.30721227,777.08537382)
\curveto(177.57244546,777.28240635)(177.78274087,777.59690219)(177.93809913,778.02886227)
\moveto(174.57905288,779.6828089)
\lineto(177.94946646,779.6828089)
\curveto(177.90399227,780.19054557)(177.77516265,780.57135077)(177.56297721,780.82522565)
\curveto(177.23710953,781.21928798)(176.81462416,781.41632152)(176.29551983,781.41632685)
\curveto(175.82566737,781.41632152)(175.42970574,781.2590736)(175.10763376,780.94458262)
\curveto(174.78934675,780.63008192)(174.6131533,780.20949111)(174.57905288,779.6828089)
}
}
{
\newrgbcolor{curcolor}{0 0 0}
\pscustom[linestyle=none,fillstyle=solid,fillcolor=curcolor]
{
\newpath
\moveto(185.67356981,779.72827823)
\lineto(180.15473071,777.36955709)
\lineto(180.15473071,778.38693319)
\lineto(184.5254694,780.20002245)
\lineto(180.15473071,781.99606072)
\lineto(180.15473071,783.01343683)
\lineto(185.67356981,780.68313401)
\lineto(185.67356981,779.72827823)
}
}
{
\newrgbcolor{curcolor}{0 0 0}
\pscustom[linestyle=none,fillstyle=solid,fillcolor=curcolor]
{
\newpath
\moveto(192.47123247,779.72827823)
\lineto(186.95239337,777.36955709)
\lineto(186.95239337,778.38693319)
\lineto(191.32313206,780.20002245)
\lineto(186.95239337,781.99606072)
\lineto(186.95239337,783.01343683)
\lineto(192.47123247,780.68313401)
\lineto(192.47123247,779.72827823)
}
}
{
\newrgbcolor{curcolor}{0 0 0}
\pscustom[linestyle=none,fillstyle=solid,fillcolor=curcolor]
{
\newpath
\moveto(139.12662775,764.6608857)
\lineto(140.61574809,764.6608857)
\lineto(140.61574809,763.7742339)
\curveto(140.80899024,764.07735726)(141.07043859,764.32364919)(141.40009391,764.5131104)
\curveto(141.72974311,764.70255983)(142.09539189,764.79728749)(142.49704134,764.79729367)
\curveto(143.19802187,764.79728749)(143.79291159,764.52257728)(144.28171227,763.97316218)
\curveto(144.77050106,763.4237364)(145.01489843,762.65833689)(145.01490511,761.67696136)
\curveto(145.01489843,760.66905599)(144.76860651,759.88471095)(144.27602861,759.32392389)
\curveto(143.78343882,758.76692454)(143.18665455,758.48842521)(142.48567401,758.48842508)
\curveto(142.15222849,758.48842521)(141.84909997,758.55473457)(141.57628754,758.68735336)
\curveto(141.30725774,758.81997203)(141.02307476,759.04731842)(140.72373773,759.36939321)
\lineto(140.72373773,756.32863222)
\lineto(139.12662775,756.32863222)
\lineto(139.12662775,764.6608857)
\moveto(140.70668674,761.74516535)
\curveto(140.70668437,761.06691217)(140.84119765,760.56485556)(141.11022698,760.23899402)
\curveto(141.37925077,759.91691835)(141.70700848,759.75588133)(142.09350109,759.75588246)
\curveto(142.46482977,759.75588133)(142.77364195,759.90365648)(143.01993855,760.19920836)
\curveto(143.26622579,760.4985462)(143.38937175,760.98734093)(143.3893768,761.66559403)
\curveto(143.38937175,762.29837177)(143.26243668,762.76822098)(143.00857122,763.07514305)
\curveto(142.75469642,763.38205623)(142.44020058,763.53551504)(142.06508277,763.53551995)
\curveto(141.67480107,763.53551504)(141.35083247,763.38395078)(141.09317598,763.08082672)
\curveto(140.83551399,762.78148285)(140.70668437,762.33626284)(140.70668674,761.74516535)
}
}
{
\newrgbcolor{curcolor}{0 0 0}
\pscustom[linestyle=none,fillstyle=solid,fillcolor=curcolor]
{
\newpath
\moveto(147.81695221,758.62483304)
\lineto(146.21984223,758.62483304)
\lineto(146.21984223,764.6608857)
\lineto(147.7032789,764.6608857)
\lineto(147.7032789,763.80265222)
\curveto(147.95714678,764.20808144)(148.18449317,764.47521344)(148.38531875,764.60404904)
\curveto(148.58992756,764.73287268)(148.82106306,764.79728749)(149.07872593,764.79729367)
\curveto(149.44247652,764.79728749)(149.79296887,764.69687617)(150.13020403,764.4960594)
\lineto(149.63572514,763.10356138)
\curveto(149.36669439,763.2778558)(149.11661336,763.36500525)(148.8854813,763.36500999)
\curveto(148.66192059,763.36500525)(148.47246526,763.30248499)(148.31711476,763.17744903)
\curveto(148.16175853,763.05619307)(148.03861257,762.83453034)(147.94767651,762.51246018)
\curveto(147.86052457,762.19038224)(147.81694984,761.51592129)(147.81695221,760.4890753)
\lineto(147.81695221,758.62483304)
}
}
{
\newrgbcolor{curcolor}{0 0 0}
\pscustom[linestyle=none,fillstyle=solid,fillcolor=curcolor]
{
\newpath
\moveto(150.81224381,765.47933352)
\lineto(150.81224381,766.95708652)
\lineto(152.40935379,766.95708652)
\lineto(152.40935379,765.47933352)
\lineto(150.81224381,765.47933352)
\moveto(150.81224381,758.62483304)
\lineto(150.81224381,764.6608857)
\lineto(152.40935379,764.6608857)
\lineto(152.40935379,758.62483304)
\lineto(150.81224381,758.62483304)
}
}
{
\newrgbcolor{curcolor}{0 0 0}
\pscustom[linestyle=none,fillstyle=solid,fillcolor=curcolor]
{
\newpath
\moveto(155.70019594,758.62483304)
\lineto(153.26758715,764.6608857)
\lineto(154.94426844,764.6608857)
\lineto(156.08100152,761.58033905)
\lineto(156.41065411,760.55159561)
\curveto(156.49780036,760.81304203)(156.5527424,760.98544638)(156.57548041,761.06880917)
\curveto(156.62852453,761.23931651)(156.68536113,761.4098263)(156.74599037,761.58033905)
\lineto(157.89409078,764.6608857)
\lineto(159.53667008,764.6608857)
\lineto(157.13816328,758.62483304)
\lineto(155.70019594,758.62483304)
}
}
{
\newrgbcolor{curcolor}{0 0 0}
\pscustom[linestyle=none,fillstyle=solid,fillcolor=curcolor]
{
\newpath
\moveto(160.5426788,765.47933352)
\lineto(160.5426788,766.95708652)
\lineto(162.13978877,766.95708652)
\lineto(162.13978877,765.47933352)
\lineto(160.5426788,765.47933352)
\moveto(160.5426788,758.62483304)
\lineto(160.5426788,764.6608857)
\lineto(162.13978877,764.6608857)
\lineto(162.13978877,758.62483304)
\lineto(160.5426788,758.62483304)
}
}
{
\newrgbcolor{curcolor}{0 0 0}
\pscustom[linestyle=none,fillstyle=solid,fillcolor=curcolor]
{
\newpath
\moveto(163.77100151,758.62483304)
\lineto(163.77100151,766.95708652)
\lineto(165.36811149,766.95708652)
\lineto(165.36811149,758.62483304)
\lineto(163.77100151,758.62483304)
}
}
{
\newrgbcolor{curcolor}{0 0 0}
\pscustom[linestyle=none,fillstyle=solid,fillcolor=curcolor]
{
\newpath
\moveto(170.49477667,760.54591195)
\lineto(172.08620298,760.27877968)
\curveto(171.88158531,759.69525562)(171.55761671,759.25003561)(171.1142962,758.94311831)
\curveto(170.6747549,758.63998947)(170.12343991,758.48842521)(169.46034957,758.48842508)
\curveto(168.41076378,758.48842521)(167.63399695,758.83133935)(167.13004676,759.51716851)
\curveto(166.73218961,760.06658806)(166.53326152,760.75999454)(166.53326189,761.59739005)
\curveto(166.53326152,762.59771119)(166.79470987,763.38016167)(167.31760771,763.94474386)
\curveto(167.84050325,764.51310451)(168.50170233,764.79728749)(169.30120694,764.79729367)
\curveto(170.19922204,764.79728749)(170.90778495,764.49984264)(171.4268978,763.9049582)
\curveto(171.94600012,763.31385231)(172.1941866,762.40636131)(172.17145796,761.18248247)
\lineto(168.17015752,761.18248247)
\curveto(168.18152284,760.70884161)(168.31035246,760.33940373)(168.55664677,760.07416772)
\curveto(168.8029363,759.81271792)(169.10985392,759.68199375)(169.47740057,759.68199481)
\curveto(169.72747828,759.68199375)(169.93777369,759.75019767)(170.10828742,759.88660676)
\curveto(170.27879327,760.02301333)(170.40762289,760.24278151)(170.49477667,760.54591195)
\moveto(170.58571532,762.16007292)
\curveto(170.57434358,762.62234038)(170.45498672,762.97283273)(170.2276444,763.21155102)
\curveto(170.00029395,763.45404925)(169.72368917,763.57530066)(169.39782925,763.57530561)
\curveto(169.04922822,763.57530066)(168.76125613,763.44836559)(168.53391211,763.19450002)
\curveto(168.30656335,762.94062532)(168.19478471,762.59581663)(168.19857585,762.16007292)
\lineto(170.58571532,762.16007292)
}
}
{
\newrgbcolor{curcolor}{0 0 0}
\pscustom[linestyle=none,fillstyle=solid,fillcolor=curcolor]
{
\newpath
\moveto(173.33092778,758.22697647)
\lineto(175.15538437,758.00531352)
\curveto(175.18569471,757.79312417)(175.25579318,757.64724357)(175.36567999,757.56767128)
\curveto(175.51724153,757.45399914)(175.75595523,757.39716255)(176.08182183,757.39716132)
\curveto(176.4986201,757.39716255)(176.81122139,757.4596828)(177.01962662,757.58472228)
\curveto(177.15981918,757.66808366)(177.26591417,757.80259694)(177.33791188,757.98826252)
\curveto(177.38716557,758.12088188)(177.41179477,758.36527925)(177.41179953,758.72145536)
\lineto(177.41179953,759.60242349)
\curveto(176.93436735,758.9506962)(176.33189942,758.62483304)(175.60439394,758.62483304)
\curveto(174.79352219,758.62483304)(174.15126864,758.96774718)(173.67763137,759.65357648)
\curveto(173.3062979,760.19541768)(173.12063168,760.86987863)(173.12063216,761.67696136)
\curveto(173.12063168,762.68864974)(173.36313449,763.46162746)(173.84814133,763.99589685)
\curveto(174.33693486,764.53015549)(174.9431919,764.79728749)(175.66691426,764.79729367)
\curveto(176.41336521,764.79728749)(177.02909501,764.46952978)(177.51410551,763.81401955)
\lineto(177.51410551,764.6608857)
\lineto(179.00890951,764.6608857)
\lineto(179.00890951,759.24435257)
\curveto(179.00890314,758.53199994)(178.95017199,757.99963048)(178.83271588,757.6472426)
\curveto(178.71524739,757.29485667)(178.55042126,757.0182519)(178.33823699,756.81742745)
\curveto(178.12604134,756.61660661)(177.84185835,756.45935869)(177.48568718,756.34568322)
\curveto(177.13329544,756.2320123)(176.68618087,756.17517571)(176.14434215,756.17517326)
\curveto(175.1212799,756.17517571)(174.39566601,756.35136916)(173.9674983,756.70375414)
\curveto(173.53932795,757.05235386)(173.32524343,757.49567931)(173.32524411,758.03373184)
\curveto(173.32524343,758.08677992)(173.32713798,758.15119474)(173.33092778,758.22697647)
\moveto(174.75752779,761.76790001)
\curveto(174.75752568,761.12753787)(174.88067164,760.65768867)(175.12696604,760.35835099)
\curveto(175.37704459,760.06279895)(175.68396221,759.9150238)(176.04771984,759.91502509)
\curveto(176.4379944,759.9150238)(176.76764666,760.06658806)(177.03667762,760.36971832)
\curveto(177.30569978,760.6766342)(177.44021306,761.12943243)(177.44021786,761.72811435)
\curveto(177.44021306,762.35331382)(177.31138344,762.81747936)(177.05372861,763.12061237)
\curveto(176.79606496,763.4237364)(176.47020181,763.57530066)(176.07613817,763.57530561)
\curveto(175.69343498,763.57530066)(175.37704459,763.42563095)(175.12696604,763.12629604)
\curveto(174.88067164,762.83074123)(174.75752568,762.37794301)(174.75752779,761.76790001)
}
}
{
\newrgbcolor{curcolor}{0 0 0}
\pscustom[linestyle=none,fillstyle=solid,fillcolor=curcolor]
{
\newpath
\moveto(184.09010555,760.54591195)
\lineto(185.68153186,760.27877968)
\curveto(185.47691419,759.69525562)(185.15294558,759.25003561)(184.70962508,758.94311831)
\curveto(184.27008377,758.63998947)(183.71876878,758.48842521)(183.05567844,758.48842508)
\curveto(182.00609265,758.48842521)(181.22932583,758.83133935)(180.72537563,759.51716851)
\curveto(180.32751849,760.06658806)(180.1285904,760.75999454)(180.12859076,761.59739005)
\curveto(180.1285904,762.59771119)(180.39003874,763.38016167)(180.91293659,763.94474386)
\curveto(181.43583213,764.51310451)(182.09703121,764.79728749)(182.89653581,764.79729367)
\curveto(183.79455091,764.79728749)(184.50311382,764.49984264)(185.02222667,763.9049582)
\curveto(185.541329,763.31385231)(185.78951547,762.40636131)(185.76678684,761.18248247)
\lineto(181.7654864,761.18248247)
\curveto(181.77685171,760.70884161)(181.90568133,760.33940373)(182.15197565,760.07416772)
\curveto(182.39826517,759.81271792)(182.7051828,759.68199375)(183.07272944,759.68199481)
\curveto(183.32280715,759.68199375)(183.53310256,759.75019767)(183.7036163,759.88660676)
\curveto(183.87412215,760.02301333)(184.00295177,760.24278151)(184.09010555,760.54591195)
\moveto(184.18104419,762.16007292)
\curveto(184.16967245,762.62234038)(184.0503156,762.97283273)(183.82297327,763.21155102)
\curveto(183.59562282,763.45404925)(183.31901805,763.57530066)(182.99315813,763.57530561)
\curveto(182.6445571,763.57530066)(182.356585,763.44836559)(182.12924098,763.19450002)
\curveto(181.90189223,762.94062532)(181.79011359,762.59581663)(181.79390473,762.16007292)
\lineto(184.18104419,762.16007292)
}
}
{
\newrgbcolor{curcolor}{0 0 0}
\pscustom[linestyle=none,fillstyle=solid,fillcolor=curcolor]
{
\newpath
\moveto(186.51134552,760.34698366)
\lineto(188.11413917,760.59138127)
\curveto(188.18234121,760.28067257)(188.32064359,760.04385342)(188.52904674,759.8809231)
\curveto(188.73744531,759.72177937)(189.02920651,759.64220813)(189.40433121,759.64220915)
\curveto(189.81734065,759.64220813)(190.12804738,759.71799026)(190.33645234,759.86955577)
\curveto(190.47664518,759.9756495)(190.54674365,760.117741)(190.54674796,760.29583067)
\curveto(190.54674365,760.41708041)(190.50885258,760.51749173)(190.43307465,760.59706494)
\curveto(190.35349922,760.6728451)(190.17541121,760.74294356)(189.8988101,760.80736056)
\curveto(188.61051024,761.09154136)(187.79395779,761.35109515)(187.44915032,761.58602272)
\curveto(186.97172169,761.91188291)(186.73300798,762.36468114)(186.73300848,762.94441875)
\curveto(186.73300798,763.46731112)(186.93951428,763.90684747)(187.352528,764.26302912)
\curveto(187.7655395,764.61919949)(188.40589849,764.79728749)(189.27360691,764.79729367)
\curveto(190.09962909,764.79728749)(190.71346433,764.66277421)(191.1151145,764.39375342)
\curveto(191.51675491,764.1247211)(191.79335968,763.72686492)(191.94492965,763.20018369)
\lineto(190.43875832,762.92168409)
\curveto(190.3743393,763.15660439)(190.25119334,763.33658695)(190.06932006,763.4616323)
\curveto(189.89122823,763.58666798)(189.63546354,763.64918823)(189.30202524,763.64919326)
\curveto(188.88143135,763.64918823)(188.58019739,763.59045708)(188.39832244,763.47299963)
\curveto(188.27706887,763.38963444)(188.21644317,763.2816449)(188.21644514,763.1490307)
\curveto(188.21644317,763.03535298)(188.26949066,762.93873077)(188.37558778,762.85916377)
\curveto(188.51957168,762.75306455)(189.01594463,762.60339484)(189.86470811,762.4101542)
\curveto(190.71725344,762.21690598)(191.31214316,761.98008683)(191.64937905,761.69969603)
\curveto(191.982815,761.41550996)(192.14953569,761.01954834)(192.1495416,760.51180996)
\curveto(192.14953569,759.95859852)(191.91840019,759.48306566)(191.45613442,759.08520994)
\curveto(190.99385821,758.6873533)(190.30992449,758.48842521)(189.40433121,758.48842508)
\curveto(188.58209194,758.48842521)(187.93036563,758.6551459)(187.44915032,758.98858763)
\curveto(186.97172169,759.32202864)(186.6591204,759.77482686)(186.51134552,760.34698366)
}
}
{
\newrgbcolor{curcolor}{1 0.50196081 0.50196081}
\pscustom[linewidth=2.32802935,linecolor=curcolor]
{
\newpath
\moveto(352.515295,660.25842368)
\lineto(427.01223194,660.25842368)
\lineto(427.01223194,706.81900578)
\lineto(378.12362028,706.81900578)
\lineto(378.12362028,716.13112406)
\lineto(352.515295,716.13112406)
\lineto(352.515295,660.25842368)
\closepath
}
}
{
\newrgbcolor{curcolor}{0 0 0}
\pscustom[linestyle=none,fillstyle=solid,fillcolor=curcolor]
{
\newpath
\moveto(363.62800523,688.93511414)
\lineto(363.62800523,689.88996993)
\lineto(369.14684433,692.22027274)
\lineto(369.14684433,691.20289663)
\lineto(364.77042198,689.40685837)
\lineto(369.14684433,687.59376911)
\lineto(369.14684433,686.576393)
\lineto(363.62800523,688.93511414)
}
}
{
\newrgbcolor{curcolor}{0 0 0}
\pscustom[linestyle=none,fillstyle=solid,fillcolor=curcolor]
{
\newpath
\moveto(370.42566923,688.93511414)
\lineto(370.42566923,689.88996993)
\lineto(375.94450833,692.22027274)
\lineto(375.94450833,691.20289663)
\lineto(371.56808597,689.40685837)
\lineto(375.94450833,687.59376911)
\lineto(375.94450833,686.576393)
\lineto(370.42566923,688.93511414)
}
}
{
\newrgbcolor{curcolor}{0 0 0}
\pscustom[linestyle=none,fillstyle=solid,fillcolor=curcolor]
{
\newpath
\moveto(378.29754598,685.29188462)
\lineto(377.34837386,685.29188462)
\lineto(377.34837386,693.62413809)
\lineto(378.37143363,693.62413809)
\lineto(378.37143363,690.65158109)
\curveto(378.80338998,691.19341796)(379.35470498,691.46433907)(380.02538026,691.46434524)
\curveto(380.39670926,691.46433907)(380.74720161,691.38855694)(381.07685836,691.23699863)
\curveto(381.41029524,691.08921753)(381.68311091,690.87892212)(381.89530618,690.60611177)
\curveto(382.11127994,690.33707989)(382.27989518,690.01121674)(382.4011524,689.62852132)
\curveto(382.52239799,689.24581723)(382.58302369,688.83659373)(382.58302969,688.40084959)
\curveto(382.58302369,687.36642042)(382.32725901,686.56691895)(381.81573486,686.00234279)
\curveto(381.30420026,685.43776522)(380.69036501,685.15547679)(379.97422727,685.15547665)
\curveto(379.26187187,685.15547679)(378.70297866,685.45292164)(378.29754598,686.04781212)
\lineto(378.29754598,685.29188462)
\moveto(378.28617865,688.35538027)
\curveto(378.28617695,687.63165787)(378.38469372,687.10876117)(378.58172925,686.78668862)
\curveto(378.90380131,686.26000132)(379.33954855,685.99665842)(379.88897229,685.99665913)
\curveto(380.33608355,685.99665842)(380.72257241,686.18990285)(381.04844003,686.576393)
\curveto(381.37429873,686.96666968)(381.53723031,687.54640297)(381.53723526,688.31559461)
\curveto(381.53723031,689.10372573)(381.37998239,689.68535358)(381.06549103,690.06047989)
\curveto(380.75477982,690.43559666)(380.37776372,690.62315743)(379.93444161,690.62316276)
\curveto(379.4873237,690.62315743)(379.10083484,690.42801845)(378.77497387,690.03774523)
\curveto(378.44910853,689.65125162)(378.28617695,689.09046386)(378.28617865,688.35538027)
}
}
{
\newrgbcolor{curcolor}{0 0 0}
\pscustom[linestyle=none,fillstyle=solid,fillcolor=curcolor]
{
\newpath
\moveto(387.77221561,686.03644479)
\curveto(387.39330025,685.71436999)(387.02765148,685.4870236)(386.67526818,685.35440494)
\curveto(386.32666678,685.22178615)(385.95154524,685.15547679)(385.54990243,685.15547665)
\curveto(384.88680632,685.15547679)(384.3771715,685.31651381)(384.02099644,685.63858821)
\curveto(383.66481948,685.96445102)(383.48673147,686.37935818)(383.4867319,686.88331093)
\curveto(383.48673147,687.17885964)(383.55304084,687.4478862)(383.68566018,687.69039142)
\curveto(383.8220674,687.93668094)(383.99826085,688.13371448)(384.21424107,688.28149262)
\curveto(384.43400809,688.42926478)(384.68030001,688.54104342)(384.95311757,688.61682888)
\curveto(385.15393832,688.66987304)(385.45706684,688.72102598)(385.86250403,688.77028784)
\curveto(386.68852645,688.86880113)(387.29667804,688.98626343)(387.68696062,689.1226751)
\curveto(387.69074511,689.26286821)(387.69263966,689.35191221)(387.69264429,689.38980737)
\curveto(387.69263966,689.80660499)(387.59601745,690.10026074)(387.40277735,690.27077551)
\curveto(387.14132467,690.50190602)(386.75294126,690.61747377)(386.23762595,690.6174791)
\curveto(385.75640625,690.61747377)(385.40023025,690.53221888)(385.16909685,690.36171415)
\curveto(384.94174836,690.1949884)(384.77313312,689.89754354)(384.66325063,689.46937869)
\lineto(383.66292552,689.60578666)
\curveto(383.75386348,690.03395137)(383.90353319,690.37876006)(384.11193509,690.64021376)
\curveto(384.3203349,690.90544586)(384.62156886,691.10816306)(385.01563789,691.24836596)
\curveto(385.40970301,691.39234605)(385.86629034,691.46433907)(386.38540125,691.46434524)
\curveto(386.90071641,691.46433907)(387.31941268,691.40371337)(387.6414913,691.28246795)
\curveto(387.96356078,691.16121055)(388.20037993,691.00775174)(388.35194948,690.82209105)
\curveto(388.50350845,690.64020841)(388.60960343,690.40907292)(388.67023474,690.12868387)
\curveto(388.70433109,689.95438014)(388.72138207,689.6398843)(388.72138773,689.18519542)
\lineto(388.72138773,687.82111572)
\curveto(388.72138207,686.87004747)(388.74222216,686.26757954)(388.78390805,686.01371012)
\curveto(388.82937161,685.76362838)(388.91652106,685.52302011)(389.04535665,685.29188462)
\lineto(387.97682756,685.29188462)
\curveto(387.87072767,685.50407458)(387.80252375,685.75226106)(387.77221561,686.03644479)
\moveto(387.68696062,688.32127828)
\curveto(387.31562357,688.16971099)(386.75862492,688.04088137)(386.015963,687.93478903)
\curveto(385.59536923,687.87416068)(385.29792437,687.80595677)(385.12362753,687.73017708)
\curveto(384.94932657,687.65439251)(384.81481329,687.54261387)(384.72008729,687.39484082)
\curveto(384.62535797,687.25085267)(384.57799414,687.08981564)(384.57799565,686.91172926)
\curveto(384.57799414,686.63891197)(384.68030001,686.41156558)(384.88491358,686.22968941)
\curveto(385.09331262,686.04781136)(385.39644114,685.95687281)(385.79430005,685.95687347)
\curveto(386.18836439,685.95687281)(386.53885674,686.0421277)(386.84577815,686.21263841)
\curveto(387.15269199,686.38693639)(387.37814383,686.62375555)(387.52213433,686.92309659)
\curveto(387.63201396,687.15423045)(387.686956,687.49525004)(387.68696062,687.94615636)
\lineto(387.68696062,688.32127828)
}
}
{
\newrgbcolor{curcolor}{0 0 0}
\pscustom[linestyle=none,fillstyle=solid,fillcolor=curcolor]
{
\newpath
\moveto(389.90359043,687.09360655)
\lineto(390.91528287,687.25274918)
\curveto(390.97211809,686.84731283)(391.12936601,686.5366061)(391.38702709,686.32062806)
\curveto(391.6484736,686.10464796)(392.01222782,685.99665842)(392.47829085,685.99665913)
\curveto(392.94813712,685.99665842)(393.29673492,686.09138609)(393.52408528,686.2808424)
\curveto(393.75142769,686.47408584)(393.86510089,686.69953768)(393.86510521,686.95719858)
\curveto(393.86510089,687.18833241)(393.76468957,687.37020952)(393.56387094,687.50283046)
\curveto(393.42366998,687.5937668)(393.07507219,687.70933455)(392.51807651,687.84953405)
\curveto(391.76783045,688.03898681)(391.24682831,688.20191839)(390.95506852,688.33832927)
\curveto(390.66709502,688.47852317)(390.44732685,688.66987304)(390.29576334,688.91237948)
\curveto(390.14798743,689.15866778)(390.07409986,689.42958889)(390.07410039,689.72514363)
\curveto(390.07409986,689.99416576)(390.13472556,690.24235223)(390.25597768,690.4697038)
\curveto(390.38101748,690.70083411)(390.54963272,690.89218399)(390.7618239,691.043754)
\curveto(390.92096516,691.16121055)(391.13694423,691.25972732)(391.40976176,691.3393046)
\curveto(391.68636466,691.4226589)(391.98191497,691.46433907)(392.29641356,691.46434524)
\curveto(392.77004912,691.46433907)(393.18495628,691.39613515)(393.54113628,691.25973329)
\curveto(393.9010974,691.12331949)(394.16633485,690.93765327)(394.33684944,690.70273408)
\curveto(394.50735444,690.47159317)(394.62481674,690.16088644)(394.68923669,689.77061295)
\lineto(393.68891158,689.63420498)
\curveto(393.64343816,689.94490737)(393.51081943,690.18741019)(393.291055,690.36171415)
\curveto(393.07507219,690.53600798)(392.76815456,690.62315743)(392.37030121,690.62316276)
\curveto(391.90044918,690.62315743)(391.56511326,690.54548075)(391.36429243,690.39013248)
\curveto(391.16346797,690.23477402)(391.06305665,690.05289691)(391.06305817,689.8445006)
\curveto(391.06305665,689.71187732)(391.10473682,689.59252047)(391.18809881,689.48642968)
\curveto(391.2714575,689.3765414)(391.40218168,689.28560285)(391.58027172,689.21361374)
\curveto(391.68257556,689.17571876)(391.98380952,689.08856931)(392.48397452,688.95216514)
\curveto(393.20769091,688.75891705)(393.71164208,688.59977457)(393.99582951,688.47473724)
\curveto(394.28379715,688.35348265)(394.50924899,688.17539465)(394.67218569,687.94047269)
\curveto(394.83511215,687.70554544)(394.91657794,687.41378425)(394.91658331,687.06518822)
\curveto(394.91657794,686.72416687)(394.81616661,686.40209282)(394.61534904,686.09896511)
\curveto(394.41831043,685.79962489)(394.13223289,685.56659484)(393.75711557,685.39987426)
\curveto(393.38198981,685.23694258)(392.95760989,685.15547679)(392.48397452,685.15547665)
\curveto(391.69962654,685.15547679)(391.10094771,685.31840836)(390.68793625,685.64427187)
\curveto(390.27871161,685.97013468)(390.01726326,686.45324575)(389.90359043,687.09360655)
}
}
{
\newrgbcolor{curcolor}{0 0 0}
\pscustom[linestyle=none,fillstyle=solid,fillcolor=curcolor]
{
\newpath
\moveto(400.26491156,687.23569819)
\lineto(401.32207332,687.10497388)
\curveto(401.15534668,686.48734771)(400.8465345,686.00802574)(400.39563586,685.66700654)
\curveto(399.94472716,685.32598658)(399.36878298,685.15547679)(398.66780158,685.15547665)
\curveto(397.78493647,685.15547679)(397.08395177,685.4263979)(396.56484538,685.9682408)
\curveto(396.0495257,686.51387146)(395.79186646,687.27737641)(395.79186689,688.25875796)
\curveto(395.79186646,689.27423553)(396.05331481,690.06236967)(396.57621271,690.62316276)
\curveto(397.0991082,691.18394519)(397.77735826,691.46433907)(398.61096493,691.46434524)
\curveto(399.41804136,691.46433907)(400.07734589,691.18962885)(400.58888049,690.64021376)
\curveto(401.10040464,690.09078797)(401.35616932,689.31781025)(401.35617531,688.32127828)
\curveto(401.35616932,688.26064954)(401.35427477,688.16971099)(401.35049165,688.04846234)
\lineto(396.84902865,688.04846234)
\curveto(396.88691823,687.38536595)(397.07447901,686.87762568)(397.41171153,686.52524001)
\curveto(397.74893996,686.17285187)(398.16953078,685.99665842)(398.67348525,685.99665913)
\curveto(399.04860348,685.99665842)(399.36878298,686.09517519)(399.6340247,686.29220973)
\curveto(399.89925788,686.48924227)(400.10955329,686.8037381)(400.26491156,687.23569819)
\moveto(396.90586531,688.88964482)
\lineto(400.27627889,688.88964482)
\curveto(400.2308047,689.39738149)(400.10197508,689.77818669)(399.88978964,690.03206156)
\curveto(399.56392196,690.42612389)(399.14143659,690.62315743)(398.62233226,690.62316276)
\curveto(398.1524798,690.62315743)(397.75651817,690.46590951)(397.43444619,690.15141853)
\curveto(397.11615918,689.83691784)(396.93996573,689.41632702)(396.90586531,688.88964482)
}
}
{
\newrgbcolor{curcolor}{0 0 0}
\pscustom[linestyle=none,fillstyle=solid,fillcolor=curcolor]
{
\newpath
\moveto(408.00038224,688.93511414)
\lineto(402.48154314,686.576393)
\lineto(402.48154314,687.59376911)
\lineto(406.85228183,689.40685837)
\lineto(402.48154314,691.20289663)
\lineto(402.48154314,692.22027274)
\lineto(408.00038224,689.88996993)
\lineto(408.00038224,688.93511414)
}
}
{
\newrgbcolor{curcolor}{0 0 0}
\pscustom[linestyle=none,fillstyle=solid,fillcolor=curcolor]
{
\newpath
\moveto(414.7980449,688.93511414)
\lineto(409.2792058,686.576393)
\lineto(409.2792058,687.59376911)
\lineto(413.64994449,689.40685837)
\lineto(409.2792058,691.20289663)
\lineto(409.2792058,692.22027274)
\lineto(414.7980449,689.88996993)
\lineto(414.7980449,688.93511414)
}
}
{
\newrgbcolor{curcolor}{0 0 0}
\pscustom[linestyle=none,fillstyle=solid,fillcolor=curcolor]
{
\newpath
\moveto(374.66793558,667.83168317)
\lineto(373.0708256,667.83168317)
\lineto(373.0708256,673.86773583)
\lineto(374.55426227,673.86773583)
\lineto(374.55426227,673.00950235)
\curveto(374.80813015,673.41493157)(375.03547654,673.68206357)(375.23630212,673.81089917)
\curveto(375.44091094,673.93972281)(375.67204643,674.00413762)(375.9297093,674.0041438)
\curveto(376.29345989,674.00413762)(376.64395224,673.9037263)(376.9811874,673.70290953)
\lineto(376.48670851,672.31041151)
\curveto(376.21767776,672.48470593)(375.96759674,672.57185538)(375.73646467,672.57186012)
\curveto(375.51290396,672.57185538)(375.32344863,672.50933512)(375.16809813,672.38429916)
\curveto(375.0127419,672.2630432)(374.88959594,672.04138047)(374.79865988,671.71931031)
\curveto(374.71150794,671.39723237)(374.66793321,670.72277142)(374.66793558,669.69592542)
\lineto(374.66793558,667.83168317)
}
}
{
\newrgbcolor{curcolor}{0 0 0}
\pscustom[linestyle=none,fillstyle=solid,fillcolor=curcolor]
{
\newpath
\moveto(377.29378893,670.93496448)
\curveto(377.29378847,671.46543629)(377.42451264,671.97886021)(377.68596185,672.4752378)
\curveto(377.94740933,672.97160611)(378.31684722,673.35051676)(378.7942766,673.61197088)
\curveto(379.27549115,673.87341345)(379.81164972,674.00413762)(380.40275391,674.0041438)
\curveto(381.31592499,674.00413762)(382.06427352,673.70669277)(382.64780174,673.11180833)
\curveto(383.23131832,672.52070244)(383.52307951,671.77235391)(383.52308621,670.8667605)
\curveto(383.52307951,669.9535828)(383.22752921,669.19576151)(382.63643441,668.59329434)
\curveto(382.0491171,667.99461475)(381.30834678,667.69527534)(380.41412124,667.6952752)
\curveto(379.86090811,667.69527534)(379.33232775,667.82031585)(378.82837859,668.07039712)
\curveto(378.32821454,668.32047791)(377.94740933,668.68612668)(377.68596185,669.16734454)
\curveto(377.42451264,669.65234884)(377.29378847,670.24155489)(377.29378893,670.93496448)
\moveto(378.93068457,670.8497095)
\curveto(378.93068247,670.25102766)(379.07277396,669.79254578)(379.35695947,669.47426247)
\curveto(379.64113993,669.15597589)(379.99163228,668.99683342)(380.40843757,668.99683458)
\curveto(380.8252357,668.99683342)(381.1738335,669.15597589)(381.45423201,669.47426247)
\curveto(381.73841036,669.79254578)(381.88050186,670.25481677)(381.88050691,670.86107683)
\curveto(381.88050186,671.45217441)(381.73841036,671.90686719)(381.45423201,672.22515653)
\curveto(381.1738335,672.54343708)(380.8252357,672.70257955)(380.40843757,672.70258442)
\curveto(379.99163228,672.70257955)(379.64113993,672.54343708)(379.35695947,672.22515653)
\curveto(379.07277396,671.90686719)(378.93068247,671.44838531)(378.93068457,670.8497095)
}
}
{
\newrgbcolor{curcolor}{0 0 0}
\pscustom[linestyle=none,fillstyle=solid,fillcolor=curcolor]
{
\newpath
\moveto(388.75205844,667.83168317)
\lineto(388.75205844,668.73538597)
\curveto(388.53228545,668.41331102)(388.24241881,668.15944088)(387.88245763,667.97377481)
\curveto(387.52627768,667.78810845)(387.14926159,667.69527534)(386.75140822,667.6952752)
\curveto(386.34597101,667.69527534)(385.98221679,667.78431934)(385.66014446,667.96240748)
\curveto(385.33806869,668.14049535)(385.10503864,668.39057638)(384.96105361,668.71265131)
\curveto(384.81706655,669.03472448)(384.74507353,669.47994449)(384.74507433,670.04831268)
\lineto(384.74507433,673.86773583)
\lineto(386.34218431,673.86773583)
\lineto(386.34218431,671.09410711)
\curveto(386.34218191,670.245344)(386.37060021,669.72434186)(386.42743929,669.53109913)
\curveto(386.48806251,669.34164211)(386.59605204,669.19007785)(386.75140822,669.0764059)
\curveto(386.90675877,668.96652056)(387.10379231,668.91157852)(387.34250942,668.9115796)
\curveto(387.61532168,668.91157852)(387.85971905,668.9854661)(388.07570225,669.13324255)
\curveto(388.29167719,669.28480551)(388.43945234,669.47047173)(388.51902815,669.69024176)
\curveto(388.59859481,669.91379718)(388.63838043,670.45753396)(388.63838513,671.32145373)
\lineto(388.63838513,673.86773583)
\lineto(390.2354951,673.86773583)
\lineto(390.2354951,667.83168317)
\lineto(388.75205844,667.83168317)
}
}
{
\newrgbcolor{curcolor}{0 0 0}
\pscustom[linestyle=none,fillstyle=solid,fillcolor=curcolor]
{
\newpath
\moveto(394.6630714,673.86773583)
\lineto(394.6630714,672.59459478)
\lineto(393.57180764,672.59459478)
\lineto(393.57180764,670.16198599)
\curveto(393.57180513,669.66939982)(393.58127789,669.38142772)(393.60022597,669.29806885)
\curveto(393.62295807,669.21849614)(393.6703219,669.15218678)(393.7423176,669.09914056)
\curveto(393.81809705,669.0460918)(393.90903561,669.01956805)(394.01513354,669.01956924)
\curveto(394.16290574,669.01956805)(394.37699025,669.07072099)(394.65738773,669.17302821)
\lineto(394.7937957,667.93398915)
\curveto(394.42245953,667.77484658)(394.00186871,667.69527534)(393.53202198,667.6952752)
\curveto(393.24404742,667.69527534)(392.98449362,667.74263917)(392.75335982,667.83736684)
\curveto(392.52222263,667.9358836)(392.35171284,668.06092412)(392.24182994,668.21248876)
\curveto(392.13573377,668.36784174)(392.0618462,668.5762426)(392.02016699,668.83769195)
\curveto(391.98606407,669.02335716)(391.96901309,669.3984787)(391.969014,669.9630577)
\lineto(391.969014,672.59459478)
\lineto(391.23582116,672.59459478)
\lineto(391.23582116,673.86773583)
\lineto(391.969014,673.86773583)
\lineto(391.969014,675.06698923)
\lineto(393.57180764,675.99911035)
\lineto(393.57180764,673.86773583)
\lineto(394.6630714,673.86773583)
}
}
{
\newrgbcolor{curcolor}{0 0 0}
\pscustom[linestyle=none,fillstyle=solid,fillcolor=curcolor]
{
\newpath
\moveto(399.2782061,669.75276208)
\lineto(400.86963241,669.48562981)
\curveto(400.66501474,668.90210575)(400.34104614,668.45688574)(399.89772563,668.14996844)
\curveto(399.45818433,667.8468396)(398.90686934,667.69527534)(398.243779,667.6952752)
\curveto(397.19419321,667.69527534)(396.41742638,668.03818948)(395.91347618,668.72401864)
\curveto(395.51561904,669.27343819)(395.31669095,669.96684467)(395.31669132,670.80424018)
\curveto(395.31669095,671.80456131)(395.5781393,672.5870118)(396.10103714,673.15159399)
\curveto(396.62393268,673.71995464)(397.28513176,674.00413762)(398.08463637,674.0041438)
\curveto(398.98265146,674.00413762)(399.69121438,673.70669277)(400.21032723,673.11180833)
\curveto(400.72942955,672.52070244)(400.97761602,671.61321144)(400.95488739,670.3893326)
\lineto(396.95358695,670.3893326)
\curveto(396.96495227,669.91569174)(397.09378189,669.54625385)(397.3400762,669.28101785)
\curveto(397.58636573,669.01956805)(397.89328335,668.88884388)(398.26082999,668.88884494)
\curveto(398.51090771,668.88884388)(398.72120312,668.9570478)(398.89171685,669.09345689)
\curveto(399.0622227,669.22986346)(399.19105232,669.44963164)(399.2782061,669.75276208)
\moveto(399.36914475,671.36692305)
\curveto(399.35777301,671.82919051)(399.23841615,672.17968286)(399.01107383,672.41840115)
\curveto(398.78372337,672.66089938)(398.5071186,672.78215079)(398.18125868,672.78215574)
\curveto(397.83265765,672.78215079)(397.54468556,672.65521572)(397.31734154,672.40135015)
\curveto(397.08999278,672.14747545)(396.97821414,671.80266676)(396.98200528,671.36692305)
\lineto(399.36914475,671.36692305)
}
}
{
\newrgbcolor{curcolor}{0 0 0}
\pscustom[linestyle=none,fillstyle=solid,fillcolor=curcolor]
{
\newpath
\moveto(401.69944786,669.55383379)
\lineto(403.3022415,669.7982314)
\curveto(403.37044354,669.4875227)(403.50874593,669.25070355)(403.71714907,669.08777323)
\curveto(403.92554764,668.9286295)(404.21730884,668.84905826)(404.59243354,668.84905928)
\curveto(405.00544298,668.84905826)(405.31614971,668.92484039)(405.52455467,669.0764059)
\curveto(405.66474751,669.18249963)(405.73484598,669.32459113)(405.73485029,669.5026808)
\curveto(405.73484598,669.62393054)(405.69695492,669.72434186)(405.62117698,669.80391507)
\curveto(405.54160155,669.87969522)(405.36351355,669.94979369)(405.08691243,670.01421069)
\curveto(403.79861257,670.29839149)(402.98206012,670.55794528)(402.63725265,670.79287285)
\curveto(402.15982402,671.11873304)(401.92111031,671.57153127)(401.92111081,672.15126888)
\curveto(401.92111031,672.67416125)(402.12761661,673.1136976)(402.54063033,673.46987925)
\curveto(402.95364183,673.82604962)(403.59400082,674.00413762)(404.46170924,674.0041438)
\curveto(405.28773142,674.00413762)(405.90156667,673.86962434)(406.30321683,673.60060355)
\curveto(406.70485724,673.33157122)(406.98146201,672.93371504)(407.13303198,672.40703382)
\lineto(405.62686065,672.12853422)
\curveto(405.56244164,672.36345452)(405.43929567,672.54343708)(405.25742239,672.66848243)
\curveto(405.07933056,672.79351811)(404.82356587,672.85603836)(404.49012757,672.85604339)
\curveto(404.06953368,672.85603836)(403.76829972,672.79730721)(403.58642477,672.67984976)
\curveto(403.4651712,672.59648457)(403.4045455,672.48849503)(403.40454747,672.35588083)
\curveto(403.4045455,672.24220311)(403.45759299,672.1455809)(403.56369011,672.0660139)
\curveto(403.70767402,671.95991468)(404.20404696,671.81024497)(405.05281044,671.61700433)
\curveto(405.90535577,671.42375611)(406.50024549,671.18693696)(406.83748138,670.90654615)
\curveto(407.17091733,670.62236009)(407.33763802,670.22639847)(407.33764393,669.71866009)
\curveto(407.33763802,669.16544865)(407.10650252,668.68991579)(406.64423675,668.29206007)
\curveto(406.18196054,667.89420343)(405.49802683,667.69527534)(404.59243354,667.6952752)
\curveto(403.77019427,667.69527534)(403.11846796,667.86199603)(402.63725265,668.19543776)
\curveto(402.15982402,668.52887877)(401.84722274,668.98167699)(401.69944786,669.55383379)
}
}
{
\newrgbcolor{curcolor}{1 0.50196081 0.50196081}
\pscustom[linewidth=2.32802935,linecolor=curcolor]
{
\newpath
\moveto(352.515295,752.21558706)
\lineto(427.01223194,752.21558706)
\lineto(427.01223194,798.77616916)
\lineto(378.12362028,798.77616916)
\lineto(378.12362028,808.08828744)
\lineto(352.515295,808.08828744)
\lineto(352.515295,752.21558706)
\closepath
}
}
{
\newrgbcolor{curcolor}{0 0 0}
\pscustom[linestyle=none,fillstyle=solid,fillcolor=curcolor]
{
\newpath
\moveto(363.62800523,779.72827823)
\lineto(363.62800523,780.68313401)
\lineto(369.14684433,783.01343683)
\lineto(369.14684433,781.99606072)
\lineto(364.77042198,780.20002245)
\lineto(369.14684433,778.38693319)
\lineto(369.14684433,777.36955709)
\lineto(363.62800523,779.72827823)
}
}
{
\newrgbcolor{curcolor}{0 0 0}
\pscustom[linestyle=none,fillstyle=solid,fillcolor=curcolor]
{
\newpath
\moveto(370.42566923,779.72827823)
\lineto(370.42566923,780.68313401)
\lineto(375.94450833,783.01343683)
\lineto(375.94450833,781.99606072)
\lineto(371.56808597,780.20002245)
\lineto(375.94450833,778.38693319)
\lineto(375.94450833,777.36955709)
\lineto(370.42566923,779.72827823)
}
}
{
\newrgbcolor{curcolor}{0 0 0}
\pscustom[linestyle=none,fillstyle=solid,fillcolor=curcolor]
{
\newpath
\moveto(378.29754598,776.08504871)
\lineto(377.34837386,776.08504871)
\lineto(377.34837386,784.41730218)
\lineto(378.37143363,784.41730218)
\lineto(378.37143363,781.44474518)
\curveto(378.80338998,781.98658204)(379.35470498,782.25750316)(380.02538026,782.25750933)
\curveto(380.39670926,782.25750316)(380.74720161,782.18172103)(381.07685836,782.03016271)
\curveto(381.41029524,781.88238161)(381.68311091,781.67208621)(381.89530618,781.39927585)
\curveto(382.11127994,781.13024398)(382.27989518,780.80438082)(382.4011524,780.4216854)
\curveto(382.52239799,780.03898131)(382.58302369,779.62975781)(382.58302969,779.19401368)
\curveto(382.58302369,778.1595845)(382.32725901,777.36008304)(381.81573486,776.79550688)
\curveto(381.30420026,776.23092931)(380.69036501,775.94864087)(379.97422727,775.94864074)
\curveto(379.26187187,775.94864087)(378.70297866,776.24608573)(378.29754598,776.8409762)
\lineto(378.29754598,776.08504871)
\moveto(378.28617865,779.14854436)
\curveto(378.28617695,778.42482195)(378.38469372,777.90192526)(378.58172925,777.57985271)
\curveto(378.90380131,777.05316541)(379.33954855,776.78982251)(379.88897229,776.78982322)
\curveto(380.33608355,776.78982251)(380.72257241,776.98306694)(381.04844003,777.36955709)
\curveto(381.37429873,777.75983377)(381.53723031,778.33956706)(381.53723526,779.1087587)
\curveto(381.53723031,779.89688982)(381.37998239,780.47851767)(381.06549103,780.85364397)
\curveto(380.75477982,781.22876075)(380.37776372,781.41632152)(379.93444161,781.41632685)
\curveto(379.4873237,781.41632152)(379.10083484,781.22118253)(378.77497387,780.83090931)
\curveto(378.44910853,780.44441571)(378.28617695,779.88362795)(378.28617865,779.14854436)
}
}
{
\newrgbcolor{curcolor}{0 0 0}
\pscustom[linestyle=none,fillstyle=solid,fillcolor=curcolor]
{
\newpath
\moveto(387.77221561,776.82960887)
\curveto(387.39330025,776.50753408)(387.02765148,776.28018769)(386.67526818,776.14756903)
\curveto(386.32666678,776.01495024)(385.95154524,775.94864087)(385.54990243,775.94864074)
\curveto(384.88680632,775.94864087)(384.3771715,776.1096779)(384.02099644,776.4317523)
\curveto(383.66481948,776.75761511)(383.48673147,777.17252226)(383.4867319,777.67647502)
\curveto(383.48673147,777.97202373)(383.55304084,778.24105029)(383.68566018,778.4835555)
\curveto(383.8220674,778.72984503)(383.99826085,778.92687856)(384.21424107,779.07465671)
\curveto(384.43400809,779.22242887)(384.68030001,779.33420751)(384.95311757,779.40999296)
\curveto(385.15393832,779.46303713)(385.45706684,779.51419007)(385.86250403,779.56345193)
\curveto(386.68852645,779.66196522)(387.29667804,779.77942752)(387.68696062,779.91583918)
\curveto(387.69074511,780.05603229)(387.69263966,780.1450763)(387.69264429,780.18297146)
\curveto(387.69263966,780.59976907)(387.59601745,780.89342482)(387.40277735,781.06393959)
\curveto(387.14132467,781.29507011)(386.75294126,781.41063786)(386.23762595,781.41064318)
\curveto(385.75640625,781.41063786)(385.40023025,781.32538296)(385.16909685,781.15487824)
\curveto(384.94174836,780.98815249)(384.77313312,780.69070763)(384.66325063,780.26254277)
\lineto(383.66292552,780.39895074)
\curveto(383.75386348,780.82711546)(383.90353319,781.17192415)(384.11193509,781.43337785)
\curveto(384.3203349,781.69860995)(384.62156886,781.90132715)(385.01563789,782.04153004)
\curveto(385.40970301,782.18551013)(385.86629034,782.25750316)(386.38540125,782.25750933)
\curveto(386.90071641,782.25750316)(387.31941268,782.19687745)(387.6414913,782.07563204)
\curveto(387.96356078,781.95437464)(388.20037993,781.80091583)(388.35194948,781.61525514)
\curveto(388.50350845,781.4333725)(388.60960343,781.202237)(388.67023474,780.92184796)
\curveto(388.70433109,780.74754423)(388.72138207,780.43304839)(388.72138773,779.9783595)
\lineto(388.72138773,778.61427981)
\curveto(388.72138207,777.66321155)(388.74222216,777.06074362)(388.78390805,776.80687421)
\curveto(388.82937161,776.55679246)(388.91652106,776.3161842)(389.04535665,776.08504871)
\lineto(387.97682756,776.08504871)
\curveto(387.87072767,776.29723867)(387.80252375,776.54542514)(387.77221561,776.82960887)
\moveto(387.68696062,779.11444236)
\curveto(387.31562357,778.96287507)(386.75862492,778.83404545)(386.015963,778.72795312)
\curveto(385.59536923,778.66732477)(385.29792437,778.59912085)(385.12362753,778.52334116)
\curveto(384.94932657,778.44755659)(384.81481329,778.33577795)(384.72008729,778.1880049)
\curveto(384.62535797,778.04401675)(384.57799414,777.88297973)(384.57799565,777.70489334)
\curveto(384.57799414,777.43207606)(384.68030001,777.20472967)(384.88491358,777.0228535)
\curveto(385.09331262,776.84097545)(385.39644114,776.75003689)(385.79430005,776.75003756)
\curveto(386.18836439,776.75003689)(386.53885674,776.83529179)(386.84577815,777.0058025)
\curveto(387.15269199,777.18010048)(387.37814383,777.41691963)(387.52213433,777.71626068)
\curveto(387.63201396,777.94739454)(387.686956,778.28841412)(387.68696062,778.73932045)
\lineto(387.68696062,779.11444236)
}
}
{
\newrgbcolor{curcolor}{0 0 0}
\pscustom[linestyle=none,fillstyle=solid,fillcolor=curcolor]
{
\newpath
\moveto(389.90359043,777.88677064)
\lineto(390.91528287,778.04591327)
\curveto(390.97211809,777.64047691)(391.12936601,777.32977018)(391.38702709,777.11379214)
\curveto(391.6484736,776.89781204)(392.01222782,776.78982251)(392.47829085,776.78982322)
\curveto(392.94813712,776.78982251)(393.29673492,776.88455017)(393.52408528,777.07400649)
\curveto(393.75142769,777.26724993)(393.86510089,777.49270176)(393.86510521,777.75036267)
\curveto(393.86510089,777.9814965)(393.76468957,778.16337361)(393.56387094,778.29599455)
\curveto(393.42366998,778.38693089)(393.07507219,778.50249864)(392.51807651,778.64269814)
\curveto(391.76783045,778.8321509)(391.24682831,778.99508248)(390.95506852,779.13149336)
\curveto(390.66709502,779.27168725)(390.44732685,779.46303713)(390.29576334,779.70554356)
\curveto(390.14798743,779.95183186)(390.07409986,780.22275298)(390.07410039,780.51830772)
\curveto(390.07409986,780.78732984)(390.13472556,781.03551632)(390.25597768,781.26286788)
\curveto(390.38101748,781.4939982)(390.54963272,781.68534808)(390.7618239,781.83691809)
\curveto(390.92096516,781.95437464)(391.13694423,782.05289141)(391.40976176,782.13246869)
\curveto(391.68636466,782.21582298)(391.98191497,782.25750316)(392.29641356,782.25750933)
\curveto(392.77004912,782.25750316)(393.18495628,782.18929924)(393.54113628,782.05289737)
\curveto(393.9010974,781.91648357)(394.16633485,781.73081736)(394.33684944,781.49589816)
\curveto(394.50735444,781.26475726)(394.62481674,780.95405053)(394.68923669,780.56377704)
\lineto(393.68891158,780.42736907)
\curveto(393.64343816,780.73807146)(393.51081943,780.98057427)(393.291055,781.15487824)
\curveto(393.07507219,781.32917207)(392.76815456,781.41632152)(392.37030121,781.41632685)
\curveto(391.90044918,781.41632152)(391.56511326,781.33864484)(391.36429243,781.18329657)
\curveto(391.16346797,781.0279381)(391.06305665,780.84606099)(391.06305817,780.63766469)
\curveto(391.06305665,780.50504141)(391.10473682,780.38568456)(391.18809881,780.27959377)
\curveto(391.2714575,780.16970549)(391.40218168,780.07876693)(391.58027172,780.00677783)
\curveto(391.68257556,779.96888284)(391.98380952,779.8817334)(392.48397452,779.74532922)
\curveto(393.20769091,779.55208113)(393.71164208,779.39293866)(393.99582951,779.26790133)
\curveto(394.28379715,779.14664674)(394.50924899,778.96855873)(394.67218569,778.73363678)
\curveto(394.83511215,778.49870953)(394.91657794,778.20694833)(394.91658331,777.85835231)
\curveto(394.91657794,777.51733095)(394.81616661,777.1952569)(394.61534904,776.89212919)
\curveto(394.41831043,776.59278897)(394.13223289,776.35975893)(393.75711557,776.19303835)
\curveto(393.38198981,776.03010666)(392.95760989,775.94864087)(392.48397452,775.94864074)
\curveto(391.69962654,775.94864087)(391.10094771,776.11157245)(390.68793625,776.43743596)
\curveto(390.27871161,776.76329877)(390.01726326,777.24640984)(389.90359043,777.88677064)
}
}
{
\newrgbcolor{curcolor}{0 0 0}
\pscustom[linestyle=none,fillstyle=solid,fillcolor=curcolor]
{
\newpath
\moveto(400.26491156,778.02886227)
\lineto(401.32207332,777.89813797)
\curveto(401.15534668,777.2805118)(400.8465345,776.80118983)(400.39563586,776.46017062)
\curveto(399.94472716,776.11915066)(399.36878298,775.94864087)(398.66780158,775.94864074)
\curveto(397.78493647,775.94864087)(397.08395177,776.21956199)(396.56484538,776.76140489)
\curveto(396.0495257,777.30703554)(395.79186646,778.0705405)(395.79186689,779.05192204)
\curveto(395.79186646,780.06739961)(396.05331481,780.85553376)(396.57621271,781.41632685)
\curveto(397.0991082,781.97710928)(397.77735826,782.25750316)(398.61096493,782.25750933)
\curveto(399.41804136,782.25750316)(400.07734589,781.98279294)(400.58888049,781.43337785)
\curveto(401.10040464,780.88395206)(401.35616932,780.11097434)(401.35617531,779.11444236)
\curveto(401.35616932,779.05381363)(401.35427477,778.96287507)(401.35049165,778.84162642)
\lineto(396.84902865,778.84162642)
\curveto(396.88691823,778.17853003)(397.07447901,777.67078977)(397.41171153,777.3184041)
\curveto(397.74893996,776.96601596)(398.16953078,776.78982251)(398.67348525,776.78982322)
\curveto(399.04860348,776.78982251)(399.36878298,776.88833928)(399.6340247,777.08537382)
\curveto(399.89925788,777.28240635)(400.10955329,777.59690219)(400.26491156,778.02886227)
\moveto(396.90586531,779.6828089)
\lineto(400.27627889,779.6828089)
\curveto(400.2308047,780.19054557)(400.10197508,780.57135077)(399.88978964,780.82522565)
\curveto(399.56392196,781.21928798)(399.14143659,781.41632152)(398.62233226,781.41632685)
\curveto(398.1524798,781.41632152)(397.75651817,781.2590736)(397.43444619,780.94458262)
\curveto(397.11615918,780.63008192)(396.93996573,780.20949111)(396.90586531,779.6828089)
}
}
{
\newrgbcolor{curcolor}{0 0 0}
\pscustom[linestyle=none,fillstyle=solid,fillcolor=curcolor]
{
\newpath
\moveto(408.00038224,779.72827823)
\lineto(402.48154314,777.36955709)
\lineto(402.48154314,778.38693319)
\lineto(406.85228183,780.20002245)
\lineto(402.48154314,781.99606072)
\lineto(402.48154314,783.01343683)
\lineto(408.00038224,780.68313401)
\lineto(408.00038224,779.72827823)
}
}
{
\newrgbcolor{curcolor}{0 0 0}
\pscustom[linestyle=none,fillstyle=solid,fillcolor=curcolor]
{
\newpath
\moveto(414.7980449,779.72827823)
\lineto(409.2792058,777.36955709)
\lineto(409.2792058,778.38693319)
\lineto(413.64994449,780.20002245)
\lineto(409.2792058,781.99606072)
\lineto(409.2792058,783.01343683)
\lineto(414.7980449,780.68313401)
\lineto(414.7980449,779.72827823)
}
}
{
\newrgbcolor{curcolor}{0 0 0}
\pscustom[linestyle=none,fillstyle=solid,fillcolor=curcolor]
{
\newpath
\moveto(365.43085945,764.6608857)
\lineto(365.43085945,763.38774465)
\lineto(364.3395957,763.38774465)
\lineto(364.3395957,760.95513586)
\curveto(364.33959319,760.46254969)(364.34906595,760.17457759)(364.36801403,760.09121872)
\curveto(364.39074612,760.01164601)(364.43810995,759.94533665)(364.51010566,759.89229043)
\curveto(364.58588511,759.83924167)(364.67682366,759.81271792)(364.7829216,759.81271911)
\curveto(364.9306938,759.81271792)(365.14477831,759.86387086)(365.42517579,759.96617808)
\lineto(365.56158376,758.72713902)
\curveto(365.19024759,758.56799645)(364.76965677,758.48842521)(364.29981004,758.48842508)
\curveto(364.01183548,758.48842521)(363.75228168,758.53578904)(363.52114788,758.63051671)
\curveto(363.29001069,758.72903347)(363.1195009,758.85407399)(363.009618,759.00563863)
\curveto(362.90352183,759.16099161)(362.82963426,759.36939247)(362.78795504,759.63084182)
\curveto(362.75385213,759.81650703)(362.73680115,760.19162857)(362.73680206,760.75620757)
\lineto(362.73680206,763.38774465)
\lineto(362.00360922,763.38774465)
\lineto(362.00360922,764.6608857)
\lineto(362.73680206,764.6608857)
\lineto(362.73680206,765.8601391)
\lineto(364.3395957,766.79226022)
\lineto(364.3395957,764.6608857)
\lineto(365.43085945,764.6608857)
}
}
{
\newrgbcolor{curcolor}{0 0 0}
\pscustom[linestyle=none,fillstyle=solid,fillcolor=curcolor]
{
\newpath
\moveto(370.04599594,760.54591195)
\lineto(371.63742225,760.27877968)
\curveto(371.43280457,759.69525562)(371.10883597,759.25003561)(370.66551546,758.94311831)
\curveto(370.22597416,758.63998947)(369.67465917,758.48842521)(369.01156883,758.48842508)
\curveto(367.96198304,758.48842521)(367.18521621,758.83133935)(366.68126602,759.51716851)
\curveto(366.28340887,760.06658806)(366.08448078,760.75999454)(366.08448115,761.59739005)
\curveto(366.08448078,762.59771119)(366.34592913,763.38016167)(366.86882698,763.94474386)
\curveto(367.39172252,764.51310451)(368.0529216,764.79728749)(368.8524262,764.79729367)
\curveto(369.7504413,764.79728749)(370.45900421,764.49984264)(370.97811706,763.9049582)
\curveto(371.49721938,763.31385231)(371.74540586,762.40636131)(371.72267723,761.18248247)
\lineto(367.72137679,761.18248247)
\curveto(367.7327421,760.70884161)(367.86157172,760.33940373)(368.10786603,760.07416772)
\curveto(368.35415556,759.81271792)(368.66107319,759.68199375)(369.02861983,759.68199481)
\curveto(369.27869754,759.68199375)(369.48899295,759.75019767)(369.65950669,759.88660676)
\curveto(369.83001254,760.02301333)(369.95884216,760.24278151)(370.04599594,760.54591195)
\moveto(370.13693458,762.16007292)
\curveto(370.12556284,762.62234038)(370.00620599,762.97283273)(369.77886366,763.21155102)
\curveto(369.55151321,763.45404925)(369.27490844,763.57530066)(368.94904851,763.57530561)
\curveto(368.60044748,763.57530066)(368.31247539,763.44836559)(368.08513137,763.19450002)
\curveto(367.85778261,762.94062532)(367.74600397,762.59581663)(367.74979511,762.16007292)
\lineto(370.13693458,762.16007292)
}
}
{
\newrgbcolor{curcolor}{0 0 0}
\pscustom[linestyle=none,fillstyle=solid,fillcolor=curcolor]
{
\newpath
\moveto(372.91056359,764.6608857)
\lineto(374.38263293,764.6608857)
\lineto(374.38263293,763.83675421)
\curveto(374.90931654,764.477108)(375.53641366,764.79728749)(376.26392618,764.79729367)
\curveto(376.65041097,764.79728749)(376.98574689,764.71771626)(377.26993495,764.55857972)
\curveto(377.55411286,764.39943131)(377.78714291,764.15882305)(377.9690258,763.83675421)
\curveto(378.23425747,764.15882305)(378.52033501,764.39943131)(378.82725927,764.55857972)
\curveto(379.13417026,764.71771626)(379.46192797,764.79728749)(379.81053338,764.79729367)
\curveto(380.25385123,764.79728749)(380.62897277,764.70634894)(380.93589913,764.52447773)
\curveto(381.24280802,764.34638382)(381.47204896,764.08304092)(381.62362265,763.73444824)
\curveto(381.73349731,763.47678389)(381.78843935,763.05998218)(381.78844894,762.48404185)
\lineto(381.78844894,758.62483304)
\lineto(380.19133897,758.62483304)
\lineto(380.19133897,762.07481794)
\curveto(380.19133097,762.67349331)(380.13638893,763.05998218)(380.02651267,763.23428568)
\curveto(379.87872969,763.46162746)(379.6513833,763.57530066)(379.34447282,763.57530561)
\curveto(379.12090839,763.57530066)(378.91061298,763.50709674)(378.71358596,763.37069365)
\curveto(378.51654591,763.23428107)(378.37445441,763.03345843)(378.28731106,762.76822512)
\curveto(378.20015552,762.50677263)(378.15658079,762.09186547)(378.15658675,761.5235024)
\lineto(378.15658675,758.62483304)
\lineto(376.55947678,758.62483304)
\lineto(376.55947678,761.93272631)
\curveto(376.55947241,762.5200345)(376.53105411,762.89894515)(376.4742218,763.06945939)
\curveto(376.41738092,763.23996473)(376.32833692,763.3668998)(376.20708952,763.45026497)
\curveto(376.08962321,763.53362048)(375.92858618,763.57530066)(375.72397796,763.57530561)
\curveto(375.47768251,763.57530066)(375.25601978,763.50899129)(375.05898911,763.37637732)
\curveto(374.86195271,763.24375384)(374.71986122,763.05240396)(374.63271421,762.80232711)
\curveto(374.54935143,762.55224191)(374.50767125,762.13733475)(374.50767357,761.55760439)
\lineto(374.50767357,758.62483304)
\lineto(372.91056359,758.62483304)
\lineto(372.91056359,764.6608857)
}
}
{
\newrgbcolor{curcolor}{0 0 0}
\pscustom[linestyle=none,fillstyle=solid,fillcolor=curcolor]
{
\newpath
\moveto(383.32872221,764.6608857)
\lineto(384.81784254,764.6608857)
\lineto(384.81784254,763.7742339)
\curveto(385.01108469,764.07735726)(385.27253304,764.32364919)(385.60218837,764.5131104)
\curveto(385.93183757,764.70255983)(386.29748634,764.79728749)(386.69913579,764.79729367)
\curveto(387.40011633,764.79728749)(387.99500604,764.52257728)(388.48380672,763.97316218)
\curveto(388.97259551,763.4237364)(389.21699288,762.65833689)(389.21699956,761.67696136)
\curveto(389.21699288,760.66905599)(388.97070096,759.88471095)(388.47812306,759.32392389)
\curveto(387.98553328,758.76692454)(387.38874901,758.48842521)(386.68776846,758.48842508)
\curveto(386.35432294,758.48842521)(386.05119442,758.55473457)(385.77838199,758.68735336)
\curveto(385.50935219,758.81997203)(385.22516921,759.04731842)(384.92583218,759.36939321)
\lineto(384.92583218,756.32863222)
\lineto(383.32872221,756.32863222)
\lineto(383.32872221,764.6608857)
\moveto(384.90878119,761.74516535)
\curveto(384.90877882,761.06691217)(385.0432921,760.56485556)(385.31232143,760.23899402)
\curveto(385.58134522,759.91691835)(385.90910293,759.75588133)(386.29559555,759.75588246)
\curveto(386.66692422,759.75588133)(386.9757364,759.90365648)(387.22203301,760.19920836)
\curveto(387.46832024,760.4985462)(387.5914662,760.98734093)(387.59147126,761.66559403)
\curveto(387.5914662,762.29837177)(387.46453114,762.76822098)(387.21066567,763.07514305)
\curveto(386.95679087,763.38205623)(386.64229503,763.53551504)(386.26717722,763.53551995)
\curveto(385.87689552,763.53551504)(385.55292692,763.38395078)(385.29527043,763.08082672)
\curveto(385.03760844,762.78148285)(384.90877882,762.33626284)(384.90878119,761.74516535)
}
}
{
\newrgbcolor{curcolor}{0 0 0}
\pscustom[linestyle=none,fillstyle=solid,fillcolor=curcolor]
{
\newpath
\moveto(390.49013978,758.62483304)
\lineto(390.49013978,766.95708652)
\lineto(392.08724976,766.95708652)
\lineto(392.08724976,758.62483304)
\lineto(390.49013978,758.62483304)
}
}
{
\newrgbcolor{curcolor}{0 0 0}
\pscustom[linestyle=none,fillstyle=solid,fillcolor=curcolor]
{
\newpath
\moveto(394.91203223,762.81937811)
\lineto(393.46269755,763.08082672)
\curveto(393.62562855,763.66434466)(393.90602243,764.0963028)(394.30388003,764.37670243)
\curveto(394.70173479,764.65709056)(395.2928354,764.79728749)(396.07718364,764.79729367)
\curveto(396.78953246,764.79728749)(397.32000737,764.7120326)(397.66860995,764.54152872)
\curveto(398.01720296,764.37480212)(398.26160033,764.16071761)(398.40180278,763.89927453)
\curveto(398.54578331,763.64161002)(398.61777633,763.16607716)(398.61778207,762.47267452)
\lineto(398.60073107,760.60843227)
\curveto(398.60072536,760.07795538)(398.62535455,759.68578286)(398.67461872,759.43191353)
\curveto(398.72766042,759.1818317)(398.82428264,758.91280514)(398.96448566,758.62483304)
\lineto(397.38442668,758.62483304)
\curveto(397.34274201,758.73092803)(397.29158907,758.88817594)(397.23096771,759.09657727)
\curveto(397.20443962,759.19130446)(397.18549409,759.25382472)(397.17413106,759.28413823)
\curveto(396.9013111,759.01890012)(396.6095499,758.81997203)(396.29884659,758.68735336)
\curveto(395.98813644,758.55473457)(395.65658962,758.48842521)(395.30420514,758.48842508)
\curveto(394.68278926,758.48842521)(394.19209997,758.65704045)(393.8321358,758.9942713)
\curveto(393.47595885,759.3315014)(393.29787084,759.75777588)(393.29787126,760.27309601)
\curveto(393.29787084,760.61411394)(393.37933663,760.91724246)(393.54226887,761.18248247)
\curveto(393.70519979,761.45150648)(393.93254618,761.65611823)(394.22430872,761.79631834)
\curveto(394.51985768,761.94030121)(394.94423761,762.06534173)(395.49744977,762.17144025)
\curveto(396.24390113,762.31163365)(396.76111416,762.44235782)(397.04909042,762.56361317)
\lineto(397.04909042,762.7227558)
\curveto(397.04908625,763.02966932)(396.97330412,763.24754295)(396.8217438,763.37637732)
\curveto(396.67017561,763.50899129)(396.38409807,763.57530066)(395.96351033,763.57530561)
\curveto(395.67932426,763.57530066)(395.45766153,763.51846406)(395.29852148,763.40479564)
\curveto(395.13937659,763.29490678)(395.01054697,763.09976779)(394.91203223,762.81937811)
\moveto(397.04909042,761.5235024)
\curveto(396.8444745,761.45529558)(396.5205059,761.37382979)(396.07718364,761.27910479)
\curveto(395.63385498,761.18437447)(395.34398834,761.09154136)(395.20758283,761.00060518)
\curveto(394.99917965,760.85282765)(394.89497922,760.66526688)(394.89498123,760.43792231)
\curveto(394.89497922,760.21436321)(394.97833956,760.02111878)(395.14506251,759.85818844)
\curveto(395.31178093,759.69525562)(395.5239709,759.61378983)(395.78163304,759.61379082)
\curveto(396.06960223,759.61378983)(396.34431245,759.7085175)(396.60576452,759.89797409)
\curveto(396.79900523,760.04195887)(396.92594029,760.21815232)(396.9865701,760.42655498)
\curveto(397.02824617,760.56296101)(397.04908625,760.8225148)(397.04909042,761.20521714)
\lineto(397.04909042,761.5235024)
}
}
{
\newrgbcolor{curcolor}{0 0 0}
\pscustom[linestyle=none,fillstyle=solid,fillcolor=curcolor]
{
\newpath
\moveto(402.96578462,764.6608857)
\lineto(402.96578462,763.38774465)
\lineto(401.87452086,763.38774465)
\lineto(401.87452086,760.95513586)
\curveto(401.87451835,760.46254969)(401.88399111,760.17457759)(401.90293919,760.09121872)
\curveto(401.92567129,760.01164601)(401.97303512,759.94533665)(402.04503082,759.89229043)
\curveto(402.12081027,759.83924167)(402.21174883,759.81271792)(402.31784676,759.81271911)
\curveto(402.46561896,759.81271792)(402.67970348,759.86387086)(402.96010095,759.96617808)
\lineto(403.09650892,758.72713902)
\curveto(402.72517275,758.56799645)(402.30458193,758.48842521)(401.8347352,758.48842508)
\curveto(401.54676064,758.48842521)(401.28720684,758.53578904)(401.05607304,758.63051671)
\curveto(400.82493585,758.72903347)(400.65442606,758.85407399)(400.54454316,759.00563863)
\curveto(400.43844699,759.16099161)(400.36455942,759.36939247)(400.32288021,759.63084182)
\curveto(400.28877729,759.81650703)(400.27172631,760.19162857)(400.27172722,760.75620757)
\lineto(400.27172722,763.38774465)
\lineto(399.53853438,763.38774465)
\lineto(399.53853438,764.6608857)
\lineto(400.27172722,764.6608857)
\lineto(400.27172722,765.8601391)
\lineto(401.87452086,766.79226022)
\lineto(401.87452086,764.6608857)
\lineto(402.96578462,764.6608857)
}
}
{
\newrgbcolor{curcolor}{0 0 0}
\pscustom[linestyle=none,fillstyle=solid,fillcolor=curcolor]
{
\newpath
\moveto(407.58092287,760.54591195)
\lineto(409.17234919,760.27877968)
\curveto(408.96773151,759.69525562)(408.64376291,759.25003561)(408.2004424,758.94311831)
\curveto(407.7609011,758.63998947)(407.20958611,758.48842521)(406.54649577,758.48842508)
\curveto(405.49690998,758.48842521)(404.72014315,758.83133935)(404.21619296,759.51716851)
\curveto(403.81833581,760.06658806)(403.61940772,760.75999454)(403.61940809,761.59739005)
\curveto(403.61940772,762.59771119)(403.88085607,763.38016167)(404.40375392,763.94474386)
\curveto(404.92664946,764.51310451)(405.58784854,764.79728749)(406.38735314,764.79729367)
\curveto(407.28536824,764.79728749)(407.99393115,764.49984264)(408.513044,763.9049582)
\curveto(409.03214632,763.31385231)(409.2803328,762.40636131)(409.25760417,761.18248247)
\lineto(405.25630373,761.18248247)
\curveto(405.26766904,760.70884161)(405.39649866,760.33940373)(405.64279297,760.07416772)
\curveto(405.8890825,759.81271792)(406.19600013,759.68199375)(406.56354677,759.68199481)
\curveto(406.81362448,759.68199375)(407.02391989,759.75019767)(407.19443363,759.88660676)
\curveto(407.36493947,760.02301333)(407.4937691,760.24278151)(407.58092287,760.54591195)
\moveto(407.67186152,762.16007292)
\curveto(407.66048978,762.62234038)(407.54113293,762.97283273)(407.3137906,763.21155102)
\curveto(407.08644015,763.45404925)(406.80983538,763.57530066)(406.48397545,763.57530561)
\curveto(406.13537442,763.57530066)(405.84740233,763.44836559)(405.62005831,763.19450002)
\curveto(405.39270955,762.94062532)(405.28093091,762.59581663)(405.28472205,762.16007292)
\lineto(407.67186152,762.16007292)
}
}
{
\newrgbcolor{curcolor}{0 0 0}
\pscustom[linestyle=none,fillstyle=solid,fillcolor=curcolor]
{
\newpath
\moveto(410.00216285,760.34698366)
\lineto(411.60495649,760.59138127)
\curveto(411.67315854,760.28067257)(411.81146092,760.04385342)(412.01986407,759.8809231)
\curveto(412.22826263,759.72177937)(412.52002383,759.64220813)(412.89514854,759.64220915)
\curveto(413.30815798,759.64220813)(413.61886471,759.71799026)(413.82726967,759.86955577)
\curveto(413.96746251,759.9756495)(414.03756098,760.117741)(414.03756529,760.29583067)
\curveto(414.03756098,760.41708041)(413.99966991,760.51749173)(413.92389198,760.59706494)
\curveto(413.84431655,760.6728451)(413.66622854,760.74294356)(413.38962743,760.80736056)
\curveto(412.10132757,761.09154136)(411.28477512,761.35109515)(410.93996764,761.58602272)
\curveto(410.46253902,761.91188291)(410.22382531,762.36468114)(410.2238258,762.94441875)
\curveto(410.22382531,763.46731112)(410.43033161,763.90684747)(410.84334533,764.26302912)
\curveto(411.25635682,764.61919949)(411.89671582,764.79728749)(412.76442424,764.79729367)
\curveto(413.59044641,764.79728749)(414.20428166,764.66277421)(414.60593183,764.39375342)
\curveto(415.00757224,764.1247211)(415.28417701,763.72686492)(415.43574697,763.20018369)
\lineto(413.92957564,762.92168409)
\curveto(413.86515663,763.15660439)(413.74201067,763.33658695)(413.56013739,763.4616323)
\curveto(413.38204556,763.58666798)(413.12628087,763.64918823)(412.79284256,763.64919326)
\curveto(412.37224868,763.64918823)(412.07101472,763.59045708)(411.88913976,763.47299963)
\curveto(411.7678862,763.38963444)(411.70726049,763.2816449)(411.70726247,763.1490307)
\curveto(411.70726049,763.03535298)(411.76030798,762.93873077)(411.8664051,762.85916377)
\curveto(412.01038901,762.75306455)(412.50676196,762.60339484)(413.35552544,762.4101542)
\curveto(414.20807077,762.21690598)(414.80296049,761.98008683)(415.14019637,761.69969603)
\curveto(415.47363233,761.41550996)(415.64035302,761.01954834)(415.64035893,760.51180996)
\curveto(415.64035302,759.95859852)(415.40921752,759.48306566)(414.94695175,759.08520994)
\curveto(414.48467554,758.6873533)(413.80074182,758.48842521)(412.89514854,758.48842508)
\curveto(412.07290927,758.48842521)(411.42118296,758.6551459)(410.93996764,758.98858763)
\curveto(410.46253902,759.32202864)(410.14993773,759.77482686)(410.00216285,760.34698366)
}
}
{
\newrgbcolor{curcolor}{0.50196081 0 1}
\pscustom[linewidth=2.32802935,linecolor=curcolor]
{
\newpath
\moveto(130.18849429,660.25842368)
\lineto(205.84945068,660.25842368)
\lineto(205.84945068,706.81900578)
\lineto(154.63280013,706.81900578)
\lineto(154.63280013,716.13112406)
\lineto(130.18849429,716.13112406)
\lineto(130.18849429,660.25842368)
\closepath
}
}
{
\newrgbcolor{curcolor}{0 0 0}
\pscustom[linestyle=none,fillstyle=solid,fillcolor=curcolor]
{
\newpath
\moveto(136.64513056,687.77109502)
\lineto(136.64513056,688.72595081)
\lineto(142.16396966,691.05625362)
\lineto(142.16396966,690.03887752)
\lineto(137.7875473,688.24283925)
\lineto(142.16396966,686.42974999)
\lineto(142.16396966,685.41237388)
\lineto(136.64513056,687.77109502)
}
}
{
\newrgbcolor{curcolor}{0 0 0}
\pscustom[linestyle=none,fillstyle=solid,fillcolor=curcolor]
{
\newpath
\moveto(143.44279455,687.77109502)
\lineto(143.44279455,688.72595081)
\lineto(148.96163365,691.05625362)
\lineto(148.96163365,690.03887752)
\lineto(144.5852113,688.24283925)
\lineto(148.96163365,686.42974999)
\lineto(148.96163365,685.41237388)
\lineto(143.44279455,687.77109502)
}
}
{
\newrgbcolor{curcolor}{0 0 0}
\pscustom[linestyle=none,fillstyle=solid,fillcolor=curcolor]
{
\newpath
\moveto(150.37118285,684.1278655)
\lineto(150.37118285,690.16391816)
\lineto(151.28625298,690.16391816)
\lineto(151.28625298,689.31705201)
\curveto(151.47570662,689.61259713)(151.7276822,689.84941628)(152.04218047,690.02751019)
\curveto(152.35667387,690.2093814)(152.71474444,690.30031995)(153.11639324,690.30032613)
\curveto(153.56350429,690.30031995)(153.92915306,690.20748684)(154.21334066,690.02182652)
\curveto(154.50130814,689.83615441)(154.70402534,689.57660062)(154.82149285,689.24316436)
\curveto(155.29891505,689.94793305)(155.92032852,690.30031995)(156.68573511,690.30032613)
\curveto(157.28440685,690.30031995)(157.74478328,690.13359927)(158.0668658,689.80016357)
\curveto(158.38893138,689.47050563)(158.54996841,688.96087081)(158.54997736,688.27125758)
\lineto(158.54997736,684.1278655)
\lineto(157.53260125,684.1278655)
\lineto(157.53260125,687.93023765)
\curveto(157.53259332,688.33945735)(157.49849136,688.6331131)(157.43029527,688.81120579)
\curveto(157.36587264,688.99307822)(157.24651578,689.13895882)(157.07222435,689.24884803)
\curveto(156.89791799,689.35872699)(156.69330624,689.41366904)(156.45838849,689.41367432)
\curveto(156.03400171,689.41366904)(155.68161481,689.27157754)(155.40122673,688.98739942)
\curveto(155.12082705,688.70700068)(154.98063011,688.25609701)(154.98063549,687.63468705)
\lineto(154.98063549,684.1278655)
\lineto(153.95757571,684.1278655)
\lineto(153.95757571,688.04959463)
\curveto(153.95757136,688.50428348)(153.87421102,688.84530307)(153.70749444,689.0726544)
\curveto(153.54076965,689.29999584)(153.26795398,689.41366904)(152.88904662,689.41367432)
\curveto(152.60107124,689.41366904)(152.33393924,689.33788691)(152.0876498,689.18632771)
\curveto(151.8451445,689.03475839)(151.66895105,688.81309566)(151.55906892,688.52133886)
\curveto(151.44918287,688.22957326)(151.39424083,687.80898244)(151.39424262,687.25956514)
\lineto(151.39424262,684.1278655)
\lineto(150.37118285,684.1278655)
}
}
{
\newrgbcolor{curcolor}{0 0 0}
\pscustom[linestyle=none,fillstyle=solid,fillcolor=curcolor]
{
\newpath
\moveto(160.10730238,691.28360024)
\lineto(160.10730238,692.46011898)
\lineto(161.13036215,692.46011898)
\lineto(161.13036215,691.28360024)
\lineto(160.10730238,691.28360024)
\moveto(160.10730238,684.1278655)
\lineto(160.10730238,690.16391816)
\lineto(161.13036215,690.16391816)
\lineto(161.13036215,684.1278655)
\lineto(160.10730238,684.1278655)
}
}
{
\newrgbcolor{curcolor}{0 0 0}
\pscustom[linestyle=none,fillstyle=solid,fillcolor=curcolor]
{
\newpath
\moveto(166.60941394,684.1278655)
\lineto(166.60941394,684.88947667)
\curveto(166.2267095,684.29079708)(165.66402719,683.99145767)(164.92136532,683.99145753)
\curveto(164.4401458,683.99145767)(163.99682034,684.1240764)(163.59138761,684.38931411)
\curveto(163.18974066,684.6545513)(162.87713938,685.02398918)(162.65358282,685.49762886)
\curveto(162.43381392,685.97505491)(162.32392983,686.5225808)(162.32393023,687.14020816)
\curveto(162.32392983,687.74267308)(162.42434115,688.28830441)(162.6251645,688.7771038)
\curveto(162.82598644,689.26968299)(163.1272204,689.64669909)(163.52886729,689.90815321)
\curveto(163.93051098,690.16959578)(164.3795201,690.30031995)(164.87589599,690.30032613)
\curveto(165.23964727,690.30031995)(165.56361587,690.22264327)(165.84780278,690.06729584)
\curveto(166.13198184,689.91572565)(166.36311734,689.71679756)(166.54120996,689.47051098)
\lineto(166.54120996,692.46011898)
\lineto(167.55858606,692.46011898)
\lineto(167.55858606,684.1278655)
\lineto(166.60941394,684.1278655)
\moveto(163.37540833,687.14020816)
\curveto(163.37540688,686.36722743)(163.53833846,685.78938869)(163.86420355,685.40669022)
\curveto(164.19006477,685.02398918)(164.57465908,684.83263931)(165.01798763,684.83264001)
\curveto(165.4650991,684.83263931)(165.84400975,685.01451642)(166.15472071,685.37827189)
\curveto(166.46921232,685.74581397)(166.62646024,686.30470717)(166.62646494,687.05495318)
\curveto(166.62646024,687.88097547)(166.46731776,688.4872325)(166.14903704,688.87372611)
\curveto(165.83074788,689.26021023)(165.43857536,689.45345466)(164.97251831,689.45345998)
\curveto(164.51782248,689.45345466)(164.13701728,689.26778844)(163.83010156,688.89646077)
\curveto(163.52697114,688.52512357)(163.37540688,687.93970662)(163.37540833,687.14020816)
}
}
{
\newrgbcolor{curcolor}{0 0 0}
\pscustom[linestyle=none,fillstyle=solid,fillcolor=curcolor]
{
\newpath
\moveto(173.08879279,684.1278655)
\lineto(173.08879279,684.88947667)
\curveto(172.70608835,684.29079708)(172.14340604,683.99145767)(171.40074417,683.99145753)
\curveto(170.91952465,683.99145767)(170.47619919,684.1240764)(170.07076646,684.38931411)
\curveto(169.66911951,684.6545513)(169.35651823,685.02398918)(169.13296167,685.49762886)
\curveto(168.91319277,685.97505491)(168.80330868,686.5225808)(168.80330908,687.14020816)
\curveto(168.80330868,687.74267308)(168.90372,688.28830441)(169.10454334,688.7771038)
\curveto(169.30536529,689.26968299)(169.60659925,689.64669909)(170.00824614,689.90815321)
\curveto(170.40988983,690.16959578)(170.85889894,690.30031995)(171.35527484,690.30032613)
\curveto(171.71902611,690.30031995)(172.04299472,690.22264327)(172.32718163,690.06729584)
\curveto(172.61136069,689.91572565)(172.84249618,689.71679756)(173.0205888,689.47051098)
\lineto(173.0205888,692.46011898)
\lineto(174.03796491,692.46011898)
\lineto(174.03796491,684.1278655)
\lineto(173.08879279,684.1278655)
\moveto(169.85478718,687.14020816)
\curveto(169.85478573,686.36722743)(170.01771731,685.78938869)(170.3435824,685.40669022)
\curveto(170.66944362,685.02398918)(171.05403793,684.83263931)(171.49736648,684.83264001)
\curveto(171.94447795,684.83263931)(172.3233886,685.01451642)(172.63409956,685.37827189)
\curveto(172.94859117,685.74581397)(173.10583908,686.30470717)(173.10584379,687.05495318)
\curveto(173.10583908,687.88097547)(172.94669661,688.4872325)(172.62841589,688.87372611)
\curveto(172.31012672,689.26021023)(171.9179542,689.45345466)(171.45189715,689.45345998)
\curveto(170.99720133,689.45345466)(170.61639613,689.26778844)(170.30948041,688.89646077)
\curveto(170.00634999,688.52512357)(169.85478573,687.93970662)(169.85478718,687.14020816)
}
}
{
\newrgbcolor{curcolor}{0 0 0}
\pscustom[linestyle=none,fillstyle=solid,fillcolor=curcolor]
{
\newpath
\moveto(175.62938974,684.1278655)
\lineto(175.62938974,692.46011898)
\lineto(176.65244951,692.46011898)
\lineto(176.65244951,684.1278655)
\lineto(175.62938974,684.1278655)
}
}
{
\newrgbcolor{curcolor}{0 0 0}
\pscustom[linestyle=none,fillstyle=solid,fillcolor=curcolor]
{
\newpath
\moveto(182.37590069,686.07167907)
\lineto(183.43306245,685.94095477)
\curveto(183.26633581,685.3233286)(182.95752363,684.84400663)(182.50662499,684.50298742)
\curveto(182.05571629,684.16196746)(181.4797721,683.99145767)(180.77879071,683.99145753)
\curveto(179.8959256,683.99145767)(179.1949409,684.26237878)(178.67583451,684.80422169)
\curveto(178.16051483,685.34985234)(177.90285559,686.1133573)(177.90285602,687.09473884)
\curveto(177.90285559,688.11021641)(178.16430394,688.89835056)(178.68720184,689.45914365)
\curveto(179.21009733,690.01992607)(179.88834738,690.30031995)(180.72195406,690.30032613)
\curveto(181.52903049,690.30031995)(182.18833502,690.02560973)(182.69986961,689.47619464)
\curveto(183.21139376,688.92676886)(183.46715845,688.15379113)(183.46716444,687.15725916)
\curveto(183.46715845,687.09663043)(183.4652639,687.00569187)(183.46148078,686.88444322)
\lineto(178.96001778,686.88444322)
\curveto(178.99790736,686.22134683)(179.18546813,685.71360656)(179.52270066,685.36122089)
\curveto(179.85992909,685.00883276)(180.28051991,684.83263931)(180.78447437,684.83264001)
\curveto(181.15959261,684.83263931)(181.4797721,684.93115608)(181.74501383,685.12819061)
\curveto(182.01024701,685.32522315)(182.22054242,685.63971899)(182.37590069,686.07167907)
\moveto(179.01685444,687.7256257)
\lineto(182.38726802,687.7256257)
\curveto(182.34179383,688.23336237)(182.21296421,688.61416757)(182.00077877,688.86804245)
\curveto(181.67491109,689.26210478)(181.25242572,689.45913832)(180.73332139,689.45914365)
\curveto(180.26346893,689.45913832)(179.8675073,689.3018904)(179.54543532,688.98739942)
\curveto(179.2271483,688.67289872)(179.05095485,688.2523079)(179.01685444,687.7256257)
}
}
{
\newrgbcolor{curcolor}{0 0 0}
\pscustom[linestyle=none,fillstyle=solid,fillcolor=curcolor]
{
\newpath
\moveto(190.11137137,687.77109502)
\lineto(184.59253226,685.41237388)
\lineto(184.59253226,686.42974999)
\lineto(188.96327095,688.24283925)
\lineto(184.59253226,690.03887752)
\lineto(184.59253226,691.05625362)
\lineto(190.11137137,688.72595081)
\lineto(190.11137137,687.77109502)
}
}
{
\newrgbcolor{curcolor}{0 0 0}
\pscustom[linestyle=none,fillstyle=solid,fillcolor=curcolor]
{
\newpath
\moveto(196.90903403,687.77109502)
\lineto(191.39019492,685.41237388)
\lineto(191.39019492,686.42974999)
\lineto(195.76093362,688.24283925)
\lineto(191.39019492,690.03887752)
\lineto(191.39019492,691.05625362)
\lineto(196.90903403,688.72595081)
\lineto(196.90903403,687.77109502)
}
}
{
\newrgbcolor{curcolor}{0 0 0}
\pscustom[linestyle=none,fillstyle=solid,fillcolor=curcolor]
{
\newpath
\moveto(156.99719962,666.66766406)
\lineto(155.40008964,666.66766406)
\lineto(155.40008964,672.70371671)
\lineto(156.88352631,672.70371671)
\lineto(156.88352631,671.84548324)
\curveto(157.13739419,672.25091245)(157.36474058,672.51804446)(157.56556616,672.64688006)
\curveto(157.77017497,672.7757037)(158.00131047,672.84011851)(158.25897333,672.84012468)
\curveto(158.62272393,672.84011851)(158.97321628,672.73970719)(159.31045143,672.53889041)
\lineto(158.81597254,671.14639239)
\curveto(158.5469418,671.32068681)(158.29686077,671.40783626)(158.06572871,671.407841)
\curveto(157.842168,671.40783626)(157.65271267,671.345316)(157.49736217,671.22028004)
\curveto(157.34200594,671.09902408)(157.21885998,670.87736135)(157.12792392,670.55529119)
\curveto(157.04077198,670.23321325)(156.99719725,669.5587523)(156.99719962,668.53190631)
\lineto(156.99719962,666.66766406)
}
}
{
\newrgbcolor{curcolor}{0 0 0}
\pscustom[linestyle=none,fillstyle=solid,fillcolor=curcolor]
{
\newpath
\moveto(159.62305297,669.77094537)
\curveto(159.6230525,670.30141717)(159.75377668,670.8148411)(160.01522588,671.31121869)
\curveto(160.27667337,671.80758699)(160.64611125,672.18649764)(161.12354064,672.44795177)
\curveto(161.60475519,672.70939433)(162.14091376,672.84011851)(162.73201794,672.84012468)
\curveto(163.64518903,672.84011851)(164.39353756,672.54267365)(164.97706578,671.94778921)
\curveto(165.56058235,671.35668332)(165.85234355,670.60833479)(165.85235025,669.70274138)
\curveto(165.85234355,668.78956368)(165.55679325,668.03174239)(164.96569844,667.42927522)
\curveto(164.37838113,666.83059564)(163.63761082,666.53125622)(162.74338527,666.53125609)
\curveto(162.19017214,666.53125622)(161.66159179,666.65629674)(161.15764263,666.906378)
\curveto(160.65747857,667.15645879)(160.27667337,667.52210757)(160.01522588,668.00332543)
\curveto(159.75377668,668.48832972)(159.6230525,669.07753578)(159.62305297,669.77094537)
\moveto(161.2599486,669.68569038)
\curveto(161.2599465,669.08700854)(161.40203799,668.62852666)(161.68622351,668.31024336)
\curveto(161.97040397,667.99195677)(162.32089632,667.8328143)(162.73770161,667.83281546)
\curveto(163.15449974,667.8328143)(163.50309754,667.99195677)(163.78349604,668.31024336)
\curveto(164.0676744,668.62852666)(164.20976589,669.09079765)(164.20977095,669.69705772)
\curveto(164.20976589,670.2881553)(164.0676744,670.74284807)(163.78349604,671.06113741)
\curveto(163.50309754,671.37941796)(163.15449974,671.53856043)(162.73770161,671.5385653)
\curveto(162.32089632,671.53856043)(161.97040397,671.37941796)(161.68622351,671.06113741)
\curveto(161.40203799,670.74284807)(161.2599465,670.28436619)(161.2599486,669.68569038)
}
}
{
\newrgbcolor{curcolor}{0 0 0}
\pscustom[linestyle=none,fillstyle=solid,fillcolor=curcolor]
{
\newpath
\moveto(167.10844036,666.66766406)
\lineto(167.10844036,674.99991753)
\lineto(168.70555034,674.99991753)
\lineto(168.70555034,666.66766406)
\lineto(167.10844036,666.66766406)
}
}
{
\newrgbcolor{curcolor}{0 0 0}
\pscustom[linestyle=none,fillstyle=solid,fillcolor=curcolor]
{
\newpath
\moveto(173.83221641,668.58874296)
\lineto(175.42364272,668.32161069)
\curveto(175.21902505,667.73808664)(174.89505644,667.29286663)(174.45173594,666.98594932)
\curveto(174.01219463,666.68282048)(173.46087964,666.53125622)(172.7977893,666.53125609)
\curveto(171.74820351,666.53125622)(170.97143669,666.87417036)(170.46748649,667.55999953)
\curveto(170.06962935,668.10941907)(169.87070126,668.80282556)(169.87070162,669.64022106)
\curveto(169.87070126,670.6405422)(170.1321496,671.42299269)(170.65504745,671.98757487)
\curveto(171.17794299,672.55593552)(171.83914207,672.84011851)(172.63864667,672.84012468)
\curveto(173.53666177,672.84011851)(174.24522468,672.54267365)(174.76433753,671.94778921)
\curveto(175.28343986,671.35668332)(175.53162633,670.44919232)(175.5088977,669.22531349)
\lineto(171.50759726,669.22531349)
\curveto(171.51896257,668.75167262)(171.64779219,668.38223474)(171.89408651,668.11699873)
\curveto(172.14037603,667.85554894)(172.44729366,667.72482476)(172.8148403,667.72482582)
\curveto(173.06491801,667.72482476)(173.27521342,667.79302868)(173.44572716,667.92943778)
\curveto(173.61623301,668.06584435)(173.74506263,668.28561252)(173.83221641,668.58874296)
\moveto(173.92315505,670.20290394)
\curveto(173.91178331,670.66517139)(173.79242646,671.01566374)(173.56508413,671.25438203)
\curveto(173.33773368,671.49688026)(173.06112891,671.61813167)(172.73526899,671.61813662)
\curveto(172.38666796,671.61813167)(172.09869586,671.4911966)(171.87135184,671.23733104)
\curveto(171.64400309,670.98345633)(171.53222444,670.63864765)(171.53601559,670.20290394)
\lineto(173.92315505,670.20290394)
}
}
{
\newrgbcolor{curcolor}{0 0 0}
\pscustom[linestyle=none,fillstyle=solid,fillcolor=curcolor]
{
\newpath
\moveto(176.25345727,668.38981467)
\lineto(177.85625092,668.63421229)
\curveto(177.92445296,668.32350359)(178.06275534,668.08668443)(178.27115849,667.92375411)
\curveto(178.47955706,667.76461038)(178.77131825,667.68503915)(179.14644296,667.68504016)
\curveto(179.5594524,667.68503915)(179.87015913,667.76082128)(180.07856409,667.91238678)
\curveto(180.21875693,668.01848052)(180.2888554,668.16057201)(180.28885971,668.33866169)
\curveto(180.2888554,668.45991142)(180.25096433,668.56032274)(180.1751864,668.63989595)
\curveto(180.09561097,668.71567611)(179.91752296,668.78577458)(179.64092185,668.85019157)
\curveto(178.35262199,669.13437237)(177.53606954,669.39392617)(177.19126206,669.62885373)
\curveto(176.71383344,669.95471393)(176.47511973,670.40751215)(176.47512022,670.98724976)
\curveto(176.47511973,671.51014213)(176.68162603,671.94967849)(177.09463975,672.30586013)
\curveto(177.50765124,672.6620305)(178.14801024,672.84011851)(179.01571866,672.84012468)
\curveto(179.84174083,672.84011851)(180.45557608,672.70560523)(180.85722625,672.43658444)
\curveto(181.25886666,672.16755211)(181.53547143,671.76969593)(181.68704139,671.2430147)
\lineto(180.18087006,670.9645151)
\curveto(180.11645105,671.1994354)(179.99330509,671.37941796)(179.81143181,671.50446331)
\curveto(179.63333998,671.62949899)(179.37757529,671.69201925)(179.04413698,671.69202427)
\curveto(178.6235431,671.69201925)(178.32230914,671.6332881)(178.14043419,671.51583064)
\curveto(178.01918062,671.43246545)(177.95855491,671.32447592)(177.95855689,671.19186172)
\curveto(177.95855491,671.078184)(178.01160241,670.98156178)(178.11769952,670.90199478)
\curveto(178.26168343,670.79589556)(178.75805638,670.64622586)(179.60681986,670.45298521)
\curveto(180.45936519,670.259737)(181.05425491,670.02291784)(181.39149079,669.74252704)
\curveto(181.72492675,669.45834098)(181.89164744,669.06237935)(181.89165335,668.55464097)
\curveto(181.89164744,668.00142954)(181.66051194,667.52589667)(181.19824617,667.12804096)
\curveto(180.73596996,666.73018431)(180.05203624,666.53125622)(179.14644296,666.53125609)
\curveto(178.32420369,666.53125622)(177.67247738,666.69797691)(177.19126206,667.03141864)
\curveto(176.71383344,667.36485965)(176.40123215,667.81765787)(176.25345727,668.38981467)
}
}
{
\newrgbcolor{curcolor}{0.50196081 0 1}
\pscustom[linewidth=2.32802935,linecolor=curcolor]
{
\newpath
\moveto(130.18849429,569.46527974)
\lineto(205.84945068,569.46527974)
\lineto(205.84945068,616.02586184)
\lineto(154.63280013,616.02586184)
\lineto(154.63280013,625.33798012)
\lineto(130.18849429,625.33798012)
\lineto(130.18849429,569.46527974)
\closepath
}
}
{
\newrgbcolor{curcolor}{0 0 0}
\pscustom[linestyle=none,fillstyle=solid,fillcolor=curcolor]
{
\newpath
\moveto(136.64513056,596.97797359)
\lineto(136.64513056,597.93282937)
\lineto(142.16396966,600.26313219)
\lineto(142.16396966,599.24575608)
\lineto(137.7875473,597.44971781)
\lineto(142.16396966,595.63662855)
\lineto(142.16396966,594.61925244)
\lineto(136.64513056,596.97797359)
}
}
{
\newrgbcolor{curcolor}{0 0 0}
\pscustom[linestyle=none,fillstyle=solid,fillcolor=curcolor]
{
\newpath
\moveto(143.44279455,596.97797359)
\lineto(143.44279455,597.93282937)
\lineto(148.96163365,600.26313219)
\lineto(148.96163365,599.24575608)
\lineto(144.5852113,597.44971781)
\lineto(148.96163365,595.63662855)
\lineto(148.96163365,594.61925244)
\lineto(143.44279455,596.97797359)
}
}
{
\newrgbcolor{curcolor}{0 0 0}
\pscustom[linestyle=none,fillstyle=solid,fillcolor=curcolor]
{
\newpath
\moveto(150.37118285,593.33474406)
\lineto(150.37118285,599.37079672)
\lineto(151.28625298,599.37079672)
\lineto(151.28625298,598.52393057)
\curveto(151.47570662,598.81947569)(151.7276822,599.05629484)(152.04218047,599.23438875)
\curveto(152.35667387,599.41625996)(152.71474444,599.50719852)(153.11639324,599.50720469)
\curveto(153.56350429,599.50719852)(153.92915306,599.41436541)(154.21334066,599.22870508)
\curveto(154.50130814,599.04303297)(154.70402534,598.78347918)(154.82149285,598.45004292)
\curveto(155.29891505,599.15481161)(155.92032852,599.50719852)(156.68573511,599.50720469)
\curveto(157.28440685,599.50719852)(157.74478328,599.34047783)(158.0668658,599.00704213)
\curveto(158.38893138,598.6773842)(158.54996841,598.16774938)(158.54997736,597.47813614)
\lineto(158.54997736,593.33474406)
\lineto(157.53260125,593.33474406)
\lineto(157.53260125,597.13711622)
\curveto(157.53259332,597.54633591)(157.49849136,597.83999167)(157.43029527,598.01808435)
\curveto(157.36587264,598.19995678)(157.24651578,598.34583738)(157.07222435,598.45572659)
\curveto(156.89791799,598.56560556)(156.69330624,598.6205476)(156.45838849,598.62055289)
\curveto(156.03400171,598.6205476)(155.68161481,598.47845611)(155.40122673,598.19427798)
\curveto(155.12082705,597.91387924)(154.98063011,597.46297557)(154.98063549,596.84156562)
\lineto(154.98063549,593.33474406)
\lineto(153.95757571,593.33474406)
\lineto(153.95757571,597.25647319)
\curveto(153.95757136,597.71116205)(153.87421102,598.05218163)(153.70749444,598.27953296)
\curveto(153.54076965,598.50687441)(153.26795398,598.6205476)(152.88904662,598.62055289)
\curveto(152.60107124,598.6205476)(152.33393924,598.54476547)(152.0876498,598.39320627)
\curveto(151.8451445,598.24163695)(151.66895105,598.01997422)(151.55906892,597.72821742)
\curveto(151.44918287,597.43645183)(151.39424083,597.01586101)(151.39424262,596.4664437)
\lineto(151.39424262,593.33474406)
\lineto(150.37118285,593.33474406)
}
}
{
\newrgbcolor{curcolor}{0 0 0}
\pscustom[linestyle=none,fillstyle=solid,fillcolor=curcolor]
{
\newpath
\moveto(160.10730238,600.4904788)
\lineto(160.10730238,601.66699754)
\lineto(161.13036215,601.66699754)
\lineto(161.13036215,600.4904788)
\lineto(160.10730238,600.4904788)
\moveto(160.10730238,593.33474406)
\lineto(160.10730238,599.37079672)
\lineto(161.13036215,599.37079672)
\lineto(161.13036215,593.33474406)
\lineto(160.10730238,593.33474406)
}
}
{
\newrgbcolor{curcolor}{0 0 0}
\pscustom[linestyle=none,fillstyle=solid,fillcolor=curcolor]
{
\newpath
\moveto(166.60941394,593.33474406)
\lineto(166.60941394,594.09635523)
\curveto(166.2267095,593.49767564)(165.66402719,593.19833623)(164.92136532,593.1983361)
\curveto(164.4401458,593.19833623)(163.99682034,593.33095496)(163.59138761,593.59619267)
\curveto(163.18974066,593.86142987)(162.87713938,594.23086775)(162.65358282,594.70450743)
\curveto(162.43381392,595.18193347)(162.32392983,595.72945936)(162.32393023,596.34708673)
\curveto(162.32392983,596.94955164)(162.42434115,597.49518298)(162.6251645,597.98398236)
\curveto(162.82598644,598.47656155)(163.1272204,598.85357765)(163.52886729,599.11503178)
\curveto(163.93051098,599.37647434)(164.3795201,599.50719852)(164.87589599,599.50720469)
\curveto(165.23964727,599.50719852)(165.56361587,599.42952183)(165.84780278,599.27417441)
\curveto(166.13198184,599.12260421)(166.36311734,598.92367612)(166.54120996,598.67738954)
\lineto(166.54120996,601.66699754)
\lineto(167.55858606,601.66699754)
\lineto(167.55858606,593.33474406)
\lineto(166.60941394,593.33474406)
\moveto(163.37540833,596.34708673)
\curveto(163.37540688,595.57410599)(163.53833846,594.99626725)(163.86420355,594.61356878)
\curveto(164.19006477,594.23086775)(164.57465908,594.03951787)(165.01798763,594.03951857)
\curveto(165.4650991,594.03951787)(165.84400975,594.22139498)(166.15472071,594.58515045)
\curveto(166.46921232,594.95269253)(166.62646024,595.51158574)(166.62646494,596.26183175)
\curveto(166.62646024,597.08785403)(166.46731776,597.69411107)(166.14903704,598.08060467)
\curveto(165.83074788,598.46708879)(165.43857536,598.66033322)(164.97251831,598.66033854)
\curveto(164.51782248,598.66033322)(164.13701728,598.474667)(163.83010156,598.10333933)
\curveto(163.52697114,597.73200213)(163.37540688,597.14658518)(163.37540833,596.34708673)
}
}
{
\newrgbcolor{curcolor}{0 0 0}
\pscustom[linestyle=none,fillstyle=solid,fillcolor=curcolor]
{
\newpath
\moveto(173.08879279,593.33474406)
\lineto(173.08879279,594.09635523)
\curveto(172.70608835,593.49767564)(172.14340604,593.19833623)(171.40074417,593.1983361)
\curveto(170.91952465,593.19833623)(170.47619919,593.33095496)(170.07076646,593.59619267)
\curveto(169.66911951,593.86142987)(169.35651823,594.23086775)(169.13296167,594.70450743)
\curveto(168.91319277,595.18193347)(168.80330868,595.72945936)(168.80330908,596.34708673)
\curveto(168.80330868,596.94955164)(168.90372,597.49518298)(169.10454334,597.98398236)
\curveto(169.30536529,598.47656155)(169.60659925,598.85357765)(170.00824614,599.11503178)
\curveto(170.40988983,599.37647434)(170.85889894,599.50719852)(171.35527484,599.50720469)
\curveto(171.71902611,599.50719852)(172.04299472,599.42952183)(172.32718163,599.27417441)
\curveto(172.61136069,599.12260421)(172.84249618,598.92367612)(173.0205888,598.67738954)
\lineto(173.0205888,601.66699754)
\lineto(174.03796491,601.66699754)
\lineto(174.03796491,593.33474406)
\lineto(173.08879279,593.33474406)
\moveto(169.85478718,596.34708673)
\curveto(169.85478573,595.57410599)(170.01771731,594.99626725)(170.3435824,594.61356878)
\curveto(170.66944362,594.23086775)(171.05403793,594.03951787)(171.49736648,594.03951857)
\curveto(171.94447795,594.03951787)(172.3233886,594.22139498)(172.63409956,594.58515045)
\curveto(172.94859117,594.95269253)(173.10583908,595.51158574)(173.10584379,596.26183175)
\curveto(173.10583908,597.08785403)(172.94669661,597.69411107)(172.62841589,598.08060467)
\curveto(172.31012672,598.46708879)(171.9179542,598.66033322)(171.45189715,598.66033854)
\curveto(170.99720133,598.66033322)(170.61639613,598.474667)(170.30948041,598.10333933)
\curveto(170.00634999,597.73200213)(169.85478573,597.14658518)(169.85478718,596.34708673)
}
}
{
\newrgbcolor{curcolor}{0 0 0}
\pscustom[linestyle=none,fillstyle=solid,fillcolor=curcolor]
{
\newpath
\moveto(175.62938974,593.33474406)
\lineto(175.62938974,601.66699754)
\lineto(176.65244951,601.66699754)
\lineto(176.65244951,593.33474406)
\lineto(175.62938974,593.33474406)
}
}
{
\newrgbcolor{curcolor}{0 0 0}
\pscustom[linestyle=none,fillstyle=solid,fillcolor=curcolor]
{
\newpath
\moveto(182.37590069,595.27855763)
\lineto(183.43306245,595.14783333)
\curveto(183.26633581,594.53020716)(182.95752363,594.05088519)(182.50662499,593.70986598)
\curveto(182.05571629,593.36884602)(181.4797721,593.19833623)(180.77879071,593.1983361)
\curveto(179.8959256,593.19833623)(179.1949409,593.46925734)(178.67583451,594.01110025)
\curveto(178.16051483,594.5567309)(177.90285559,595.32023586)(177.90285602,596.3016174)
\curveto(177.90285559,597.31709497)(178.16430394,598.10522912)(178.68720184,598.66602221)
\curveto(179.21009733,599.22680464)(179.88834738,599.50719852)(180.72195406,599.50720469)
\curveto(181.52903049,599.50719852)(182.18833502,599.2324883)(182.69986961,598.6830732)
\curveto(183.21139376,598.13364742)(183.46715845,597.3606697)(183.46716444,596.36413772)
\curveto(183.46715845,596.30350899)(183.4652639,596.21257043)(183.46148078,596.09132178)
\lineto(178.96001778,596.09132178)
\curveto(178.99790736,595.42822539)(179.18546813,594.92048513)(179.52270066,594.56809946)
\curveto(179.85992909,594.21571132)(180.28051991,594.03951787)(180.78447437,594.03951857)
\curveto(181.15959261,594.03951787)(181.4797721,594.13803464)(181.74501383,594.33506918)
\curveto(182.01024701,594.53210171)(182.22054242,594.84659755)(182.37590069,595.27855763)
\moveto(179.01685444,596.93250426)
\lineto(182.38726802,596.93250426)
\curveto(182.34179383,597.44024093)(182.21296421,597.82104613)(182.00077877,598.07492101)
\curveto(181.67491109,598.46898334)(181.25242572,598.66601688)(180.73332139,598.66602221)
\curveto(180.26346893,598.66601688)(179.8675073,598.50876896)(179.54543532,598.19427798)
\curveto(179.2271483,597.87977728)(179.05095485,597.45918646)(179.01685444,596.93250426)
}
}
{
\newrgbcolor{curcolor}{0 0 0}
\pscustom[linestyle=none,fillstyle=solid,fillcolor=curcolor]
{
\newpath
\moveto(190.11137137,596.97797359)
\lineto(184.59253226,594.61925244)
\lineto(184.59253226,595.63662855)
\lineto(188.96327095,597.44971781)
\lineto(184.59253226,599.24575608)
\lineto(184.59253226,600.26313219)
\lineto(190.11137137,597.93282937)
\lineto(190.11137137,596.97797359)
}
}
{
\newrgbcolor{curcolor}{0 0 0}
\pscustom[linestyle=none,fillstyle=solid,fillcolor=curcolor]
{
\newpath
\moveto(196.90903403,596.97797359)
\lineto(191.39019492,594.61925244)
\lineto(191.39019492,595.63662855)
\lineto(195.76093362,597.44971781)
\lineto(191.39019492,599.24575608)
\lineto(191.39019492,600.26313219)
\lineto(196.90903403,597.93282937)
\lineto(196.90903403,596.97797359)
}
}
{
\newrgbcolor{curcolor}{0 0 0}
\pscustom[linestyle=none,fillstyle=solid,fillcolor=curcolor]
{
\newpath
\moveto(157.11315172,575.87451419)
\lineto(157.11315172,576.77821699)
\curveto(156.89337874,576.45614203)(156.60351209,576.2022719)(156.24355091,576.01660582)
\curveto(155.88737097,575.83093946)(155.51035487,575.73810635)(155.1125015,575.73810622)
\curveto(154.7070643,575.73810635)(154.34331008,575.82715036)(154.02123774,576.00523849)
\curveto(153.69916198,576.18332637)(153.46613193,576.43340739)(153.3221469,576.75548232)
\curveto(153.17815984,577.07755549)(153.10616681,577.5227755)(153.10616761,578.09114369)
\lineto(153.10616761,581.91056684)
\lineto(154.70327759,581.91056684)
\lineto(154.70327759,579.13693813)
\curveto(154.70327519,578.28817501)(154.73169349,577.76717287)(154.78853257,577.57393014)
\curveto(154.84915579,577.38447312)(154.95714533,577.23290886)(155.1125015,577.11923691)
\curveto(155.26785206,577.00935158)(155.46488559,576.95440953)(155.7036027,576.95441061)
\curveto(155.97641497,576.95440953)(156.22081234,577.02829711)(156.43679554,577.17607356)
\curveto(156.65277048,577.32763652)(156.80054563,577.51330274)(156.88012144,577.73307277)
\curveto(156.9596881,577.9566282)(156.99947372,578.50036498)(156.99947841,579.36428474)
\lineto(156.99947841,581.91056684)
\lineto(158.59658839,581.91056684)
\lineto(158.59658839,575.87451419)
\lineto(157.11315172,575.87451419)
}
}
{
\newrgbcolor{curcolor}{0 0 0}
\pscustom[linestyle=none,fillstyle=solid,fillcolor=curcolor]
{
\newpath
\moveto(159.69353587,577.5966648)
\lineto(161.29632951,577.84106242)
\curveto(161.36453155,577.53035372)(161.50283394,577.29353456)(161.71123709,577.13060424)
\curveto(161.91963565,576.97146051)(162.21139685,576.89188928)(162.58652156,576.89189029)
\curveto(162.999531,576.89188928)(163.31023773,576.96767141)(163.51864268,577.11923691)
\curveto(163.65883553,577.22533065)(163.72893399,577.36742214)(163.7289383,577.54551181)
\curveto(163.72893399,577.66676155)(163.69104293,577.76717287)(163.615265,577.84674608)
\curveto(163.53568956,577.92252624)(163.35760156,577.99262471)(163.08100045,578.0570417)
\curveto(161.79270059,578.3412225)(160.97614814,578.6007763)(160.63134066,578.83570386)
\curveto(160.15391203,579.16156406)(159.91519833,579.61436228)(159.91519882,580.19409989)
\curveto(159.91519833,580.71699226)(160.12170463,581.15652862)(160.53471835,581.51271026)
\curveto(160.94772984,581.86888063)(161.58808884,582.04696864)(162.45579725,582.04697481)
\curveto(163.28181943,582.04696864)(163.89565468,581.91245536)(164.29730484,581.64343457)
\curveto(164.69894525,581.37440224)(164.97555003,580.97654606)(165.12711999,580.44986483)
\lineto(163.62094866,580.17136523)
\curveto(163.55652965,580.40628553)(163.43338369,580.58626809)(163.25151041,580.71131344)
\curveto(163.07341857,580.83634912)(162.81765389,580.89886938)(162.48421558,580.8988744)
\curveto(162.0636217,580.89886938)(161.76238773,580.84013822)(161.58051278,580.72268077)
\curveto(161.45925922,580.63931558)(161.39863351,580.53132605)(161.39863549,580.39871184)
\curveto(161.39863351,580.28503413)(161.451681,580.18841191)(161.55777812,580.10884491)
\curveto(161.70176203,580.00274569)(162.19813498,579.85307599)(163.04689846,579.65983534)
\curveto(163.89944379,579.46658713)(164.4943335,579.22976797)(164.83156939,578.94937717)
\curveto(165.16500535,578.66519111)(165.33172603,578.26922948)(165.33173195,577.7614911)
\curveto(165.33172603,577.20827967)(165.10059054,576.7327468)(164.63832477,576.33489108)
\curveto(164.17604856,575.93703444)(163.49211484,575.73810635)(162.58652156,575.73810622)
\curveto(161.76428229,575.73810635)(161.11255597,575.90482704)(160.63134066,576.23826877)
\curveto(160.15391203,576.57170978)(159.84131075,577.024508)(159.69353587,577.5966648)
}
}
{
\newrgbcolor{curcolor}{0 0 0}
\pscustom[linestyle=none,fillstyle=solid,fillcolor=curcolor]
{
\newpath
\moveto(170.23105093,577.79559309)
\lineto(171.82247724,577.52846082)
\curveto(171.61785956,576.94493677)(171.29389096,576.49971676)(170.85057045,576.19279945)
\curveto(170.41102915,575.88967061)(169.85971416,575.73810635)(169.19662382,575.73810622)
\curveto(168.14703803,575.73810635)(167.3702712,576.08102049)(166.86632101,576.76684966)
\curveto(166.46846386,577.3162692)(166.26953577,578.00967569)(166.26953614,578.84707119)
\curveto(166.26953577,579.84739233)(166.53098412,580.62984282)(167.05388197,581.194425)
\curveto(167.57677751,581.76278565)(168.23797659,582.04696864)(169.03748119,582.04697481)
\curveto(169.93549629,582.04696864)(170.6440592,581.74952378)(171.16317205,581.15463934)
\curveto(171.68227437,580.56353345)(171.93046085,579.65604245)(171.90773222,578.43216362)
\lineto(167.90643178,578.43216362)
\curveto(167.91779709,577.95852275)(168.04662671,577.58908487)(168.29292102,577.32384886)
\curveto(168.53921055,577.06239907)(168.84612818,576.93167489)(169.21367482,576.93167595)
\curveto(169.46375253,576.93167489)(169.67404794,576.99987881)(169.84456168,577.13628791)
\curveto(170.01506753,577.27269448)(170.14389715,577.49246265)(170.23105093,577.79559309)
\moveto(170.32198957,579.40975407)
\curveto(170.31061783,579.87202152)(170.19126098,580.22251387)(169.96391865,580.46123216)
\curveto(169.7365682,580.70373039)(169.45996343,580.8249818)(169.1341035,580.82498675)
\curveto(168.78550247,580.8249818)(168.49753038,580.69804673)(168.27018636,580.44418117)
\curveto(168.0428376,580.19030646)(167.93105896,579.84549777)(167.9348501,579.40975407)
\lineto(170.32198957,579.40975407)
}
}
{
\newrgbcolor{curcolor}{0 0 0}
\pscustom[linestyle=none,fillstyle=solid,fillcolor=curcolor]
{
\newpath
\moveto(174.74388155,575.87451419)
\lineto(173.14677157,575.87451419)
\lineto(173.14677157,581.91056684)
\lineto(174.63020824,581.91056684)
\lineto(174.63020824,581.05233337)
\curveto(174.88407612,581.45776258)(175.11142251,581.72489459)(175.31224809,581.85373019)
\curveto(175.5168569,581.98255383)(175.7479924,582.04696864)(176.00565527,582.04697481)
\curveto(176.36940586,582.04696864)(176.71989821,581.94655732)(177.05713336,581.74574054)
\lineto(176.56265447,580.35324252)
\curveto(176.29362373,580.52753694)(176.0435427,580.61468639)(175.81241064,580.61469113)
\curveto(175.58884993,580.61468639)(175.3993946,580.55216613)(175.2440441,580.42713017)
\curveto(175.08868787,580.30587421)(174.96554191,580.08421148)(174.87460585,579.76214132)
\curveto(174.78745391,579.44006338)(174.74387918,578.76560243)(174.74388155,577.73875644)
\lineto(174.74388155,575.87451419)
}
}
{
\newrgbcolor{curcolor}{0 0 0}
\pscustom[linestyle=none,fillstyle=solid,fillcolor=curcolor]
{
\newpath
\moveto(177.17648939,577.5966648)
\lineto(178.77928303,577.84106242)
\curveto(178.84748507,577.53035372)(178.98578746,577.29353456)(179.19419061,577.13060424)
\curveto(179.40258917,576.97146051)(179.69435037,576.89188928)(180.06947508,576.89189029)
\curveto(180.48248452,576.89188928)(180.79319125,576.96767141)(181.0015962,577.11923691)
\curveto(181.14178904,577.22533065)(181.21188751,577.36742214)(181.21189182,577.54551181)
\curveto(181.21188751,577.66676155)(181.17399645,577.76717287)(181.09821851,577.84674608)
\curveto(181.01864308,577.92252624)(180.84055508,577.99262471)(180.56395397,578.0570417)
\curveto(179.2756541,578.3412225)(178.45910166,578.6007763)(178.11429418,578.83570386)
\curveto(177.63686555,579.16156406)(177.39815185,579.61436228)(177.39815234,580.19409989)
\curveto(177.39815185,580.71699226)(177.60465815,581.15652862)(178.01767187,581.51271026)
\curveto(178.43068336,581.86888063)(179.07104235,582.04696864)(179.93875077,582.04697481)
\curveto(180.76477295,582.04696864)(181.3786082,581.91245536)(181.78025836,581.64343457)
\curveto(182.18189877,581.37440224)(182.45850354,580.97654606)(182.61007351,580.44986483)
\lineto(181.10390218,580.17136523)
\curveto(181.03948317,580.40628553)(180.91633721,580.58626809)(180.73446393,580.71131344)
\curveto(180.55637209,580.83634912)(180.30060741,580.89886938)(179.9671691,580.8988744)
\curveto(179.54657522,580.89886938)(179.24534125,580.84013822)(179.0634663,580.72268077)
\curveto(178.94221273,580.63931558)(178.88158703,580.53132605)(178.88158901,580.39871184)
\curveto(178.88158703,580.28503413)(178.93463452,580.18841191)(179.04073164,580.10884491)
\curveto(179.18471555,580.00274569)(179.6810885,579.85307599)(180.52985197,579.65983534)
\curveto(181.38239731,579.46658713)(181.97728702,579.22976797)(182.31452291,578.94937717)
\curveto(182.64795887,578.66519111)(182.81467955,578.26922948)(182.81468546,577.7614911)
\curveto(182.81467955,577.20827967)(182.58354406,576.7327468)(182.12127829,576.33489108)
\curveto(181.65900208,575.93703444)(180.97506836,575.73810635)(180.06947508,575.73810622)
\curveto(179.24723581,575.73810635)(178.59550949,575.90482704)(178.11429418,576.23826877)
\curveto(177.63686555,576.57170978)(177.32426427,577.024508)(177.17648939,577.5966648)
}
}
{
\newrgbcolor{curcolor}{0 1 0.25098041}
\pscustom[linewidth=2.32802935,linecolor=curcolor]
{
\newpath
\moveto(131.35250442,478.6721358)
\lineto(205.84945068,478.6721358)
\lineto(205.84945068,525.2327179)
\lineto(155.79681957,525.2327179)
\lineto(155.79681957,534.54483618)
\lineto(131.35250442,534.54483618)
\lineto(131.35250442,478.6721358)
\closepath
}
}
{
\newrgbcolor{curcolor}{0 0 0}
\pscustom[linestyle=none,fillstyle=solid,fillcolor=curcolor]
{
\newpath
\moveto(144.79323594,506.18482371)
\lineto(144.79323594,507.1396795)
\lineto(150.31207504,509.46998232)
\lineto(150.31207504,508.45260621)
\lineto(145.93565268,506.65656794)
\lineto(150.31207504,504.84347868)
\lineto(150.31207504,503.82610257)
\lineto(144.79323594,506.18482371)
}
}
{
\newrgbcolor{curcolor}{0 0 0}
\pscustom[linestyle=none,fillstyle=solid,fillcolor=curcolor]
{
\newpath
\moveto(151.59089993,506.18482371)
\lineto(151.59089993,507.1396795)
\lineto(157.10973903,509.46998232)
\lineto(157.10973903,508.45260621)
\lineto(152.73331668,506.65656794)
\lineto(157.10973903,504.84347868)
\lineto(157.10973903,503.82610257)
\lineto(151.59089993,506.18482371)
}
}
{
\newrgbcolor{curcolor}{0 0 0}
\pscustom[linestyle=none,fillstyle=solid,fillcolor=curcolor]
{
\newpath
\moveto(162.45806835,503.28615436)
\curveto(162.079153,502.96407957)(161.71350422,502.73673318)(161.36112093,502.60411451)
\curveto(161.01251952,502.47149572)(160.63739798,502.40518636)(160.23575518,502.40518622)
\curveto(159.57265906,502.40518636)(159.06302424,502.56622339)(158.70684919,502.88829778)
\curveto(158.35067222,503.21416059)(158.17258422,503.62906775)(158.17258464,504.13302051)
\curveto(158.17258422,504.42856922)(158.23889358,504.69759578)(158.37151293,504.94010099)
\curveto(158.50792014,505.18639052)(158.68411359,505.38342405)(158.90009381,505.53120219)
\curveto(159.11986084,505.67897436)(159.36615276,505.790753)(159.63897031,505.86653845)
\curveto(159.83979107,505.91958262)(160.14291959,505.97073556)(160.54835678,506.01999742)
\curveto(161.37437919,506.11851071)(161.98253078,506.23597301)(162.37281337,506.37238467)
\curveto(162.37659785,506.51257778)(162.37849241,506.60162178)(162.37849703,506.63951695)
\curveto(162.37849241,507.05631456)(162.28187019,507.34997031)(162.0886301,507.52048508)
\curveto(161.82717742,507.7516156)(161.438794,507.86718335)(160.92347869,507.86718867)
\curveto(160.442259,507.86718335)(160.08608299,507.78192845)(159.8549496,507.61142373)
\curveto(159.62760111,507.44469798)(159.45898587,507.14725312)(159.34910338,506.71908826)
\lineto(158.34877827,506.85549623)
\curveto(158.43971623,507.28366095)(158.58938593,507.62846964)(158.79778783,507.88992333)
\curveto(159.00618764,508.15515544)(159.30742161,508.35787264)(159.70149063,508.49807553)
\curveto(160.09555576,508.64205562)(160.55214309,508.71404864)(161.07125399,508.71405482)
\curveto(161.58656915,508.71404864)(162.00526542,508.65342294)(162.32734405,508.53217752)
\curveto(162.64941352,508.41092013)(162.88623268,508.25746131)(163.03780222,508.07180063)
\curveto(163.18936119,507.88991799)(163.29545617,507.65878249)(163.35608748,507.37839345)
\curveto(163.39018384,507.20408971)(163.40723482,506.88959388)(163.40724047,506.43490499)
\lineto(163.40724047,505.0708253)
\curveto(163.40723482,504.11975704)(163.4280749,503.51728911)(163.46976079,503.2634197)
\curveto(163.51522435,503.01333795)(163.6023738,502.77272969)(163.7312094,502.54159419)
\lineto(162.6626803,502.54159419)
\curveto(162.55658041,502.75378416)(162.4883765,503.00197063)(162.45806835,503.28615436)
\moveto(162.37281337,505.57098785)
\curveto(162.00147631,505.41942056)(161.44447766,505.29059094)(160.70181574,505.1844986)
\curveto(160.28122197,505.12387026)(159.98377711,505.05566634)(159.80948027,504.97988665)
\curveto(159.63517932,504.90410208)(159.50066604,504.79232344)(159.40594003,504.64455039)
\curveto(159.31121071,504.50056224)(159.26384688,504.33952522)(159.2638484,504.16143883)
\curveto(159.26384688,503.88862155)(159.36615276,503.66127516)(159.57076633,503.47939899)
\curveto(159.77916536,503.29752094)(160.08229388,503.20658238)(160.48015279,503.20658305)
\curveto(160.87421714,503.20658238)(161.22470949,503.29183728)(161.53163089,503.46234799)
\curveto(161.83854473,503.63664597)(162.06399657,503.87346512)(162.20798707,504.17280616)
\curveto(162.3178667,504.40394003)(162.37280875,504.74495961)(162.37281337,505.19586594)
\lineto(162.37281337,505.57098785)
}
}
{
\newrgbcolor{curcolor}{0 0 0}
\pscustom[linestyle=none,fillstyle=solid,fillcolor=curcolor]
{
\newpath
\moveto(164.99866619,500.22834238)
\lineto(164.99866619,508.57764685)
\lineto(165.93078731,508.57764685)
\lineto(165.93078731,507.79330102)
\curveto(166.15055379,508.1002134)(166.39874026,508.32945434)(166.67534748,508.48102454)
\curveto(166.95194981,508.63637196)(167.28728573,508.71404864)(167.68135626,508.71405482)
\curveto(168.19667129,508.71404864)(168.65136407,508.58142992)(169.04543595,508.31619824)
\curveto(169.43949821,508.05095501)(169.73694307,507.67583347)(169.93777142,507.19083249)
\curveto(170.13858836,506.70961132)(170.23899968,506.18103097)(170.23900569,505.60508984)
\curveto(170.23899968,504.98746243)(170.12722104,504.43046377)(169.90366943,503.93409222)
\curveto(169.68389558,503.44150698)(169.36182153,503.06259633)(168.93744631,502.79735914)
\curveto(168.51685079,502.53591053)(168.07352533,502.40518636)(167.60746861,502.40518622)
\curveto(167.26644565,502.40518636)(166.95952802,502.47717938)(166.68671481,502.62116551)
\curveto(166.4176858,502.76515148)(166.19602307,502.94702859)(166.02172596,503.16679739)
\lineto(166.02172596,500.22834238)
\lineto(164.99866619,500.22834238)
\moveto(165.92510365,505.52551853)
\curveto(165.92510196,504.74874872)(166.08234987,504.17469909)(166.39684788,503.80336791)
\curveto(166.71134155,503.43203422)(167.09214675,503.246368)(167.53926462,503.2463687)
\curveto(167.99395409,503.246368)(168.38233751,503.43771788)(168.70441603,503.82041891)
\curveto(169.03027471,504.20690649)(169.19320629,504.80369076)(169.19321125,505.61077351)
\curveto(169.19320629,506.37995906)(169.03406382,506.95590324)(168.71578336,507.33860779)
\curveto(168.40128304,507.72130275)(168.02426694,507.91265262)(167.58473395,507.912658)
\curveto(167.14898335,507.91265262)(166.76249449,507.70804088)(166.4252662,507.29882213)
\curveto(166.09182264,506.89338298)(165.92510196,506.30228237)(165.92510365,505.52551853)
}
}
{
\newrgbcolor{curcolor}{0 0 0}
\pscustom[linestyle=none,fillstyle=solid,fillcolor=curcolor]
{
\newpath
\moveto(171.47804504,500.22834238)
\lineto(171.47804504,508.57764685)
\lineto(172.41016616,508.57764685)
\lineto(172.41016616,507.79330102)
\curveto(172.62993264,508.1002134)(172.87811911,508.32945434)(173.15472633,508.48102454)
\curveto(173.43132866,508.63637196)(173.76666458,508.71404864)(174.16073511,508.71405482)
\curveto(174.67605014,508.71404864)(175.13074291,508.58142992)(175.5248148,508.31619824)
\curveto(175.91887706,508.05095501)(176.21632192,507.67583347)(176.41715027,507.19083249)
\curveto(176.61796721,506.70961132)(176.71837853,506.18103097)(176.71838454,505.60508984)
\curveto(176.71837853,504.98746243)(176.60659989,504.43046377)(176.38304828,503.93409222)
\curveto(176.16327443,503.44150698)(175.84120038,503.06259633)(175.41682516,502.79735914)
\curveto(174.99622963,502.53591053)(174.55290418,502.40518636)(174.08684746,502.40518622)
\curveto(173.7458245,502.40518636)(173.43890687,502.47717938)(173.16609366,502.62116551)
\curveto(172.89706465,502.76515148)(172.67540192,502.94702859)(172.50110481,503.16679739)
\lineto(172.50110481,500.22834238)
\lineto(171.47804504,500.22834238)
\moveto(172.4044825,505.52551853)
\curveto(172.4044808,504.74874872)(172.56172872,504.17469909)(172.87622673,503.80336791)
\curveto(173.1907204,503.43203422)(173.5715256,503.246368)(174.01864347,503.2463687)
\curveto(174.47333294,503.246368)(174.86171635,503.43771788)(175.18379488,503.82041891)
\curveto(175.50965356,504.20690649)(175.67258514,504.80369076)(175.6725901,505.61077351)
\curveto(175.67258514,506.37995906)(175.51344267,506.95590324)(175.19516221,507.33860779)
\curveto(174.88066189,507.72130275)(174.50364579,507.91265262)(174.06411279,507.912658)
\curveto(173.6283622,507.91265262)(173.24187334,507.70804088)(172.90464505,507.29882213)
\curveto(172.57120149,506.89338298)(172.4044808,506.30228237)(172.4044825,505.52551853)
}
}
{
\newrgbcolor{curcolor}{0 0 0}
\pscustom[linestyle=none,fillstyle=solid,fillcolor=curcolor]
{
\newpath
\moveto(183.34553868,506.18482371)
\lineto(177.82669958,503.82610257)
\lineto(177.82669958,504.84347868)
\lineto(182.19743827,506.65656794)
\lineto(177.82669958,508.45260621)
\lineto(177.82669958,509.46998232)
\lineto(183.34553868,507.1396795)
\lineto(183.34553868,506.18482371)
}
}
{
\newrgbcolor{curcolor}{0 0 0}
\pscustom[linestyle=none,fillstyle=solid,fillcolor=curcolor]
{
\newpath
\moveto(190.14320135,506.18482371)
\lineto(184.62436224,503.82610257)
\lineto(184.62436224,504.84347868)
\lineto(188.99510094,506.65656794)
\lineto(184.62436224,508.45260621)
\lineto(184.62436224,509.46998232)
\lineto(190.14320135,507.1396795)
\lineto(190.14320135,506.18482371)
}
}
{
\newrgbcolor{curcolor}{0 0 0}
\pscustom[linestyle=none,fillstyle=solid,fillcolor=curcolor]
{
\newpath
\moveto(150.25523839,489.3327176)
\lineto(148.68086307,489.04853433)
\curveto(148.62781106,489.3630262)(148.50655965,489.59984536)(148.31710848,489.75899251)
\curveto(148.13143811,489.9181303)(147.88893529,489.99770154)(147.58959931,489.99770645)
\curveto(147.1917397,489.99770154)(146.87345476,489.85939915)(146.63474353,489.58279888)
\curveto(146.39981645,489.30997871)(146.28235415,488.85149683)(146.28235627,488.20735185)
\curveto(146.28235415,487.4912076)(146.401711,486.98536189)(146.64042719,486.68981319)
\curveto(146.88292752,486.39426128)(147.20689613,486.24648613)(147.61233397,486.24648729)
\curveto(147.91545904,486.24648613)(148.16364551,486.33174102)(148.35689414,486.50225223)
\curveto(148.55013437,486.67654971)(148.68654221,486.97399457)(148.76611805,487.3945877)
\lineto(150.3348097,487.12745543)
\curveto(150.17187194,486.40752315)(149.85927066,485.86378637)(149.39700491,485.49624346)
\curveto(148.93472868,485.12869972)(148.31520977,484.94492805)(147.53844632,484.94492792)
\curveto(146.65558113,484.94492805)(145.95080733,485.22342738)(145.4241228,485.78042673)
\curveto(144.90122484,486.33742468)(144.63977649,487.10850785)(144.63977697,488.09367855)
\curveto(144.63977649,489.09021054)(144.90311939,489.86508281)(145.42980646,490.41829769)
\curveto(145.95649099,490.97529101)(146.66884301,491.25379033)(147.56686465,491.25379651)
\curveto(148.3019479,491.25379033)(148.8854703,491.09464786)(149.31743359,490.77636861)
\curveto(149.75317568,490.46186708)(150.06577696,489.98065056)(150.25523839,489.3327176)
}
}
{
\newrgbcolor{curcolor}{0 0 0}
\pscustom[linestyle=none,fillstyle=solid,fillcolor=curcolor]
{
\newpath
\moveto(151.10210438,488.18461719)
\curveto(151.10210391,488.715089)(151.23282809,489.22851292)(151.49427729,489.72489052)
\curveto(151.75572478,490.22125882)(152.12516266,490.60016947)(152.60259205,490.86162359)
\curveto(153.0838066,491.12306616)(153.61996517,491.25379033)(154.21106935,491.25379651)
\curveto(155.12424044,491.25379033)(155.87258897,490.95634548)(156.45611719,490.36146104)
\curveto(157.03963376,489.77035515)(157.33139496,489.02200662)(157.33140166,488.11641321)
\curveto(157.33139496,487.20323551)(157.03584466,486.44541422)(156.44474985,485.84294705)
\curveto(155.85743254,485.24426746)(155.11666223,484.94492805)(154.22243668,484.94492792)
\curveto(153.66922355,484.94492805)(153.1406432,485.06996857)(152.63669404,485.32004983)
\curveto(152.13652998,485.57013062)(151.75572478,485.9357794)(151.49427729,486.41699725)
\curveto(151.23282809,486.90200155)(151.10210391,487.4912076)(151.10210438,488.18461719)
\moveto(152.73900002,488.09936221)
\curveto(152.73899791,487.50068037)(152.8810894,487.04219849)(153.16527492,486.72391518)
\curveto(153.44945538,486.4056286)(153.79994773,486.24648613)(154.21675302,486.24648729)
\curveto(154.63355115,486.24648613)(154.98214895,486.4056286)(155.26254745,486.72391518)
\curveto(155.54672581,487.04219849)(155.6888173,487.50446948)(155.68882236,488.11072954)
\curveto(155.6888173,488.70182712)(155.54672581,489.1565199)(155.26254745,489.47480924)
\curveto(154.98214895,489.79308979)(154.63355115,489.95223226)(154.21675302,489.95223713)
\curveto(153.79994773,489.95223226)(153.44945538,489.79308979)(153.16527492,489.47480924)
\curveto(152.8810894,489.1565199)(152.73899791,488.69803802)(152.73900002,488.09936221)
}
}
{
\newrgbcolor{curcolor}{0 0 0}
\pscustom[linestyle=none,fillstyle=solid,fillcolor=curcolor]
{
\newpath
\moveto(164.0779121,485.08133588)
\lineto(162.48080212,485.08133588)
\lineto(162.48080212,488.16188253)
\curveto(162.48079739,488.81360576)(162.44669544,489.23419658)(162.37849615,489.42365625)
\curveto(162.3102876,489.61689634)(162.19850896,489.76656604)(162.04315989,489.87266582)
\curveto(161.89159134,489.97875601)(161.70781967,490.0318035)(161.49184434,490.03180845)
\curveto(161.21523583,490.0318035)(160.96704936,489.95602137)(160.74728418,489.80446183)
\curveto(160.52751301,489.65289285)(160.37594875,489.45207021)(160.29259094,489.2019933)
\curveto(160.21301717,488.95190815)(160.17323155,488.48963716)(160.17323397,487.81517894)
\lineto(160.17323397,485.08133588)
\lineto(158.57612399,485.08133588)
\lineto(158.57612399,491.11738854)
\lineto(160.05956066,491.11738854)
\lineto(160.05956066,490.23073674)
\curveto(160.58624416,490.91277075)(161.24933779,491.25379033)(162.04884355,491.25379651)
\curveto(162.40122616,491.25379033)(162.72330021,491.18937552)(163.01506667,491.06055188)
\curveto(163.30682261,490.93550539)(163.52659078,490.77446837)(163.67437186,490.57744033)
\curveto(163.82593019,490.38040129)(163.93013062,490.15684401)(163.98697345,489.90676781)
\curveto(164.04759292,489.65668196)(164.07790577,489.29861139)(164.0779121,488.83255505)
\lineto(164.0779121,485.08133588)
}
}
{
\newrgbcolor{curcolor}{0 0 0}
\pscustom[linestyle=none,fillstyle=solid,fillcolor=curcolor]
{
\newpath
\moveto(168.4713864,491.11738854)
\lineto(168.4713864,489.84424749)
\lineto(167.38012264,489.84424749)
\lineto(167.38012264,487.4116387)
\curveto(167.38012013,486.91905253)(167.3895929,486.63108043)(167.40854097,486.54772156)
\curveto(167.43127307,486.46814886)(167.4786369,486.40183949)(167.55063261,486.34879327)
\curveto(167.62641205,486.29574451)(167.71735061,486.26922077)(167.82344854,486.26922195)
\curveto(167.97122074,486.26922077)(168.18530526,486.3203737)(168.46570273,486.42268092)
\lineto(168.6021107,485.18364186)
\curveto(168.23077454,485.02449929)(167.81018372,484.94492805)(167.34033699,484.94492792)
\curveto(167.05236242,484.94492805)(166.79280863,484.99229188)(166.56167483,485.08701955)
\curveto(166.33053764,485.18553631)(166.16002785,485.31057683)(166.05014494,485.46214147)
\curveto(165.94404878,485.61749445)(165.8701612,485.82589531)(165.82848199,486.08734466)
\curveto(165.79437907,486.27300987)(165.77732809,486.64813141)(165.777329,487.21271041)
\lineto(165.777329,489.84424749)
\lineto(165.04413616,489.84424749)
\lineto(165.04413616,491.11738854)
\lineto(165.777329,491.11738854)
\lineto(165.777329,492.31664194)
\lineto(167.38012264,493.24876306)
\lineto(167.38012264,491.11738854)
\lineto(168.4713864,491.11738854)
}
}
{
\newrgbcolor{curcolor}{0 0 0}
\pscustom[linestyle=none,fillstyle=solid,fillcolor=curcolor]
{
\newpath
\moveto(170.78463839,489.27588095)
\lineto(169.33530372,489.53732956)
\curveto(169.49823472,490.1208475)(169.7786286,490.55280564)(170.1764862,490.83320527)
\curveto(170.57434096,491.1135934)(171.16544157,491.25379033)(171.9497898,491.25379651)
\curveto(172.66213862,491.25379033)(173.19261353,491.16853544)(173.54121611,490.99803156)
\curveto(173.88980912,490.83130496)(174.13420649,490.61722045)(174.27440895,490.35577737)
\curveto(174.41838948,490.09811286)(174.4903825,489.62258)(174.49038823,488.92917736)
\lineto(174.47333724,487.06493511)
\curveto(174.47333152,486.53445822)(174.49796071,486.1422857)(174.54722489,485.88841637)
\curveto(174.60026659,485.63833454)(174.6968888,485.36930798)(174.83709182,485.08133588)
\lineto(173.25703284,485.08133588)
\curveto(173.21534817,485.18743087)(173.16419523,485.34467878)(173.10357388,485.55308011)
\curveto(173.07704578,485.6478073)(173.05810025,485.71032756)(173.04673722,485.74064107)
\curveto(172.77391727,485.47540296)(172.48215607,485.27647487)(172.17145275,485.1438562)
\curveto(171.8607426,485.01123741)(171.52919579,484.94492805)(171.17681131,484.94492792)
\curveto(170.55539542,484.94492805)(170.06470613,485.11354329)(169.70474197,485.45077414)
\curveto(169.34856501,485.78800424)(169.17047701,486.21427872)(169.17047742,486.72959885)
\curveto(169.17047701,487.07061678)(169.2519428,487.3737453)(169.41487503,487.63898531)
\curveto(169.57780595,487.90800932)(169.80515234,488.11262107)(170.09691488,488.25282118)
\curveto(170.39246384,488.39680405)(170.81684377,488.52184457)(171.37005593,488.62794309)
\curveto(172.11650729,488.76813649)(172.63372033,488.89886066)(172.92169658,489.02011601)
\lineto(172.92169658,489.17925864)
\curveto(172.92169242,489.48617216)(172.84591029,489.70404579)(172.69434997,489.83288016)
\curveto(172.54278177,489.96549413)(172.25670423,490.0318035)(171.83611649,490.03180845)
\curveto(171.55193043,490.0318035)(171.3302677,489.9749669)(171.17112764,489.86129848)
\curveto(171.01198275,489.75140962)(170.88315313,489.55627063)(170.78463839,489.27588095)
\moveto(172.92169658,487.98000524)
\curveto(172.71708067,487.91179842)(172.39311206,487.83033263)(171.9497898,487.73560763)
\curveto(171.50646115,487.64087731)(171.2165945,487.5480442)(171.08018899,487.45710802)
\curveto(170.87178581,487.30933049)(170.76758539,487.12176972)(170.7675874,486.89442515)
\curveto(170.76758539,486.67086605)(170.85094573,486.47762162)(171.01766868,486.31469128)
\curveto(171.1843871,486.15175846)(171.39657706,486.07029268)(171.6542392,486.07029366)
\curveto(171.94220839,486.07029268)(172.21691861,486.16502034)(172.47837068,486.35447693)
\curveto(172.67161139,486.49846171)(172.79854646,486.67465516)(172.85917626,486.88305782)
\curveto(172.90085233,487.01946385)(172.92169242,487.27901764)(172.92169658,487.66171998)
\lineto(172.92169658,487.98000524)
}
}
{
\newrgbcolor{curcolor}{0 0 0}
\pscustom[linestyle=none,fillstyle=solid,fillcolor=curcolor]
{
\newpath
\moveto(181.33352167,489.3327176)
\lineto(179.75914635,489.04853433)
\curveto(179.70609434,489.3630262)(179.58484293,489.59984536)(179.39539177,489.75899251)
\curveto(179.20972139,489.9181303)(178.96721858,489.99770154)(178.6678826,489.99770645)
\curveto(178.27002298,489.99770154)(177.95173804,489.85939915)(177.71302681,489.58279888)
\curveto(177.47809973,489.30997871)(177.36063743,488.85149683)(177.36063955,488.20735185)
\curveto(177.36063743,487.4912076)(177.47999428,486.98536189)(177.71871048,486.68981319)
\curveto(177.96121081,486.39426128)(178.28517941,486.24648613)(178.69061726,486.24648729)
\curveto(178.99374232,486.24648613)(179.24192879,486.33174102)(179.43517743,486.50225223)
\curveto(179.62841766,486.67654971)(179.76482549,486.97399457)(179.84440133,487.3945877)
\lineto(181.41309298,487.12745543)
\curveto(181.25015523,486.40752315)(180.93755394,485.86378637)(180.47528819,485.49624346)
\curveto(180.01301196,485.12869972)(179.39349305,484.94492805)(178.61672961,484.94492792)
\curveto(177.73386442,484.94492805)(177.02909061,485.22342738)(176.50240608,485.78042673)
\curveto(175.97950812,486.33742468)(175.71805977,487.10850785)(175.71806025,488.09367855)
\curveto(175.71805977,489.09021054)(175.98140267,489.86508281)(176.50808975,490.41829769)
\curveto(177.03477427,490.97529101)(177.74712629,491.25379033)(178.64514794,491.25379651)
\curveto(179.38023118,491.25379033)(179.96375358,491.09464786)(180.39571688,490.77636861)
\curveto(180.83145896,490.46186708)(181.14406025,489.98065056)(181.33352167,489.3327176)
}
}
{
\newrgbcolor{curcolor}{0 0 0}
\pscustom[linestyle=none,fillstyle=solid,fillcolor=curcolor]
{
\newpath
\moveto(185.31776963,491.11738854)
\lineto(185.31776963,489.84424749)
\lineto(184.22650587,489.84424749)
\lineto(184.22650587,487.4116387)
\curveto(184.22650336,486.91905253)(184.23597613,486.63108043)(184.2549242,486.54772156)
\curveto(184.2776563,486.46814886)(184.32502013,486.40183949)(184.39701584,486.34879327)
\curveto(184.47279528,486.29574451)(184.56373384,486.26922077)(184.66983178,486.26922195)
\curveto(184.81760397,486.26922077)(185.03168849,486.3203737)(185.31208597,486.42268092)
\lineto(185.44849393,485.18364186)
\curveto(185.07715777,485.02449929)(184.65656695,484.94492805)(184.18672022,484.94492792)
\curveto(183.89874565,484.94492805)(183.63919186,484.99229188)(183.40805806,485.08701955)
\curveto(183.17692087,485.18553631)(183.00641108,485.31057683)(182.89652817,485.46214147)
\curveto(182.79043201,485.61749445)(182.71654443,485.82589531)(182.67486522,486.08734466)
\curveto(182.6407623,486.27300987)(182.62371132,486.64813141)(182.62371223,487.21271041)
\lineto(182.62371223,489.84424749)
\lineto(181.8905194,489.84424749)
\lineto(181.8905194,491.11738854)
\lineto(182.62371223,491.11738854)
\lineto(182.62371223,492.31664194)
\lineto(184.22650587,493.24876306)
\lineto(184.22650587,491.11738854)
\lineto(185.31776963,491.11738854)
}
}
{
\newrgbcolor{curcolor}{0 0 0}
\pscustom[linestyle=none,fillstyle=solid,fillcolor=curcolor]
{
\newpath
\moveto(185.87477079,486.8034865)
\lineto(187.47756444,487.04788411)
\curveto(187.54576648,486.73717542)(187.68406886,486.50035626)(187.89247201,486.33742594)
\curveto(188.10087058,486.17828221)(188.39263177,486.09871097)(188.76775648,486.09871199)
\curveto(189.18076592,486.09871097)(189.49147265,486.1744931)(189.69987761,486.32605861)
\curveto(189.84007045,486.43215234)(189.91016892,486.57424384)(189.91017323,486.75233351)
\curveto(189.91016892,486.87358325)(189.87227785,486.97399457)(189.79649992,487.05356778)
\curveto(189.71692449,487.12934794)(189.53883648,487.19944641)(189.26223537,487.2638634)
\curveto(187.97393551,487.5480442)(187.15738306,487.80759799)(186.81257558,488.04252556)
\curveto(186.33514696,488.36838575)(186.09643325,488.82118398)(186.09643374,489.40092159)
\curveto(186.09643325,489.92381396)(186.30293955,490.36335031)(186.71595327,490.71953196)
\curveto(187.12896476,491.07570233)(187.76932376,491.25379033)(188.63703218,491.25379651)
\curveto(189.46305435,491.25379033)(190.0768896,491.11927706)(190.47853977,490.85025626)
\curveto(190.88018018,490.58122394)(191.15678495,490.18336776)(191.30835491,489.65668653)
\lineto(189.80218358,489.37818693)
\curveto(189.73776457,489.61310723)(189.61461861,489.79308979)(189.43274533,489.91813514)
\curveto(189.2546535,490.04317082)(188.99888881,490.10569107)(188.6654505,490.1056961)
\curveto(188.24485662,490.10569107)(187.94362266,490.04695992)(187.76174771,489.92950247)
\curveto(187.64049414,489.84613728)(187.57986844,489.73814774)(187.57987041,489.60553354)
\curveto(187.57986844,489.49185582)(187.63291593,489.39523361)(187.73901304,489.31566661)
\curveto(187.88299695,489.20956739)(188.3793699,489.05989769)(189.22813338,488.86665704)
\curveto(190.08067871,488.67340882)(190.67556843,488.43658967)(191.01280431,488.15619887)
\curveto(191.34624027,487.8720128)(191.51296096,487.47605118)(191.51296687,486.9683128)
\curveto(191.51296096,486.41510136)(191.28182546,485.9395685)(190.81955969,485.54171278)
\curveto(190.35728348,485.14385614)(189.67334976,484.94492805)(188.76775648,484.94492792)
\curveto(187.94551721,484.94492805)(187.2937909,485.11164874)(186.81257558,485.44509047)
\curveto(186.33514696,485.77853148)(186.02254567,486.2313297)(185.87477079,486.8034865)
}
}
{
\newrgbcolor{curcolor}{0.50196081 0.50196081 0}
\pscustom[linewidth=2.32802935,linecolor=curcolor]
{
\newpath
\moveto(130.18849429,386.71497242)
\lineto(204.68543124,386.71497242)
\lineto(204.68543124,433.27556384)
\lineto(154.63280013,433.27556384)
\lineto(154.63280013,442.58768212)
\lineto(130.18849429,442.58768212)
\lineto(130.18849429,386.71497242)
\closepath
}
}
{
\newrgbcolor{curcolor}{0 0 0}
\pscustom[linestyle=none,fillstyle=solid,fillcolor=curcolor]
{
\newpath
\moveto(145.95725505,415.39170228)
\lineto(145.95725505,416.34655806)
\lineto(151.47609416,418.67686088)
\lineto(151.47609416,417.65948477)
\lineto(147.0996718,415.8634465)
\lineto(151.47609416,414.05035724)
\lineto(151.47609416,413.03298114)
\lineto(145.95725505,415.39170228)
}
}
{
\newrgbcolor{curcolor}{0 0 0}
\pscustom[linestyle=none,fillstyle=solid,fillcolor=curcolor]
{
\newpath
\moveto(152.75491905,415.39170228)
\lineto(152.75491905,416.34655806)
\lineto(158.27375815,418.67686088)
\lineto(158.27375815,417.65948477)
\lineto(153.89733579,415.8634465)
\lineto(158.27375815,414.05035724)
\lineto(158.27375815,413.03298114)
\lineto(152.75491905,415.39170228)
}
}
{
\newrgbcolor{curcolor}{0 0 0}
\pscustom[linestyle=none,fillstyle=solid,fillcolor=curcolor]
{
\newpath
\moveto(163.63913846,411.74847276)
\lineto(163.63913846,412.63512456)
\curveto(163.16928454,411.95308451)(162.53082009,411.61206492)(161.72374322,411.61206479)
\curveto(161.36756441,411.61206492)(161.03412304,411.68026884)(160.72341811,411.81667674)
\curveto(160.41649868,411.95308451)(160.18725774,412.1235943)(160.0356946,412.32820663)
\curveto(159.88791833,412.5366069)(159.7837179,412.79047704)(159.723093,413.08981779)
\curveto(159.68141202,413.29063909)(159.66057194,413.60892404)(159.66057268,414.04467358)
\lineto(159.66057268,417.78452541)
\lineto(160.68363245,417.78452541)
\lineto(160.68363245,414.43684649)
\curveto(160.68363069,413.90257979)(160.70447077,413.54261467)(160.74615277,413.35695006)
\curveto(160.81056575,413.0879219)(160.94697359,412.87573193)(161.15537668,412.72037954)
\curveto(161.3637753,412.56881431)(161.62143454,412.49303218)(161.92835518,412.49303292)
\curveto(162.23526979,412.49303218)(162.52324188,412.57070886)(162.79227232,412.7260632)
\curveto(163.061295,412.8852047)(163.25075032,413.09928922)(163.36063886,413.36831739)
\curveto(163.47430761,413.64113144)(163.5311442,414.03519851)(163.53114882,414.5505198)
\lineto(163.53114882,417.78452541)
\lineto(164.55420859,417.78452541)
\lineto(164.55420859,411.74847276)
\lineto(163.63913846,411.74847276)
}
}
{
\newrgbcolor{curcolor}{0 0 0}
\pscustom[linestyle=none,fillstyle=solid,fillcolor=curcolor]
{
\newpath
\moveto(168.39636581,412.66354289)
\lineto(168.54414111,411.75984009)
\curveto(168.25616587,411.69921437)(167.99850663,411.66890152)(167.77116261,411.66890144)
\curveto(167.3998278,411.66890152)(167.11185571,411.72763267)(166.90724547,411.84509507)
\curveto(166.70263221,411.96255727)(166.55864616,412.11601608)(166.4752869,412.30547197)
\curveto(166.39192548,412.49871584)(166.35024531,412.90225568)(166.35024626,413.5160927)
\lineto(166.35024626,416.98881225)
\lineto(165.60000243,416.98881225)
\lineto(165.60000243,417.78452541)
\lineto(166.35024626,417.78452541)
\lineto(166.35024626,419.27932941)
\lineto(167.36762237,419.89316527)
\lineto(167.36762237,417.78452541)
\lineto(168.39636581,417.78452541)
\lineto(168.39636581,416.98881225)
\lineto(167.36762237,416.98881225)
\lineto(167.36762237,413.45925604)
\curveto(167.3676204,413.16749313)(167.38467138,412.97993236)(167.41877536,412.89657317)
\curveto(167.4566644,412.81321168)(167.51539555,412.74690231)(167.59496899,412.69764488)
\curveto(167.67832713,412.64838554)(167.79578943,412.62375635)(167.94735624,412.62375723)
\curveto(168.06102688,412.62375635)(168.21069659,412.63701823)(168.39636581,412.66354289)
}
}
{
\newrgbcolor{curcolor}{0 0 0}
\pscustom[linestyle=none,fillstyle=solid,fillcolor=curcolor]
{
\newpath
\moveto(169.39669169,418.90420749)
\lineto(169.39669169,420.08072623)
\lineto(170.41975146,420.08072623)
\lineto(170.41975146,418.90420749)
\lineto(169.39669169,418.90420749)
\moveto(169.39669169,411.74847276)
\lineto(169.39669169,417.78452541)
\lineto(170.41975146,417.78452541)
\lineto(170.41975146,411.74847276)
\lineto(169.39669169,411.74847276)
}
}
{
\newrgbcolor{curcolor}{0 0 0}
\pscustom[linestyle=none,fillstyle=solid,fillcolor=curcolor]
{
\newpath
\moveto(171.9600249,411.74847276)
\lineto(171.9600249,420.08072623)
\lineto(172.98308467,420.08072623)
\lineto(172.98308467,411.74847276)
\lineto(171.9600249,411.74847276)
}
}
{
\newrgbcolor{curcolor}{0 0 0}
\pscustom[linestyle=none,fillstyle=solid,fillcolor=curcolor]
{
\newpath
\moveto(179.9626259,415.39170228)
\lineto(174.4437868,413.03298114)
\lineto(174.4437868,414.05035724)
\lineto(178.81452549,415.8634465)
\lineto(174.4437868,417.65948477)
\lineto(174.4437868,418.67686088)
\lineto(179.9626259,416.34655806)
\lineto(179.9626259,415.39170228)
}
}
{
\newrgbcolor{curcolor}{0 0 0}
\pscustom[linestyle=none,fillstyle=solid,fillcolor=curcolor]
{
\newpath
\moveto(186.76028856,415.39170228)
\lineto(181.24144946,413.03298114)
\lineto(181.24144946,414.05035724)
\lineto(185.61218815,415.8634465)
\lineto(181.24144946,417.65948477)
\lineto(181.24144946,418.67686088)
\lineto(186.76028856,416.34655806)
\lineto(186.76028856,415.39170228)
}
}
{
\newrgbcolor{curcolor}{0 0 0}
\pscustom[linestyle=none,fillstyle=solid,fillcolor=curcolor]
{
\newpath
\moveto(138.00807796,401.14274335)
\lineto(138.00807796,402.62049635)
\lineto(139.60518794,402.62049635)
\lineto(139.60518794,401.14274335)
\lineto(138.00807796,401.14274335)
\moveto(138.00807796,394.28824288)
\lineto(138.00807796,400.32429553)
\lineto(139.60518794,400.32429553)
\lineto(139.60518794,394.28824288)
\lineto(138.00807796,394.28824288)
}
}
{
\newrgbcolor{curcolor}{0 0 0}
\pscustom[linestyle=none,fillstyle=solid,fillcolor=curcolor]
{
\newpath
\moveto(146.72682079,394.28824288)
\lineto(145.12971081,394.28824288)
\lineto(145.12971081,397.36878952)
\curveto(145.12970608,398.02051276)(145.09560412,398.44110358)(145.02740483,398.63056324)
\curveto(144.95919629,398.82380333)(144.84741765,398.97347304)(144.69206857,399.07957281)
\curveto(144.54050002,399.185663)(144.35672836,399.23871049)(144.14075303,399.23871544)
\curveto(143.86414452,399.23871049)(143.61595804,399.16292836)(143.39619286,399.01136883)
\curveto(143.17642169,398.85979984)(143.02485743,398.6589772)(142.94149963,398.40890029)
\curveto(142.86192586,398.15881514)(142.82214024,397.69654415)(142.82214266,397.02208594)
\lineto(142.82214266,394.28824288)
\lineto(141.22503268,394.28824288)
\lineto(141.22503268,400.32429553)
\lineto(142.70846935,400.32429553)
\lineto(142.70846935,399.43764373)
\curveto(143.23515284,400.11967775)(143.89824648,400.46069733)(144.69775224,400.4607035)
\curveto(145.05013485,400.46069733)(145.3722089,400.39628252)(145.66397536,400.26745888)
\curveto(145.95573129,400.14241239)(146.17549947,399.98137536)(146.32328054,399.78434732)
\curveto(146.47483888,399.58730829)(146.57903931,399.363751)(146.63588214,399.1136748)
\curveto(146.69650161,398.86358895)(146.72681446,398.50551839)(146.72682079,398.03946204)
\lineto(146.72682079,394.28824288)
}
}
{
\newrgbcolor{curcolor}{0 0 0}
\pscustom[linestyle=none,fillstyle=solid,fillcolor=curcolor]
{
\newpath
\moveto(150.01197922,394.28824288)
\lineto(147.57937043,400.32429553)
\lineto(149.25605173,400.32429553)
\lineto(150.39278481,397.24374889)
\lineto(150.7224374,396.21500545)
\curveto(150.80958364,396.47645187)(150.86452569,396.64885621)(150.8872637,396.732219)
\curveto(150.94030782,396.90272635)(150.99714441,397.07323614)(151.05777366,397.24374889)
\lineto(152.20587407,400.32429553)
\lineto(153.84845337,400.32429553)
\lineto(151.44994657,394.28824288)
\lineto(150.01197922,394.28824288)
}
}
{
\newrgbcolor{curcolor}{0 0 0}
\pscustom[linestyle=none,fillstyle=solid,fillcolor=curcolor]
{
\newpath
\moveto(154.85446297,401.14274335)
\lineto(154.85446297,402.62049635)
\lineto(156.45157295,402.62049635)
\lineto(156.45157295,401.14274335)
\lineto(154.85446297,401.14274335)
\moveto(154.85446297,394.28824288)
\lineto(154.85446297,400.32429553)
\lineto(156.45157295,400.32429553)
\lineto(156.45157295,394.28824288)
\lineto(154.85446297,394.28824288)
}
}
{
\newrgbcolor{curcolor}{0 0 0}
\pscustom[linestyle=none,fillstyle=solid,fillcolor=curcolor]
{
\newpath
\moveto(160.85072896,400.32429553)
\lineto(160.85072896,399.05115448)
\lineto(159.7594652,399.05115448)
\lineto(159.7594652,396.61854569)
\curveto(159.75946269,396.12595952)(159.76893546,395.83798743)(159.78788353,395.75462855)
\curveto(159.81061563,395.67505585)(159.85797946,395.60874649)(159.92997516,395.55570026)
\curveto(160.00575461,395.5026515)(160.09669317,395.47612776)(160.2027911,395.47612895)
\curveto(160.3505633,395.47612776)(160.56464782,395.5272807)(160.84504529,395.62958791)
\lineto(160.98145326,394.39054886)
\curveto(160.61011709,394.23140628)(160.18952627,394.15183505)(159.71967954,394.15183491)
\curveto(159.43170498,394.15183505)(159.17215119,394.19919888)(158.94101738,394.29392654)
\curveto(158.7098802,394.39244331)(158.5393704,394.51748382)(158.4294875,394.66904846)
\curveto(158.32339134,394.82440145)(158.24950376,395.0328023)(158.20782455,395.29425165)
\curveto(158.17372163,395.47991687)(158.15667065,395.85503841)(158.15667156,396.4196174)
\lineto(158.15667156,399.05115448)
\lineto(157.42347872,399.05115448)
\lineto(157.42347872,400.32429553)
\lineto(158.15667156,400.32429553)
\lineto(158.15667156,401.52354893)
\lineto(159.7594652,402.45567006)
\lineto(159.7594652,400.32429553)
\lineto(160.85072896,400.32429553)
}
}
{
\newrgbcolor{curcolor}{0 0 0}
\pscustom[linestyle=none,fillstyle=solid,fillcolor=curcolor]
{
\newpath
\moveto(163.16398095,398.48278794)
\lineto(161.71464628,398.74423655)
\curveto(161.87757727,399.32775449)(162.15797115,399.75971263)(162.55582875,400.04011226)
\curveto(162.95368351,400.32050039)(163.54478412,400.46069733)(164.32913236,400.4607035)
\curveto(165.04148118,400.46069733)(165.57195609,400.37544243)(165.92055867,400.20493856)
\curveto(166.26915168,400.03821196)(166.51354905,399.82412744)(166.65375151,399.56268437)
\curveto(166.79773203,399.30501985)(166.86972506,398.82948699)(166.86973079,398.13608435)
\lineto(166.8526798,396.2718421)
\curveto(166.85267408,395.74136521)(166.87730327,395.34919269)(166.92656745,395.09532337)
\curveto(166.97960915,394.84524153)(167.07623136,394.57621497)(167.21643438,394.28824288)
\lineto(165.6363754,394.28824288)
\curveto(165.59469073,394.39433786)(165.54353779,394.55158578)(165.48291644,394.75998711)
\curveto(165.45638834,394.8547143)(165.43744281,394.91723455)(165.42607978,394.94754807)
\curveto(165.15325982,394.68230995)(164.86149862,394.48338186)(164.55079531,394.3507632)
\curveto(164.24008516,394.21814441)(163.90853835,394.15183505)(163.55615386,394.15183491)
\curveto(162.93473798,394.15183505)(162.44404869,394.32045028)(162.08408453,394.65768113)
\curveto(161.72790757,394.99491124)(161.54981956,395.42118572)(161.54981998,395.93650584)
\curveto(161.54981956,396.27752378)(161.63128535,396.5806523)(161.79421759,396.84589231)
\curveto(161.95714851,397.11491631)(162.1844949,397.31952806)(162.47625744,397.45972817)
\curveto(162.7718064,397.60371105)(163.19618633,397.72875156)(163.74939849,397.83485009)
\curveto(164.49584985,397.97504348)(165.01306288,398.10576765)(165.30103914,398.227023)
\lineto(165.30103914,398.38616563)
\curveto(165.30103498,398.69307916)(165.22525285,398.91095278)(165.07369253,399.03978715)
\curveto(164.92212433,399.17240113)(164.63604679,399.23871049)(164.21545905,399.23871544)
\curveto(163.93127298,399.23871049)(163.70961026,399.18187389)(163.5504702,399.06820548)
\curveto(163.39132531,398.95831661)(163.26249569,398.76317763)(163.16398095,398.48278794)
\moveto(165.30103914,397.18691223)
\curveto(165.09642323,397.11870542)(164.77245462,397.03723963)(164.32913236,396.94251462)
\curveto(163.88580371,396.8477843)(163.59593706,396.7549512)(163.45953155,396.66401502)
\curveto(163.25112837,396.51623749)(163.14692794,396.32867672)(163.14692996,396.10133214)
\curveto(163.14692794,395.87777305)(163.23028829,395.68452862)(163.39701123,395.52159827)
\curveto(163.56372966,395.35866546)(163.77591962,395.27719967)(164.03358176,395.27720066)
\curveto(164.32155095,395.27719967)(164.59626117,395.37192733)(164.85771324,395.56138393)
\curveto(165.05095395,395.7053687)(165.17788902,395.88156215)(165.23851882,396.08996481)
\curveto(165.28019489,396.22637084)(165.30103498,396.48592464)(165.30103914,396.86862697)
\lineto(165.30103914,397.18691223)
}
}
{
\newrgbcolor{curcolor}{0 0 0}
\pscustom[linestyle=none,fillstyle=solid,fillcolor=curcolor]
{
\newpath
\moveto(171.21773512,400.32429553)
\lineto(171.21773512,399.05115448)
\lineto(170.12647136,399.05115448)
\lineto(170.12647136,396.61854569)
\curveto(170.12646885,396.12595952)(170.13594161,395.83798743)(170.15488969,395.75462855)
\curveto(170.17762179,395.67505585)(170.22498562,395.60874649)(170.29698132,395.55570026)
\curveto(170.37276077,395.5026515)(170.46369932,395.47612776)(170.56979726,395.47612895)
\curveto(170.71756946,395.47612776)(170.93165397,395.5272807)(171.21205145,395.62958791)
\lineto(171.34845942,394.39054886)
\curveto(170.97712325,394.23140628)(170.55653243,394.15183505)(170.0866857,394.15183491)
\curveto(169.79871114,394.15183505)(169.53915734,394.19919888)(169.30802354,394.29392654)
\curveto(169.07688635,394.39244331)(168.90637656,394.51748382)(168.79649366,394.66904846)
\curveto(168.69039749,394.82440145)(168.61650992,395.0328023)(168.57483071,395.29425165)
\curveto(168.54072779,395.47991687)(168.52367681,395.85503841)(168.52367772,396.4196174)
\lineto(168.52367772,399.05115448)
\lineto(167.79048488,399.05115448)
\lineto(167.79048488,400.32429553)
\lineto(168.52367772,400.32429553)
\lineto(168.52367772,401.52354893)
\lineto(170.12647136,402.45567006)
\lineto(170.12647136,400.32429553)
\lineto(171.21773512,400.32429553)
}
}
{
\newrgbcolor{curcolor}{0 0 0}
\pscustom[linestyle=none,fillstyle=solid,fillcolor=curcolor]
{
\newpath
\moveto(172.3374156,401.14274335)
\lineto(172.3374156,402.62049635)
\lineto(173.93452558,402.62049635)
\lineto(173.93452558,401.14274335)
\lineto(172.3374156,401.14274335)
\moveto(172.3374156,394.28824288)
\lineto(172.3374156,400.32429553)
\lineto(173.93452558,400.32429553)
\lineto(173.93452558,394.28824288)
\lineto(172.3374156,394.28824288)
}
}
{
\newrgbcolor{curcolor}{0 0 0}
\pscustom[linestyle=none,fillstyle=solid,fillcolor=curcolor]
{
\newpath
\moveto(175.19630007,397.39152419)
\curveto(175.1962996,397.92199599)(175.32702377,398.43541992)(175.58847298,398.93179751)
\curveto(175.84992047,399.42816581)(176.21935835,399.80707646)(176.69678773,400.06853059)
\curveto(177.17800229,400.32997316)(177.71416085,400.46069733)(178.30526504,400.4607035)
\curveto(179.21843613,400.46069733)(179.96678465,400.16325247)(180.55031287,399.56836803)
\curveto(181.13382945,398.97726214)(181.42559065,398.22891361)(181.42559734,397.3233202)
\curveto(181.42559065,396.41014251)(181.13004034,395.65232121)(180.53894554,395.04985404)
\curveto(179.95162823,394.45117446)(179.21085791,394.15183505)(178.31663237,394.15183491)
\curveto(177.76341924,394.15183505)(177.23483888,394.27687556)(176.73088972,394.52695683)
\curveto(176.23072567,394.77703761)(175.84992047,395.14268639)(175.58847298,395.62390425)
\curveto(175.32702377,396.10890854)(175.1962996,396.6981146)(175.19630007,397.39152419)
\moveto(176.8331957,397.30626921)
\curveto(176.8331936,396.70758736)(176.97528509,396.24910548)(177.25947061,395.93082218)
\curveto(177.54365106,395.61253559)(177.89414341,395.45339312)(178.3109487,395.45339429)
\curveto(178.72774684,395.45339312)(179.07634463,395.61253559)(179.35674314,395.93082218)
\curveto(179.6409215,396.24910548)(179.78301299,396.71137647)(179.78301804,397.31763654)
\curveto(179.78301299,397.90873412)(179.6409215,398.36342689)(179.35674314,398.68171623)
\curveto(179.07634463,398.99999678)(178.72774684,399.15913925)(178.3109487,399.15914413)
\curveto(177.89414341,399.15913925)(177.54365106,398.99999678)(177.25947061,398.68171623)
\curveto(176.97528509,398.36342689)(176.8331936,397.90494501)(176.8331957,397.30626921)
}
}
{
\newrgbcolor{curcolor}{0 0 0}
\pscustom[linestyle=none,fillstyle=solid,fillcolor=curcolor]
{
\newpath
\moveto(188.1721109,394.28824288)
\lineto(186.57500092,394.28824288)
\lineto(186.57500092,397.36878952)
\curveto(186.57499619,398.02051276)(186.54089423,398.44110358)(186.47269494,398.63056324)
\curveto(186.4044864,398.82380333)(186.29270776,398.97347304)(186.13735868,399.07957281)
\curveto(185.98579013,399.185663)(185.80201847,399.23871049)(185.58604314,399.23871544)
\curveto(185.30943463,399.23871049)(185.06124815,399.16292836)(184.84148297,399.01136883)
\curveto(184.6217118,398.85979984)(184.47014754,398.6589772)(184.38678974,398.40890029)
\curveto(184.30721596,398.15881514)(184.26743035,397.69654415)(184.26743277,397.02208594)
\lineto(184.26743277,394.28824288)
\lineto(182.67032279,394.28824288)
\lineto(182.67032279,400.32429553)
\lineto(184.15375946,400.32429553)
\lineto(184.15375946,399.43764373)
\curveto(184.68044295,400.11967775)(185.34353658,400.46069733)(186.14304235,400.4607035)
\curveto(186.49542495,400.46069733)(186.817499,400.39628252)(187.10926547,400.26745888)
\curveto(187.4010214,400.14241239)(187.62078958,399.98137536)(187.76857065,399.78434732)
\curveto(187.92012899,399.58730829)(188.02432942,399.363751)(188.08117225,399.1136748)
\curveto(188.14179172,398.86358895)(188.17210457,398.50551839)(188.1721109,398.03946204)
\lineto(188.1721109,394.28824288)
}
}
{
\newrgbcolor{curcolor}{0 0 0}
\pscustom[linestyle=none,fillstyle=solid,fillcolor=curcolor]
{
\newpath
\moveto(189.2349555,396.01039349)
\lineto(190.83774914,396.25479111)
\curveto(190.90595118,395.94408241)(191.04425356,395.70726325)(191.25265671,395.54433293)
\curveto(191.46105528,395.3851892)(191.75281648,395.30561797)(192.12794118,395.30561899)
\curveto(192.54095062,395.30561797)(192.85165735,395.3814001)(193.06006231,395.5329656)
\curveto(193.20025515,395.63905934)(193.27035362,395.78115083)(193.27035793,395.95924051)
\curveto(193.27035362,396.08049024)(193.23246256,396.18090156)(193.15668462,396.26047477)
\curveto(193.07710919,396.33625493)(192.89902119,396.4063534)(192.62242007,396.47077039)
\curveto(191.33412021,396.7549512)(190.51756776,397.01450499)(190.17276029,397.24943255)
\curveto(189.69533166,397.57529275)(189.45661795,398.02809097)(189.45661845,398.60782858)
\curveto(189.45661795,399.13072096)(189.66312425,399.57025731)(190.07613797,399.92643895)
\curveto(190.48914947,400.28260932)(191.12950846,400.46069733)(191.99721688,400.4607035)
\curveto(192.82323906,400.46069733)(193.4370743,400.32618405)(193.83872447,400.05716326)
\curveto(194.24036488,399.78813093)(194.51696965,399.39027475)(194.66853962,398.86359352)
\lineto(193.16236829,398.58509392)
\curveto(193.09794928,398.82001422)(192.97480331,398.99999678)(192.79293003,399.12504213)
\curveto(192.6148382,399.25007781)(192.35907351,399.31259807)(192.02563521,399.31260309)
\curveto(191.60504132,399.31259807)(191.30380736,399.25386692)(191.12193241,399.13640946)
\curveto(191.00067884,399.05304427)(190.94005314,398.94505474)(190.94005511,398.81244054)
\curveto(190.94005314,398.69876282)(190.99310063,398.6021406)(191.09919775,398.5225736)
\curveto(191.24318165,398.41647439)(191.7395546,398.26680468)(192.58831808,398.07356403)
\curveto(193.44086341,397.88031582)(194.03575313,397.64349666)(194.37298902,397.36310586)
\curveto(194.70642497,397.0789198)(194.87314566,396.68295817)(194.87315157,396.17521979)
\curveto(194.87314566,395.62200836)(194.64201016,395.1464755)(194.17974439,394.74861978)
\curveto(193.71746818,394.35076314)(193.03353447,394.15183505)(192.12794118,394.15183491)
\curveto(191.30570191,394.15183505)(190.6539756,394.31855573)(190.17276029,394.65199746)
\curveto(189.69533166,394.98543847)(189.38273038,395.43823669)(189.2349555,396.01039349)
}
}
{
\newrgbcolor{curcolor}{0.50196081 0 1}
\pscustom[linewidth=2.32802935,linecolor=curcolor]
{
\newpath
\moveto(352.515295,478.6721358)
\lineto(428.17625139,478.6721358)
\lineto(428.17625139,525.2327179)
\lineto(378.12362028,525.2327179)
\lineto(378.12362028,534.54483618)
\lineto(352.515295,534.54483618)
\lineto(352.515295,478.6721358)
\closepath
}
}
{
\newrgbcolor{curcolor}{0 0 0}
\pscustom[linestyle=none,fillstyle=solid,fillcolor=curcolor]
{
\newpath
\moveto(358.97192877,506.18482371)
\lineto(358.97192877,507.1396795)
\lineto(364.49076787,509.46998232)
\lineto(364.49076787,508.45260621)
\lineto(360.11434551,506.65656794)
\lineto(364.49076787,504.84347868)
\lineto(364.49076787,503.82610257)
\lineto(358.97192877,506.18482371)
}
}
{
\newrgbcolor{curcolor}{0 0 0}
\pscustom[linestyle=none,fillstyle=solid,fillcolor=curcolor]
{
\newpath
\moveto(365.76959276,506.18482371)
\lineto(365.76959276,507.1396795)
\lineto(371.28843186,509.46998232)
\lineto(371.28843186,508.45260621)
\lineto(366.91200951,506.65656794)
\lineto(371.28843186,504.84347868)
\lineto(371.28843186,503.82610257)
\lineto(365.76959276,506.18482371)
}
}
{
\newrgbcolor{curcolor}{0 0 0}
\pscustom[linestyle=none,fillstyle=solid,fillcolor=curcolor]
{
\newpath
\moveto(372.69798106,502.54159419)
\lineto(372.69798106,508.57764685)
\lineto(373.61305119,508.57764685)
\lineto(373.61305119,507.7307807)
\curveto(373.80250483,508.02632582)(374.05448041,508.26314497)(374.36897869,508.44123888)
\curveto(374.68347209,508.62311009)(375.04154265,508.71404864)(375.44319145,508.71405482)
\curveto(375.8903025,508.71404864)(376.25595127,508.62121554)(376.54013887,508.43555521)
\curveto(376.82810635,508.2498831)(377.03082355,507.99032931)(377.14829107,507.65689305)
\curveto(377.62571326,508.36166174)(378.24712673,508.71404864)(379.01253332,508.71405482)
\curveto(379.61120506,508.71404864)(380.0715815,508.54732796)(380.39366401,508.21389226)
\curveto(380.7157296,507.88423433)(380.87676662,507.37459951)(380.87677557,506.68498627)
\lineto(380.87677557,502.54159419)
\lineto(379.85939946,502.54159419)
\lineto(379.85939946,506.34396635)
\curveto(379.85939153,506.75318604)(379.82528957,507.04684179)(379.75709348,507.22493448)
\curveto(379.69267085,507.40680691)(379.57331399,507.55268751)(379.39902256,507.66257672)
\curveto(379.2247162,507.77245569)(379.02010445,507.82739773)(378.7851867,507.82740301)
\curveto(378.36079992,507.82739773)(378.00841302,507.68530624)(377.72802494,507.40112811)
\curveto(377.44762526,507.12072937)(377.30742832,506.6698257)(377.3074337,506.04841575)
\lineto(377.3074337,502.54159419)
\lineto(376.28437393,502.54159419)
\lineto(376.28437393,506.46332332)
\curveto(376.28436957,506.91801217)(376.20100923,507.25903176)(376.03429265,507.48638309)
\curveto(375.86756786,507.71372453)(375.59475219,507.82739773)(375.21584483,507.82740301)
\curveto(374.92786945,507.82739773)(374.66073745,507.7516156)(374.41444801,507.6000564)
\curveto(374.17194271,507.44848708)(373.99574926,507.22682435)(373.88586713,506.93506755)
\curveto(373.77598108,506.64330196)(373.72103904,506.22271114)(373.72104083,505.67329383)
\lineto(373.72104083,502.54159419)
\lineto(372.69798106,502.54159419)
}
}
{
\newrgbcolor{curcolor}{0 0 0}
\pscustom[linestyle=none,fillstyle=solid,fillcolor=curcolor]
{
\newpath
\moveto(382.43410059,509.69732893)
\lineto(382.43410059,510.87384767)
\lineto(383.45716037,510.87384767)
\lineto(383.45716037,509.69732893)
\lineto(382.43410059,509.69732893)
\moveto(382.43410059,502.54159419)
\lineto(382.43410059,508.57764685)
\lineto(383.45716037,508.57764685)
\lineto(383.45716037,502.54159419)
\lineto(382.43410059,502.54159419)
}
}
{
\newrgbcolor{curcolor}{0 0 0}
\pscustom[linestyle=none,fillstyle=solid,fillcolor=curcolor]
{
\newpath
\moveto(388.93621215,502.54159419)
\lineto(388.93621215,503.30320536)
\curveto(388.55350771,502.70452577)(387.9908254,502.40518636)(387.24816353,502.40518622)
\curveto(386.76694401,502.40518636)(386.32361855,502.53780509)(385.91818582,502.8030428)
\curveto(385.51653887,503.06827999)(385.20393759,503.43771788)(384.98038103,503.91135756)
\curveto(384.76061213,504.3887836)(384.65072804,504.93630949)(384.65072844,505.55393686)
\curveto(384.65072804,506.15640177)(384.75113936,506.70203311)(384.95196271,507.19083249)
\curveto(385.15278465,507.68341168)(385.45401862,508.06042778)(385.85566551,508.3218819)
\curveto(386.25730919,508.58332447)(386.70631831,508.71404864)(387.20269421,508.71405482)
\curveto(387.56644548,508.71404864)(387.89041408,508.63637196)(388.17460099,508.48102454)
\curveto(388.45878005,508.32945434)(388.68991555,508.13052625)(388.86800817,507.88423967)
\lineto(388.86800817,510.87384767)
\lineto(389.88538427,510.87384767)
\lineto(389.88538427,502.54159419)
\lineto(388.93621215,502.54159419)
\moveto(385.70220654,505.55393686)
\curveto(385.70220509,504.78095612)(385.86513667,504.20311738)(386.19100176,503.82041891)
\curveto(386.51686298,503.43771788)(386.90145729,503.246368)(387.34478584,503.2463687)
\curveto(387.79189731,503.246368)(388.17080796,503.42824511)(388.48151892,503.79200058)
\curveto(388.79601053,504.15954266)(388.95325845,504.71843587)(388.95326315,505.46868187)
\curveto(388.95325845,506.29470416)(388.79411598,506.9009612)(388.47583525,507.2874548)
\curveto(388.15754609,507.67393892)(387.76537357,507.86718335)(387.29931652,507.86718867)
\curveto(386.84462069,507.86718335)(386.46381549,507.68151713)(386.15689977,507.31018946)
\curveto(385.85376935,506.93885226)(385.70220509,506.35343531)(385.70220654,505.55393686)
}
}
{
\newrgbcolor{curcolor}{0 0 0}
\pscustom[linestyle=none,fillstyle=solid,fillcolor=curcolor]
{
\newpath
\moveto(395.415591,502.54159419)
\lineto(395.415591,503.30320536)
\curveto(395.03288656,502.70452577)(394.47020425,502.40518636)(393.72754238,502.40518622)
\curveto(393.24632286,502.40518636)(392.8029974,502.53780509)(392.39756467,502.8030428)
\curveto(391.99591772,503.06827999)(391.68331644,503.43771788)(391.45975988,503.91135756)
\curveto(391.23999098,504.3887836)(391.13010689,504.93630949)(391.13010729,505.55393686)
\curveto(391.13010689,506.15640177)(391.23051821,506.70203311)(391.43134156,507.19083249)
\curveto(391.6321635,507.68341168)(391.93339747,508.06042778)(392.33504435,508.3218819)
\curveto(392.73668804,508.58332447)(393.18569716,508.71404864)(393.68207305,508.71405482)
\curveto(394.04582433,508.71404864)(394.36979293,508.63637196)(394.65397984,508.48102454)
\curveto(394.9381589,508.32945434)(395.1692944,508.13052625)(395.34738702,507.88423967)
\lineto(395.34738702,510.87384767)
\lineto(396.36476312,510.87384767)
\lineto(396.36476312,502.54159419)
\lineto(395.415591,502.54159419)
\moveto(392.18158539,505.55393686)
\curveto(392.18158394,504.78095612)(392.34451552,504.20311738)(392.67038061,503.82041891)
\curveto(392.99624183,503.43771788)(393.38083614,503.246368)(393.82416469,503.2463687)
\curveto(394.27127616,503.246368)(394.65018681,503.42824511)(394.96089777,503.79200058)
\curveto(395.27538938,504.15954266)(395.4326373,504.71843587)(395.432642,505.46868187)
\curveto(395.4326373,506.29470416)(395.27349482,506.9009612)(394.9552141,507.2874548)
\curveto(394.63692494,507.67393892)(394.24475242,507.86718335)(393.77869537,507.86718867)
\curveto(393.32399954,507.86718335)(392.94319434,507.68151713)(392.63627862,507.31018946)
\curveto(392.3331482,506.93885226)(392.18158394,506.35343531)(392.18158539,505.55393686)
}
}
{
\newrgbcolor{curcolor}{0 0 0}
\pscustom[linestyle=none,fillstyle=solid,fillcolor=curcolor]
{
\newpath
\moveto(397.95618795,502.54159419)
\lineto(397.95618795,510.87384767)
\lineto(398.97924772,510.87384767)
\lineto(398.97924772,502.54159419)
\lineto(397.95618795,502.54159419)
}
}
{
\newrgbcolor{curcolor}{0 0 0}
\pscustom[linestyle=none,fillstyle=solid,fillcolor=curcolor]
{
\newpath
\moveto(404.7026989,504.48540776)
\lineto(405.75986066,504.35468346)
\curveto(405.59313402,503.73705729)(405.28432184,503.25773532)(404.8334232,502.91671611)
\curveto(404.3825145,502.57569615)(403.80657032,502.40518636)(403.10558892,502.40518622)
\curveto(402.22272381,502.40518636)(401.52173911,502.67610747)(401.00263272,503.21795038)
\curveto(400.48731304,503.76358103)(400.2296538,504.52708599)(400.22965423,505.50846753)
\curveto(400.2296538,506.5239451)(400.49110215,507.31207925)(401.01400005,507.87287234)
\curveto(401.53689554,508.43365477)(402.2151456,508.71404864)(403.04875227,508.71405482)
\curveto(403.8558287,508.71404864)(404.51513323,508.43933843)(405.02666783,507.88992333)
\curveto(405.53819198,507.34049755)(405.79395666,506.56751983)(405.79396265,505.57098785)
\curveto(405.79395666,505.51035912)(405.79206211,505.41942056)(405.78827899,505.29817191)
\lineto(401.28681599,505.29817191)
\curveto(401.32470557,504.63507552)(401.51226635,504.12733525)(401.84949887,503.77494959)
\curveto(402.1867273,503.42256145)(402.60731812,503.246368)(403.11127259,503.2463687)
\curveto(403.48639082,503.246368)(403.80657032,503.34488477)(404.07181204,503.5419193)
\curveto(404.33704522,503.73895184)(404.54734063,504.05344768)(404.7026989,504.48540776)
\moveto(401.34365265,506.13935439)
\lineto(404.71406623,506.13935439)
\curveto(404.66859204,506.64709106)(404.53976242,507.02789626)(404.32757698,507.28177114)
\curveto(404.0017093,507.67583347)(403.57922393,507.87286701)(403.0601196,507.87287234)
\curveto(402.59026714,507.87286701)(402.19430551,507.71561909)(401.87223353,507.40112811)
\curveto(401.55394652,507.08662741)(401.37775307,506.66603659)(401.34365265,506.13935439)
}
}
{
\newrgbcolor{curcolor}{0 0 0}
\pscustom[linestyle=none,fillstyle=solid,fillcolor=curcolor]
{
\newpath
\moveto(412.43816958,506.18482371)
\lineto(406.91933047,503.82610257)
\lineto(406.91933047,504.84347868)
\lineto(411.29006917,506.65656794)
\lineto(406.91933047,508.45260621)
\lineto(406.91933047,509.46998232)
\lineto(412.43816958,507.1396795)
\lineto(412.43816958,506.18482371)
}
}
{
\newrgbcolor{curcolor}{0 0 0}
\pscustom[linestyle=none,fillstyle=solid,fillcolor=curcolor]
{
\newpath
\moveto(419.23583224,506.18482371)
\lineto(413.71699314,503.82610257)
\lineto(413.71699314,504.84347868)
\lineto(418.08773183,506.65656794)
\lineto(413.71699314,508.45260621)
\lineto(413.71699314,509.46998232)
\lineto(419.23583224,507.1396795)
\lineto(419.23583224,506.18482371)
}
}
{
\newrgbcolor{curcolor}{0 0 0}
\pscustom[linestyle=none,fillstyle=solid,fillcolor=curcolor]
{
\newpath
\moveto(370.24833691,486.8034865)
\lineto(371.85113055,487.04788411)
\curveto(371.9193326,486.73717542)(372.05763498,486.50035626)(372.26603813,486.33742594)
\curveto(372.47443669,486.17828221)(372.76619789,486.09871097)(373.1413226,486.09871199)
\curveto(373.55433204,486.09871097)(373.86503877,486.1744931)(374.07344372,486.32605861)
\curveto(374.21363657,486.43215234)(374.28373504,486.57424384)(374.28373934,486.75233351)
\curveto(374.28373504,486.87358325)(374.24584397,486.97399457)(374.17006604,487.05356778)
\curveto(374.09049061,487.12934794)(373.9124026,487.19944641)(373.63580149,487.2638634)
\curveto(372.34750163,487.5480442)(371.53094918,487.80759799)(371.1861417,488.04252556)
\curveto(370.70871308,488.36838575)(370.46999937,488.82118398)(370.46999986,489.40092159)
\curveto(370.46999937,489.92381396)(370.67650567,490.36335031)(371.08951939,490.71953196)
\curveto(371.50253088,491.07570233)(372.14288988,491.25379033)(373.0105983,491.25379651)
\curveto(373.83662047,491.25379033)(374.45045572,491.11927706)(374.85210588,490.85025626)
\curveto(375.25374629,490.58122394)(375.53035107,490.18336776)(375.68192103,489.65668653)
\lineto(374.1757497,489.37818693)
\curveto(374.11133069,489.61310723)(373.98818473,489.79308979)(373.80631145,489.91813514)
\curveto(373.62821962,490.04317082)(373.37245493,490.10569107)(373.03901662,490.1056961)
\curveto(372.61842274,490.10569107)(372.31718878,490.04695992)(372.13531382,489.92950247)
\curveto(372.01406026,489.84613728)(371.95343455,489.73814774)(371.95343653,489.60553354)
\curveto(371.95343455,489.49185582)(372.00648204,489.39523361)(372.11257916,489.31566661)
\curveto(372.25656307,489.20956739)(372.75293602,489.05989769)(373.6016995,488.86665704)
\curveto(374.45424483,488.67340882)(375.04913454,488.43658967)(375.38637043,488.15619887)
\curveto(375.71980639,487.8720128)(375.88652708,487.47605118)(375.88653299,486.9683128)
\curveto(375.88652708,486.41510136)(375.65539158,485.9395685)(375.19312581,485.54171278)
\curveto(374.7308496,485.14385614)(374.04691588,484.94492805)(373.1413226,484.94492792)
\curveto(372.31908333,484.94492805)(371.66735701,485.11164874)(371.1861417,485.44509047)
\curveto(370.70871308,485.77853148)(370.39611179,486.2313297)(370.24833691,486.8034865)
}
}
{
\newrgbcolor{curcolor}{0 0 0}
\pscustom[linestyle=none,fillstyle=solid,fillcolor=curcolor]
{
\newpath
\moveto(377.24492887,491.11738854)
\lineto(378.7340492,491.11738854)
\lineto(378.7340492,490.23073674)
\curveto(378.92729135,490.5338601)(379.1887397,490.78015203)(379.51839503,490.96961324)
\curveto(379.84804423,491.15906267)(380.213693,491.25379033)(380.61534245,491.25379651)
\curveto(381.31632299,491.25379033)(381.9112127,490.97908012)(382.40001338,490.42966502)
\curveto(382.88880217,489.88023924)(383.13319954,489.11483973)(383.13320622,488.1334642)
\curveto(383.13319954,487.12555883)(382.88690762,486.34121379)(382.39432972,485.78042673)
\curveto(381.90173994,485.22342738)(381.30495567,484.94492805)(380.60397512,484.94492792)
\curveto(380.2705296,484.94492805)(379.96740108,485.01123741)(379.69458865,485.1438562)
\curveto(379.42555886,485.27647487)(379.14137587,485.50382126)(378.84203884,485.82589605)
\lineto(378.84203884,482.78513506)
\lineto(377.24492887,482.78513506)
\lineto(377.24492887,491.11738854)
\moveto(378.82498785,488.20166819)
\curveto(378.82498548,487.52341501)(378.95949876,487.0213584)(379.22852809,486.69549686)
\curveto(379.49755188,486.37342119)(379.82530959,486.21238417)(380.21180221,486.2123853)
\curveto(380.58313088,486.21238417)(380.89194306,486.36015932)(381.13823967,486.6557112)
\curveto(381.3845269,486.95504904)(381.50767286,487.44384377)(381.50767792,488.12209687)
\curveto(381.50767286,488.75487461)(381.3807378,489.22472382)(381.12687234,489.53164589)
\curveto(380.87299753,489.83855907)(380.55850169,489.99201788)(380.18338388,489.99202279)
\curveto(379.79310218,489.99201788)(379.46913358,489.84045362)(379.2114771,489.53732956)
\curveto(378.9538151,489.23798569)(378.82498548,488.79276568)(378.82498785,488.20166819)
}
}
{
\newrgbcolor{curcolor}{0 0 0}
\pscustom[linestyle=none,fillstyle=solid,fillcolor=curcolor]
{
\newpath
\moveto(385.59991662,489.27588095)
\lineto(384.15058194,489.53732956)
\curveto(384.31351294,490.1208475)(384.59390682,490.55280564)(384.99176442,490.83320527)
\curveto(385.38961918,491.1135934)(385.98071979,491.25379033)(386.76506803,491.25379651)
\curveto(387.47741685,491.25379033)(388.00789176,491.16853544)(388.35649434,490.99803156)
\curveto(388.70508735,490.83130496)(388.94948471,490.61722045)(389.08968717,490.35577737)
\curveto(389.2336677,490.09811286)(389.30566072,489.62258)(389.30566646,488.92917736)
\lineto(389.28861546,487.06493511)
\curveto(389.28860974,486.53445822)(389.31323894,486.1422857)(389.36250311,485.88841637)
\curveto(389.41554481,485.63833454)(389.51216703,485.36930798)(389.65237005,485.08133588)
\lineto(388.07231107,485.08133588)
\curveto(388.03062639,485.18743087)(387.97947346,485.34467878)(387.9188521,485.55308011)
\curveto(387.89232401,485.6478073)(387.87337848,485.71032756)(387.86201545,485.74064107)
\curveto(387.58919549,485.47540296)(387.29743429,485.27647487)(386.98673098,485.1438562)
\curveto(386.67602083,485.01123741)(386.34447401,484.94492805)(385.99208953,484.94492792)
\curveto(385.37067365,484.94492805)(384.87998436,485.11354329)(384.52002019,485.45077414)
\curveto(384.16384323,485.78800424)(383.98575523,486.21427872)(383.98575565,486.72959885)
\curveto(383.98575523,487.07061678)(384.06722102,487.3737453)(384.23015326,487.63898531)
\curveto(384.39308418,487.90800932)(384.62043057,488.11262107)(384.91219311,488.25282118)
\curveto(385.20774207,488.39680405)(385.63212199,488.52184457)(386.18533415,488.62794309)
\curveto(386.93178552,488.76813649)(387.44899855,488.89886066)(387.73697481,489.02011601)
\lineto(387.73697481,489.17925864)
\curveto(387.73697064,489.48617216)(387.66118851,489.70404579)(387.50962819,489.83288016)
\curveto(387.35805999,489.96549413)(387.07198246,490.0318035)(386.65139472,490.03180845)
\curveto(386.36720865,490.0318035)(386.14554592,489.9749669)(385.98640587,489.86129848)
\curveto(385.82726098,489.75140962)(385.69843136,489.55627063)(385.59991662,489.27588095)
\moveto(387.73697481,487.98000524)
\curveto(387.53235889,487.91179842)(387.20839029,487.83033263)(386.76506803,487.73560763)
\curveto(386.32173937,487.64087731)(386.03187273,487.5480442)(385.89546722,487.45710802)
\curveto(385.68706404,487.30933049)(385.58286361,487.12176972)(385.58286562,486.89442515)
\curveto(385.58286361,486.67086605)(385.66622395,486.47762162)(385.8329469,486.31469128)
\curveto(385.99966532,486.15175846)(386.21185529,486.07029268)(386.46951742,486.07029366)
\curveto(386.75748662,486.07029268)(387.03219684,486.16502034)(387.29364891,486.35447693)
\curveto(387.48688961,486.49846171)(387.61382468,486.67465516)(387.67445449,486.88305782)
\curveto(387.71613056,487.01946385)(387.73697064,487.27901764)(387.73697481,487.66171998)
\lineto(387.73697481,487.98000524)
}
}
{
\newrgbcolor{curcolor}{0 0 0}
\pscustom[linestyle=none,fillstyle=solid,fillcolor=curcolor]
{
\newpath
\moveto(396.14879989,489.3327176)
\lineto(394.57442458,489.04853433)
\curveto(394.52137256,489.3630262)(394.40012116,489.59984536)(394.21066999,489.75899251)
\curveto(394.02499961,489.9181303)(393.7824968,489.99770154)(393.48316082,489.99770645)
\curveto(393.08530121,489.99770154)(392.76701626,489.85939915)(392.52830503,489.58279888)
\curveto(392.29337795,489.30997871)(392.17591565,488.85149683)(392.17591778,488.20735185)
\curveto(392.17591565,487.4912076)(392.29527251,486.98536189)(392.5339887,486.68981319)
\curveto(392.77648903,486.39426128)(393.10045763,486.24648613)(393.50589548,486.24648729)
\curveto(393.80902054,486.24648613)(394.05720702,486.33174102)(394.25045565,486.50225223)
\curveto(394.44369588,486.67654971)(394.58010371,486.97399457)(394.65967956,487.3945877)
\lineto(396.22837121,487.12745543)
\curveto(396.06543345,486.40752315)(395.75283217,485.86378637)(395.29056642,485.49624346)
\curveto(394.82829019,485.12869972)(394.20877128,484.94492805)(393.43200783,484.94492792)
\curveto(392.54914264,484.94492805)(391.84436884,485.22342738)(391.3176843,485.78042673)
\curveto(390.79478634,486.33742468)(390.533338,487.10850785)(390.53333848,488.09367855)
\curveto(390.533338,489.09021054)(390.7966809,489.86508281)(391.32336797,490.41829769)
\curveto(391.8500525,490.97529101)(392.56240451,491.25379033)(393.46042616,491.25379651)
\curveto(394.19550941,491.25379033)(394.7790318,491.09464786)(395.2109951,490.77636861)
\curveto(395.64673719,490.46186708)(395.95933847,489.98065056)(396.14879989,489.3327176)
}
}
{
\newrgbcolor{curcolor}{0 0 0}
\pscustom[linestyle=none,fillstyle=solid,fillcolor=curcolor]
{
\newpath
\moveto(400.8605588,487.00241479)
\lineto(402.45198511,486.73528252)
\curveto(402.24736744,486.15175846)(401.92339884,485.70653845)(401.48007833,485.39962115)
\curveto(401.04053703,485.09649231)(400.48922204,484.94492805)(399.8261317,484.94492792)
\curveto(398.77654591,484.94492805)(397.99977908,485.28784219)(397.49582889,485.97367135)
\curveto(397.09797174,486.5230909)(396.89904365,487.21649738)(396.89904402,488.05389289)
\curveto(396.89904365,489.05421403)(397.160492,489.83666451)(397.68338985,490.4012467)
\curveto(398.20628539,490.96960735)(398.86748447,491.25379033)(399.66698907,491.25379651)
\curveto(400.56500417,491.25379033)(401.27356708,490.95634548)(401.79267993,490.36146104)
\curveto(402.31178225,489.77035515)(402.55996873,488.86286415)(402.5372401,487.63898531)
\lineto(398.53593966,487.63898531)
\curveto(398.54730497,487.16534445)(398.67613459,486.79590657)(398.9224289,486.53067056)
\curveto(399.16871843,486.26922077)(399.47563605,486.13849659)(399.8431827,486.13849765)
\curveto(400.09326041,486.13849659)(400.30355582,486.20670051)(400.47406956,486.3431096)
\curveto(400.6445754,486.47951617)(400.77340502,486.69928435)(400.8605588,487.00241479)
\moveto(400.95149745,488.61657576)
\curveto(400.94012571,489.07884322)(400.82076885,489.42933557)(400.59342653,489.66805386)
\curveto(400.36607608,489.91055209)(400.0894713,490.0318035)(399.76361138,490.03180845)
\curveto(399.41501035,490.0318035)(399.12703826,489.90486843)(398.89969424,489.65100287)
\curveto(398.67234548,489.39712816)(398.56056684,489.05231947)(398.56435798,488.61657576)
\lineto(400.95149745,488.61657576)
}
}
{
\newrgbcolor{curcolor}{0 0 0}
\pscustom[linestyle=none,fillstyle=solid,fillcolor=curcolor]
{
\newpath
\moveto(403.28180056,486.8034865)
\lineto(404.8845942,487.04788411)
\curveto(404.95279624,486.73717542)(405.09109863,486.50035626)(405.29950177,486.33742594)
\curveto(405.50790034,486.17828221)(405.79966154,486.09871097)(406.17478625,486.09871199)
\curveto(406.58779569,486.09871097)(406.89850242,486.1744931)(407.10690737,486.32605861)
\curveto(407.24710021,486.43215234)(407.31719868,486.57424384)(407.31720299,486.75233351)
\curveto(407.31719868,486.87358325)(407.27930762,486.97399457)(407.20352968,487.05356778)
\curveto(407.12395425,487.12934794)(406.94586625,487.19944641)(406.66926514,487.2638634)
\curveto(405.38096527,487.5480442)(404.56441283,487.80759799)(404.21960535,488.04252556)
\curveto(403.74217672,488.36838575)(403.50346301,488.82118398)(403.50346351,489.40092159)
\curveto(403.50346301,489.92381396)(403.70996932,490.36335031)(404.12298304,490.71953196)
\curveto(404.53599453,491.07570233)(405.17635352,491.25379033)(406.04406194,491.25379651)
\curveto(406.87008412,491.25379033)(407.48391937,491.11927706)(407.88556953,490.85025626)
\curveto(408.28720994,490.58122394)(408.56381471,490.18336776)(408.71538468,489.65668653)
\lineto(407.20921335,489.37818693)
\curveto(407.14479434,489.61310723)(407.02164838,489.79308979)(406.8397751,489.91813514)
\curveto(406.66168326,490.04317082)(406.40591857,490.10569107)(406.07248027,490.1056961)
\curveto(405.65188639,490.10569107)(405.35065242,490.04695992)(405.16877747,489.92950247)
\curveto(405.0475239,489.84613728)(404.9868982,489.73814774)(404.98690018,489.60553354)
\curveto(404.9868982,489.49185582)(405.03994569,489.39523361)(405.14604281,489.31566661)
\curveto(405.29002672,489.20956739)(405.78639967,489.05989769)(406.63516314,488.86665704)
\curveto(407.48770847,488.67340882)(408.08259819,488.43658967)(408.41983408,488.15619887)
\curveto(408.75327004,487.8720128)(408.91999072,487.47605118)(408.91999663,486.9683128)
\curveto(408.91999072,486.41510136)(408.68885523,485.9395685)(408.22658945,485.54171278)
\curveto(407.76431325,485.14385614)(407.08037953,484.94492805)(406.17478625,484.94492792)
\curveto(405.35254697,484.94492805)(404.70082066,485.11164874)(404.21960535,485.44509047)
\curveto(403.74217672,485.77853148)(403.42957544,486.2313297)(403.28180056,486.8034865)
}
}
{
\newrgbcolor{curcolor}{0.50196081 0 1}
\pscustom[linewidth=2.32802935,linecolor=curcolor]
{
\newpath
\moveto(240.76988492,386.71497242)
\lineto(316.43084131,386.71497242)
\lineto(316.43084131,433.27556384)
\lineto(266.3782102,433.27556384)
\lineto(266.3782102,442.58768212)
\lineto(240.76988492,442.58768212)
\lineto(240.76988492,386.71497242)
\closepath
}
}
{
\newrgbcolor{curcolor}{0 0 0}
\pscustom[linestyle=none,fillstyle=solid,fillcolor=curcolor]
{
\newpath
\moveto(247.22654854,415.39170228)
\lineto(247.22654854,416.34655806)
\lineto(252.74538764,418.67686088)
\lineto(252.74538764,417.65948477)
\lineto(248.36896528,415.8634465)
\lineto(252.74538764,414.05035724)
\lineto(252.74538764,413.03298114)
\lineto(247.22654854,415.39170228)
}
}
{
\newrgbcolor{curcolor}{0 0 0}
\pscustom[linestyle=none,fillstyle=solid,fillcolor=curcolor]
{
\newpath
\moveto(254.02421253,415.39170228)
\lineto(254.02421253,416.34655806)
\lineto(259.54305163,418.67686088)
\lineto(259.54305163,417.65948477)
\lineto(255.16662928,415.8634465)
\lineto(259.54305163,414.05035724)
\lineto(259.54305163,413.03298114)
\lineto(254.02421253,415.39170228)
}
}
{
\newrgbcolor{curcolor}{0 0 0}
\pscustom[linestyle=none,fillstyle=solid,fillcolor=curcolor]
{
\newpath
\moveto(260.95260083,411.74847276)
\lineto(260.95260083,417.78452541)
\lineto(261.86767096,417.78452541)
\lineto(261.86767096,416.93765927)
\curveto(262.0571246,417.23320438)(262.30910018,417.47002354)(262.62359845,417.64811744)
\curveto(262.93809185,417.82998865)(263.29616242,417.92092721)(263.69781122,417.92093338)
\curveto(264.14492227,417.92092721)(264.51057104,417.8280941)(264.79475864,417.64243377)
\curveto(265.08272612,417.45676166)(265.28544332,417.19720787)(265.40291083,416.86377161)
\curveto(265.88033303,417.5685403)(266.5017465,417.92092721)(267.26715309,417.92093338)
\curveto(267.86582483,417.92092721)(268.32620126,417.75420652)(268.64828378,417.42077082)
\curveto(268.97034936,417.09111289)(269.13138639,416.58147807)(269.13139534,415.89186483)
\lineto(269.13139534,411.74847276)
\lineto(268.11401923,411.74847276)
\lineto(268.11401923,415.55084491)
\curveto(268.1140113,415.9600646)(268.07990934,416.25372036)(268.01171325,416.43181304)
\curveto(267.94729062,416.61368547)(267.82793376,416.75956607)(267.65364233,416.86945528)
\curveto(267.47933597,416.97933425)(267.27472422,417.03427629)(267.03980647,417.03428158)
\curveto(266.61541969,417.03427629)(266.26303279,416.8921848)(265.98264471,416.60800667)
\curveto(265.70224503,416.32760793)(265.56204809,415.87670426)(265.56205347,415.25529431)
\lineto(265.56205347,411.74847276)
\lineto(264.53899369,411.74847276)
\lineto(264.53899369,415.67020188)
\curveto(264.53898934,416.12489074)(264.455629,416.46591032)(264.28891242,416.69326165)
\curveto(264.12218763,416.9206031)(263.84937196,417.03427629)(263.4704646,417.03428158)
\curveto(263.18248922,417.03427629)(262.91535722,416.95849416)(262.66906778,416.80693496)
\curveto(262.42656248,416.65536564)(262.25036903,416.43370291)(262.1404869,416.14194611)
\curveto(262.03060085,415.85018052)(261.97565881,415.4295897)(261.9756606,414.88017239)
\lineto(261.9756606,411.74847276)
\lineto(260.95260083,411.74847276)
}
}
{
\newrgbcolor{curcolor}{0 0 0}
\pscustom[linestyle=none,fillstyle=solid,fillcolor=curcolor]
{
\newpath
\moveto(270.68872036,418.90420749)
\lineto(270.68872036,420.08072623)
\lineto(271.71178013,420.08072623)
\lineto(271.71178013,418.90420749)
\lineto(270.68872036,418.90420749)
\moveto(270.68872036,411.74847276)
\lineto(270.68872036,417.78452541)
\lineto(271.71178013,417.78452541)
\lineto(271.71178013,411.74847276)
\lineto(270.68872036,411.74847276)
}
}
{
\newrgbcolor{curcolor}{0 0 0}
\pscustom[linestyle=none,fillstyle=solid,fillcolor=curcolor]
{
\newpath
\moveto(277.19083192,411.74847276)
\lineto(277.19083192,412.51008392)
\curveto(276.80812748,411.91140433)(276.24544517,411.61206492)(275.5027833,411.61206479)
\curveto(275.02156378,411.61206492)(274.57823832,411.74468365)(274.17280559,412.00992136)
\curveto(273.77115864,412.27515856)(273.45855736,412.64459644)(273.2350008,413.11823612)
\curveto(273.0152319,413.59566216)(272.90534781,414.14318805)(272.90534821,414.76081542)
\curveto(272.90534781,415.36328033)(273.00575913,415.90891167)(273.20658248,416.39771105)
\curveto(273.40740442,416.89029024)(273.70863838,417.26730634)(274.11028527,417.52876047)
\curveto(274.51192896,417.79020303)(274.96093808,417.92092721)(275.45731397,417.92093338)
\curveto(275.82106525,417.92092721)(276.14503385,417.84325052)(276.42922076,417.6879031)
\curveto(276.71339982,417.5363329)(276.94453532,417.33740481)(277.12262794,417.09111823)
\lineto(277.12262794,420.08072623)
\lineto(278.14000404,420.08072623)
\lineto(278.14000404,411.74847276)
\lineto(277.19083192,411.74847276)
\moveto(273.95682631,414.76081542)
\curveto(273.95682486,413.98783468)(274.11975644,413.40999595)(274.44562153,413.02729747)
\curveto(274.77148275,412.64459644)(275.15607706,412.45324656)(275.59940561,412.45324727)
\curveto(276.04651708,412.45324656)(276.42542773,412.63512367)(276.73613869,412.99887914)
\curveto(277.0506303,413.36642122)(277.20787822,413.92531443)(277.20788292,414.67556044)
\curveto(277.20787822,415.50158272)(277.04873574,416.10783976)(276.73045502,416.49433336)
\curveto(276.41216586,416.88081748)(276.01999334,417.07406191)(275.55393629,417.07406723)
\curveto(275.09924046,417.07406191)(274.71843526,416.88839569)(274.41151954,416.51706803)
\curveto(274.10838912,416.14573082)(273.95682486,415.56031387)(273.95682631,414.76081542)
}
}
{
\newrgbcolor{curcolor}{0 0 0}
\pscustom[linestyle=none,fillstyle=solid,fillcolor=curcolor]
{
\newpath
\moveto(283.67021077,411.74847276)
\lineto(283.67021077,412.51008392)
\curveto(283.28750633,411.91140433)(282.72482402,411.61206492)(281.98216215,411.61206479)
\curveto(281.50094263,411.61206492)(281.05761717,411.74468365)(280.65218444,412.00992136)
\curveto(280.25053749,412.27515856)(279.93793621,412.64459644)(279.71437965,413.11823612)
\curveto(279.49461075,413.59566216)(279.38472666,414.14318805)(279.38472706,414.76081542)
\curveto(279.38472666,415.36328033)(279.48513798,415.90891167)(279.68596132,416.39771105)
\curveto(279.88678327,416.89029024)(280.18801723,417.26730634)(280.58966412,417.52876047)
\curveto(280.99130781,417.79020303)(281.44031692,417.92092721)(281.93669282,417.92093338)
\curveto(282.30044409,417.92092721)(282.6244127,417.84325052)(282.90859961,417.6879031)
\curveto(283.19277867,417.5363329)(283.42391416,417.33740481)(283.60200678,417.09111823)
\lineto(283.60200678,420.08072623)
\lineto(284.61938289,420.08072623)
\lineto(284.61938289,411.74847276)
\lineto(283.67021077,411.74847276)
\moveto(280.43620516,414.76081542)
\curveto(280.43620371,413.98783468)(280.59913529,413.40999595)(280.92500038,413.02729747)
\curveto(281.2508616,412.64459644)(281.63545591,412.45324656)(282.07878446,412.45324727)
\curveto(282.52589593,412.45324656)(282.90480658,412.63512367)(283.21551754,412.99887914)
\curveto(283.53000915,413.36642122)(283.68725706,413.92531443)(283.68726177,414.67556044)
\curveto(283.68725706,415.50158272)(283.52811459,416.10783976)(283.20983387,416.49433336)
\curveto(282.8915447,416.88081748)(282.49937218,417.07406191)(282.03331513,417.07406723)
\curveto(281.57861931,417.07406191)(281.19781411,416.88839569)(280.89089839,416.51706803)
\curveto(280.58776797,416.14573082)(280.43620371,415.56031387)(280.43620516,414.76081542)
}
}
{
\newrgbcolor{curcolor}{0 0 0}
\pscustom[linestyle=none,fillstyle=solid,fillcolor=curcolor]
{
\newpath
\moveto(286.21080772,411.74847276)
\lineto(286.21080772,420.08072623)
\lineto(287.23386749,420.08072623)
\lineto(287.23386749,411.74847276)
\lineto(286.21080772,411.74847276)
}
}
{
\newrgbcolor{curcolor}{0 0 0}
\pscustom[linestyle=none,fillstyle=solid,fillcolor=curcolor]
{
\newpath
\moveto(292.95731867,413.69228632)
\lineto(294.01448043,413.56156202)
\curveto(293.84775379,412.94393585)(293.53894161,412.46461388)(293.08804297,412.12359467)
\curveto(292.63713427,411.78257471)(292.06119009,411.61206492)(291.36020869,411.61206479)
\curveto(290.47734358,411.61206492)(289.77635888,411.88298604)(289.25725249,412.42482894)
\curveto(288.74193281,412.97045959)(288.48427357,413.73396455)(288.484274,414.71534609)
\curveto(288.48427357,415.73082366)(288.74572192,416.51895781)(289.26861982,417.0797509)
\curveto(289.79151531,417.64053333)(290.46976536,417.92092721)(291.30337204,417.92093338)
\curveto(292.11044847,417.92092721)(292.769753,417.64621699)(293.28128759,417.0968019)
\curveto(293.79281174,416.54737611)(294.04857643,415.77439839)(294.04858242,414.77786641)
\curveto(294.04857643,414.71723768)(294.04668188,414.62629913)(294.04289876,414.50505047)
\lineto(289.54143576,414.50505047)
\curveto(289.57932534,413.84195408)(289.76688611,413.33421382)(290.10411864,412.98182815)
\curveto(290.44134707,412.62944001)(290.86193789,412.45324656)(291.36589235,412.45324727)
\curveto(291.74101059,412.45324656)(292.06119009,412.55176333)(292.32643181,412.74879787)
\curveto(292.59166499,412.9458304)(292.8019604,413.26032624)(292.95731867,413.69228632)
\moveto(289.59827242,415.34623295)
\lineto(292.968686,415.34623295)
\curveto(292.92321181,415.85396962)(292.79438219,416.23477482)(292.58219675,416.4886497)
\curveto(292.25632907,416.88271203)(291.8338437,417.07974557)(291.31473937,417.0797509)
\curveto(290.84488691,417.07974557)(290.44892528,416.92249765)(290.1268533,416.60800667)
\curveto(289.80856628,416.29350597)(289.63237283,415.87291516)(289.59827242,415.34623295)
}
}
{
\newrgbcolor{curcolor}{0 0 0}
\pscustom[linestyle=none,fillstyle=solid,fillcolor=curcolor]
{
\newpath
\moveto(300.69278935,415.39170228)
\lineto(295.17395024,413.03298114)
\lineto(295.17395024,414.05035724)
\lineto(299.54468893,415.8634465)
\lineto(295.17395024,417.65948477)
\lineto(295.17395024,418.67686088)
\lineto(300.69278935,416.34655806)
\lineto(300.69278935,415.39170228)
}
}
{
\newrgbcolor{curcolor}{0 0 0}
\pscustom[linestyle=none,fillstyle=solid,fillcolor=curcolor]
{
\newpath
\moveto(307.49045201,415.39170228)
\lineto(301.9716129,413.03298114)
\lineto(301.9716129,414.05035724)
\lineto(306.3423516,415.8634465)
\lineto(301.9716129,417.65948477)
\lineto(301.9716129,418.67686088)
\lineto(307.49045201,416.34655806)
\lineto(307.49045201,415.39170228)
}
}
{
\newrgbcolor{curcolor}{0 0 0}
\pscustom[linestyle=none,fillstyle=solid,fillcolor=curcolor]
{
\newpath
\moveto(252.44641173,394.28824288)
\lineto(250.84930176,394.28824288)
\lineto(250.84930176,400.32429553)
\lineto(252.33273843,400.32429553)
\lineto(252.33273843,399.46606206)
\curveto(252.58660631,399.87149127)(252.8139527,400.13862328)(253.01477827,400.26745888)
\curveto(253.21938709,400.39628252)(253.45052259,400.46069733)(253.70818545,400.4607035)
\curveto(254.07193605,400.46069733)(254.4224284,400.36028601)(254.75966355,400.15946924)
\lineto(254.26518466,398.76697121)
\curveto(253.99615392,398.94126563)(253.74607289,399.02841508)(253.51494083,399.02841982)
\curveto(253.29138011,399.02841508)(253.10192479,398.96589482)(252.94657429,398.84085886)
\curveto(252.79121806,398.7196029)(252.6680721,398.49794017)(252.57713604,398.17587001)
\curveto(252.48998409,397.85379207)(252.44640937,397.17933112)(252.44641173,396.15248513)
\lineto(252.44641173,394.28824288)
}
}
{
\newrgbcolor{curcolor}{0 0 0}
\pscustom[linestyle=none,fillstyle=solid,fillcolor=curcolor]
{
\newpath
\moveto(258.93715756,396.20932178)
\lineto(260.52858387,395.94218951)
\curveto(260.3239662,395.35866546)(259.9999976,394.91344545)(259.55667709,394.60652814)
\curveto(259.11713579,394.3033993)(258.56582079,394.15183505)(257.90273046,394.15183491)
\curveto(256.85314467,394.15183505)(256.07637784,394.49474918)(255.57242764,395.18057835)
\curveto(255.1745705,395.72999789)(254.97564241,396.42340438)(254.97564278,397.26079988)
\curveto(254.97564241,398.26112102)(255.23709075,399.04357151)(255.7599886,399.60815369)
\curveto(256.28288414,400.17651434)(256.94408322,400.46069733)(257.74358783,400.4607035)
\curveto(258.64160292,400.46069733)(259.35016583,400.16325247)(259.86927869,399.56836803)
\curveto(260.38838101,398.97726214)(260.63656748,398.06977114)(260.61383885,396.84589231)
\lineto(256.61253841,396.84589231)
\curveto(256.62390373,396.37225144)(256.75273335,396.00281356)(256.99902766,395.73757756)
\curveto(257.24531719,395.47612776)(257.55223481,395.34540359)(257.91978145,395.34540464)
\curveto(258.16985917,395.34540359)(258.38015458,395.4136075)(258.55066831,395.5500166)
\curveto(258.72117416,395.68642317)(258.85000378,395.90619134)(258.93715756,396.20932178)
\moveto(259.02809621,397.82348276)
\curveto(259.01672446,398.28575021)(258.89736761,398.63624256)(258.67002529,398.87496086)
\curveto(258.44267483,399.11745908)(258.16607006,399.23871049)(257.84021014,399.23871544)
\curveto(257.49160911,399.23871049)(257.20363702,399.11177542)(256.976293,398.85790986)
\curveto(256.74894424,398.60403516)(256.6371656,398.25922647)(256.64095674,397.82348276)
\lineto(259.02809621,397.82348276)
}
}
{
\newrgbcolor{curcolor}{0 0 0}
\pscustom[linestyle=none,fillstyle=solid,fillcolor=curcolor]
{
\newpath
\moveto(261.35839843,396.01039349)
\lineto(262.96119207,396.25479111)
\curveto(263.02939411,395.94408241)(263.1676965,395.70726325)(263.37609964,395.54433293)
\curveto(263.58449821,395.3851892)(263.87625941,395.30561797)(264.25138411,395.30561899)
\curveto(264.66439355,395.30561797)(264.97510029,395.3814001)(265.18350524,395.5329656)
\curveto(265.32369808,395.63905934)(265.39379655,395.78115083)(265.39380086,395.95924051)
\curveto(265.39379655,396.08049024)(265.35590549,396.18090156)(265.28012755,396.26047477)
\curveto(265.20055212,396.33625493)(265.02246412,396.4063534)(264.745863,396.47077039)
\curveto(263.45756314,396.7549512)(262.6410107,397.01450499)(262.29620322,397.24943255)
\curveto(261.81877459,397.57529275)(261.58006088,398.02809097)(261.58006138,398.60782858)
\curveto(261.58006088,399.13072096)(261.78656719,399.57025731)(262.19958091,399.92643895)
\curveto(262.6125924,400.28260932)(263.25295139,400.46069733)(264.12065981,400.4607035)
\curveto(264.94668199,400.46069733)(265.56051724,400.32618405)(265.9621674,400.05716326)
\curveto(266.36380781,399.78813093)(266.64041258,399.39027475)(266.79198255,398.86359352)
\lineto(265.28581122,398.58509392)
\curveto(265.22139221,398.82001422)(265.09824625,398.99999678)(264.91637297,399.12504213)
\curveto(264.73828113,399.25007781)(264.48251644,399.31259807)(264.14907814,399.31260309)
\curveto(263.72848425,399.31259807)(263.42725029,399.25386692)(263.24537534,399.13640946)
\curveto(263.12412177,399.05304427)(263.06349607,398.94505474)(263.06349805,398.81244054)
\curveto(263.06349607,398.69876282)(263.11654356,398.6021406)(263.22264068,398.5225736)
\curveto(263.36662459,398.41647439)(263.86299753,398.26680468)(264.71176101,398.07356403)
\curveto(265.56430634,397.88031582)(266.15919606,397.64349666)(266.49643195,397.36310586)
\curveto(266.82986791,397.0789198)(266.99658859,396.68295817)(266.9965945,396.17521979)
\curveto(266.99658859,395.62200836)(266.7654531,395.1464755)(266.30318732,394.74861978)
\curveto(265.84091111,394.35076314)(265.1569774,394.15183505)(264.25138411,394.15183491)
\curveto(263.42914484,394.15183505)(262.77741853,394.31855573)(262.29620322,394.65199746)
\curveto(261.81877459,394.98543847)(261.50617331,395.43823669)(261.35839843,396.01039349)
}
}
{
\newrgbcolor{curcolor}{0 0 0}
\pscustom[linestyle=none,fillstyle=solid,fillcolor=curcolor]
{
\newpath
\moveto(268.0310219,397.39152419)
\curveto(268.03102143,397.92199599)(268.16174561,398.43541992)(268.42319481,398.93179751)
\curveto(268.6846423,399.42816581)(269.05408018,399.80707646)(269.53150956,400.06853059)
\curveto(270.01272412,400.32997316)(270.54888269,400.46069733)(271.13998687,400.4607035)
\curveto(272.05315796,400.46069733)(272.80150649,400.16325247)(273.3850347,399.56836803)
\curveto(273.96855128,398.97726214)(274.26031248,398.22891361)(274.26031918,397.3233202)
\curveto(274.26031248,396.41014251)(273.96476217,395.65232121)(273.37366737,395.04985404)
\curveto(272.78635006,394.45117446)(272.04557974,394.15183505)(271.1513542,394.15183491)
\curveto(270.59814107,394.15183505)(270.06956072,394.27687556)(269.56561156,394.52695683)
\curveto(269.0654475,394.77703761)(268.6846423,395.14268639)(268.42319481,395.62390425)
\curveto(268.16174561,396.10890854)(268.03102143,396.6981146)(268.0310219,397.39152419)
\moveto(269.66791753,397.30626921)
\curveto(269.66791543,396.70758736)(269.81000692,396.24910548)(270.09419244,395.93082218)
\curveto(270.37837289,395.61253559)(270.72886524,395.45339312)(271.14567054,395.45339429)
\curveto(271.56246867,395.45339312)(271.91106646,395.61253559)(272.19146497,395.93082218)
\curveto(272.47564333,396.24910548)(272.61773482,396.71137647)(272.61773988,397.31763654)
\curveto(272.61773482,397.90873412)(272.47564333,398.36342689)(272.19146497,398.68171623)
\curveto(271.91106646,398.99999678)(271.56246867,399.15913925)(271.14567054,399.15914413)
\curveto(270.72886524,399.15913925)(270.37837289,398.99999678)(270.09419244,398.68171623)
\curveto(269.81000692,398.36342689)(269.66791543,397.90494501)(269.66791753,397.30626921)
}
}
{
\newrgbcolor{curcolor}{0 0 0}
\pscustom[linestyle=none,fillstyle=solid,fillcolor=curcolor]
{
\newpath
\moveto(279.48929051,394.28824288)
\lineto(279.48929051,395.19194568)
\curveto(279.26951753,394.86987072)(278.97965088,394.61600059)(278.61968971,394.43033451)
\curveto(278.26350976,394.24466815)(277.88649366,394.15183505)(277.48864029,394.15183491)
\curveto(277.08320309,394.15183505)(276.71944887,394.24087905)(276.39737654,394.41896718)
\curveto(276.07530077,394.59705506)(275.84227072,394.84713608)(275.69828569,395.16921102)
\curveto(275.55429863,395.49128419)(275.48230561,395.9365042)(275.48230641,396.50487238)
\lineto(275.48230641,400.32429553)
\lineto(277.07941638,400.32429553)
\lineto(277.07941638,397.55066682)
\curveto(277.07941398,396.7019037)(277.10783228,396.18090156)(277.16467136,395.98765883)
\curveto(277.22529458,395.79820181)(277.33328412,395.64663755)(277.48864029,395.5329656)
\curveto(277.64399085,395.42308027)(277.84102439,395.36813822)(278.07974149,395.3681393)
\curveto(278.35255376,395.36813822)(278.59695113,395.4420258)(278.81293433,395.58980226)
\curveto(279.02890927,395.74136521)(279.17668442,395.92703143)(279.25626023,396.14680146)
\curveto(279.33582689,396.37035689)(279.37561251,396.91409367)(279.3756172,397.77801343)
\lineto(279.3756172,400.32429553)
\lineto(280.97272718,400.32429553)
\lineto(280.97272718,394.28824288)
\lineto(279.48929051,394.28824288)
}
}
{
\newrgbcolor{curcolor}{0 0 0}
\pscustom[linestyle=none,fillstyle=solid,fillcolor=curcolor]
{
\newpath
\moveto(284.16126264,394.28824288)
\lineto(282.56415266,394.28824288)
\lineto(282.56415266,400.32429553)
\lineto(284.04758933,400.32429553)
\lineto(284.04758933,399.46606206)
\curveto(284.30145722,399.87149127)(284.5288036,400.13862328)(284.72962918,400.26745888)
\curveto(284.934238,400.39628252)(285.16537349,400.46069733)(285.42303636,400.4607035)
\curveto(285.78678695,400.46069733)(286.1372793,400.36028601)(286.47451446,400.15946924)
\lineto(285.98003557,398.76697121)
\curveto(285.71100482,398.94126563)(285.4609238,399.02841508)(285.22979174,399.02841982)
\curveto(285.00623102,399.02841508)(284.8167757,398.96589482)(284.6614252,398.84085886)
\curveto(284.50606897,398.7196029)(284.382923,398.49794017)(284.29198694,398.17587001)
\curveto(284.204835,397.85379207)(284.16126028,397.17933112)(284.16126264,396.15248513)
\lineto(284.16126264,394.28824288)
}
}
{
\newrgbcolor{curcolor}{0 0 0}
\pscustom[linestyle=none,fillstyle=solid,fillcolor=curcolor]
{
\newpath
\moveto(292.41963107,398.5396246)
\lineto(290.84525576,398.25544133)
\curveto(290.79220374,398.5699332)(290.67095233,398.80675235)(290.48150117,398.9658995)
\curveto(290.29583079,399.1250373)(290.05332798,399.20460853)(289.753992,399.20461345)
\curveto(289.35613239,399.20460853)(289.03784744,399.06630615)(288.79913621,398.78970587)
\curveto(288.56420913,398.51688571)(288.44674683,398.05840382)(288.44674896,397.41425885)
\curveto(288.44674683,396.6981146)(288.56610369,396.19226888)(288.80481988,395.89672019)
\curveto(289.04732021,395.60116827)(289.37128881,395.45339312)(289.77672666,395.45339429)
\curveto(290.07985172,395.45339312)(290.3280382,395.53864802)(290.52128683,395.70915923)
\curveto(290.71452706,395.88345671)(290.85093489,396.18090156)(290.93051074,396.6014947)
\lineto(292.49920239,396.33436242)
\curveto(292.33626463,395.61443015)(292.02366335,395.07069337)(291.5613976,394.70315045)
\curveto(291.09912137,394.33560671)(290.47960246,394.15183505)(289.70283901,394.15183491)
\curveto(288.81997382,394.15183505)(288.11520001,394.43033437)(287.58851548,394.98733372)
\curveto(287.06561752,395.54433168)(286.80416917,396.31541484)(286.80416966,397.30058554)
\curveto(286.80416917,398.29711753)(287.06751207,399.07198981)(287.59419915,399.62520469)
\curveto(288.12088367,400.182198)(288.83323569,400.46069733)(289.73125734,400.4607035)
\curveto(290.46634058,400.46069733)(291.04986298,400.30155486)(291.48182628,399.98327561)
\curveto(291.91756836,399.66877408)(292.23016965,399.18755755)(292.41963107,398.5396246)
}
}
{
\newrgbcolor{curcolor}{0 0 0}
\pscustom[linestyle=none,fillstyle=solid,fillcolor=curcolor]
{
\newpath
\moveto(297.1313882,396.20932178)
\lineto(298.72281452,395.94218951)
\curveto(298.51819684,395.35866546)(298.19422824,394.91344545)(297.75090773,394.60652814)
\curveto(297.31136643,394.3033993)(296.76005144,394.15183505)(296.0969611,394.15183491)
\curveto(295.04737531,394.15183505)(294.27060848,394.49474918)(293.76665829,395.18057835)
\curveto(293.36880114,395.72999789)(293.16987305,396.42340438)(293.16987342,397.26079988)
\curveto(293.16987305,398.26112102)(293.4313214,399.04357151)(293.95421925,399.60815369)
\curveto(294.47711479,400.17651434)(295.13831387,400.46069733)(295.93781847,400.4607035)
\curveto(296.83583357,400.46069733)(297.54439648,400.16325247)(298.06350933,399.56836803)
\curveto(298.58261165,398.97726214)(298.83079813,398.06977114)(298.8080695,396.84589231)
\lineto(294.80676906,396.84589231)
\curveto(294.81813437,396.37225144)(294.94696399,396.00281356)(295.1932583,395.73757756)
\curveto(295.43954783,395.47612776)(295.74646546,395.34540359)(296.1140121,395.34540464)
\curveto(296.36408981,395.34540359)(296.57438522,395.4136075)(296.74489896,395.5500166)
\curveto(296.9154048,395.68642317)(297.04423442,395.90619134)(297.1313882,396.20932178)
\moveto(297.22232685,397.82348276)
\curveto(297.21095511,398.28575021)(297.09159825,398.63624256)(296.86425593,398.87496086)
\curveto(296.63690548,399.11745908)(296.3603007,399.23871049)(296.03444078,399.23871544)
\curveto(295.68583975,399.23871049)(295.39786766,399.11177542)(295.17052364,398.85790986)
\curveto(294.94317488,398.60403516)(294.83139624,398.25922647)(294.83518738,397.82348276)
\lineto(297.22232685,397.82348276)
}
}
{
\newrgbcolor{curcolor}{0 0 0}
\pscustom[linestyle=none,fillstyle=solid,fillcolor=curcolor]
{
\newpath
\moveto(299.55262818,396.01039349)
\lineto(301.15542182,396.25479111)
\curveto(301.22362386,395.94408241)(301.36192625,395.70726325)(301.5703294,395.54433293)
\curveto(301.77872796,395.3851892)(302.07048916,395.30561797)(302.44561387,395.30561899)
\curveto(302.85862331,395.30561797)(303.16933004,395.3814001)(303.37773499,395.5329656)
\curveto(303.51792784,395.63905934)(303.58802631,395.78115083)(303.58803061,395.95924051)
\curveto(303.58802631,396.08049024)(303.55013524,396.18090156)(303.47435731,396.26047477)
\curveto(303.39478188,396.33625493)(303.21669387,396.4063534)(302.94009276,396.47077039)
\curveto(301.6517929,396.7549512)(300.83524045,397.01450499)(300.49043297,397.24943255)
\curveto(300.01300435,397.57529275)(299.77429064,398.02809097)(299.77429113,398.60782858)
\curveto(299.77429064,399.13072096)(299.98079694,399.57025731)(300.39381066,399.92643895)
\curveto(300.80682215,400.28260932)(301.44718115,400.46069733)(302.31488957,400.4607035)
\curveto(303.14091174,400.46069733)(303.75474699,400.32618405)(304.15639715,400.05716326)
\curveto(304.55803756,399.78813093)(304.83464234,399.39027475)(304.9862123,398.86359352)
\lineto(303.48004097,398.58509392)
\curveto(303.41562196,398.82001422)(303.292476,398.99999678)(303.11060272,399.12504213)
\curveto(302.93251089,399.25007781)(302.6767462,399.31259807)(302.34330789,399.31260309)
\curveto(301.92271401,399.31259807)(301.62148004,399.25386692)(301.43960509,399.13640946)
\curveto(301.31835153,399.05304427)(301.25772582,398.94505474)(301.2577278,398.81244054)
\curveto(301.25772582,398.69876282)(301.31077331,398.6021406)(301.41687043,398.5225736)
\curveto(301.56085434,398.41647439)(302.05722729,398.26680468)(302.90599077,398.07356403)
\curveto(303.7585361,397.88031582)(304.35342581,397.64349666)(304.6906617,397.36310586)
\curveto(305.02409766,397.0789198)(305.19081835,396.68295817)(305.19082426,396.17521979)
\curveto(305.19081835,395.62200836)(304.95968285,395.1464755)(304.49741708,394.74861978)
\curveto(304.03514087,394.35076314)(303.35120715,394.15183505)(302.44561387,394.15183491)
\curveto(301.6233746,394.15183505)(300.97164828,394.31855573)(300.49043297,394.65199746)
\curveto(300.01300435,394.98543847)(299.70040306,395.43823669)(299.55262818,396.01039349)
}
}
{
\newrgbcolor{curcolor}{0 1 0.25098041}
\pscustom[linewidth=2.32802935,linecolor=curcolor]
{
\newpath
\moveto(130.18849429,295.92182848)
\lineto(204.68543124,295.92182848)
\lineto(204.68543124,342.4824199)
\lineto(154.63280013,342.4824199)
\lineto(154.63280013,351.79453818)
\lineto(130.18849429,351.79453818)
\lineto(130.18849429,295.92182848)
\closepath
}
}
{
\newrgbcolor{curcolor}{0 0 0}
\pscustom[linestyle=none,fillstyle=solid,fillcolor=curcolor]
{
\newpath
\moveto(143.62921682,323.43453329)
\lineto(143.62921682,324.38938908)
\lineto(149.14805592,326.71969189)
\lineto(149.14805592,325.70231578)
\lineto(144.77163357,323.90627752)
\lineto(149.14805592,322.09318826)
\lineto(149.14805592,321.07581215)
\lineto(143.62921682,323.43453329)
}
}
{
\newrgbcolor{curcolor}{0 0 0}
\pscustom[linestyle=none,fillstyle=solid,fillcolor=curcolor]
{
\newpath
\moveto(150.42688081,323.43453329)
\lineto(150.42688081,324.38938908)
\lineto(155.94571992,326.71969189)
\lineto(155.94571992,325.70231578)
\lineto(151.56929756,323.90627752)
\lineto(155.94571992,322.09318826)
\lineto(155.94571992,321.07581215)
\lineto(150.42688081,323.43453329)
}
}
{
\newrgbcolor{curcolor}{0 0 0}
\pscustom[linestyle=none,fillstyle=solid,fillcolor=curcolor]
{
\newpath
\moveto(161.29404923,320.53586394)
\curveto(160.91513388,320.21378914)(160.5494851,319.98644275)(160.19710181,319.85382409)
\curveto(159.84850041,319.7212053)(159.47337887,319.65489594)(159.07173606,319.6548958)
\curveto(158.40863995,319.65489594)(157.89900512,319.81593296)(157.54283007,320.13800736)
\curveto(157.18665311,320.46387017)(157.0085651,320.87877733)(157.00856552,321.38273008)
\curveto(157.0085651,321.6782788)(157.07487447,321.94730535)(157.20749381,322.18981057)
\curveto(157.34390103,322.43610009)(157.52009448,322.63313363)(157.73607469,322.78091177)
\curveto(157.95584172,322.92868393)(158.20213364,323.04046257)(158.4749512,323.11624803)
\curveto(158.67577195,323.16929219)(158.97890047,323.22044513)(159.38433766,323.26970699)
\curveto(160.21036008,323.36822028)(160.81851166,323.48568258)(161.20879425,323.62209425)
\curveto(161.21257874,323.76228736)(161.21447329,323.85133136)(161.21447792,323.88922652)
\curveto(161.21447329,324.30602414)(161.11785108,324.59967989)(160.92461098,324.77019466)
\curveto(160.6631583,325.00132517)(160.27477489,325.11689292)(159.75945958,325.11689825)
\curveto(159.27823988,325.11689292)(158.92206387,325.03163803)(158.69093048,324.86113331)
\curveto(158.46358199,324.69440755)(158.29496675,324.39696269)(158.18508426,323.96879784)
\lineto(157.18475915,324.10520581)
\curveto(157.27569711,324.53337053)(157.42536681,324.87817921)(157.63376872,325.13963291)
\curveto(157.84216853,325.40486501)(158.14340249,325.60758221)(158.53747152,325.74778511)
\curveto(158.93153664,325.8917652)(159.38812397,325.96375822)(159.90723488,325.96376439)
\curveto(160.42255004,325.96375822)(160.8412463,325.90313252)(161.16332493,325.7818871)
\curveto(161.4853944,325.6606297)(161.72221356,325.50717089)(161.8737831,325.3215102)
\curveto(162.02534208,325.13962756)(162.13143706,324.90849207)(162.19206837,324.62810302)
\curveto(162.22616472,324.45379929)(162.2432157,324.13930345)(162.24322135,323.68461457)
\lineto(162.24322135,322.32053487)
\curveto(162.2432157,321.36946662)(162.26405579,320.76699869)(162.30574167,320.51312928)
\curveto(162.35120523,320.26304753)(162.43835468,320.02243926)(162.56719028,319.79130377)
\lineto(161.49866119,319.79130377)
\curveto(161.3925613,320.00349373)(161.32435738,320.25168021)(161.29404923,320.53586394)
\moveto(161.20879425,322.82069743)
\curveto(160.8374572,322.66913014)(160.28045854,322.54030052)(159.53779663,322.43420818)
\curveto(159.11720286,322.37357983)(158.819758,322.30537592)(158.64546116,322.22959623)
\curveto(158.4711602,322.15381166)(158.33664692,322.04203302)(158.24192091,321.89425997)
\curveto(158.1471916,321.75027182)(158.09982777,321.58923479)(158.09982928,321.41114841)
\curveto(158.09982777,321.13833112)(158.20213364,320.91098473)(158.40674721,320.72910856)
\curveto(158.61514625,320.54723051)(158.91827477,320.45629196)(159.31613367,320.45629262)
\curveto(159.71019802,320.45629196)(160.06069037,320.54154685)(160.36761177,320.71205756)
\curveto(160.67452562,320.88635554)(160.89997745,321.1231747)(161.04396796,321.42251574)
\curveto(161.15384759,321.6536496)(161.20878963,321.99466919)(161.20879425,322.44557551)
\lineto(161.20879425,322.82069743)
}
}
{
\newrgbcolor{curcolor}{0 0 0}
\pscustom[linestyle=none,fillstyle=solid,fillcolor=curcolor]
{
\newpath
\moveto(163.83464707,317.47805195)
\lineto(163.83464707,325.82735642)
\lineto(164.7667682,325.82735642)
\lineto(164.7667682,325.0430106)
\curveto(164.98653467,325.34992297)(165.23472115,325.57916391)(165.51132837,325.73073411)
\curveto(165.78793069,325.88608154)(166.12326662,325.96375822)(166.51733714,325.96376439)
\curveto(167.03265217,325.96375822)(167.48734495,325.83113949)(167.88141684,325.56590781)
\curveto(168.2754791,325.30066459)(168.57292395,324.92554305)(168.7737523,324.44054207)
\curveto(168.97456924,323.95932089)(169.07498056,323.43074054)(169.07498657,322.85479942)
\curveto(169.07498056,322.237172)(168.96320192,321.68017335)(168.73965031,321.18380179)
\curveto(168.51987646,320.69121656)(168.19780241,320.31230591)(167.77342719,320.04706871)
\curveto(167.35283167,319.78562011)(166.90950621,319.65489594)(166.44344949,319.6548958)
\curveto(166.10242653,319.65489594)(165.79550891,319.72688896)(165.5226957,319.87087509)
\curveto(165.25366668,320.01486105)(165.03200395,320.19673816)(164.85770684,320.41650696)
\lineto(164.85770684,317.47805195)
\lineto(163.83464707,317.47805195)
\moveto(164.76108453,322.7752281)
\curveto(164.76108284,321.99845829)(164.91833076,321.42440866)(165.23282876,321.05307749)
\curveto(165.54732243,320.68174379)(165.92812763,320.49607757)(166.37524551,320.49607828)
\curveto(166.82993498,320.49607757)(167.21831839,320.68742745)(167.54039691,321.07012848)
\curveto(167.8662556,321.45661607)(168.02918718,322.05340034)(168.02919214,322.86048308)
\curveto(168.02918718,323.62966863)(167.8700447,324.20561281)(167.55176424,324.58831737)
\curveto(167.23726392,324.97101232)(166.86024783,325.1623622)(166.42071483,325.16236757)
\curveto(165.98496423,325.1623622)(165.59847537,324.95775045)(165.26124709,324.54853171)
\curveto(164.92780352,324.14309256)(164.76108284,323.55199195)(164.76108453,322.7752281)
}
}
{
\newrgbcolor{curcolor}{0 0 0}
\pscustom[linestyle=none,fillstyle=solid,fillcolor=curcolor]
{
\newpath
\moveto(170.31402592,317.47805195)
\lineto(170.31402592,325.82735642)
\lineto(171.24614705,325.82735642)
\lineto(171.24614705,325.0430106)
\curveto(171.46591352,325.34992297)(171.7141,325.57916391)(171.99070721,325.73073411)
\curveto(172.26730954,325.88608154)(172.60264547,325.96375822)(172.99671599,325.96376439)
\curveto(173.51203102,325.96375822)(173.9667238,325.83113949)(174.36079569,325.56590781)
\curveto(174.75485795,325.30066459)(175.0523028,324.92554305)(175.25313115,324.44054207)
\curveto(175.45394809,323.95932089)(175.55435941,323.43074054)(175.55436542,322.85479942)
\curveto(175.55435941,322.237172)(175.44258077,321.68017335)(175.21902916,321.18380179)
\curveto(174.99925531,320.69121656)(174.67718126,320.31230591)(174.25280604,320.04706871)
\curveto(173.83221052,319.78562011)(173.38888506,319.65489594)(172.92282834,319.6548958)
\curveto(172.58180538,319.65489594)(172.27488776,319.72688896)(172.00207455,319.87087509)
\curveto(171.73304553,320.01486105)(171.5113828,320.19673816)(171.33708569,320.41650696)
\lineto(171.33708569,317.47805195)
\lineto(170.31402592,317.47805195)
\moveto(171.24046338,322.7752281)
\curveto(171.24046169,321.99845829)(171.39770961,321.42440866)(171.71220761,321.05307749)
\curveto(172.02670128,320.68174379)(172.40750648,320.49607757)(172.85462436,320.49607828)
\curveto(173.30931382,320.49607757)(173.69769724,320.68742745)(174.01977576,321.07012848)
\curveto(174.34563445,321.45661607)(174.50856602,322.05340034)(174.50857099,322.86048308)
\curveto(174.50856602,323.62966863)(174.34942355,324.20561281)(174.03114309,324.58831737)
\curveto(173.71664277,324.97101232)(173.33962668,325.1623622)(172.90009368,325.16236757)
\curveto(172.46434308,325.1623622)(172.07785422,324.95775045)(171.74062594,324.54853171)
\curveto(171.40718237,324.14309256)(171.24046169,323.55199195)(171.24046338,322.7752281)
}
}
{
\newrgbcolor{curcolor}{0 0 0}
\pscustom[linestyle=none,fillstyle=solid,fillcolor=curcolor]
{
\newpath
\moveto(182.18151957,323.43453329)
\lineto(176.66268047,321.07581215)
\lineto(176.66268047,322.09318826)
\lineto(181.03341916,323.90627752)
\lineto(176.66268047,325.70231578)
\lineto(176.66268047,326.71969189)
\lineto(182.18151957,324.38938908)
\lineto(182.18151957,323.43453329)
}
}
{
\newrgbcolor{curcolor}{0 0 0}
\pscustom[linestyle=none,fillstyle=solid,fillcolor=curcolor]
{
\newpath
\moveto(188.97918223,323.43453329)
\lineto(183.46034313,321.07581215)
\lineto(183.46034313,322.09318826)
\lineto(187.83108182,323.90627752)
\lineto(183.46034313,325.70231578)
\lineto(183.46034313,326.71969189)
\lineto(188.97918223,324.38938908)
\lineto(188.97918223,323.43453329)
}
}
{
\newrgbcolor{curcolor}{0 0 0}
\pscustom[linestyle=none,fillstyle=solid,fillcolor=curcolor]
{
\newpath
\moveto(158.63069038,302.33107389)
\lineto(157.0335804,302.33107389)
\lineto(157.0335804,305.41162054)
\curveto(157.03357568,306.06334377)(156.99947372,306.48393459)(156.93127443,306.67339426)
\curveto(156.86306588,306.86663434)(156.75128724,307.01630405)(156.59593817,307.12240382)
\curveto(156.44436962,307.22849401)(156.26059795,307.2815415)(156.04462263,307.28154645)
\curveto(155.76801411,307.2815415)(155.51982764,307.20575937)(155.30006246,307.05419984)
\curveto(155.08029129,306.90263086)(154.92872703,306.70180821)(154.84536923,306.45173131)
\curveto(154.76579545,306.20164616)(154.72600983,305.73937517)(154.72601225,305.06491695)
\lineto(154.72601225,302.33107389)
\lineto(153.12890228,302.33107389)
\lineto(153.12890228,308.36712655)
\lineto(154.61233895,308.36712655)
\lineto(154.61233895,307.48047474)
\curveto(155.13902244,308.16250876)(155.80211607,308.50352834)(156.60162183,308.50353452)
\curveto(156.95400444,308.50352834)(157.27607849,308.43911353)(157.56784495,308.31028989)
\curveto(157.85960089,308.1852434)(158.07936906,308.02420637)(158.22715014,307.82717833)
\curveto(158.37870848,307.6301393)(158.4829089,307.40658202)(158.53975174,307.15650582)
\curveto(158.6003712,306.90641996)(158.63068406,306.5483494)(158.63069038,306.08229306)
\lineto(158.63069038,302.33107389)
}
}
{
\newrgbcolor{curcolor}{0 0 0}
\pscustom[linestyle=none,fillstyle=solid,fillcolor=curcolor]
{
\newpath
\moveto(159.88678049,305.4343552)
\curveto(159.88678003,305.964827)(160.0175042,306.47825093)(160.27895341,306.97462852)
\curveto(160.54040089,307.47099683)(160.90983878,307.84990748)(161.38726816,308.1113616)
\curveto(161.86848271,308.37280417)(162.40464128,308.50352834)(162.99574547,308.50353452)
\curveto(163.90891655,308.50352834)(164.65726508,308.20608348)(165.2407933,307.61119905)
\curveto(165.82430988,307.02009316)(166.11607108,306.27174463)(166.11607777,305.36615122)
\curveto(166.11607108,304.45297352)(165.82052077,303.69515222)(165.22942597,303.09268506)
\curveto(164.64210866,302.49400547)(163.90133834,302.19466606)(163.0071128,302.19466592)
\curveto(162.45389967,302.19466606)(161.92531931,302.31970657)(161.42137015,302.56978784)
\curveto(160.9212061,302.81986863)(160.54040089,303.1855174)(160.27895341,303.66673526)
\curveto(160.0175042,304.15173955)(159.88678003,304.74094561)(159.88678049,305.4343552)
\moveto(161.52367613,305.34910022)
\curveto(161.52367403,304.75041838)(161.66576552,304.29193649)(161.94995103,303.97365319)
\curveto(162.23413149,303.65536661)(162.58462384,303.49622413)(163.00142913,303.4962253)
\curveto(163.41822726,303.49622413)(163.76682506,303.65536661)(164.04722357,303.97365319)
\curveto(164.33140192,304.29193649)(164.47349342,304.75420748)(164.47349847,305.36046755)
\curveto(164.47349342,305.95156513)(164.33140192,306.40625791)(164.04722357,306.72454725)
\curveto(163.76682506,307.0428278)(163.41822726,307.20197027)(163.00142913,307.20197514)
\curveto(162.58462384,307.20197027)(162.23413149,307.0428278)(161.94995103,306.72454725)
\curveto(161.66576552,306.40625791)(161.52367403,305.94777602)(161.52367613,305.34910022)
}
}
{
\newrgbcolor{curcolor}{0 0 0}
\pscustom[linestyle=none,fillstyle=solid,fillcolor=curcolor]
{
\newpath
\moveto(170.14011293,308.36712655)
\lineto(170.14011293,307.0939855)
\lineto(169.04884917,307.0939855)
\lineto(169.04884917,304.66137671)
\curveto(169.04884666,304.16879053)(169.05831943,303.88081844)(169.0772675,303.79745957)
\curveto(169.0999996,303.71788686)(169.14736343,303.6515775)(169.21935914,303.59853128)
\curveto(169.29513858,303.54548252)(169.38607714,303.51895877)(169.49217508,303.51895996)
\curveto(169.63994727,303.51895877)(169.85403179,303.57011171)(170.13442927,303.67241893)
\lineto(170.27083724,302.43337987)
\curveto(169.89950107,302.2742373)(169.47891025,302.19466606)(169.00906352,302.19466592)
\curveto(168.72108895,302.19466606)(168.46153516,302.24202989)(168.23040136,302.33675756)
\curveto(167.99926417,302.43527432)(167.82875438,302.56031483)(167.71887147,302.71187947)
\curveto(167.61277531,302.86723246)(167.53888773,303.07563332)(167.49720852,303.33708267)
\curveto(167.4631056,303.52274788)(167.44605462,303.89786942)(167.44605553,304.46244842)
\lineto(167.44605553,307.0939855)
\lineto(166.7128627,307.0939855)
\lineto(166.7128627,308.36712655)
\lineto(167.44605553,308.36712655)
\lineto(167.44605553,309.56637994)
\lineto(169.04884917,310.49850107)
\lineto(169.04884917,308.36712655)
\lineto(170.14011293,308.36712655)
}
}
{
\newrgbcolor{curcolor}{0 0 0}
\pscustom[linestyle=none,fillstyle=solid,fillcolor=curcolor]
{
\newpath
\moveto(174.75524941,304.2521528)
\lineto(176.34667572,303.98502052)
\curveto(176.14205805,303.40149647)(175.81808945,302.95627646)(175.37476894,302.64935915)
\curveto(174.93522764,302.34623032)(174.38391265,302.19466606)(173.72082231,302.19466592)
\curveto(172.67123652,302.19466606)(171.89446969,302.5375802)(171.3905195,303.22340936)
\curveto(170.99266235,303.77282891)(170.79373426,304.46623539)(170.79373463,305.3036309)
\curveto(170.79373426,306.30395203)(171.05518261,307.08640252)(171.57808045,307.65098471)
\curveto(172.10097599,308.21934536)(172.76217507,308.50352834)(173.56167968,308.50353452)
\curveto(174.45969478,308.50352834)(175.16825769,308.20608348)(175.68737054,307.61119905)
\curveto(176.20647286,307.02009316)(176.45465934,306.11260216)(176.4319307,304.88872332)
\lineto(172.43063026,304.88872332)
\curveto(172.44199558,304.41508245)(172.5708252,304.04564457)(172.81711951,303.78040857)
\curveto(173.06340904,303.51895877)(173.37032666,303.3882346)(173.73787331,303.38823566)
\curveto(173.98795102,303.3882346)(174.19824643,303.45643852)(174.36876016,303.59284761)
\curveto(174.53926601,303.72925418)(174.66809563,303.94902236)(174.75524941,304.2521528)
\moveto(174.84618806,305.86631377)
\curveto(174.83481632,306.32858123)(174.71545946,306.67907357)(174.48811714,306.91779187)
\curveto(174.26076669,307.1602901)(173.98416191,307.2815415)(173.65830199,307.28154645)
\curveto(173.30970096,307.2815415)(173.02172887,307.15460644)(172.79438485,306.90074087)
\curveto(172.56703609,306.64686617)(172.45525745,306.30205748)(172.45904859,305.86631377)
\lineto(174.84618806,305.86631377)
}
}
{
\newrgbcolor{curcolor}{0 0 0}
\pscustom[linestyle=none,fillstyle=solid,fillcolor=curcolor]
{
\newpath
\moveto(177.17649117,304.05322451)
\lineto(178.77928481,304.29762212)
\curveto(178.84748685,303.98691342)(178.98578924,303.75009427)(179.19419238,303.58716395)
\curveto(179.40259095,303.42802022)(179.69435215,303.34844898)(180.06947685,303.34845)
\curveto(180.48248629,303.34844898)(180.79319303,303.42423111)(181.00159798,303.57579661)
\curveto(181.14179082,303.68189035)(181.21188929,303.82398184)(181.2118936,304.00207152)
\curveto(181.21188929,304.12332126)(181.17399823,304.22373258)(181.09822029,304.30330579)
\curveto(181.01864486,304.37908594)(180.84055686,304.44918441)(180.56395574,304.51360141)
\curveto(179.27565588,304.79778221)(178.45910344,305.057336)(178.11429596,305.29226356)
\curveto(177.63686733,305.61812376)(177.39815362,306.07092198)(177.39815412,306.6506596)
\curveto(177.39815362,307.17355197)(177.60465993,307.61308832)(178.01767365,307.96926997)
\curveto(178.43068514,308.32544034)(179.07104413,308.50352834)(179.93875255,308.50353452)
\curveto(180.76477473,308.50352834)(181.37860998,308.36901506)(181.78026014,308.09999427)
\curveto(182.18190055,307.83096194)(182.45850532,307.43310576)(182.61007529,306.90642454)
\lineto(181.10390396,306.62792493)
\curveto(181.03948495,306.86284524)(180.91633899,307.0428278)(180.73446571,307.16787315)
\curveto(180.55637387,307.29290882)(180.30060918,307.35542908)(179.96717088,307.3554341)
\curveto(179.54657699,307.35542908)(179.24534303,307.29669793)(179.06346808,307.17924048)
\curveto(178.94221451,307.09587529)(178.88158881,306.98788575)(178.88159079,306.85527155)
\curveto(178.88158881,306.74159383)(178.9346363,306.64497162)(179.04073342,306.56540461)
\curveto(179.18471733,306.4593054)(179.68109027,306.30963569)(180.52985375,306.11639505)
\curveto(181.38239908,305.92314683)(181.9772888,305.68632768)(182.31452469,305.40593687)
\curveto(182.64796065,305.12175081)(182.81468133,304.72578919)(182.81468724,304.2180508)
\curveto(182.81468133,303.66483937)(182.58354584,303.18930651)(182.12128006,302.79145079)
\curveto(181.65900386,302.39359415)(180.97507014,302.19466606)(180.06947685,302.19466592)
\curveto(179.24723758,302.19466606)(178.59551127,302.36138674)(178.11429596,302.69482848)
\curveto(177.63686733,303.02826948)(177.32426605,303.48106771)(177.17649117,304.05322451)
}
}
{
\newrgbcolor{curcolor}{0 1 0.25098041}
\pscustom[linewidth=2.32802935,linecolor=curcolor]
{
\newpath
\moveto(130.18849429,114.33554061)
\lineto(204.68543124,114.33554061)
\lineto(204.68543124,160.89613202)
\lineto(154.63280013,160.89613202)
\lineto(154.63280013,170.2082503)
\lineto(130.18849429,170.2082503)
\lineto(130.18849429,114.33554061)
\closepath
}
}
{
\newrgbcolor{curcolor}{0 0 0}
\pscustom[linestyle=none,fillstyle=solid,fillcolor=curcolor]
{
\newpath
\moveto(143.62921682,141.84823355)
\lineto(143.62921682,142.80308934)
\lineto(149.14805592,145.13339215)
\lineto(149.14805592,144.11601604)
\lineto(144.77163357,142.31997778)
\lineto(149.14805592,140.50688852)
\lineto(149.14805592,139.48951241)
\lineto(143.62921682,141.84823355)
}
}
{
\newrgbcolor{curcolor}{0 0 0}
\pscustom[linestyle=none,fillstyle=solid,fillcolor=curcolor]
{
\newpath
\moveto(150.42688081,141.84823355)
\lineto(150.42688081,142.80308934)
\lineto(155.94571992,145.13339215)
\lineto(155.94571992,144.11601604)
\lineto(151.56929756,142.31997778)
\lineto(155.94571992,140.50688852)
\lineto(155.94571992,139.48951241)
\lineto(150.42688081,141.84823355)
}
}
{
\newrgbcolor{curcolor}{0 0 0}
\pscustom[linestyle=none,fillstyle=solid,fillcolor=curcolor]
{
\newpath
\moveto(161.29404923,138.9495642)
\curveto(160.91513388,138.6274894)(160.5494851,138.40014301)(160.19710181,138.26752435)
\curveto(159.84850041,138.13490556)(159.47337887,138.0685962)(159.07173606,138.06859606)
\curveto(158.40863995,138.0685962)(157.89900512,138.22963322)(157.54283007,138.55170762)
\curveto(157.18665311,138.87757043)(157.0085651,139.29247759)(157.00856552,139.79643034)
\curveto(157.0085651,140.09197905)(157.07487447,140.36100561)(157.20749381,140.60351083)
\curveto(157.34390103,140.84980035)(157.52009448,141.04683389)(157.73607469,141.19461203)
\curveto(157.95584172,141.34238419)(158.20213364,141.45416283)(158.4749512,141.52994829)
\curveto(158.67577195,141.58299245)(158.97890047,141.63414539)(159.38433766,141.68340725)
\curveto(160.21036008,141.78192054)(160.81851166,141.89938284)(161.20879425,142.03579451)
\curveto(161.21257874,142.17598762)(161.21447329,142.26503162)(161.21447792,142.30292678)
\curveto(161.21447329,142.7197244)(161.11785108,143.01338015)(160.92461098,143.18389492)
\curveto(160.6631583,143.41502543)(160.27477489,143.53059318)(159.75945958,143.53059851)
\curveto(159.27823988,143.53059318)(158.92206387,143.44533829)(158.69093048,143.27483356)
\curveto(158.46358199,143.10810781)(158.29496675,142.81066295)(158.18508426,142.3824981)
\lineto(157.18475915,142.51890607)
\curveto(157.27569711,142.94707078)(157.42536681,143.29187947)(157.63376872,143.55333317)
\curveto(157.84216853,143.81856527)(158.14340249,144.02128247)(158.53747152,144.16148537)
\curveto(158.93153664,144.30546546)(159.38812397,144.37745848)(159.90723488,144.37746465)
\curveto(160.42255004,144.37745848)(160.8412463,144.31683278)(161.16332493,144.19558736)
\curveto(161.4853944,144.07432996)(161.72221356,143.92087115)(161.8737831,143.73521046)
\curveto(162.02534208,143.55332782)(162.13143706,143.32219233)(162.19206837,143.04180328)
\curveto(162.22616472,142.86749955)(162.2432157,142.55300371)(162.24322135,142.09831483)
\lineto(162.24322135,140.73423513)
\curveto(162.2432157,139.78316688)(162.26405579,139.18069895)(162.30574167,138.92682953)
\curveto(162.35120523,138.67674779)(162.43835468,138.43613952)(162.56719028,138.20500403)
\lineto(161.49866119,138.20500403)
\curveto(161.3925613,138.41719399)(161.32435738,138.66538047)(161.29404923,138.9495642)
\moveto(161.20879425,141.23439769)
\curveto(160.8374572,141.0828304)(160.28045854,140.95400078)(159.53779663,140.84790844)
\curveto(159.11720286,140.78728009)(158.819758,140.71907618)(158.64546116,140.64329648)
\curveto(158.4711602,140.56751192)(158.33664692,140.45573328)(158.24192091,140.30796023)
\curveto(158.1471916,140.16397208)(158.09982777,140.00293505)(158.09982928,139.82484867)
\curveto(158.09982777,139.55203138)(158.20213364,139.32468499)(158.40674721,139.14280882)
\curveto(158.61514625,138.96093077)(158.91827477,138.86999222)(159.31613367,138.86999288)
\curveto(159.71019802,138.86999222)(160.06069037,138.95524711)(160.36761177,139.12575782)
\curveto(160.67452562,139.3000558)(160.89997745,139.53687496)(161.04396796,139.836216)
\curveto(161.15384759,140.06734986)(161.20878963,140.40836944)(161.20879425,140.85927577)
\lineto(161.20879425,141.23439769)
}
}
{
\newrgbcolor{curcolor}{0 0 0}
\pscustom[linestyle=none,fillstyle=solid,fillcolor=curcolor]
{
\newpath
\moveto(163.83464707,135.89175221)
\lineto(163.83464707,144.24105668)
\lineto(164.7667682,144.24105668)
\lineto(164.7667682,143.45671086)
\curveto(164.98653467,143.76362323)(165.23472115,143.99286417)(165.51132837,144.14443437)
\curveto(165.78793069,144.2997818)(166.12326662,144.37745848)(166.51733714,144.37746465)
\curveto(167.03265217,144.37745848)(167.48734495,144.24483975)(167.88141684,143.97960807)
\curveto(168.2754791,143.71436485)(168.57292395,143.3392433)(168.7737523,142.85424232)
\curveto(168.97456924,142.37302115)(169.07498056,141.8444408)(169.07498657,141.26849968)
\curveto(169.07498056,140.65087226)(168.96320192,140.09387361)(168.73965031,139.59750205)
\curveto(168.51987646,139.10491682)(168.19780241,138.72600617)(167.77342719,138.46076897)
\curveto(167.35283167,138.19932037)(166.90950621,138.0685962)(166.44344949,138.06859606)
\curveto(166.10242653,138.0685962)(165.79550891,138.14058922)(165.5226957,138.28457534)
\curveto(165.25366668,138.42856131)(165.03200395,138.61043842)(164.85770684,138.83020722)
\lineto(164.85770684,135.89175221)
\lineto(163.83464707,135.89175221)
\moveto(164.76108453,141.18892836)
\curveto(164.76108284,140.41215855)(164.91833076,139.83810892)(165.23282876,139.46677775)
\curveto(165.54732243,139.09544405)(165.92812763,138.90977783)(166.37524551,138.90977854)
\curveto(166.82993498,138.90977783)(167.21831839,139.10112771)(167.54039691,139.48382874)
\curveto(167.8662556,139.87031633)(168.02918718,140.4671006)(168.02919214,141.27418334)
\curveto(168.02918718,142.04336889)(167.8700447,142.61931307)(167.55176424,143.00201763)
\curveto(167.23726392,143.38471258)(166.86024783,143.57606246)(166.42071483,143.57606783)
\curveto(165.98496423,143.57606246)(165.59847537,143.37145071)(165.26124709,142.96223197)
\curveto(164.92780352,142.55679282)(164.76108284,141.96569221)(164.76108453,141.18892836)
}
}
{
\newrgbcolor{curcolor}{0 0 0}
\pscustom[linestyle=none,fillstyle=solid,fillcolor=curcolor]
{
\newpath
\moveto(170.31402592,135.89175221)
\lineto(170.31402592,144.24105668)
\lineto(171.24614705,144.24105668)
\lineto(171.24614705,143.45671086)
\curveto(171.46591352,143.76362323)(171.7141,143.99286417)(171.99070721,144.14443437)
\curveto(172.26730954,144.2997818)(172.60264547,144.37745848)(172.99671599,144.37746465)
\curveto(173.51203102,144.37745848)(173.9667238,144.24483975)(174.36079569,143.97960807)
\curveto(174.75485795,143.71436485)(175.0523028,143.3392433)(175.25313115,142.85424232)
\curveto(175.45394809,142.37302115)(175.55435941,141.8444408)(175.55436542,141.26849968)
\curveto(175.55435941,140.65087226)(175.44258077,140.09387361)(175.21902916,139.59750205)
\curveto(174.99925531,139.10491682)(174.67718126,138.72600617)(174.25280604,138.46076897)
\curveto(173.83221052,138.19932037)(173.38888506,138.0685962)(172.92282834,138.06859606)
\curveto(172.58180538,138.0685962)(172.27488776,138.14058922)(172.00207455,138.28457534)
\curveto(171.73304553,138.42856131)(171.5113828,138.61043842)(171.33708569,138.83020722)
\lineto(171.33708569,135.89175221)
\lineto(170.31402592,135.89175221)
\moveto(171.24046338,141.18892836)
\curveto(171.24046169,140.41215855)(171.39770961,139.83810892)(171.71220761,139.46677775)
\curveto(172.02670128,139.09544405)(172.40750648,138.90977783)(172.85462436,138.90977854)
\curveto(173.30931382,138.90977783)(173.69769724,139.10112771)(174.01977576,139.48382874)
\curveto(174.34563445,139.87031633)(174.50856602,140.4671006)(174.50857099,141.27418334)
\curveto(174.50856602,142.04336889)(174.34942355,142.61931307)(174.03114309,143.00201763)
\curveto(173.71664277,143.38471258)(173.33962668,143.57606246)(172.90009368,143.57606783)
\curveto(172.46434308,143.57606246)(172.07785422,143.37145071)(171.74062594,142.96223197)
\curveto(171.40718237,142.55679282)(171.24046169,141.96569221)(171.24046338,141.18892836)
}
}
{
\newrgbcolor{curcolor}{0 0 0}
\pscustom[linestyle=none,fillstyle=solid,fillcolor=curcolor]
{
\newpath
\moveto(182.18151957,141.84823355)
\lineto(176.66268047,139.48951241)
\lineto(176.66268047,140.50688852)
\lineto(181.03341916,142.31997778)
\lineto(176.66268047,144.11601604)
\lineto(176.66268047,145.13339215)
\lineto(182.18151957,142.80308934)
\lineto(182.18151957,141.84823355)
}
}
{
\newrgbcolor{curcolor}{0 0 0}
\pscustom[linestyle=none,fillstyle=solid,fillcolor=curcolor]
{
\newpath
\moveto(188.97918223,141.84823355)
\lineto(183.46034313,139.48951241)
\lineto(183.46034313,140.50688852)
\lineto(187.83108182,142.31997778)
\lineto(183.46034313,144.11601604)
\lineto(183.46034313,145.13339215)
\lineto(188.97918223,142.80308934)
\lineto(188.97918223,141.84823355)
}
}
{
\newrgbcolor{curcolor}{0 0 0}
\pscustom[linestyle=none,fillstyle=solid,fillcolor=curcolor]
{
\newpath
\moveto(155.9332219,126.7808268)
\lineto(156.8198737,126.7808268)
\lineto(156.8198737,127.23552004)
\curveto(156.81987267,127.74325381)(156.87292016,128.12216446)(156.97901633,128.37225312)
\curveto(157.08889923,128.62232652)(157.28782732,128.82504371)(157.5758012,128.98040531)
\curveto(157.86756062,129.13953955)(158.23510394,129.21911079)(158.67843228,129.21911926)
\curveto(159.13312218,129.21911079)(159.57834219,129.15090687)(160.01409365,129.01450731)
\lineto(159.79811437,127.90050889)
\curveto(159.54424023,127.96112744)(159.29984286,127.99144029)(159.06492153,127.99144753)
\curveto(158.83378277,127.99144029)(158.66706208,127.93649824)(158.56475897,127.82662124)
\curveto(158.46623944,127.72051917)(158.41698105,127.51401287)(158.41698367,127.20710171)
\lineto(158.41698367,126.7808268)
\lineto(159.61055341,126.7808268)
\lineto(159.61055341,125.52473675)
\lineto(158.41698367,125.52473675)
\lineto(158.41698367,120.74477415)
\lineto(156.8198737,120.74477415)
\lineto(156.8198737,125.52473675)
\lineto(155.9332219,125.52473675)
\lineto(155.9332219,126.7808268)
}
}
{
\newrgbcolor{curcolor}{0 0 0}
\pscustom[linestyle=none,fillstyle=solid,fillcolor=curcolor]
{
\newpath
\moveto(160.51994005,127.59927462)
\lineto(160.51994005,129.07702763)
\lineto(162.11705003,129.07702763)
\lineto(162.11705003,127.59927462)
\lineto(160.51994005,127.59927462)
\moveto(160.51994005,120.74477415)
\lineto(160.51994005,126.7808268)
\lineto(162.11705003,126.7808268)
\lineto(162.11705003,120.74477415)
\lineto(160.51994005,120.74477415)
}
}
{
\newrgbcolor{curcolor}{0 0 0}
\pscustom[linestyle=none,fillstyle=solid,fillcolor=curcolor]
{
\newpath
\moveto(163.74826188,120.74477415)
\lineto(163.74826188,129.07702763)
\lineto(165.34537185,129.07702763)
\lineto(165.34537185,120.74477415)
\lineto(163.74826188,120.74477415)
}
}
{
\newrgbcolor{curcolor}{0 0 0}
\pscustom[linestyle=none,fillstyle=solid,fillcolor=curcolor]
{
\newpath
\moveto(170.47203793,122.66585306)
\lineto(172.06346424,122.39872078)
\curveto(171.85884656,121.81519673)(171.53487796,121.36997672)(171.09155745,121.06305941)
\curveto(170.65201615,120.75993058)(170.10070116,120.60836632)(169.43761082,120.60836618)
\curveto(168.38802503,120.60836632)(167.6112582,120.95128045)(167.10730801,121.63710962)
\curveto(166.70945086,122.18652917)(166.51052277,122.87993565)(166.51052314,123.71733115)
\curveto(166.51052277,124.71765229)(166.77197112,125.50010278)(167.29486897,126.06468496)
\curveto(167.81776451,126.63304562)(168.47896359,126.9172286)(169.27846819,126.91723477)
\curveto(170.17648329,126.9172286)(170.8850462,126.61978374)(171.40415905,126.02489931)
\curveto(171.92326137,125.43379342)(172.17144785,124.52630242)(172.14871922,123.30242358)
\lineto(168.14741878,123.30242358)
\curveto(168.15878409,122.82878271)(168.28761371,122.45934483)(168.53390802,122.19410883)
\curveto(168.78019755,121.93265903)(169.08711518,121.80193486)(169.45466182,121.80193592)
\curveto(169.70473953,121.80193486)(169.91503494,121.87013878)(170.08554868,122.00654787)
\curveto(170.25605453,122.14295444)(170.38488415,122.36272262)(170.47203793,122.66585306)
\moveto(170.56297657,124.28001403)
\curveto(170.55160483,124.74228148)(170.43224798,125.09277383)(170.20490565,125.33149213)
\curveto(169.9775552,125.57399036)(169.70095043,125.69524176)(169.3750905,125.69524671)
\curveto(169.02648947,125.69524176)(168.73851738,125.5683067)(168.51117336,125.31444113)
\curveto(168.2838246,125.06056643)(168.17204596,124.71575774)(168.1758371,124.28001403)
\lineto(170.56297657,124.28001403)
}
}
{
\newrgbcolor{curcolor}{0 0 0}
\pscustom[linestyle=none,fillstyle=solid,fillcolor=curcolor]
{
\newpath
\moveto(172.89327968,122.46692477)
\lineto(174.49607332,122.71132238)
\curveto(174.56427536,122.40061368)(174.70257775,122.16379453)(174.9109809,122.0008642)
\curveto(175.11937946,121.84172048)(175.41114066,121.76214924)(175.78626537,121.76215026)
\curveto(176.19927481,121.76214924)(176.50998154,121.83793137)(176.71838649,121.98949687)
\curveto(176.85857933,122.09559061)(176.9286778,122.2376821)(176.92868211,122.41577178)
\curveto(176.9286778,122.53702151)(176.89078674,122.63743284)(176.8150088,122.71700604)
\curveto(176.73543337,122.7927862)(176.55734537,122.86288467)(176.28074426,122.92730166)
\curveto(174.9924444,123.21148247)(174.17589195,123.47103626)(173.83108447,123.70596382)
\curveto(173.35365584,124.03182402)(173.11494214,124.48462224)(173.11494263,125.06435985)
\curveto(173.11494214,125.58725223)(173.32144844,126.02678858)(173.73446216,126.38297023)
\curveto(174.14747365,126.7391406)(174.78783265,126.9172286)(175.65554106,126.91723477)
\curveto(176.48156324,126.9172286)(177.09539849,126.78271532)(177.49704865,126.51369453)
\curveto(177.89868906,126.2446622)(178.17529384,125.84680602)(178.3268638,125.3201248)
\lineto(176.82069247,125.04162519)
\curveto(176.75627346,125.2765455)(176.6331275,125.45652806)(176.45125422,125.58157341)
\curveto(176.27316238,125.70660908)(176.0173977,125.76912934)(175.68395939,125.76913436)
\curveto(175.26336551,125.76912934)(174.96213154,125.71039819)(174.78025659,125.59294074)
\curveto(174.65900303,125.50957555)(174.59837732,125.40158601)(174.5983793,125.26897181)
\curveto(174.59837732,125.15529409)(174.65142481,125.05867187)(174.75752193,124.97910487)
\curveto(174.90150584,124.87300566)(175.39787879,124.72333595)(176.24664226,124.53009531)
\curveto(177.0991876,124.33684709)(177.69407731,124.10002794)(178.0313132,123.81963713)
\curveto(178.36474916,123.53545107)(178.53146984,123.13948944)(178.53147576,122.63175106)
\curveto(178.53146984,122.07853963)(178.30033435,121.60300677)(177.83806858,121.20515105)
\curveto(177.37579237,120.80729441)(176.69185865,120.60836632)(175.78626537,120.60836618)
\curveto(174.9640261,120.60836632)(174.31229978,120.775087)(173.83108447,121.10852874)
\curveto(173.35365584,121.44196974)(173.04105456,121.89476797)(172.89327968,122.46692477)
}
}
{
\newrgbcolor{curcolor}{0 1 0.25098041}
\pscustom[linewidth=2.32802935,linecolor=curcolor]
{
\newpath
\moveto(241.93390437,205.12868455)
\lineto(316.43084131,205.12868455)
\lineto(316.43084131,251.68927596)
\lineto(266.3782102,251.68927596)
\lineto(266.3782102,261.00139424)
\lineto(241.93390437,261.00139424)
\lineto(241.93390437,205.12868455)
\closepath
}
}
{
\newrgbcolor{curcolor}{0 0 0}
\pscustom[linestyle=none,fillstyle=solid,fillcolor=curcolor]
{
\newpath
\moveto(255.37462549,232.64138342)
\lineto(255.37462549,233.59623921)
\lineto(260.89346459,235.92654202)
\lineto(260.89346459,234.90916591)
\lineto(256.51704223,233.11312765)
\lineto(260.89346459,231.30003839)
\lineto(260.89346459,230.28266228)
\lineto(255.37462549,232.64138342)
}
}
{
\newrgbcolor{curcolor}{0 0 0}
\pscustom[linestyle=none,fillstyle=solid,fillcolor=curcolor]
{
\newpath
\moveto(262.17228948,232.64138342)
\lineto(262.17228948,233.59623921)
\lineto(267.69112858,235.92654202)
\lineto(267.69112858,234.90916591)
\lineto(263.31470622,233.11312765)
\lineto(267.69112858,231.30003839)
\lineto(267.69112858,230.28266228)
\lineto(262.17228948,232.64138342)
}
}
{
\newrgbcolor{curcolor}{0 0 0}
\pscustom[linestyle=none,fillstyle=solid,fillcolor=curcolor]
{
\newpath
\moveto(273.0394579,229.74271407)
\curveto(272.66054254,229.42063927)(272.29489377,229.19329288)(271.94251048,229.06067422)
\curveto(271.59390907,228.92805543)(271.21878753,228.86174607)(270.81714473,228.86174593)
\curveto(270.15404861,228.86174607)(269.64441379,229.02278309)(269.28823873,229.34485749)
\curveto(268.93206177,229.6707203)(268.75397377,230.08562746)(268.75397419,230.58958021)
\curveto(268.75397377,230.88512892)(268.82028313,231.15415548)(268.95290248,231.3966607)
\curveto(269.08930969,231.64295022)(269.26550314,231.83998376)(269.48148336,231.9877619)
\curveto(269.70125039,232.13553406)(269.94754231,232.2473127)(270.22035986,232.32309816)
\curveto(270.42118062,232.37614232)(270.72430913,232.42729526)(271.12974632,232.47655712)
\curveto(271.95576874,232.57507041)(272.56392033,232.69253271)(272.95420292,232.82894438)
\curveto(272.9579874,232.96913749)(272.95988196,233.05818149)(272.95988658,233.09607665)
\curveto(272.95988196,233.51287427)(272.86325974,233.80653002)(272.67001965,233.97704479)
\curveto(272.40856696,234.2081753)(272.02018355,234.32374305)(271.50486824,234.32374838)
\curveto(271.02364855,234.32374305)(270.66747254,234.23848816)(270.43633914,234.06798343)
\curveto(270.20899065,233.90125768)(270.04037542,233.60381282)(269.93049292,233.17564797)
\lineto(268.93016781,233.31205594)
\curveto(269.02110577,233.74022065)(269.17077548,234.08502934)(269.37917738,234.34648304)
\curveto(269.58757719,234.61171514)(269.88881116,234.81443234)(270.28288018,234.95463524)
\curveto(270.6769453,235.09861533)(271.13353263,235.17060835)(271.65264354,235.17061452)
\curveto(272.1679587,235.17060835)(272.58665497,235.10998265)(272.90873359,234.98873723)
\curveto(273.23080307,234.86747983)(273.46762222,234.71402102)(273.61919177,234.52836033)
\curveto(273.77075074,234.34647769)(273.87684572,234.1153422)(273.93747703,233.83495315)
\curveto(273.97157338,233.66064942)(273.98862436,233.34615358)(273.98863002,232.8914647)
\lineto(273.98863002,231.527385)
\curveto(273.98862436,230.57631675)(274.00946445,229.97384882)(274.05115034,229.7199794)
\curveto(274.0966139,229.46989766)(274.18376335,229.22928939)(274.31259895,228.9981539)
\lineto(273.24406985,228.9981539)
\curveto(273.13796996,229.21034386)(273.06976604,229.45853034)(273.0394579,229.74271407)
\moveto(272.95420292,232.02754756)
\curveto(272.58286586,231.87598027)(272.02586721,231.74715065)(271.28320529,231.64105831)
\curveto(270.86261152,231.58042996)(270.56516666,231.51222605)(270.39086982,231.43644636)
\curveto(270.21656887,231.36066179)(270.08205559,231.24888315)(269.98732958,231.1011101)
\curveto(269.89260026,230.95712195)(269.84523643,230.79608492)(269.84523794,230.61799854)
\curveto(269.84523643,230.34518125)(269.94754231,230.11783486)(270.15215587,229.93595869)
\curveto(270.36055491,229.75408064)(270.66368343,229.66314209)(271.06154234,229.66314275)
\curveto(271.45560668,229.66314209)(271.80609903,229.74839698)(272.11302044,229.91890769)
\curveto(272.41993428,230.09320567)(272.64538612,230.33002483)(272.78937662,230.62936587)
\curveto(272.89925625,230.86049973)(272.9541983,231.20151932)(272.95420292,231.65242564)
\lineto(272.95420292,232.02754756)
}
}
{
\newrgbcolor{curcolor}{0 0 0}
\pscustom[linestyle=none,fillstyle=solid,fillcolor=curcolor]
{
\newpath
\moveto(275.58005574,226.68490208)
\lineto(275.58005574,235.03420655)
\lineto(276.51217686,235.03420655)
\lineto(276.51217686,234.24986073)
\curveto(276.73194334,234.5567731)(276.98012981,234.78601404)(277.25673703,234.93758424)
\curveto(277.53333936,235.09293167)(277.86867528,235.17060835)(278.2627458,235.17061452)
\curveto(278.77806084,235.17060835)(279.23275361,235.03798962)(279.6268255,234.77275794)
\curveto(280.02088776,234.50751472)(280.31833262,234.13239317)(280.51916097,233.6473922)
\curveto(280.7199779,233.16617102)(280.82038923,232.63759067)(280.82039523,232.06164955)
\curveto(280.82038923,231.44402213)(280.70861059,230.88702348)(280.48505898,230.39065192)
\curveto(280.26528513,229.89806669)(279.94321108,229.51915604)(279.51883586,229.25391884)
\curveto(279.09824033,228.99247024)(278.65491488,228.86174607)(278.18885815,228.86174593)
\curveto(277.8478352,228.86174607)(277.54091757,228.93373909)(277.26810436,229.07772521)
\curveto(276.99907534,229.22171118)(276.77741262,229.40358829)(276.60311551,229.62335709)
\lineto(276.60311551,226.68490208)
\lineto(275.58005574,226.68490208)
\moveto(276.5064932,231.98207823)
\curveto(276.5064915,231.20530842)(276.66373942,230.63125879)(276.97823742,230.25992762)
\curveto(277.2927311,229.88859392)(277.6735363,229.7029277)(278.12065417,229.70292841)
\curveto(278.57534364,229.7029277)(278.96372705,229.89427758)(279.28580558,230.27697861)
\curveto(279.61166426,230.6634662)(279.77459584,231.26025047)(279.7746008,232.06733321)
\curveto(279.77459584,232.83651876)(279.61545337,233.41246294)(279.29717291,233.7951675)
\curveto(278.98267259,234.17786245)(278.60565649,234.36921233)(278.16612349,234.3692177)
\curveto(277.73037289,234.36921233)(277.34388403,234.16460058)(277.00665575,233.75538184)
\curveto(276.67321219,233.34994269)(276.5064915,232.75884208)(276.5064932,231.98207823)
}
}
{
\newrgbcolor{curcolor}{0 0 0}
\pscustom[linestyle=none,fillstyle=solid,fillcolor=curcolor]
{
\newpath
\moveto(282.05943459,226.68490208)
\lineto(282.05943459,235.03420655)
\lineto(282.99155571,235.03420655)
\lineto(282.99155571,234.24986073)
\curveto(283.21132219,234.5567731)(283.45950866,234.78601404)(283.73611588,234.93758424)
\curveto(284.01271821,235.09293167)(284.34805413,235.17060835)(284.74212465,235.17061452)
\curveto(285.25743968,235.17060835)(285.71213246,235.03798962)(286.10620435,234.77275794)
\curveto(286.50026661,234.50751472)(286.79771147,234.13239317)(286.99853982,233.6473922)
\curveto(287.19935675,233.16617102)(287.29976808,232.63759067)(287.29977408,232.06164955)
\curveto(287.29976808,231.44402213)(287.18798943,230.88702348)(286.96443782,230.39065192)
\curveto(286.74466398,229.89806669)(286.42258993,229.51915604)(285.99821471,229.25391884)
\curveto(285.57761918,228.99247024)(285.13429372,228.86174607)(284.668237,228.86174593)
\curveto(284.32721404,228.86174607)(284.02029642,228.93373909)(283.74748321,229.07772521)
\curveto(283.47845419,229.22171118)(283.25679147,229.40358829)(283.08249436,229.62335709)
\lineto(283.08249436,226.68490208)
\lineto(282.05943459,226.68490208)
\moveto(282.98587205,231.98207823)
\curveto(282.98587035,231.20530842)(283.14311827,230.63125879)(283.45761627,230.25992762)
\curveto(283.77210995,229.88859392)(284.15291515,229.7029277)(284.60003302,229.70292841)
\curveto(285.05472249,229.7029277)(285.4431059,229.89427758)(285.76518443,230.27697861)
\curveto(286.09104311,230.6634662)(286.25397469,231.26025047)(286.25397965,232.06733321)
\curveto(286.25397469,232.83651876)(286.09483222,233.41246294)(285.77655176,233.7951675)
\curveto(285.46205143,234.17786245)(285.08503534,234.36921233)(284.64550234,234.3692177)
\curveto(284.20975174,234.36921233)(283.82326288,234.16460058)(283.4860346,233.75538184)
\curveto(283.15259104,233.34994269)(282.98587035,232.75884208)(282.98587205,231.98207823)
}
}
{
\newrgbcolor{curcolor}{0 0 0}
\pscustom[linestyle=none,fillstyle=solid,fillcolor=curcolor]
{
\newpath
\moveto(293.92692823,232.64138342)
\lineto(288.40808913,230.28266228)
\lineto(288.40808913,231.30003839)
\lineto(292.77882782,233.11312765)
\lineto(288.40808913,234.90916591)
\lineto(288.40808913,235.92654202)
\lineto(293.92692823,233.59623921)
\lineto(293.92692823,232.64138342)
}
}
{
\newrgbcolor{curcolor}{0 0 0}
\pscustom[linestyle=none,fillstyle=solid,fillcolor=curcolor]
{
\newpath
\moveto(300.72459089,232.64138342)
\lineto(295.20575179,230.28266228)
\lineto(295.20575179,231.30003839)
\lineto(299.57649048,233.11312765)
\lineto(295.20575179,234.90916591)
\lineto(295.20575179,235.92654202)
\lineto(300.72459089,233.59623921)
\lineto(300.72459089,232.64138342)
}
}
{
\newrgbcolor{curcolor}{0 0 0}
\pscustom[linestyle=none,fillstyle=solid,fillcolor=curcolor]
{
\newpath
\moveto(263.72508446,213.45900293)
\lineto(265.31651077,213.19187065)
\curveto(265.1118931,212.6083466)(264.78792449,212.16312659)(264.34460399,211.85620928)
\curveto(263.90506269,211.55308045)(263.35374769,211.40151619)(262.69065736,211.40151605)
\curveto(261.64107157,211.40151619)(260.86430474,211.74443032)(260.36035454,212.43025949)
\curveto(259.9624974,212.97967904)(259.76356931,213.67308552)(259.76356968,214.51048103)
\curveto(259.76356931,215.51080216)(260.02501765,216.29325265)(260.5479155,216.85783483)
\curveto(261.07081104,217.42619549)(261.73201012,217.71037847)(262.53151472,217.71038464)
\curveto(263.42952982,217.71037847)(264.13809273,217.41293361)(264.65720558,216.81804918)
\curveto(265.17630791,216.22694329)(265.42449438,215.31945229)(265.40176575,214.09557345)
\lineto(261.40046531,214.09557345)
\curveto(261.41183062,213.62193258)(261.54066024,213.2524947)(261.78695456,212.9872587)
\curveto(262.03324409,212.7258089)(262.34016171,212.59508473)(262.70770835,212.59508579)
\curveto(262.95778607,212.59508473)(263.16808148,212.66328865)(263.33859521,212.79969774)
\curveto(263.50910106,212.93610431)(263.63793068,213.15587249)(263.72508446,213.45900293)
\moveto(263.8160231,215.0731639)
\curveto(263.80465136,215.53543135)(263.68529451,215.8859237)(263.45795218,216.124642)
\curveto(263.23060173,216.36714023)(262.95399696,216.48839163)(262.62813704,216.48839658)
\curveto(262.27953601,216.48839163)(261.99156391,216.36145657)(261.7642199,216.107591)
\curveto(261.53687114,215.8537163)(261.4250925,215.50890761)(261.42888364,215.0731639)
\lineto(263.8160231,215.0731639)
}
}
{
\newrgbcolor{curcolor}{0 0 0}
\pscustom[linestyle=none,fillstyle=solid,fillcolor=curcolor]
{
\newpath
\moveto(268.36863894,211.53792402)
\lineto(265.93603015,217.57397668)
\lineto(267.61271144,217.57397668)
\lineto(268.74944452,214.49343003)
\lineto(269.07909711,213.46468659)
\curveto(269.16624336,213.72613301)(269.2211854,213.89853736)(269.24392341,213.98190014)
\curveto(269.29696753,214.15240749)(269.35380413,214.32291728)(269.41443337,214.49343003)
\lineto(270.56253378,217.57397668)
\lineto(272.20511308,217.57397668)
\lineto(269.80660629,211.53792402)
\lineto(268.36863894,211.53792402)
}
}
{
\newrgbcolor{curcolor}{0 0 0}
\pscustom[linestyle=none,fillstyle=solid,fillcolor=curcolor]
{
\newpath
\moveto(276.70657557,213.45900293)
\lineto(278.29800188,213.19187065)
\curveto(278.09338421,212.6083466)(277.76941561,212.16312659)(277.3260951,211.85620928)
\curveto(276.8865538,211.55308045)(276.33523881,211.40151619)(275.67214847,211.40151605)
\curveto(274.62256268,211.40151619)(273.84579585,211.74443032)(273.34184566,212.43025949)
\curveto(272.94398851,212.97967904)(272.74506042,213.67308552)(272.74506079,214.51048103)
\curveto(272.74506042,215.51080216)(273.00650877,216.29325265)(273.52940661,216.85783483)
\curveto(274.05230215,217.42619549)(274.71350123,217.71037847)(275.51300584,217.71038464)
\curveto(276.41102094,217.71037847)(277.11958385,217.41293361)(277.6386967,216.81804918)
\curveto(278.15779902,216.22694329)(278.4059855,215.31945229)(278.38325686,214.09557345)
\lineto(274.38195642,214.09557345)
\curveto(274.39332174,213.62193258)(274.52215136,213.2524947)(274.76844567,212.9872587)
\curveto(275.0147352,212.7258089)(275.32165282,212.59508473)(275.68919947,212.59508579)
\curveto(275.93927718,212.59508473)(276.14957259,212.66328865)(276.32008633,212.79969774)
\curveto(276.49059217,212.93610431)(276.61942179,213.15587249)(276.70657557,213.45900293)
\moveto(276.79751422,215.0731639)
\curveto(276.78614248,215.53543135)(276.66678562,215.8859237)(276.4394433,216.124642)
\curveto(276.21209285,216.36714023)(275.93548807,216.48839163)(275.60962815,216.48839658)
\curveto(275.26102712,216.48839163)(274.97305503,216.36145657)(274.74571101,216.107591)
\curveto(274.51836225,215.8537163)(274.40658361,215.50890761)(274.41037475,215.0731639)
\lineto(276.79751422,215.0731639)
}
}
{
\newrgbcolor{curcolor}{0 0 0}
\pscustom[linestyle=none,fillstyle=solid,fillcolor=curcolor]
{
\newpath
\moveto(285.18092098,211.53792402)
\lineto(283.583811,211.53792402)
\lineto(283.583811,214.61847067)
\curveto(283.58380627,215.2701939)(283.54970431,215.69078472)(283.48150502,215.88024439)
\curveto(283.41329648,216.07348447)(283.30151784,216.22315418)(283.14616876,216.32925395)
\curveto(282.99460021,216.43534414)(282.81082855,216.48839163)(282.59485322,216.48839658)
\curveto(282.31824471,216.48839163)(282.07005823,216.4126095)(281.85029305,216.26104997)
\curveto(281.63052188,216.10948099)(281.47895762,215.90865834)(281.39559982,215.65858144)
\curveto(281.31602604,215.40849629)(281.27624043,214.9462253)(281.27624285,214.27176708)
\lineto(281.27624285,211.53792402)
\lineto(279.67913287,211.53792402)
\lineto(279.67913287,217.57397668)
\lineto(281.16256954,217.57397668)
\lineto(281.16256954,216.68732487)
\curveto(281.68925303,217.36935889)(282.35234667,217.71037847)(283.15185243,217.71038464)
\curveto(283.50423503,217.71037847)(283.82630909,217.64596366)(284.11807555,217.51714002)
\curveto(284.40983148,217.39209353)(284.62959966,217.2310565)(284.77738073,217.03402846)
\curveto(284.92893907,216.83698943)(285.0331395,216.61343215)(285.08998233,216.36335595)
\curveto(285.1506018,216.11327009)(285.18091465,215.75519953)(285.18092098,215.28914318)
\lineto(285.18092098,211.53792402)
}
}
{
\newrgbcolor{curcolor}{0 0 0}
\pscustom[linestyle=none,fillstyle=solid,fillcolor=curcolor]
{
\newpath
\moveto(289.5743935,217.57397668)
\lineto(289.5743935,216.30083563)
\lineto(288.48312974,216.30083563)
\lineto(288.48312974,213.86822684)
\curveto(288.48312723,213.37564066)(288.4926,213.08766857)(288.51154807,213.00430969)
\curveto(288.53428017,212.92473699)(288.581644,212.85842763)(288.65363971,212.80538141)
\curveto(288.72941915,212.75233265)(288.82035771,212.7258089)(288.92645564,212.72581009)
\curveto(289.07422784,212.7258089)(289.28831236,212.77696184)(289.56870983,212.87926906)
\lineto(289.7051178,211.64023)
\curveto(289.33378164,211.48108742)(288.91319082,211.40151619)(288.44334409,211.40151605)
\curveto(288.15536952,211.40151619)(287.89581573,211.44888002)(287.66468193,211.54360769)
\curveto(287.43354474,211.64212445)(287.26303495,211.76716496)(287.15315204,211.9187296)
\curveto(287.04705588,212.07408259)(286.9731683,212.28248344)(286.93148909,212.5439328)
\curveto(286.89738617,212.72959801)(286.88033519,213.10471955)(286.8803361,213.66929855)
\lineto(286.8803361,216.30083563)
\lineto(286.14714326,216.30083563)
\lineto(286.14714326,217.57397668)
\lineto(286.8803361,217.57397668)
\lineto(286.8803361,218.77323007)
\lineto(288.48312974,219.7053512)
\lineto(288.48312974,217.57397668)
\lineto(289.5743935,217.57397668)
}
}
{
\newrgbcolor{curcolor}{0 0 0}
\pscustom[linestyle=none,fillstyle=solid,fillcolor=curcolor]
{
\newpath
\moveto(290.13139466,213.26007464)
\lineto(291.73418831,213.50447225)
\curveto(291.80239035,213.19376355)(291.94069273,212.9569444)(292.14909588,212.79401408)
\curveto(292.35749444,212.63487035)(292.64925564,212.55529911)(293.02438035,212.55530013)
\curveto(293.43738979,212.55529911)(293.74809652,212.63108124)(293.95650148,212.78264674)
\curveto(294.09669432,212.88874048)(294.16679279,213.03083197)(294.1667971,213.20892165)
\curveto(294.16679279,213.33017139)(294.12890172,213.43058271)(294.05312379,213.51015592)
\curveto(293.97354836,213.58593607)(293.79546035,213.65603454)(293.51885924,213.72045153)
\curveto(292.23055938,214.00463234)(291.41400693,214.26418613)(291.06919945,214.49911369)
\curveto(290.59177083,214.82497389)(290.35305712,215.27777211)(290.35305761,215.85750972)
\curveto(290.35305712,216.3804021)(290.55956342,216.81993845)(290.97257714,217.1761201)
\curveto(291.38558863,217.53229047)(292.02594763,217.71037847)(292.89365605,217.71038464)
\curveto(293.71967822,217.71037847)(294.33351347,217.57586519)(294.73516364,217.3068444)
\curveto(295.13680405,217.03781207)(295.41340882,216.63995589)(295.56497878,216.11327467)
\lineto(294.05880745,215.83477506)
\curveto(293.99438844,216.06969537)(293.87124248,216.24967793)(293.6893692,216.37472328)
\curveto(293.51127737,216.49975895)(293.25551268,216.56227921)(292.92207437,216.56228423)
\curveto(292.50148049,216.56227921)(292.20024653,216.50354806)(292.01837158,216.38609061)
\curveto(291.89711801,216.30272542)(291.8364923,216.19473588)(291.83649428,216.06212168)
\curveto(291.8364923,215.94844396)(291.8895398,215.85182175)(291.99563691,215.77225474)
\curveto(292.13962082,215.66615553)(292.63599377,215.51648582)(293.48475725,215.32324518)
\curveto(294.33730258,215.12999696)(294.9321923,214.89317781)(295.26942818,214.612787)
\curveto(295.60286414,214.32860094)(295.76958483,213.93263932)(295.76959074,213.42490093)
\curveto(295.76958483,212.8716895)(295.53844933,212.39615664)(295.07618356,211.99830092)
\curveto(294.61390735,211.60044428)(293.92997363,211.40151619)(293.02438035,211.40151605)
\curveto(292.20214108,211.40151619)(291.55041477,211.56823687)(291.06919945,211.90167861)
\curveto(290.59177083,212.23511961)(290.27916954,212.68791784)(290.13139466,213.26007464)
}
}
{
\newrgbcolor{curcolor}{0 1 0.25098041}
\pscustom[linewidth=2.32802935,linecolor=curcolor]
{
\newpath
\moveto(18.44308422,22.37839586)
\lineto(92.94002116,22.37839586)
\lineto(92.94002116,68.93896864)
\lineto(44.0514067,68.93896864)
\lineto(44.0514067,78.25108692)
\lineto(18.44308422,78.25108692)
\lineto(18.44308422,22.37839586)
\closepath
}
}
{
\newrgbcolor{curcolor}{0 0 0}
\pscustom[linestyle=none,fillstyle=solid,fillcolor=curcolor]
{
\newpath
\moveto(31.88381527,49.89106456)
\lineto(31.88381527,50.84592035)
\lineto(37.40265437,53.17622316)
\lineto(37.40265437,52.15884706)
\lineto(33.02623201,50.36280879)
\lineto(37.40265437,48.54971953)
\lineto(37.40265437,47.53234342)
\lineto(31.88381527,49.89106456)
}
}
{
\newrgbcolor{curcolor}{0 0 0}
\pscustom[linestyle=none,fillstyle=solid,fillcolor=curcolor]
{
\newpath
\moveto(38.68147926,49.89106456)
\lineto(38.68147926,50.84592035)
\lineto(44.20031836,53.17622316)
\lineto(44.20031836,52.15884706)
\lineto(39.823896,50.36280879)
\lineto(44.20031836,48.54971953)
\lineto(44.20031836,47.53234342)
\lineto(38.68147926,49.89106456)
}
}
{
\newrgbcolor{curcolor}{0 0 0}
\pscustom[linestyle=none,fillstyle=solid,fillcolor=curcolor]
{
\newpath
\moveto(49.54864768,46.99239521)
\curveto(49.16973232,46.67032041)(48.80408355,46.44297403)(48.45170026,46.31035536)
\curveto(48.10309885,46.17773657)(47.72797731,46.11142721)(47.32633451,46.11142707)
\curveto(46.66323839,46.11142721)(46.15360357,46.27246423)(45.79742851,46.59453863)
\curveto(45.44125155,46.92040144)(45.26316355,47.3353086)(45.26316397,47.83926135)
\curveto(45.26316355,48.13481007)(45.32947291,48.40383663)(45.46209226,48.64634184)
\curveto(45.59849947,48.89263136)(45.77469292,49.0896649)(45.99067314,49.23744304)
\curveto(46.21044017,49.3852152)(46.45673209,49.49699385)(46.72954964,49.5727793)
\curveto(46.9303704,49.62582347)(47.23349891,49.6769764)(47.6389361,49.72623827)
\curveto(48.46495852,49.82475156)(49.07311011,49.94221386)(49.4633927,50.07862552)
\curveto(49.46717718,50.21881863)(49.46907174,50.30786263)(49.46907636,50.34575779)
\curveto(49.46907174,50.76255541)(49.37244952,51.05621116)(49.17920943,51.22672593)
\curveto(48.91775674,51.45785645)(48.52937333,51.57342419)(48.01405802,51.57342952)
\curveto(47.53283833,51.57342419)(47.17666232,51.4881693)(46.94552893,51.31766458)
\curveto(46.71818043,51.15093882)(46.5495652,50.85349396)(46.4396827,50.42532911)
\lineto(45.43935759,50.56173708)
\curveto(45.53029555,50.9899018)(45.67996526,51.33471049)(45.88836716,51.59616418)
\curveto(46.09676697,51.86139629)(46.39800094,52.06411348)(46.79206996,52.20431638)
\curveto(47.18613508,52.34829647)(47.64272241,52.42028949)(48.16183332,52.42029566)
\curveto(48.67714848,52.42028949)(49.09584475,52.35966379)(49.41792337,52.23841837)
\curveto(49.73999285,52.11716097)(49.976812,51.96370216)(50.12838155,51.77804147)
\curveto(50.27994052,51.59615883)(50.3860355,51.36502334)(50.44666681,51.0846343)
\curveto(50.48076316,50.91033056)(50.49781414,50.59583472)(50.4978198,50.14114584)
\lineto(50.4978198,48.77706614)
\curveto(50.49781414,47.82599789)(50.51865423,47.22352996)(50.56034012,46.96966055)
\curveto(50.60580368,46.7195788)(50.69295313,46.47897054)(50.82178873,46.24783504)
\lineto(49.75325963,46.24783504)
\curveto(49.64715974,46.460025)(49.57895582,46.70821148)(49.54864768,46.99239521)
\moveto(49.4633927,49.2772287)
\curveto(49.09205564,49.12566141)(48.53505699,48.99683179)(47.79239507,48.89073945)
\curveto(47.3718013,48.83011111)(47.07435644,48.76190719)(46.9000596,48.6861275)
\curveto(46.72575865,48.61034293)(46.59124537,48.49856429)(46.49651936,48.35079124)
\curveto(46.40179004,48.20680309)(46.35442621,48.04576607)(46.35442772,47.86767968)
\curveto(46.35442621,47.59486239)(46.45673209,47.36751601)(46.66134566,47.18563983)
\curveto(46.86974469,47.00376178)(47.17287321,46.91282323)(47.57073212,46.91282389)
\curveto(47.96479646,46.91282323)(48.31528881,46.99807812)(48.62221022,47.16858884)
\curveto(48.92912406,47.34288681)(49.1545759,47.57970597)(49.2985664,47.87904701)
\curveto(49.40844603,48.11018088)(49.46338808,48.45120046)(49.4633927,48.90210678)
\lineto(49.4633927,49.2772287)
}
}
{
\newrgbcolor{curcolor}{0 0 0}
\pscustom[linestyle=none,fillstyle=solid,fillcolor=curcolor]
{
\newpath
\moveto(52.08924552,43.93458323)
\lineto(52.08924552,52.2838877)
\lineto(53.02136664,52.2838877)
\lineto(53.02136664,51.49954187)
\curveto(53.24113312,51.80645424)(53.48931959,52.03569519)(53.76592681,52.18726538)
\curveto(54.04252914,52.34261281)(54.37786506,52.42028949)(54.77193559,52.42029566)
\curveto(55.28725062,52.42028949)(55.74194339,52.28767077)(56.13601528,52.02243909)
\curveto(56.53007754,51.75719586)(56.8275224,51.38207432)(57.02835075,50.89707334)
\curveto(57.22916769,50.41585217)(57.32957901,49.88727181)(57.32958501,49.31133069)
\curveto(57.32957901,48.69370327)(57.21780037,48.13670462)(56.99424876,47.64033306)
\curveto(56.77447491,47.14774783)(56.45240086,46.76883718)(56.02802564,46.50359999)
\curveto(55.60743011,46.24215138)(55.16410466,46.11142721)(54.69804793,46.11142707)
\curveto(54.35702498,46.11142721)(54.05010735,46.18342023)(53.77729414,46.32740636)
\curveto(53.50826513,46.47139232)(53.2866024,46.65326944)(53.11230529,46.87303824)
\lineto(53.11230529,43.93458323)
\lineto(52.08924552,43.93458323)
\moveto(53.01568298,49.23175938)
\curveto(53.01568128,48.45498956)(53.1729292,47.88093993)(53.48742721,47.50960876)
\curveto(53.80192088,47.13827506)(54.18272608,46.95260885)(54.62984395,46.95260955)
\curveto(55.08453342,46.95260885)(55.47291683,47.14395872)(55.79499536,47.52665976)
\curveto(56.12085404,47.91314734)(56.28378562,48.50993161)(56.28379058,49.31701436)
\curveto(56.28378562,50.0861999)(56.12464315,50.66214409)(55.80636269,51.04484864)
\curveto(55.49186237,51.4275436)(55.11484627,51.61889347)(54.67531327,51.61889884)
\curveto(54.23956268,51.61889347)(53.85307381,51.41428172)(53.51584553,51.00506298)
\curveto(53.18240197,50.59962383)(53.01568128,50.00852322)(53.01568298,49.23175938)
}
}
{
\newrgbcolor{curcolor}{0 0 0}
\pscustom[linestyle=none,fillstyle=solid,fillcolor=curcolor]
{
\newpath
\moveto(58.56862437,43.93458323)
\lineto(58.56862437,52.2838877)
\lineto(59.50074549,52.2838877)
\lineto(59.50074549,51.49954187)
\curveto(59.72051197,51.80645424)(59.96869844,52.03569519)(60.24530566,52.18726538)
\curveto(60.52190799,52.34261281)(60.85724391,52.42028949)(61.25131443,52.42029566)
\curveto(61.76662947,52.42028949)(62.22132224,52.28767077)(62.61539413,52.02243909)
\curveto(63.00945639,51.75719586)(63.30690125,51.38207432)(63.5077296,50.89707334)
\curveto(63.70854653,50.41585217)(63.80895786,49.88727181)(63.80896386,49.31133069)
\curveto(63.80895786,48.69370327)(63.69717921,48.13670462)(63.47362761,47.64033306)
\curveto(63.25385376,47.14774783)(62.93177971,46.76883718)(62.50740449,46.50359999)
\curveto(62.08680896,46.24215138)(61.6434835,46.11142721)(61.17742678,46.11142707)
\curveto(60.83640383,46.11142721)(60.5294862,46.18342023)(60.25667299,46.32740636)
\curveto(59.98764397,46.47139232)(59.76598125,46.65326944)(59.59168414,46.87303824)
\lineto(59.59168414,43.93458323)
\lineto(58.56862437,43.93458323)
\moveto(59.49506183,49.23175938)
\curveto(59.49506013,48.45498956)(59.65230805,47.88093993)(59.96680605,47.50960876)
\curveto(60.28129973,47.13827506)(60.66210493,46.95260885)(61.1092228,46.95260955)
\curveto(61.56391227,46.95260885)(61.95229568,47.14395872)(62.27437421,47.52665976)
\curveto(62.60023289,47.91314734)(62.76316447,48.50993161)(62.76316943,49.31701436)
\curveto(62.76316447,50.0861999)(62.604022,50.66214409)(62.28574154,51.04484864)
\curveto(61.97124121,51.4275436)(61.59422512,51.61889347)(61.15469212,51.61889884)
\curveto(60.71894152,51.61889347)(60.33245266,51.41428172)(59.99522438,51.00506298)
\curveto(59.66178082,50.59962383)(59.49506013,50.00852322)(59.49506183,49.23175938)
}
}
{
\newrgbcolor{curcolor}{0 0 0}
\pscustom[linestyle=none,fillstyle=solid,fillcolor=curcolor]
{
\newpath
\moveto(70.43611801,49.89106456)
\lineto(64.91727891,47.53234342)
\lineto(64.91727891,48.54971953)
\lineto(69.2880176,50.36280879)
\lineto(64.91727891,52.15884706)
\lineto(64.91727891,53.17622316)
\lineto(70.43611801,50.84592035)
\lineto(70.43611801,49.89106456)
}
}
{
\newrgbcolor{curcolor}{0 0 0}
\pscustom[linestyle=none,fillstyle=solid,fillcolor=curcolor]
{
\newpath
\moveto(77.23378067,49.89106456)
\lineto(71.71494157,47.53234342)
\lineto(71.71494157,48.54971953)
\lineto(76.08568026,50.36280879)
\lineto(71.71494157,52.15884706)
\lineto(71.71494157,53.17622316)
\lineto(77.23378067,50.84592035)
\lineto(77.23378067,49.89106456)
}
}
{
\newrgbcolor{curcolor}{0 0 0}
\pscustom[linestyle=none,fillstyle=solid,fillcolor=curcolor]
{
\newpath
\moveto(37.85734849,34.82371468)
\lineto(39.34646882,34.82371468)
\lineto(39.34646882,33.93706288)
\curveto(39.53971097,34.24018625)(39.80115932,34.48647817)(40.13081465,34.67593938)
\curveto(40.46046385,34.86538882)(40.82611262,34.96011648)(41.22776207,34.96012265)
\curveto(41.92874261,34.96011648)(42.52363232,34.68540626)(43.012433,34.13599117)
\curveto(43.50122179,33.58656538)(43.74561916,32.82116587)(43.74562584,31.83979035)
\curveto(43.74561916,30.83188497)(43.49932724,30.04753993)(43.00674934,29.48675287)
\curveto(42.51415956,28.92975352)(41.91737529,28.6512542)(41.21639474,28.65125406)
\curveto(40.88294922,28.6512542)(40.5798207,28.71756356)(40.30700827,28.85018235)
\curveto(40.03797847,28.98280101)(39.75379549,29.2101474)(39.45445846,29.5322222)
\lineto(39.45445846,26.49146121)
\lineto(37.85734849,26.49146121)
\lineto(37.85734849,34.82371468)
\moveto(39.43740747,31.90799433)
\curveto(39.4374051,31.22974115)(39.57191838,30.72768455)(39.84094771,30.401823)
\curveto(40.1099715,30.07974734)(40.43772921,29.91871031)(40.82422183,29.91871144)
\curveto(41.1955505,29.91871031)(41.50436268,30.06648547)(41.75065929,30.36203734)
\curveto(41.99694652,30.66137518)(42.12009248,31.15016992)(42.12009754,31.82842302)
\curveto(42.12009248,32.46120076)(41.99315742,32.93104996)(41.73929195,33.23797204)
\curveto(41.48541715,33.54488521)(41.17092131,33.69834402)(40.7958035,33.69834893)
\curveto(40.4055218,33.69834402)(40.0815532,33.54677976)(39.82389672,33.2436557)
\curveto(39.56623472,32.94431183)(39.4374051,32.49909182)(39.43740747,31.90799433)
}
}
{
\newrgbcolor{curcolor}{0 0 0}
\pscustom[linestyle=none,fillstyle=solid,fillcolor=curcolor]
{
\newpath
\moveto(46.61019326,37.1199155)
\lineto(46.61019326,34.05641985)
\curveto(47.12550931,34.65888251)(47.74123912,34.96011648)(48.45738451,34.96012265)
\curveto(48.82492357,34.96011648)(49.15647039,34.89191256)(49.45202596,34.7555107)
\curveto(49.747571,34.6190969)(49.96923372,34.444798)(50.11701481,34.23261348)
\curveto(50.26857314,34.02041807)(50.37087901,33.78549347)(50.42393274,33.52783897)
\curveto(50.4807631,33.27017499)(50.5091814,32.87042426)(50.50918772,32.32858557)
\lineto(50.50918772,28.78766203)
\lineto(48.91207775,28.78766203)
\lineto(48.91207775,31.97619832)
\curveto(48.91207302,32.60897591)(48.88176017,33.0106212)(48.8211391,33.18113538)
\curveto(48.76050876,33.35164078)(48.65251922,33.48615406)(48.49717017,33.58467563)
\curveto(48.3456016,33.6869767)(48.15425172,33.73812964)(47.92311997,33.73813459)
\curveto(47.65787877,33.73812964)(47.42105962,33.67371483)(47.21266179,33.54488997)
\curveto(47.00425791,33.41605559)(46.85079909,33.22091661)(46.7522849,32.95947243)
\curveto(46.65755466,32.70180902)(46.61019083,32.31910927)(46.61019326,31.81137202)
\lineto(46.61019326,28.78766203)
\lineto(45.01308328,28.78766203)
\lineto(45.01308328,37.1199155)
\lineto(46.61019326,37.1199155)
}
}
{
\newrgbcolor{curcolor}{0 0 0}
\pscustom[linestyle=none,fillstyle=solid,fillcolor=curcolor]
{
\newpath
\moveto(51.76527784,31.89094334)
\curveto(51.76527737,32.42141514)(51.89600154,32.93483907)(52.15745075,33.43121666)
\curveto(52.41889824,33.92758496)(52.78833612,34.30649561)(53.2657655,34.56794974)
\curveto(53.74698006,34.82939231)(54.28313862,34.96011648)(54.87424281,34.96012265)
\curveto(55.78741389,34.96011648)(56.53576242,34.66267162)(57.11929064,34.06778718)
\curveto(57.70280722,33.47668129)(57.99456842,32.72833277)(57.99457511,31.82273935)
\curveto(57.99456842,30.90956166)(57.69901811,30.15174036)(57.10792331,29.54927319)
\curveto(56.520606,28.95059361)(55.77983568,28.6512542)(54.88561014,28.65125406)
\curveto(54.33239701,28.6512542)(53.80381665,28.77629471)(53.29986749,29.02637598)
\curveto(52.79970344,29.27645677)(52.41889824,29.64210554)(52.15745075,30.1233234)
\curveto(51.89600154,30.60832769)(51.76527737,31.19753375)(51.76527784,31.89094334)
\moveto(53.40217347,31.80568836)
\curveto(53.40217137,31.20700651)(53.54426286,30.74852463)(53.82844838,30.43024133)
\curveto(54.11262883,30.11195474)(54.46312118,29.95281227)(54.87992647,29.95281344)
\curveto(55.29672461,29.95281227)(55.6453224,30.11195474)(55.92572091,30.43024133)
\curveto(56.20989927,30.74852463)(56.35199076,31.21079562)(56.35199581,31.81705569)
\curveto(56.35199076,32.40815327)(56.20989927,32.86284605)(55.92572091,33.18113538)
\curveto(55.6453224,33.49941593)(55.29672461,33.65855841)(54.87992647,33.65856328)
\curveto(54.46312118,33.65855841)(54.11262883,33.49941593)(53.82844838,33.18113538)
\curveto(53.54426286,32.86284605)(53.40217137,32.40436416)(53.40217347,31.80568836)
}
}
{
\newrgbcolor{curcolor}{0 0 0}
\pscustom[linestyle=none,fillstyle=solid,fillcolor=curcolor]
{
\newpath
\moveto(62.01861116,34.82371468)
\lineto(62.01861116,33.55057363)
\lineto(60.92734741,33.55057363)
\lineto(60.92734741,31.11796484)
\curveto(60.92734489,30.62537867)(60.93681766,30.33740658)(60.95576573,30.2540477)
\curveto(60.97849783,30.174475)(61.02586166,30.10816564)(61.09785737,30.05511941)
\curveto(61.17363681,30.00207066)(61.26457537,29.97554691)(61.37067331,29.9755481)
\curveto(61.5184455,29.97554691)(61.73253002,30.02669985)(62.0129275,30.12900706)
\lineto(62.14933547,28.88996801)
\curveto(61.7779993,28.73082543)(61.35740848,28.6512542)(60.88756175,28.65125406)
\curveto(60.59958718,28.6512542)(60.34003339,28.69861803)(60.10889959,28.7933457)
\curveto(59.8777624,28.89186246)(59.70725261,29.01690297)(59.5973697,29.16846761)
\curveto(59.49127354,29.3238206)(59.41738596,29.53222145)(59.37570675,29.79367081)
\curveto(59.34160383,29.97933602)(59.32455285,30.35445756)(59.32455376,30.91903655)
\lineto(59.32455376,33.55057363)
\lineto(58.59136093,33.55057363)
\lineto(58.59136093,34.82371468)
\lineto(59.32455376,34.82371468)
\lineto(59.32455376,36.02296808)
\lineto(60.92734741,36.95508921)
\lineto(60.92734741,34.82371468)
\lineto(62.01861116,34.82371468)
}
}
{
\newrgbcolor{curcolor}{0 0 0}
\pscustom[linestyle=none,fillstyle=solid,fillcolor=curcolor]
{
\newpath
\moveto(62.76885339,31.89094334)
\curveto(62.76885293,32.42141514)(62.8995771,32.93483907)(63.16102631,33.43121666)
\curveto(63.4224738,33.92758496)(63.79191168,34.30649561)(64.26934106,34.56794974)
\curveto(64.75055562,34.82939231)(65.28671418,34.96011648)(65.87781837,34.96012265)
\curveto(66.79098945,34.96011648)(67.53933798,34.66267162)(68.1228662,34.06778718)
\curveto(68.70638278,33.47668129)(68.99814398,32.72833277)(68.99815067,31.82273935)
\curveto(68.99814398,30.90956166)(68.70259367,30.15174036)(68.11149887,29.54927319)
\curveto(67.52418156,28.95059361)(66.78341124,28.6512542)(65.8891857,28.65125406)
\curveto(65.33597257,28.6512542)(64.80739221,28.77629471)(64.30344305,29.02637598)
\curveto(63.803279,29.27645677)(63.4224738,29.64210554)(63.16102631,30.1233234)
\curveto(62.8995771,30.60832769)(62.76885293,31.19753375)(62.76885339,31.89094334)
\moveto(64.40574903,31.80568836)
\curveto(64.40574693,31.20700651)(64.54783842,30.74852463)(64.83202393,30.43024133)
\curveto(65.11620439,30.11195474)(65.46669674,29.95281227)(65.88350203,29.95281344)
\curveto(66.30030016,29.95281227)(66.64889796,30.11195474)(66.92929647,30.43024133)
\curveto(67.21347482,30.74852463)(67.35556632,31.21079562)(67.35557137,31.81705569)
\curveto(67.35556632,32.40815327)(67.21347482,32.86284605)(66.92929647,33.18113538)
\curveto(66.64889796,33.49941593)(66.30030016,33.65855841)(65.88350203,33.65856328)
\curveto(65.46669674,33.65855841)(65.11620439,33.49941593)(64.83202393,33.18113538)
\curveto(64.54783842,32.86284605)(64.40574693,32.40436416)(64.40574903,31.80568836)
}
}
{
\newrgbcolor{curcolor}{0 0 0}
\pscustom[linestyle=none,fillstyle=solid,fillcolor=curcolor]
{
\newpath
\moveto(69.69155702,30.50981265)
\lineto(71.29435066,30.75421026)
\curveto(71.3625527,30.44350156)(71.50085509,30.20668241)(71.70925824,30.04375208)
\curveto(71.9176568,29.88460835)(72.209418,29.80503712)(72.58454271,29.80503814)
\curveto(72.99755215,29.80503712)(73.30825888,29.88081925)(73.51666383,30.03238475)
\curveto(73.65685667,30.13847849)(73.72695514,30.28056998)(73.72695945,30.45865966)
\curveto(73.72695514,30.57990939)(73.68906408,30.68032071)(73.61328614,30.75989392)
\curveto(73.53371071,30.83567408)(73.35562271,30.90577255)(73.0790216,30.97018954)
\curveto(71.79072173,31.25437035)(70.97416929,31.51392414)(70.62936181,31.7488517)
\curveto(70.15193318,32.0747119)(69.91321948,32.52751012)(69.91321997,33.10724773)
\curveto(69.91321948,33.63014011)(70.11972578,34.06967646)(70.5327395,34.4258581)
\curveto(70.94575099,34.78202848)(71.58610999,34.96011648)(72.4538184,34.96012265)
\curveto(73.27984058,34.96011648)(73.89367583,34.8256032)(74.29532599,34.55658241)
\curveto(74.6969664,34.28755008)(74.97357118,33.8896939)(75.12514114,33.36301268)
\lineto(73.61896981,33.08451307)
\curveto(73.5545508,33.31943338)(73.43140484,33.49941593)(73.24953156,33.62446128)
\curveto(73.07143972,33.74949696)(72.81567504,33.81201722)(72.48223673,33.81202224)
\curveto(72.06164285,33.81201722)(71.76040888,33.75328607)(71.57853393,33.63582861)
\curveto(71.45728037,33.55246342)(71.39665466,33.44447389)(71.39665664,33.31185969)
\curveto(71.39665466,33.19818197)(71.44970215,33.10155975)(71.55579927,33.02199275)
\curveto(71.69978318,32.91589354)(72.19615613,32.76622383)(73.0449196,32.57298318)
\curveto(73.89746494,32.37973497)(74.49235465,32.14291581)(74.82959054,31.86252501)
\curveto(75.1630265,31.57833895)(75.32974718,31.18237732)(75.32975309,30.67463894)
\curveto(75.32974718,30.12142751)(75.09861169,29.64589465)(74.63634592,29.24803893)
\curveto(74.17406971,28.85018229)(73.49013599,28.6512542)(72.58454271,28.65125406)
\curveto(71.76230344,28.6512542)(71.11057712,28.81797488)(70.62936181,29.15141662)
\curveto(70.15193318,29.48485762)(69.8393319,29.93765585)(69.69155702,30.50981265)
}
}
{
\newrgbcolor{curcolor}{0 1 0.25098041}
\pscustom[linewidth=2.32802935,linecolor=curcolor]
{
\newpath
\moveto(241.93390437,22.37839586)
\lineto(316.43084131,22.37839586)
\lineto(316.43084131,68.93896864)
\lineto(266.3782102,68.93896864)
\lineto(266.3782102,78.25108692)
\lineto(241.93390437,78.25108692)
\lineto(241.93390437,22.37839586)
\closepath
}
}
{
\newrgbcolor{curcolor}{0 0 0}
\pscustom[linestyle=none,fillstyle=solid,fillcolor=curcolor]
{
\newpath
\moveto(255.37462549,49.89106456)
\lineto(255.37462549,50.84592035)
\lineto(260.89346459,53.17622316)
\lineto(260.89346459,52.15884706)
\lineto(256.51704223,50.36280879)
\lineto(260.89346459,48.54971953)
\lineto(260.89346459,47.53234342)
\lineto(255.37462549,49.89106456)
}
}
{
\newrgbcolor{curcolor}{0 0 0}
\pscustom[linestyle=none,fillstyle=solid,fillcolor=curcolor]
{
\newpath
\moveto(262.17228948,49.89106456)
\lineto(262.17228948,50.84592035)
\lineto(267.69112858,53.17622316)
\lineto(267.69112858,52.15884706)
\lineto(263.31470622,50.36280879)
\lineto(267.69112858,48.54971953)
\lineto(267.69112858,47.53234342)
\lineto(262.17228948,49.89106456)
}
}
{
\newrgbcolor{curcolor}{0 0 0}
\pscustom[linestyle=none,fillstyle=solid,fillcolor=curcolor]
{
\newpath
\moveto(273.0394579,46.99239521)
\curveto(272.66054254,46.67032041)(272.29489377,46.44297403)(271.94251048,46.31035536)
\curveto(271.59390907,46.17773657)(271.21878753,46.11142721)(270.81714473,46.11142707)
\curveto(270.15404861,46.11142721)(269.64441379,46.27246423)(269.28823873,46.59453863)
\curveto(268.93206177,46.92040144)(268.75397377,47.3353086)(268.75397419,47.83926135)
\curveto(268.75397377,48.13481007)(268.82028313,48.40383663)(268.95290248,48.64634184)
\curveto(269.08930969,48.89263136)(269.26550314,49.0896649)(269.48148336,49.23744304)
\curveto(269.70125039,49.3852152)(269.94754231,49.49699385)(270.22035986,49.5727793)
\curveto(270.42118062,49.62582347)(270.72430913,49.6769764)(271.12974632,49.72623827)
\curveto(271.95576874,49.82475156)(272.56392033,49.94221386)(272.95420292,50.07862552)
\curveto(272.9579874,50.21881863)(272.95988196,50.30786263)(272.95988658,50.34575779)
\curveto(272.95988196,50.76255541)(272.86325974,51.05621116)(272.67001965,51.22672593)
\curveto(272.40856696,51.45785645)(272.02018355,51.57342419)(271.50486824,51.57342952)
\curveto(271.02364855,51.57342419)(270.66747254,51.4881693)(270.43633914,51.31766458)
\curveto(270.20899065,51.15093882)(270.04037542,50.85349396)(269.93049292,50.42532911)
\lineto(268.93016781,50.56173708)
\curveto(269.02110577,50.9899018)(269.17077548,51.33471049)(269.37917738,51.59616418)
\curveto(269.58757719,51.86139629)(269.88881116,52.06411348)(270.28288018,52.20431638)
\curveto(270.6769453,52.34829647)(271.13353263,52.42028949)(271.65264354,52.42029566)
\curveto(272.1679587,52.42028949)(272.58665497,52.35966379)(272.90873359,52.23841837)
\curveto(273.23080307,52.11716097)(273.46762222,51.96370216)(273.61919177,51.77804147)
\curveto(273.77075074,51.59615883)(273.87684572,51.36502334)(273.93747703,51.0846343)
\curveto(273.97157338,50.91033056)(273.98862436,50.59583472)(273.98863002,50.14114584)
\lineto(273.98863002,48.77706614)
\curveto(273.98862436,47.82599789)(274.00946445,47.22352996)(274.05115034,46.96966055)
\curveto(274.0966139,46.7195788)(274.18376335,46.47897054)(274.31259895,46.24783504)
\lineto(273.24406985,46.24783504)
\curveto(273.13796996,46.460025)(273.06976604,46.70821148)(273.0394579,46.99239521)
\moveto(272.95420292,49.2772287)
\curveto(272.58286586,49.12566141)(272.02586721,48.99683179)(271.28320529,48.89073945)
\curveto(270.86261152,48.83011111)(270.56516666,48.76190719)(270.39086982,48.6861275)
\curveto(270.21656887,48.61034293)(270.08205559,48.49856429)(269.98732958,48.35079124)
\curveto(269.89260026,48.20680309)(269.84523643,48.04576607)(269.84523794,47.86767968)
\curveto(269.84523643,47.59486239)(269.94754231,47.36751601)(270.15215587,47.18563983)
\curveto(270.36055491,47.00376178)(270.66368343,46.91282323)(271.06154234,46.91282389)
\curveto(271.45560668,46.91282323)(271.80609903,46.99807812)(272.11302044,47.16858884)
\curveto(272.41993428,47.34288681)(272.64538612,47.57970597)(272.78937662,47.87904701)
\curveto(272.89925625,48.11018088)(272.9541983,48.45120046)(272.95420292,48.90210678)
\lineto(272.95420292,49.2772287)
}
}
{
\newrgbcolor{curcolor}{0 0 0}
\pscustom[linestyle=none,fillstyle=solid,fillcolor=curcolor]
{
\newpath
\moveto(275.58005574,43.93458323)
\lineto(275.58005574,52.2838877)
\lineto(276.51217686,52.2838877)
\lineto(276.51217686,51.49954187)
\curveto(276.73194334,51.80645424)(276.98012981,52.03569519)(277.25673703,52.18726538)
\curveto(277.53333936,52.34261281)(277.86867528,52.42028949)(278.2627458,52.42029566)
\curveto(278.77806084,52.42028949)(279.23275361,52.28767077)(279.6268255,52.02243909)
\curveto(280.02088776,51.75719586)(280.31833262,51.38207432)(280.51916097,50.89707334)
\curveto(280.7199779,50.41585217)(280.82038923,49.88727181)(280.82039523,49.31133069)
\curveto(280.82038923,48.69370327)(280.70861059,48.13670462)(280.48505898,47.64033306)
\curveto(280.26528513,47.14774783)(279.94321108,46.76883718)(279.51883586,46.50359999)
\curveto(279.09824033,46.24215138)(278.65491488,46.11142721)(278.18885815,46.11142707)
\curveto(277.8478352,46.11142721)(277.54091757,46.18342023)(277.26810436,46.32740636)
\curveto(276.99907534,46.47139232)(276.77741262,46.65326944)(276.60311551,46.87303824)
\lineto(276.60311551,43.93458323)
\lineto(275.58005574,43.93458323)
\moveto(276.5064932,49.23175938)
\curveto(276.5064915,48.45498956)(276.66373942,47.88093993)(276.97823742,47.50960876)
\curveto(277.2927311,47.13827506)(277.6735363,46.95260885)(278.12065417,46.95260955)
\curveto(278.57534364,46.95260885)(278.96372705,47.14395872)(279.28580558,47.52665976)
\curveto(279.61166426,47.91314734)(279.77459584,48.50993161)(279.7746008,49.31701436)
\curveto(279.77459584,50.0861999)(279.61545337,50.66214409)(279.29717291,51.04484864)
\curveto(278.98267259,51.4275436)(278.60565649,51.61889347)(278.16612349,51.61889884)
\curveto(277.73037289,51.61889347)(277.34388403,51.41428172)(277.00665575,51.00506298)
\curveto(276.67321219,50.59962383)(276.5064915,50.00852322)(276.5064932,49.23175938)
}
}
{
\newrgbcolor{curcolor}{0 0 0}
\pscustom[linestyle=none,fillstyle=solid,fillcolor=curcolor]
{
\newpath
\moveto(282.05943459,43.93458323)
\lineto(282.05943459,52.2838877)
\lineto(282.99155571,52.2838877)
\lineto(282.99155571,51.49954187)
\curveto(283.21132219,51.80645424)(283.45950866,52.03569519)(283.73611588,52.18726538)
\curveto(284.01271821,52.34261281)(284.34805413,52.42028949)(284.74212465,52.42029566)
\curveto(285.25743968,52.42028949)(285.71213246,52.28767077)(286.10620435,52.02243909)
\curveto(286.50026661,51.75719586)(286.79771147,51.38207432)(286.99853982,50.89707334)
\curveto(287.19935675,50.41585217)(287.29976808,49.88727181)(287.29977408,49.31133069)
\curveto(287.29976808,48.69370327)(287.18798943,48.13670462)(286.96443782,47.64033306)
\curveto(286.74466398,47.14774783)(286.42258993,46.76883718)(285.99821471,46.50359999)
\curveto(285.57761918,46.24215138)(285.13429372,46.11142721)(284.668237,46.11142707)
\curveto(284.32721404,46.11142721)(284.02029642,46.18342023)(283.74748321,46.32740636)
\curveto(283.47845419,46.47139232)(283.25679147,46.65326944)(283.08249436,46.87303824)
\lineto(283.08249436,43.93458323)
\lineto(282.05943459,43.93458323)
\moveto(282.98587205,49.23175938)
\curveto(282.98587035,48.45498956)(283.14311827,47.88093993)(283.45761627,47.50960876)
\curveto(283.77210995,47.13827506)(284.15291515,46.95260885)(284.60003302,46.95260955)
\curveto(285.05472249,46.95260885)(285.4431059,47.14395872)(285.76518443,47.52665976)
\curveto(286.09104311,47.91314734)(286.25397469,48.50993161)(286.25397965,49.31701436)
\curveto(286.25397469,50.0861999)(286.09483222,50.66214409)(285.77655176,51.04484864)
\curveto(285.46205143,51.4275436)(285.08503534,51.61889347)(284.64550234,51.61889884)
\curveto(284.20975174,51.61889347)(283.82326288,51.41428172)(283.4860346,51.00506298)
\curveto(283.15259104,50.59962383)(282.98587035,50.00852322)(282.98587205,49.23175938)
}
}
{
\newrgbcolor{curcolor}{0 0 0}
\pscustom[linestyle=none,fillstyle=solid,fillcolor=curcolor]
{
\newpath
\moveto(293.92692823,49.89106456)
\lineto(288.40808913,47.53234342)
\lineto(288.40808913,48.54971953)
\lineto(292.77882782,50.36280879)
\lineto(288.40808913,52.15884706)
\lineto(288.40808913,53.17622316)
\lineto(293.92692823,50.84592035)
\lineto(293.92692823,49.89106456)
}
}
{
\newrgbcolor{curcolor}{0 0 0}
\pscustom[linestyle=none,fillstyle=solid,fillcolor=curcolor]
{
\newpath
\moveto(300.72459089,49.89106456)
\lineto(295.20575179,47.53234342)
\lineto(295.20575179,48.54971953)
\lineto(299.57649048,50.36280879)
\lineto(295.20575179,52.15884706)
\lineto(295.20575179,53.17622316)
\lineto(300.72459089,50.84592035)
\lineto(300.72459089,49.89106456)
}
}
{
\newrgbcolor{curcolor}{0 0 0}
\pscustom[linestyle=none,fillstyle=solid,fillcolor=curcolor]
{
\newpath
\moveto(261.88926053,28.78766203)
\lineto(259.45665174,34.82371468)
\lineto(261.13333304,34.82371468)
\lineto(262.27006612,31.74316804)
\lineto(262.59971871,30.7144246)
\curveto(262.68686495,30.97587102)(262.741807,31.14827536)(262.76454501,31.23163815)
\curveto(262.81758913,31.4021455)(262.87442572,31.57265529)(262.93505497,31.74316804)
\lineto(264.08315538,34.82371468)
\lineto(265.72573468,34.82371468)
\lineto(263.32722788,28.78766203)
\lineto(261.88926053,28.78766203)
}
}
{
\newrgbcolor{curcolor}{0 0 0}
\pscustom[linestyle=none,fillstyle=solid,fillcolor=curcolor]
{
\newpath
\moveto(266.73174339,35.6421625)
\lineto(266.73174339,37.1199155)
\lineto(268.32885337,37.1199155)
\lineto(268.32885337,35.6421625)
\lineto(266.73174339,35.6421625)
\moveto(266.73174339,28.78766203)
\lineto(266.73174339,34.82371468)
\lineto(268.32885337,34.82371468)
\lineto(268.32885337,28.78766203)
\lineto(266.73174339,28.78766203)
}
}
{
\newrgbcolor{curcolor}{0 0 0}
\pscustom[linestyle=none,fillstyle=solid,fillcolor=curcolor]
{
\newpath
\moveto(275.49595532,28.78766203)
\lineto(274.01251865,28.78766203)
\lineto(274.01251865,29.67431383)
\curveto(273.76622184,29.32950426)(273.47446064,29.07184502)(273.13723418,28.90133534)
\curveto(272.8037888,28.73461454)(272.46655832,28.6512542)(272.12554174,28.65125406)
\curveto(271.43213225,28.6512542)(270.83724253,28.92975352)(270.3408708,29.48675287)
\curveto(269.84828574,30.04753993)(269.60199382,30.82809587)(269.6019943,31.82842302)
\curveto(269.60199382,32.85147873)(269.84260208,33.62824555)(270.32381981,34.15872583)
\curveto(270.80503513,34.69298447)(271.41318672,34.96011648)(272.1482764,34.96012265)
\curveto(272.82273433,34.96011648)(273.40625672,34.6797226)(273.89884534,34.11894017)
\lineto(273.89884534,37.1199155)
\lineto(275.49595532,37.1199155)
\lineto(275.49595532,28.78766203)
\moveto(271.23320627,31.93641266)
\curveto(271.23320416,31.29226141)(271.32224816,30.82620131)(271.50033854,30.53823097)
\curveto(271.75799541,30.12142751)(272.11796052,29.91302665)(272.58023497,29.91302778)
\curveto(272.94777484,29.91302665)(273.26037613,30.06838002)(273.51803976,30.37908834)
\curveto(273.77569461,30.69358259)(273.90452423,31.16153724)(273.90452901,31.78295369)
\curveto(273.90452423,32.47635718)(273.77948371,32.97462469)(273.52940709,33.27775769)
\curveto(273.27932166,33.58467083)(272.95914216,33.73812964)(272.56886764,33.73813459)
\curveto(272.18995355,33.73812964)(271.8716686,33.58656538)(271.61401185,33.28344136)
\curveto(271.36013923,32.98409745)(271.23320416,32.53508833)(271.23320627,31.93641266)
}
}
{
\newrgbcolor{curcolor}{0 0 0}
\pscustom[linestyle=none,fillstyle=solid,fillcolor=curcolor]
{
\newpath
\moveto(280.57146947,30.70874093)
\lineto(282.16289578,30.44160866)
\curveto(281.95827811,29.85808461)(281.6343095,29.4128646)(281.19098899,29.10594729)
\curveto(280.75144769,28.80281846)(280.2001327,28.6512542)(279.53704236,28.65125406)
\curveto(278.48745657,28.6512542)(277.71068975,28.99416833)(277.20673955,29.6799975)
\curveto(276.8088824,30.22941704)(276.60995431,30.92282353)(276.60995468,31.76021903)
\curveto(276.60995431,32.76054017)(276.87140266,33.54299066)(277.39430051,34.10757284)
\curveto(277.91719605,34.67593349)(278.57839513,34.96011648)(279.37789973,34.96012265)
\curveto(280.27591483,34.96011648)(280.98447774,34.66267162)(281.50359059,34.06778718)
\curveto(282.02269292,33.47668129)(282.27087939,32.56919029)(282.24815076,31.34531146)
\lineto(278.24685032,31.34531146)
\curveto(278.25821563,30.87167059)(278.38704525,30.50223271)(278.63333957,30.23699671)
\curveto(278.87962909,29.97554691)(279.18654672,29.84482274)(279.55409336,29.84482379)
\curveto(279.80417107,29.84482274)(280.01446648,29.91302665)(280.18498022,30.04943575)
\curveto(280.35548607,30.18584232)(280.48431569,30.4056105)(280.57146947,30.70874093)
\moveto(280.66240811,32.32290191)
\curveto(280.65103637,32.78516936)(280.53167952,33.13566171)(280.30433719,33.37438001)
\curveto(280.07698674,33.61687823)(279.80038197,33.73812964)(279.47452204,33.73813459)
\curveto(279.12592101,33.73812964)(278.83794892,33.61119457)(278.6106049,33.35732901)
\curveto(278.38325615,33.10345431)(278.2714775,32.75864562)(278.27526865,32.32290191)
\lineto(280.66240811,32.32290191)
}
}
{
\newrgbcolor{curcolor}{0 0 0}
\pscustom[linestyle=none,fillstyle=solid,fillcolor=curcolor]
{
\newpath
\moveto(283.18595407,31.89094334)
\curveto(283.1859536,32.42141514)(283.31667777,32.93483907)(283.57812698,33.43121666)
\curveto(283.83957447,33.92758496)(284.20901235,34.30649561)(284.68644173,34.56794974)
\curveto(285.16765629,34.82939231)(285.70381486,34.96011648)(286.29491904,34.96012265)
\curveto(287.20809013,34.96011648)(287.95643866,34.66267162)(288.53996687,34.06778718)
\curveto(289.12348345,33.47668129)(289.41524465,32.72833277)(289.41525134,31.82273935)
\curveto(289.41524465,30.90956166)(289.11969434,30.15174036)(288.52859954,29.54927319)
\curveto(287.94128223,28.95059361)(287.20051191,28.6512542)(286.30628637,28.65125406)
\curveto(285.75307324,28.6512542)(285.22449289,28.77629471)(284.72054373,29.02637598)
\curveto(284.22037967,29.27645677)(283.83957447,29.64210554)(283.57812698,30.1233234)
\curveto(283.31667777,30.60832769)(283.1859536,31.19753375)(283.18595407,31.89094334)
\moveto(284.8228497,31.80568836)
\curveto(284.8228476,31.20700651)(284.96493909,30.74852463)(285.24912461,30.43024133)
\curveto(285.53330506,30.11195474)(285.88379741,29.95281227)(286.30060271,29.95281344)
\curveto(286.71740084,29.95281227)(287.06599863,30.11195474)(287.34639714,30.43024133)
\curveto(287.6305755,30.74852463)(287.77266699,31.21079562)(287.77267204,31.81705569)
\curveto(287.77266699,32.40815327)(287.6305755,32.86284605)(287.34639714,33.18113538)
\curveto(287.06599863,33.49941593)(286.71740084,33.65855841)(286.30060271,33.65856328)
\curveto(285.88379741,33.65855841)(285.53330506,33.49941593)(285.24912461,33.18113538)
\curveto(284.96493909,32.86284605)(284.8228476,32.40436416)(284.8228497,31.80568836)
}
}
{
\newrgbcolor{curcolor}{0 0 0}
\pscustom[linestyle=none,fillstyle=solid,fillcolor=curcolor]
{
\newpath
\moveto(290.10865947,30.50981265)
\lineto(291.71145311,30.75421026)
\curveto(291.77965515,30.44350156)(291.91795754,30.20668241)(292.12636069,30.04375208)
\curveto(292.33475925,29.88460835)(292.62652045,29.80503712)(293.00164516,29.80503814)
\curveto(293.4146546,29.80503712)(293.72536133,29.88081925)(293.93376628,30.03238475)
\curveto(294.07395913,30.13847849)(294.14405759,30.28056998)(294.1440619,30.45865966)
\curveto(294.14405759,30.57990939)(294.10616653,30.68032071)(294.0303886,30.75989392)
\curveto(293.95081316,30.83567408)(293.77272516,30.90577255)(293.49612405,30.97018954)
\curveto(292.20782419,31.25437035)(291.39127174,31.51392414)(291.04646426,31.7488517)
\curveto(290.56903563,32.0747119)(290.33032193,32.52751012)(290.33032242,33.10724773)
\curveto(290.33032193,33.63014011)(290.53682823,34.06967646)(290.94984195,34.4258581)
\curveto(291.36285344,34.78202848)(292.00321244,34.96011648)(292.87092085,34.96012265)
\curveto(293.69694303,34.96011648)(294.31077828,34.8256032)(294.71242844,34.55658241)
\curveto(295.11406885,34.28755008)(295.39067363,33.8896939)(295.54224359,33.36301268)
\lineto(294.03607226,33.08451307)
\curveto(293.97165325,33.31943338)(293.84850729,33.49941593)(293.66663401,33.62446128)
\curveto(293.48854217,33.74949696)(293.23277749,33.81201722)(292.89933918,33.81202224)
\curveto(292.4787453,33.81201722)(292.17751133,33.75328607)(291.99563638,33.63582861)
\curveto(291.87438282,33.55246342)(291.81375711,33.44447389)(291.81375909,33.31185969)
\curveto(291.81375711,33.19818197)(291.8668046,33.10155975)(291.97290172,33.02199275)
\curveto(292.11688563,32.91589354)(292.61325858,32.76622383)(293.46202206,32.57298318)
\curveto(294.31456739,32.37973497)(294.9094571,32.14291581)(295.24669299,31.86252501)
\curveto(295.58012895,31.57833895)(295.74684963,31.18237732)(295.74685555,30.67463894)
\curveto(295.74684963,30.12142751)(295.51571414,29.64589465)(295.05344837,29.24803893)
\curveto(294.59117216,28.85018229)(293.90723844,28.6512542)(293.00164516,28.65125406)
\curveto(292.17940589,28.6512542)(291.52767957,28.81797488)(291.04646426,29.15141662)
\curveto(290.56903563,29.48485762)(290.25643435,29.93765585)(290.10865947,30.50981265)
}
}
{
\newrgbcolor{curcolor}{0 1 0.25098041}
\pscustom[linewidth=2.32802935,linecolor=curcolor]
{
\newpath
\moveto(18.44308422,114.33554061)
\lineto(92.94002116,114.33554061)
\lineto(92.94002116,160.89613202)
\lineto(44.0514067,160.89613202)
\lineto(44.0514067,170.2082503)
\lineto(18.44308422,170.2082503)
\lineto(18.44308422,114.33554061)
\closepath
}
}
{
\newrgbcolor{curcolor}{0 0 0}
\pscustom[linestyle=none,fillstyle=solid,fillcolor=curcolor]
{
\newpath
\moveto(31.88381527,141.84823355)
\lineto(31.88381527,142.80308934)
\lineto(37.40265437,145.13339215)
\lineto(37.40265437,144.11601604)
\lineto(33.02623201,142.31997778)
\lineto(37.40265437,140.50688852)
\lineto(37.40265437,139.48951241)
\lineto(31.88381527,141.84823355)
}
}
{
\newrgbcolor{curcolor}{0 0 0}
\pscustom[linestyle=none,fillstyle=solid,fillcolor=curcolor]
{
\newpath
\moveto(38.68147926,141.84823355)
\lineto(38.68147926,142.80308934)
\lineto(44.20031836,145.13339215)
\lineto(44.20031836,144.11601604)
\lineto(39.823896,142.31997778)
\lineto(44.20031836,140.50688852)
\lineto(44.20031836,139.48951241)
\lineto(38.68147926,141.84823355)
}
}
{
\newrgbcolor{curcolor}{0 0 0}
\pscustom[linestyle=none,fillstyle=solid,fillcolor=curcolor]
{
\newpath
\moveto(49.54864768,138.9495642)
\curveto(49.16973232,138.6274894)(48.80408355,138.40014301)(48.45170026,138.26752435)
\curveto(48.10309885,138.13490556)(47.72797731,138.0685962)(47.32633451,138.06859606)
\curveto(46.66323839,138.0685962)(46.15360357,138.22963322)(45.79742851,138.55170762)
\curveto(45.44125155,138.87757043)(45.26316355,139.29247759)(45.26316397,139.79643034)
\curveto(45.26316355,140.09197905)(45.32947291,140.36100561)(45.46209226,140.60351083)
\curveto(45.59849947,140.84980035)(45.77469292,141.04683389)(45.99067314,141.19461203)
\curveto(46.21044017,141.34238419)(46.45673209,141.45416283)(46.72954964,141.52994829)
\curveto(46.9303704,141.58299245)(47.23349891,141.63414539)(47.6389361,141.68340725)
\curveto(48.46495852,141.78192054)(49.07311011,141.89938284)(49.4633927,142.03579451)
\curveto(49.46717718,142.17598762)(49.46907174,142.26503162)(49.46907636,142.30292678)
\curveto(49.46907174,142.7197244)(49.37244952,143.01338015)(49.17920943,143.18389492)
\curveto(48.91775674,143.41502543)(48.52937333,143.53059318)(48.01405802,143.53059851)
\curveto(47.53283833,143.53059318)(47.17666232,143.44533829)(46.94552893,143.27483356)
\curveto(46.71818043,143.10810781)(46.5495652,142.81066295)(46.4396827,142.3824981)
\lineto(45.43935759,142.51890607)
\curveto(45.53029555,142.94707078)(45.67996526,143.29187947)(45.88836716,143.55333317)
\curveto(46.09676697,143.81856527)(46.39800094,144.02128247)(46.79206996,144.16148537)
\curveto(47.18613508,144.30546546)(47.64272241,144.37745848)(48.16183332,144.37746465)
\curveto(48.67714848,144.37745848)(49.09584475,144.31683278)(49.41792337,144.19558736)
\curveto(49.73999285,144.07432996)(49.976812,143.92087115)(50.12838155,143.73521046)
\curveto(50.27994052,143.55332782)(50.3860355,143.32219233)(50.44666681,143.04180328)
\curveto(50.48076316,142.86749955)(50.49781414,142.55300371)(50.4978198,142.09831483)
\lineto(50.4978198,140.73423513)
\curveto(50.49781414,139.78316688)(50.51865423,139.18069895)(50.56034012,138.92682953)
\curveto(50.60580368,138.67674779)(50.69295313,138.43613952)(50.82178873,138.20500403)
\lineto(49.75325963,138.20500403)
\curveto(49.64715974,138.41719399)(49.57895582,138.66538047)(49.54864768,138.9495642)
\moveto(49.4633927,141.23439769)
\curveto(49.09205564,141.0828304)(48.53505699,140.95400078)(47.79239507,140.84790844)
\curveto(47.3718013,140.78728009)(47.07435644,140.71907618)(46.9000596,140.64329648)
\curveto(46.72575865,140.56751192)(46.59124537,140.45573328)(46.49651936,140.30796023)
\curveto(46.40179004,140.16397208)(46.35442621,140.00293505)(46.35442772,139.82484867)
\curveto(46.35442621,139.55203138)(46.45673209,139.32468499)(46.66134566,139.14280882)
\curveto(46.86974469,138.96093077)(47.17287321,138.86999222)(47.57073212,138.86999288)
\curveto(47.96479646,138.86999222)(48.31528881,138.95524711)(48.62221022,139.12575782)
\curveto(48.92912406,139.3000558)(49.1545759,139.53687496)(49.2985664,139.836216)
\curveto(49.40844603,140.06734986)(49.46338808,140.40836944)(49.4633927,140.85927577)
\lineto(49.4633927,141.23439769)
}
}
{
\newrgbcolor{curcolor}{0 0 0}
\pscustom[linestyle=none,fillstyle=solid,fillcolor=curcolor]
{
\newpath
\moveto(52.08924552,135.89175221)
\lineto(52.08924552,144.24105668)
\lineto(53.02136664,144.24105668)
\lineto(53.02136664,143.45671086)
\curveto(53.24113312,143.76362323)(53.48931959,143.99286417)(53.76592681,144.14443437)
\curveto(54.04252914,144.2997818)(54.37786506,144.37745848)(54.77193559,144.37746465)
\curveto(55.28725062,144.37745848)(55.74194339,144.24483975)(56.13601528,143.97960807)
\curveto(56.53007754,143.71436485)(56.8275224,143.3392433)(57.02835075,142.85424232)
\curveto(57.22916769,142.37302115)(57.32957901,141.8444408)(57.32958501,141.26849968)
\curveto(57.32957901,140.65087226)(57.21780037,140.09387361)(56.99424876,139.59750205)
\curveto(56.77447491,139.10491682)(56.45240086,138.72600617)(56.02802564,138.46076897)
\curveto(55.60743011,138.19932037)(55.16410466,138.0685962)(54.69804793,138.06859606)
\curveto(54.35702498,138.0685962)(54.05010735,138.14058922)(53.77729414,138.28457534)
\curveto(53.50826513,138.42856131)(53.2866024,138.61043842)(53.11230529,138.83020722)
\lineto(53.11230529,135.89175221)
\lineto(52.08924552,135.89175221)
\moveto(53.01568298,141.18892836)
\curveto(53.01568128,140.41215855)(53.1729292,139.83810892)(53.48742721,139.46677775)
\curveto(53.80192088,139.09544405)(54.18272608,138.90977783)(54.62984395,138.90977854)
\curveto(55.08453342,138.90977783)(55.47291683,139.10112771)(55.79499536,139.48382874)
\curveto(56.12085404,139.87031633)(56.28378562,140.4671006)(56.28379058,141.27418334)
\curveto(56.28378562,142.04336889)(56.12464315,142.61931307)(55.80636269,143.00201763)
\curveto(55.49186237,143.38471258)(55.11484627,143.57606246)(54.67531327,143.57606783)
\curveto(54.23956268,143.57606246)(53.85307381,143.37145071)(53.51584553,142.96223197)
\curveto(53.18240197,142.55679282)(53.01568128,141.96569221)(53.01568298,141.18892836)
}
}
{
\newrgbcolor{curcolor}{0 0 0}
\pscustom[linestyle=none,fillstyle=solid,fillcolor=curcolor]
{
\newpath
\moveto(58.56862437,135.89175221)
\lineto(58.56862437,144.24105668)
\lineto(59.50074549,144.24105668)
\lineto(59.50074549,143.45671086)
\curveto(59.72051197,143.76362323)(59.96869844,143.99286417)(60.24530566,144.14443437)
\curveto(60.52190799,144.2997818)(60.85724391,144.37745848)(61.25131443,144.37746465)
\curveto(61.76662947,144.37745848)(62.22132224,144.24483975)(62.61539413,143.97960807)
\curveto(63.00945639,143.71436485)(63.30690125,143.3392433)(63.5077296,142.85424232)
\curveto(63.70854653,142.37302115)(63.80895786,141.8444408)(63.80896386,141.26849968)
\curveto(63.80895786,140.65087226)(63.69717921,140.09387361)(63.47362761,139.59750205)
\curveto(63.25385376,139.10491682)(62.93177971,138.72600617)(62.50740449,138.46076897)
\curveto(62.08680896,138.19932037)(61.6434835,138.0685962)(61.17742678,138.06859606)
\curveto(60.83640383,138.0685962)(60.5294862,138.14058922)(60.25667299,138.28457534)
\curveto(59.98764397,138.42856131)(59.76598125,138.61043842)(59.59168414,138.83020722)
\lineto(59.59168414,135.89175221)
\lineto(58.56862437,135.89175221)
\moveto(59.49506183,141.18892836)
\curveto(59.49506013,140.41215855)(59.65230805,139.83810892)(59.96680605,139.46677775)
\curveto(60.28129973,139.09544405)(60.66210493,138.90977783)(61.1092228,138.90977854)
\curveto(61.56391227,138.90977783)(61.95229568,139.10112771)(62.27437421,139.48382874)
\curveto(62.60023289,139.87031633)(62.76316447,140.4671006)(62.76316943,141.27418334)
\curveto(62.76316447,142.04336889)(62.604022,142.61931307)(62.28574154,143.00201763)
\curveto(61.97124121,143.38471258)(61.59422512,143.57606246)(61.15469212,143.57606783)
\curveto(60.71894152,143.57606246)(60.33245266,143.37145071)(59.99522438,142.96223197)
\curveto(59.66178082,142.55679282)(59.49506013,141.96569221)(59.49506183,141.18892836)
}
}
{
\newrgbcolor{curcolor}{0 0 0}
\pscustom[linestyle=none,fillstyle=solid,fillcolor=curcolor]
{
\newpath
\moveto(70.43611801,141.84823355)
\lineto(64.91727891,139.48951241)
\lineto(64.91727891,140.50688852)
\lineto(69.2880176,142.31997778)
\lineto(64.91727891,144.11601604)
\lineto(64.91727891,145.13339215)
\lineto(70.43611801,142.80308934)
\lineto(70.43611801,141.84823355)
}
}
{
\newrgbcolor{curcolor}{0 0 0}
\pscustom[linestyle=none,fillstyle=solid,fillcolor=curcolor]
{
\newpath
\moveto(77.23378067,141.84823355)
\lineto(71.71494157,139.48951241)
\lineto(71.71494157,140.50688852)
\lineto(76.08568026,142.31997778)
\lineto(71.71494157,144.11601604)
\lineto(71.71494157,145.13339215)
\lineto(77.23378067,142.80308934)
\lineto(77.23378067,141.84823355)
}
}
{
\newrgbcolor{curcolor}{0 0 0}
\pscustom[linestyle=none,fillstyle=solid,fillcolor=curcolor]
{
\newpath
\moveto(42.55887651,120.74477415)
\lineto(42.55887651,129.07702763)
\lineto(44.15598648,129.07702763)
\lineto(44.15598648,120.74477415)
\lineto(42.55887651,120.74477415)
}
}
{
\newrgbcolor{curcolor}{0 0 0}
\pscustom[linestyle=none,fillstyle=solid,fillcolor=curcolor]
{
\newpath
\moveto(45.78719856,127.59927462)
\lineto(45.78719856,129.07702763)
\lineto(47.38430853,129.07702763)
\lineto(47.38430853,127.59927462)
\lineto(45.78719856,127.59927462)
\moveto(45.78719856,120.74477415)
\lineto(45.78719856,126.7808268)
\lineto(47.38430853,126.7808268)
\lineto(47.38430853,120.74477415)
\lineto(45.78719856,120.74477415)
}
}
{
\newrgbcolor{curcolor}{0 0 0}
\pscustom[linestyle=none,fillstyle=solid,fillcolor=curcolor]
{
\newpath
\moveto(54.50594138,120.74477415)
\lineto(52.9088314,120.74477415)
\lineto(52.9088314,123.8253208)
\curveto(52.90882668,124.47704403)(52.87472472,124.89763485)(52.80652543,125.08709452)
\curveto(52.73831688,125.2803346)(52.62653824,125.43000431)(52.47118917,125.53610408)
\curveto(52.31962062,125.64219427)(52.13584895,125.69524176)(51.91987362,125.69524671)
\curveto(51.64326511,125.69524176)(51.39507864,125.61945963)(51.17531346,125.4679001)
\curveto(50.95554229,125.31633112)(50.80397803,125.11550847)(50.72062023,124.86543157)
\curveto(50.64104645,124.61534642)(50.60126083,124.15307543)(50.60126325,123.47861721)
\lineto(50.60126325,120.74477415)
\lineto(49.00415328,120.74477415)
\lineto(49.00415328,126.7808268)
\lineto(50.48758994,126.7808268)
\lineto(50.48758994,125.894175)
\curveto(51.01427344,126.57620902)(51.67736707,126.9172286)(52.47687283,126.91723477)
\curveto(52.82925544,126.9172286)(53.15132949,126.85281379)(53.44309595,126.72399015)
\curveto(53.73485189,126.59894366)(53.95462006,126.43790663)(54.10240114,126.24087859)
\curveto(54.25395947,126.04383956)(54.3581599,125.82028228)(54.41500273,125.57020607)
\curveto(54.4756222,125.32012022)(54.50593505,124.96204966)(54.50594138,124.49599331)
\lineto(54.50594138,120.74477415)
}
}
{
\newrgbcolor{curcolor}{0 0 0}
\pscustom[linestyle=none,fillstyle=solid,fillcolor=curcolor]
{
\newpath
\moveto(56.07463353,120.74477415)
\lineto(56.07463353,129.07702763)
\lineto(57.67174351,129.07702763)
\lineto(57.67174351,124.65513595)
\lineto(59.54166943,126.7808268)
\lineto(61.50821765,126.7808268)
\lineto(59.44504712,124.57556463)
\lineto(61.65599296,120.74477415)
\lineto(59.93384234,120.74477415)
\lineto(58.41630368,123.45588255)
\lineto(57.67174351,122.67722039)
\lineto(57.67174351,120.74477415)
\lineto(56.07463353,120.74477415)
}
}
{
\newrgbcolor{curcolor}{0 0 0}
\pscustom[linestyle=none,fillstyle=solid,fillcolor=curcolor]
{
\newpath
\moveto(62.04816439,122.46692477)
\lineto(63.65095803,122.71132238)
\curveto(63.71916007,122.40061368)(63.85746246,122.16379453)(64.0658656,122.0008642)
\curveto(64.27426417,121.84172048)(64.56602537,121.76214924)(64.94115007,121.76215026)
\curveto(65.35415951,121.76214924)(65.66486624,121.83793137)(65.8732712,121.98949687)
\curveto(66.01346404,122.09559061)(66.08356251,122.2376821)(66.08356682,122.41577178)
\curveto(66.08356251,122.53702151)(66.04567145,122.63743284)(65.96989351,122.71700604)
\curveto(65.89031808,122.7927862)(65.71223008,122.86288467)(65.43562896,122.92730166)
\curveto(64.1473291,123.21148247)(63.33077666,123.47103626)(62.98596918,123.70596382)
\curveto(62.50854055,124.03182402)(62.26982684,124.48462224)(62.26982734,125.06435985)
\curveto(62.26982684,125.58725223)(62.47633314,126.02678858)(62.88934686,126.38297023)
\curveto(63.30235836,126.7391406)(63.94271735,126.9172286)(64.81042577,126.91723477)
\curveto(65.63644795,126.9172286)(66.2502832,126.78271532)(66.65193336,126.51369453)
\curveto(67.05357377,126.2446622)(67.33017854,125.84680602)(67.48174851,125.3201248)
\lineto(65.97557718,125.04162519)
\curveto(65.91115817,125.2765455)(65.78801221,125.45652806)(65.60613892,125.58157341)
\curveto(65.42804709,125.70660908)(65.1722824,125.76912934)(64.8388441,125.76913436)
\curveto(64.41825021,125.76912934)(64.11701625,125.71039819)(63.9351413,125.59294074)
\curveto(63.81388773,125.50957555)(63.75326203,125.40158601)(63.75326401,125.26897181)
\curveto(63.75326203,125.15529409)(63.80630952,125.05867187)(63.91240664,124.97910487)
\curveto(64.05639055,124.87300566)(64.55276349,124.72333595)(65.40152697,124.53009531)
\curveto(66.2540723,124.33684709)(66.84896202,124.10002794)(67.18619791,123.81963713)
\curveto(67.51963386,123.53545107)(67.68635455,123.13948944)(67.68636046,122.63175106)
\curveto(67.68635455,122.07853963)(67.45521905,121.60300677)(66.99295328,121.20515105)
\curveto(66.53067707,120.80729441)(65.84674336,120.60836632)(64.94115007,120.60836618)
\curveto(64.1189108,120.60836632)(63.46718449,120.775087)(62.98596918,121.10852874)
\curveto(62.50854055,121.44196974)(62.19593927,121.89476797)(62.04816439,122.46692477)
}
}
{
\newrgbcolor{curcolor}{0 1 0.25098041}
\pscustom[linewidth=2.32802935,linecolor=curcolor]
{
\newpath
\moveto(18.44308422,205.12868455)
\lineto(92.94002116,205.12868455)
\lineto(92.94002116,251.68927596)
\lineto(44.0514067,251.68927596)
\lineto(44.0514067,261.00139424)
\lineto(18.44308422,261.00139424)
\lineto(18.44308422,205.12868455)
\closepath
}
}
{
\newrgbcolor{curcolor}{0 0 0}
\pscustom[linestyle=none,fillstyle=solid,fillcolor=curcolor]
{
\newpath
\moveto(31.88381527,232.64138342)
\lineto(31.88381527,233.59623921)
\lineto(37.40265437,235.92654202)
\lineto(37.40265437,234.90916591)
\lineto(33.02623201,233.11312765)
\lineto(37.40265437,231.30003839)
\lineto(37.40265437,230.28266228)
\lineto(31.88381527,232.64138342)
}
}
{
\newrgbcolor{curcolor}{0 0 0}
\pscustom[linestyle=none,fillstyle=solid,fillcolor=curcolor]
{
\newpath
\moveto(38.68147926,232.64138342)
\lineto(38.68147926,233.59623921)
\lineto(44.20031836,235.92654202)
\lineto(44.20031836,234.90916591)
\lineto(39.823896,233.11312765)
\lineto(44.20031836,231.30003839)
\lineto(44.20031836,230.28266228)
\lineto(38.68147926,232.64138342)
}
}
{
\newrgbcolor{curcolor}{0 0 0}
\pscustom[linestyle=none,fillstyle=solid,fillcolor=curcolor]
{
\newpath
\moveto(49.54864768,229.74271407)
\curveto(49.16973232,229.42063927)(48.80408355,229.19329288)(48.45170026,229.06067422)
\curveto(48.10309885,228.92805543)(47.72797731,228.86174607)(47.32633451,228.86174593)
\curveto(46.66323839,228.86174607)(46.15360357,229.02278309)(45.79742851,229.34485749)
\curveto(45.44125155,229.6707203)(45.26316355,230.08562746)(45.26316397,230.58958021)
\curveto(45.26316355,230.88512892)(45.32947291,231.15415548)(45.46209226,231.3966607)
\curveto(45.59849947,231.64295022)(45.77469292,231.83998376)(45.99067314,231.9877619)
\curveto(46.21044017,232.13553406)(46.45673209,232.2473127)(46.72954964,232.32309816)
\curveto(46.9303704,232.37614232)(47.23349891,232.42729526)(47.6389361,232.47655712)
\curveto(48.46495852,232.57507041)(49.07311011,232.69253271)(49.4633927,232.82894438)
\curveto(49.46717718,232.96913749)(49.46907174,233.05818149)(49.46907636,233.09607665)
\curveto(49.46907174,233.51287427)(49.37244952,233.80653002)(49.17920943,233.97704479)
\curveto(48.91775674,234.2081753)(48.52937333,234.32374305)(48.01405802,234.32374838)
\curveto(47.53283833,234.32374305)(47.17666232,234.23848816)(46.94552893,234.06798343)
\curveto(46.71818043,233.90125768)(46.5495652,233.60381282)(46.4396827,233.17564797)
\lineto(45.43935759,233.31205594)
\curveto(45.53029555,233.74022065)(45.67996526,234.08502934)(45.88836716,234.34648304)
\curveto(46.09676697,234.61171514)(46.39800094,234.81443234)(46.79206996,234.95463524)
\curveto(47.18613508,235.09861533)(47.64272241,235.17060835)(48.16183332,235.17061452)
\curveto(48.67714848,235.17060835)(49.09584475,235.10998265)(49.41792337,234.98873723)
\curveto(49.73999285,234.86747983)(49.976812,234.71402102)(50.12838155,234.52836033)
\curveto(50.27994052,234.34647769)(50.3860355,234.1153422)(50.44666681,233.83495315)
\curveto(50.48076316,233.66064942)(50.49781414,233.34615358)(50.4978198,232.8914647)
\lineto(50.4978198,231.527385)
\curveto(50.49781414,230.57631675)(50.51865423,229.97384882)(50.56034012,229.7199794)
\curveto(50.60580368,229.46989766)(50.69295313,229.22928939)(50.82178873,228.9981539)
\lineto(49.75325963,228.9981539)
\curveto(49.64715974,229.21034386)(49.57895582,229.45853034)(49.54864768,229.74271407)
\moveto(49.4633927,232.02754756)
\curveto(49.09205564,231.87598027)(48.53505699,231.74715065)(47.79239507,231.64105831)
\curveto(47.3718013,231.58042996)(47.07435644,231.51222605)(46.9000596,231.43644636)
\curveto(46.72575865,231.36066179)(46.59124537,231.24888315)(46.49651936,231.1011101)
\curveto(46.40179004,230.95712195)(46.35442621,230.79608492)(46.35442772,230.61799854)
\curveto(46.35442621,230.34518125)(46.45673209,230.11783486)(46.66134566,229.93595869)
\curveto(46.86974469,229.75408064)(47.17287321,229.66314209)(47.57073212,229.66314275)
\curveto(47.96479646,229.66314209)(48.31528881,229.74839698)(48.62221022,229.91890769)
\curveto(48.92912406,230.09320567)(49.1545759,230.33002483)(49.2985664,230.62936587)
\curveto(49.40844603,230.86049973)(49.46338808,231.20151932)(49.4633927,231.65242564)
\lineto(49.4633927,232.02754756)
}
}
{
\newrgbcolor{curcolor}{0 0 0}
\pscustom[linestyle=none,fillstyle=solid,fillcolor=curcolor]
{
\newpath
\moveto(52.08924552,226.68490208)
\lineto(52.08924552,235.03420655)
\lineto(53.02136664,235.03420655)
\lineto(53.02136664,234.24986073)
\curveto(53.24113312,234.5567731)(53.48931959,234.78601404)(53.76592681,234.93758424)
\curveto(54.04252914,235.09293167)(54.37786506,235.17060835)(54.77193559,235.17061452)
\curveto(55.28725062,235.17060835)(55.74194339,235.03798962)(56.13601528,234.77275794)
\curveto(56.53007754,234.50751472)(56.8275224,234.13239317)(57.02835075,233.6473922)
\curveto(57.22916769,233.16617102)(57.32957901,232.63759067)(57.32958501,232.06164955)
\curveto(57.32957901,231.44402213)(57.21780037,230.88702348)(56.99424876,230.39065192)
\curveto(56.77447491,229.89806669)(56.45240086,229.51915604)(56.02802564,229.25391884)
\curveto(55.60743011,228.99247024)(55.16410466,228.86174607)(54.69804793,228.86174593)
\curveto(54.35702498,228.86174607)(54.05010735,228.93373909)(53.77729414,229.07772521)
\curveto(53.50826513,229.22171118)(53.2866024,229.40358829)(53.11230529,229.62335709)
\lineto(53.11230529,226.68490208)
\lineto(52.08924552,226.68490208)
\moveto(53.01568298,231.98207823)
\curveto(53.01568128,231.20530842)(53.1729292,230.63125879)(53.48742721,230.25992762)
\curveto(53.80192088,229.88859392)(54.18272608,229.7029277)(54.62984395,229.70292841)
\curveto(55.08453342,229.7029277)(55.47291683,229.89427758)(55.79499536,230.27697861)
\curveto(56.12085404,230.6634662)(56.28378562,231.26025047)(56.28379058,232.06733321)
\curveto(56.28378562,232.83651876)(56.12464315,233.41246294)(55.80636269,233.7951675)
\curveto(55.49186237,234.17786245)(55.11484627,234.36921233)(54.67531327,234.3692177)
\curveto(54.23956268,234.36921233)(53.85307381,234.16460058)(53.51584553,233.75538184)
\curveto(53.18240197,233.34994269)(53.01568128,232.75884208)(53.01568298,231.98207823)
}
}
{
\newrgbcolor{curcolor}{0 0 0}
\pscustom[linestyle=none,fillstyle=solid,fillcolor=curcolor]
{
\newpath
\moveto(58.56862437,226.68490208)
\lineto(58.56862437,235.03420655)
\lineto(59.50074549,235.03420655)
\lineto(59.50074549,234.24986073)
\curveto(59.72051197,234.5567731)(59.96869844,234.78601404)(60.24530566,234.93758424)
\curveto(60.52190799,235.09293167)(60.85724391,235.17060835)(61.25131443,235.17061452)
\curveto(61.76662947,235.17060835)(62.22132224,235.03798962)(62.61539413,234.77275794)
\curveto(63.00945639,234.50751472)(63.30690125,234.13239317)(63.5077296,233.6473922)
\curveto(63.70854653,233.16617102)(63.80895786,232.63759067)(63.80896386,232.06164955)
\curveto(63.80895786,231.44402213)(63.69717921,230.88702348)(63.47362761,230.39065192)
\curveto(63.25385376,229.89806669)(62.93177971,229.51915604)(62.50740449,229.25391884)
\curveto(62.08680896,228.99247024)(61.6434835,228.86174607)(61.17742678,228.86174593)
\curveto(60.83640383,228.86174607)(60.5294862,228.93373909)(60.25667299,229.07772521)
\curveto(59.98764397,229.22171118)(59.76598125,229.40358829)(59.59168414,229.62335709)
\lineto(59.59168414,226.68490208)
\lineto(58.56862437,226.68490208)
\moveto(59.49506183,231.98207823)
\curveto(59.49506013,231.20530842)(59.65230805,230.63125879)(59.96680605,230.25992762)
\curveto(60.28129973,229.88859392)(60.66210493,229.7029277)(61.1092228,229.70292841)
\curveto(61.56391227,229.7029277)(61.95229568,229.89427758)(62.27437421,230.27697861)
\curveto(62.60023289,230.6634662)(62.76316447,231.26025047)(62.76316943,232.06733321)
\curveto(62.76316447,232.83651876)(62.604022,233.41246294)(62.28574154,233.7951675)
\curveto(61.97124121,234.17786245)(61.59422512,234.36921233)(61.15469212,234.3692177)
\curveto(60.71894152,234.36921233)(60.33245266,234.16460058)(59.99522438,233.75538184)
\curveto(59.66178082,233.34994269)(59.49506013,232.75884208)(59.49506183,231.98207823)
}
}
{
\newrgbcolor{curcolor}{0 0 0}
\pscustom[linestyle=none,fillstyle=solid,fillcolor=curcolor]
{
\newpath
\moveto(70.43611801,232.64138342)
\lineto(64.91727891,230.28266228)
\lineto(64.91727891,231.30003839)
\lineto(69.2880176,233.11312765)
\lineto(64.91727891,234.90916591)
\lineto(64.91727891,235.92654202)
\lineto(70.43611801,233.59623921)
\lineto(70.43611801,232.64138342)
}
}
{
\newrgbcolor{curcolor}{0 0 0}
\pscustom[linestyle=none,fillstyle=solid,fillcolor=curcolor]
{
\newpath
\moveto(77.23378067,232.64138342)
\lineto(71.71494157,230.28266228)
\lineto(71.71494157,231.30003839)
\lineto(76.08568026,233.11312765)
\lineto(71.71494157,234.90916591)
\lineto(71.71494157,235.92654202)
\lineto(77.23378067,233.59623921)
\lineto(77.23378067,232.64138342)
}
}
{
\newrgbcolor{curcolor}{0 0 0}
\pscustom[linestyle=none,fillstyle=solid,fillcolor=curcolor]
{
\newpath
\moveto(30.2196407,217.57397668)
\lineto(31.1062925,217.57397668)
\lineto(31.1062925,218.02866991)
\curveto(31.10629148,218.53640368)(31.15933897,218.91531433)(31.26543514,219.16540299)
\curveto(31.37531804,219.41547639)(31.57424613,219.61819358)(31.86222,219.77355518)
\curveto(32.15397942,219.93268942)(32.52152275,220.01226066)(32.96485109,220.01226913)
\curveto(33.41954099,220.01226066)(33.864761,219.94405674)(34.30051246,219.80765718)
\lineto(34.08453317,218.69365876)
\curveto(33.83065904,218.75427731)(33.58626167,218.78459016)(33.35134034,218.7845974)
\curveto(33.12020157,218.78459016)(32.95348089,218.72964811)(32.85117778,218.61977111)
\curveto(32.75265825,218.51366905)(32.70339986,218.30716274)(32.70340248,218.00025158)
\lineto(32.70340248,217.57397668)
\lineto(33.89697222,217.57397668)
\lineto(33.89697222,216.31788662)
\lineto(32.70340248,216.31788662)
\lineto(32.70340248,211.53792402)
\lineto(31.1062925,211.53792402)
\lineto(31.1062925,216.31788662)
\lineto(30.2196407,216.31788662)
\lineto(30.2196407,217.57397668)
}
}
{
\newrgbcolor{curcolor}{0 0 0}
\pscustom[linestyle=none,fillstyle=solid,fillcolor=curcolor]
{
\newpath
\moveto(38.30181308,213.45900293)
\lineto(39.89323939,213.19187065)
\curveto(39.68862172,212.6083466)(39.36465311,212.16312659)(38.9213326,211.85620928)
\curveto(38.4817913,211.55308045)(37.93047631,211.40151619)(37.26738597,211.40151605)
\curveto(36.21780018,211.40151619)(35.44103336,211.74443032)(34.93708316,212.43025949)
\curveto(34.53922601,212.97967904)(34.34029792,213.67308552)(34.34029829,214.51048103)
\curveto(34.34029792,215.51080216)(34.60174627,216.29325265)(35.12464412,216.85783483)
\curveto(35.64753966,217.42619549)(36.30873874,217.71037847)(37.10824334,217.71038464)
\curveto(38.00625844,217.71037847)(38.71482135,217.41293361)(39.2339342,216.81804918)
\curveto(39.75303653,216.22694329)(40.001223,215.31945229)(39.97849437,214.09557345)
\lineto(35.97719393,214.09557345)
\curveto(35.98855924,213.62193258)(36.11738886,213.2524947)(36.36368318,212.9872587)
\curveto(36.6099727,212.7258089)(36.91689033,212.59508473)(37.28443697,212.59508579)
\curveto(37.53451468,212.59508473)(37.74481009,212.66328865)(37.91532383,212.79969774)
\curveto(38.08582968,212.93610431)(38.2146593,213.15587249)(38.30181308,213.45900293)
\moveto(38.39275172,215.0731639)
\curveto(38.38137998,215.53543135)(38.26202313,215.8859237)(38.0346808,216.124642)
\curveto(37.80733035,216.36714023)(37.53072558,216.48839163)(37.20486565,216.48839658)
\curveto(36.85626462,216.48839163)(36.56829253,216.36145657)(36.34094851,216.107591)
\curveto(36.11359975,215.8537163)(36.00182111,215.50890761)(36.00561225,215.0731639)
\lineto(38.39275172,215.0731639)
}
}
{
\newrgbcolor{curcolor}{0 0 0}
\pscustom[linestyle=none,fillstyle=solid,fillcolor=curcolor]
{
\newpath
\moveto(44.78119192,213.45900293)
\lineto(46.37261824,213.19187065)
\curveto(46.16800056,212.6083466)(45.84403196,212.16312659)(45.40071145,211.85620928)
\curveto(44.96117015,211.55308045)(44.40985516,211.40151619)(43.74676482,211.40151605)
\curveto(42.69717903,211.40151619)(41.9204122,211.74443032)(41.41646201,212.43025949)
\curveto(41.01860486,212.97967904)(40.81967677,213.67308552)(40.81967714,214.51048103)
\curveto(40.81967677,215.51080216)(41.08112512,216.29325265)(41.60402297,216.85783483)
\curveto(42.12691851,217.42619549)(42.78811759,217.71037847)(43.58762219,217.71038464)
\curveto(44.48563729,217.71037847)(45.1942002,217.41293361)(45.71331305,216.81804918)
\curveto(46.23241537,216.22694329)(46.48060185,215.31945229)(46.45787322,214.09557345)
\lineto(42.45657278,214.09557345)
\curveto(42.46793809,213.62193258)(42.59676771,213.2524947)(42.84306202,212.9872587)
\curveto(43.08935155,212.7258089)(43.39626918,212.59508473)(43.76381582,212.59508579)
\curveto(44.01389353,212.59508473)(44.22418894,212.66328865)(44.39470268,212.79969774)
\curveto(44.56520852,212.93610431)(44.69403815,213.15587249)(44.78119192,213.45900293)
\moveto(44.87213057,215.0731639)
\curveto(44.86075883,215.53543135)(44.74140198,215.8859237)(44.51405965,216.124642)
\curveto(44.2867092,216.36714023)(44.01010443,216.48839163)(43.6842445,216.48839658)
\curveto(43.33564347,216.48839163)(43.04767138,216.36145657)(42.82032736,216.107591)
\curveto(42.5929786,215.8537163)(42.48119996,215.50890761)(42.4849911,215.0731639)
\lineto(44.87213057,215.0731639)
}
}
{
\newrgbcolor{curcolor}{0 0 0}
\pscustom[linestyle=none,fillstyle=solid,fillcolor=curcolor]
{
\newpath
\moveto(53.30100488,211.53792402)
\lineto(51.81756821,211.53792402)
\lineto(51.81756821,212.42457582)
\curveto(51.5712714,212.07976625)(51.2795102,211.82210701)(50.94228373,211.65159733)
\curveto(50.60883835,211.48487653)(50.27160788,211.40151619)(49.93059129,211.40151605)
\curveto(49.23718181,211.40151619)(48.64229209,211.68001551)(48.14592036,212.23701487)
\curveto(47.6533353,212.79780193)(47.40704338,213.57835786)(47.40704386,214.57868501)
\curveto(47.40704338,215.60174072)(47.64765164,216.37850755)(48.12886936,216.90898782)
\curveto(48.61008469,217.44324647)(49.21823628,217.71037847)(49.95332596,217.71038464)
\curveto(50.62778388,217.71037847)(51.21130628,217.42998459)(51.7038949,216.86920217)
\lineto(51.7038949,219.8701775)
\lineto(53.30100488,219.8701775)
\lineto(53.30100488,211.53792402)
\moveto(49.03825583,214.68667465)
\curveto(49.03825372,214.0425234)(49.12729772,213.57646331)(49.3053881,213.28849296)
\curveto(49.56304496,212.8716895)(49.92301008,212.66328865)(50.38528453,212.66328977)
\curveto(50.7528244,212.66328865)(51.06542568,212.81864201)(51.32308932,213.12935033)
\curveto(51.58074416,213.44384458)(51.70957378,213.91179923)(51.70957856,214.53321569)
\curveto(51.70957378,215.22661918)(51.58453327,215.72488668)(51.33445665,216.02801969)
\curveto(51.08437121,216.33493282)(50.76419172,216.48839163)(50.37391719,216.48839658)
\curveto(49.9950031,216.48839163)(49.67671816,216.33682737)(49.41906141,216.03370335)
\curveto(49.16518878,215.73435944)(49.03825372,215.28535033)(49.03825583,214.68667465)
}
}
{
\newrgbcolor{curcolor}{0 0 0}
\pscustom[linestyle=none,fillstyle=solid,fillcolor=curcolor]
{
\newpath
\moveto(54.81286082,211.53792402)
\lineto(54.81286082,219.8701775)
\lineto(56.4099708,219.8701775)
\lineto(56.4099708,216.86920217)
\curveto(56.90255227,217.42998459)(57.48607467,217.71037847)(58.16053974,217.71038464)
\curveto(58.89562228,217.71037847)(59.50377387,217.44324647)(59.98499633,216.90898782)
\curveto(60.46620691,216.37850755)(60.70681518,215.61500259)(60.70682184,214.61847067)
\curveto(60.70681518,213.58783063)(60.46052325,212.79401282)(59.96794533,212.23701487)
\curveto(59.47914468,211.68001551)(58.88425496,211.40151619)(58.1832744,211.40151605)
\curveto(57.83846157,211.40151619)(57.49744199,211.48677108)(57.16021463,211.657281)
\curveto(56.82677014,211.83157977)(56.53879805,212.08734446)(56.29629749,212.42457582)
\lineto(56.29629749,211.53792402)
\lineto(54.81286082,211.53792402)
\moveto(56.39860346,214.68667465)
\curveto(56.39860111,214.06146894)(56.49711788,213.59919795)(56.69415407,213.2998603)
\curveto(56.97075619,212.87547861)(57.33829952,212.66328865)(57.79678515,212.66328977)
\curveto(58.1491683,212.66328865)(58.44850772,212.81295835)(58.69480429,213.11229934)
\curveto(58.94488066,213.41542628)(59.06992118,213.89095914)(59.0699262,214.53889935)
\curveto(59.06992118,215.22851373)(58.94488066,215.72488668)(58.69480429,216.02801969)
\curveto(58.44471861,216.33493282)(58.12453911,216.48839163)(57.73426483,216.48839658)
\curveto(57.35156139,216.48839163)(57.03327645,216.33872193)(56.77940905,216.03938702)
\curveto(56.52553618,215.74383221)(56.39860111,215.29292854)(56.39860346,214.68667465)
}
}
{
\newrgbcolor{curcolor}{0 0 0}
\pscustom[linestyle=none,fillstyle=solid,fillcolor=curcolor]
{
\newpath
\moveto(63.19058456,215.73246909)
\lineto(61.74124989,215.99391769)
\curveto(61.90418089,216.57743564)(62.18457476,217.00939377)(62.58243237,217.28979341)
\curveto(62.98028712,217.57018153)(63.57138774,217.71037847)(64.35573597,217.71038464)
\curveto(65.06808479,217.71037847)(65.5985597,217.62512358)(65.94716228,217.4546197)
\curveto(66.29575529,217.2878931)(66.54015266,217.07380858)(66.68035512,216.81236551)
\curveto(66.82433565,216.554701)(66.89632867,216.07916813)(66.8963344,215.3857655)
\lineto(66.87928341,213.52152325)
\curveto(66.87927769,212.99104636)(66.90390688,212.59887384)(66.95317106,212.34500451)
\curveto(67.00621276,212.09492267)(67.10283497,211.82589611)(67.24303799,211.53792402)
\lineto(65.66297901,211.53792402)
\curveto(65.62129434,211.644019)(65.5701414,211.80126692)(65.50952005,212.00966825)
\curveto(65.48299195,212.10439544)(65.46404642,212.1669157)(65.45268339,212.19722921)
\curveto(65.17986343,211.9319911)(64.88810224,211.73306301)(64.57739892,211.60044434)
\curveto(64.26668877,211.46782555)(63.93514196,211.40151619)(63.58275748,211.40151605)
\curveto(62.96134159,211.40151619)(62.4706523,211.57013143)(62.11068814,211.90736227)
\curveto(61.75451118,212.24459238)(61.57642318,212.67086686)(61.57642359,213.18618699)
\curveto(61.57642318,213.52720492)(61.65788896,213.83033344)(61.8208212,214.09557345)
\curveto(61.98375212,214.36459745)(62.21109851,214.5692092)(62.50286105,214.70940931)
\curveto(62.79841001,214.85339219)(63.22278994,214.9784327)(63.7760021,215.08453123)
\curveto(64.52245346,215.22472462)(65.03966649,215.3554488)(65.32764275,215.47670414)
\lineto(65.32764275,215.63584677)
\curveto(65.32763859,215.9427603)(65.25185646,216.16063392)(65.10029614,216.2894683)
\curveto(64.94872794,216.42208227)(64.6626504,216.48839163)(64.24206266,216.48839658)
\curveto(63.9578766,216.48839163)(63.73621387,216.43155504)(63.57707381,216.31788662)
\curveto(63.41792892,216.20799775)(63.2890993,216.01285877)(63.19058456,215.73246909)
\moveto(65.32764275,214.43659338)
\curveto(65.12302684,214.36838656)(64.79905823,214.28692077)(64.35573597,214.19219576)
\curveto(63.91240732,214.09746545)(63.62254067,214.00463234)(63.48613516,213.91369616)
\curveto(63.27773198,213.76591863)(63.17353156,213.57835786)(63.17353357,213.35101328)
\curveto(63.17353156,213.12745419)(63.2568919,212.93420976)(63.42361484,212.77127941)
\curveto(63.59033327,212.6083466)(63.80252323,212.52688081)(64.06018537,212.5268818)
\curveto(64.34815456,212.52688081)(64.62286478,212.62160847)(64.88431685,212.81106507)
\curveto(65.07755756,212.95504984)(65.20449263,213.1312433)(65.26512243,213.33964595)
\curveto(65.3067985,213.47605198)(65.32763859,213.73560578)(65.32764275,214.11830811)
\lineto(65.32764275,214.43659338)
}
}
{
\newrgbcolor{curcolor}{0 0 0}
\pscustom[linestyle=none,fillstyle=solid,fillcolor=curcolor]
{
\newpath
\moveto(73.73946606,215.78930574)
\lineto(72.16509075,215.50512247)
\curveto(72.11203873,215.81961434)(71.99078732,216.0564335)(71.80133616,216.21558064)
\curveto(71.61566578,216.37471844)(71.37316297,216.45428968)(71.07382699,216.45429459)
\curveto(70.67596738,216.45428968)(70.35768243,216.31598729)(70.1189712,216.03938702)
\curveto(69.88404412,215.76656685)(69.76658182,215.30808497)(69.76658395,214.66393999)
\curveto(69.76658182,213.94779574)(69.88593868,213.44195003)(70.12465487,213.14640133)
\curveto(70.3671552,212.85084942)(70.6911238,212.70307426)(71.09656165,212.70307543)
\curveto(71.39968671,212.70307426)(71.64787319,212.78832916)(71.84112182,212.95884037)
\curveto(72.03436205,213.13313785)(72.17076988,213.43058271)(72.25034573,213.85117584)
\lineto(73.81903738,213.58404357)
\curveto(73.65609962,212.86411129)(73.34349834,212.32037451)(72.88123259,211.9528316)
\curveto(72.41895635,211.58528785)(71.79943745,211.40151619)(71.022674,211.40151605)
\curveto(70.13980881,211.40151619)(69.435035,211.68001551)(68.90835047,212.23701487)
\curveto(68.38545251,212.79401282)(68.12400416,213.56509599)(68.12400465,214.55026668)
\curveto(68.12400416,215.54679867)(68.38734706,216.32167095)(68.91403414,216.87488583)
\curveto(69.44071866,217.43187915)(70.15307068,217.71037847)(71.05109233,217.71038464)
\curveto(71.78617557,217.71037847)(72.36969797,217.551236)(72.80166127,217.23295675)
\curveto(73.23740335,216.91845522)(73.55000464,216.4372387)(73.73946606,215.78930574)
}
}
{
\newrgbcolor{curcolor}{0 0 0}
\pscustom[linestyle=none,fillstyle=solid,fillcolor=curcolor]
{
\newpath
\moveto(74.89893587,211.53792402)
\lineto(74.89893587,219.8701775)
\lineto(76.49604585,219.8701775)
\lineto(76.49604585,215.44828582)
\lineto(78.36597177,217.57397668)
\lineto(80.33251999,217.57397668)
\lineto(78.26934946,215.3687145)
\lineto(80.4802953,211.53792402)
\lineto(78.75814468,211.53792402)
\lineto(77.24060602,214.24903242)
\lineto(76.49604585,213.47037026)
\lineto(76.49604585,211.53792402)
\lineto(74.89893587,211.53792402)
}
}
{
\newrgbcolor{curcolor}{0.50196081 0.50196081 0}
\pscustom[linewidth=2.32802935,linecolor=curcolor]
{
\newpath
\moveto(241.93390437,295.92182848)
\lineto(316.43084131,295.92182848)
\lineto(316.43084131,342.4824199)
\lineto(266.3782102,342.4824199)
\lineto(266.3782102,351.79453818)
\lineto(241.93390437,351.79453818)
\lineto(241.93390437,295.92182848)
\closepath
}
}
{
\newrgbcolor{curcolor}{0 0 0}
\pscustom[linestyle=none,fillstyle=solid,fillcolor=curcolor]
{
\newpath
\moveto(257.70266372,323.43453329)
\lineto(257.70266372,324.38938908)
\lineto(263.22150282,326.71969189)
\lineto(263.22150282,325.70231578)
\lineto(258.84508046,323.90627752)
\lineto(263.22150282,322.09318826)
\lineto(263.22150282,321.07581215)
\lineto(257.70266372,323.43453329)
}
}
{
\newrgbcolor{curcolor}{0 0 0}
\pscustom[linestyle=none,fillstyle=solid,fillcolor=curcolor]
{
\newpath
\moveto(264.50032771,323.43453329)
\lineto(264.50032771,324.38938908)
\lineto(270.01916681,326.71969189)
\lineto(270.01916681,325.70231578)
\lineto(265.64274446,323.90627752)
\lineto(270.01916681,322.09318826)
\lineto(270.01916681,321.07581215)
\lineto(264.50032771,323.43453329)
}
}
{
\newrgbcolor{curcolor}{0 0 0}
\pscustom[linestyle=none,fillstyle=solid,fillcolor=curcolor]
{
\newpath
\moveto(275.38454713,319.79130377)
\lineto(275.38454713,320.67795557)
\curveto(274.9146932,319.99591552)(274.27622876,319.65489594)(273.46915189,319.6548958)
\curveto(273.11297307,319.65489594)(272.7795317,319.72309985)(272.46882678,319.85950775)
\curveto(272.16190734,319.99591552)(271.9326664,320.16642531)(271.78110326,320.37103764)
\curveto(271.63332699,320.57943792)(271.52912656,320.83330805)(271.46850167,321.1326488)
\curveto(271.42682069,321.33347011)(271.4059806,321.65175505)(271.40598135,322.08750459)
\lineto(271.40598135,325.82735642)
\lineto(272.42904112,325.82735642)
\lineto(272.42904112,322.4796775)
\curveto(272.42903935,321.9454108)(272.44987944,321.58544569)(272.49156144,321.39978108)
\curveto(272.55597442,321.13075291)(272.69238225,320.91856295)(272.90078535,320.76321055)
\curveto(273.10918396,320.61164532)(273.3668432,320.53586319)(273.67376384,320.53586394)
\curveto(273.98067845,320.53586319)(274.26865055,320.61353988)(274.53768098,320.76889422)
\curveto(274.80670366,320.92803571)(274.99615899,321.14212023)(275.10604752,321.41114841)
\curveto(275.21971627,321.68396245)(275.27655287,322.07802953)(275.27655748,322.59335081)
\lineto(275.27655748,325.82735642)
\lineto(276.29961725,325.82735642)
\lineto(276.29961725,319.79130377)
\lineto(275.38454713,319.79130377)
}
}
{
\newrgbcolor{curcolor}{0 0 0}
\pscustom[linestyle=none,fillstyle=solid,fillcolor=curcolor]
{
\newpath
\moveto(280.14177447,320.7063739)
\lineto(280.28954977,319.8026711)
\curveto(280.00157453,319.74204539)(279.74391529,319.71173253)(279.51657128,319.71173245)
\curveto(279.14523647,319.71173253)(278.85726437,319.77046368)(278.65265414,319.88792608)
\curveto(278.44804087,320.00538829)(278.30405483,320.1588471)(278.22069557,320.34830298)
\curveto(278.13733414,320.54154685)(278.09565397,320.94508669)(278.09565493,321.55892371)
\lineto(278.09565493,325.03164327)
\lineto(277.34541109,325.03164327)
\lineto(277.34541109,325.82735642)
\lineto(278.09565493,325.82735642)
\lineto(278.09565493,327.32216042)
\lineto(279.11303103,327.93599629)
\lineto(279.11303103,325.82735642)
\lineto(280.14177447,325.82735642)
\lineto(280.14177447,325.03164327)
\lineto(279.11303103,325.03164327)
\lineto(279.11303103,321.50208705)
\curveto(279.11302906,321.21032415)(279.13008004,321.02276337)(279.16418402,320.93940418)
\curveto(279.20207306,320.85604269)(279.26080421,320.78973333)(279.34037765,320.74047589)
\curveto(279.42373579,320.69121656)(279.54119809,320.66658737)(279.6927649,320.66658824)
\curveto(279.80643555,320.66658737)(279.95610525,320.67984924)(280.14177447,320.7063739)
}
}
{
\newrgbcolor{curcolor}{0 0 0}
\pscustom[linestyle=none,fillstyle=solid,fillcolor=curcolor]
{
\newpath
\moveto(281.14210035,326.94703851)
\lineto(281.14210035,328.12355724)
\lineto(282.16516012,328.12355724)
\lineto(282.16516012,326.94703851)
\lineto(281.14210035,326.94703851)
\moveto(281.14210035,319.79130377)
\lineto(281.14210035,325.82735642)
\lineto(282.16516012,325.82735642)
\lineto(282.16516012,319.79130377)
\lineto(281.14210035,319.79130377)
}
}
{
\newrgbcolor{curcolor}{0 0 0}
\pscustom[linestyle=none,fillstyle=solid,fillcolor=curcolor]
{
\newpath
\moveto(283.70543356,319.79130377)
\lineto(283.70543356,328.12355724)
\lineto(284.72849334,328.12355724)
\lineto(284.72849334,319.79130377)
\lineto(283.70543356,319.79130377)
}
}
{
\newrgbcolor{curcolor}{0 0 0}
\pscustom[linestyle=none,fillstyle=solid,fillcolor=curcolor]
{
\newpath
\moveto(291.70803456,323.43453329)
\lineto(286.18919546,321.07581215)
\lineto(286.18919546,322.09318826)
\lineto(290.55993415,323.90627752)
\lineto(286.18919546,325.70231578)
\lineto(286.18919546,326.71969189)
\lineto(291.70803456,324.38938908)
\lineto(291.70803456,323.43453329)
}
}
{
\newrgbcolor{curcolor}{0 0 0}
\pscustom[linestyle=none,fillstyle=solid,fillcolor=curcolor]
{
\newpath
\moveto(298.50569722,323.43453329)
\lineto(292.98685812,321.07581215)
\lineto(292.98685812,322.09318826)
\lineto(297.35759681,323.90627752)
\lineto(292.98685812,325.70231578)
\lineto(292.98685812,326.71969189)
\lineto(298.50569722,324.38938908)
\lineto(298.50569722,323.43453329)
}
}
{
\newrgbcolor{curcolor}{0 0 0}
\pscustom[linestyle=none,fillstyle=solid,fillcolor=curcolor]
{
\newpath
\moveto(256.1805799,306.58245561)
\lineto(254.60620459,306.29827234)
\curveto(254.55315257,306.61276421)(254.43190116,306.84958337)(254.24245,307.00873052)
\curveto(254.05677962,307.16786831)(253.81427681,307.24743955)(253.51494083,307.24744446)
\curveto(253.11708122,307.24743955)(252.79879627,307.10913716)(252.56008504,306.83253689)
\curveto(252.32515796,306.55971672)(252.20769566,306.10123484)(252.20769779,305.45708986)
\curveto(252.20769566,304.74094561)(252.32705252,304.2350999)(252.56576871,303.9395512)
\curveto(252.80826904,303.64399929)(253.13223764,303.49622413)(253.53767549,303.4962253)
\curveto(253.84080055,303.49622413)(254.08898703,303.58147903)(254.28223566,303.75199024)
\curveto(254.47547589,303.92628772)(254.61188372,304.22373258)(254.69145957,304.64432571)
\lineto(256.26015122,304.37719344)
\curveto(256.09721346,303.65726116)(255.78461218,303.11352438)(255.32234643,302.74598147)
\curveto(254.8600702,302.37843772)(254.24055129,302.19466606)(253.46378784,302.19466592)
\curveto(252.58092265,302.19466606)(251.87614884,302.47316539)(251.34946431,303.03016474)
\curveto(250.82656635,303.58716269)(250.565118,304.35824586)(250.56511849,305.34341655)
\curveto(250.565118,306.33994854)(250.8284609,307.11482082)(251.35514798,307.6680357)
\curveto(251.8818325,308.22502902)(252.59418452,308.50352834)(253.49220617,308.50353452)
\curveto(254.22728941,308.50352834)(254.81081181,308.34438587)(255.24277511,308.02610662)
\curveto(255.67851719,307.71160509)(255.99111848,307.23038857)(256.1805799,306.58245561)
}
}
{
\newrgbcolor{curcolor}{0 0 0}
\pscustom[linestyle=none,fillstyle=solid,fillcolor=curcolor]
{
\newpath
\moveto(257.0274459,305.4343552)
\curveto(257.02744543,305.964827)(257.1581696,306.47825093)(257.41961881,306.97462852)
\curveto(257.6810663,307.47099683)(258.05050418,307.84990748)(258.52793356,308.1113616)
\curveto(259.00914812,308.37280417)(259.54530668,308.50352834)(260.13641087,308.50353452)
\curveto(261.04958195,308.50352834)(261.79793048,308.20608348)(262.3814587,307.61119905)
\curveto(262.96497528,307.02009316)(263.25673648,306.27174463)(263.25674317,305.36615122)
\curveto(263.25673648,304.45297352)(262.96118617,303.69515222)(262.37009137,303.09268506)
\curveto(261.78277406,302.49400547)(261.04200374,302.19466606)(260.1477782,302.19466592)
\curveto(259.59456507,302.19466606)(259.06598471,302.31970657)(258.56203555,302.56978784)
\curveto(258.0618715,302.81986863)(257.6810663,303.1855174)(257.41961881,303.66673526)
\curveto(257.1581696,304.15173955)(257.02744543,304.74094561)(257.0274459,305.4343552)
\moveto(258.66434153,305.34910022)
\curveto(258.66433943,304.75041838)(258.80643092,304.29193649)(259.09061644,303.97365319)
\curveto(259.37479689,303.65536661)(259.72528924,303.49622413)(260.14209453,303.4962253)
\curveto(260.55889267,303.49622413)(260.90749046,303.65536661)(261.18788897,303.97365319)
\curveto(261.47206733,304.29193649)(261.61415882,304.75420748)(261.61416387,305.36046755)
\curveto(261.61415882,305.95156513)(261.47206733,306.40625791)(261.18788897,306.72454725)
\curveto(260.90749046,307.0428278)(260.55889267,307.20197027)(260.14209453,307.20197514)
\curveto(259.72528924,307.20197027)(259.37479689,307.0428278)(259.09061644,306.72454725)
\curveto(258.80643092,306.40625791)(258.66433943,305.94777602)(258.66434153,305.34910022)
}
}
{
\newrgbcolor{curcolor}{0 0 0}
\pscustom[linestyle=none,fillstyle=solid,fillcolor=curcolor]
{
\newpath
\moveto(264.39347587,308.36712655)
\lineto(265.86554521,308.36712655)
\lineto(265.86554521,307.54299506)
\curveto(266.39222882,308.18334885)(267.01932594,308.50352834)(267.74683845,308.50353452)
\curveto(268.13332324,308.50352834)(268.46865917,308.42395711)(268.75284723,308.26482057)
\curveto(269.03702514,308.10567216)(269.27005519,307.8650639)(269.45193807,307.54299506)
\curveto(269.71716975,307.8650639)(270.00324729,308.10567216)(270.31017155,308.26482057)
\curveto(270.61708254,308.42395711)(270.94484025,308.50352834)(271.29344566,308.50353452)
\curveto(271.7367635,308.50352834)(272.11188504,308.41258979)(272.41881141,308.23071858)
\curveto(272.72572029,308.05262467)(272.95496123,307.78928177)(273.10653492,307.44068909)
\curveto(273.21640958,307.18302474)(273.27135163,306.76622302)(273.27136122,306.1902827)
\lineto(273.27136122,302.33107389)
\lineto(271.67425124,302.33107389)
\lineto(271.67425124,305.78105879)
\curveto(271.67424325,306.37973416)(271.6193012,306.76622302)(271.50942495,306.94052653)
\curveto(271.36164196,307.16786831)(271.13429557,307.2815415)(270.8273851,307.28154645)
\curveto(270.60382067,307.2815415)(270.39352526,307.21333759)(270.19649824,307.0769345)
\curveto(269.99945818,306.94052192)(269.85736669,306.73969928)(269.77022333,306.47446597)
\curveto(269.68306779,306.21301348)(269.63949307,305.79810632)(269.63949903,305.22974325)
\lineto(269.63949903,302.33107389)
\lineto(268.04238905,302.33107389)
\lineto(268.04238905,305.63896715)
\curveto(268.04238469,306.22627535)(268.01396639,306.605186)(267.95713407,306.77570023)
\curveto(267.90029319,306.94620558)(267.81124919,307.07314065)(267.6900018,307.15650582)
\curveto(267.57253548,307.23986133)(267.41149846,307.2815415)(267.20689024,307.28154645)
\curveto(266.96059479,307.2815415)(266.73893206,307.21523214)(266.54190139,307.08261817)
\curveto(266.34486499,306.94999469)(266.20277349,306.75864481)(266.11562648,306.50856796)
\curveto(266.0322637,306.25848276)(265.99058353,305.8435756)(265.99058584,305.26384524)
\lineto(265.99058584,302.33107389)
\lineto(264.39347587,302.33107389)
\lineto(264.39347587,308.36712655)
}
}
{
\newrgbcolor{curcolor}{0 0 0}
\pscustom[linestyle=none,fillstyle=solid,fillcolor=curcolor]
{
\newpath
\moveto(274.73774683,308.36712655)
\lineto(276.20981617,308.36712655)
\lineto(276.20981617,307.54299506)
\curveto(276.73649978,308.18334885)(277.3635969,308.50352834)(278.09110942,308.50353452)
\curveto(278.47759421,308.50352834)(278.81293013,308.42395711)(279.09711819,308.26482057)
\curveto(279.3812961,308.10567216)(279.61432615,307.8650639)(279.79620904,307.54299506)
\curveto(280.06144072,307.8650639)(280.34751826,308.10567216)(280.65444251,308.26482057)
\curveto(280.9613535,308.42395711)(281.28911121,308.50352834)(281.63771663,308.50353452)
\curveto(282.08103447,308.50352834)(282.45615601,308.41258979)(282.76308238,308.23071858)
\curveto(283.06999126,308.05262467)(283.2992322,307.78928177)(283.45080589,307.44068909)
\curveto(283.56068055,307.18302474)(283.61562259,306.76622302)(283.61563219,306.1902827)
\lineto(283.61563219,302.33107389)
\lineto(282.01852221,302.33107389)
\lineto(282.01852221,305.78105879)
\curveto(282.01851421,306.37973416)(281.96357217,306.76622302)(281.85369591,306.94052653)
\curveto(281.70591293,307.16786831)(281.47856654,307.2815415)(281.17165606,307.28154645)
\curveto(280.94809163,307.2815415)(280.73779622,307.21333759)(280.5407692,307.0769345)
\curveto(280.34372915,306.94052192)(280.20163766,306.73969928)(280.1144943,306.47446597)
\curveto(280.02733876,306.21301348)(279.98376403,305.79810632)(279.98377,305.22974325)
\lineto(279.98377,302.33107389)
\lineto(278.38666002,302.33107389)
\lineto(278.38666002,305.63896715)
\curveto(278.38665565,306.22627535)(278.35823735,306.605186)(278.30140504,306.77570023)
\curveto(278.24456416,306.94620558)(278.15552016,307.07314065)(278.03427276,307.15650582)
\curveto(277.91680645,307.23986133)(277.75576943,307.2815415)(277.55116121,307.28154645)
\curveto(277.30486575,307.2815415)(277.08320303,307.21523214)(276.88617235,307.08261817)
\curveto(276.68913595,306.94999469)(276.54704446,306.75864481)(276.45989745,306.50856796)
\curveto(276.37653467,306.25848276)(276.3348545,305.8435756)(276.33485681,305.26384524)
\lineto(276.33485681,302.33107389)
\lineto(274.73774683,302.33107389)
\lineto(274.73774683,308.36712655)
}
}
{
\newrgbcolor{curcolor}{0 0 0}
\pscustom[linestyle=none,fillstyle=solid,fillcolor=curcolor]
{
\newpath
\moveto(288.69683077,304.2521528)
\lineto(290.28825708,303.98502052)
\curveto(290.08363941,303.40149647)(289.7596708,302.95627646)(289.3163503,302.64935915)
\curveto(288.876809,302.34623032)(288.325494,302.19466606)(287.66240367,302.19466592)
\curveto(286.61281788,302.19466606)(285.83605105,302.5375802)(285.33210085,303.22340936)
\curveto(284.93424371,303.77282891)(284.73531562,304.46623539)(284.73531599,305.3036309)
\curveto(284.73531562,306.30395203)(284.99676396,307.08640252)(285.51966181,307.65098471)
\curveto(286.04255735,308.21934536)(286.70375643,308.50352834)(287.50326104,308.50353452)
\curveto(288.40127613,308.50352834)(289.10983904,308.20608348)(289.62895189,307.61119905)
\curveto(290.14805422,307.02009316)(290.39624069,306.11260216)(290.37351206,304.88872332)
\lineto(286.37221162,304.88872332)
\curveto(286.38357693,304.41508245)(286.51240655,304.04564457)(286.75870087,303.78040857)
\curveto(287.0049904,303.51895877)(287.31190802,303.3882346)(287.67945466,303.38823566)
\curveto(287.92953238,303.3882346)(288.13982779,303.45643852)(288.31034152,303.59284761)
\curveto(288.48084737,303.72925418)(288.60967699,303.94902236)(288.69683077,304.2521528)
\moveto(288.78776942,305.86631377)
\curveto(288.77639767,306.32858123)(288.65704082,306.67907357)(288.4296985,306.91779187)
\curveto(288.20234804,307.1602901)(287.92574327,307.2815415)(287.59988335,307.28154645)
\curveto(287.25128232,307.2815415)(286.96331023,307.15460644)(286.73596621,306.90074087)
\curveto(286.50861745,306.64686617)(286.39683881,306.30205748)(286.40062995,305.86631377)
\lineto(288.78776942,305.86631377)
}
}
{
\newrgbcolor{curcolor}{0 0 0}
\pscustom[linestyle=none,fillstyle=solid,fillcolor=curcolor]
{
\newpath
\moveto(297.1711744,302.33107389)
\lineto(295.57406442,302.33107389)
\lineto(295.57406442,305.41162054)
\curveto(295.57405969,306.06334377)(295.53995773,306.48393459)(295.47175844,306.67339426)
\curveto(295.4035499,306.86663434)(295.29177126,307.01630405)(295.13642218,307.12240382)
\curveto(294.98485363,307.22849401)(294.80108197,307.2815415)(294.58510664,307.28154645)
\curveto(294.30849813,307.2815415)(294.06031165,307.20575937)(293.84054647,307.05419984)
\curveto(293.6207753,306.90263086)(293.46921104,306.70180821)(293.38585324,306.45173131)
\curveto(293.30627946,306.20164616)(293.26649385,305.73937517)(293.26649627,305.06491695)
\lineto(293.26649627,302.33107389)
\lineto(291.66938629,302.33107389)
\lineto(291.66938629,308.36712655)
\lineto(293.15282296,308.36712655)
\lineto(293.15282296,307.48047474)
\curveto(293.67950645,308.16250876)(294.34260009,308.50352834)(295.14210585,308.50353452)
\curveto(295.49448845,308.50352834)(295.8165625,308.43911353)(296.10832897,308.31028989)
\curveto(296.4000849,308.1852434)(296.61985308,308.02420637)(296.76763415,307.82717833)
\curveto(296.91919249,307.6301393)(297.02339292,307.40658202)(297.08023575,307.15650582)
\curveto(297.14085522,306.90641996)(297.17116807,306.5483494)(297.1711744,306.08229306)
\lineto(297.1711744,302.33107389)
}
}
{
\newrgbcolor{curcolor}{0 0 0}
\pscustom[linestyle=none,fillstyle=solid,fillcolor=curcolor]
{
\newpath
\moveto(301.56464692,308.36712655)
\lineto(301.56464692,307.0939855)
\lineto(300.47338316,307.0939855)
\lineto(300.47338316,304.66137671)
\curveto(300.47338065,304.16879053)(300.48285342,303.88081844)(300.50180149,303.79745957)
\curveto(300.52453359,303.71788686)(300.57189742,303.6515775)(300.64389313,303.59853128)
\curveto(300.71967257,303.54548252)(300.81061113,303.51895877)(300.91670906,303.51895996)
\curveto(301.06448126,303.51895877)(301.27856578,303.57011171)(301.55896325,303.67241893)
\lineto(301.69537122,302.43337987)
\curveto(301.32403505,302.2742373)(300.90344424,302.19466606)(300.43359751,302.19466592)
\curveto(300.14562294,302.19466606)(299.88606915,302.24202989)(299.65493535,302.33675756)
\curveto(299.42379816,302.43527432)(299.25328837,302.56031483)(299.14340546,302.71187947)
\curveto(299.0373093,302.86723246)(298.96342172,303.07563332)(298.92174251,303.33708267)
\curveto(298.88763959,303.52274788)(298.87058861,303.89786942)(298.87058952,304.46244842)
\lineto(298.87058952,307.0939855)
\lineto(298.13739668,307.0939855)
\lineto(298.13739668,308.36712655)
\lineto(298.87058952,308.36712655)
\lineto(298.87058952,309.56637994)
\lineto(300.47338316,310.49850107)
\lineto(300.47338316,308.36712655)
\lineto(301.56464692,308.36712655)
}
}
{
\newrgbcolor{curcolor}{0 0 0}
\pscustom[linestyle=none,fillstyle=solid,fillcolor=curcolor]
{
\newpath
\moveto(302.12164808,304.05322451)
\lineto(303.72444172,304.29762212)
\curveto(303.79264377,303.98691342)(303.93094615,303.75009427)(304.1393493,303.58716395)
\curveto(304.34774786,303.42802022)(304.63950906,303.34844898)(305.01463377,303.34845)
\curveto(305.42764321,303.34844898)(305.73834994,303.42423111)(305.9467549,303.57579661)
\curveto(306.08694774,303.68189035)(306.15704621,303.82398184)(306.15705052,304.00207152)
\curveto(306.15704621,304.12332126)(306.11915514,304.22373258)(306.04337721,304.30330579)
\curveto(305.96380178,304.37908594)(305.78571377,304.44918441)(305.50911266,304.51360141)
\curveto(304.2208128,304.79778221)(303.40426035,305.057336)(303.05945287,305.29226356)
\curveto(302.58202425,305.61812376)(302.34331054,306.07092198)(302.34331103,306.6506596)
\curveto(302.34331054,307.17355197)(302.54981684,307.61308832)(302.96283056,307.96926997)
\curveto(303.37584205,308.32544034)(304.01620105,308.50352834)(304.88390947,308.50353452)
\curveto(305.70993164,308.50352834)(306.32376689,308.36901506)(306.72541706,308.09999427)
\curveto(307.12705747,307.83096194)(307.40366224,307.43310576)(307.5552322,306.90642454)
\lineto(306.04906087,306.62792493)
\curveto(305.98464186,306.86284524)(305.8614959,307.0428278)(305.67962262,307.16787315)
\curveto(305.50153079,307.29290882)(305.2457661,307.35542908)(304.91232779,307.3554341)
\curveto(304.49173391,307.35542908)(304.19049995,307.29669793)(304.00862499,307.17924048)
\curveto(303.88737143,307.09587529)(303.82674572,306.98788575)(303.8267477,306.85527155)
\curveto(303.82674572,306.74159383)(303.87979321,306.64497162)(303.98589033,306.56540461)
\curveto(304.12987424,306.4593054)(304.62624719,306.30963569)(305.47501067,306.11639505)
\curveto(306.327556,305.92314683)(306.92244572,305.68632768)(307.2596816,305.40593687)
\curveto(307.59311756,305.12175081)(307.75983825,304.72578919)(307.75984416,304.2180508)
\curveto(307.75983825,303.66483937)(307.52870275,303.18930651)(307.06643698,302.79145079)
\curveto(306.60416077,302.39359415)(305.92022705,302.19466606)(305.01463377,302.19466592)
\curveto(304.1923945,302.19466606)(303.54066819,302.36138674)(303.05945287,302.69482848)
\curveto(302.58202425,303.02826948)(302.26942296,303.48106771)(302.12164808,304.05322451)
}
}
{
\newrgbcolor{curcolor}{0.50196081 0.50196081 0}
\pscustom[linewidth=2.32802935,linecolor=curcolor]
{
\newpath
\moveto(464.26070507,295.92182848)
\lineto(538.75764202,295.92182848)
\lineto(538.75764202,342.4824199)
\lineto(488.70501091,342.4824199)
\lineto(488.70501091,351.79453818)
\lineto(464.26070507,351.79453818)
\lineto(464.26070507,295.92182848)
\closepath
}
}
{
\newrgbcolor{curcolor}{0 0 0}
\pscustom[linestyle=none,fillstyle=solid,fillcolor=curcolor]
{
\newpath
\moveto(480.02946193,323.43453329)
\lineto(480.02946193,324.38938908)
\lineto(485.54830103,326.71969189)
\lineto(485.54830103,325.70231578)
\lineto(481.17187867,323.90627752)
\lineto(485.54830103,322.09318826)
\lineto(485.54830103,321.07581215)
\lineto(480.02946193,323.43453329)
}
}
{
\newrgbcolor{curcolor}{0 0 0}
\pscustom[linestyle=none,fillstyle=solid,fillcolor=curcolor]
{
\newpath
\moveto(486.82712592,323.43453329)
\lineto(486.82712592,324.38938908)
\lineto(492.34596502,326.71969189)
\lineto(492.34596502,325.70231578)
\lineto(487.96954267,323.90627752)
\lineto(492.34596502,322.09318826)
\lineto(492.34596502,321.07581215)
\lineto(486.82712592,323.43453329)
}
}
{
\newrgbcolor{curcolor}{0 0 0}
\pscustom[linestyle=none,fillstyle=solid,fillcolor=curcolor]
{
\newpath
\moveto(497.71134534,319.79130377)
\lineto(497.71134534,320.67795557)
\curveto(497.24149141,319.99591552)(496.60302697,319.65489594)(495.7959501,319.6548958)
\curveto(495.43977128,319.65489594)(495.10632991,319.72309985)(494.79562499,319.85950775)
\curveto(494.48870556,319.99591552)(494.25946461,320.16642531)(494.10790147,320.37103764)
\curveto(493.9601252,320.57943792)(493.85592477,320.83330805)(493.79529988,321.1326488)
\curveto(493.7536189,321.33347011)(493.73277881,321.65175505)(493.73277956,322.08750459)
\lineto(493.73277956,325.82735642)
\lineto(494.75583933,325.82735642)
\lineto(494.75583933,322.4796775)
\curveto(494.75583756,321.9454108)(494.77667765,321.58544569)(494.81835965,321.39978108)
\curveto(494.88277263,321.13075291)(495.01918046,320.91856295)(495.22758356,320.76321055)
\curveto(495.43598218,320.61164532)(495.69364142,320.53586319)(496.00056205,320.53586394)
\curveto(496.30747666,320.53586319)(496.59544876,320.61353988)(496.86447919,320.76889422)
\curveto(497.13350188,320.92803571)(497.3229572,321.14212023)(497.43284573,321.41114841)
\curveto(497.54651448,321.68396245)(497.60335108,322.07802953)(497.60335569,322.59335081)
\lineto(497.60335569,325.82735642)
\lineto(498.62641547,325.82735642)
\lineto(498.62641547,319.79130377)
\lineto(497.71134534,319.79130377)
}
}
{
\newrgbcolor{curcolor}{0 0 0}
\pscustom[linestyle=none,fillstyle=solid,fillcolor=curcolor]
{
\newpath
\moveto(502.46857268,320.7063739)
\lineto(502.61634798,319.8026711)
\curveto(502.32837274,319.74204539)(502.0707135,319.71173253)(501.84336949,319.71173245)
\curveto(501.47203468,319.71173253)(501.18406259,319.77046368)(500.97945235,319.88792608)
\curveto(500.77483909,320.00538829)(500.63085304,320.1588471)(500.54749378,320.34830298)
\curveto(500.46413236,320.54154685)(500.42245218,320.94508669)(500.42245314,321.55892371)
\lineto(500.42245314,325.03164327)
\lineto(499.67220931,325.03164327)
\lineto(499.67220931,325.82735642)
\lineto(500.42245314,325.82735642)
\lineto(500.42245314,327.32216042)
\lineto(501.43982925,327.93599629)
\lineto(501.43982925,325.82735642)
\lineto(502.46857268,325.82735642)
\lineto(502.46857268,325.03164327)
\lineto(501.43982925,325.03164327)
\lineto(501.43982925,321.50208705)
\curveto(501.43982727,321.21032415)(501.45687825,321.02276337)(501.49098223,320.93940418)
\curveto(501.52887128,320.85604269)(501.58760243,320.78973333)(501.66717586,320.74047589)
\curveto(501.750534,320.69121656)(501.8679963,320.66658737)(502.01956312,320.66658824)
\curveto(502.13323376,320.66658737)(502.28290346,320.67984924)(502.46857268,320.7063739)
}
}
{
\newrgbcolor{curcolor}{0 0 0}
\pscustom[linestyle=none,fillstyle=solid,fillcolor=curcolor]
{
\newpath
\moveto(503.46889856,326.94703851)
\lineto(503.46889856,328.12355724)
\lineto(504.49195833,328.12355724)
\lineto(504.49195833,326.94703851)
\lineto(503.46889856,326.94703851)
\moveto(503.46889856,319.79130377)
\lineto(503.46889856,325.82735642)
\lineto(504.49195833,325.82735642)
\lineto(504.49195833,319.79130377)
\lineto(503.46889856,319.79130377)
}
}
{
\newrgbcolor{curcolor}{0 0 0}
\pscustom[linestyle=none,fillstyle=solid,fillcolor=curcolor]
{
\newpath
\moveto(506.03223178,319.79130377)
\lineto(506.03223178,328.12355724)
\lineto(507.05529155,328.12355724)
\lineto(507.05529155,319.79130377)
\lineto(506.03223178,319.79130377)
}
}
{
\newrgbcolor{curcolor}{0 0 0}
\pscustom[linestyle=none,fillstyle=solid,fillcolor=curcolor]
{
\newpath
\moveto(514.03483277,323.43453329)
\lineto(508.51599367,321.07581215)
\lineto(508.51599367,322.09318826)
\lineto(512.88673236,323.90627752)
\lineto(508.51599367,325.70231578)
\lineto(508.51599367,326.71969189)
\lineto(514.03483277,324.38938908)
\lineto(514.03483277,323.43453329)
}
}
{
\newrgbcolor{curcolor}{0 0 0}
\pscustom[linestyle=none,fillstyle=solid,fillcolor=curcolor]
{
\newpath
\moveto(520.83249544,323.43453329)
\lineto(515.31365633,321.07581215)
\lineto(515.31365633,322.09318826)
\lineto(519.68439502,323.90627752)
\lineto(515.31365633,325.70231578)
\lineto(515.31365633,326.71969189)
\lineto(520.83249544,324.38938908)
\lineto(520.83249544,323.43453329)
}
}
{
\newrgbcolor{curcolor}{0 0 0}
\pscustom[linestyle=none,fillstyle=solid,fillcolor=curcolor]
{
\newpath
\moveto(484.08533444,302.33107389)
\lineto(482.48822447,302.33107389)
\lineto(482.48822447,308.36712655)
\lineto(483.97166113,308.36712655)
\lineto(483.97166113,307.50889307)
\curveto(484.22552902,307.91432229)(484.45287541,308.18145429)(484.65370098,308.31028989)
\curveto(484.8583098,308.43911353)(485.08944529,308.50352834)(485.34710816,308.50353452)
\curveto(485.71085876,308.50352834)(486.06135111,308.40311702)(486.39858626,308.20230025)
\lineto(485.90410737,306.80980223)
\curveto(485.63507663,306.98409665)(485.3849956,307.07124609)(485.15386354,307.07125083)
\curveto(484.93030282,307.07124609)(484.7408475,307.00872584)(484.585497,306.88368988)
\curveto(484.43014077,306.76243392)(484.30699481,306.54077119)(484.21605875,306.21870102)
\curveto(484.1289068,305.89662309)(484.08533208,305.22216213)(484.08533444,304.19531614)
\lineto(484.08533444,302.33107389)
}
}
{
\newrgbcolor{curcolor}{0 0 0}
\pscustom[linestyle=none,fillstyle=solid,fillcolor=curcolor]
{
\newpath
\moveto(488.27419578,306.52561896)
\lineto(486.82486111,306.78706756)
\curveto(486.9877921,307.37058551)(487.26818598,307.80254364)(487.66604358,308.08294328)
\curveto(488.06389834,308.3633314)(488.65499895,308.50352834)(489.43934719,308.50353452)
\curveto(490.15169601,308.50352834)(490.68217092,308.41827345)(491.0307735,308.24776957)
\curveto(491.37936651,308.08104297)(491.62376388,307.86695845)(491.76396634,307.60551538)
\curveto(491.90794686,307.34785087)(491.97993989,306.872318)(491.97994562,306.17891537)
\lineto(491.96289463,304.31467312)
\curveto(491.96288891,303.78419623)(491.9875181,303.39202371)(492.03678228,303.13815438)
\curveto(492.08982397,302.88807254)(492.18644619,302.61904598)(492.32664921,302.33107389)
\lineto(490.74659023,302.33107389)
\curveto(490.70490556,302.43716887)(490.65375262,302.59441679)(490.59313126,302.80281812)
\curveto(490.56660317,302.89754531)(490.54765764,302.96006557)(490.53629461,302.99037908)
\curveto(490.26347465,302.72514097)(489.97171345,302.52621288)(489.66101014,302.39359421)
\curveto(489.35029999,302.26097542)(489.01875318,302.19466606)(488.66636869,302.19466592)
\curveto(488.04495281,302.19466606)(487.55426352,302.3632813)(487.19429936,302.70051214)
\curveto(486.8381224,303.03774225)(486.66003439,303.46401673)(486.66003481,303.97933686)
\curveto(486.66003439,304.32035479)(486.74150018,304.62348331)(486.90443242,304.88872332)
\curveto(487.06736334,305.15774732)(487.29470973,305.36235907)(487.58647227,305.50255918)
\curveto(487.88202123,305.64654206)(488.30640116,305.77158257)(488.85961332,305.8776811)
\curveto(489.60606468,306.01787449)(490.12327771,306.14859867)(490.41125397,306.26985401)
\lineto(490.41125397,306.42899664)
\curveto(490.41124981,306.73591017)(490.33546768,306.95378379)(490.18390736,307.08261817)
\curveto(490.03233916,307.21523214)(489.74626162,307.2815415)(489.32567388,307.28154645)
\curveto(489.04148781,307.2815415)(488.81982509,307.22470491)(488.66068503,307.11103649)
\curveto(488.50154014,307.00114762)(488.37271052,306.80600864)(488.27419578,306.52561896)
\moveto(490.41125397,305.22974325)
\curveto(490.20663806,305.16153643)(489.88266945,305.08007064)(489.43934719,304.98534563)
\curveto(488.99601854,304.89061532)(488.70615189,304.79778221)(488.56974638,304.70684603)
\curveto(488.3613432,304.5590685)(488.25714277,304.37150773)(488.25714479,304.14416315)
\curveto(488.25714277,303.92060406)(488.34050312,303.72735963)(488.50722606,303.56442928)
\curveto(488.67394449,303.40149647)(488.88613445,303.32003068)(489.14379659,303.32003167)
\curveto(489.43176578,303.32003068)(489.706476,303.41475834)(489.96792807,303.60421494)
\curveto(490.16116878,303.74819971)(490.28810385,303.92439317)(490.34873365,304.13279582)
\curveto(490.39040972,304.26920186)(490.41124981,304.52875565)(490.41125397,304.91145798)
\lineto(490.41125397,305.22974325)
}
}
{
\newrgbcolor{curcolor}{0 0 0}
\pscustom[linestyle=none,fillstyle=solid,fillcolor=curcolor]
{
\newpath
\moveto(496.32794906,308.36712655)
\lineto(496.32794906,307.0939855)
\lineto(495.2366853,307.0939855)
\lineto(495.2366853,304.66137671)
\curveto(495.23668279,304.16879053)(495.24615556,303.88081844)(495.26510363,303.79745957)
\curveto(495.28783573,303.71788686)(495.33519956,303.6515775)(495.40719526,303.59853128)
\curveto(495.48297471,303.54548252)(495.57391327,303.51895877)(495.6800112,303.51895996)
\curveto(495.8277834,303.51895877)(496.04186792,303.57011171)(496.32226539,303.67241893)
\lineto(496.45867336,302.43337987)
\curveto(496.08733719,302.2742373)(495.66674637,302.19466606)(495.19689964,302.19466592)
\curveto(494.90892508,302.19466606)(494.64937129,302.24202989)(494.41823748,302.33675756)
\curveto(494.1871003,302.43527432)(494.0165905,302.56031483)(493.9067076,302.71187947)
\curveto(493.80061143,302.86723246)(493.72672386,303.07563332)(493.68504465,303.33708267)
\curveto(493.65094173,303.52274788)(493.63389075,303.89786942)(493.63389166,304.46244842)
\lineto(493.63389166,307.0939855)
\lineto(492.90069882,307.0939855)
\lineto(492.90069882,308.36712655)
\lineto(493.63389166,308.36712655)
\lineto(493.63389166,309.56637994)
\lineto(495.2366853,310.49850107)
\lineto(495.2366853,308.36712655)
\lineto(496.32794906,308.36712655)
}
}
{
\newrgbcolor{curcolor}{0 0 0}
\pscustom[linestyle=none,fillstyle=solid,fillcolor=curcolor]
{
\newpath
\moveto(497.44763132,309.18557436)
\lineto(497.44763132,310.66332737)
\lineto(499.0447413,310.66332737)
\lineto(499.0447413,309.18557436)
\lineto(497.44763132,309.18557436)
\moveto(497.44763132,302.33107389)
\lineto(497.44763132,308.36712655)
\lineto(499.0447413,308.36712655)
\lineto(499.0447413,302.33107389)
\lineto(497.44763132,302.33107389)
}
}
{
\newrgbcolor{curcolor}{0 0 0}
\pscustom[linestyle=none,fillstyle=solid,fillcolor=curcolor]
{
\newpath
\moveto(506.16637481,302.33107389)
\lineto(504.56926483,302.33107389)
\lineto(504.56926483,305.41162054)
\curveto(504.5692601,306.06334377)(504.53515815,306.48393459)(504.46695886,306.67339426)
\curveto(504.39875031,306.86663434)(504.28697167,307.01630405)(504.1316226,307.12240382)
\curveto(503.98005405,307.22849401)(503.79628238,307.2815415)(503.58030705,307.28154645)
\curveto(503.30369854,307.2815415)(503.05551207,307.20575937)(502.83574689,307.05419984)
\curveto(502.61597572,306.90263086)(502.46441146,306.70180821)(502.38105365,306.45173131)
\curveto(502.30147988,306.20164616)(502.26169426,305.73937517)(502.26169668,305.06491695)
\lineto(502.26169668,302.33107389)
\lineto(500.6645867,302.33107389)
\lineto(500.6645867,308.36712655)
\lineto(502.14802337,308.36712655)
\lineto(502.14802337,307.48047474)
\curveto(502.67470687,308.16250876)(503.3378005,308.50352834)(504.13730626,308.50353452)
\curveto(504.48968887,308.50352834)(504.81176292,308.43911353)(505.10352938,308.31028989)
\curveto(505.39528532,308.1852434)(505.61505349,308.02420637)(505.76283457,307.82717833)
\curveto(505.9143929,307.6301393)(506.01859333,307.40658202)(506.07543616,307.15650582)
\curveto(506.13605563,306.90641996)(506.16636848,306.5483494)(506.16637481,306.08229306)
\lineto(506.16637481,302.33107389)
}
}
{
\newrgbcolor{curcolor}{0 0 0}
\pscustom[linestyle=none,fillstyle=solid,fillcolor=curcolor]
{
\newpath
\moveto(507.64412698,301.93321731)
\lineto(509.46858358,301.71155436)
\curveto(509.49889392,301.49936502)(509.56899239,301.35348442)(509.6788792,301.27391213)
\curveto(509.83044073,301.16023999)(510.06915444,301.10340339)(510.39502104,301.10340217)
\curveto(510.81181931,301.10340339)(511.12442059,301.16592365)(511.33282583,301.29096312)
\curveto(511.47301839,301.37432451)(511.57911337,301.50883779)(511.65111109,301.69450337)
\curveto(511.70036478,301.82712273)(511.72499397,302.0715201)(511.72499874,302.4276962)
\lineto(511.72499874,303.30866434)
\curveto(511.24756656,302.65693705)(510.64509863,302.33107389)(509.91759314,302.33107389)
\curveto(509.1067214,302.33107389)(508.46446785,302.67398803)(507.99083057,303.35981733)
\curveto(507.6194971,303.90165853)(507.43383089,304.57611948)(507.43383136,305.38320221)
\curveto(507.43383089,306.39489059)(507.6763337,307.16786831)(508.16134054,307.70213769)
\curveto(508.65013407,308.23639634)(509.2563911,308.50352834)(509.98011346,308.50353452)
\curveto(510.72656441,308.50352834)(511.34229422,308.17577063)(511.82730472,307.5202604)
\lineto(511.82730472,308.36712655)
\lineto(513.32210872,308.36712655)
\lineto(513.32210872,302.95059342)
\curveto(513.32210235,302.23824078)(513.2633712,301.70587132)(513.14591509,301.35348344)
\curveto(513.0284466,301.00109752)(512.86362047,300.72449275)(512.6514362,300.5236683)
\curveto(512.43924054,300.32284746)(512.15505756,300.16559954)(511.79888639,300.05192407)
\curveto(511.44649465,299.93825315)(510.99938008,299.88141656)(510.45754136,299.88141411)
\curveto(509.43447911,299.88141656)(508.70886522,300.05761001)(508.28069751,300.40999499)
\curveto(507.85252715,300.7585947)(507.63844264,301.20192016)(507.63844332,301.73997269)
\curveto(507.63844264,301.79302077)(507.64033719,301.85743558)(507.64412698,301.93321731)
\moveto(509.070727,305.47414086)
\curveto(509.07072488,304.83377872)(509.19387085,304.36392952)(509.44016525,304.06459184)
\curveto(509.69024379,303.7690398)(509.99716142,303.62126465)(510.36091904,303.62126594)
\curveto(510.75119361,303.62126465)(511.08084587,303.77282891)(511.34987682,304.07595917)
\curveto(511.61889899,304.38287505)(511.75341227,304.83567327)(511.75341707,305.4343552)
\curveto(511.75341227,306.05955467)(511.62458265,306.52372021)(511.36692782,306.82685322)
\curveto(511.10926417,307.12997724)(510.78340101,307.2815415)(510.38933737,307.28154645)
\curveto(510.00663418,307.2815415)(509.69024379,307.1318718)(509.44016525,306.83253689)
\curveto(509.19387085,306.53698208)(509.07072488,306.08418386)(509.070727,305.47414086)
}
}
{
\newrgbcolor{curcolor}{0 0 0}
\pscustom[linestyle=none,fillstyle=solid,fillcolor=curcolor]
{
\newpath
\moveto(514.34516766,304.05322451)
\lineto(515.9479613,304.29762212)
\curveto(516.01616334,303.98691342)(516.15446573,303.75009427)(516.36286888,303.58716395)
\curveto(516.57126744,303.42802022)(516.86302864,303.34844898)(517.23815335,303.34845)
\curveto(517.65116279,303.34844898)(517.96186952,303.42423111)(518.17027447,303.57579661)
\curveto(518.31046731,303.68189035)(518.38056578,303.82398184)(518.38057009,304.00207152)
\curveto(518.38056578,304.12332126)(518.34267472,304.22373258)(518.26689678,304.30330579)
\curveto(518.18732135,304.37908594)(518.00923335,304.44918441)(517.73263224,304.51360141)
\curveto(516.44433237,304.79778221)(515.62777993,305.057336)(515.28297245,305.29226356)
\curveto(514.80554382,305.61812376)(514.56683012,306.07092198)(514.56683061,306.6506596)
\curveto(514.56683012,307.17355197)(514.77333642,307.61308832)(515.18635014,307.96926997)
\curveto(515.59936163,308.32544034)(516.23972062,308.50352834)(517.10742904,308.50353452)
\curveto(517.93345122,308.50352834)(518.54728647,308.36901506)(518.94893663,308.09999427)
\curveto(519.35057704,307.83096194)(519.62718181,307.43310576)(519.77875178,306.90642454)
\lineto(518.27258045,306.62792493)
\curveto(518.20816144,306.86284524)(518.08501548,307.0428278)(517.9031422,307.16787315)
\curveto(517.72505036,307.29290882)(517.46928568,307.35542908)(517.13584737,307.3554341)
\curveto(516.71525349,307.35542908)(516.41401952,307.29669793)(516.23214457,307.17924048)
\curveto(516.110891,307.09587529)(516.0502653,306.98788575)(516.05026728,306.85527155)
\curveto(516.0502653,306.74159383)(516.10331279,306.64497162)(516.20940991,306.56540461)
\curveto(516.35339382,306.4593054)(516.84976677,306.30963569)(517.69853024,306.11639505)
\curveto(518.55107558,305.92314683)(519.14596529,305.68632768)(519.48320118,305.40593687)
\curveto(519.81663714,305.12175081)(519.98335782,304.72578919)(519.98336373,304.2180508)
\curveto(519.98335782,303.66483937)(519.75222233,303.18930651)(519.28995656,302.79145079)
\curveto(518.82768035,302.39359415)(518.14374663,302.19466606)(517.23815335,302.19466592)
\curveto(516.41591408,302.19466606)(515.76418776,302.36138674)(515.28297245,302.69482848)
\curveto(514.80554382,303.02826948)(514.49294254,303.48106771)(514.34516766,304.05322451)
}
}
{
\newrgbcolor{curcolor}{0 1 0.25098041}
\pscustom[linewidth=2.32802935,linecolor=curcolor]
{
\newpath
\moveto(18.44308422,295.92182848)
\lineto(92.94002116,295.92182848)
\lineto(92.94002116,342.4824199)
\lineto(44.0514067,342.4824199)
\lineto(44.0514067,351.79453818)
\lineto(18.44308422,351.79453818)
\lineto(18.44308422,295.92182848)
\closepath
}
}
{
\newrgbcolor{curcolor}{0 0 0}
\pscustom[linestyle=none,fillstyle=solid,fillcolor=curcolor]
{
\newpath
\moveto(31.88381527,323.43453329)
\lineto(31.88381527,324.38938908)
\lineto(37.40265437,326.71969189)
\lineto(37.40265437,325.70231578)
\lineto(33.02623201,323.90627752)
\lineto(37.40265437,322.09318826)
\lineto(37.40265437,321.07581215)
\lineto(31.88381527,323.43453329)
}
}
{
\newrgbcolor{curcolor}{0 0 0}
\pscustom[linestyle=none,fillstyle=solid,fillcolor=curcolor]
{
\newpath
\moveto(38.68147926,323.43453329)
\lineto(38.68147926,324.38938908)
\lineto(44.20031836,326.71969189)
\lineto(44.20031836,325.70231578)
\lineto(39.823896,323.90627752)
\lineto(44.20031836,322.09318826)
\lineto(44.20031836,321.07581215)
\lineto(38.68147926,323.43453329)
}
}
{
\newrgbcolor{curcolor}{0 0 0}
\pscustom[linestyle=none,fillstyle=solid,fillcolor=curcolor]
{
\newpath
\moveto(49.54864768,320.53586394)
\curveto(49.16973232,320.21378914)(48.80408355,319.98644275)(48.45170026,319.85382409)
\curveto(48.10309885,319.7212053)(47.72797731,319.65489594)(47.32633451,319.6548958)
\curveto(46.66323839,319.65489594)(46.15360357,319.81593296)(45.79742851,320.13800736)
\curveto(45.44125155,320.46387017)(45.26316355,320.87877733)(45.26316397,321.38273008)
\curveto(45.26316355,321.6782788)(45.32947291,321.94730535)(45.46209226,322.18981057)
\curveto(45.59849947,322.43610009)(45.77469292,322.63313363)(45.99067314,322.78091177)
\curveto(46.21044017,322.92868393)(46.45673209,323.04046257)(46.72954964,323.11624803)
\curveto(46.9303704,323.16929219)(47.23349891,323.22044513)(47.6389361,323.26970699)
\curveto(48.46495852,323.36822028)(49.07311011,323.48568258)(49.4633927,323.62209425)
\curveto(49.46717718,323.76228736)(49.46907174,323.85133136)(49.46907636,323.88922652)
\curveto(49.46907174,324.30602414)(49.37244952,324.59967989)(49.17920943,324.77019466)
\curveto(48.91775674,325.00132517)(48.52937333,325.11689292)(48.01405802,325.11689825)
\curveto(47.53283833,325.11689292)(47.17666232,325.03163803)(46.94552893,324.86113331)
\curveto(46.71818043,324.69440755)(46.5495652,324.39696269)(46.4396827,323.96879784)
\lineto(45.43935759,324.10520581)
\curveto(45.53029555,324.53337053)(45.67996526,324.87817921)(45.88836716,325.13963291)
\curveto(46.09676697,325.40486501)(46.39800094,325.60758221)(46.79206996,325.74778511)
\curveto(47.18613508,325.8917652)(47.64272241,325.96375822)(48.16183332,325.96376439)
\curveto(48.67714848,325.96375822)(49.09584475,325.90313252)(49.41792337,325.7818871)
\curveto(49.73999285,325.6606297)(49.976812,325.50717089)(50.12838155,325.3215102)
\curveto(50.27994052,325.13962756)(50.3860355,324.90849207)(50.44666681,324.62810302)
\curveto(50.48076316,324.45379929)(50.49781414,324.13930345)(50.4978198,323.68461457)
\lineto(50.4978198,322.32053487)
\curveto(50.49781414,321.36946662)(50.51865423,320.76699869)(50.56034012,320.51312928)
\curveto(50.60580368,320.26304753)(50.69295313,320.02243926)(50.82178873,319.79130377)
\lineto(49.75325963,319.79130377)
\curveto(49.64715974,320.00349373)(49.57895582,320.25168021)(49.54864768,320.53586394)
\moveto(49.4633927,322.82069743)
\curveto(49.09205564,322.66913014)(48.53505699,322.54030052)(47.79239507,322.43420818)
\curveto(47.3718013,322.37357983)(47.07435644,322.30537592)(46.9000596,322.22959623)
\curveto(46.72575865,322.15381166)(46.59124537,322.04203302)(46.49651936,321.89425997)
\curveto(46.40179004,321.75027182)(46.35442621,321.58923479)(46.35442772,321.41114841)
\curveto(46.35442621,321.13833112)(46.45673209,320.91098473)(46.66134566,320.72910856)
\curveto(46.86974469,320.54723051)(47.17287321,320.45629196)(47.57073212,320.45629262)
\curveto(47.96479646,320.45629196)(48.31528881,320.54154685)(48.62221022,320.71205756)
\curveto(48.92912406,320.88635554)(49.1545759,321.1231747)(49.2985664,321.42251574)
\curveto(49.40844603,321.6536496)(49.46338808,321.99466919)(49.4633927,322.44557551)
\lineto(49.4633927,322.82069743)
}
}
{
\newrgbcolor{curcolor}{0 0 0}
\pscustom[linestyle=none,fillstyle=solid,fillcolor=curcolor]
{
\newpath
\moveto(52.08924552,317.47805195)
\lineto(52.08924552,325.82735642)
\lineto(53.02136664,325.82735642)
\lineto(53.02136664,325.0430106)
\curveto(53.24113312,325.34992297)(53.48931959,325.57916391)(53.76592681,325.73073411)
\curveto(54.04252914,325.88608154)(54.37786506,325.96375822)(54.77193559,325.96376439)
\curveto(55.28725062,325.96375822)(55.74194339,325.83113949)(56.13601528,325.56590781)
\curveto(56.53007754,325.30066459)(56.8275224,324.92554305)(57.02835075,324.44054207)
\curveto(57.22916769,323.95932089)(57.32957901,323.43074054)(57.32958501,322.85479942)
\curveto(57.32957901,322.237172)(57.21780037,321.68017335)(56.99424876,321.18380179)
\curveto(56.77447491,320.69121656)(56.45240086,320.31230591)(56.02802564,320.04706871)
\curveto(55.60743011,319.78562011)(55.16410466,319.65489594)(54.69804793,319.6548958)
\curveto(54.35702498,319.65489594)(54.05010735,319.72688896)(53.77729414,319.87087509)
\curveto(53.50826513,320.01486105)(53.2866024,320.19673816)(53.11230529,320.41650696)
\lineto(53.11230529,317.47805195)
\lineto(52.08924552,317.47805195)
\moveto(53.01568298,322.7752281)
\curveto(53.01568128,321.99845829)(53.1729292,321.42440866)(53.48742721,321.05307749)
\curveto(53.80192088,320.68174379)(54.18272608,320.49607757)(54.62984395,320.49607828)
\curveto(55.08453342,320.49607757)(55.47291683,320.68742745)(55.79499536,321.07012848)
\curveto(56.12085404,321.45661607)(56.28378562,322.05340034)(56.28379058,322.86048308)
\curveto(56.28378562,323.62966863)(56.12464315,324.20561281)(55.80636269,324.58831737)
\curveto(55.49186237,324.97101232)(55.11484627,325.1623622)(54.67531327,325.16236757)
\curveto(54.23956268,325.1623622)(53.85307381,324.95775045)(53.51584553,324.54853171)
\curveto(53.18240197,324.14309256)(53.01568128,323.55199195)(53.01568298,322.7752281)
}
}
{
\newrgbcolor{curcolor}{0 0 0}
\pscustom[linestyle=none,fillstyle=solid,fillcolor=curcolor]
{
\newpath
\moveto(58.56862437,317.47805195)
\lineto(58.56862437,325.82735642)
\lineto(59.50074549,325.82735642)
\lineto(59.50074549,325.0430106)
\curveto(59.72051197,325.34992297)(59.96869844,325.57916391)(60.24530566,325.73073411)
\curveto(60.52190799,325.88608154)(60.85724391,325.96375822)(61.25131443,325.96376439)
\curveto(61.76662947,325.96375822)(62.22132224,325.83113949)(62.61539413,325.56590781)
\curveto(63.00945639,325.30066459)(63.30690125,324.92554305)(63.5077296,324.44054207)
\curveto(63.70854653,323.95932089)(63.80895786,323.43074054)(63.80896386,322.85479942)
\curveto(63.80895786,322.237172)(63.69717921,321.68017335)(63.47362761,321.18380179)
\curveto(63.25385376,320.69121656)(62.93177971,320.31230591)(62.50740449,320.04706871)
\curveto(62.08680896,319.78562011)(61.6434835,319.65489594)(61.17742678,319.6548958)
\curveto(60.83640383,319.65489594)(60.5294862,319.72688896)(60.25667299,319.87087509)
\curveto(59.98764397,320.01486105)(59.76598125,320.19673816)(59.59168414,320.41650696)
\lineto(59.59168414,317.47805195)
\lineto(58.56862437,317.47805195)
\moveto(59.49506183,322.7752281)
\curveto(59.49506013,321.99845829)(59.65230805,321.42440866)(59.96680605,321.05307749)
\curveto(60.28129973,320.68174379)(60.66210493,320.49607757)(61.1092228,320.49607828)
\curveto(61.56391227,320.49607757)(61.95229568,320.68742745)(62.27437421,321.07012848)
\curveto(62.60023289,321.45661607)(62.76316447,322.05340034)(62.76316943,322.86048308)
\curveto(62.76316447,323.62966863)(62.604022,324.20561281)(62.28574154,324.58831737)
\curveto(61.97124121,324.97101232)(61.59422512,325.1623622)(61.15469212,325.16236757)
\curveto(60.71894152,325.1623622)(60.33245266,324.95775045)(59.99522438,324.54853171)
\curveto(59.66178082,324.14309256)(59.49506013,323.55199195)(59.49506183,322.7752281)
}
}
{
\newrgbcolor{curcolor}{0 0 0}
\pscustom[linestyle=none,fillstyle=solid,fillcolor=curcolor]
{
\newpath
\moveto(70.43611801,323.43453329)
\lineto(64.91727891,321.07581215)
\lineto(64.91727891,322.09318826)
\lineto(69.2880176,323.90627752)
\lineto(64.91727891,325.70231578)
\lineto(64.91727891,326.71969189)
\lineto(70.43611801,324.38938908)
\lineto(70.43611801,323.43453329)
}
}
{
\newrgbcolor{curcolor}{0 0 0}
\pscustom[linestyle=none,fillstyle=solid,fillcolor=curcolor]
{
\newpath
\moveto(77.23378067,323.43453329)
\lineto(71.71494157,321.07581215)
\lineto(71.71494157,322.09318826)
\lineto(76.08568026,323.90627752)
\lineto(71.71494157,325.70231578)
\lineto(71.71494157,326.71969189)
\lineto(77.23378067,324.38938908)
\lineto(77.23378067,323.43453329)
}
}
{
\newrgbcolor{curcolor}{0 0 0}
\pscustom[linestyle=none,fillstyle=solid,fillcolor=curcolor]
{
\newpath
\moveto(47.65484912,308.36712655)
\lineto(47.65484912,307.0939855)
\lineto(46.56358537,307.0939855)
\lineto(46.56358537,304.66137671)
\curveto(46.56358286,304.16879053)(46.57305562,303.88081844)(46.59200369,303.79745957)
\curveto(46.61473579,303.71788686)(46.66209962,303.6515775)(46.73409533,303.59853128)
\curveto(46.80987478,303.54548252)(46.90081333,303.51895877)(47.00691127,303.51895996)
\curveto(47.15468347,303.51895877)(47.36876798,303.57011171)(47.64916546,303.67241893)
\lineto(47.78557343,302.43337987)
\curveto(47.41423726,302.2742373)(46.99364644,302.19466606)(46.52379971,302.19466592)
\curveto(46.23582515,302.19466606)(45.97627135,302.24202989)(45.74513755,302.33675756)
\curveto(45.51400036,302.43527432)(45.34349057,302.56031483)(45.23360766,302.71187947)
\curveto(45.1275115,302.86723246)(45.05362392,303.07563332)(45.01194471,303.33708267)
\curveto(44.9778418,303.52274788)(44.96079082,303.89786942)(44.96079173,304.46244842)
\lineto(44.96079173,307.0939855)
\lineto(44.22759889,307.0939855)
\lineto(44.22759889,308.36712655)
\lineto(44.96079173,308.36712655)
\lineto(44.96079173,309.56637994)
\lineto(46.56358537,310.49850107)
\lineto(46.56358537,308.36712655)
\lineto(47.65484912,308.36712655)
}
}
{
\newrgbcolor{curcolor}{0 0 0}
\pscustom[linestyle=none,fillstyle=solid,fillcolor=curcolor]
{
\newpath
\moveto(49.96810112,306.52561896)
\lineto(48.51876644,306.78706756)
\curveto(48.68169744,307.37058551)(48.96209132,307.80254364)(49.35994892,308.08294328)
\curveto(49.75780368,308.3633314)(50.34890429,308.50352834)(51.13325253,308.50353452)
\curveto(51.84560135,308.50352834)(52.37607626,308.41827345)(52.72467884,308.24776957)
\curveto(53.07327185,308.08104297)(53.31766921,307.86695845)(53.45787167,307.60551538)
\curveto(53.6018522,307.34785087)(53.67384522,306.872318)(53.67385096,306.17891537)
\lineto(53.65679996,304.31467312)
\curveto(53.65679424,303.78419623)(53.68142344,303.39202371)(53.73068761,303.13815438)
\curveto(53.78372931,302.88807254)(53.88035153,302.61904598)(54.02055455,302.33107389)
\lineto(52.44049557,302.33107389)
\curveto(52.39881089,302.43716887)(52.34765796,302.59441679)(52.2870366,302.80281812)
\curveto(52.26050851,302.89754531)(52.24156298,302.96006557)(52.23019995,302.99037908)
\curveto(51.95737999,302.72514097)(51.66561879,302.52621288)(51.35491548,302.39359421)
\curveto(51.04420533,302.26097542)(50.71265851,302.19466606)(50.36027403,302.19466592)
\curveto(49.73885815,302.19466606)(49.24816886,302.3632813)(48.88820469,302.70051214)
\curveto(48.53202773,303.03774225)(48.35393973,303.46401673)(48.35394015,303.97933686)
\curveto(48.35393973,304.32035479)(48.43540552,304.62348331)(48.59833776,304.88872332)
\curveto(48.76126868,305.15774732)(48.98861507,305.36235907)(49.28037761,305.50255918)
\curveto(49.57592657,305.64654206)(50.00030649,305.77158257)(50.55351865,305.8776811)
\curveto(51.29997002,306.01787449)(51.81718305,306.14859867)(52.10515931,306.26985401)
\lineto(52.10515931,306.42899664)
\curveto(52.10515514,306.73591017)(52.02937301,306.95378379)(51.87781269,307.08261817)
\curveto(51.72624449,307.21523214)(51.44016696,307.2815415)(51.01957922,307.28154645)
\curveto(50.73539315,307.2815415)(50.51373042,307.22470491)(50.35459037,307.11103649)
\curveto(50.19544548,307.00114762)(50.06661586,306.80600864)(49.96810112,306.52561896)
\moveto(52.10515931,305.22974325)
\curveto(51.90054339,305.16153643)(51.57657479,305.08007064)(51.13325253,304.98534563)
\curveto(50.68992387,304.89061532)(50.40005723,304.79778221)(50.26365172,304.70684603)
\curveto(50.05524854,304.5590685)(49.95104811,304.37150773)(49.95105012,304.14416315)
\curveto(49.95104811,303.92060406)(50.03440845,303.72735963)(50.2011314,303.56442928)
\curveto(50.36784982,303.40149647)(50.58003979,303.32003068)(50.83770192,303.32003167)
\curveto(51.12567112,303.32003068)(51.40038134,303.41475834)(51.66183341,303.60421494)
\curveto(51.85507411,303.74819971)(51.98200918,303.92439317)(52.04263899,304.13279582)
\curveto(52.08431506,304.26920186)(52.10515514,304.52875565)(52.10515931,304.91145798)
\lineto(52.10515931,305.22974325)
}
}
{
\newrgbcolor{curcolor}{0 0 0}
\pscustom[linestyle=none,fillstyle=solid,fillcolor=curcolor]
{
\newpath
\moveto(55.10613493,301.93321731)
\lineto(56.93059153,301.71155436)
\curveto(56.96090187,301.49936502)(57.03100034,301.35348442)(57.14088715,301.27391213)
\curveto(57.29244868,301.16023999)(57.53116239,301.10340339)(57.85702899,301.10340217)
\curveto(58.27382726,301.10340339)(58.58642854,301.16592365)(58.79483378,301.29096312)
\curveto(58.93502634,301.37432451)(59.04112132,301.50883779)(59.11311904,301.69450337)
\curveto(59.16237273,301.82712273)(59.18700192,302.0715201)(59.18700669,302.4276962)
\lineto(59.18700669,303.30866434)
\curveto(58.7095745,302.65693705)(58.10710658,302.33107389)(57.37960109,302.33107389)
\curveto(56.56872935,302.33107389)(55.9264758,302.67398803)(55.45283852,303.35981733)
\curveto(55.08150505,303.90165853)(54.89583884,304.57611948)(54.89583931,305.38320221)
\curveto(54.89583884,306.39489059)(55.13834165,307.16786831)(55.62334848,307.70213769)
\curveto(56.11214202,308.23639634)(56.71839905,308.50352834)(57.44212141,308.50353452)
\curveto(58.18857236,308.50352834)(58.80430217,308.17577063)(59.28931267,307.5202604)
\lineto(59.28931267,308.36712655)
\lineto(60.78411667,308.36712655)
\lineto(60.78411667,302.95059342)
\curveto(60.7841103,302.23824078)(60.72537915,301.70587132)(60.60792304,301.35348344)
\curveto(60.49045455,301.00109752)(60.32562842,300.72449275)(60.11344415,300.5236683)
\curveto(59.90124849,300.32284746)(59.61706551,300.16559954)(59.26089434,300.05192407)
\curveto(58.90850259,299.93825315)(58.46138803,299.88141656)(57.91954931,299.88141411)
\curveto(56.89648706,299.88141656)(56.17087317,300.05761001)(55.74270546,300.40999499)
\curveto(55.3145351,300.7585947)(55.10045059,301.20192016)(55.10045127,301.73997269)
\curveto(55.10045059,301.79302077)(55.10234514,301.85743558)(55.10613493,301.93321731)
\moveto(56.53273495,305.47414086)
\curveto(56.53273283,304.83377872)(56.65587879,304.36392952)(56.9021732,304.06459184)
\curveto(57.15225174,303.7690398)(57.45916937,303.62126465)(57.82292699,303.62126594)
\curveto(58.21320156,303.62126465)(58.54285382,303.77282891)(58.81188477,304.07595917)
\curveto(59.08090694,304.38287505)(59.21542022,304.83567327)(59.21542502,305.4343552)
\curveto(59.21542022,306.05955467)(59.0865906,306.52372021)(58.82893577,306.82685322)
\curveto(58.57127212,307.12997724)(58.24540896,307.2815415)(57.85134532,307.28154645)
\curveto(57.46864213,307.2815415)(57.15225174,307.1318718)(56.9021732,306.83253689)
\curveto(56.65587879,306.53698208)(56.53273283,306.08418386)(56.53273495,305.47414086)
}
}
{
\newrgbcolor{curcolor}{0 0 0}
\pscustom[linestyle=none,fillstyle=solid,fillcolor=curcolor]
{
\newpath
\moveto(61.80717561,304.05322451)
\lineto(63.40996925,304.29762212)
\curveto(63.47817129,303.98691342)(63.61647368,303.75009427)(63.82487683,303.58716395)
\curveto(64.03327539,303.42802022)(64.32503659,303.34844898)(64.7001613,303.34845)
\curveto(65.11317074,303.34844898)(65.42387747,303.42423111)(65.63228242,303.57579661)
\curveto(65.77247526,303.68189035)(65.84257373,303.82398184)(65.84257804,304.00207152)
\curveto(65.84257373,304.12332126)(65.80468267,304.22373258)(65.72890473,304.30330579)
\curveto(65.6493293,304.37908594)(65.4712413,304.44918441)(65.19464019,304.51360141)
\curveto(63.90634032,304.79778221)(63.08978788,305.057336)(62.7449804,305.29226356)
\curveto(62.26755177,305.61812376)(62.02883806,306.07092198)(62.02883856,306.6506596)
\curveto(62.02883806,307.17355197)(62.23534437,307.61308832)(62.64835809,307.96926997)
\curveto(63.06136958,308.32544034)(63.70172857,308.50352834)(64.56943699,308.50353452)
\curveto(65.39545917,308.50352834)(66.00929442,308.36901506)(66.41094458,308.09999427)
\curveto(66.81258499,307.83096194)(67.08918976,307.43310576)(67.24075973,306.90642454)
\lineto(65.7345884,306.62792493)
\curveto(65.67016939,306.86284524)(65.54702343,307.0428278)(65.36515015,307.16787315)
\curveto(65.18705831,307.29290882)(64.93129363,307.35542908)(64.59785532,307.3554341)
\curveto(64.17726144,307.35542908)(63.87602747,307.29669793)(63.69415252,307.17924048)
\curveto(63.57289895,307.09587529)(63.51227325,306.98788575)(63.51227523,306.85527155)
\curveto(63.51227325,306.74159383)(63.56532074,306.64497162)(63.67141786,306.56540461)
\curveto(63.81540177,306.4593054)(64.31177472,306.30963569)(65.16053819,306.11639505)
\curveto(66.01308352,305.92314683)(66.60797324,305.68632768)(66.94520913,305.40593687)
\curveto(67.27864509,305.12175081)(67.44536577,304.72578919)(67.44537168,304.2180508)
\curveto(67.44536577,303.66483937)(67.21423028,303.18930651)(66.75196451,302.79145079)
\curveto(66.2896883,302.39359415)(65.60575458,302.19466606)(64.7001613,302.19466592)
\curveto(63.87792203,302.19466606)(63.22619571,302.36138674)(62.7449804,302.69482848)
\curveto(62.26755177,303.02826948)(61.95495049,303.48106771)(61.80717561,304.05322451)
}
}
{
\newrgbcolor{curcolor}{0 1 0.25098041}
\pscustom[linewidth=2.32802935,linecolor=curcolor]
{
\newpath
\moveto(352.515295,295.92182848)
\lineto(427.01223194,295.92182848)
\lineto(427.01223194,342.4824199)
\lineto(378.12362028,342.4824199)
\lineto(378.12362028,351.79453818)
\lineto(352.515295,351.79453818)
\lineto(352.515295,295.92182848)
\closepath
}
}
{
\newrgbcolor{curcolor}{0 0 0}
\pscustom[linestyle=none,fillstyle=solid,fillcolor=curcolor]
{
\newpath
\moveto(365.95601503,325.76257152)
\lineto(365.95601503,326.71742731)
\lineto(371.47485413,329.04773012)
\lineto(371.47485413,328.03035402)
\lineto(367.09843178,326.23431575)
\lineto(371.47485413,324.42122649)
\lineto(371.47485413,323.40385038)
\lineto(365.95601503,325.76257152)
}
}
{
\newrgbcolor{curcolor}{0 0 0}
\pscustom[linestyle=none,fillstyle=solid,fillcolor=curcolor]
{
\newpath
\moveto(372.75367903,325.76257152)
\lineto(372.75367903,326.71742731)
\lineto(378.27251813,329.04773012)
\lineto(378.27251813,328.03035402)
\lineto(373.89609577,326.23431575)
\lineto(378.27251813,324.42122649)
\lineto(378.27251813,323.40385038)
\lineto(372.75367903,325.76257152)
}
}
{
\newrgbcolor{curcolor}{0 0 0}
\pscustom[linestyle=none,fillstyle=solid,fillcolor=curcolor]
{
\newpath
\moveto(383.62084744,322.86390217)
\curveto(383.24193209,322.54182737)(382.87628332,322.31448099)(382.52390002,322.18186232)
\curveto(382.17529862,322.04924353)(381.80017708,321.98293417)(381.39853427,321.98293403)
\curveto(380.73543816,321.98293417)(380.22580334,322.14397119)(379.86962828,322.46604559)
\curveto(379.51345132,322.7919084)(379.33536331,323.20681556)(379.33536373,323.71076831)
\curveto(379.33536331,324.00631703)(379.40167268,324.27534359)(379.53429202,324.5178488)
\curveto(379.67069924,324.76413832)(379.84689269,324.96117186)(380.06287291,325.10895)
\curveto(380.28263993,325.25672216)(380.52893185,325.36850081)(380.80174941,325.44428626)
\curveto(381.00257016,325.49733043)(381.30569868,325.54848336)(381.71113587,325.59774523)
\curveto(382.53715829,325.69625852)(383.14530988,325.81372082)(383.53559246,325.95013248)
\curveto(383.53937695,326.09032559)(383.5412715,326.17936959)(383.54127613,326.21726475)
\curveto(383.5412715,326.63406237)(383.44464929,326.92771812)(383.25140919,327.09823289)
\curveto(382.98995651,327.32936341)(382.6015731,327.44493115)(382.08625779,327.44493648)
\curveto(381.60503809,327.44493115)(381.24886208,327.35967626)(381.01772869,327.18917154)
\curveto(380.7903802,327.02244578)(380.62176496,326.72500092)(380.51188247,326.29683607)
\lineto(379.51155736,326.43324404)
\curveto(379.60249532,326.86140876)(379.75216503,327.20621745)(379.96056693,327.46767114)
\curveto(380.16896674,327.73290325)(380.4702007,327.93562044)(380.86426973,328.07582334)
\curveto(381.25833485,328.21980343)(381.71492218,328.29179645)(382.23403309,328.29180262)
\curveto(382.74934825,328.29179645)(383.16804451,328.23117075)(383.49012314,328.10992533)
\curveto(383.81219262,327.98866793)(384.04901177,327.83520912)(384.20058132,327.64954843)
\curveto(384.35214029,327.46766579)(384.45823527,327.2365303)(384.51886658,326.95614126)
\curveto(384.55296293,326.78183752)(384.57001391,326.46734168)(384.57001957,326.0126528)
\lineto(384.57001957,324.6485731)
\curveto(384.57001391,323.69750485)(384.590854,323.09503692)(384.63253989,322.84116751)
\curveto(384.67800345,322.59108576)(384.76515289,322.3504775)(384.89398849,322.119342)
\lineto(383.8254594,322.119342)
\curveto(383.71935951,322.33153196)(383.65115559,322.57971844)(383.62084744,322.86390217)
\moveto(383.53559246,325.14873566)
\curveto(383.16425541,324.99716837)(382.60725676,324.86833875)(381.86459484,324.76224641)
\curveto(381.44400107,324.70161807)(381.14655621,324.63341415)(380.97225937,324.55763446)
\curveto(380.79795841,324.48184989)(380.66344513,324.37007125)(380.56871913,324.2222982)
\curveto(380.47398981,324.07831005)(380.42662598,323.91727303)(380.42662749,323.73918664)
\curveto(380.42662598,323.46636935)(380.52893185,323.23902297)(380.73354542,323.05714679)
\curveto(380.94194446,322.87526874)(381.24507298,322.78433019)(381.64293189,322.78433085)
\curveto(382.03699623,322.78433019)(382.38748858,322.86958508)(382.69440999,323.0400958)
\curveto(383.00132383,323.21439377)(383.22677567,323.45121293)(383.37076617,323.75055397)
\curveto(383.4806458,323.98168784)(383.53558784,324.32270742)(383.53559246,324.77361374)
\lineto(383.53559246,325.14873566)
}
}
{
\newrgbcolor{curcolor}{0 0 0}
\pscustom[linestyle=none,fillstyle=solid,fillcolor=curcolor]
{
\newpath
\moveto(386.16144528,319.80609018)
\lineto(386.16144528,328.15539466)
\lineto(387.09356641,328.15539466)
\lineto(387.09356641,327.37104883)
\curveto(387.31333289,327.6779612)(387.56151936,327.90720214)(387.83812658,328.05877234)
\curveto(388.11472891,328.21411977)(388.45006483,328.29179645)(388.84413535,328.29180262)
\curveto(389.35945038,328.29179645)(389.81414316,328.15917773)(390.20821505,327.89394605)
\curveto(390.60227731,327.62870282)(390.89972217,327.25358128)(391.10055052,326.7685803)
\curveto(391.30136745,326.28735913)(391.40177877,325.75877877)(391.40178478,325.18283765)
\curveto(391.40177877,324.56521023)(391.29000013,324.00821158)(391.06644852,323.51184002)
\curveto(390.84667468,323.01925479)(390.52460062,322.64034414)(390.10022541,322.37510694)
\curveto(389.67962988,322.11365834)(389.23630442,321.98293417)(388.7702477,321.98293403)
\curveto(388.42922474,321.98293417)(388.12230712,322.05492719)(387.84949391,322.19891332)
\curveto(387.58046489,322.34289928)(387.35880216,322.5247764)(387.18450506,322.7445452)
\lineto(387.18450506,319.80609018)
\lineto(386.16144528,319.80609018)
\moveto(387.08788274,325.10326634)
\curveto(387.08788105,324.32649652)(387.24512897,323.75244689)(387.55962697,323.38111572)
\curveto(387.87412064,323.00978202)(388.25492585,322.82411581)(388.70204372,322.82411651)
\curveto(389.15673319,322.82411581)(389.5451166,323.01546568)(389.86719512,323.39816672)
\curveto(390.19305381,323.7846543)(390.35598539,324.38143857)(390.35599035,325.18852132)
\curveto(390.35598539,325.95770686)(390.19684291,326.53365105)(389.87856246,326.9163556)
\curveto(389.56406213,327.29905056)(389.18704604,327.49040043)(388.74751304,327.4904058)
\curveto(388.31176244,327.49040043)(387.92527358,327.28578868)(387.5880453,326.87656994)
\curveto(387.25460174,326.47113079)(387.08788105,325.88003018)(387.08788274,325.10326634)
}
}
{
\newrgbcolor{curcolor}{0 0 0}
\pscustom[linestyle=none,fillstyle=solid,fillcolor=curcolor]
{
\newpath
\moveto(392.64082413,319.80609018)
\lineto(392.64082413,328.15539466)
\lineto(393.57294526,328.15539466)
\lineto(393.57294526,327.37104883)
\curveto(393.79271174,327.6779612)(394.04089821,327.90720214)(394.31750543,328.05877234)
\curveto(394.59410775,328.21411977)(394.92944368,328.29179645)(395.3235142,328.29180262)
\curveto(395.83882923,328.29179645)(396.29352201,328.15917773)(396.6875939,327.89394605)
\curveto(397.08165616,327.62870282)(397.37910102,327.25358128)(397.57992936,326.7685803)
\curveto(397.7807463,326.28735913)(397.88115762,325.75877877)(397.88116363,325.18283765)
\curveto(397.88115762,324.56521023)(397.76937898,324.00821158)(397.54582737,323.51184002)
\curveto(397.32605352,323.01925479)(397.00397947,322.64034414)(396.57960425,322.37510694)
\curveto(396.15900873,322.11365834)(395.71568327,321.98293417)(395.24962655,321.98293403)
\curveto(394.90860359,321.98293417)(394.60168597,322.05492719)(394.32887276,322.19891332)
\curveto(394.05984374,322.34289928)(393.83818101,322.5247764)(393.66388391,322.7445452)
\lineto(393.66388391,319.80609018)
\lineto(392.64082413,319.80609018)
\moveto(393.56726159,325.10326634)
\curveto(393.5672599,324.32649652)(393.72450782,323.75244689)(394.03900582,323.38111572)
\curveto(394.35349949,323.00978202)(394.73430469,322.82411581)(395.18142257,322.82411651)
\curveto(395.63611204,322.82411581)(396.02449545,323.01546568)(396.34657397,323.39816672)
\curveto(396.67243266,323.7846543)(396.83536424,324.38143857)(396.8353692,325.18852132)
\curveto(396.83536424,325.95770686)(396.67622176,326.53365105)(396.3579413,326.9163556)
\curveto(396.04344098,327.29905056)(395.66642489,327.49040043)(395.22689189,327.4904058)
\curveto(394.79114129,327.49040043)(394.40465243,327.28578868)(394.06742415,326.87656994)
\curveto(393.73398058,326.47113079)(393.5672599,325.88003018)(393.56726159,325.10326634)
}
}
{
\newrgbcolor{curcolor}{0 0 0}
\pscustom[linestyle=none,fillstyle=solid,fillcolor=curcolor]
{
\newpath
\moveto(404.50831778,325.76257152)
\lineto(398.98947868,323.40385038)
\lineto(398.98947868,324.42122649)
\lineto(403.36021737,326.23431575)
\lineto(398.98947868,328.03035402)
\lineto(398.98947868,329.04773012)
\lineto(404.50831778,326.71742731)
\lineto(404.50831778,325.76257152)
}
}
{
\newrgbcolor{curcolor}{0 0 0}
\pscustom[linestyle=none,fillstyle=solid,fillcolor=curcolor]
{
\newpath
\moveto(411.30598044,325.76257152)
\lineto(405.78714134,323.40385038)
\lineto(405.78714134,324.42122649)
\lineto(410.15788003,326.23431575)
\lineto(405.78714134,328.03035402)
\lineto(405.78714134,329.04773012)
\lineto(411.30598044,326.71742731)
\lineto(411.30598044,325.76257152)
}
}
{
\newrgbcolor{curcolor}{0 0 0}
\pscustom[linestyle=none,fillstyle=solid,fillcolor=curcolor]
{
\newpath
\moveto(359.66649667,305.82313124)
\lineto(357.23388788,311.85918389)
\lineto(358.91056917,311.85918389)
\lineto(360.04730225,308.77863725)
\lineto(360.37695484,307.74989381)
\curveto(360.46410109,308.01134023)(360.51904313,308.18374458)(360.54178114,308.26710736)
\curveto(360.59482526,308.43761471)(360.65166186,308.6081245)(360.7122911,308.77863725)
\lineto(361.86039151,311.85918389)
\lineto(363.50297081,311.85918389)
\lineto(361.10446402,305.82313124)
\lineto(359.66649667,305.82313124)
}
}
{
\newrgbcolor{curcolor}{0 0 0}
\pscustom[linestyle=none,fillstyle=solid,fillcolor=curcolor]
{
\newpath
\moveto(365.70254926,310.0176763)
\lineto(364.25321458,310.27912491)
\curveto(364.41614558,310.86264285)(364.69653946,311.29460099)(365.09439706,311.57500062)
\curveto(365.49225182,311.85538875)(366.08335243,311.99558569)(366.86770067,311.99559186)
\curveto(367.58004949,311.99558569)(368.1105244,311.9103308)(368.45912698,311.73982692)
\curveto(368.80771999,311.57310032)(369.05211736,311.3590158)(369.19231981,311.09757273)
\curveto(369.33630034,310.83990822)(369.40829337,310.36437535)(369.4082991,309.67097272)
\lineto(369.3912481,307.80673046)
\curveto(369.39124239,307.27625357)(369.41587158,306.88408105)(369.46513575,306.63021173)
\curveto(369.51817745,306.38012989)(369.61479967,306.11110333)(369.75500269,305.82313124)
\lineto(368.17494371,305.82313124)
\curveto(368.13325904,305.92922622)(368.0821061,306.08647414)(368.02148474,306.29487547)
\curveto(367.99495665,306.38960266)(367.97601112,306.45212292)(367.96464809,306.48243643)
\curveto(367.69182813,306.21719831)(367.40006693,306.01827022)(367.08936362,305.88565156)
\curveto(366.77865347,305.75303277)(366.44710665,305.68672341)(366.09472217,305.68672327)
\curveto(365.47330629,305.68672341)(364.982617,305.85533865)(364.62265283,306.19256949)
\curveto(364.26647588,306.5297996)(364.08838787,306.95607408)(364.08838829,307.47139421)
\curveto(364.08838787,307.81241214)(364.16985366,308.11554066)(364.3327859,308.38078067)
\curveto(364.49571682,308.64980467)(364.72306321,308.85441642)(365.01482575,308.99461653)
\curveto(365.31037471,309.13859941)(365.73475464,309.26363992)(366.2879668,309.36973845)
\curveto(367.03441816,309.50993184)(367.55163119,309.64065602)(367.83960745,309.76191136)
\lineto(367.83960745,309.92105399)
\curveto(367.83960328,310.22796752)(367.76382115,310.44584114)(367.61226083,310.57467551)
\curveto(367.46069264,310.70728949)(367.1746151,310.77359885)(366.75402736,310.7736038)
\curveto(366.46984129,310.77359885)(366.24817856,310.71676226)(366.08903851,310.60309384)
\curveto(365.92989362,310.49320497)(365.801064,310.29806599)(365.70254926,310.0176763)
\moveto(367.83960745,308.72180059)
\curveto(367.63499153,308.65359378)(367.31102293,308.57212799)(366.86770067,308.47740298)
\curveto(366.42437202,308.38267267)(366.13450537,308.28983956)(365.99809986,308.19890338)
\curveto(365.78969668,308.05112585)(365.68549625,307.86356508)(365.68549826,307.6362205)
\curveto(365.68549625,307.41266141)(365.76885659,307.21941698)(365.93557954,307.05648663)
\curveto(366.10229796,306.89355382)(366.31448793,306.81208803)(366.57215007,306.81208902)
\curveto(366.86011926,306.81208803)(367.13482948,306.90681569)(367.39628155,307.09627229)
\curveto(367.58952226,307.24025706)(367.71645732,307.41645051)(367.77708713,307.62485317)
\curveto(367.8187632,307.7612592)(367.83960328,308.020813)(367.83960745,308.40351533)
\lineto(367.83960745,308.72180059)
}
}
{
\newrgbcolor{curcolor}{0 0 0}
\pscustom[linestyle=none,fillstyle=solid,fillcolor=curcolor]
{
\newpath
\moveto(370.98835749,305.82313124)
\lineto(370.98835749,314.15538471)
\lineto(372.58546746,314.15538471)
\lineto(372.58546746,305.82313124)
\lineto(370.98835749,305.82313124)
}
}
{
\newrgbcolor{curcolor}{0 0 0}
\pscustom[linestyle=none,fillstyle=solid,fillcolor=curcolor]
{
\newpath
\moveto(373.84724195,308.92641255)
\curveto(373.84724149,309.45688435)(373.97796566,309.97030828)(374.23941487,310.46668587)
\curveto(374.50086235,310.96305418)(374.87030024,311.34196482)(375.34772962,311.60341895)
\curveto(375.82894417,311.86486152)(376.36510274,311.99558569)(376.95620693,311.99559186)
\curveto(377.86937801,311.99558569)(378.61772654,311.69814083)(379.20125476,311.1032564)
\curveto(379.78477134,310.51215051)(380.07653253,309.76380198)(380.07653923,308.85820856)
\curveto(380.07653253,307.94503087)(379.78098223,307.18720957)(379.18988743,306.5847424)
\curveto(378.60257012,305.98606282)(377.8617998,305.68672341)(376.96757426,305.68672327)
\curveto(376.41436112,305.68672341)(375.88578077,305.81176392)(375.38183161,306.06184519)
\curveto(374.88166756,306.31192598)(374.50086235,306.67757475)(374.23941487,307.15879261)
\curveto(373.97796566,307.6437969)(373.84724149,308.23300296)(373.84724195,308.92641255)
\moveto(375.48413759,308.84115757)
\curveto(375.48413549,308.24247573)(375.62622698,307.78399384)(375.91041249,307.46571054)
\curveto(376.19459295,307.14742395)(376.5450853,306.98828148)(376.96189059,306.98828265)
\curveto(377.37868872,306.98828148)(377.72728652,307.14742395)(378.00768502,307.46571054)
\curveto(378.29186338,307.78399384)(378.43395488,308.24626483)(378.43395993,308.8525249)
\curveto(378.43395488,309.44362248)(378.29186338,309.89831526)(378.00768502,310.21660459)
\curveto(377.72728652,310.53488514)(377.37868872,310.69402762)(376.96189059,310.69403249)
\curveto(376.5450853,310.69402762)(376.19459295,310.53488514)(375.91041249,310.21660459)
\curveto(375.62622698,309.89831526)(375.48413549,309.43983337)(375.48413759,308.84115757)
}
}
{
\newrgbcolor{curcolor}{0 0 0}
\pscustom[linestyle=none,fillstyle=solid,fillcolor=curcolor]
{
\newpath
\moveto(382.86153445,305.82313124)
\lineto(381.26442447,305.82313124)
\lineto(381.26442447,311.85918389)
\lineto(382.74786114,311.85918389)
\lineto(382.74786114,311.00095042)
\curveto(383.00172902,311.40637963)(383.22907541,311.67351164)(383.42990099,311.80234724)
\curveto(383.6345098,311.93117088)(383.8656453,311.99558569)(384.12330816,311.99559186)
\curveto(384.48705876,311.99558569)(384.83755111,311.89517437)(385.17478626,311.6943576)
\lineto(384.68030737,310.30185957)
\curveto(384.41127663,310.47615399)(384.1611956,310.56330344)(383.93006354,310.56330818)
\curveto(383.70650283,310.56330344)(383.5170475,310.50078319)(383.361697,310.37574722)
\curveto(383.20634077,310.25449127)(383.08319481,310.03282854)(382.99225875,309.71075837)
\curveto(382.90510681,309.38868044)(382.86153208,308.71421948)(382.86153445,307.68737349)
\lineto(382.86153445,305.82313124)
}
}
{
\newrgbcolor{curcolor}{0 0 0}
\pscustom[linestyle=none,fillstyle=solid,fillcolor=curcolor]
{
\newpath
\moveto(387.05039667,310.0176763)
\lineto(385.601062,310.27912491)
\curveto(385.763993,310.86264285)(386.04438688,311.29460099)(386.44224448,311.57500062)
\curveto(386.84009924,311.85538875)(387.43119985,311.99558569)(388.21554808,311.99559186)
\curveto(388.9278969,311.99558569)(389.45837181,311.9103308)(389.80697439,311.73982692)
\curveto(390.1555674,311.57310032)(390.39996477,311.3590158)(390.54016723,311.09757273)
\curveto(390.68414776,310.83990822)(390.75614078,310.36437535)(390.75614651,309.67097272)
\lineto(390.73909552,307.80673046)
\curveto(390.7390898,307.27625357)(390.76371899,306.88408105)(390.81298317,306.63021173)
\curveto(390.86602487,306.38012989)(390.96264708,306.11110333)(391.1028501,305.82313124)
\lineto(389.52279112,305.82313124)
\curveto(389.48110645,305.92922622)(389.42995351,306.08647414)(389.36933216,306.29487547)
\curveto(389.34280406,306.38960266)(389.32385853,306.45212292)(389.3124955,306.48243643)
\curveto(389.03967554,306.21719831)(388.74791435,306.01827022)(388.43721103,305.88565156)
\curveto(388.12650088,305.75303277)(387.79495407,305.68672341)(387.44256959,305.68672327)
\curveto(386.8211537,305.68672341)(386.33046441,305.85533865)(385.97050025,306.19256949)
\curveto(385.61432329,306.5297996)(385.43623529,306.95607408)(385.4362357,307.47139421)
\curveto(385.43623529,307.81241214)(385.51770107,308.11554066)(385.68063331,308.38078067)
\curveto(385.84356423,308.64980467)(386.07091062,308.85441642)(386.36267316,308.99461653)
\curveto(386.65822212,309.13859941)(387.08260205,309.26363992)(387.63581421,309.36973845)
\curveto(388.38226557,309.50993184)(388.89947861,309.64065602)(389.18745486,309.76191136)
\lineto(389.18745486,309.92105399)
\curveto(389.1874507,310.22796752)(389.11166857,310.44584114)(388.96010825,310.57467551)
\curveto(388.80854005,310.70728949)(388.52246251,310.77359885)(388.10187477,310.7736038)
\curveto(387.81768871,310.77359885)(387.59602598,310.71676226)(387.43688592,310.60309384)
\curveto(387.27774103,310.49320497)(387.14891141,310.29806599)(387.05039667,310.0176763)
\moveto(389.18745486,308.72180059)
\curveto(388.98283895,308.65359378)(388.65887034,308.57212799)(388.21554808,308.47740298)
\curveto(387.77221943,308.38267267)(387.48235278,308.28983956)(387.34594727,308.19890338)
\curveto(387.13754409,308.05112585)(387.03334367,307.86356508)(387.03334568,307.6362205)
\curveto(387.03334367,307.41266141)(387.11670401,307.21941698)(387.28342695,307.05648663)
\curveto(387.45014538,306.89355382)(387.66233534,306.81208803)(387.91999748,306.81208902)
\curveto(388.20796667,306.81208803)(388.48267689,306.90681569)(388.74412896,307.09627229)
\curveto(388.93736967,307.24025706)(389.06430474,307.41645051)(389.12493454,307.62485317)
\curveto(389.16661061,307.7612592)(389.1874507,308.020813)(389.18745486,308.40351533)
\lineto(389.18745486,308.72180059)
}
}
{
\newrgbcolor{curcolor}{0 0 0}
\pscustom[linestyle=none,fillstyle=solid,fillcolor=curcolor]
{
\newpath
\moveto(395.10414906,311.85918389)
\lineto(395.10414906,310.58604284)
\lineto(394.0128853,310.58604284)
\lineto(394.0128853,308.15343405)
\curveto(394.01288279,307.66084788)(394.02235556,307.37287579)(394.04130363,307.28951691)
\curveto(394.06403573,307.20994421)(394.11139956,307.14363485)(394.18339527,307.09058862)
\curveto(394.25917471,307.03753987)(394.35011327,307.01101612)(394.45621121,307.01101731)
\curveto(394.6039834,307.01101612)(394.81806792,307.06216906)(395.0984654,307.16447627)
\lineto(395.23487337,305.92543722)
\curveto(394.8635372,305.76629464)(394.44294638,305.68672341)(393.97309965,305.68672327)
\curveto(393.68512508,305.68672341)(393.42557129,305.73408724)(393.19443749,305.82881491)
\curveto(392.9633003,305.92733167)(392.79279051,306.05237218)(392.6829076,306.20393682)
\curveto(392.57681144,306.35928981)(392.50292386,306.56769066)(392.46124465,306.82914002)
\curveto(392.42714173,307.01480523)(392.41009075,307.38992677)(392.41009166,307.95450577)
\lineto(392.41009166,310.58604284)
\lineto(391.67689883,310.58604284)
\lineto(391.67689883,311.85918389)
\lineto(392.41009166,311.85918389)
\lineto(392.41009166,313.05843729)
\lineto(394.0128853,313.99055842)
\lineto(394.0128853,311.85918389)
\lineto(395.10414906,311.85918389)
}
}
{
\newrgbcolor{curcolor}{0 0 0}
\pscustom[linestyle=none,fillstyle=solid,fillcolor=curcolor]
{
\newpath
\moveto(396.2238331,312.67763171)
\lineto(396.2238331,314.15538471)
\lineto(397.82094308,314.15538471)
\lineto(397.82094308,312.67763171)
\lineto(396.2238331,312.67763171)
\moveto(396.2238331,305.82313124)
\lineto(396.2238331,311.85918389)
\lineto(397.82094308,311.85918389)
\lineto(397.82094308,305.82313124)
\lineto(396.2238331,305.82313124)
}
}
{
\newrgbcolor{curcolor}{0 0 0}
\pscustom[linestyle=none,fillstyle=solid,fillcolor=curcolor]
{
\newpath
\moveto(399.08271756,308.92641255)
\curveto(399.0827171,309.45688435)(399.21344127,309.97030828)(399.47489048,310.46668587)
\curveto(399.73633797,310.96305418)(400.10577585,311.34196482)(400.58320523,311.60341895)
\curveto(401.06441979,311.86486152)(401.60057835,311.99558569)(402.19168254,311.99559186)
\curveto(403.10485362,311.99558569)(403.85320215,311.69814083)(404.43673037,311.1032564)
\curveto(405.02024695,310.51215051)(405.31200815,309.76380198)(405.31201484,308.85820856)
\curveto(405.31200815,307.94503087)(405.01645784,307.18720957)(404.42536304,306.5847424)
\curveto(403.83804573,305.98606282)(403.09727541,305.68672341)(402.20304987,305.68672327)
\curveto(401.64983674,305.68672341)(401.12125638,305.81176392)(400.61730722,306.06184519)
\curveto(400.11714317,306.31192598)(399.73633797,306.67757475)(399.47489048,307.15879261)
\curveto(399.21344127,307.6437969)(399.0827171,308.23300296)(399.08271756,308.92641255)
\moveto(400.7196132,308.84115757)
\curveto(400.7196111,308.24247573)(400.86170259,307.78399384)(401.1458881,307.46571054)
\curveto(401.43006856,307.14742395)(401.78056091,306.98828148)(402.1973662,306.98828265)
\curveto(402.61416433,306.98828148)(402.96276213,307.14742395)(403.24316064,307.46571054)
\curveto(403.527339,307.78399384)(403.66943049,308.24626483)(403.66943554,308.8525249)
\curveto(403.66943049,309.44362248)(403.527339,309.89831526)(403.24316064,310.21660459)
\curveto(402.96276213,310.53488514)(402.61416433,310.69402762)(402.1973662,310.69403249)
\curveto(401.78056091,310.69402762)(401.43006856,310.53488514)(401.1458881,310.21660459)
\curveto(400.86170259,309.89831526)(400.7196111,309.43983337)(400.7196132,308.84115757)
}
}
{
\newrgbcolor{curcolor}{0 0 0}
\pscustom[linestyle=none,fillstyle=solid,fillcolor=curcolor]
{
\newpath
\moveto(412.05852484,305.82313124)
\lineto(410.46141486,305.82313124)
\lineto(410.46141486,308.90367789)
\curveto(410.46141013,309.55540112)(410.42730818,309.97599194)(410.35910889,310.16545161)
\curveto(410.29090034,310.35869169)(410.1791217,310.5083614)(410.02377263,310.61446117)
\curveto(409.87220408,310.72055136)(409.68843241,310.77359885)(409.47245708,310.7736038)
\curveto(409.19584857,310.77359885)(408.9476621,310.69781672)(408.72789692,310.54625719)
\curveto(408.50812575,310.3946882)(408.35656149,310.19386556)(408.27320368,309.94378865)
\curveto(408.19362991,309.69370351)(408.15384429,309.23143252)(408.15384671,308.5569743)
\lineto(408.15384671,305.82313124)
\lineto(406.55673673,305.82313124)
\lineto(406.55673673,311.85918389)
\lineto(408.0401734,311.85918389)
\lineto(408.0401734,310.97253209)
\curveto(408.5668569,311.65456611)(409.22995053,311.99558569)(410.02945629,311.99559186)
\curveto(410.3818389,311.99558569)(410.70391295,311.93117088)(410.99567941,311.80234724)
\curveto(411.28743535,311.67730075)(411.50720352,311.51626372)(411.6549846,311.31923568)
\curveto(411.80654293,311.12219665)(411.91074336,310.89863937)(411.96758619,310.64856316)
\curveto(412.02820566,310.39847731)(412.05851851,310.04040675)(412.05852484,309.5743504)
\lineto(412.05852484,305.82313124)
}
}
{
\newrgbcolor{curcolor}{0 0 0}
\pscustom[linestyle=none,fillstyle=solid,fillcolor=curcolor]
{
\newpath
\moveto(413.12136944,307.54528186)
\lineto(414.72416308,307.78967947)
\curveto(414.79236512,307.47897077)(414.93066751,307.24215162)(415.13907066,307.07922129)
\curveto(415.34746922,306.92007757)(415.63923042,306.84050633)(416.01435513,306.84050735)
\curveto(416.42736457,306.84050633)(416.7380713,306.91628846)(416.94647625,307.06785396)
\curveto(417.08666909,307.1739477)(417.15676756,307.31603919)(417.15677187,307.49412887)
\curveto(417.15676756,307.6153786)(417.1188765,307.71578993)(417.04309856,307.79536313)
\curveto(416.96352313,307.87114329)(416.78543513,307.94124176)(416.50883402,308.00565875)
\curveto(415.22053415,308.28983956)(414.40398171,308.54939335)(414.05917423,308.78432091)
\curveto(413.5817456,309.11018111)(413.3430319,309.56297933)(413.34303239,310.14271694)
\curveto(413.3430319,310.66560932)(413.5495382,311.10514567)(413.96255192,311.46132732)
\curveto(414.37556341,311.81749769)(415.0159224,311.99558569)(415.88363082,311.99559186)
\curveto(416.709653,311.99558569)(417.32348825,311.86107241)(417.72513841,311.59205162)
\curveto(418.12677882,311.32301929)(418.40338359,310.92516311)(418.55495356,310.39848189)
\lineto(417.04878223,310.11998228)
\curveto(416.98436322,310.35490259)(416.86121726,310.53488514)(416.67934398,310.65993049)
\curveto(416.50125214,310.78496617)(416.24548746,310.84748643)(415.91204915,310.84749145)
\curveto(415.49145527,310.84748643)(415.1902213,310.78875528)(415.00834635,310.67129783)
\curveto(414.88709278,310.58793263)(414.82646708,310.4799431)(414.82646906,310.3473289)
\curveto(414.82646708,310.23365118)(414.87951457,310.13702896)(414.98561169,310.05746196)
\curveto(415.1295956,309.95136275)(415.62596855,309.80169304)(416.47473202,309.6084524)
\curveto(417.32727736,309.41520418)(417.92216707,309.17838503)(418.25940296,308.89799422)
\curveto(418.59283892,308.61380816)(418.7595596,308.21784653)(418.75956551,307.71010815)
\curveto(418.7595596,307.15689672)(418.52842411,306.68136386)(418.06615834,306.28350814)
\curveto(417.60388213,305.8856515)(416.91994841,305.68672341)(416.01435513,305.68672327)
\curveto(415.19211586,305.68672341)(414.54038954,305.85344409)(414.05917423,306.18688583)
\curveto(413.5817456,306.52032683)(413.26914432,306.97312506)(413.12136944,307.54528186)
}
}
{
\newrgbcolor{curcolor}{0 1 0.25098041}
\pscustom[linewidth=2.32802935,linecolor=curcolor]
{
\newpath
\moveto(352.515295,205.12868455)
\lineto(427.01223194,205.12868455)
\lineto(427.01223194,251.68927596)
\lineto(378.12362028,251.68927596)
\lineto(378.12362028,261.00139424)
\lineto(352.515295,261.00139424)
\lineto(352.515295,205.12868455)
\closepath
}
}
{
\newrgbcolor{curcolor}{0 0 0}
\pscustom[linestyle=none,fillstyle=solid,fillcolor=curcolor]
{
\newpath
\moveto(365.95601503,232.64138342)
\lineto(365.95601503,233.59623921)
\lineto(371.47485413,235.92654202)
\lineto(371.47485413,234.90916591)
\lineto(367.09843178,233.11312765)
\lineto(371.47485413,231.30003839)
\lineto(371.47485413,230.28266228)
\lineto(365.95601503,232.64138342)
}
}
{
\newrgbcolor{curcolor}{0 0 0}
\pscustom[linestyle=none,fillstyle=solid,fillcolor=curcolor]
{
\newpath
\moveto(372.75367903,232.64138342)
\lineto(372.75367903,233.59623921)
\lineto(378.27251813,235.92654202)
\lineto(378.27251813,234.90916591)
\lineto(373.89609577,233.11312765)
\lineto(378.27251813,231.30003839)
\lineto(378.27251813,230.28266228)
\lineto(372.75367903,232.64138342)
}
}
{
\newrgbcolor{curcolor}{0 0 0}
\pscustom[linestyle=none,fillstyle=solid,fillcolor=curcolor]
{
\newpath
\moveto(383.62084744,229.74271407)
\curveto(383.24193209,229.42063927)(382.87628332,229.19329288)(382.52390002,229.06067422)
\curveto(382.17529862,228.92805543)(381.80017708,228.86174607)(381.39853427,228.86174593)
\curveto(380.73543816,228.86174607)(380.22580334,229.02278309)(379.86962828,229.34485749)
\curveto(379.51345132,229.6707203)(379.33536331,230.08562746)(379.33536373,230.58958021)
\curveto(379.33536331,230.88512892)(379.40167268,231.15415548)(379.53429202,231.3966607)
\curveto(379.67069924,231.64295022)(379.84689269,231.83998376)(380.06287291,231.9877619)
\curveto(380.28263993,232.13553406)(380.52893185,232.2473127)(380.80174941,232.32309816)
\curveto(381.00257016,232.37614232)(381.30569868,232.42729526)(381.71113587,232.47655712)
\curveto(382.53715829,232.57507041)(383.14530988,232.69253271)(383.53559246,232.82894438)
\curveto(383.53937695,232.96913749)(383.5412715,233.05818149)(383.54127613,233.09607665)
\curveto(383.5412715,233.51287427)(383.44464929,233.80653002)(383.25140919,233.97704479)
\curveto(382.98995651,234.2081753)(382.6015731,234.32374305)(382.08625779,234.32374838)
\curveto(381.60503809,234.32374305)(381.24886208,234.23848816)(381.01772869,234.06798343)
\curveto(380.7903802,233.90125768)(380.62176496,233.60381282)(380.51188247,233.17564797)
\lineto(379.51155736,233.31205594)
\curveto(379.60249532,233.74022065)(379.75216503,234.08502934)(379.96056693,234.34648304)
\curveto(380.16896674,234.61171514)(380.4702007,234.81443234)(380.86426973,234.95463524)
\curveto(381.25833485,235.09861533)(381.71492218,235.17060835)(382.23403309,235.17061452)
\curveto(382.74934825,235.17060835)(383.16804451,235.10998265)(383.49012314,234.98873723)
\curveto(383.81219262,234.86747983)(384.04901177,234.71402102)(384.20058132,234.52836033)
\curveto(384.35214029,234.34647769)(384.45823527,234.1153422)(384.51886658,233.83495315)
\curveto(384.55296293,233.66064942)(384.57001391,233.34615358)(384.57001957,232.8914647)
\lineto(384.57001957,231.527385)
\curveto(384.57001391,230.57631675)(384.590854,229.97384882)(384.63253989,229.7199794)
\curveto(384.67800345,229.46989766)(384.76515289,229.22928939)(384.89398849,228.9981539)
\lineto(383.8254594,228.9981539)
\curveto(383.71935951,229.21034386)(383.65115559,229.45853034)(383.62084744,229.74271407)
\moveto(383.53559246,232.02754756)
\curveto(383.16425541,231.87598027)(382.60725676,231.74715065)(381.86459484,231.64105831)
\curveto(381.44400107,231.58042996)(381.14655621,231.51222605)(380.97225937,231.43644636)
\curveto(380.79795841,231.36066179)(380.66344513,231.24888315)(380.56871913,231.1011101)
\curveto(380.47398981,230.95712195)(380.42662598,230.79608492)(380.42662749,230.61799854)
\curveto(380.42662598,230.34518125)(380.52893185,230.11783486)(380.73354542,229.93595869)
\curveto(380.94194446,229.75408064)(381.24507298,229.66314209)(381.64293189,229.66314275)
\curveto(382.03699623,229.66314209)(382.38748858,229.74839698)(382.69440999,229.91890769)
\curveto(383.00132383,230.09320567)(383.22677567,230.33002483)(383.37076617,230.62936587)
\curveto(383.4806458,230.86049973)(383.53558784,231.20151932)(383.53559246,231.65242564)
\lineto(383.53559246,232.02754756)
}
}
{
\newrgbcolor{curcolor}{0 0 0}
\pscustom[linestyle=none,fillstyle=solid,fillcolor=curcolor]
{
\newpath
\moveto(386.16144528,226.68490208)
\lineto(386.16144528,235.03420655)
\lineto(387.09356641,235.03420655)
\lineto(387.09356641,234.24986073)
\curveto(387.31333289,234.5567731)(387.56151936,234.78601404)(387.83812658,234.93758424)
\curveto(388.11472891,235.09293167)(388.45006483,235.17060835)(388.84413535,235.17061452)
\curveto(389.35945038,235.17060835)(389.81414316,235.03798962)(390.20821505,234.77275794)
\curveto(390.60227731,234.50751472)(390.89972217,234.13239317)(391.10055052,233.6473922)
\curveto(391.30136745,233.16617102)(391.40177877,232.63759067)(391.40178478,232.06164955)
\curveto(391.40177877,231.44402213)(391.29000013,230.88702348)(391.06644852,230.39065192)
\curveto(390.84667468,229.89806669)(390.52460062,229.51915604)(390.10022541,229.25391884)
\curveto(389.67962988,228.99247024)(389.23630442,228.86174607)(388.7702477,228.86174593)
\curveto(388.42922474,228.86174607)(388.12230712,228.93373909)(387.84949391,229.07772521)
\curveto(387.58046489,229.22171118)(387.35880216,229.40358829)(387.18450506,229.62335709)
\lineto(387.18450506,226.68490208)
\lineto(386.16144528,226.68490208)
\moveto(387.08788274,231.98207823)
\curveto(387.08788105,231.20530842)(387.24512897,230.63125879)(387.55962697,230.25992762)
\curveto(387.87412064,229.88859392)(388.25492585,229.7029277)(388.70204372,229.70292841)
\curveto(389.15673319,229.7029277)(389.5451166,229.89427758)(389.86719512,230.27697861)
\curveto(390.19305381,230.6634662)(390.35598539,231.26025047)(390.35599035,232.06733321)
\curveto(390.35598539,232.83651876)(390.19684291,233.41246294)(389.87856246,233.7951675)
\curveto(389.56406213,234.17786245)(389.18704604,234.36921233)(388.74751304,234.3692177)
\curveto(388.31176244,234.36921233)(387.92527358,234.16460058)(387.5880453,233.75538184)
\curveto(387.25460174,233.34994269)(387.08788105,232.75884208)(387.08788274,231.98207823)
}
}
{
\newrgbcolor{curcolor}{0 0 0}
\pscustom[linestyle=none,fillstyle=solid,fillcolor=curcolor]
{
\newpath
\moveto(392.64082413,226.68490208)
\lineto(392.64082413,235.03420655)
\lineto(393.57294526,235.03420655)
\lineto(393.57294526,234.24986073)
\curveto(393.79271174,234.5567731)(394.04089821,234.78601404)(394.31750543,234.93758424)
\curveto(394.59410775,235.09293167)(394.92944368,235.17060835)(395.3235142,235.17061452)
\curveto(395.83882923,235.17060835)(396.29352201,235.03798962)(396.6875939,234.77275794)
\curveto(397.08165616,234.50751472)(397.37910102,234.13239317)(397.57992936,233.6473922)
\curveto(397.7807463,233.16617102)(397.88115762,232.63759067)(397.88116363,232.06164955)
\curveto(397.88115762,231.44402213)(397.76937898,230.88702348)(397.54582737,230.39065192)
\curveto(397.32605352,229.89806669)(397.00397947,229.51915604)(396.57960425,229.25391884)
\curveto(396.15900873,228.99247024)(395.71568327,228.86174607)(395.24962655,228.86174593)
\curveto(394.90860359,228.86174607)(394.60168597,228.93373909)(394.32887276,229.07772521)
\curveto(394.05984374,229.22171118)(393.83818101,229.40358829)(393.66388391,229.62335709)
\lineto(393.66388391,226.68490208)
\lineto(392.64082413,226.68490208)
\moveto(393.56726159,231.98207823)
\curveto(393.5672599,231.20530842)(393.72450782,230.63125879)(394.03900582,230.25992762)
\curveto(394.35349949,229.88859392)(394.73430469,229.7029277)(395.18142257,229.70292841)
\curveto(395.63611204,229.7029277)(396.02449545,229.89427758)(396.34657397,230.27697861)
\curveto(396.67243266,230.6634662)(396.83536424,231.26025047)(396.8353692,232.06733321)
\curveto(396.83536424,232.83651876)(396.67622176,233.41246294)(396.3579413,233.7951675)
\curveto(396.04344098,234.17786245)(395.66642489,234.36921233)(395.22689189,234.3692177)
\curveto(394.79114129,234.36921233)(394.40465243,234.16460058)(394.06742415,233.75538184)
\curveto(393.73398058,233.34994269)(393.5672599,232.75884208)(393.56726159,231.98207823)
}
}
{
\newrgbcolor{curcolor}{0 0 0}
\pscustom[linestyle=none,fillstyle=solid,fillcolor=curcolor]
{
\newpath
\moveto(404.50831778,232.64138342)
\lineto(398.98947868,230.28266228)
\lineto(398.98947868,231.30003839)
\lineto(403.36021737,233.11312765)
\lineto(398.98947868,234.90916591)
\lineto(398.98947868,235.92654202)
\lineto(404.50831778,233.59623921)
\lineto(404.50831778,232.64138342)
}
}
{
\newrgbcolor{curcolor}{0 0 0}
\pscustom[linestyle=none,fillstyle=solid,fillcolor=curcolor]
{
\newpath
\moveto(411.30598044,232.64138342)
\lineto(405.78714134,230.28266228)
\lineto(405.78714134,231.30003839)
\lineto(410.15788003,233.11312765)
\lineto(405.78714134,234.90916591)
\lineto(405.78714134,235.92654202)
\lineto(411.30598044,233.59623921)
\lineto(411.30598044,232.64138342)
}
}
{
\newrgbcolor{curcolor}{0 0 0}
\pscustom[linestyle=none,fillstyle=solid,fillcolor=curcolor]
{
\newpath
\moveto(376.06840406,213.26007464)
\lineto(377.6711977,213.50447225)
\curveto(377.73939974,213.19376355)(377.87770213,212.9569444)(378.08610528,212.79401408)
\curveto(378.29450384,212.63487035)(378.58626504,212.55529911)(378.96138975,212.55530013)
\curveto(379.37439919,212.55529911)(379.68510592,212.63108124)(379.89351087,212.78264674)
\curveto(380.03370372,212.88874048)(380.10380218,213.03083197)(380.10380649,213.20892165)
\curveto(380.10380218,213.33017139)(380.06591112,213.43058271)(379.99013319,213.51015592)
\curveto(379.91055775,213.58593607)(379.73246975,213.65603454)(379.45586864,213.72045153)
\curveto(378.16756878,214.00463234)(377.35101633,214.26418613)(377.00620885,214.49911369)
\curveto(376.52878022,214.82497389)(376.29006652,215.27777211)(376.29006701,215.85750972)
\curveto(376.29006652,216.3804021)(376.49657282,216.81993845)(376.90958654,217.1761201)
\curveto(377.32259803,217.53229047)(377.96295703,217.71037847)(378.83066544,217.71038464)
\curveto(379.65668762,217.71037847)(380.27052287,217.57586519)(380.67217303,217.3068444)
\curveto(381.07381344,217.03781207)(381.35041822,216.63995589)(381.50198818,216.11327467)
\lineto(379.99581685,215.83477506)
\curveto(379.93139784,216.06969537)(379.80825188,216.24967793)(379.6263786,216.37472328)
\curveto(379.44828676,216.49975895)(379.19252208,216.56227921)(378.85908377,216.56228423)
\curveto(378.43848989,216.56227921)(378.13725592,216.50354806)(377.95538097,216.38609061)
\curveto(377.83412741,216.30272542)(377.7735017,216.19473588)(377.77350368,216.06212168)
\curveto(377.7735017,215.94844396)(377.82654919,215.85182175)(377.93264631,215.77225474)
\curveto(378.07663022,215.66615553)(378.57300317,215.51648582)(379.42176665,215.32324518)
\curveto(380.27431198,215.12999696)(380.86920169,214.89317781)(381.20643758,214.612787)
\curveto(381.53987354,214.32860094)(381.70659422,213.93263932)(381.70660014,213.42490093)
\curveto(381.70659422,212.8716895)(381.47545873,212.39615664)(381.01319296,211.99830092)
\curveto(380.55091675,211.60044428)(379.86698303,211.40151619)(378.96138975,211.40151605)
\curveto(378.13915048,211.40151619)(377.48742416,211.56823687)(377.00620885,211.90167861)
\curveto(376.52878022,212.23511961)(376.21617894,212.68791784)(376.06840406,213.26007464)
}
}
{
\newrgbcolor{curcolor}{0 0 0}
\pscustom[linestyle=none,fillstyle=solid,fillcolor=curcolor]
{
\newpath
\moveto(385.87841039,217.57397668)
\lineto(385.87841039,216.30083563)
\lineto(384.78714663,216.30083563)
\lineto(384.78714663,213.86822684)
\curveto(384.78714412,213.37564066)(384.79661689,213.08766857)(384.81556496,213.00430969)
\curveto(384.83829706,212.92473699)(384.88566089,212.85842763)(384.95765659,212.80538141)
\curveto(385.03343604,212.75233265)(385.1243746,212.7258089)(385.23047253,212.72581009)
\curveto(385.37824473,212.7258089)(385.59232925,212.77696184)(385.87272672,212.87926906)
\lineto(386.00913469,211.64023)
\curveto(385.63779852,211.48108742)(385.2172077,211.40151619)(384.74736097,211.40151605)
\curveto(384.45938641,211.40151619)(384.19983262,211.44888002)(383.96869881,211.54360769)
\curveto(383.73756163,211.64212445)(383.56705183,211.76716496)(383.45716893,211.9187296)
\curveto(383.35107277,212.07408259)(383.27718519,212.28248344)(383.23550598,212.5439328)
\curveto(383.20140306,212.72959801)(383.18435208,213.10471955)(383.18435299,213.66929855)
\lineto(383.18435299,216.30083563)
\lineto(382.45116015,216.30083563)
\lineto(382.45116015,217.57397668)
\lineto(383.18435299,217.57397668)
\lineto(383.18435299,218.77323007)
\lineto(384.78714663,219.7053512)
\lineto(384.78714663,217.57397668)
\lineto(385.87841039,217.57397668)
}
}
{
\newrgbcolor{curcolor}{0 0 0}
\pscustom[linestyle=none,fillstyle=solid,fillcolor=curcolor]
{
\newpath
\moveto(388.19166283,215.73246909)
\lineto(386.74232815,215.99391769)
\curveto(386.90525915,216.57743564)(387.18565303,217.00939377)(387.58351063,217.28979341)
\curveto(387.98136539,217.57018153)(388.572466,217.71037847)(389.35681423,217.71038464)
\curveto(390.06916306,217.71037847)(390.59963796,217.62512358)(390.94824055,217.4546197)
\curveto(391.29683356,217.2878931)(391.54123092,217.07380858)(391.68143338,216.81236551)
\curveto(391.82541391,216.554701)(391.89740693,216.07916813)(391.89741267,215.3857655)
\lineto(391.88036167,213.52152325)
\curveto(391.88035595,212.99104636)(391.90498514,212.59887384)(391.95424932,212.34500451)
\curveto(392.00729102,212.09492267)(392.10391323,211.82589611)(392.24411626,211.53792402)
\lineto(390.66405728,211.53792402)
\curveto(390.6223726,211.644019)(390.57121967,211.80126692)(390.51059831,212.00966825)
\curveto(390.48407022,212.10439544)(390.46512468,212.1669157)(390.45376166,212.19722921)
\curveto(390.1809417,211.9319911)(389.8891805,211.73306301)(389.57847718,211.60044434)
\curveto(389.26776704,211.46782555)(388.93622022,211.40151619)(388.58383574,211.40151605)
\curveto(387.96241986,211.40151619)(387.47173057,211.57013143)(387.1117664,211.90736227)
\curveto(386.75558944,212.24459238)(386.57750144,212.67086686)(386.57750185,213.18618699)
\curveto(386.57750144,213.52720492)(386.65896723,213.83033344)(386.82189947,214.09557345)
\curveto(386.98483039,214.36459745)(387.21217677,214.5692092)(387.50393931,214.70940931)
\curveto(387.79948828,214.85339219)(388.2238682,214.9784327)(388.77708036,215.08453123)
\curveto(389.52353172,215.22472462)(390.04074476,215.3554488)(390.32872102,215.47670414)
\lineto(390.32872102,215.63584677)
\curveto(390.32871685,215.9427603)(390.25293472,216.16063392)(390.1013744,216.2894683)
\curveto(389.9498062,216.42208227)(389.66372866,216.48839163)(389.24314093,216.48839658)
\curveto(388.95895486,216.48839163)(388.73729213,216.43155504)(388.57815207,216.31788662)
\curveto(388.41900719,216.20799775)(388.29017757,216.01285877)(388.19166283,215.73246909)
\moveto(390.32872102,214.43659338)
\curveto(390.1241051,214.36838656)(389.8001365,214.28692077)(389.35681423,214.19219576)
\curveto(388.91348558,214.09746545)(388.62361894,214.00463234)(388.48721343,213.91369616)
\curveto(388.27881025,213.76591863)(388.17460982,213.57835786)(388.17461183,213.35101328)
\curveto(388.17460982,213.12745419)(388.25797016,212.93420976)(388.42469311,212.77127941)
\curveto(388.59141153,212.6083466)(388.80360149,212.52688081)(389.06126363,212.5268818)
\curveto(389.34923283,212.52688081)(389.62394305,212.62160847)(389.88539512,212.81106507)
\curveto(390.07863582,212.95504984)(390.20557089,213.1312433)(390.2662007,213.33964595)
\curveto(390.30787676,213.47605198)(390.32871685,213.73560578)(390.32872102,214.11830811)
\lineto(390.32872102,214.43659338)
}
}
{
\newrgbcolor{curcolor}{0 0 0}
\pscustom[linestyle=none,fillstyle=solid,fillcolor=curcolor]
{
\newpath
\moveto(396.24541521,217.57397668)
\lineto(396.24541521,216.30083563)
\lineto(395.15415146,216.30083563)
\lineto(395.15415146,213.86822684)
\curveto(395.15414895,213.37564066)(395.16362171,213.08766857)(395.18256978,213.00430969)
\curveto(395.20530188,212.92473699)(395.25266571,212.85842763)(395.32466142,212.80538141)
\curveto(395.40044087,212.75233265)(395.49137942,212.7258089)(395.59747736,212.72581009)
\curveto(395.74524956,212.7258089)(395.95933407,212.77696184)(396.23973155,212.87926906)
\lineto(396.37613952,211.64023)
\curveto(396.00480335,211.48108742)(395.58421253,211.40151619)(395.1143658,211.40151605)
\curveto(394.82639124,211.40151619)(394.56683744,211.44888002)(394.33570364,211.54360769)
\curveto(394.10456645,211.64212445)(393.93405666,211.76716496)(393.82417375,211.9187296)
\curveto(393.71807759,212.07408259)(393.64419001,212.28248344)(393.6025108,212.5439328)
\curveto(393.56840789,212.72959801)(393.55135691,213.10471955)(393.55135782,213.66929855)
\lineto(393.55135782,216.30083563)
\lineto(392.81816498,216.30083563)
\lineto(392.81816498,217.57397668)
\lineto(393.55135782,217.57397668)
\lineto(393.55135782,218.77323007)
\lineto(395.15415146,219.7053512)
\lineto(395.15415146,217.57397668)
\lineto(396.24541521,217.57397668)
}
}
{
\newrgbcolor{curcolor}{0 0 0}
\pscustom[linestyle=none,fillstyle=solid,fillcolor=curcolor]
{
\newpath
\moveto(396.80241638,213.26007464)
\lineto(398.40521002,213.50447225)
\curveto(398.47341206,213.19376355)(398.61171445,212.9569444)(398.82011759,212.79401408)
\curveto(399.02851616,212.63487035)(399.32027736,212.55529911)(399.69540207,212.55530013)
\curveto(400.10841151,212.55529911)(400.41911824,212.63108124)(400.62752319,212.78264674)
\curveto(400.76771603,212.88874048)(400.8378145,213.03083197)(400.83781881,213.20892165)
\curveto(400.8378145,213.33017139)(400.79992344,213.43058271)(400.7241455,213.51015592)
\curveto(400.64457007,213.58593607)(400.46648207,213.65603454)(400.18988095,213.72045153)
\curveto(398.90158109,214.00463234)(398.08502865,214.26418613)(397.74022117,214.49911369)
\curveto(397.26279254,214.82497389)(397.02407883,215.27777211)(397.02407933,215.85750972)
\curveto(397.02407883,216.3804021)(397.23058514,216.81993845)(397.64359886,217.1761201)
\curveto(398.05661035,217.53229047)(398.69696934,217.71037847)(399.56467776,217.71038464)
\curveto(400.39069994,217.71037847)(401.00453519,217.57586519)(401.40618535,217.3068444)
\curveto(401.80782576,217.03781207)(402.08443053,216.63995589)(402.2360005,216.11327467)
\lineto(400.72982917,215.83477506)
\curveto(400.66541016,216.06969537)(400.5422642,216.24967793)(400.36039092,216.37472328)
\curveto(400.18229908,216.49975895)(399.92653439,216.56227921)(399.59309609,216.56228423)
\curveto(399.17250221,216.56227921)(398.87126824,216.50354806)(398.68939329,216.38609061)
\curveto(398.56813972,216.30272542)(398.50751402,216.19473588)(398.507516,216.06212168)
\curveto(398.50751402,215.94844396)(398.56056151,215.85182175)(398.66665863,215.77225474)
\curveto(398.81064254,215.66615553)(399.30701549,215.51648582)(400.15577896,215.32324518)
\curveto(401.00832429,215.12999696)(401.60321401,214.89317781)(401.9404499,214.612787)
\curveto(402.27388586,214.32860094)(402.44060654,213.93263932)(402.44061245,213.42490093)
\curveto(402.44060654,212.8716895)(402.20947105,212.39615664)(401.74720527,211.99830092)
\curveto(401.28492907,211.60044428)(400.60099535,211.40151619)(399.69540207,211.40151605)
\curveto(398.87316279,211.40151619)(398.22143648,211.56823687)(397.74022117,211.90167861)
\curveto(397.26279254,212.23511961)(396.95019126,212.68791784)(396.80241638,213.26007464)
}
}
{
\newrgbcolor{curcolor}{0 1 0.25098041}
\pscustom[linewidth=2.32802935,linecolor=curcolor]
{
\newpath
\moveto(581.82618442,386.71497242)
\lineto(656.32312136,386.71497242)
\lineto(656.32312136,433.27556384)
\lineto(606.27049025,433.27556384)
\lineto(606.27049025,442.58768212)
\lineto(581.82618442,442.58768212)
\lineto(581.82618442,386.71497242)
\closepath
}
}
{
\newrgbcolor{curcolor}{0 0 0}
\pscustom[linestyle=none,fillstyle=solid,fillcolor=curcolor]
{
\newpath
\moveto(595.26689951,415.39170228)
\lineto(595.26689951,416.34655806)
\lineto(600.78573861,418.67686088)
\lineto(600.78573861,417.65948477)
\lineto(596.40931625,415.8634465)
\lineto(600.78573861,414.05035724)
\lineto(600.78573861,413.03298114)
\lineto(595.26689951,415.39170228)
}
}
{
\newrgbcolor{curcolor}{0 0 0}
\pscustom[linestyle=none,fillstyle=solid,fillcolor=curcolor]
{
\newpath
\moveto(602.0645635,415.39170228)
\lineto(602.0645635,416.34655806)
\lineto(607.5834026,418.67686088)
\lineto(607.5834026,417.65948477)
\lineto(603.20698025,415.8634465)
\lineto(607.5834026,414.05035724)
\lineto(607.5834026,413.03298114)
\lineto(602.0645635,415.39170228)
}
}
{
\newrgbcolor{curcolor}{0 0 0}
\pscustom[linestyle=none,fillstyle=solid,fillcolor=curcolor]
{
\newpath
\moveto(612.93173192,412.49303292)
\curveto(612.55281657,412.17095813)(612.18716779,411.94361174)(611.8347845,411.81099308)
\curveto(611.48618309,411.67837429)(611.11106155,411.61206492)(610.70941875,411.61206479)
\curveto(610.04632263,411.61206492)(609.53668781,411.77310195)(609.18051276,412.09517635)
\curveto(608.82433579,412.42103916)(608.64624779,412.83594632)(608.64624821,413.33989907)
\curveto(608.64624779,413.63544778)(608.71255715,413.90447434)(608.8451765,414.14697955)
\curveto(608.98158371,414.39326908)(609.15777716,414.59030261)(609.37375738,414.73808076)
\curveto(609.59352441,414.88585292)(609.83981633,414.99763156)(610.11263388,415.07341701)
\curveto(610.31345464,415.12646118)(610.61658316,415.17761412)(611.02202035,415.22687598)
\curveto(611.84804276,415.32538927)(612.45619435,415.44285157)(612.84647694,415.57926323)
\curveto(612.85026143,415.71945634)(612.85215598,415.80850035)(612.85216061,415.84639551)
\curveto(612.85215598,416.26319312)(612.75553376,416.55684887)(612.56229367,416.72736365)
\curveto(612.30084099,416.95849416)(611.91245757,417.07406191)(611.39714226,417.07406723)
\curveto(610.91592257,417.07406191)(610.55974656,416.98880701)(610.32861317,416.81830229)
\curveto(610.10126468,416.65157654)(609.93264944,416.35413168)(609.82276695,415.92596682)
\lineto(608.82244184,416.06237479)
\curveto(608.9133798,416.49053951)(609.0630495,416.8353482)(609.2714514,417.0968019)
\curveto(609.47985121,417.362034)(609.78108518,417.5647512)(610.1751542,417.70495409)
\curveto(610.56921933,417.84893418)(611.02580666,417.92092721)(611.54491756,417.92093338)
\curveto(612.06023273,417.92092721)(612.47892899,417.8603015)(612.80100762,417.73905609)
\curveto(613.12307709,417.61779869)(613.35989625,417.46433988)(613.51146579,417.27867919)
\curveto(613.66302476,417.09679655)(613.76911975,416.86566105)(613.82975105,416.58527201)
\curveto(613.86384741,416.41096828)(613.88089839,416.09647244)(613.88090404,415.64178355)
\lineto(613.88090404,414.27770386)
\curveto(613.88089839,413.3266356)(613.90173847,412.72416767)(613.94342436,412.47029826)
\curveto(613.98888792,412.22021651)(614.07603737,411.97960825)(614.20487297,411.74847276)
\lineto(613.13634388,411.74847276)
\curveto(613.03024398,411.96066272)(612.96204007,412.20884919)(612.93173192,412.49303292)
\moveto(612.84647694,414.77786641)
\curveto(612.47513988,414.62629913)(611.91814123,414.49746951)(611.17547931,414.39137717)
\curveto(610.75488554,414.33074882)(610.45744069,414.2625449)(610.28314385,414.18676521)
\curveto(610.10884289,414.11098064)(609.97432961,413.999202)(609.8796036,413.85142895)
\curveto(609.78487429,413.7074408)(609.73751046,413.54640378)(609.73751197,413.36831739)
\curveto(609.73751046,413.09550011)(609.83981633,412.86815372)(610.0444299,412.68627755)
\curveto(610.25282894,412.5043995)(610.55595745,412.41346094)(610.95381636,412.41346161)
\curveto(611.34788071,412.41346094)(611.69837306,412.49871584)(612.00529446,412.66922655)
\curveto(612.31220831,412.84352453)(612.53766014,413.08034368)(612.68165064,413.37968473)
\curveto(612.79153028,413.61081859)(612.84647232,413.95183817)(612.84647694,414.4027445)
\lineto(612.84647694,414.77786641)
}
}
{
\newrgbcolor{curcolor}{0 0 0}
\pscustom[linestyle=none,fillstyle=solid,fillcolor=curcolor]
{
\newpath
\moveto(615.47232976,409.43522094)
\lineto(615.47232976,417.78452541)
\lineto(616.40445089,417.78452541)
\lineto(616.40445089,417.00017958)
\curveto(616.62421736,417.30709196)(616.87240384,417.5363329)(617.14901105,417.6879031)
\curveto(617.42561338,417.84325052)(617.76094931,417.92092721)(618.15501983,417.92093338)
\curveto(618.67033486,417.92092721)(619.12502764,417.78830848)(619.51909952,417.5230768)
\curveto(619.91316178,417.25783357)(620.21060664,416.88271203)(620.41143499,416.39771105)
\curveto(620.61225193,415.91648988)(620.71266325,415.38790953)(620.71266926,414.81196841)
\curveto(620.71266325,414.19434099)(620.60088461,413.63734233)(620.377333,413.14097078)
\curveto(620.15755915,412.64838554)(619.8354851,412.2694749)(619.41110988,412.0042377)
\curveto(618.99051436,411.7427891)(618.5471889,411.61206492)(618.08113218,411.61206479)
\curveto(617.74010922,411.61206492)(617.43319159,411.68405795)(617.16037838,411.82804407)
\curveto(616.89134937,411.97203004)(616.66968664,412.15390715)(616.49538953,412.37367595)
\lineto(616.49538953,409.43522094)
\lineto(615.47232976,409.43522094)
\moveto(616.39876722,414.73239709)
\curveto(616.39876553,413.95562728)(616.55601345,413.38157765)(616.87051145,413.01024647)
\curveto(617.18500512,412.63891278)(617.56581032,412.45324656)(618.01292819,412.45324727)
\curveto(618.46761766,412.45324656)(618.85600108,412.64459644)(619.1780796,413.02729747)
\curveto(619.50393828,413.41378505)(619.66686986,414.01056932)(619.66687482,414.81765207)
\curveto(619.66686986,415.58683762)(619.50772739,416.1627818)(619.18944693,416.54548635)
\curveto(618.87494661,416.92818131)(618.49793051,417.11953119)(618.05839752,417.11953656)
\curveto(617.62264692,417.11953119)(617.23615806,416.91491944)(616.89892978,416.50570069)
\curveto(616.56548621,416.10026154)(616.39876553,415.50916093)(616.39876722,414.73239709)
}
}
{
\newrgbcolor{curcolor}{0 0 0}
\pscustom[linestyle=none,fillstyle=solid,fillcolor=curcolor]
{
\newpath
\moveto(621.95170861,409.43522094)
\lineto(621.95170861,417.78452541)
\lineto(622.88382974,417.78452541)
\lineto(622.88382974,417.00017958)
\curveto(623.10359621,417.30709196)(623.35178269,417.5363329)(623.6283899,417.6879031)
\curveto(623.90499223,417.84325052)(624.24032815,417.92092721)(624.63439868,417.92093338)
\curveto(625.14971371,417.92092721)(625.60440649,417.78830848)(625.99847837,417.5230768)
\curveto(626.39254063,417.25783357)(626.68998549,416.88271203)(626.89081384,416.39771105)
\curveto(627.09163078,415.91648988)(627.1920421,415.38790953)(627.19204811,414.81196841)
\curveto(627.1920421,414.19434099)(627.08026346,413.63734233)(626.85671185,413.14097078)
\curveto(626.636938,412.64838554)(626.31486395,412.2694749)(625.89048873,412.0042377)
\curveto(625.46989321,411.7427891)(625.02656775,411.61206492)(624.56051103,411.61206479)
\curveto(624.21948807,411.61206492)(623.91257044,411.68405795)(623.63975723,411.82804407)
\curveto(623.37072822,411.97203004)(623.14906549,412.15390715)(622.97476838,412.37367595)
\lineto(622.97476838,409.43522094)
\lineto(621.95170861,409.43522094)
\moveto(622.87814607,414.73239709)
\curveto(622.87814438,413.95562728)(623.03539229,413.38157765)(623.3498903,413.01024647)
\curveto(623.66438397,412.63891278)(624.04518917,412.45324656)(624.49230704,412.45324727)
\curveto(624.94699651,412.45324656)(625.33537993,412.64459644)(625.65745845,413.02729747)
\curveto(625.98331713,413.41378505)(626.14624871,414.01056932)(626.14625367,414.81765207)
\curveto(626.14624871,415.58683762)(625.98710624,416.1627818)(625.66882578,416.54548635)
\curveto(625.35432546,416.92818131)(624.97730936,417.11953119)(624.53777637,417.11953656)
\curveto(624.10202577,417.11953119)(623.71553691,416.91491944)(623.37830862,416.50570069)
\curveto(623.04486506,416.10026154)(622.87814438,415.50916093)(622.87814607,414.73239709)
}
}
{
\newrgbcolor{curcolor}{0 0 0}
\pscustom[linestyle=none,fillstyle=solid,fillcolor=curcolor]
{
\newpath
\moveto(633.81920226,415.39170228)
\lineto(628.30036315,413.03298114)
\lineto(628.30036315,414.05035724)
\lineto(632.67110185,415.8634465)
\lineto(628.30036315,417.65948477)
\lineto(628.30036315,418.67686088)
\lineto(633.81920226,416.34655806)
\lineto(633.81920226,415.39170228)
}
}
{
\newrgbcolor{curcolor}{0 0 0}
\pscustom[linestyle=none,fillstyle=solid,fillcolor=curcolor]
{
\newpath
\moveto(640.61686492,415.39170228)
\lineto(635.09802582,413.03298114)
\lineto(635.09802582,414.05035724)
\lineto(639.46876451,415.8634465)
\lineto(635.09802582,417.65948477)
\lineto(635.09802582,418.67686088)
\lineto(640.61686492,416.34655806)
\lineto(640.61686492,415.39170228)
}
}
{
\newrgbcolor{curcolor}{0 0 0}
\pscustom[linestyle=none,fillstyle=solid,fillcolor=curcolor]
{
\newpath
\moveto(595.3180525,393.8903863)
\lineto(597.14250909,393.66872335)
\curveto(597.17281943,393.45653401)(597.2429179,393.31065341)(597.35280471,393.23108111)
\curveto(597.50436625,393.11740898)(597.74307995,393.06057238)(598.06894655,393.06057115)
\curveto(598.48574482,393.06057238)(598.79834611,393.12309264)(599.00675134,393.24813211)
\curveto(599.1469439,393.33149349)(599.25303889,393.46600677)(599.3250366,393.65167235)
\curveto(599.37429029,393.78429172)(599.39891948,394.02868909)(599.39892425,394.38486519)
\lineto(599.39892425,395.26583333)
\curveto(598.92149207,394.61410604)(598.31902414,394.28824288)(597.59151866,394.28824288)
\curveto(596.78064691,394.28824288)(596.13839336,394.63115702)(595.66475609,395.31698632)
\curveto(595.29342262,395.85882751)(595.1077564,396.53328847)(595.10775688,397.3403712)
\curveto(595.1077564,398.35205958)(595.35025921,399.1250373)(595.83526605,399.65930668)
\curveto(596.32405958,400.19356532)(596.93031662,400.46069733)(597.65403898,400.4607035)
\curveto(598.40048993,400.46069733)(599.01621973,400.13293962)(599.50123023,399.47742939)
\lineto(599.50123023,400.32429553)
\lineto(600.99603423,400.32429553)
\lineto(600.99603423,394.90776241)
\curveto(600.99602786,394.19540977)(600.93729671,393.66304031)(600.8198406,393.31065243)
\curveto(600.70237211,392.95826651)(600.53754598,392.68166173)(600.32536171,392.48083728)
\curveto(600.11316606,392.28001645)(599.82898307,392.12276853)(599.4728119,392.00909305)
\curveto(599.12042016,391.89542214)(598.67330559,391.83858554)(598.13146687,391.83858309)
\curveto(597.10840462,391.83858554)(596.38279073,392.01477899)(595.95462302,392.36716397)
\curveto(595.52645267,392.71576369)(595.31236815,393.15908915)(595.31236883,393.69714168)
\curveto(595.31236815,393.75018976)(595.3142627,393.81460457)(595.3180525,393.8903863)
\moveto(596.74465251,397.43130984)
\curveto(596.7446504,396.79094771)(596.86779636,396.3210985)(597.11409076,396.02176083)
\curveto(597.36416931,395.72620879)(597.67108693,395.57843363)(598.03484456,395.57843492)
\curveto(598.42511912,395.57843363)(598.75477138,395.72999789)(599.02380234,396.03312816)
\curveto(599.2928245,396.34004404)(599.42733778,396.79284226)(599.42734258,397.39152419)
\curveto(599.42733778,398.01672365)(599.29850816,398.4808892)(599.04085333,398.78402221)
\curveto(598.78318968,399.08714623)(598.45732653,399.23871049)(598.06326288,399.23871544)
\curveto(597.6805597,399.23871049)(597.36416931,399.08904078)(597.11409076,398.78970587)
\curveto(596.86779636,398.49415107)(596.7446504,398.04135284)(596.74465251,397.43130984)
}
}
{
\newrgbcolor{curcolor}{0 0 0}
\pscustom[linestyle=none,fillstyle=solid,fillcolor=curcolor]
{
\newpath
\moveto(606.07723116,396.20932178)
\lineto(607.66865747,395.94218951)
\curveto(607.4640398,395.35866546)(607.14007119,394.91344545)(606.69675068,394.60652814)
\curveto(606.25720938,394.3033993)(605.70589439,394.15183505)(605.04280405,394.15183491)
\curveto(603.99321826,394.15183505)(603.21645143,394.49474918)(602.71250124,395.18057835)
\curveto(602.31464409,395.72999789)(602.115716,396.42340438)(602.11571637,397.26079988)
\curveto(602.115716,398.26112102)(602.37716435,399.04357151)(602.9000622,399.60815369)
\curveto(603.42295774,400.17651434)(604.08415682,400.46069733)(604.88366142,400.4607035)
\curveto(605.78167652,400.46069733)(606.49023943,400.16325247)(607.00935228,399.56836803)
\curveto(607.52845461,398.97726214)(607.77664108,398.06977114)(607.75391245,396.84589231)
\lineto(603.75261201,396.84589231)
\curveto(603.76397732,396.37225144)(603.89280694,396.00281356)(604.13910125,395.73757756)
\curveto(604.38539078,395.47612776)(604.69230841,395.34540359)(605.05985505,395.34540464)
\curveto(605.30993276,395.34540359)(605.52022817,395.4136075)(605.69074191,395.5500166)
\curveto(605.86124776,395.68642317)(605.99007738,395.90619134)(606.07723116,396.20932178)
\moveto(606.1681698,397.82348276)
\curveto(606.15679806,398.28575021)(606.03744121,398.63624256)(605.81009888,398.87496086)
\curveto(605.58274843,399.11745908)(605.30614366,399.23871049)(604.98028373,399.23871544)
\curveto(604.6316827,399.23871049)(604.34371061,399.11177542)(604.11636659,398.85790986)
\curveto(603.88901783,398.60403516)(603.77723919,398.25922647)(603.78103033,397.82348276)
\lineto(606.1681698,397.82348276)
}
}
{
\newrgbcolor{curcolor}{0 0 0}
\pscustom[linestyle=none,fillstyle=solid,fillcolor=curcolor]
{
\newpath
\moveto(608.49847202,396.01039349)
\lineto(610.10126566,396.25479111)
\curveto(610.16946771,395.94408241)(610.30777009,395.70726325)(610.51617324,395.54433293)
\curveto(610.7245718,395.3851892)(611.016333,395.30561797)(611.39145771,395.30561899)
\curveto(611.80446715,395.30561797)(612.11517388,395.3814001)(612.32357883,395.5329656)
\curveto(612.46377168,395.63905934)(612.53387015,395.78115083)(612.53387445,395.95924051)
\curveto(612.53387015,396.08049024)(612.49597908,396.18090156)(612.42020115,396.26047477)
\curveto(612.34062572,396.33625493)(612.16253771,396.4063534)(611.8859366,396.47077039)
\curveto(610.59763674,396.7549512)(609.78108429,397.01450499)(609.43627681,397.24943255)
\curveto(608.95884819,397.57529275)(608.72013448,398.02809097)(608.72013497,398.60782858)
\curveto(608.72013448,399.13072096)(608.92664078,399.57025731)(609.3396545,399.92643895)
\curveto(609.75266599,400.28260932)(610.39302499,400.46069733)(611.26073341,400.4607035)
\curveto(612.08675558,400.46069733)(612.70059083,400.32618405)(613.10224099,400.05716326)
\curveto(613.5038814,399.78813093)(613.78048618,399.39027475)(613.93205614,398.86359352)
\lineto(612.42588481,398.58509392)
\curveto(612.3614658,398.82001422)(612.23831984,398.99999678)(612.05644656,399.12504213)
\curveto(611.87835473,399.25007781)(611.62259004,399.31259807)(611.28915173,399.31260309)
\curveto(610.86855785,399.31259807)(610.56732389,399.25386692)(610.38544893,399.13640946)
\curveto(610.26419537,399.05304427)(610.20356966,398.94505474)(610.20357164,398.81244054)
\curveto(610.20356966,398.69876282)(610.25661715,398.6021406)(610.36271427,398.5225736)
\curveto(610.50669818,398.41647439)(611.00307113,398.26680468)(611.85183461,398.07356403)
\curveto(612.70437994,397.88031582)(613.29926965,397.64349666)(613.63650554,397.36310586)
\curveto(613.9699415,397.0789198)(614.13666219,396.68295817)(614.1366681,396.17521979)
\curveto(614.13666219,395.62200836)(613.90552669,395.1464755)(613.44326092,394.74861978)
\curveto(612.98098471,394.35076314)(612.29705099,394.15183505)(611.39145771,394.15183491)
\curveto(610.56921844,394.15183505)(609.91749212,394.31855573)(609.43627681,394.65199746)
\curveto(608.95884819,394.98543847)(608.6462469,395.43823669)(608.49847202,396.01039349)
}
}
{
\newrgbcolor{curcolor}{0 0 0}
\pscustom[linestyle=none,fillstyle=solid,fillcolor=curcolor]
{
\newpath
\moveto(618.30847879,400.32429553)
\lineto(618.30847879,399.05115448)
\lineto(617.21721504,399.05115448)
\lineto(617.21721504,396.61854569)
\curveto(617.21721253,396.12595952)(617.22668529,395.83798743)(617.24563336,395.75462855)
\curveto(617.26836546,395.67505585)(617.31572929,395.60874649)(617.387725,395.55570026)
\curveto(617.46350445,395.5026515)(617.554443,395.47612776)(617.66054094,395.47612895)
\curveto(617.80831314,395.47612776)(618.02239765,395.5272807)(618.30279513,395.62958791)
\lineto(618.4392031,394.39054886)
\curveto(618.06786693,394.23140628)(617.64727611,394.15183505)(617.17742938,394.15183491)
\curveto(616.88945482,394.15183505)(616.62990102,394.19919888)(616.39876722,394.29392654)
\curveto(616.16763003,394.39244331)(615.99712024,394.51748382)(615.88723733,394.66904846)
\curveto(615.78114117,394.82440145)(615.70725359,395.0328023)(615.66557438,395.29425165)
\curveto(615.63147147,395.47991687)(615.61442049,395.85503841)(615.6144214,396.4196174)
\lineto(615.6144214,399.05115448)
\lineto(614.88122856,399.05115448)
\lineto(614.88122856,400.32429553)
\lineto(615.6144214,400.32429553)
\lineto(615.6144214,401.52354893)
\lineto(617.21721504,402.45567006)
\lineto(617.21721504,400.32429553)
\lineto(618.30847879,400.32429553)
}
}
{
\newrgbcolor{curcolor}{0 0 0}
\pscustom[linestyle=none,fillstyle=solid,fillcolor=curcolor]
{
\newpath
\moveto(619.42816105,401.14274335)
\lineto(619.42816105,402.62049635)
\lineto(621.02527103,402.62049635)
\lineto(621.02527103,401.14274335)
\lineto(619.42816105,401.14274335)
\moveto(619.42816105,394.28824288)
\lineto(619.42816105,400.32429553)
\lineto(621.02527103,400.32429553)
\lineto(621.02527103,394.28824288)
\lineto(619.42816105,394.28824288)
}
}
{
\newrgbcolor{curcolor}{0 0 0}
\pscustom[linestyle=none,fillstyle=solid,fillcolor=curcolor]
{
\newpath
\moveto(622.28704552,397.39152419)
\curveto(622.28704505,397.92199599)(622.41776923,398.43541992)(622.67921843,398.93179751)
\curveto(622.94066592,399.42816581)(623.3101038,399.80707646)(623.78753319,400.06853059)
\curveto(624.26874774,400.32997316)(624.80490631,400.46069733)(625.39601049,400.4607035)
\curveto(626.30918158,400.46069733)(627.05753011,400.16325247)(627.64105833,399.56836803)
\curveto(628.2245749,398.97726214)(628.5163361,398.22891361)(628.5163428,397.3233202)
\curveto(628.5163361,396.41014251)(628.2207858,395.65232121)(627.629691,395.04985404)
\curveto(627.04237368,394.45117446)(626.30160337,394.15183505)(625.40737782,394.15183491)
\curveto(624.85416469,394.15183505)(624.32558434,394.27687556)(623.82163518,394.52695683)
\curveto(623.32147112,394.77703761)(622.94066592,395.14268639)(622.67921843,395.62390425)
\curveto(622.41776923,396.10890854)(622.28704505,396.6981146)(622.28704552,397.39152419)
\moveto(623.92394116,397.30626921)
\curveto(623.92393905,396.70758736)(624.06603055,396.24910548)(624.35021606,395.93082218)
\curveto(624.63439652,395.61253559)(624.98488887,395.45339312)(625.40169416,395.45339429)
\curveto(625.81849229,395.45339312)(626.16709009,395.61253559)(626.44748859,395.93082218)
\curveto(626.73166695,396.24910548)(626.87375844,396.71137647)(626.8737635,397.31763654)
\curveto(626.87375844,397.90873412)(626.73166695,398.36342689)(626.44748859,398.68171623)
\curveto(626.16709009,398.99999678)(625.81849229,399.15913925)(625.40169416,399.15914413)
\curveto(624.98488887,399.15913925)(624.63439652,398.99999678)(624.35021606,398.68171623)
\curveto(624.06603055,398.36342689)(623.92393905,397.90494501)(623.92394116,397.30626921)
}
}
{
\newrgbcolor{curcolor}{0 0 0}
\pscustom[linestyle=none,fillstyle=solid,fillcolor=curcolor]
{
\newpath
\moveto(635.26285457,394.28824288)
\lineto(633.6657446,394.28824288)
\lineto(633.6657446,397.36878952)
\curveto(633.66573987,398.02051276)(633.63163791,398.44110358)(633.56343862,398.63056324)
\curveto(633.49523008,398.82380333)(633.38345143,398.97347304)(633.22810236,399.07957281)
\curveto(633.07653381,399.185663)(632.89276215,399.23871049)(632.67678682,399.23871544)
\curveto(632.4001783,399.23871049)(632.15199183,399.16292836)(631.93222665,399.01136883)
\curveto(631.71245548,398.85979984)(631.56089122,398.6589772)(631.47753342,398.40890029)
\curveto(631.39795964,398.15881514)(631.35817402,397.69654415)(631.35817644,397.02208594)
\lineto(631.35817644,394.28824288)
\lineto(629.76106647,394.28824288)
\lineto(629.76106647,400.32429553)
\lineto(631.24450314,400.32429553)
\lineto(631.24450314,399.43764373)
\curveto(631.77118663,400.11967775)(632.43428026,400.46069733)(633.23378603,400.4607035)
\curveto(633.58616863,400.46069733)(633.90824268,400.39628252)(634.20000914,400.26745888)
\curveto(634.49176508,400.14241239)(634.71153325,399.98137536)(634.85931433,399.78434732)
\curveto(635.01087267,399.58730829)(635.11507309,399.363751)(635.17191593,399.1136748)
\curveto(635.23253539,398.86358895)(635.26284825,398.50551839)(635.26285457,398.03946204)
\lineto(635.26285457,394.28824288)
}
}
{
\newrgbcolor{curcolor}{0 0 0}
\pscustom[linestyle=none,fillstyle=solid,fillcolor=curcolor]
{
\newpath
\moveto(636.32569917,396.01039349)
\lineto(637.92849281,396.25479111)
\curveto(637.99669486,395.94408241)(638.13499724,395.70726325)(638.34340039,395.54433293)
\curveto(638.55179895,395.3851892)(638.84356015,395.30561797)(639.21868486,395.30561899)
\curveto(639.6316943,395.30561797)(639.94240103,395.3814001)(640.15080599,395.5329656)
\curveto(640.29099883,395.63905934)(640.3610973,395.78115083)(640.36110161,395.95924051)
\curveto(640.3610973,396.08049024)(640.32320623,396.18090156)(640.2474283,396.26047477)
\curveto(640.16785287,396.33625493)(639.98976486,396.4063534)(639.71316375,396.47077039)
\curveto(638.42486389,396.7549512)(637.60831144,397.01450499)(637.26350396,397.24943255)
\curveto(636.78607534,397.57529275)(636.54736163,398.02809097)(636.54736212,398.60782858)
\curveto(636.54736163,399.13072096)(636.75386793,399.57025731)(637.16688165,399.92643895)
\curveto(637.57989314,400.28260932)(638.22025214,400.46069733)(639.08796056,400.4607035)
\curveto(639.91398273,400.46069733)(640.52781798,400.32618405)(640.92946815,400.05716326)
\curveto(641.33110856,399.78813093)(641.60771333,399.39027475)(641.75928329,398.86359352)
\lineto(640.25311196,398.58509392)
\curveto(640.18869295,398.82001422)(640.06554699,398.99999678)(639.88367371,399.12504213)
\curveto(639.70558188,399.25007781)(639.44981719,399.31259807)(639.11637888,399.31260309)
\curveto(638.695785,399.31259807)(638.39455104,399.25386692)(638.21267608,399.13640946)
\curveto(638.09142252,399.05304427)(638.03079681,398.94505474)(638.03079879,398.81244054)
\curveto(638.03079681,398.69876282)(638.0838443,398.6021406)(638.18994142,398.5225736)
\curveto(638.33392533,398.41647439)(638.83029828,398.26680468)(639.67906176,398.07356403)
\curveto(640.53160709,397.88031582)(641.12649681,397.64349666)(641.46373269,397.36310586)
\curveto(641.79716865,397.0789198)(641.96388934,396.68295817)(641.96389525,396.17521979)
\curveto(641.96388934,395.62200836)(641.73275384,395.1464755)(641.27048807,394.74861978)
\curveto(640.80821186,394.35076314)(640.12427814,394.15183505)(639.21868486,394.15183491)
\curveto(638.39644559,394.15183505)(637.74471928,394.31855573)(637.26350396,394.65199746)
\curveto(636.78607534,394.98543847)(636.47347405,395.43823669)(636.32569917,396.01039349)
}
}
{
\newrgbcolor{curcolor}{0 1 0.25098041}
\pscustom[linewidth=2.32802935,linecolor=curcolor]
{
\newpath
\moveto(456.11259692,386.71497242)
\lineto(538.75764202,386.71497242)
\lineto(538.75764202,433.27556384)
\lineto(484.04895176,433.27556384)
\lineto(484.04895176,442.58768212)
\lineto(456.11259692,442.58768212)
\lineto(456.11259692,386.71497242)
\closepath
}
}
{
\newrgbcolor{curcolor}{0 0 0}
\pscustom[linestyle=none,fillstyle=solid,fillcolor=curcolor]
{
\newpath
\moveto(474.20942321,416.55572139)
\lineto(474.20942321,417.51057718)
\lineto(479.72826231,419.84087999)
\lineto(479.72826231,418.82350389)
\lineto(475.35183996,417.02746562)
\lineto(479.72826231,415.21437636)
\lineto(479.72826231,414.19700025)
\lineto(474.20942321,416.55572139)
}
}
{
\newrgbcolor{curcolor}{0 0 0}
\pscustom[linestyle=none,fillstyle=solid,fillcolor=curcolor]
{
\newpath
\moveto(481.00708721,416.55572139)
\lineto(481.00708721,417.51057718)
\lineto(486.52592631,419.84087999)
\lineto(486.52592631,418.82350389)
\lineto(482.14950395,417.02746562)
\lineto(486.52592631,415.21437636)
\lineto(486.52592631,414.19700025)
\lineto(481.00708721,416.55572139)
}
}
{
\newrgbcolor{curcolor}{0 0 0}
\pscustom[linestyle=none,fillstyle=solid,fillcolor=curcolor]
{
\newpath
\moveto(491.87425563,413.65705204)
\curveto(491.49534027,413.33497724)(491.1296915,413.10763086)(490.7773082,412.97501219)
\curveto(490.4287068,412.8423934)(490.05358526,412.77608404)(489.65194245,412.7760839)
\curveto(488.98884634,412.77608404)(488.47921152,412.93712106)(488.12303646,413.25919546)
\curveto(487.7668595,413.58505827)(487.58877149,413.99996543)(487.58877191,414.50391818)
\curveto(487.58877149,414.7994669)(487.65508086,415.06849346)(487.7877002,415.31099867)
\curveto(487.92410742,415.55728819)(488.10030087,415.75432173)(488.31628109,415.90209987)
\curveto(488.53604811,416.04987203)(488.78234003,416.16165068)(489.05515759,416.23743613)
\curveto(489.25597834,416.2904803)(489.55910686,416.34163323)(489.96454405,416.3908951)
\curveto(490.79056647,416.48940839)(491.39871806,416.60687069)(491.78900064,416.74328235)
\curveto(491.79278513,416.88347546)(491.79467968,416.97251946)(491.79468431,417.01041462)
\curveto(491.79467968,417.42721224)(491.69805747,417.72086799)(491.50481737,417.89138276)
\curveto(491.24336469,418.12251328)(490.85498128,418.23808103)(490.33966597,418.23808635)
\curveto(489.85844627,418.23808103)(489.50227026,418.15282613)(489.27113687,417.98232141)
\curveto(489.04378838,417.81559565)(488.87517314,417.51815079)(488.76529065,417.08998594)
\lineto(487.76496554,417.22639391)
\curveto(487.8559035,417.65455863)(488.00557321,417.99936732)(488.21397511,418.26082101)
\curveto(488.42237492,418.52605312)(488.72360888,418.72877031)(489.11767791,418.86897321)
\curveto(489.51174303,419.0129533)(489.96833036,419.08494632)(490.48744127,419.0849525)
\curveto(491.00275643,419.08494632)(491.4214527,419.02432062)(491.74353132,418.9030752)
\curveto(492.0656008,418.7818178)(492.30241995,418.62835899)(492.4539895,418.44269831)
\curveto(492.60554847,418.26081566)(492.71164345,418.02968017)(492.77227476,417.74929113)
\curveto(492.80637111,417.57498739)(492.82342209,417.26049155)(492.82342775,416.80580267)
\lineto(492.82342775,415.44172297)
\curveto(492.82342209,414.49065472)(492.84426218,413.88818679)(492.88594807,413.63431738)
\curveto(492.93141163,413.38423563)(493.01856107,413.14362737)(493.14739667,412.91249187)
\lineto(492.07886758,412.91249187)
\curveto(491.97276769,413.12468184)(491.90456377,413.37286831)(491.87425563,413.65705204)
\moveto(491.78900064,415.94188553)
\curveto(491.41766359,415.79031824)(490.86066494,415.66148862)(490.11800302,415.55539628)
\curveto(489.69740925,415.49476794)(489.39996439,415.42656402)(489.22566755,415.35078433)
\curveto(489.05136659,415.27499976)(488.91685331,415.16322112)(488.82212731,415.01544807)
\curveto(488.72739799,414.87145992)(488.68003416,414.7104229)(488.68003567,414.53233651)
\curveto(488.68003416,414.25951922)(488.78234003,414.03217284)(488.9869536,413.85029666)
\curveto(489.19535264,413.66841861)(489.49848116,413.57748006)(489.89634007,413.57748072)
\curveto(490.29040441,413.57748006)(490.64089676,413.66273495)(490.94781817,413.83324567)
\curveto(491.25473201,414.00754364)(491.48018385,414.2443628)(491.62417435,414.54370384)
\curveto(491.73405398,414.77483771)(491.78899602,415.11585729)(491.78900064,415.56676361)
\lineto(491.78900064,415.94188553)
}
}
{
\newrgbcolor{curcolor}{0 0 0}
\pscustom[linestyle=none,fillstyle=solid,fillcolor=curcolor]
{
\newpath
\moveto(494.41485346,410.59924006)
\lineto(494.41485346,418.94854453)
\lineto(495.34697459,418.94854453)
\lineto(495.34697459,418.1641987)
\curveto(495.56674107,418.47111107)(495.81492754,418.70035202)(496.09153476,418.85192221)
\curveto(496.36813709,419.00726964)(496.70347301,419.08494632)(497.09754353,419.0849525)
\curveto(497.61285856,419.08494632)(498.06755134,418.9523276)(498.46162323,418.68709592)
\curveto(498.85568549,418.42185269)(499.15313035,418.04673115)(499.3539587,417.56173017)
\curveto(499.55477563,417.080509)(499.65518695,416.55192864)(499.65519296,415.97598752)
\curveto(499.65518695,415.3583601)(499.54340831,414.80136145)(499.3198567,414.3049899)
\curveto(499.10008286,413.81240466)(498.7780088,413.43349401)(498.35363359,413.16825682)
\curveto(497.93303806,412.90680821)(497.4897126,412.77608404)(497.02365588,412.7760839)
\curveto(496.68263292,412.77608404)(496.3757153,412.84807706)(496.10290209,412.99206319)
\curveto(495.83387307,413.13604915)(495.61221034,413.31792627)(495.43791324,413.53769507)
\lineto(495.43791324,410.59924006)
\lineto(494.41485346,410.59924006)
\moveto(495.34129092,415.89641621)
\curveto(495.34128923,415.1196464)(495.49853715,414.54559676)(495.81303515,414.17426559)
\curveto(496.12752882,413.80293189)(496.50833403,413.61726568)(496.9554519,413.61726638)
\curveto(497.41014137,413.61726568)(497.79852478,413.80861555)(498.1206033,414.19131659)
\curveto(498.44646199,414.57780417)(498.60939357,415.17458844)(498.60939853,415.98167119)
\curveto(498.60939357,416.75085673)(498.45025109,417.32680092)(498.13197064,417.70950547)
\curveto(497.81747031,418.09220043)(497.44045422,418.2835503)(497.00092122,418.28355567)
\curveto(496.56517062,418.2835503)(496.17868176,418.07893855)(495.84145348,417.66971981)
\curveto(495.50800992,417.26428066)(495.34128923,416.67318005)(495.34129092,415.89641621)
}
}
{
\newrgbcolor{curcolor}{0 0 0}
\pscustom[linestyle=none,fillstyle=solid,fillcolor=curcolor]
{
\newpath
\moveto(500.89423231,410.59924006)
\lineto(500.89423231,418.94854453)
\lineto(501.82635344,418.94854453)
\lineto(501.82635344,418.1641987)
\curveto(502.04611992,418.47111107)(502.29430639,418.70035202)(502.57091361,418.85192221)
\curveto(502.84751594,419.00726964)(503.18285186,419.08494632)(503.57692238,419.0849525)
\curveto(504.09223741,419.08494632)(504.54693019,418.9523276)(504.94100208,418.68709592)
\curveto(505.33506434,418.42185269)(505.6325092,418.04673115)(505.83333754,417.56173017)
\curveto(506.03415448,417.080509)(506.1345658,416.55192864)(506.13457181,415.97598752)
\curveto(506.1345658,415.3583601)(506.02278716,414.80136145)(505.79923555,414.3049899)
\curveto(505.5794617,413.81240466)(505.25738765,413.43349401)(504.83301243,413.16825682)
\curveto(504.41241691,412.90680821)(503.96909145,412.77608404)(503.50303473,412.7760839)
\curveto(503.16201177,412.77608404)(502.85509415,412.84807706)(502.58228094,412.99206319)
\curveto(502.31325192,413.13604915)(502.09158919,413.31792627)(501.91729209,413.53769507)
\lineto(501.91729209,410.59924006)
\lineto(500.89423231,410.59924006)
\moveto(501.82066977,415.89641621)
\curveto(501.82066808,415.1196464)(501.977916,414.54559676)(502.292414,414.17426559)
\curveto(502.60690767,413.80293189)(502.98771287,413.61726568)(503.43483075,413.61726638)
\curveto(503.88952022,413.61726568)(504.27790363,413.80861555)(504.59998215,414.19131659)
\curveto(504.92584084,414.57780417)(505.08877242,415.17458844)(505.08877738,415.98167119)
\curveto(505.08877242,416.75085673)(504.92962994,417.32680092)(504.61134948,417.70950547)
\curveto(504.29684916,418.09220043)(503.91983307,418.2835503)(503.48030007,418.28355567)
\curveto(503.04454947,418.2835503)(502.65806061,418.07893855)(502.32083233,417.66971981)
\curveto(501.98738876,417.26428066)(501.82066808,416.67318005)(501.82066977,415.89641621)
}
}
{
\newrgbcolor{curcolor}{0 0 0}
\pscustom[linestyle=none,fillstyle=solid,fillcolor=curcolor]
{
\newpath
\moveto(512.76172596,416.55572139)
\lineto(507.24288686,414.19700025)
\lineto(507.24288686,415.21437636)
\lineto(511.61362555,417.02746562)
\lineto(507.24288686,418.82350389)
\lineto(507.24288686,419.84087999)
\lineto(512.76172596,417.51057718)
\lineto(512.76172596,416.55572139)
}
}
{
\newrgbcolor{curcolor}{0 0 0}
\pscustom[linestyle=none,fillstyle=solid,fillcolor=curcolor]
{
\newpath
\moveto(519.55938862,416.55572139)
\lineto(514.04054952,414.19700025)
\lineto(514.04054952,415.21437636)
\lineto(518.41128821,417.02746562)
\lineto(514.04054952,418.82350389)
\lineto(514.04054952,419.84087999)
\lineto(519.55938862,417.51057718)
\lineto(519.55938862,416.55572139)
}
}
{
\newrgbcolor{curcolor}{0 0 0}
\pscustom[linestyle=none,fillstyle=solid,fillcolor=curcolor]
{
\newpath
\moveto(468.0312345,399.70364371)
\lineto(466.45685918,399.41946044)
\curveto(466.40380717,399.73395231)(466.28255576,399.97077147)(466.0931046,400.12991862)
\curveto(465.90743422,400.28905641)(465.66493141,400.36862765)(465.36559543,400.36863256)
\curveto(464.96773581,400.36862765)(464.64945087,400.23032526)(464.41073964,399.95372499)
\curveto(464.17581256,399.68090482)(464.05835026,399.22242294)(464.05835239,398.57827796)
\curveto(464.05835026,397.86213371)(464.17770711,397.356288)(464.41642331,397.0607393)
\curveto(464.65892364,396.76518739)(464.98289224,396.61741224)(465.38833009,396.6174134)
\curveto(465.69145515,396.61741224)(465.93964163,396.70266713)(466.13289026,396.87317834)
\curveto(466.32613049,397.04747582)(466.46253832,397.34492068)(466.54211417,397.76551381)
\lineto(468.11080582,397.49838154)
\curveto(467.94786806,396.77844926)(467.63526677,396.23471248)(467.17300102,395.86716957)
\curveto(466.71072479,395.49962583)(466.09120588,395.31585416)(465.31444244,395.31585403)
\curveto(464.43157725,395.31585416)(463.72680344,395.59435349)(463.20011891,396.15135284)
\curveto(462.67722095,396.70835079)(462.4157726,397.47943396)(462.41577309,398.46460466)
\curveto(462.4157726,399.46113665)(462.6791155,400.23600892)(463.20580258,400.7892238)
\curveto(463.7324871,401.34621712)(464.44483912,401.62471645)(465.34286077,401.62472262)
\curveto(466.07794401,401.62471645)(466.66146641,401.46557397)(467.09342971,401.14729472)
\curveto(467.52917179,400.83279319)(467.84177308,400.35157667)(468.0312345,399.70364371)
}
}
{
\newrgbcolor{curcolor}{0 0 0}
\pscustom[linestyle=none,fillstyle=solid,fillcolor=curcolor]
{
\newpath
\moveto(468.87810049,398.5555433)
\curveto(468.87810003,399.08601511)(469.0088242,399.59943903)(469.27027341,400.09581663)
\curveto(469.5317209,400.59218493)(469.90115878,400.97109558)(470.37858816,401.23254971)
\curveto(470.85980272,401.49399227)(471.39596128,401.62471645)(471.98706547,401.62472262)
\curveto(472.90023655,401.62471645)(473.64858508,401.32727159)(474.2321133,400.73238715)
\curveto(474.81562988,400.14128126)(475.10739108,399.39293273)(475.10739777,398.48733932)
\curveto(475.10739108,397.57416162)(474.81184077,396.81634033)(474.22074597,396.21387316)
\curveto(473.63342866,395.61519357)(472.89265834,395.31585416)(471.9984328,395.31585403)
\curveto(471.44521967,395.31585416)(470.91663931,395.44089468)(470.41269015,395.69097594)
\curveto(469.9125261,395.94105673)(469.5317209,396.30670551)(469.27027341,396.78792336)
\curveto(469.0088242,397.27292766)(468.87810003,397.86213371)(468.87810049,398.5555433)
\moveto(470.51499613,398.47028832)
\curveto(470.51499403,397.87160648)(470.65708552,397.4131246)(470.94127103,397.0948413)
\curveto(471.22545149,396.77655471)(471.57594384,396.61741224)(471.99274913,396.6174134)
\curveto(472.40954726,396.61741224)(472.75814506,396.77655471)(473.03854357,397.0948413)
\curveto(473.32272193,397.4131246)(473.46481342,397.87539559)(473.46481847,398.48165565)
\curveto(473.46481342,399.07275323)(473.32272193,399.52744601)(473.03854357,399.84573535)
\curveto(472.75814506,400.1640159)(472.40954726,400.32315837)(471.99274913,400.32316324)
\curveto(471.57594384,400.32315837)(471.22545149,400.1640159)(470.94127103,399.84573535)
\curveto(470.65708552,399.52744601)(470.51499403,399.06896413)(470.51499613,398.47028832)
}
}
{
\newrgbcolor{curcolor}{0 0 0}
\pscustom[linestyle=none,fillstyle=solid,fillcolor=curcolor]
{
\newpath
\moveto(476.24413047,401.48831465)
\lineto(477.7161998,401.48831465)
\lineto(477.7161998,400.66418317)
\curveto(478.24288342,401.30453695)(478.86998054,401.62471645)(479.59749305,401.62472262)
\curveto(479.98397784,401.62471645)(480.31931376,401.54514521)(480.60350183,401.38600867)
\curveto(480.88767974,401.22686027)(481.12070978,400.986252)(481.30259267,400.66418317)
\curveto(481.56782435,400.986252)(481.85390189,401.22686027)(482.16082615,401.38600867)
\curveto(482.46773714,401.54514521)(482.79549485,401.62471645)(483.14410026,401.62472262)
\curveto(483.5874181,401.62471645)(483.96253964,401.53377789)(484.26946601,401.35190668)
\curveto(484.57637489,401.17381277)(484.80561583,400.91046987)(484.95718952,400.56187719)
\curveto(485.06706418,400.30421284)(485.12200622,399.88741113)(485.12201582,399.3114708)
\lineto(485.12201582,395.452262)
\lineto(483.52490584,395.452262)
\lineto(483.52490584,398.90224689)
\curveto(483.52489784,399.50092227)(483.4699558,399.88741113)(483.36007954,400.06171463)
\curveto(483.21229656,400.28905641)(482.98495017,400.40272961)(482.6780397,400.40273456)
\curveto(482.45447526,400.40272961)(482.24417985,400.33452569)(482.04715284,400.1981226)
\curveto(481.85011278,400.06171002)(481.70802129,399.86088738)(481.62087793,399.59565407)
\curveto(481.53372239,399.33420158)(481.49014767,398.91929442)(481.49015363,398.35093135)
\lineto(481.49015363,395.452262)
\lineto(479.89304365,395.452262)
\lineto(479.89304365,398.76015526)
\curveto(479.89303929,399.34746345)(479.86462099,399.7263741)(479.80778867,399.89688834)
\curveto(479.75094779,400.06739368)(479.66190379,400.19432875)(479.5406564,400.27769392)
\curveto(479.42319008,400.36104944)(479.26215306,400.40272961)(479.05754484,400.40273456)
\curveto(478.81124939,400.40272961)(478.58958666,400.33642024)(478.39255599,400.20380627)
\curveto(478.19551958,400.07118279)(478.05342809,399.87983291)(477.96628108,399.62975606)
\curveto(477.8829183,399.37967086)(477.84123813,398.9647637)(477.84124044,398.38503334)
\lineto(477.84124044,395.452262)
\lineto(476.24413047,395.452262)
\lineto(476.24413047,401.48831465)
}
}
{
\newrgbcolor{curcolor}{0 0 0}
\pscustom[linestyle=none,fillstyle=solid,fillcolor=curcolor]
{
\newpath
\moveto(486.58840143,401.48831465)
\lineto(488.06047077,401.48831465)
\lineto(488.06047077,400.66418317)
\curveto(488.58715438,401.30453695)(489.2142515,401.62471645)(489.94176402,401.62472262)
\curveto(490.32824881,401.62471645)(490.66358473,401.54514521)(490.94777279,401.38600867)
\curveto(491.2319507,401.22686027)(491.46498075,400.986252)(491.64686364,400.66418317)
\curveto(491.91209531,400.986252)(492.19817285,401.22686027)(492.50509711,401.38600867)
\curveto(492.8120081,401.54514521)(493.13976581,401.62471645)(493.48837123,401.62472262)
\curveto(493.93168907,401.62471645)(494.30681061,401.53377789)(494.61373697,401.35190668)
\curveto(494.92064586,401.17381277)(495.1498868,400.91046987)(495.30146049,400.56187719)
\curveto(495.41133515,400.30421284)(495.46627719,399.88741113)(495.46628678,399.3114708)
\lineto(495.46628678,395.452262)
\lineto(493.86917681,395.452262)
\lineto(493.86917681,398.90224689)
\curveto(493.86916881,399.50092227)(493.81422677,399.88741113)(493.70435051,400.06171463)
\curveto(493.55656753,400.28905641)(493.32922114,400.40272961)(493.02231066,400.40273456)
\curveto(492.79874623,400.40272961)(492.58845082,400.33452569)(492.3914238,400.1981226)
\curveto(492.19438375,400.06171002)(492.05229225,399.86088738)(491.9651489,399.59565407)
\curveto(491.87799336,399.33420158)(491.83441863,398.91929442)(491.83442459,398.35093135)
\lineto(491.83442459,395.452262)
\lineto(490.23731462,395.452262)
\lineto(490.23731462,398.76015526)
\curveto(490.23731025,399.34746345)(490.20889195,399.7263741)(490.15205964,399.89688834)
\curveto(490.09521876,400.06739368)(490.00617476,400.19432875)(489.88492736,400.27769392)
\curveto(489.76746105,400.36104944)(489.60642402,400.40272961)(489.4018158,400.40273456)
\curveto(489.15552035,400.40272961)(488.93385762,400.33642024)(488.73682695,400.20380627)
\curveto(488.53979055,400.07118279)(488.39769906,399.87983291)(488.31055205,399.62975606)
\curveto(488.22718927,399.37967086)(488.1855091,398.9647637)(488.18551141,398.38503334)
\lineto(488.18551141,395.452262)
\lineto(486.58840143,395.452262)
\lineto(486.58840143,401.48831465)
}
}
{
\newrgbcolor{curcolor}{0 0 0}
\pscustom[linestyle=none,fillstyle=solid,fillcolor=curcolor]
{
\newpath
\moveto(501.02491326,395.452262)
\lineto(501.02491326,396.35596479)
\curveto(500.80514028,396.03388984)(500.51527363,395.78001971)(500.15531246,395.59435363)
\curveto(499.79913251,395.40868727)(499.42211641,395.31585416)(499.02426304,395.31585403)
\curveto(498.61882584,395.31585416)(498.25507162,395.40489816)(497.93299928,395.5829863)
\curveto(497.61092352,395.76107417)(497.37789347,396.0111552)(497.23390844,396.33323013)
\curveto(497.08992138,396.6553033)(497.01792835,397.10052331)(497.01792916,397.6688915)
\lineto(497.01792916,401.48831465)
\lineto(498.61503913,401.48831465)
\lineto(498.61503913,398.71468593)
\curveto(498.61503673,397.86592282)(498.64345503,397.34492068)(498.70029411,397.15167795)
\curveto(498.76091733,396.96222093)(498.86890687,396.81065667)(499.02426304,396.69698472)
\curveto(499.1796136,396.58709938)(499.37664714,396.53215734)(499.61536424,396.53215842)
\curveto(499.88817651,396.53215734)(500.13257388,396.60604492)(500.34855708,396.75382137)
\curveto(500.56453202,396.90538433)(500.71230717,397.09105055)(500.79188298,397.31082058)
\curveto(500.87144964,397.534376)(500.91123526,398.07811278)(500.91123995,398.94203255)
\lineto(500.91123995,401.48831465)
\lineto(502.50834993,401.48831465)
\lineto(502.50834993,395.452262)
\lineto(501.02491326,395.452262)
}
}
{
\newrgbcolor{curcolor}{0 0 0}
\pscustom[linestyle=none,fillstyle=solid,fillcolor=curcolor]
{
\newpath
\moveto(509.65840017,395.452262)
\lineto(508.0612902,395.452262)
\lineto(508.0612902,398.53280864)
\curveto(508.06128547,399.18453187)(508.02718351,399.60512269)(507.95898422,399.79458236)
\curveto(507.89077567,399.98782245)(507.77899703,400.13749215)(507.62364796,400.24359193)
\curveto(507.47207941,400.34968212)(507.28830775,400.40272961)(507.07233242,400.40273456)
\curveto(506.7957239,400.40272961)(506.54753743,400.32694748)(506.32777225,400.17538794)
\curveto(506.10800108,400.02381896)(505.95643682,399.82299632)(505.87307902,399.57291941)
\curveto(505.79350524,399.32283426)(505.75371962,398.86056327)(505.75372204,398.18610505)
\lineto(505.75372204,395.452262)
\lineto(504.15661207,395.452262)
\lineto(504.15661207,401.48831465)
\lineto(505.64004874,401.48831465)
\lineto(505.64004874,400.60166285)
\curveto(506.16673223,401.28369686)(506.82982586,401.62471645)(507.62933162,401.62472262)
\curveto(507.98171423,401.62471645)(508.30378828,401.56030164)(508.59555474,401.43147799)
\curveto(508.88731068,401.3064315)(509.10707885,401.14539448)(509.25485993,400.94836644)
\curveto(509.40641827,400.7513274)(509.51061869,400.52777012)(509.56746153,400.27769392)
\curveto(509.62808099,400.02760807)(509.65839385,399.6695375)(509.65840017,399.20348116)
\lineto(509.65840017,395.452262)
}
}
{
\newrgbcolor{curcolor}{0 0 0}
\pscustom[linestyle=none,fillstyle=solid,fillcolor=curcolor]
{
\newpath
\moveto(511.28392765,402.30676247)
\lineto(511.28392765,403.78451547)
\lineto(512.88103762,403.78451547)
\lineto(512.88103762,402.30676247)
\lineto(511.28392765,402.30676247)
\moveto(511.28392765,395.452262)
\lineto(511.28392765,401.48831465)
\lineto(512.88103762,401.48831465)
\lineto(512.88103762,395.452262)
\lineto(511.28392765,395.452262)
}
}
{
\newrgbcolor{curcolor}{0 0 0}
\pscustom[linestyle=none,fillstyle=solid,fillcolor=curcolor]
{
\newpath
\moveto(517.28019541,401.48831465)
\lineto(517.28019541,400.2151736)
\lineto(516.18893166,400.2151736)
\lineto(516.18893166,397.78256481)
\curveto(516.18892914,397.28997864)(516.19840191,397.00200654)(516.21734998,396.91864767)
\curveto(516.24008208,396.83907497)(516.28744591,396.7727656)(516.35944162,396.71971938)
\curveto(516.43522106,396.66667062)(516.52615962,396.64014688)(516.63225756,396.64014806)
\curveto(516.78002975,396.64014688)(516.99411427,396.69129981)(517.27451175,396.79360703)
\lineto(517.41091972,395.55456797)
\curveto(517.03958355,395.3954254)(516.61899273,395.31585416)(516.149146,395.31585403)
\curveto(515.86117143,395.31585416)(515.60161764,395.36321799)(515.37048384,395.45794566)
\curveto(515.13934665,395.55646242)(514.96883686,395.68150294)(514.85895395,395.83306758)
\curveto(514.75285779,395.98842056)(514.67897021,396.19682142)(514.637291,396.45827077)
\curveto(514.60318808,396.64393598)(514.5861371,397.01905752)(514.58613801,397.58363652)
\lineto(514.58613801,400.2151736)
\lineto(513.85294518,400.2151736)
\lineto(513.85294518,401.48831465)
\lineto(514.58613801,401.48831465)
\lineto(514.58613801,402.68756805)
\lineto(516.18893166,403.61968917)
\lineto(516.18893166,401.48831465)
\lineto(517.28019541,401.48831465)
}
}
{
\newrgbcolor{curcolor}{0 0 0}
\pscustom[linestyle=none,fillstyle=solid,fillcolor=curcolor]
{
\newpath
\moveto(518.3998759,402.30676247)
\lineto(518.3998759,403.78451547)
\lineto(519.99698587,403.78451547)
\lineto(519.99698587,402.30676247)
\lineto(518.3998759,402.30676247)
\moveto(518.3998759,395.452262)
\lineto(518.3998759,401.48831465)
\lineto(519.99698587,401.48831465)
\lineto(519.99698587,395.452262)
\lineto(518.3998759,395.452262)
}
}
{
\newrgbcolor{curcolor}{0 0 0}
\pscustom[linestyle=none,fillstyle=solid,fillcolor=curcolor]
{
\newpath
\moveto(525.12365283,397.3733409)
\lineto(526.71507914,397.10620863)
\curveto(526.51046147,396.52268457)(526.18649287,396.07746456)(525.74317236,395.77054726)
\curveto(525.30363106,395.46741842)(524.75231607,395.31585416)(524.08922573,395.31585403)
\curveto(523.03963994,395.31585416)(522.26287311,395.6587683)(521.75892292,396.34459746)
\curveto(521.36106577,396.89401701)(521.16213768,397.58742349)(521.16213805,398.424819)
\curveto(521.16213768,399.42514014)(521.42358603,400.20759062)(521.94648387,400.77217281)
\curveto(522.46937941,401.34053346)(523.13057849,401.62471645)(523.9300831,401.62472262)
\curveto(524.8280982,401.62471645)(525.53666111,401.32727159)(526.05577396,400.73238715)
\curveto(526.57487628,400.14128126)(526.82306276,399.23379026)(526.80033412,398.00991142)
\lineto(522.79903368,398.00991142)
\curveto(522.810399,397.53627056)(522.93922862,397.16683268)(523.18552293,396.90159667)
\curveto(523.43181246,396.64014688)(523.73873008,396.5094227)(524.10627673,396.50942376)
\curveto(524.35635444,396.5094227)(524.56664985,396.57762662)(524.73716359,396.71403571)
\curveto(524.90766943,396.85044228)(525.03649905,397.07021046)(525.12365283,397.3733409)
\moveto(525.21459148,398.98750187)
\curveto(525.20321974,399.44976933)(525.08386288,399.80026168)(524.85652056,400.03897997)
\curveto(524.62917011,400.2814782)(524.35256533,400.40272961)(524.02670541,400.40273456)
\curveto(523.67810438,400.40272961)(523.39013229,400.27579454)(523.16278827,400.02192898)
\curveto(522.93543951,399.76805427)(522.82366087,399.42324558)(522.82745201,398.98750187)
\lineto(525.21459148,398.98750187)
}
}
{
\newrgbcolor{curcolor}{0 0 0}
\pscustom[linestyle=none,fillstyle=solid,fillcolor=curcolor]
{
\newpath
\moveto(527.54489281,397.17441261)
\lineto(529.14768645,397.41881022)
\curveto(529.21588849,397.10810153)(529.35419088,396.87128237)(529.56259403,396.70835205)
\curveto(529.77099259,396.54920832)(530.06275379,396.46963708)(530.4378785,396.4696381)
\curveto(530.85088794,396.46963708)(531.16159467,396.54541921)(531.36999962,396.69698472)
\curveto(531.51019246,396.80307845)(531.58029093,396.94516995)(531.58029524,397.12325962)
\curveto(531.58029093,397.24450936)(531.54239987,397.34492068)(531.46662193,397.42449389)
\curveto(531.3870465,397.50027405)(531.2089585,397.57037252)(530.93235739,397.63478951)
\curveto(529.64405753,397.91897031)(528.82750508,398.1785241)(528.4826976,398.41345167)
\curveto(528.00526897,398.73931186)(527.76655527,399.19211009)(527.76655576,399.7718477)
\curveto(527.76655527,400.29474007)(527.97306157,400.73427642)(528.38607529,401.09045807)
\curveto(528.79908678,401.44662844)(529.43944578,401.62471645)(530.30715419,401.62472262)
\curveto(531.13317637,401.62471645)(531.74701162,401.49020317)(532.14866178,401.22118237)
\curveto(532.55030219,400.95215005)(532.82690697,400.55429387)(532.97847693,400.02761264)
\lineto(531.4723056,399.74911304)
\curveto(531.40788659,399.98403334)(531.28474063,400.1640159)(531.10286735,400.28906125)
\curveto(530.92477551,400.41409693)(530.66901083,400.47661718)(530.33557252,400.47662221)
\curveto(529.91497864,400.47661718)(529.61374467,400.41788603)(529.43186972,400.30042858)
\curveto(529.31061616,400.21706339)(529.24999045,400.10907385)(529.24999243,399.97645965)
\curveto(529.24999045,399.86278193)(529.30303794,399.76615972)(529.40913506,399.68659272)
\curveto(529.55311897,399.5804935)(530.04949192,399.4308238)(530.8982554,399.23758315)
\curveto(531.75080073,399.04433493)(532.34569044,398.80751578)(532.68292633,398.52712498)
\curveto(533.01636229,398.24293892)(533.18308297,397.84697729)(533.18308889,397.33923891)
\curveto(533.18308297,396.78602747)(532.95194748,396.31049461)(532.48968171,395.91263889)
\curveto(532.0274055,395.51478225)(531.34347178,395.31585416)(530.4378785,395.31585403)
\curveto(529.61563923,395.31585416)(528.96391291,395.48257485)(528.4826976,395.81601658)
\curveto(528.00526897,396.14945759)(527.69266769,396.60225581)(527.54489281,397.17441261)
}
}
{
\newrgbcolor{curcolor}{0 1 0.25098041}
\pscustom[linewidth=2.32802935,linecolor=curcolor]
{
\newpath
\moveto(581.82618442,295.92182848)
\lineto(656.32312136,295.92182848)
\lineto(656.32312136,342.4824199)
\lineto(606.27049025,342.4824199)
\lineto(606.27049025,351.79453818)
\lineto(581.82618442,351.79453818)
\lineto(581.82618442,295.92182848)
\closepath
}
}
{
\newrgbcolor{curcolor}{0 0 0}
\pscustom[linestyle=none,fillstyle=solid,fillcolor=curcolor]
{
\newpath
\moveto(595.26689951,323.43453329)
\lineto(595.26689951,324.38938908)
\lineto(600.78573861,326.71969189)
\lineto(600.78573861,325.70231578)
\lineto(596.40931625,323.90627752)
\lineto(600.78573861,322.09318826)
\lineto(600.78573861,321.07581215)
\lineto(595.26689951,323.43453329)
}
}
{
\newrgbcolor{curcolor}{0 0 0}
\pscustom[linestyle=none,fillstyle=solid,fillcolor=curcolor]
{
\newpath
\moveto(602.0645635,323.43453329)
\lineto(602.0645635,324.38938908)
\lineto(607.5834026,326.71969189)
\lineto(607.5834026,325.70231578)
\lineto(603.20698025,323.90627752)
\lineto(607.5834026,322.09318826)
\lineto(607.5834026,321.07581215)
\lineto(602.0645635,323.43453329)
}
}
{
\newrgbcolor{curcolor}{0 0 0}
\pscustom[linestyle=none,fillstyle=solid,fillcolor=curcolor]
{
\newpath
\moveto(612.93173192,320.53586394)
\curveto(612.55281657,320.21378914)(612.18716779,319.98644275)(611.8347845,319.85382409)
\curveto(611.48618309,319.7212053)(611.11106155,319.65489594)(610.70941875,319.6548958)
\curveto(610.04632263,319.65489594)(609.53668781,319.81593296)(609.18051276,320.13800736)
\curveto(608.82433579,320.46387017)(608.64624779,320.87877733)(608.64624821,321.38273008)
\curveto(608.64624779,321.6782788)(608.71255715,321.94730535)(608.8451765,322.18981057)
\curveto(608.98158371,322.43610009)(609.15777716,322.63313363)(609.37375738,322.78091177)
\curveto(609.59352441,322.92868393)(609.83981633,323.04046257)(610.11263388,323.11624803)
\curveto(610.31345464,323.16929219)(610.61658316,323.22044513)(611.02202035,323.26970699)
\curveto(611.84804276,323.36822028)(612.45619435,323.48568258)(612.84647694,323.62209425)
\curveto(612.85026143,323.76228736)(612.85215598,323.85133136)(612.85216061,323.88922652)
\curveto(612.85215598,324.30602414)(612.75553376,324.59967989)(612.56229367,324.77019466)
\curveto(612.30084099,325.00132517)(611.91245757,325.11689292)(611.39714226,325.11689825)
\curveto(610.91592257,325.11689292)(610.55974656,325.03163803)(610.32861317,324.86113331)
\curveto(610.10126468,324.69440755)(609.93264944,324.39696269)(609.82276695,323.96879784)
\lineto(608.82244184,324.10520581)
\curveto(608.9133798,324.53337053)(609.0630495,324.87817921)(609.2714514,325.13963291)
\curveto(609.47985121,325.40486501)(609.78108518,325.60758221)(610.1751542,325.74778511)
\curveto(610.56921933,325.8917652)(611.02580666,325.96375822)(611.54491756,325.96376439)
\curveto(612.06023273,325.96375822)(612.47892899,325.90313252)(612.80100762,325.7818871)
\curveto(613.12307709,325.6606297)(613.35989625,325.50717089)(613.51146579,325.3215102)
\curveto(613.66302476,325.13962756)(613.76911975,324.90849207)(613.82975105,324.62810302)
\curveto(613.86384741,324.45379929)(613.88089839,324.13930345)(613.88090404,323.68461457)
\lineto(613.88090404,322.32053487)
\curveto(613.88089839,321.36946662)(613.90173847,320.76699869)(613.94342436,320.51312928)
\curveto(613.98888792,320.26304753)(614.07603737,320.02243926)(614.20487297,319.79130377)
\lineto(613.13634388,319.79130377)
\curveto(613.03024398,320.00349373)(612.96204007,320.25168021)(612.93173192,320.53586394)
\moveto(612.84647694,322.82069743)
\curveto(612.47513988,322.66913014)(611.91814123,322.54030052)(611.17547931,322.43420818)
\curveto(610.75488554,322.37357983)(610.45744069,322.30537592)(610.28314385,322.22959623)
\curveto(610.10884289,322.15381166)(609.97432961,322.04203302)(609.8796036,321.89425997)
\curveto(609.78487429,321.75027182)(609.73751046,321.58923479)(609.73751197,321.41114841)
\curveto(609.73751046,321.13833112)(609.83981633,320.91098473)(610.0444299,320.72910856)
\curveto(610.25282894,320.54723051)(610.55595745,320.45629196)(610.95381636,320.45629262)
\curveto(611.34788071,320.45629196)(611.69837306,320.54154685)(612.00529446,320.71205756)
\curveto(612.31220831,320.88635554)(612.53766014,321.1231747)(612.68165064,321.42251574)
\curveto(612.79153028,321.6536496)(612.84647232,321.99466919)(612.84647694,322.44557551)
\lineto(612.84647694,322.82069743)
}
}
{
\newrgbcolor{curcolor}{0 0 0}
\pscustom[linestyle=none,fillstyle=solid,fillcolor=curcolor]
{
\newpath
\moveto(615.47232976,317.47805195)
\lineto(615.47232976,325.82735642)
\lineto(616.40445089,325.82735642)
\lineto(616.40445089,325.0430106)
\curveto(616.62421736,325.34992297)(616.87240384,325.57916391)(617.14901105,325.73073411)
\curveto(617.42561338,325.88608154)(617.76094931,325.96375822)(618.15501983,325.96376439)
\curveto(618.67033486,325.96375822)(619.12502764,325.83113949)(619.51909952,325.56590781)
\curveto(619.91316178,325.30066459)(620.21060664,324.92554305)(620.41143499,324.44054207)
\curveto(620.61225193,323.95932089)(620.71266325,323.43074054)(620.71266926,322.85479942)
\curveto(620.71266325,322.237172)(620.60088461,321.68017335)(620.377333,321.18380179)
\curveto(620.15755915,320.69121656)(619.8354851,320.31230591)(619.41110988,320.04706871)
\curveto(618.99051436,319.78562011)(618.5471889,319.65489594)(618.08113218,319.6548958)
\curveto(617.74010922,319.65489594)(617.43319159,319.72688896)(617.16037838,319.87087509)
\curveto(616.89134937,320.01486105)(616.66968664,320.19673816)(616.49538953,320.41650696)
\lineto(616.49538953,317.47805195)
\lineto(615.47232976,317.47805195)
\moveto(616.39876722,322.7752281)
\curveto(616.39876553,321.99845829)(616.55601345,321.42440866)(616.87051145,321.05307749)
\curveto(617.18500512,320.68174379)(617.56581032,320.49607757)(618.01292819,320.49607828)
\curveto(618.46761766,320.49607757)(618.85600108,320.68742745)(619.1780796,321.07012848)
\curveto(619.50393828,321.45661607)(619.66686986,322.05340034)(619.66687482,322.86048308)
\curveto(619.66686986,323.62966863)(619.50772739,324.20561281)(619.18944693,324.58831737)
\curveto(618.87494661,324.97101232)(618.49793051,325.1623622)(618.05839752,325.16236757)
\curveto(617.62264692,325.1623622)(617.23615806,324.95775045)(616.89892978,324.54853171)
\curveto(616.56548621,324.14309256)(616.39876553,323.55199195)(616.39876722,322.7752281)
}
}
{
\newrgbcolor{curcolor}{0 0 0}
\pscustom[linestyle=none,fillstyle=solid,fillcolor=curcolor]
{
\newpath
\moveto(621.95170861,317.47805195)
\lineto(621.95170861,325.82735642)
\lineto(622.88382974,325.82735642)
\lineto(622.88382974,325.0430106)
\curveto(623.10359621,325.34992297)(623.35178269,325.57916391)(623.6283899,325.73073411)
\curveto(623.90499223,325.88608154)(624.24032815,325.96375822)(624.63439868,325.96376439)
\curveto(625.14971371,325.96375822)(625.60440649,325.83113949)(625.99847837,325.56590781)
\curveto(626.39254063,325.30066459)(626.68998549,324.92554305)(626.89081384,324.44054207)
\curveto(627.09163078,323.95932089)(627.1920421,323.43074054)(627.19204811,322.85479942)
\curveto(627.1920421,322.237172)(627.08026346,321.68017335)(626.85671185,321.18380179)
\curveto(626.636938,320.69121656)(626.31486395,320.31230591)(625.89048873,320.04706871)
\curveto(625.46989321,319.78562011)(625.02656775,319.65489594)(624.56051103,319.6548958)
\curveto(624.21948807,319.65489594)(623.91257044,319.72688896)(623.63975723,319.87087509)
\curveto(623.37072822,320.01486105)(623.14906549,320.19673816)(622.97476838,320.41650696)
\lineto(622.97476838,317.47805195)
\lineto(621.95170861,317.47805195)
\moveto(622.87814607,322.7752281)
\curveto(622.87814438,321.99845829)(623.03539229,321.42440866)(623.3498903,321.05307749)
\curveto(623.66438397,320.68174379)(624.04518917,320.49607757)(624.49230704,320.49607828)
\curveto(624.94699651,320.49607757)(625.33537993,320.68742745)(625.65745845,321.07012848)
\curveto(625.98331713,321.45661607)(626.14624871,322.05340034)(626.14625367,322.86048308)
\curveto(626.14624871,323.62966863)(625.98710624,324.20561281)(625.66882578,324.58831737)
\curveto(625.35432546,324.97101232)(624.97730936,325.1623622)(624.53777637,325.16236757)
\curveto(624.10202577,325.1623622)(623.71553691,324.95775045)(623.37830862,324.54853171)
\curveto(623.04486506,324.14309256)(622.87814438,323.55199195)(622.87814607,322.7752281)
}
}
{
\newrgbcolor{curcolor}{0 0 0}
\pscustom[linestyle=none,fillstyle=solid,fillcolor=curcolor]
{
\newpath
\moveto(633.81920226,323.43453329)
\lineto(628.30036315,321.07581215)
\lineto(628.30036315,322.09318826)
\lineto(632.67110185,323.90627752)
\lineto(628.30036315,325.70231578)
\lineto(628.30036315,326.71969189)
\lineto(633.81920226,324.38938908)
\lineto(633.81920226,323.43453329)
}
}
{
\newrgbcolor{curcolor}{0 0 0}
\pscustom[linestyle=none,fillstyle=solid,fillcolor=curcolor]
{
\newpath
\moveto(640.61686492,323.43453329)
\lineto(635.09802582,321.07581215)
\lineto(635.09802582,322.09318826)
\lineto(639.46876451,323.90627752)
\lineto(635.09802582,325.70231578)
\lineto(635.09802582,326.71969189)
\lineto(640.61686492,324.38938908)
\lineto(640.61686492,323.43453329)
}
}
{
\newrgbcolor{curcolor}{0 0 0}
\pscustom[linestyle=none,fillstyle=solid,fillcolor=curcolor]
{
\newpath
\moveto(594.90314492,304.05322451)
\lineto(596.50593857,304.29762212)
\curveto(596.57414061,303.98691342)(596.71244299,303.75009427)(596.92084614,303.58716395)
\curveto(597.12924471,303.42802022)(597.4210059,303.34844898)(597.79613061,303.34845)
\curveto(598.20914005,303.34844898)(598.51984678,303.42423111)(598.72825174,303.57579661)
\curveto(598.86844458,303.68189035)(598.93854305,303.82398184)(598.93854736,304.00207152)
\curveto(598.93854305,304.12332126)(598.90065198,304.22373258)(598.82487405,304.30330579)
\curveto(598.74529862,304.37908594)(598.56721061,304.44918441)(598.2906095,304.51360141)
\curveto(597.00230964,304.79778221)(596.18575719,305.057336)(595.84094971,305.29226356)
\curveto(595.36352109,305.61812376)(595.12480738,306.07092198)(595.12480787,306.6506596)
\curveto(595.12480738,307.17355197)(595.33131368,307.61308832)(595.7443274,307.96926997)
\curveto(596.15733889,308.32544034)(596.79769789,308.50352834)(597.66540631,308.50353452)
\curveto(598.49142848,308.50352834)(599.10526373,308.36901506)(599.5069139,308.09999427)
\curveto(599.90855431,307.83096194)(600.18515908,307.43310576)(600.33672904,306.90642454)
\lineto(598.83055771,306.62792493)
\curveto(598.7661387,306.86284524)(598.64299274,307.0428278)(598.46111946,307.16787315)
\curveto(598.28302763,307.29290882)(598.02726294,307.35542908)(597.69382463,307.3554341)
\curveto(597.27323075,307.35542908)(596.97199679,307.29669793)(596.79012184,307.17924048)
\curveto(596.66886827,307.09587529)(596.60824257,306.98788575)(596.60824454,306.85527155)
\curveto(596.60824257,306.74159383)(596.66129006,306.64497162)(596.76738717,306.56540461)
\curveto(596.91137108,306.4593054)(597.40774403,306.30963569)(598.25650751,306.11639505)
\curveto(599.10905284,305.92314683)(599.70394256,305.68632768)(600.04117844,305.40593687)
\curveto(600.3746144,305.12175081)(600.54133509,304.72578919)(600.541341,304.2180508)
\curveto(600.54133509,303.66483937)(600.31019959,303.18930651)(599.84793382,302.79145079)
\curveto(599.38565761,302.39359415)(598.70172389,302.19466606)(597.79613061,302.19466592)
\curveto(596.97389134,302.19466606)(596.32216503,302.36138674)(595.84094971,302.69482848)
\curveto(595.36352109,303.02826948)(595.0509198,303.48106771)(594.90314492,304.05322451)
}
}
{
\newrgbcolor{curcolor}{0 0 0}
\pscustom[linestyle=none,fillstyle=solid,fillcolor=curcolor]
{
\newpath
\moveto(605.91808832,302.33107389)
\lineto(605.91808832,303.23477669)
\curveto(605.69831533,302.91270174)(605.40844869,302.6588316)(605.04848751,302.47316553)
\curveto(604.69230756,302.28749917)(604.31529147,302.19466606)(603.9174381,302.19466592)
\curveto(603.5120009,302.19466606)(603.14824667,302.28371006)(602.82617434,302.4617982)
\curveto(602.50409857,302.63988607)(602.27106852,302.8899671)(602.1270835,303.21204203)
\curveto(601.98309643,303.5341152)(601.91110341,303.97933521)(601.91110421,304.5477034)
\lineto(601.91110421,308.36712655)
\lineto(603.50821419,308.36712655)
\lineto(603.50821419,305.59349783)
\curveto(603.50821179,304.74473472)(603.53663009,304.22373258)(603.59346917,304.03048985)
\curveto(603.65409239,303.84103282)(603.76208192,303.68946856)(603.9174381,303.57579661)
\curveto(604.07278865,303.46591128)(604.26982219,303.41096924)(604.5085393,303.41097032)
\curveto(604.78135156,303.41096924)(605.02574893,303.48485681)(605.24173213,303.63263327)
\curveto(605.45770707,303.78419623)(605.60548222,303.96986244)(605.68505803,304.18963248)
\curveto(605.7646247,304.4131879)(605.80441031,304.95692468)(605.80441501,305.82084445)
\lineto(605.80441501,308.36712655)
\lineto(607.40152499,308.36712655)
\lineto(607.40152499,302.33107389)
\lineto(605.91808832,302.33107389)
}
}
{
\newrgbcolor{curcolor}{0 0 0}
\pscustom[linestyle=none,fillstyle=solid,fillcolor=curcolor]
{
\newpath
\moveto(608.99295091,302.33107389)
\lineto(608.99295091,310.66332737)
\lineto(610.59006089,310.66332737)
\lineto(610.59006089,307.66235204)
\curveto(611.08264237,308.22313446)(611.66616476,308.50352834)(612.34062983,308.50353452)
\curveto(613.07571237,308.50352834)(613.68386396,308.23639634)(614.16508642,307.70213769)
\curveto(614.64629701,307.17165742)(614.88690527,306.40815246)(614.88691193,305.41162054)
\curveto(614.88690527,304.3809805)(614.64061335,303.58716269)(614.14803543,303.03016474)
\curveto(613.65923477,302.47316539)(613.06434505,302.19466606)(612.36336449,302.19466592)
\curveto(612.01855167,302.19466606)(611.67753208,302.27992095)(611.34030472,302.45043087)
\curveto(611.00686024,302.62472964)(610.71888814,302.88049433)(610.47638758,303.21772569)
\lineto(610.47638758,302.33107389)
\lineto(608.99295091,302.33107389)
\moveto(610.57869356,305.47982452)
\curveto(610.5786912,304.85461881)(610.67720797,304.39234782)(610.87424416,304.09301017)
\curveto(611.15084628,303.66862848)(611.51838961,303.45643852)(611.97687525,303.45643964)
\curveto(612.3292584,303.45643852)(612.62859781,303.60610822)(612.87489438,303.90544921)
\curveto(613.12497076,304.20857615)(613.25001127,304.68410901)(613.25001629,305.33204922)
\curveto(613.25001127,306.0216636)(613.12497076,306.51803655)(612.87489438,306.82116956)
\curveto(612.6248087,307.12808269)(612.3046292,307.2815415)(611.91435493,307.28154645)
\curveto(611.53165148,307.2815415)(611.21336654,307.1318718)(610.95949914,306.83253689)
\curveto(610.70562627,306.53698208)(610.5786912,306.08607841)(610.57869356,305.47982452)
}
}
{
\newrgbcolor{curcolor}{0 0 0}
\pscustom[linestyle=none,fillstyle=solid,fillcolor=curcolor]
{
\newpath
\moveto(616.14300293,309.18557436)
\lineto(616.14300293,310.66332737)
\lineto(617.74011291,310.66332737)
\lineto(617.74011291,309.18557436)
\lineto(616.14300293,309.18557436)
\moveto(617.74011291,308.36712655)
\lineto(617.74011291,302.51863485)
\curveto(617.74011051,301.74944605)(617.68895757,301.20760382)(617.58665394,300.89310655)
\curveto(617.48813493,300.57482304)(617.2948905,300.32663657)(617.00692007,300.14854638)
\curveto(616.72273542,299.97046056)(616.3589812,299.88141656)(615.91565631,299.88141411)
\curveto(615.75651327,299.88141656)(615.58410892,299.89657298)(615.39844276,299.92688343)
\curveto(615.2165656,299.95340958)(615.01953206,299.99508975)(614.80734156,300.05192407)
\lineto(615.08584117,301.41600376)
\curveto(615.16162355,301.40084825)(615.23361657,301.38948093)(615.30182045,301.38190177)
\curveto(615.3662353,301.3705354)(615.426861,301.36485174)(615.48369774,301.36485077)
\curveto(615.64662918,301.36485174)(615.77924791,301.40084825)(615.88155432,301.47284042)
\curveto(615.98764876,301.54104519)(616.05774723,301.62440553)(616.09184994,301.72292169)
\curveto(616.12595115,301.82143907)(616.14300213,302.11698938)(616.14300293,302.6095735)
\lineto(616.14300293,308.36712655)
\lineto(617.74011291,308.36712655)
}
}
{
\newrgbcolor{curcolor}{0 0 0}
\pscustom[linestyle=none,fillstyle=solid,fillcolor=curcolor]
{
\newpath
\moveto(622.90088008,304.2521528)
\lineto(624.49230639,303.98502052)
\curveto(624.28768872,303.40149647)(623.96372012,302.95627646)(623.52039961,302.64935915)
\curveto(623.08085831,302.34623032)(622.52954332,302.19466606)(621.86645298,302.19466592)
\curveto(620.81686719,302.19466606)(620.04010036,302.5375802)(619.53615017,303.22340936)
\curveto(619.13829302,303.77282891)(618.93936493,304.46623539)(618.9393653,305.3036309)
\curveto(618.93936493,306.30395203)(619.20081328,307.08640252)(619.72371112,307.65098471)
\curveto(620.24660666,308.21934536)(620.90780574,308.50352834)(621.70731035,308.50353452)
\curveto(622.60532545,308.50352834)(623.31388836,308.20608348)(623.83300121,307.61119905)
\curveto(624.35210353,307.02009316)(624.60029001,306.11260216)(624.57756137,304.88872332)
\lineto(620.57626093,304.88872332)
\curveto(620.58762625,304.41508245)(620.71645587,304.04564457)(620.96275018,303.78040857)
\curveto(621.20903971,303.51895877)(621.51595733,303.3882346)(621.88350398,303.38823566)
\curveto(622.13358169,303.3882346)(622.3438771,303.45643852)(622.51439084,303.59284761)
\curveto(622.68489668,303.72925418)(622.8137263,303.94902236)(622.90088008,304.2521528)
\moveto(622.99181873,305.86631377)
\curveto(622.98044699,306.32858123)(622.86109013,306.67907357)(622.63374781,306.91779187)
\curveto(622.40639736,307.1602901)(622.12979258,307.2815415)(621.80393266,307.28154645)
\curveto(621.45533163,307.2815415)(621.16735954,307.15460644)(620.94001552,306.90074087)
\curveto(620.71266676,306.64686617)(620.60088812,306.30205748)(620.60467926,305.86631377)
\lineto(622.99181873,305.86631377)
}
}
{
\newrgbcolor{curcolor}{0 0 0}
\pscustom[linestyle=none,fillstyle=solid,fillcolor=curcolor]
{
\newpath
\moveto(631.14787887,306.58245561)
\lineto(629.57350355,306.29827234)
\curveto(629.52045154,306.61276421)(629.39920013,306.84958337)(629.20974897,307.00873052)
\curveto(629.02407859,307.16786831)(628.78157578,307.24743955)(628.4822398,307.24744446)
\curveto(628.08438019,307.24743955)(627.76609524,307.10913716)(627.52738401,306.83253689)
\curveto(627.29245693,306.55971672)(627.17499463,306.10123484)(627.17499676,305.45708986)
\curveto(627.17499463,304.74094561)(627.29435149,304.2350999)(627.53306768,303.9395512)
\curveto(627.77556801,303.64399929)(628.09953661,303.49622413)(628.50497446,303.4962253)
\curveto(628.80809952,303.49622413)(629.056286,303.58147903)(629.24953463,303.75199024)
\curveto(629.44277486,303.92628772)(629.57918269,304.22373258)(629.65875854,304.64432571)
\lineto(631.22745019,304.37719344)
\curveto(631.06451243,303.65726116)(630.75191114,303.11352438)(630.2896454,302.74598147)
\curveto(629.82736916,302.37843772)(629.20785026,302.19466606)(628.43108681,302.19466592)
\curveto(627.54822162,302.19466606)(626.84344781,302.47316539)(626.31676328,303.03016474)
\curveto(625.79386532,303.58716269)(625.53241697,304.35824586)(625.53241746,305.34341655)
\curveto(625.53241697,306.33994854)(625.79575987,307.11482082)(626.32244695,307.6680357)
\curveto(626.84913147,308.22502902)(627.56148349,308.50352834)(628.45950514,308.50353452)
\curveto(629.19458838,308.50352834)(629.77811078,308.34438587)(630.21007408,308.02610662)
\curveto(630.64581616,307.71160509)(630.95841745,307.23038857)(631.14787887,306.58245561)
}
}
{
\newrgbcolor{curcolor}{0 0 0}
\pscustom[linestyle=none,fillstyle=solid,fillcolor=curcolor]
{
\newpath
\moveto(635.13212683,308.36712655)
\lineto(635.13212683,307.0939855)
\lineto(634.04086308,307.0939855)
\lineto(634.04086308,304.66137671)
\curveto(634.04086056,304.16879053)(634.05033333,303.88081844)(634.0692814,303.79745957)
\curveto(634.0920135,303.71788686)(634.13937733,303.6515775)(634.21137304,303.59853128)
\curveto(634.28715248,303.54548252)(634.37809104,303.51895877)(634.48418898,303.51895996)
\curveto(634.63196117,303.51895877)(634.84604569,303.57011171)(635.12644317,303.67241893)
\lineto(635.26285114,302.43337987)
\curveto(634.89151497,302.2742373)(634.47092415,302.19466606)(634.00107742,302.19466592)
\curveto(633.71310285,302.19466606)(633.45354906,302.24202989)(633.22241526,302.33675756)
\curveto(632.99127807,302.43527432)(632.82076828,302.56031483)(632.71088537,302.71187947)
\curveto(632.60478921,302.86723246)(632.53090163,303.07563332)(632.48922242,303.33708267)
\curveto(632.4551195,303.52274788)(632.43806852,303.89786942)(632.43806943,304.46244842)
\lineto(632.43806943,307.0939855)
\lineto(631.7048766,307.0939855)
\lineto(631.7048766,308.36712655)
\lineto(632.43806943,308.36712655)
\lineto(632.43806943,309.56637994)
\lineto(634.04086308,310.49850107)
\lineto(634.04086308,308.36712655)
\lineto(635.13212683,308.36712655)
}
}
{
\newrgbcolor{curcolor}{0 0 0}
\pscustom[linestyle=none,fillstyle=solid,fillcolor=curcolor]
{
\newpath
\moveto(635.689128,304.05322451)
\lineto(637.29192164,304.29762212)
\curveto(637.36012368,303.98691342)(637.49842607,303.75009427)(637.70682921,303.58716395)
\curveto(637.91522778,303.42802022)(638.20698898,303.34844898)(638.58211368,303.34845)
\curveto(638.99512312,303.34844898)(639.30582985,303.42423111)(639.51423481,303.57579661)
\curveto(639.65442765,303.68189035)(639.72452612,303.82398184)(639.72453043,304.00207152)
\curveto(639.72452612,304.12332126)(639.68663506,304.22373258)(639.61085712,304.30330579)
\curveto(639.53128169,304.37908594)(639.35319369,304.44918441)(639.07659257,304.51360141)
\curveto(637.78829271,304.79778221)(636.97174026,305.057336)(636.62693279,305.29226356)
\curveto(636.14950416,305.61812376)(635.91079045,306.07092198)(635.91079095,306.6506596)
\curveto(635.91079045,307.17355197)(636.11729675,307.61308832)(636.53031047,307.96926997)
\curveto(636.94332197,308.32544034)(637.58368096,308.50352834)(638.45138938,308.50353452)
\curveto(639.27741156,308.50352834)(639.89124681,308.36901506)(640.29289697,308.09999427)
\curveto(640.69453738,307.83096194)(640.97114215,307.43310576)(641.12271212,306.90642454)
\lineto(639.61654079,306.62792493)
\curveto(639.55212178,306.86284524)(639.42897581,307.0428278)(639.24710253,307.16787315)
\curveto(639.0690107,307.29290882)(638.81324601,307.35542908)(638.47980771,307.3554341)
\curveto(638.05921382,307.35542908)(637.75797986,307.29669793)(637.57610491,307.17924048)
\curveto(637.45485134,307.09587529)(637.39422564,306.98788575)(637.39422762,306.85527155)
\curveto(637.39422564,306.74159383)(637.44727313,306.64497162)(637.55337025,306.56540461)
\curveto(637.69735416,306.4593054)(638.1937271,306.30963569)(639.04249058,306.11639505)
\curveto(639.89503591,305.92314683)(640.48992563,305.68632768)(640.82716152,305.40593687)
\curveto(641.16059747,305.12175081)(641.32731816,304.72578919)(641.32732407,304.2180508)
\curveto(641.32731816,303.66483937)(641.09618266,303.18930651)(640.63391689,302.79145079)
\curveto(640.17164068,302.39359415)(639.48770697,302.19466606)(638.58211368,302.19466592)
\curveto(637.75987441,302.19466606)(637.1081481,302.36138674)(636.62693279,302.69482848)
\curveto(636.14950416,303.02826948)(635.83690288,303.48106771)(635.689128,304.05322451)
}
}
{
\newrgbcolor{curcolor}{0 1 0.25098041}
\pscustom[linewidth=2.32802935,linecolor=curcolor]
{
\newpath
\moveto(686.58750578,205.12868455)
\lineto(761.08444272,205.12868455)
\lineto(761.08444272,251.68927596)
\lineto(712.19583105,251.68927596)
\lineto(712.19583105,261.00139424)
\lineto(686.58750578,261.00139424)
\lineto(686.58750578,205.12868455)
\closepath
}
}
{
\newrgbcolor{curcolor}{0 0 0}
\pscustom[linestyle=none,fillstyle=solid,fillcolor=curcolor]
{
\newpath
\moveto(700.02822191,232.64138342)
\lineto(700.02822191,233.59623921)
\lineto(705.54706101,235.92654202)
\lineto(705.54706101,234.90916591)
\lineto(701.17063865,233.11312765)
\lineto(705.54706101,231.30003839)
\lineto(705.54706101,230.28266228)
\lineto(700.02822191,232.64138342)
}
}
{
\newrgbcolor{curcolor}{0 0 0}
\pscustom[linestyle=none,fillstyle=solid,fillcolor=curcolor]
{
\newpath
\moveto(706.8258859,232.64138342)
\lineto(706.8258859,233.59623921)
\lineto(712.344725,235.92654202)
\lineto(712.344725,234.90916591)
\lineto(707.96830265,233.11312765)
\lineto(712.344725,231.30003839)
\lineto(712.344725,230.28266228)
\lineto(706.8258859,232.64138342)
}
}
{
\newrgbcolor{curcolor}{0 0 0}
\pscustom[linestyle=none,fillstyle=solid,fillcolor=curcolor]
{
\newpath
\moveto(717.69305432,229.74271407)
\curveto(717.31413897,229.42063927)(716.94849019,229.19329288)(716.5961069,229.06067422)
\curveto(716.24750549,228.92805543)(715.87238395,228.86174607)(715.47074115,228.86174593)
\curveto(714.80764503,228.86174607)(714.29801021,229.02278309)(713.94183516,229.34485749)
\curveto(713.58565819,229.6707203)(713.40757019,230.08562746)(713.40757061,230.58958021)
\curveto(713.40757019,230.88512892)(713.47387955,231.15415548)(713.6064989,231.3966607)
\curveto(713.74290611,231.64295022)(713.91909956,231.83998376)(714.13507978,231.9877619)
\curveto(714.35484681,232.13553406)(714.60113873,232.2473127)(714.87395628,232.32309816)
\curveto(715.07477704,232.37614232)(715.37790556,232.42729526)(715.78334275,232.47655712)
\curveto(716.60936516,232.57507041)(717.21751675,232.69253271)(717.60779934,232.82894438)
\curveto(717.61158383,232.96913749)(717.61347838,233.05818149)(717.613483,233.09607665)
\curveto(717.61347838,233.51287427)(717.51685616,233.80653002)(717.32361607,233.97704479)
\curveto(717.06216339,234.2081753)(716.67377997,234.32374305)(716.15846466,234.32374838)
\curveto(715.67724497,234.32374305)(715.32106896,234.23848816)(715.08993557,234.06798343)
\curveto(714.86258708,233.90125768)(714.69397184,233.60381282)(714.58408935,233.17564797)
\lineto(713.58376424,233.31205594)
\curveto(713.6747022,233.74022065)(713.8243719,234.08502934)(714.0327738,234.34648304)
\curveto(714.24117361,234.61171514)(714.54240758,234.81443234)(714.9364766,234.95463524)
\curveto(715.33054173,235.09861533)(715.78712906,235.17060835)(716.30623996,235.17061452)
\curveto(716.82155512,235.17060835)(717.24025139,235.10998265)(717.56233002,234.98873723)
\curveto(717.88439949,234.86747983)(718.12121865,234.71402102)(718.27278819,234.52836033)
\curveto(718.42434716,234.34647769)(718.53044215,234.1153422)(718.59107345,233.83495315)
\curveto(718.62516981,233.66064942)(718.64222079,233.34615358)(718.64222644,232.8914647)
\lineto(718.64222644,231.527385)
\curveto(718.64222079,230.57631675)(718.66306087,229.97384882)(718.70474676,229.7199794)
\curveto(718.75021032,229.46989766)(718.83735977,229.22928939)(718.96619537,228.9981539)
\lineto(717.89766627,228.9981539)
\curveto(717.79156638,229.21034386)(717.72336247,229.45853034)(717.69305432,229.74271407)
\moveto(717.60779934,232.02754756)
\curveto(717.23646228,231.87598027)(716.67946363,231.74715065)(715.93680171,231.64105831)
\curveto(715.51620794,231.58042996)(715.21876309,231.51222605)(715.04446624,231.43644636)
\curveto(714.87016529,231.36066179)(714.73565201,231.24888315)(714.640926,231.1011101)
\curveto(714.54619669,230.95712195)(714.49883285,230.79608492)(714.49883437,230.61799854)
\curveto(714.49883285,230.34518125)(714.60113873,230.11783486)(714.8057523,229.93595869)
\curveto(715.01415134,229.75408064)(715.31727985,229.66314209)(715.71513876,229.66314275)
\curveto(716.10920311,229.66314209)(716.45969546,229.74839698)(716.76661686,229.91890769)
\curveto(717.07353071,230.09320567)(717.29898254,230.33002483)(717.44297304,230.62936587)
\curveto(717.55285267,230.86049973)(717.60779472,231.20151932)(717.60779934,231.65242564)
\lineto(717.60779934,232.02754756)
}
}
{
\newrgbcolor{curcolor}{0 0 0}
\pscustom[linestyle=none,fillstyle=solid,fillcolor=curcolor]
{
\newpath
\moveto(720.23365216,226.68490208)
\lineto(720.23365216,235.03420655)
\lineto(721.16577329,235.03420655)
\lineto(721.16577329,234.24986073)
\curveto(721.38553976,234.5567731)(721.63372624,234.78601404)(721.91033345,234.93758424)
\curveto(722.18693578,235.09293167)(722.5222717,235.17060835)(722.91634223,235.17061452)
\curveto(723.43165726,235.17060835)(723.88635004,235.03798962)(724.28042192,234.77275794)
\curveto(724.67448418,234.50751472)(724.97192904,234.13239317)(725.17275739,233.6473922)
\curveto(725.37357433,233.16617102)(725.47398565,232.63759067)(725.47399166,232.06164955)
\curveto(725.47398565,231.44402213)(725.36220701,230.88702348)(725.1386554,230.39065192)
\curveto(724.91888155,229.89806669)(724.5968075,229.51915604)(724.17243228,229.25391884)
\curveto(723.75183676,228.99247024)(723.3085113,228.86174607)(722.84245458,228.86174593)
\curveto(722.50143162,228.86174607)(722.19451399,228.93373909)(721.92170078,229.07772521)
\curveto(721.65267177,229.22171118)(721.43100904,229.40358829)(721.25671193,229.62335709)
\lineto(721.25671193,226.68490208)
\lineto(720.23365216,226.68490208)
\moveto(721.16008962,231.98207823)
\curveto(721.16008793,231.20530842)(721.31733584,230.63125879)(721.63183385,230.25992762)
\curveto(721.94632752,229.88859392)(722.32713272,229.7029277)(722.77425059,229.70292841)
\curveto(723.22894006,229.7029277)(723.61732348,229.89427758)(723.939402,230.27697861)
\curveto(724.26526068,230.6634662)(724.42819226,231.26025047)(724.42819722,232.06733321)
\curveto(724.42819226,232.83651876)(724.26904979,233.41246294)(723.95076933,233.7951675)
\curveto(723.63626901,234.17786245)(723.25925291,234.36921233)(722.81971992,234.3692177)
\curveto(722.38396932,234.36921233)(721.99748046,234.16460058)(721.66025217,233.75538184)
\curveto(721.32680861,233.34994269)(721.16008793,232.75884208)(721.16008962,231.98207823)
}
}
{
\newrgbcolor{curcolor}{0 0 0}
\pscustom[linestyle=none,fillstyle=solid,fillcolor=curcolor]
{
\newpath
\moveto(726.71303101,226.68490208)
\lineto(726.71303101,235.03420655)
\lineto(727.64515213,235.03420655)
\lineto(727.64515213,234.24986073)
\curveto(727.86491861,234.5567731)(728.11310508,234.78601404)(728.3897123,234.93758424)
\curveto(728.66631463,235.09293167)(729.00165055,235.17060835)(729.39572108,235.17061452)
\curveto(729.91103611,235.17060835)(730.36572889,235.03798962)(730.75980077,234.77275794)
\curveto(731.15386303,234.50751472)(731.45130789,234.13239317)(731.65213624,233.6473922)
\curveto(731.85295318,233.16617102)(731.9533645,232.63759067)(731.95337051,232.06164955)
\curveto(731.9533645,231.44402213)(731.84158586,230.88702348)(731.61803425,230.39065192)
\curveto(731.3982604,229.89806669)(731.07618635,229.51915604)(730.65181113,229.25391884)
\curveto(730.23121561,228.99247024)(729.78789015,228.86174607)(729.32183343,228.86174593)
\curveto(728.98081047,228.86174607)(728.67389284,228.93373909)(728.40107963,229.07772521)
\curveto(728.13205062,229.22171118)(727.91038789,229.40358829)(727.73609078,229.62335709)
\lineto(727.73609078,226.68490208)
\lineto(726.71303101,226.68490208)
\moveto(727.63946847,231.98207823)
\curveto(727.63946678,231.20530842)(727.79671469,230.63125879)(728.1112127,230.25992762)
\curveto(728.42570637,229.88859392)(728.80651157,229.7029277)(729.25362944,229.70292841)
\curveto(729.70831891,229.7029277)(730.09670233,229.89427758)(730.41878085,230.27697861)
\curveto(730.74463953,230.6634662)(730.90757111,231.26025047)(730.90757607,232.06733321)
\curveto(730.90757111,232.83651876)(730.74842864,233.41246294)(730.43014818,233.7951675)
\curveto(730.11564786,234.17786245)(729.73863176,234.36921233)(729.29909877,234.3692177)
\curveto(728.86334817,234.36921233)(728.47685931,234.16460058)(728.13963102,233.75538184)
\curveto(727.80618746,233.34994269)(727.63946678,232.75884208)(727.63946847,231.98207823)
}
}
{
\newrgbcolor{curcolor}{0 0 0}
\pscustom[linestyle=none,fillstyle=solid,fillcolor=curcolor]
{
\newpath
\moveto(738.58052466,232.64138342)
\lineto(733.06168555,230.28266228)
\lineto(733.06168555,231.30003839)
\lineto(737.43242425,233.11312765)
\lineto(733.06168555,234.90916591)
\lineto(733.06168555,235.92654202)
\lineto(738.58052466,233.59623921)
\lineto(738.58052466,232.64138342)
}
}
{
\newrgbcolor{curcolor}{0 0 0}
\pscustom[linestyle=none,fillstyle=solid,fillcolor=curcolor]
{
\newpath
\moveto(745.37818732,232.64138342)
\lineto(739.85934821,230.28266228)
\lineto(739.85934821,231.30003839)
\lineto(744.23008691,233.11312765)
\lineto(739.85934821,234.90916591)
\lineto(739.85934821,235.92654202)
\lineto(745.37818732,233.59623921)
\lineto(745.37818732,232.64138342)
}
}
{
\newrgbcolor{curcolor}{0 0 0}
\pscustom[linestyle=none,fillstyle=solid,fillcolor=curcolor]
{
\newpath
\moveto(709.56885374,215.73246909)
\lineto(708.11951907,215.99391769)
\curveto(708.28245007,216.57743564)(708.56284395,217.00939377)(708.96070155,217.28979341)
\curveto(709.35855631,217.57018153)(709.94965692,217.71037847)(710.73400515,217.71038464)
\curveto(711.44635397,217.71037847)(711.97682888,217.62512358)(712.32543146,217.4546197)
\curveto(712.67402447,217.2878931)(712.91842184,217.07380858)(713.0586243,216.81236551)
\curveto(713.20260483,216.554701)(713.27459785,216.07916813)(713.27460358,215.3857655)
\lineto(713.25755259,213.52152325)
\curveto(713.25754687,212.99104636)(713.28217606,212.59887384)(713.33144024,212.34500451)
\curveto(713.38448194,212.09492267)(713.48110415,211.82589611)(713.62130717,211.53792402)
\lineto(712.04124819,211.53792402)
\curveto(711.99956352,211.644019)(711.94841058,211.80126692)(711.88778923,212.00966825)
\curveto(711.86126113,212.10439544)(711.8423156,212.1669157)(711.83095257,212.19722921)
\curveto(711.55813261,211.9319911)(711.26637142,211.73306301)(710.9556681,211.60044434)
\curveto(710.64495795,211.46782555)(710.31341114,211.40151619)(709.96102666,211.40151605)
\curveto(709.33961077,211.40151619)(708.84892148,211.57013143)(708.48895732,211.90736227)
\curveto(708.13278036,212.24459238)(707.95469236,212.67086686)(707.95469277,213.18618699)
\curveto(707.95469236,213.52720492)(708.03615815,213.83033344)(708.19909038,214.09557345)
\curveto(708.3620213,214.36459745)(708.58936769,214.5692092)(708.88113023,214.70940931)
\curveto(709.17667919,214.85339219)(709.60105912,214.9784327)(710.15427128,215.08453123)
\curveto(710.90072264,215.22472462)(711.41793568,215.3554488)(711.70591193,215.47670414)
\lineto(711.70591193,215.63584677)
\curveto(711.70590777,215.9427603)(711.63012564,216.16063392)(711.47856532,216.2894683)
\curveto(711.32699712,216.42208227)(711.04091958,216.48839163)(710.62033184,216.48839658)
\curveto(710.33614578,216.48839163)(710.11448305,216.43155504)(709.95534299,216.31788662)
\curveto(709.7961981,216.20799775)(709.66736848,216.01285877)(709.56885374,215.73246909)
\moveto(711.70591193,214.43659338)
\curveto(711.50129602,214.36838656)(711.17732741,214.28692077)(710.73400515,214.19219576)
\curveto(710.2906765,214.09746545)(710.00080985,214.00463234)(709.86440434,213.91369616)
\curveto(709.65600116,213.76591863)(709.55180074,213.57835786)(709.55180275,213.35101328)
\curveto(709.55180074,213.12745419)(709.63516108,212.93420976)(709.80188403,212.77127941)
\curveto(709.96860245,212.6083466)(710.18079241,212.52688081)(710.43845455,212.5268818)
\curveto(710.72642374,212.52688081)(711.00113396,212.62160847)(711.26258603,212.81106507)
\curveto(711.45582674,212.95504984)(711.58276181,213.1312433)(711.64339161,213.33964595)
\curveto(711.68506768,213.47605198)(711.70590777,213.73560578)(711.70591193,214.11830811)
\lineto(711.70591193,214.43659338)
}
}
{
\newrgbcolor{curcolor}{0 0 0}
\pscustom[linestyle=none,fillstyle=solid,fillcolor=curcolor]
{
\newpath
\moveto(716.38356841,211.53792402)
\lineto(714.78645843,211.53792402)
\lineto(714.78645843,217.57397668)
\lineto(716.2698951,217.57397668)
\lineto(716.2698951,216.7157432)
\curveto(716.52376298,217.12117242)(716.75110937,217.38830442)(716.95193495,217.51714002)
\curveto(717.15654376,217.64596366)(717.38767926,217.71037847)(717.64534213,217.71038464)
\curveto(718.00909272,217.71037847)(718.35958507,217.60996715)(718.69682022,217.40915038)
\lineto(718.20234133,216.01665236)
\curveto(717.93331059,216.19094678)(717.68322956,216.27809622)(717.4520975,216.27810096)
\curveto(717.22853679,216.27809622)(717.03908146,216.21557597)(716.88373096,216.09054001)
\curveto(716.72837473,215.96928405)(716.60522877,215.74762132)(716.51429271,215.42555115)
\curveto(716.42714077,215.10347322)(716.38356604,214.42901226)(716.38356841,213.40216627)
\lineto(716.38356841,211.53792402)
}
}
{
\newrgbcolor{curcolor}{0 0 0}
\pscustom[linestyle=none,fillstyle=solid,fillcolor=curcolor]
{
\newpath
\moveto(722.87431379,213.45900293)
\lineto(724.4657401,213.19187065)
\curveto(724.26112243,212.6083466)(723.93715382,212.16312659)(723.49383332,211.85620928)
\curveto(723.05429202,211.55308045)(722.50297702,211.40151619)(721.83988669,211.40151605)
\curveto(720.7903009,211.40151619)(720.01353407,211.74443032)(719.50958387,212.43025949)
\curveto(719.11172673,212.97967904)(718.91279864,213.67308552)(718.91279901,214.51048103)
\curveto(718.91279864,215.51080216)(719.17424698,216.29325265)(719.69714483,216.85783483)
\curveto(720.22004037,217.42619549)(720.88123945,217.71037847)(721.68074406,217.71038464)
\curveto(722.57875915,217.71037847)(723.28732206,217.41293361)(723.80643491,216.81804918)
\curveto(724.32553724,216.22694329)(724.57372371,215.31945229)(724.55099508,214.09557345)
\lineto(720.54969464,214.09557345)
\curveto(720.56105995,213.62193258)(720.68988957,213.2524947)(720.93618389,212.9872587)
\curveto(721.18247342,212.7258089)(721.48939104,212.59508473)(721.85693768,212.59508579)
\curveto(722.1070154,212.59508473)(722.31731081,212.66328865)(722.48782454,212.79969774)
\curveto(722.65833039,212.93610431)(722.78716001,213.15587249)(722.87431379,213.45900293)
\moveto(722.96525244,215.0731639)
\curveto(722.95388069,215.53543135)(722.83452384,215.8859237)(722.60718152,216.124642)
\curveto(722.37983106,216.36714023)(722.10322629,216.48839163)(721.77736637,216.48839658)
\curveto(721.42876534,216.48839163)(721.14079324,216.36145657)(720.91344923,216.107591)
\curveto(720.68610047,215.8537163)(720.57432183,215.50890761)(720.57811297,215.0731639)
\lineto(722.96525244,215.0731639)
}
}
{
\newrgbcolor{curcolor}{0 0 0}
\pscustom[linestyle=none,fillstyle=solid,fillcolor=curcolor]
{
\newpath
\moveto(727.05180815,215.73246909)
\lineto(725.60247348,215.99391769)
\curveto(725.76540447,216.57743564)(726.04579835,217.00939377)(726.44365595,217.28979341)
\curveto(726.84151071,217.57018153)(727.43261132,217.71037847)(728.21695956,217.71038464)
\curveto(728.92930838,217.71037847)(729.45978329,217.62512358)(729.80838587,217.4546197)
\curveto(730.15697888,217.2878931)(730.40137625,217.07380858)(730.54157871,216.81236551)
\curveto(730.68555923,216.554701)(730.75755226,216.07916813)(730.75755799,215.3857655)
\lineto(730.740507,213.52152325)
\curveto(730.74050128,212.99104636)(730.76513047,212.59887384)(730.81439465,212.34500451)
\curveto(730.86743634,212.09492267)(730.96405856,211.82589611)(731.10426158,211.53792402)
\lineto(729.5242026,211.53792402)
\curveto(729.48251793,211.644019)(729.43136499,211.80126692)(729.37074363,212.00966825)
\curveto(729.34421554,212.10439544)(729.32527001,212.1669157)(729.31390698,212.19722921)
\curveto(729.04108702,211.9319911)(728.74932582,211.73306301)(728.43862251,211.60044434)
\curveto(728.12791236,211.46782555)(727.79636555,211.40151619)(727.44398106,211.40151605)
\curveto(726.82256518,211.40151619)(726.33187589,211.57013143)(725.97191173,211.90736227)
\curveto(725.61573477,212.24459238)(725.43764676,212.67086686)(725.43764718,213.18618699)
\curveto(725.43764676,213.52720492)(725.51911255,213.83033344)(725.68204479,214.09557345)
\curveto(725.84497571,214.36459745)(726.0723221,214.5692092)(726.36408464,214.70940931)
\curveto(726.6596336,214.85339219)(727.08401353,214.9784327)(727.63722569,215.08453123)
\curveto(728.38367705,215.22472462)(728.90089008,215.3554488)(729.18886634,215.47670414)
\lineto(729.18886634,215.63584677)
\curveto(729.18886218,215.9427603)(729.11308005,216.16063392)(728.96151973,216.2894683)
\curveto(728.80995153,216.42208227)(728.52387399,216.48839163)(728.10328625,216.48839658)
\curveto(727.81910018,216.48839163)(727.59743746,216.43155504)(727.4382974,216.31788662)
\curveto(727.27915251,216.20799775)(727.15032289,216.01285877)(727.05180815,215.73246909)
\moveto(729.18886634,214.43659338)
\curveto(728.98425043,214.36838656)(728.66028182,214.28692077)(728.21695956,214.19219576)
\curveto(727.77363091,214.09746545)(727.48376426,214.00463234)(727.34735875,213.91369616)
\curveto(727.13895557,213.76591863)(727.03475514,213.57835786)(727.03475716,213.35101328)
\curveto(727.03475514,213.12745419)(727.11811549,212.93420976)(727.28483843,212.77127941)
\curveto(727.45155686,212.6083466)(727.66374682,212.52688081)(727.92140896,212.5268818)
\curveto(728.20937815,212.52688081)(728.48408837,212.62160847)(728.74554044,212.81106507)
\curveto(728.93878115,212.95504984)(729.06571621,213.1312433)(729.12634602,213.33964595)
\curveto(729.16802209,213.47605198)(729.18886218,213.73560578)(729.18886634,214.11830811)
\lineto(729.18886634,214.43659338)
}
}
{
\newrgbcolor{curcolor}{0 0 0}
\pscustom[linestyle=none,fillstyle=solid,fillcolor=curcolor]
{
\newpath
\moveto(731.77493439,213.26007464)
\lineto(733.37772804,213.50447225)
\curveto(733.44593008,213.19376355)(733.58423246,212.9569444)(733.79263561,212.79401408)
\curveto(734.00103417,212.63487035)(734.29279537,212.55529911)(734.66792008,212.55530013)
\curveto(735.08092952,212.55529911)(735.39163625,212.63108124)(735.60004121,212.78264674)
\curveto(735.74023405,212.88874048)(735.81033252,213.03083197)(735.81033683,213.20892165)
\curveto(735.81033252,213.33017139)(735.77244145,213.43058271)(735.69666352,213.51015592)
\curveto(735.61708809,213.58593607)(735.43900008,213.65603454)(735.16239897,213.72045153)
\curveto(733.87409911,214.00463234)(733.05754666,214.26418613)(732.71273918,214.49911369)
\curveto(732.23531056,214.82497389)(731.99659685,215.27777211)(731.99659734,215.85750972)
\curveto(731.99659685,216.3804021)(732.20310315,216.81993845)(732.61611687,217.1761201)
\curveto(733.02912836,217.53229047)(733.66948736,217.71037847)(734.53719578,217.71038464)
\curveto(735.36321795,217.71037847)(735.9770532,217.57586519)(736.37870337,217.3068444)
\curveto(736.78034378,217.03781207)(737.05694855,216.63995589)(737.20851851,216.11327467)
\lineto(735.70234718,215.83477506)
\curveto(735.63792817,216.06969537)(735.51478221,216.24967793)(735.33290893,216.37472328)
\curveto(735.1548171,216.49975895)(734.89905241,216.56227921)(734.5656141,216.56228423)
\curveto(734.14502022,216.56227921)(733.84378626,216.50354806)(733.6619113,216.38609061)
\curveto(733.54065774,216.30272542)(733.48003203,216.19473588)(733.48003401,216.06212168)
\curveto(733.48003203,215.94844396)(733.53307952,215.85182175)(733.63917664,215.77225474)
\curveto(733.78316055,215.66615553)(734.2795335,215.51648582)(735.12829698,215.32324518)
\curveto(735.98084231,215.12999696)(736.57573203,214.89317781)(736.91296791,214.612787)
\curveto(737.24640387,214.32860094)(737.41312456,213.93263932)(737.41313047,213.42490093)
\curveto(737.41312456,212.8716895)(737.18198906,212.39615664)(736.71972329,211.99830092)
\curveto(736.25744708,211.60044428)(735.57351336,211.40151619)(734.66792008,211.40151605)
\curveto(733.84568081,211.40151619)(733.1939545,211.56823687)(732.71273918,211.90167861)
\curveto(732.23531056,212.23511961)(731.92270927,212.68791784)(731.77493439,213.26007464)
}
}
{
\newrgbcolor{curcolor}{0 1 0.25098041}
\pscustom[linewidth=2.32802935,linecolor=curcolor]
{
\newpath
\moveto(581.82618442,205.12868455)
\lineto(656.32312136,205.12868455)
\lineto(656.32312136,251.68927596)
\lineto(606.27049025,251.68927596)
\lineto(606.27049025,261.00139424)
\lineto(581.82618442,261.00139424)
\lineto(581.82618442,205.12868455)
\closepath
}
}
{
\newrgbcolor{curcolor}{0 0 0}
\pscustom[linestyle=none,fillstyle=solid,fillcolor=curcolor]
{
\newpath
\moveto(595.26689951,232.64138342)
\lineto(595.26689951,233.59623921)
\lineto(600.78573861,235.92654202)
\lineto(600.78573861,234.90916591)
\lineto(596.40931625,233.11312765)
\lineto(600.78573861,231.30003839)
\lineto(600.78573861,230.28266228)
\lineto(595.26689951,232.64138342)
}
}
{
\newrgbcolor{curcolor}{0 0 0}
\pscustom[linestyle=none,fillstyle=solid,fillcolor=curcolor]
{
\newpath
\moveto(602.0645635,232.64138342)
\lineto(602.0645635,233.59623921)
\lineto(607.5834026,235.92654202)
\lineto(607.5834026,234.90916591)
\lineto(603.20698025,233.11312765)
\lineto(607.5834026,231.30003839)
\lineto(607.5834026,230.28266228)
\lineto(602.0645635,232.64138342)
}
}
{
\newrgbcolor{curcolor}{0 0 0}
\pscustom[linestyle=none,fillstyle=solid,fillcolor=curcolor]
{
\newpath
\moveto(612.93173192,229.74271407)
\curveto(612.55281657,229.42063927)(612.18716779,229.19329288)(611.8347845,229.06067422)
\curveto(611.48618309,228.92805543)(611.11106155,228.86174607)(610.70941875,228.86174593)
\curveto(610.04632263,228.86174607)(609.53668781,229.02278309)(609.18051276,229.34485749)
\curveto(608.82433579,229.6707203)(608.64624779,230.08562746)(608.64624821,230.58958021)
\curveto(608.64624779,230.88512892)(608.71255715,231.15415548)(608.8451765,231.3966607)
\curveto(608.98158371,231.64295022)(609.15777716,231.83998376)(609.37375738,231.9877619)
\curveto(609.59352441,232.13553406)(609.83981633,232.2473127)(610.11263388,232.32309816)
\curveto(610.31345464,232.37614232)(610.61658316,232.42729526)(611.02202035,232.47655712)
\curveto(611.84804276,232.57507041)(612.45619435,232.69253271)(612.84647694,232.82894438)
\curveto(612.85026143,232.96913749)(612.85215598,233.05818149)(612.85216061,233.09607665)
\curveto(612.85215598,233.51287427)(612.75553376,233.80653002)(612.56229367,233.97704479)
\curveto(612.30084099,234.2081753)(611.91245757,234.32374305)(611.39714226,234.32374838)
\curveto(610.91592257,234.32374305)(610.55974656,234.23848816)(610.32861317,234.06798343)
\curveto(610.10126468,233.90125768)(609.93264944,233.60381282)(609.82276695,233.17564797)
\lineto(608.82244184,233.31205594)
\curveto(608.9133798,233.74022065)(609.0630495,234.08502934)(609.2714514,234.34648304)
\curveto(609.47985121,234.61171514)(609.78108518,234.81443234)(610.1751542,234.95463524)
\curveto(610.56921933,235.09861533)(611.02580666,235.17060835)(611.54491756,235.17061452)
\curveto(612.06023273,235.17060835)(612.47892899,235.10998265)(612.80100762,234.98873723)
\curveto(613.12307709,234.86747983)(613.35989625,234.71402102)(613.51146579,234.52836033)
\curveto(613.66302476,234.34647769)(613.76911975,234.1153422)(613.82975105,233.83495315)
\curveto(613.86384741,233.66064942)(613.88089839,233.34615358)(613.88090404,232.8914647)
\lineto(613.88090404,231.527385)
\curveto(613.88089839,230.57631675)(613.90173847,229.97384882)(613.94342436,229.7199794)
\curveto(613.98888792,229.46989766)(614.07603737,229.22928939)(614.20487297,228.9981539)
\lineto(613.13634388,228.9981539)
\curveto(613.03024398,229.21034386)(612.96204007,229.45853034)(612.93173192,229.74271407)
\moveto(612.84647694,232.02754756)
\curveto(612.47513988,231.87598027)(611.91814123,231.74715065)(611.17547931,231.64105831)
\curveto(610.75488554,231.58042996)(610.45744069,231.51222605)(610.28314385,231.43644636)
\curveto(610.10884289,231.36066179)(609.97432961,231.24888315)(609.8796036,231.1011101)
\curveto(609.78487429,230.95712195)(609.73751046,230.79608492)(609.73751197,230.61799854)
\curveto(609.73751046,230.34518125)(609.83981633,230.11783486)(610.0444299,229.93595869)
\curveto(610.25282894,229.75408064)(610.55595745,229.66314209)(610.95381636,229.66314275)
\curveto(611.34788071,229.66314209)(611.69837306,229.74839698)(612.00529446,229.91890769)
\curveto(612.31220831,230.09320567)(612.53766014,230.33002483)(612.68165064,230.62936587)
\curveto(612.79153028,230.86049973)(612.84647232,231.20151932)(612.84647694,231.65242564)
\lineto(612.84647694,232.02754756)
}
}
{
\newrgbcolor{curcolor}{0 0 0}
\pscustom[linestyle=none,fillstyle=solid,fillcolor=curcolor]
{
\newpath
\moveto(615.47232976,226.68490208)
\lineto(615.47232976,235.03420655)
\lineto(616.40445089,235.03420655)
\lineto(616.40445089,234.24986073)
\curveto(616.62421736,234.5567731)(616.87240384,234.78601404)(617.14901105,234.93758424)
\curveto(617.42561338,235.09293167)(617.76094931,235.17060835)(618.15501983,235.17061452)
\curveto(618.67033486,235.17060835)(619.12502764,235.03798962)(619.51909952,234.77275794)
\curveto(619.91316178,234.50751472)(620.21060664,234.13239317)(620.41143499,233.6473922)
\curveto(620.61225193,233.16617102)(620.71266325,232.63759067)(620.71266926,232.06164955)
\curveto(620.71266325,231.44402213)(620.60088461,230.88702348)(620.377333,230.39065192)
\curveto(620.15755915,229.89806669)(619.8354851,229.51915604)(619.41110988,229.25391884)
\curveto(618.99051436,228.99247024)(618.5471889,228.86174607)(618.08113218,228.86174593)
\curveto(617.74010922,228.86174607)(617.43319159,228.93373909)(617.16037838,229.07772521)
\curveto(616.89134937,229.22171118)(616.66968664,229.40358829)(616.49538953,229.62335709)
\lineto(616.49538953,226.68490208)
\lineto(615.47232976,226.68490208)
\moveto(616.39876722,231.98207823)
\curveto(616.39876553,231.20530842)(616.55601345,230.63125879)(616.87051145,230.25992762)
\curveto(617.18500512,229.88859392)(617.56581032,229.7029277)(618.01292819,229.70292841)
\curveto(618.46761766,229.7029277)(618.85600108,229.89427758)(619.1780796,230.27697861)
\curveto(619.50393828,230.6634662)(619.66686986,231.26025047)(619.66687482,232.06733321)
\curveto(619.66686986,232.83651876)(619.50772739,233.41246294)(619.18944693,233.7951675)
\curveto(618.87494661,234.17786245)(618.49793051,234.36921233)(618.05839752,234.3692177)
\curveto(617.62264692,234.36921233)(617.23615806,234.16460058)(616.89892978,233.75538184)
\curveto(616.56548621,233.34994269)(616.39876553,232.75884208)(616.39876722,231.98207823)
}
}
{
\newrgbcolor{curcolor}{0 0 0}
\pscustom[linestyle=none,fillstyle=solid,fillcolor=curcolor]
{
\newpath
\moveto(621.95170861,226.68490208)
\lineto(621.95170861,235.03420655)
\lineto(622.88382974,235.03420655)
\lineto(622.88382974,234.24986073)
\curveto(623.10359621,234.5567731)(623.35178269,234.78601404)(623.6283899,234.93758424)
\curveto(623.90499223,235.09293167)(624.24032815,235.17060835)(624.63439868,235.17061452)
\curveto(625.14971371,235.17060835)(625.60440649,235.03798962)(625.99847837,234.77275794)
\curveto(626.39254063,234.50751472)(626.68998549,234.13239317)(626.89081384,233.6473922)
\curveto(627.09163078,233.16617102)(627.1920421,232.63759067)(627.19204811,232.06164955)
\curveto(627.1920421,231.44402213)(627.08026346,230.88702348)(626.85671185,230.39065192)
\curveto(626.636938,229.89806669)(626.31486395,229.51915604)(625.89048873,229.25391884)
\curveto(625.46989321,228.99247024)(625.02656775,228.86174607)(624.56051103,228.86174593)
\curveto(624.21948807,228.86174607)(623.91257044,228.93373909)(623.63975723,229.07772521)
\curveto(623.37072822,229.22171118)(623.14906549,229.40358829)(622.97476838,229.62335709)
\lineto(622.97476838,226.68490208)
\lineto(621.95170861,226.68490208)
\moveto(622.87814607,231.98207823)
\curveto(622.87814438,231.20530842)(623.03539229,230.63125879)(623.3498903,230.25992762)
\curveto(623.66438397,229.88859392)(624.04518917,229.7029277)(624.49230704,229.70292841)
\curveto(624.94699651,229.7029277)(625.33537993,229.89427758)(625.65745845,230.27697861)
\curveto(625.98331713,230.6634662)(626.14624871,231.26025047)(626.14625367,232.06733321)
\curveto(626.14624871,232.83651876)(625.98710624,233.41246294)(625.66882578,233.7951675)
\curveto(625.35432546,234.17786245)(624.97730936,234.36921233)(624.53777637,234.3692177)
\curveto(624.10202577,234.36921233)(623.71553691,234.16460058)(623.37830862,233.75538184)
\curveto(623.04486506,233.34994269)(622.87814438,232.75884208)(622.87814607,231.98207823)
}
}
{
\newrgbcolor{curcolor}{0 0 0}
\pscustom[linestyle=none,fillstyle=solid,fillcolor=curcolor]
{
\newpath
\moveto(633.81920226,232.64138342)
\lineto(628.30036315,230.28266228)
\lineto(628.30036315,231.30003839)
\lineto(632.67110185,233.11312765)
\lineto(628.30036315,234.90916591)
\lineto(628.30036315,235.92654202)
\lineto(633.81920226,233.59623921)
\lineto(633.81920226,232.64138342)
}
}
{
\newrgbcolor{curcolor}{0 0 0}
\pscustom[linestyle=none,fillstyle=solid,fillcolor=curcolor]
{
\newpath
\moveto(640.61686492,232.64138342)
\lineto(635.09802582,230.28266228)
\lineto(635.09802582,231.30003839)
\lineto(639.46876451,233.11312765)
\lineto(635.09802582,234.90916591)
\lineto(635.09802582,235.92654202)
\lineto(640.61686492,233.59623921)
\lineto(640.61686492,232.64138342)
}
}
{
\newrgbcolor{curcolor}{0 0 0}
\pscustom[linestyle=none,fillstyle=solid,fillcolor=curcolor]
{
\newpath
\moveto(599.97412896,211.14006744)
\lineto(601.79858555,210.91840449)
\curveto(601.82889589,210.70621515)(601.89899436,210.56033455)(602.00888117,210.48076226)
\curveto(602.16044271,210.36709012)(602.39915642,210.31025352)(602.72502301,210.3102523)
\curveto(603.14182129,210.31025352)(603.45442257,210.37277378)(603.66282781,210.49781325)
\curveto(603.80302037,210.58117464)(603.90911535,210.71568792)(603.98111307,210.9013535)
\curveto(604.03036676,211.03397286)(604.05499595,211.27837023)(604.05500072,211.63454633)
\lineto(604.05500072,212.51551447)
\curveto(603.57756853,211.86378718)(602.9751006,211.53792402)(602.24759512,211.53792402)
\curveto(601.43672337,211.53792402)(600.79446983,211.88083816)(600.32083255,212.56666746)
\curveto(599.94949908,213.10850866)(599.76383286,213.78296961)(599.76383334,214.59005234)
\curveto(599.76383286,215.60174072)(600.00633568,216.37471844)(600.49134251,216.90898782)
\curveto(600.98013604,217.44324647)(601.58639308,217.71037847)(602.31011544,217.71038464)
\curveto(603.05656639,217.71037847)(603.6722962,217.38262076)(604.1573067,216.72711053)
\lineto(604.1573067,217.57397668)
\lineto(605.65211069,217.57397668)
\lineto(605.65211069,212.15744355)
\curveto(605.65210433,211.44509091)(605.59337318,210.91272145)(605.47591707,210.56033357)
\curveto(605.35844858,210.20794765)(605.19362245,209.93134288)(604.98143818,209.73051843)
\curveto(604.76924252,209.52969759)(604.48505953,209.37244967)(604.12888837,209.2587742)
\curveto(603.77649662,209.14510328)(603.32938206,209.08826668)(602.78754333,209.08826424)
\curveto(601.76448108,209.08826668)(601.03886719,209.26446014)(600.61069949,209.61684512)
\curveto(600.18252913,209.96544483)(599.96844461,210.40877029)(599.9684453,210.94682282)
\curveto(599.96844461,210.9998709)(599.97033917,211.06428571)(599.97412896,211.14006744)
\moveto(601.40072898,214.68099099)
\curveto(601.40072686,214.04062885)(601.52387282,213.57077965)(601.77016723,213.27144197)
\curveto(602.02024577,212.97588993)(602.3271634,212.82811478)(602.69092102,212.82811607)
\curveto(603.08119558,212.82811478)(603.41084785,212.97967904)(603.6798788,213.2828093)
\curveto(603.94890097,213.58972518)(604.08341425,214.0425234)(604.08341905,214.64120533)
\curveto(604.08341425,215.26640479)(603.95458463,215.73057034)(603.6969298,216.03370335)
\curveto(603.43926615,216.33682737)(603.11340299,216.48839163)(602.71933935,216.48839658)
\curveto(602.33663616,216.48839163)(602.02024577,216.33872193)(601.77016723,216.03938702)
\curveto(601.52387282,215.74383221)(601.40072686,215.29103399)(601.40072898,214.68099099)
}
}
{
\newrgbcolor{curcolor}{0 0 0}
\pscustom[linestyle=none,fillstyle=solid,fillcolor=curcolor]
{
\newpath
\moveto(608.76675939,211.53792402)
\lineto(607.16964942,211.53792402)
\lineto(607.16964942,217.57397668)
\lineto(608.65308608,217.57397668)
\lineto(608.65308608,216.7157432)
\curveto(608.90695397,217.12117242)(609.13430036,217.38830442)(609.33512593,217.51714002)
\curveto(609.53973475,217.64596366)(609.77087024,217.71037847)(610.02853311,217.71038464)
\curveto(610.39228371,217.71037847)(610.74277606,217.60996715)(611.08001121,217.40915038)
\lineto(610.58553232,216.01665236)
\curveto(610.31650158,216.19094678)(610.06642055,216.27809622)(609.83528849,216.27810096)
\curveto(609.61172777,216.27809622)(609.42227245,216.21557597)(609.26692195,216.09054001)
\curveto(609.11156572,215.96928405)(608.98841976,215.74762132)(608.8974837,215.42555115)
\curveto(608.81033175,215.10347322)(608.76675703,214.42901226)(608.76675939,213.40216627)
\lineto(608.76675939,211.53792402)
}
}
{
\newrgbcolor{curcolor}{0 0 0}
\pscustom[linestyle=none,fillstyle=solid,fillcolor=curcolor]
{
\newpath
\moveto(611.39261275,214.64120533)
\curveto(611.39261228,215.17167713)(611.52333645,215.68510106)(611.78478566,216.18147865)
\curveto(612.04623315,216.67784696)(612.41567103,217.0567576)(612.89310041,217.31821173)
\curveto(613.37431497,217.5796543)(613.91047353,217.71037847)(614.50157772,217.71038464)
\curveto(615.41474881,217.71037847)(616.16309734,217.41293361)(616.74662555,216.81804918)
\curveto(617.33014213,216.22694329)(617.62190333,215.47859476)(617.62191002,214.57300134)
\curveto(617.62190333,213.65982365)(617.32635302,212.90200235)(616.73525822,212.29953519)
\curveto(616.14794091,211.7008556)(615.40717059,211.40151619)(614.51294505,211.40151605)
\curveto(613.95973192,211.40151619)(613.43115157,211.5265567)(612.9272024,211.77663797)
\curveto(612.42703835,212.02671876)(612.04623315,212.39236753)(611.78478566,212.87358539)
\curveto(611.52333645,213.35858968)(611.39261228,213.94779574)(611.39261275,214.64120533)
\moveto(613.02950838,214.55595035)
\curveto(613.02950628,213.95726851)(613.17159777,213.49878662)(613.45578329,213.18050332)
\curveto(613.73996374,212.86221674)(614.09045609,212.70307426)(614.50726139,212.70307543)
\curveto(614.92405952,212.70307426)(615.27265731,212.86221674)(615.55305582,213.18050332)
\curveto(615.83723418,213.49878662)(615.97932567,213.96105761)(615.97933072,214.56731768)
\curveto(615.97932567,215.15841526)(615.83723418,215.61310804)(615.55305582,215.93139737)
\curveto(615.27265731,216.24967793)(614.92405952,216.4088204)(614.50726139,216.40882527)
\curveto(614.09045609,216.4088204)(613.73996374,216.24967793)(613.45578329,215.93139737)
\curveto(613.17159777,215.61310804)(613.02950628,215.15462615)(613.02950838,214.55595035)
}
}
{
\newrgbcolor{curcolor}{0 0 0}
\pscustom[linestyle=none,fillstyle=solid,fillcolor=curcolor]
{
\newpath
\moveto(622.85088314,211.53792402)
\lineto(622.85088314,212.44162682)
\curveto(622.63111015,212.11955187)(622.34124351,211.86568173)(621.98128233,211.68001566)
\curveto(621.62510238,211.4943493)(621.24808629,211.40151619)(620.85023292,211.40151605)
\curveto(620.44479572,211.40151619)(620.0810415,211.49056019)(619.75896916,211.66864833)
\curveto(619.43689339,211.8467362)(619.20386335,212.09681723)(619.05987832,212.41889216)
\curveto(618.91589125,212.74096533)(618.84389823,213.18618534)(618.84389903,213.75455353)
\lineto(618.84389903,217.57397668)
\lineto(620.44100901,217.57397668)
\lineto(620.44100901,214.80034796)
\curveto(620.44100661,213.95158485)(620.46942491,213.43058271)(620.52626399,213.23733998)
\curveto(620.58688721,213.04788295)(620.69487674,212.89631869)(620.85023292,212.78264674)
\curveto(621.00558348,212.67276141)(621.20261701,212.61781937)(621.44133412,212.61782045)
\curveto(621.71414639,212.61781937)(621.95854375,212.69170694)(622.17452696,212.8394834)
\curveto(622.39050189,212.99104636)(622.53827705,213.17671257)(622.61785286,213.39648261)
\curveto(622.69741952,213.62003803)(622.73720514,214.16377481)(622.73720983,215.02769458)
\lineto(622.73720983,217.57397668)
\lineto(624.33431981,217.57397668)
\lineto(624.33431981,211.53792402)
\lineto(622.85088314,211.53792402)
}
}
{
\newrgbcolor{curcolor}{0 0 0}
\pscustom[linestyle=none,fillstyle=solid,fillcolor=curcolor]
{
\newpath
\moveto(625.94847995,217.57397668)
\lineto(627.43760028,217.57397668)
\lineto(627.43760028,216.68732487)
\curveto(627.63084244,216.99044824)(627.89229078,217.23674016)(628.22194611,217.42620137)
\curveto(628.55159531,217.61565081)(628.91724408,217.71037847)(629.31889353,217.71038464)
\curveto(630.01987407,217.71037847)(630.61476379,217.43566825)(631.10356447,216.88625316)
\curveto(631.59235326,216.33682737)(631.83675062,215.57142787)(631.8367573,214.59005234)
\curveto(631.83675062,213.58214697)(631.5904587,212.79780193)(631.0978808,212.23701487)
\curveto(630.60529102,211.68001551)(630.00850675,211.40151619)(629.3075262,211.40151605)
\curveto(628.97408068,211.40151619)(628.67095216,211.46782555)(628.39813974,211.60044434)
\curveto(628.12910994,211.73306301)(627.84492695,211.96040939)(627.54558993,212.28248419)
\lineto(627.54558993,209.2417232)
\lineto(625.94847995,209.2417232)
\lineto(625.94847995,217.57397668)
\moveto(627.52853893,214.65825633)
\curveto(627.52853656,213.98000315)(627.66304984,213.47794654)(627.93207917,213.152085)
\curveto(628.20110296,212.83000933)(628.52886067,212.66897231)(628.91535329,212.66897344)
\curveto(629.28668197,212.66897231)(629.59549414,212.81674746)(629.84179075,213.11229934)
\curveto(630.08807799,213.41163717)(630.21122395,213.90043191)(630.211229,214.57868501)
\curveto(630.21122395,215.21146275)(630.08428888,215.68131195)(629.83042342,215.98823403)
\curveto(629.57654861,216.2951472)(629.26205277,216.44860602)(628.88693496,216.44861093)
\curveto(628.49665327,216.44860602)(628.17268466,216.29704176)(627.91502818,215.99391769)
\curveto(627.65736618,215.69457383)(627.52853656,215.24935382)(627.52853893,214.65825633)
}
}
{
\newrgbcolor{curcolor}{0 0 0}
\pscustom[linestyle=none,fillstyle=solid,fillcolor=curcolor]
{
\newpath
\moveto(632.54721465,213.26007464)
\lineto(634.15000829,213.50447225)
\curveto(634.21821033,213.19376355)(634.35651272,212.9569444)(634.56491586,212.79401408)
\curveto(634.77331443,212.63487035)(635.06507563,212.55529911)(635.44020034,212.55530013)
\curveto(635.85320978,212.55529911)(636.16391651,212.63108124)(636.37232146,212.78264674)
\curveto(636.5125143,212.88874048)(636.58261277,213.03083197)(636.58261708,213.20892165)
\curveto(636.58261277,213.33017139)(636.54472171,213.43058271)(636.46894377,213.51015592)
\curveto(636.38936834,213.58593607)(636.21128034,213.65603454)(635.93467923,213.72045153)
\curveto(634.64637936,214.00463234)(633.82982692,214.26418613)(633.48501944,214.49911369)
\curveto(633.00759081,214.82497389)(632.7688771,215.27777211)(632.7688776,215.85750972)
\curveto(632.7688771,216.3804021)(632.97538341,216.81993845)(633.38839713,217.1761201)
\curveto(633.80140862,217.53229047)(634.44176761,217.71037847)(635.30947603,217.71038464)
\curveto(636.13549821,217.71037847)(636.74933346,217.57586519)(637.15098362,217.3068444)
\curveto(637.55262403,217.03781207)(637.8292288,216.63995589)(637.98079877,216.11327467)
\lineto(636.47462744,215.83477506)
\curveto(636.41020843,216.06969537)(636.28706247,216.24967793)(636.10518919,216.37472328)
\curveto(635.92709735,216.49975895)(635.67133267,216.56227921)(635.33789436,216.56228423)
\curveto(634.91730048,216.56227921)(634.61606651,216.50354806)(634.43419156,216.38609061)
\curveto(634.31293799,216.30272542)(634.25231229,216.19473588)(634.25231427,216.06212168)
\curveto(634.25231229,215.94844396)(634.30535978,215.85182175)(634.4114569,215.77225474)
\curveto(634.55544081,215.66615553)(635.05181376,215.51648582)(635.90057723,215.32324518)
\curveto(636.75312256,215.12999696)(637.34801228,214.89317781)(637.68524817,214.612787)
\curveto(638.01868413,214.32860094)(638.18540481,213.93263932)(638.18541072,213.42490093)
\curveto(638.18540481,212.8716895)(637.95426932,212.39615664)(637.49200355,211.99830092)
\curveto(637.02972734,211.60044428)(636.34579362,211.40151619)(635.44020034,211.40151605)
\curveto(634.61796107,211.40151619)(633.96623475,211.56823687)(633.48501944,211.90167861)
\curveto(633.00759081,212.23511961)(632.69498953,212.68791784)(632.54721465,213.26007464)
}
}
{
\newrgbcolor{curcolor}{0 1 0.25098041}
\pscustom[linewidth=2.32802935,linecolor=curcolor]
{
\newpath
\moveto(686.58750578,114.33554061)
\lineto(761.08444272,114.33554061)
\lineto(761.08444272,160.89613202)
\lineto(712.19583105,160.89613202)
\lineto(712.19583105,170.2082503)
\lineto(686.58750578,170.2082503)
\lineto(686.58750578,114.33554061)
\closepath
}
}
{
\newrgbcolor{curcolor}{0 0 0}
\pscustom[linestyle=none,fillstyle=solid,fillcolor=curcolor]
{
\newpath
\moveto(700.02822191,141.84823355)
\lineto(700.02822191,142.80308934)
\lineto(705.54706101,145.13339215)
\lineto(705.54706101,144.11601604)
\lineto(701.17063865,142.31997778)
\lineto(705.54706101,140.50688852)
\lineto(705.54706101,139.48951241)
\lineto(700.02822191,141.84823355)
}
}
{
\newrgbcolor{curcolor}{0 0 0}
\pscustom[linestyle=none,fillstyle=solid,fillcolor=curcolor]
{
\newpath
\moveto(706.8258859,141.84823355)
\lineto(706.8258859,142.80308934)
\lineto(712.344725,145.13339215)
\lineto(712.344725,144.11601604)
\lineto(707.96830265,142.31997778)
\lineto(712.344725,140.50688852)
\lineto(712.344725,139.48951241)
\lineto(706.8258859,141.84823355)
}
}
{
\newrgbcolor{curcolor}{0 0 0}
\pscustom[linestyle=none,fillstyle=solid,fillcolor=curcolor]
{
\newpath
\moveto(717.69305432,138.9495642)
\curveto(717.31413897,138.6274894)(716.94849019,138.40014301)(716.5961069,138.26752435)
\curveto(716.24750549,138.13490556)(715.87238395,138.0685962)(715.47074115,138.06859606)
\curveto(714.80764503,138.0685962)(714.29801021,138.22963322)(713.94183516,138.55170762)
\curveto(713.58565819,138.87757043)(713.40757019,139.29247759)(713.40757061,139.79643034)
\curveto(713.40757019,140.09197905)(713.47387955,140.36100561)(713.6064989,140.60351083)
\curveto(713.74290611,140.84980035)(713.91909956,141.04683389)(714.13507978,141.19461203)
\curveto(714.35484681,141.34238419)(714.60113873,141.45416283)(714.87395628,141.52994829)
\curveto(715.07477704,141.58299245)(715.37790556,141.63414539)(715.78334275,141.68340725)
\curveto(716.60936516,141.78192054)(717.21751675,141.89938284)(717.60779934,142.03579451)
\curveto(717.61158383,142.17598762)(717.61347838,142.26503162)(717.613483,142.30292678)
\curveto(717.61347838,142.7197244)(717.51685616,143.01338015)(717.32361607,143.18389492)
\curveto(717.06216339,143.41502543)(716.67377997,143.53059318)(716.15846466,143.53059851)
\curveto(715.67724497,143.53059318)(715.32106896,143.44533829)(715.08993557,143.27483356)
\curveto(714.86258708,143.10810781)(714.69397184,142.81066295)(714.58408935,142.3824981)
\lineto(713.58376424,142.51890607)
\curveto(713.6747022,142.94707078)(713.8243719,143.29187947)(714.0327738,143.55333317)
\curveto(714.24117361,143.81856527)(714.54240758,144.02128247)(714.9364766,144.16148537)
\curveto(715.33054173,144.30546546)(715.78712906,144.37745848)(716.30623996,144.37746465)
\curveto(716.82155512,144.37745848)(717.24025139,144.31683278)(717.56233002,144.19558736)
\curveto(717.88439949,144.07432996)(718.12121865,143.92087115)(718.27278819,143.73521046)
\curveto(718.42434716,143.55332782)(718.53044215,143.32219233)(718.59107345,143.04180328)
\curveto(718.62516981,142.86749955)(718.64222079,142.55300371)(718.64222644,142.09831483)
\lineto(718.64222644,140.73423513)
\curveto(718.64222079,139.78316688)(718.66306087,139.18069895)(718.70474676,138.92682953)
\curveto(718.75021032,138.67674779)(718.83735977,138.43613952)(718.96619537,138.20500403)
\lineto(717.89766627,138.20500403)
\curveto(717.79156638,138.41719399)(717.72336247,138.66538047)(717.69305432,138.9495642)
\moveto(717.60779934,141.23439769)
\curveto(717.23646228,141.0828304)(716.67946363,140.95400078)(715.93680171,140.84790844)
\curveto(715.51620794,140.78728009)(715.21876309,140.71907618)(715.04446624,140.64329648)
\curveto(714.87016529,140.56751192)(714.73565201,140.45573328)(714.640926,140.30796023)
\curveto(714.54619669,140.16397208)(714.49883285,140.00293505)(714.49883437,139.82484867)
\curveto(714.49883285,139.55203138)(714.60113873,139.32468499)(714.8057523,139.14280882)
\curveto(715.01415134,138.96093077)(715.31727985,138.86999222)(715.71513876,138.86999288)
\curveto(716.10920311,138.86999222)(716.45969546,138.95524711)(716.76661686,139.12575782)
\curveto(717.07353071,139.3000558)(717.29898254,139.53687496)(717.44297304,139.836216)
\curveto(717.55285267,140.06734986)(717.60779472,140.40836944)(717.60779934,140.85927577)
\lineto(717.60779934,141.23439769)
}
}
{
\newrgbcolor{curcolor}{0 0 0}
\pscustom[linestyle=none,fillstyle=solid,fillcolor=curcolor]
{
\newpath
\moveto(720.23365216,135.89175221)
\lineto(720.23365216,144.24105668)
\lineto(721.16577329,144.24105668)
\lineto(721.16577329,143.45671086)
\curveto(721.38553976,143.76362323)(721.63372624,143.99286417)(721.91033345,144.14443437)
\curveto(722.18693578,144.2997818)(722.5222717,144.37745848)(722.91634223,144.37746465)
\curveto(723.43165726,144.37745848)(723.88635004,144.24483975)(724.28042192,143.97960807)
\curveto(724.67448418,143.71436485)(724.97192904,143.3392433)(725.17275739,142.85424232)
\curveto(725.37357433,142.37302115)(725.47398565,141.8444408)(725.47399166,141.26849968)
\curveto(725.47398565,140.65087226)(725.36220701,140.09387361)(725.1386554,139.59750205)
\curveto(724.91888155,139.10491682)(724.5968075,138.72600617)(724.17243228,138.46076897)
\curveto(723.75183676,138.19932037)(723.3085113,138.0685962)(722.84245458,138.06859606)
\curveto(722.50143162,138.0685962)(722.19451399,138.14058922)(721.92170078,138.28457534)
\curveto(721.65267177,138.42856131)(721.43100904,138.61043842)(721.25671193,138.83020722)
\lineto(721.25671193,135.89175221)
\lineto(720.23365216,135.89175221)
\moveto(721.16008962,141.18892836)
\curveto(721.16008793,140.41215855)(721.31733584,139.83810892)(721.63183385,139.46677775)
\curveto(721.94632752,139.09544405)(722.32713272,138.90977783)(722.77425059,138.90977854)
\curveto(723.22894006,138.90977783)(723.61732348,139.10112771)(723.939402,139.48382874)
\curveto(724.26526068,139.87031633)(724.42819226,140.4671006)(724.42819722,141.27418334)
\curveto(724.42819226,142.04336889)(724.26904979,142.61931307)(723.95076933,143.00201763)
\curveto(723.63626901,143.38471258)(723.25925291,143.57606246)(722.81971992,143.57606783)
\curveto(722.38396932,143.57606246)(721.99748046,143.37145071)(721.66025217,142.96223197)
\curveto(721.32680861,142.55679282)(721.16008793,141.96569221)(721.16008962,141.18892836)
}
}
{
\newrgbcolor{curcolor}{0 0 0}
\pscustom[linestyle=none,fillstyle=solid,fillcolor=curcolor]
{
\newpath
\moveto(726.71303101,135.89175221)
\lineto(726.71303101,144.24105668)
\lineto(727.64515213,144.24105668)
\lineto(727.64515213,143.45671086)
\curveto(727.86491861,143.76362323)(728.11310508,143.99286417)(728.3897123,144.14443437)
\curveto(728.66631463,144.2997818)(729.00165055,144.37745848)(729.39572108,144.37746465)
\curveto(729.91103611,144.37745848)(730.36572889,144.24483975)(730.75980077,143.97960807)
\curveto(731.15386303,143.71436485)(731.45130789,143.3392433)(731.65213624,142.85424232)
\curveto(731.85295318,142.37302115)(731.9533645,141.8444408)(731.95337051,141.26849968)
\curveto(731.9533645,140.65087226)(731.84158586,140.09387361)(731.61803425,139.59750205)
\curveto(731.3982604,139.10491682)(731.07618635,138.72600617)(730.65181113,138.46076897)
\curveto(730.23121561,138.19932037)(729.78789015,138.0685962)(729.32183343,138.06859606)
\curveto(728.98081047,138.0685962)(728.67389284,138.14058922)(728.40107963,138.28457534)
\curveto(728.13205062,138.42856131)(727.91038789,138.61043842)(727.73609078,138.83020722)
\lineto(727.73609078,135.89175221)
\lineto(726.71303101,135.89175221)
\moveto(727.63946847,141.18892836)
\curveto(727.63946678,140.41215855)(727.79671469,139.83810892)(728.1112127,139.46677775)
\curveto(728.42570637,139.09544405)(728.80651157,138.90977783)(729.25362944,138.90977854)
\curveto(729.70831891,138.90977783)(730.09670233,139.10112771)(730.41878085,139.48382874)
\curveto(730.74463953,139.87031633)(730.90757111,140.4671006)(730.90757607,141.27418334)
\curveto(730.90757111,142.04336889)(730.74842864,142.61931307)(730.43014818,143.00201763)
\curveto(730.11564786,143.38471258)(729.73863176,143.57606246)(729.29909877,143.57606783)
\curveto(728.86334817,143.57606246)(728.47685931,143.37145071)(728.13963102,142.96223197)
\curveto(727.80618746,142.55679282)(727.63946678,141.96569221)(727.63946847,141.18892836)
}
}
{
\newrgbcolor{curcolor}{0 0 0}
\pscustom[linestyle=none,fillstyle=solid,fillcolor=curcolor]
{
\newpath
\moveto(738.58052466,141.84823355)
\lineto(733.06168555,139.48951241)
\lineto(733.06168555,140.50688852)
\lineto(737.43242425,142.31997778)
\lineto(733.06168555,144.11601604)
\lineto(733.06168555,145.13339215)
\lineto(738.58052466,142.80308934)
\lineto(738.58052466,141.84823355)
}
}
{
\newrgbcolor{curcolor}{0 0 0}
\pscustom[linestyle=none,fillstyle=solid,fillcolor=curcolor]
{
\newpath
\moveto(745.37818732,141.84823355)
\lineto(739.85934821,139.48951241)
\lineto(739.85934821,140.50688852)
\lineto(744.23008691,142.31997778)
\lineto(739.85934821,144.11601604)
\lineto(739.85934821,145.13339215)
\lineto(745.37818732,142.80308934)
\lineto(745.37818732,141.84823355)
}
}
{
\newrgbcolor{curcolor}{0 0 0}
\pscustom[linestyle=none,fillstyle=solid,fillcolor=curcolor]
{
\newpath
\moveto(709.97920994,126.7808268)
\lineto(709.97920994,125.50768576)
\lineto(708.88794619,125.50768576)
\lineto(708.88794619,123.07507696)
\curveto(708.88794367,122.58249079)(708.89741644,122.2945187)(708.91636451,122.21115982)
\curveto(708.93909661,122.13158712)(708.98646044,122.06527776)(709.05845615,122.01223154)
\curveto(709.1342356,121.95918278)(709.22517415,121.93265903)(709.33127209,121.93266022)
\curveto(709.47904428,121.93265903)(709.6931288,121.98381197)(709.97352628,122.08611919)
\lineto(710.10993425,120.84708013)
\curveto(709.73859808,120.68793755)(709.31800726,120.60836632)(708.84816053,120.60836618)
\curveto(708.56018596,120.60836632)(708.30063217,120.65573015)(708.06949837,120.75045782)
\curveto(707.83836118,120.84897458)(707.66785139,120.97401509)(707.55796848,121.12557973)
\curveto(707.45187232,121.28093272)(707.37798474,121.48933357)(707.33630553,121.75078293)
\curveto(707.30220261,121.93644814)(707.28515163,122.31156968)(707.28515254,122.87614868)
\lineto(707.28515254,125.50768576)
\lineto(706.55195971,125.50768576)
\lineto(706.55195971,126.7808268)
\lineto(707.28515254,126.7808268)
\lineto(707.28515254,127.9800802)
\lineto(708.88794619,128.91220133)
\lineto(708.88794619,126.7808268)
\lineto(709.97920994,126.7808268)
}
}
{
\newrgbcolor{curcolor}{0 0 0}
\pscustom[linestyle=none,fillstyle=solid,fillcolor=curcolor]
{
\newpath
\moveto(714.59434642,122.66585306)
\lineto(716.18577274,122.39872078)
\curveto(715.98115506,121.81519673)(715.65718646,121.36997672)(715.21386595,121.06305941)
\curveto(714.77432465,120.75993058)(714.22300966,120.60836632)(713.55991932,120.60836618)
\curveto(712.51033353,120.60836632)(711.7335667,120.95128045)(711.22961651,121.63710962)
\curveto(710.83175936,122.18652917)(710.63283127,122.87993565)(710.63283164,123.71733115)
\curveto(710.63283127,124.71765229)(710.89427962,125.50010278)(711.41717747,126.06468496)
\curveto(711.94007301,126.63304562)(712.60127209,126.9172286)(713.40077669,126.91723477)
\curveto(714.29879179,126.9172286)(715.0073547,126.61978374)(715.52646755,126.02489931)
\curveto(716.04556987,125.43379342)(716.29375635,124.52630242)(716.27102772,123.30242358)
\lineto(712.26972728,123.30242358)
\curveto(712.28109259,122.82878271)(712.40992221,122.45934483)(712.65621652,122.19410883)
\curveto(712.90250605,121.93265903)(713.20942368,121.80193486)(713.57697032,121.80193592)
\curveto(713.82704803,121.80193486)(714.03734344,121.87013878)(714.20785718,122.00654787)
\curveto(714.37836302,122.14295444)(714.50719264,122.36272262)(714.59434642,122.66585306)
\moveto(714.68528507,124.28001403)
\curveto(714.67391333,124.74228148)(714.55455647,125.09277383)(714.32721415,125.33149213)
\curveto(714.0998637,125.57399036)(713.82325892,125.69524176)(713.497399,125.69524671)
\curveto(713.14879797,125.69524176)(712.86082588,125.5683067)(712.63348186,125.31444113)
\curveto(712.4061331,125.06056643)(712.29435446,124.71575774)(712.2981456,124.28001403)
\lineto(714.68528507,124.28001403)
}
}
{
\newrgbcolor{curcolor}{0 0 0}
\pscustom[linestyle=none,fillstyle=solid,fillcolor=curcolor]
{
\newpath
\moveto(718.77184079,124.93931922)
\lineto(717.32250611,125.20076782)
\curveto(717.48543711,125.78428577)(717.76583099,126.2162439)(718.16368859,126.49664353)
\curveto(718.56154335,126.77703166)(719.15264396,126.9172286)(719.93699219,126.91723477)
\curveto(720.64934102,126.9172286)(721.17981592,126.83197371)(721.5284185,126.66146983)
\curveto(721.87701151,126.49474323)(722.12140888,126.28065871)(722.26161134,126.01921564)
\curveto(722.40559187,125.76155113)(722.47758489,125.28601826)(722.47759063,124.59261563)
\lineto(722.46053963,122.72837338)
\curveto(722.46053391,122.19789649)(722.4851631,121.80572396)(722.53442728,121.55185464)
\curveto(722.58746898,121.3017728)(722.68409119,121.03274624)(722.82429422,120.74477415)
\lineto(721.24423523,120.74477415)
\curveto(721.20255056,120.85086913)(721.15139762,121.00811705)(721.09077627,121.21651838)
\curveto(721.06424818,121.31124557)(721.04530264,121.37376583)(721.03393961,121.40407934)
\curveto(720.76111966,121.13884123)(720.46935846,120.93991314)(720.15865514,120.80729447)
\curveto(719.847945,120.67467568)(719.51639818,120.60836632)(719.1640137,120.60836618)
\curveto(718.54259782,120.60836632)(718.05190853,120.77698156)(717.69194436,121.1142124)
\curveto(717.3357674,121.45144251)(717.1576794,121.87771699)(717.15767981,122.39303712)
\curveto(717.1576794,122.73405505)(717.23914519,123.03718357)(717.40207743,123.30242358)
\curveto(717.56500834,123.57144758)(717.79235473,123.77605933)(718.08411727,123.91625944)
\curveto(718.37966624,124.06024232)(718.80404616,124.18528283)(719.35725832,124.29138136)
\curveto(720.10370968,124.43157475)(720.62092272,124.56229893)(720.90889898,124.68355427)
\lineto(720.90889898,124.8426969)
\curveto(720.90889481,125.14961043)(720.83311268,125.36748405)(720.68155236,125.49631842)
\curveto(720.52998416,125.6289324)(720.24390662,125.69524176)(719.82331888,125.69524671)
\curveto(719.53913282,125.69524176)(719.31747009,125.63840517)(719.15833003,125.52473675)
\curveto(718.99918515,125.41484788)(718.87035553,125.2197089)(718.77184079,124.93931922)
\moveto(720.90889898,123.6434435)
\curveto(720.70428306,123.57523669)(720.38031446,123.4937709)(719.93699219,123.39904589)
\curveto(719.49366354,123.30431558)(719.2037969,123.21148247)(719.06739139,123.12054629)
\curveto(718.85898821,122.97276876)(718.75478778,122.78520799)(718.75478979,122.55786341)
\curveto(718.75478778,122.33430432)(718.83814812,122.14105989)(719.00487107,121.97812954)
\curveto(719.17158949,121.81519673)(719.38377945,121.73373094)(719.64144159,121.73373193)
\curveto(719.92941079,121.73373094)(720.20412101,121.8284586)(720.46557308,122.0179152)
\curveto(720.65881378,122.16189997)(720.78574885,122.33809342)(720.84637866,122.54649608)
\curveto(720.88805472,122.68290211)(720.90889481,122.94245591)(720.90889898,123.32515824)
\lineto(720.90889898,123.6434435)
}
}
{
\newrgbcolor{curcolor}{0 0 0}
\pscustom[linestyle=none,fillstyle=solid,fillcolor=curcolor]
{
\newpath
\moveto(723.93829115,126.7808268)
\lineto(725.41036049,126.7808268)
\lineto(725.41036049,125.95669532)
\curveto(725.9370441,126.5970491)(726.56414122,126.9172286)(727.29165374,126.91723477)
\curveto(727.67813853,126.9172286)(728.01347445,126.83765737)(728.29766251,126.67852083)
\curveto(728.58184042,126.51937242)(728.81487047,126.27876416)(728.99675336,125.95669532)
\curveto(729.26198503,126.27876416)(729.54806257,126.51937242)(729.85498683,126.67852083)
\curveto(730.16189782,126.83765737)(730.48965553,126.9172286)(730.83826094,126.91723477)
\curveto(731.28157879,126.9172286)(731.65670033,126.82629005)(731.96362669,126.64441884)
\curveto(732.27053558,126.46632493)(732.49977652,126.20298203)(732.65135021,125.85438934)
\curveto(732.76122487,125.59672499)(732.81616691,125.17992328)(732.8161765,124.60398296)
\lineto(732.8161765,120.74477415)
\lineto(731.21906653,120.74477415)
\lineto(731.21906653,124.19475905)
\curveto(731.21905853,124.79343442)(731.16411649,125.17992328)(731.05424023,125.35422679)
\curveto(730.90645725,125.58156857)(730.67911086,125.69524176)(730.37220038,125.69524671)
\curveto(730.14863595,125.69524176)(729.93834054,125.62703785)(729.74131352,125.49063476)
\curveto(729.54427347,125.35422218)(729.40218197,125.15339954)(729.31503862,124.88816623)
\curveto(729.22788308,124.62671374)(729.18430835,124.21180658)(729.18431431,123.6434435)
\lineto(729.18431431,120.74477415)
\lineto(727.58720434,120.74477415)
\lineto(727.58720434,124.05266741)
\curveto(727.58719997,124.63997561)(727.55878167,125.01888626)(727.50194936,125.18940049)
\curveto(727.44510848,125.35990584)(727.35606448,125.48684091)(727.23481708,125.57020607)
\curveto(727.11735077,125.65356159)(726.95631374,125.69524176)(726.75170552,125.69524671)
\curveto(726.50541007,125.69524176)(726.28374734,125.6289324)(726.08671667,125.49631842)
\curveto(725.88968027,125.36369495)(725.74758878,125.17234507)(725.66044177,124.92226822)
\curveto(725.57707899,124.67218301)(725.53539882,124.25727586)(725.53540113,123.6775455)
\lineto(725.53540113,120.74477415)
\lineto(723.93829115,120.74477415)
\lineto(723.93829115,126.7808268)
}
}
{
\newrgbcolor{curcolor}{0 0 0}
\pscustom[linestyle=none,fillstyle=solid,fillcolor=curcolor]
{
\newpath
\moveto(733.83923622,122.46692477)
\lineto(735.44202986,122.71132238)
\curveto(735.5102319,122.40061368)(735.64853429,122.16379453)(735.85693743,122.0008642)
\curveto(736.065336,121.84172048)(736.3570972,121.76214924)(736.7322219,121.76215026)
\curveto(737.14523134,121.76214924)(737.45593808,121.83793137)(737.66434303,121.98949687)
\curveto(737.80453587,122.09559061)(737.87463434,122.2376821)(737.87463865,122.41577178)
\curveto(737.87463434,122.53702151)(737.83674328,122.63743284)(737.76096534,122.71700604)
\curveto(737.68138991,122.7927862)(737.50330191,122.86288467)(737.22670079,122.92730166)
\curveto(735.93840093,123.21148247)(735.12184849,123.47103626)(734.77704101,123.70596382)
\curveto(734.29961238,124.03182402)(734.06089867,124.48462224)(734.06089917,125.06435985)
\curveto(734.06089867,125.58725223)(734.26740498,126.02678858)(734.6804187,126.38297023)
\curveto(735.09343019,126.7391406)(735.73378918,126.9172286)(736.6014976,126.91723477)
\curveto(737.42751978,126.9172286)(738.04135503,126.78271532)(738.44300519,126.51369453)
\curveto(738.8446456,126.2446622)(739.12125037,125.84680602)(739.27282034,125.3201248)
\lineto(737.76664901,125.04162519)
\curveto(737.70223,125.2765455)(737.57908404,125.45652806)(737.39721076,125.58157341)
\curveto(737.21911892,125.70660908)(736.96335423,125.76912934)(736.62991593,125.76913436)
\curveto(736.20932204,125.76912934)(735.90808808,125.71039819)(735.72621313,125.59294074)
\curveto(735.60495956,125.50957555)(735.54433386,125.40158601)(735.54433584,125.26897181)
\curveto(735.54433386,125.15529409)(735.59738135,125.05867187)(735.70347847,124.97910487)
\curveto(735.84746238,124.87300566)(736.34383532,124.72333595)(737.1925988,124.53009531)
\curveto(738.04514413,124.33684709)(738.64003385,124.10002794)(738.97726974,123.81963713)
\curveto(739.3107057,123.53545107)(739.47742638,123.13948944)(739.47743229,122.63175106)
\curveto(739.47742638,122.07853963)(739.24629089,121.60300677)(738.78402511,121.20515105)
\curveto(738.32174891,120.80729441)(737.63781519,120.60836632)(736.7322219,120.60836618)
\curveto(735.90998263,120.60836632)(735.25825632,120.775087)(734.77704101,121.10852874)
\curveto(734.29961238,121.44196974)(733.9870111,121.89476797)(733.83923622,122.46692477)
}
}
{
\newrgbcolor{curcolor}{0 1 0.25098041}
\pscustom[linewidth=2.32802935,linecolor=curcolor]
{
\newpath
\moveto(464.26070507,205.12868455)
\lineto(538.75764202,205.12868455)
\lineto(538.75764202,251.68927596)
\lineto(488.70501091,251.68927596)
\lineto(488.70501091,261.00139424)
\lineto(464.26070507,261.00139424)
\lineto(464.26070507,205.12868455)
\closepath
}
}
{
\newrgbcolor{curcolor}{0 0 0}
\pscustom[linestyle=none,fillstyle=solid,fillcolor=curcolor]
{
\newpath
\moveto(477.7014237,232.64138342)
\lineto(477.7014237,233.59623921)
\lineto(483.2202628,235.92654202)
\lineto(483.2202628,234.90916591)
\lineto(478.84384044,233.11312765)
\lineto(483.2202628,231.30003839)
\lineto(483.2202628,230.28266228)
\lineto(477.7014237,232.64138342)
}
}
{
\newrgbcolor{curcolor}{0 0 0}
\pscustom[linestyle=none,fillstyle=solid,fillcolor=curcolor]
{
\newpath
\moveto(484.49908769,232.64138342)
\lineto(484.49908769,233.59623921)
\lineto(490.01792679,235.92654202)
\lineto(490.01792679,234.90916591)
\lineto(485.64150444,233.11312765)
\lineto(490.01792679,231.30003839)
\lineto(490.01792679,230.28266228)
\lineto(484.49908769,232.64138342)
}
}
{
\newrgbcolor{curcolor}{0 0 0}
\pscustom[linestyle=none,fillstyle=solid,fillcolor=curcolor]
{
\newpath
\moveto(495.36625611,229.74271407)
\curveto(494.98734076,229.42063927)(494.62169198,229.19329288)(494.26930869,229.06067422)
\curveto(493.92070728,228.92805543)(493.54558574,228.86174607)(493.14394294,228.86174593)
\curveto(492.48084682,228.86174607)(491.971212,229.02278309)(491.61503695,229.34485749)
\curveto(491.25885998,229.6707203)(491.08077198,230.08562746)(491.0807724,230.58958021)
\curveto(491.08077198,230.88512892)(491.14708134,231.15415548)(491.27970069,231.3966607)
\curveto(491.4161079,231.64295022)(491.59230135,231.83998376)(491.80828157,231.9877619)
\curveto(492.0280486,232.13553406)(492.27434052,232.2473127)(492.54715807,232.32309816)
\curveto(492.74797883,232.37614232)(493.05110735,232.42729526)(493.45654453,232.47655712)
\curveto(494.28256695,232.57507041)(494.89071854,232.69253271)(495.28100113,232.82894438)
\curveto(495.28478561,232.96913749)(495.28668017,233.05818149)(495.28668479,233.09607665)
\curveto(495.28668017,233.51287427)(495.19005795,233.80653002)(494.99681786,233.97704479)
\curveto(494.73536517,234.2081753)(494.34698176,234.32374305)(493.83166645,234.32374838)
\curveto(493.35044676,234.32374305)(492.99427075,234.23848816)(492.76313736,234.06798343)
\curveto(492.53578886,233.90125768)(492.36717363,233.60381282)(492.25729114,233.17564797)
\lineto(491.25696603,233.31205594)
\curveto(491.34790398,233.74022065)(491.49757369,234.08502934)(491.70597559,234.34648304)
\curveto(491.9143754,234.61171514)(492.21560937,234.81443234)(492.60967839,234.95463524)
\curveto(493.00374351,235.09861533)(493.46033085,235.17060835)(493.97944175,235.17061452)
\curveto(494.49475691,235.17060835)(494.91345318,235.10998265)(495.2355318,234.98873723)
\curveto(495.55760128,234.86747983)(495.79442043,234.71402102)(495.94598998,234.52836033)
\curveto(496.09754895,234.34647769)(496.20364393,234.1153422)(496.26427524,233.83495315)
\curveto(496.2983716,233.66064942)(496.31542258,233.34615358)(496.31542823,232.8914647)
\lineto(496.31542823,231.527385)
\curveto(496.31542258,230.57631675)(496.33626266,229.97384882)(496.37794855,229.7199794)
\curveto(496.42341211,229.46989766)(496.51056156,229.22928939)(496.63939716,228.9981539)
\lineto(495.57086806,228.9981539)
\curveto(495.46476817,229.21034386)(495.39656425,229.45853034)(495.36625611,229.74271407)
\moveto(495.28100113,232.02754756)
\curveto(494.90966407,231.87598027)(494.35266542,231.74715065)(493.6100035,231.64105831)
\curveto(493.18940973,231.58042996)(492.89196487,231.51222605)(492.71766803,231.43644636)
\curveto(492.54336708,231.36066179)(492.4088538,231.24888315)(492.31412779,231.1011101)
\curveto(492.21939847,230.95712195)(492.17203464,230.79608492)(492.17203615,230.61799854)
\curveto(492.17203464,230.34518125)(492.27434052,230.11783486)(492.47895409,229.93595869)
\curveto(492.68735312,229.75408064)(492.99048164,229.66314209)(493.38834055,229.66314275)
\curveto(493.7824049,229.66314209)(494.13289724,229.74839698)(494.43981865,229.91890769)
\curveto(494.74673249,230.09320567)(494.97218433,230.33002483)(495.11617483,230.62936587)
\curveto(495.22605446,230.86049973)(495.28099651,231.20151932)(495.28100113,231.65242564)
\lineto(495.28100113,232.02754756)
}
}
{
\newrgbcolor{curcolor}{0 0 0}
\pscustom[linestyle=none,fillstyle=solid,fillcolor=curcolor]
{
\newpath
\moveto(497.90685395,226.68490208)
\lineto(497.90685395,235.03420655)
\lineto(498.83897507,235.03420655)
\lineto(498.83897507,234.24986073)
\curveto(499.05874155,234.5567731)(499.30692802,234.78601404)(499.58353524,234.93758424)
\curveto(499.86013757,235.09293167)(500.19547349,235.17060835)(500.58954402,235.17061452)
\curveto(501.10485905,235.17060835)(501.55955182,235.03798962)(501.95362371,234.77275794)
\curveto(502.34768597,234.50751472)(502.64513083,234.13239317)(502.84595918,233.6473922)
\curveto(503.04677612,233.16617102)(503.14718744,232.63759067)(503.14719345,232.06164955)
\curveto(503.14718744,231.44402213)(503.0354088,230.88702348)(502.81185719,230.39065192)
\curveto(502.59208334,229.89806669)(502.27000929,229.51915604)(501.84563407,229.25391884)
\curveto(501.42503854,228.99247024)(500.98171309,228.86174607)(500.51565637,228.86174593)
\curveto(500.17463341,228.86174607)(499.86771578,228.93373909)(499.59490257,229.07772521)
\curveto(499.32587356,229.22171118)(499.10421083,229.40358829)(498.92991372,229.62335709)
\lineto(498.92991372,226.68490208)
\lineto(497.90685395,226.68490208)
\moveto(498.83329141,231.98207823)
\curveto(498.83328971,231.20530842)(498.99053763,230.63125879)(499.30503564,230.25992762)
\curveto(499.61952931,229.88859392)(500.00033451,229.7029277)(500.44745238,229.70292841)
\curveto(500.90214185,229.7029277)(501.29052526,229.89427758)(501.61260379,230.27697861)
\curveto(501.93846247,230.6634662)(502.10139405,231.26025047)(502.10139901,232.06733321)
\curveto(502.10139405,232.83651876)(501.94225158,233.41246294)(501.62397112,233.7951675)
\curveto(501.3094708,234.17786245)(500.9324547,234.36921233)(500.4929217,234.3692177)
\curveto(500.05717111,234.36921233)(499.67068225,234.16460058)(499.33345396,233.75538184)
\curveto(499.0000104,233.34994269)(498.83328971,232.75884208)(498.83329141,231.98207823)
}
}
{
\newrgbcolor{curcolor}{0 0 0}
\pscustom[linestyle=none,fillstyle=solid,fillcolor=curcolor]
{
\newpath
\moveto(504.3862328,226.68490208)
\lineto(504.3862328,235.03420655)
\lineto(505.31835392,235.03420655)
\lineto(505.31835392,234.24986073)
\curveto(505.5381204,234.5567731)(505.78630687,234.78601404)(506.06291409,234.93758424)
\curveto(506.33951642,235.09293167)(506.67485234,235.17060835)(507.06892287,235.17061452)
\curveto(507.5842379,235.17060835)(508.03893067,235.03798962)(508.43300256,234.77275794)
\curveto(508.82706482,234.50751472)(509.12450968,234.13239317)(509.32533803,233.6473922)
\curveto(509.52615497,233.16617102)(509.62656629,232.63759067)(509.62657229,232.06164955)
\curveto(509.62656629,231.44402213)(509.51478765,230.88702348)(509.29123604,230.39065192)
\curveto(509.07146219,229.89806669)(508.74938814,229.51915604)(508.32501292,229.25391884)
\curveto(507.90441739,228.99247024)(507.46109194,228.86174607)(506.99503522,228.86174593)
\curveto(506.65401226,228.86174607)(506.34709463,228.93373909)(506.07428142,229.07772521)
\curveto(505.80525241,229.22171118)(505.58358968,229.40358829)(505.40929257,229.62335709)
\lineto(505.40929257,226.68490208)
\lineto(504.3862328,226.68490208)
\moveto(505.31267026,231.98207823)
\curveto(505.31266856,231.20530842)(505.46991648,230.63125879)(505.78441449,230.25992762)
\curveto(506.09890816,229.88859392)(506.47971336,229.7029277)(506.92683123,229.70292841)
\curveto(507.3815207,229.7029277)(507.76990411,229.89427758)(508.09198264,230.27697861)
\curveto(508.41784132,230.6634662)(508.5807729,231.26025047)(508.58077786,232.06733321)
\curveto(508.5807729,232.83651876)(508.42163043,233.41246294)(508.10334997,233.7951675)
\curveto(507.78884965,234.17786245)(507.41183355,234.36921233)(506.97230055,234.3692177)
\curveto(506.53654996,234.36921233)(506.15006109,234.16460058)(505.81283281,233.75538184)
\curveto(505.47938925,233.34994269)(505.31266856,232.75884208)(505.31267026,231.98207823)
}
}
{
\newrgbcolor{curcolor}{0 0 0}
\pscustom[linestyle=none,fillstyle=solid,fillcolor=curcolor]
{
\newpath
\moveto(516.25372644,232.64138342)
\lineto(510.73488734,230.28266228)
\lineto(510.73488734,231.30003839)
\lineto(515.10562603,233.11312765)
\lineto(510.73488734,234.90916591)
\lineto(510.73488734,235.92654202)
\lineto(516.25372644,233.59623921)
\lineto(516.25372644,232.64138342)
}
}
{
\newrgbcolor{curcolor}{0 0 0}
\pscustom[linestyle=none,fillstyle=solid,fillcolor=curcolor]
{
\newpath
\moveto(523.0513891,232.64138342)
\lineto(517.53255,230.28266228)
\lineto(517.53255,231.30003839)
\lineto(521.90328869,233.11312765)
\lineto(517.53255,234.90916591)
\lineto(517.53255,235.92654202)
\lineto(523.0513891,233.59623921)
\lineto(523.0513891,232.64138342)
}
}
{
\newrgbcolor{curcolor}{0 0 0}
\pscustom[linestyle=none,fillstyle=solid,fillcolor=curcolor]
{
\newpath
\moveto(488.98352173,215.78930574)
\lineto(487.40914641,215.50512247)
\curveto(487.35609439,215.81961434)(487.23484299,216.0564335)(487.04539182,216.21558064)
\curveto(486.85972145,216.37471844)(486.61721863,216.45428968)(486.31788265,216.45429459)
\curveto(485.92002304,216.45428968)(485.6017381,216.31598729)(485.36302687,216.03938702)
\curveto(485.12809979,215.76656685)(485.01063749,215.30808497)(485.01063961,214.66393999)
\curveto(485.01063749,213.94779574)(485.12999434,213.44195003)(485.36871053,213.14640133)
\curveto(485.61121086,212.85084942)(485.93517947,212.70307426)(486.34061731,212.70307543)
\curveto(486.64374238,212.70307426)(486.89192885,212.78832916)(487.08517748,212.95884037)
\curveto(487.27841771,213.13313785)(487.41482555,213.43058271)(487.49440139,213.85117584)
\lineto(489.06309304,213.58404357)
\curveto(488.90015528,212.86411129)(488.587554,212.32037451)(488.12528825,211.9528316)
\curveto(487.66301202,211.58528785)(487.04349311,211.40151619)(486.26672966,211.40151605)
\curveto(485.38386447,211.40151619)(484.67909067,211.68001551)(484.15240614,212.23701487)
\curveto(483.62950818,212.79401282)(483.36805983,213.56509599)(483.36806031,214.55026668)
\curveto(483.36805983,215.54679867)(483.63140273,216.32167095)(484.1580898,216.87488583)
\curveto(484.68477433,217.43187915)(485.39712635,217.71037847)(486.29514799,217.71038464)
\curveto(487.03023124,217.71037847)(487.61375364,217.551236)(488.04571693,217.23295675)
\curveto(488.48145902,216.91845522)(488.7940603,216.4372387)(488.98352173,215.78930574)
}
}
{
\newrgbcolor{curcolor}{0 0 0}
\pscustom[linestyle=none,fillstyle=solid,fillcolor=curcolor]
{
\newpath
\moveto(491.3933957,215.73246909)
\lineto(489.94406103,215.99391769)
\curveto(490.10699203,216.57743564)(490.38738591,217.00939377)(490.78524351,217.28979341)
\curveto(491.18309827,217.57018153)(491.77419888,217.71037847)(492.55854711,217.71038464)
\curveto(493.27089593,217.71037847)(493.80137084,217.62512358)(494.14997342,217.4546197)
\curveto(494.49856643,217.2878931)(494.7429638,217.07380858)(494.88316626,216.81236551)
\curveto(495.02714679,216.554701)(495.09913981,216.07916813)(495.09914554,215.3857655)
\lineto(495.08209455,213.52152325)
\curveto(495.08208883,212.99104636)(495.10671802,212.59887384)(495.1559822,212.34500451)
\curveto(495.2090239,212.09492267)(495.30564611,211.82589611)(495.44584913,211.53792402)
\lineto(493.86579015,211.53792402)
\curveto(493.82410548,211.644019)(493.77295254,211.80126692)(493.71233119,212.00966825)
\curveto(493.68580309,212.10439544)(493.66685756,212.1669157)(493.65549453,212.19722921)
\curveto(493.38267458,211.9319911)(493.09091338,211.73306301)(492.78021006,211.60044434)
\curveto(492.46949992,211.46782555)(492.1379531,211.40151619)(491.78556862,211.40151605)
\curveto(491.16415273,211.40151619)(490.67346345,211.57013143)(490.31349928,211.90736227)
\curveto(489.95732232,212.24459238)(489.77923432,212.67086686)(489.77923473,213.18618699)
\curveto(489.77923432,213.52720492)(489.86070011,213.83033344)(490.02363234,214.09557345)
\curveto(490.18656326,214.36459745)(490.41390965,214.5692092)(490.70567219,214.70940931)
\curveto(491.00122116,214.85339219)(491.42560108,214.9784327)(491.97881324,215.08453123)
\curveto(492.7252646,215.22472462)(493.24247764,215.3554488)(493.53045389,215.47670414)
\lineto(493.53045389,215.63584677)
\curveto(493.53044973,215.9427603)(493.4546676,216.16063392)(493.30310728,216.2894683)
\curveto(493.15153908,216.42208227)(492.86546154,216.48839163)(492.4448738,216.48839658)
\curveto(492.16068774,216.48839163)(491.93902501,216.43155504)(491.77988495,216.31788662)
\curveto(491.62074006,216.20799775)(491.49191044,216.01285877)(491.3933957,215.73246909)
\moveto(493.53045389,214.43659338)
\curveto(493.32583798,214.36838656)(493.00186938,214.28692077)(492.55854711,214.19219576)
\curveto(492.11521846,214.09746545)(491.82535181,214.00463234)(491.68894631,213.91369616)
\curveto(491.48054312,213.76591863)(491.3763427,213.57835786)(491.37634471,213.35101328)
\curveto(491.3763427,213.12745419)(491.45970304,212.93420976)(491.62642599,212.77127941)
\curveto(491.79314441,212.6083466)(492.00533437,212.52688081)(492.26299651,212.5268818)
\curveto(492.5509657,212.52688081)(492.82567592,212.62160847)(493.08712799,212.81106507)
\curveto(493.2803687,212.95504984)(493.40730377,213.1312433)(493.46793358,213.33964595)
\curveto(493.50960964,213.47605198)(493.53044973,213.73560578)(493.53045389,214.11830811)
\lineto(493.53045389,214.43659338)
}
}
{
\newrgbcolor{curcolor}{0 0 0}
\pscustom[linestyle=none,fillstyle=solid,fillcolor=curcolor]
{
\newpath
\moveto(498.20811037,211.53792402)
\lineto(496.61100039,211.53792402)
\lineto(496.61100039,217.57397668)
\lineto(498.09443706,217.57397668)
\lineto(498.09443706,216.7157432)
\curveto(498.34830494,217.12117242)(498.57565133,217.38830442)(498.77647691,217.51714002)
\curveto(498.98108573,217.64596366)(499.21222122,217.71037847)(499.46988409,217.71038464)
\curveto(499.83363468,217.71037847)(500.18412703,217.60996715)(500.52136219,217.40915038)
\lineto(500.0268833,216.01665236)
\curveto(499.75785255,216.19094678)(499.50777153,216.27809622)(499.27663946,216.27810096)
\curveto(499.05307875,216.27809622)(498.86362342,216.21557597)(498.70827292,216.09054001)
\curveto(498.55291669,215.96928405)(498.42977073,215.74762132)(498.33883467,215.42555115)
\curveto(498.25168273,215.10347322)(498.208108,214.42901226)(498.20811037,213.40216627)
\lineto(498.20811037,211.53792402)
}
}
{
\newrgbcolor{curcolor}{0 0 0}
\pscustom[linestyle=none,fillstyle=solid,fillcolor=curcolor]
{
\newpath
\moveto(504.69885619,213.45900293)
\lineto(506.29028251,213.19187065)
\curveto(506.08566483,212.6083466)(505.76169623,212.16312659)(505.31837572,211.85620928)
\curveto(504.87883442,211.55308045)(504.32751943,211.40151619)(503.66442909,211.40151605)
\curveto(502.6148433,211.40151619)(501.83807647,211.74443032)(501.33412628,212.43025949)
\curveto(500.93626913,212.97967904)(500.73734104,213.67308552)(500.73734141,214.51048103)
\curveto(500.73734104,215.51080216)(500.99878939,216.29325265)(501.52168724,216.85783483)
\curveto(502.04458278,217.42619549)(502.70578186,217.71037847)(503.50528646,217.71038464)
\curveto(504.40330156,217.71037847)(505.11186447,217.41293361)(505.63097732,216.81804918)
\curveto(506.15007964,216.22694329)(506.39826612,215.31945229)(506.37553749,214.09557345)
\lineto(502.37423705,214.09557345)
\curveto(502.38560236,213.62193258)(502.51443198,213.2524947)(502.76072629,212.9872587)
\curveto(503.00701582,212.7258089)(503.31393345,212.59508473)(503.68148009,212.59508579)
\curveto(503.9315578,212.59508473)(504.14185321,212.66328865)(504.31236695,212.79969774)
\curveto(504.48287279,212.93610431)(504.61170241,213.15587249)(504.69885619,213.45900293)
\moveto(504.78979484,215.0731639)
\curveto(504.7784231,215.53543135)(504.65906625,215.8859237)(504.43172392,216.124642)
\curveto(504.20437347,216.36714023)(503.9277687,216.48839163)(503.60190877,216.48839658)
\curveto(503.25330774,216.48839163)(502.96533565,216.36145657)(502.73799163,216.107591)
\curveto(502.51064287,215.8537163)(502.39886423,215.50890761)(502.40265537,215.0731639)
\lineto(504.78979484,215.0731639)
}
}
{
\newrgbcolor{curcolor}{0 0 0}
\pscustom[linestyle=none,fillstyle=solid,fillcolor=curcolor]
{
\newpath
\moveto(511.17823504,213.45900293)
\lineto(512.76966135,213.19187065)
\curveto(512.56504368,212.6083466)(512.24107508,212.16312659)(511.79775457,211.85620928)
\curveto(511.35821327,211.55308045)(510.80689828,211.40151619)(510.14380794,211.40151605)
\curveto(509.09422215,211.40151619)(508.31745532,211.74443032)(507.81350513,212.43025949)
\curveto(507.41564798,212.97967904)(507.21671989,213.67308552)(507.21672026,214.51048103)
\curveto(507.21671989,215.51080216)(507.47816824,216.29325265)(508.00106609,216.85783483)
\curveto(508.52396163,217.42619549)(509.18516071,217.71037847)(509.98466531,217.71038464)
\curveto(510.88268041,217.71037847)(511.59124332,217.41293361)(512.11035617,216.81804918)
\curveto(512.62945849,216.22694329)(512.87764497,215.31945229)(512.85491634,214.09557345)
\lineto(508.8536159,214.09557345)
\curveto(508.86498121,213.62193258)(508.99381083,213.2524947)(509.24010514,212.9872587)
\curveto(509.48639467,212.7258089)(509.7933123,212.59508473)(510.16085894,212.59508579)
\curveto(510.41093665,212.59508473)(510.62123206,212.66328865)(510.7917458,212.79969774)
\curveto(510.96225164,212.93610431)(511.09108126,213.15587249)(511.17823504,213.45900293)
\moveto(511.26917369,215.0731639)
\curveto(511.25780195,215.53543135)(511.13844509,215.8859237)(510.91110277,216.124642)
\curveto(510.68375232,216.36714023)(510.40714754,216.48839163)(510.08128762,216.48839658)
\curveto(509.73268659,216.48839163)(509.4447145,216.36145657)(509.21737048,216.107591)
\curveto(508.99002172,215.8537163)(508.87824308,215.50890761)(508.88203422,215.0731639)
\lineto(511.26917369,215.0731639)
}
}
{
\newrgbcolor{curcolor}{0 0 0}
\pscustom[linestyle=none,fillstyle=solid,fillcolor=curcolor]
{
\newpath
\moveto(515.69106566,211.53792402)
\lineto(514.09395569,211.53792402)
\lineto(514.09395569,217.57397668)
\lineto(515.57739236,217.57397668)
\lineto(515.57739236,216.7157432)
\curveto(515.83126024,217.12117242)(516.05860663,217.38830442)(516.2594322,217.51714002)
\curveto(516.46404102,217.64596366)(516.69517652,217.71037847)(516.95283938,217.71038464)
\curveto(517.31658998,217.71037847)(517.66708233,217.60996715)(518.00431748,217.40915038)
\lineto(517.50983859,216.01665236)
\curveto(517.24080785,216.19094678)(516.99072682,216.27809622)(516.75959476,216.27810096)
\curveto(516.53603404,216.27809622)(516.34657872,216.21557597)(516.19122822,216.09054001)
\curveto(516.03587199,215.96928405)(515.91272603,215.74762132)(515.82178997,215.42555115)
\curveto(515.73463802,215.10347322)(515.6910633,214.42901226)(515.69106566,213.40216627)
\lineto(515.69106566,211.53792402)
}
}
{
\newrgbcolor{curcolor}{0.50196081 0.50196081 0}
\pscustom[linewidth=2.32802935,linecolor=curcolor]
{
\newpath
\moveto(464.26070507,114.33554061)
\lineto(538.75764202,114.33554061)
\lineto(538.75764202,160.89613202)
\lineto(488.70501091,160.89613202)
\lineto(488.70501091,170.2082503)
\lineto(464.26070507,170.2082503)
\lineto(464.26070507,114.33554061)
\closepath
}
}
{
\newrgbcolor{curcolor}{0 0 0}
\pscustom[linestyle=none,fillstyle=solid,fillcolor=curcolor]
{
\newpath
\moveto(480.02946193,141.84823355)
\lineto(480.02946193,142.80308934)
\lineto(485.54830103,145.13339215)
\lineto(485.54830103,144.11601604)
\lineto(481.17187867,142.31997778)
\lineto(485.54830103,140.50688852)
\lineto(485.54830103,139.48951241)
\lineto(480.02946193,141.84823355)
}
}
{
\newrgbcolor{curcolor}{0 0 0}
\pscustom[linestyle=none,fillstyle=solid,fillcolor=curcolor]
{
\newpath
\moveto(486.82712592,141.84823355)
\lineto(486.82712592,142.80308934)
\lineto(492.34596502,145.13339215)
\lineto(492.34596502,144.11601604)
\lineto(487.96954267,142.31997778)
\lineto(492.34596502,140.50688852)
\lineto(492.34596502,139.48951241)
\lineto(486.82712592,141.84823355)
}
}
{
\newrgbcolor{curcolor}{0 0 0}
\pscustom[linestyle=none,fillstyle=solid,fillcolor=curcolor]
{
\newpath
\moveto(497.71134534,138.20500403)
\lineto(497.71134534,139.09165583)
\curveto(497.24149141,138.40961578)(496.60302697,138.0685962)(495.7959501,138.06859606)
\curveto(495.43977128,138.0685962)(495.10632991,138.13680011)(494.79562499,138.27320801)
\curveto(494.48870556,138.40961578)(494.25946461,138.58012557)(494.10790147,138.7847379)
\curveto(493.9601252,138.99313818)(493.85592477,139.24700831)(493.79529988,139.54634906)
\curveto(493.7536189,139.74717036)(493.73277881,140.06545531)(493.73277956,140.50120485)
\lineto(493.73277956,144.24105668)
\lineto(494.75583933,144.24105668)
\lineto(494.75583933,140.89337776)
\curveto(494.75583756,140.35911106)(494.77667765,139.99914595)(494.81835965,139.81348134)
\curveto(494.88277263,139.54445317)(495.01918046,139.33226321)(495.22758356,139.17691081)
\curveto(495.43598218,139.02534558)(495.69364142,138.94956345)(496.00056205,138.9495642)
\curveto(496.30747666,138.94956345)(496.59544876,139.02724013)(496.86447919,139.18259448)
\curveto(497.13350188,139.34173597)(497.3229572,139.55582049)(497.43284573,139.82484867)
\curveto(497.54651448,140.09766271)(497.60335108,140.49172979)(497.60335569,141.00705107)
\lineto(497.60335569,144.24105668)
\lineto(498.62641547,144.24105668)
\lineto(498.62641547,138.20500403)
\lineto(497.71134534,138.20500403)
}
}
{
\newrgbcolor{curcolor}{0 0 0}
\pscustom[linestyle=none,fillstyle=solid,fillcolor=curcolor]
{
\newpath
\moveto(502.46857268,139.12007416)
\lineto(502.61634798,138.21637136)
\curveto(502.32837274,138.15574564)(502.0707135,138.12543279)(501.84336949,138.12543271)
\curveto(501.47203468,138.12543279)(501.18406259,138.18416394)(500.97945235,138.30162634)
\curveto(500.77483909,138.41908854)(500.63085304,138.57254736)(500.54749378,138.76200324)
\curveto(500.46413236,138.95524711)(500.42245218,139.35878695)(500.42245314,139.97262397)
\lineto(500.42245314,143.44534353)
\lineto(499.67220931,143.44534353)
\lineto(499.67220931,144.24105668)
\lineto(500.42245314,144.24105668)
\lineto(500.42245314,145.73586068)
\lineto(501.43982925,146.34969654)
\lineto(501.43982925,144.24105668)
\lineto(502.46857268,144.24105668)
\lineto(502.46857268,143.44534353)
\lineto(501.43982925,143.44534353)
\lineto(501.43982925,139.91578731)
\curveto(501.43982727,139.6240244)(501.45687825,139.43646363)(501.49098223,139.35310444)
\curveto(501.52887128,139.26974295)(501.58760243,139.20343359)(501.66717586,139.15417615)
\curveto(501.750534,139.10491682)(501.8679963,139.08028762)(502.01956312,139.0802885)
\curveto(502.13323376,139.08028762)(502.28290346,139.0935495)(502.46857268,139.12007416)
}
}
{
\newrgbcolor{curcolor}{0 0 0}
\pscustom[linestyle=none,fillstyle=solid,fillcolor=curcolor]
{
\newpath
\moveto(503.46889856,145.36073877)
\lineto(503.46889856,146.5372575)
\lineto(504.49195833,146.5372575)
\lineto(504.49195833,145.36073877)
\lineto(503.46889856,145.36073877)
\moveto(503.46889856,138.20500403)
\lineto(503.46889856,144.24105668)
\lineto(504.49195833,144.24105668)
\lineto(504.49195833,138.20500403)
\lineto(503.46889856,138.20500403)
}
}
{
\newrgbcolor{curcolor}{0 0 0}
\pscustom[linestyle=none,fillstyle=solid,fillcolor=curcolor]
{
\newpath
\moveto(506.03223178,138.20500403)
\lineto(506.03223178,146.5372575)
\lineto(507.05529155,146.5372575)
\lineto(507.05529155,138.20500403)
\lineto(506.03223178,138.20500403)
}
}
{
\newrgbcolor{curcolor}{0 0 0}
\pscustom[linestyle=none,fillstyle=solid,fillcolor=curcolor]
{
\newpath
\moveto(514.03483277,141.84823355)
\lineto(508.51599367,139.48951241)
\lineto(508.51599367,140.50688852)
\lineto(512.88673236,142.31997778)
\lineto(508.51599367,144.11601604)
\lineto(508.51599367,145.13339215)
\lineto(514.03483277,142.80308934)
\lineto(514.03483277,141.84823355)
}
}
{
\newrgbcolor{curcolor}{0 0 0}
\pscustom[linestyle=none,fillstyle=solid,fillcolor=curcolor]
{
\newpath
\moveto(520.83249544,141.84823355)
\lineto(515.31365633,139.48951241)
\lineto(515.31365633,140.50688852)
\lineto(519.68439502,142.31997778)
\lineto(515.31365633,144.11601604)
\lineto(515.31365633,145.13339215)
\lineto(520.83249544,142.80308934)
\lineto(520.83249544,141.84823355)
}
}
{
\newrgbcolor{curcolor}{0 0 0}
\pscustom[linestyle=none,fillstyle=solid,fillcolor=curcolor]
{
\newpath
\moveto(474.2605762,120.34691757)
\lineto(476.08503279,120.12525462)
\curveto(476.11534313,119.91306528)(476.1854416,119.76718468)(476.29532841,119.68761239)
\curveto(476.44688995,119.57394025)(476.68560366,119.51710365)(477.01147025,119.51710243)
\curveto(477.42826853,119.51710365)(477.74086981,119.57962391)(477.94927505,119.70466338)
\curveto(478.08946761,119.78802477)(478.19556259,119.92253805)(478.26756031,120.10820363)
\curveto(478.316814,120.24082299)(478.34144319,120.48522036)(478.34144796,120.84139646)
\lineto(478.34144796,121.7223646)
\curveto(477.86401577,121.07063731)(477.26154784,120.74477415)(476.53404236,120.74477415)
\curveto(475.72317061,120.74477415)(475.08091707,121.08768829)(474.60727979,121.77351759)
\curveto(474.23594632,122.31535879)(474.0502801,122.98981974)(474.05028058,123.79690247)
\curveto(474.0502801,124.80859085)(474.29278292,125.58156857)(474.77778975,126.11583795)
\curveto(475.26658328,126.6500966)(475.87284032,126.9172286)(476.59656268,126.91723477)
\curveto(477.34301363,126.9172286)(477.95874343,126.58947089)(478.44375393,125.93396066)
\lineto(478.44375393,126.7808268)
\lineto(479.93855793,126.7808268)
\lineto(479.93855793,121.36429368)
\curveto(479.93855157,120.65194104)(479.87982042,120.11957158)(479.76236431,119.7671837)
\curveto(479.64489582,119.41479778)(479.48006969,119.13819301)(479.26788542,118.93736855)
\curveto(479.05568976,118.73654772)(478.77150677,118.5792998)(478.41533561,118.46562433)
\curveto(478.06294386,118.35195341)(477.6158293,118.29511681)(477.07399057,118.29511436)
\curveto(476.05092832,118.29511681)(475.32531443,118.47131027)(474.89714673,118.82369525)
\curveto(474.46897637,119.17229496)(474.25489185,119.61562042)(474.25489254,120.15367295)
\curveto(474.25489185,120.20672103)(474.25678641,120.27113584)(474.2605762,120.34691757)
\moveto(475.68717622,123.88784112)
\curveto(475.6871741,123.24747898)(475.81032006,122.77762978)(476.05661447,122.4782921)
\curveto(476.30669301,122.18274006)(476.61361064,122.03496491)(476.97736826,122.0349662)
\curveto(477.36764282,122.03496491)(477.69729509,122.18652917)(477.96632604,122.48965943)
\curveto(478.23534821,122.79657531)(478.36986149,123.24937353)(478.36986628,123.84805546)
\curveto(478.36986149,124.47325492)(478.24103187,124.93742047)(477.98337704,125.24055348)
\curveto(477.72571339,125.5436775)(477.39985023,125.69524176)(477.00578659,125.69524671)
\curveto(476.6230834,125.69524176)(476.30669301,125.54557206)(476.05661447,125.24623715)
\curveto(475.81032006,124.95068234)(475.6871741,124.49788412)(475.68717622,123.88784112)
}
}
{
\newrgbcolor{curcolor}{0 0 0}
\pscustom[linestyle=none,fillstyle=solid,fillcolor=curcolor]
{
\newpath
\moveto(483.05320663,120.74477415)
\lineto(481.45609665,120.74477415)
\lineto(481.45609665,126.7808268)
\lineto(482.93953332,126.7808268)
\lineto(482.93953332,125.92259333)
\curveto(483.19340121,126.32802254)(483.4207476,126.59515455)(483.62157317,126.72399015)
\curveto(483.82618199,126.85281379)(484.05731748,126.9172286)(484.31498035,126.91723477)
\curveto(484.67873095,126.9172286)(485.02922329,126.81681728)(485.36645845,126.61600051)
\lineto(484.87197956,125.22350249)
\curveto(484.60294882,125.3977969)(484.35286779,125.48494635)(484.12173573,125.48495109)
\curveto(483.89817501,125.48494635)(483.70871969,125.4224261)(483.55336919,125.29739014)
\curveto(483.39801296,125.17613418)(483.274867,124.95447145)(483.18393094,124.63240128)
\curveto(483.09677899,124.31032335)(483.05320427,123.63586239)(483.05320663,122.6090164)
\lineto(483.05320663,120.74477415)
}
}
{
\newrgbcolor{curcolor}{0 0 0}
\pscustom[linestyle=none,fillstyle=solid,fillcolor=curcolor]
{
\newpath
\moveto(485.67905999,123.84805546)
\curveto(485.67905952,124.37852726)(485.80978369,124.89195119)(486.0712329,125.38832878)
\curveto(486.33268039,125.88469709)(486.70211827,126.26360773)(487.17954765,126.52506186)
\curveto(487.66076221,126.78650443)(488.19692077,126.9172286)(488.78802496,126.91723477)
\curveto(489.70119605,126.9172286)(490.44954457,126.61978374)(491.03307279,126.02489931)
\curveto(491.61658937,125.43379342)(491.90835057,124.68544489)(491.90835726,123.77985147)
\curveto(491.90835057,122.86667378)(491.61280026,122.10885248)(491.02170546,121.50638531)
\curveto(490.43438815,120.90770573)(489.69361783,120.60836632)(488.79939229,120.60836618)
\curveto(488.24617916,120.60836632)(487.71759881,120.73340683)(487.21364964,120.9834881)
\curveto(486.71348559,121.23356889)(486.33268039,121.59921766)(486.0712329,122.08043552)
\curveto(485.80978369,122.56543981)(485.67905952,123.15464587)(485.67905999,123.84805546)
\moveto(487.31595562,123.76280048)
\curveto(487.31595352,123.16411864)(487.45804501,122.70563675)(487.74223053,122.38735345)
\curveto(488.02641098,122.06906687)(488.37690333,121.90992439)(488.79370863,121.90992556)
\curveto(489.21050676,121.90992439)(489.55910455,122.06906687)(489.83950306,122.38735345)
\curveto(490.12368142,122.70563675)(490.26577291,123.16790774)(490.26577796,123.77416781)
\curveto(490.26577291,124.36526539)(490.12368142,124.81995817)(489.83950306,125.1382475)
\curveto(489.55910455,125.45652806)(489.21050676,125.61567053)(488.79370863,125.6156754)
\curveto(488.37690333,125.61567053)(488.02641098,125.45652806)(487.74223053,125.1382475)
\curveto(487.45804501,124.81995817)(487.31595352,124.36147628)(487.31595562,123.76280048)
}
}
{
\newrgbcolor{curcolor}{0 0 0}
\pscustom[linestyle=none,fillstyle=solid,fillcolor=curcolor]
{
\newpath
\moveto(497.13733038,120.74477415)
\lineto(497.13733038,121.64847695)
\curveto(496.91755739,121.326402)(496.62769075,121.07253186)(496.26772957,120.88686579)
\curveto(495.91154962,120.70119943)(495.53453353,120.60836632)(495.13668016,120.60836618)
\curveto(494.73124296,120.60836632)(494.36748873,120.69741032)(494.0454164,120.87549846)
\curveto(493.72334063,121.05358633)(493.49031059,121.30366736)(493.34632556,121.62574229)
\curveto(493.20233849,121.94781546)(493.13034547,122.39303547)(493.13034627,122.96140366)
\lineto(493.13034627,126.7808268)
\lineto(494.72745625,126.7808268)
\lineto(494.72745625,124.00719809)
\curveto(494.72745385,123.15843498)(494.75587215,122.63743284)(494.81271123,122.44419011)
\curveto(494.87333445,122.25473308)(494.98132398,122.10316882)(495.13668016,121.98949687)
\curveto(495.29203071,121.87961154)(495.48906425,121.8246695)(495.72778136,121.82467058)
\curveto(496.00059363,121.8246695)(496.24499099,121.89855707)(496.4609742,122.04633353)
\curveto(496.67694913,122.19789649)(496.82472428,122.3835627)(496.9043001,122.60333274)
\curveto(496.98386676,122.82688816)(497.02365237,123.37062494)(497.02365707,124.23454471)
\lineto(497.02365707,126.7808268)
\lineto(498.62076705,126.7808268)
\lineto(498.62076705,120.74477415)
\lineto(497.13733038,120.74477415)
}
}
{
\newrgbcolor{curcolor}{0 0 0}
\pscustom[linestyle=none,fillstyle=solid,fillcolor=curcolor]
{
\newpath
\moveto(500.23492719,126.7808268)
\lineto(501.72404752,126.7808268)
\lineto(501.72404752,125.894175)
\curveto(501.91728968,126.19729837)(502.17873802,126.44359029)(502.50839335,126.6330515)
\curveto(502.83804255,126.82250094)(503.20369132,126.9172286)(503.60534077,126.91723477)
\curveto(504.30632131,126.9172286)(504.90121103,126.64251838)(505.39001171,126.09310329)
\curveto(505.8788005,125.5436775)(506.12319786,124.778278)(506.12320454,123.79690247)
\curveto(506.12319786,122.7889971)(505.87690594,122.00465205)(505.38432804,121.443865)
\curveto(504.89173826,120.88686564)(504.29495399,120.60836632)(503.59397344,120.60836618)
\curveto(503.26052792,120.60836632)(502.9573994,120.67467568)(502.68458698,120.80729447)
\curveto(502.41555718,120.93991314)(502.13137419,121.16725952)(501.83203717,121.48933432)
\lineto(501.83203717,118.44857333)
\lineto(500.23492719,118.44857333)
\lineto(500.23492719,126.7808268)
\moveto(501.81498617,123.86510646)
\curveto(501.8149838,123.18685328)(501.94949708,122.68479667)(502.21852641,122.35893512)
\curveto(502.4875502,122.03685946)(502.81530791,121.87582243)(503.20180053,121.87582357)
\curveto(503.57312921,121.87582243)(503.88194138,122.02359759)(504.12823799,122.31914947)
\curveto(504.37452523,122.6184873)(504.49767119,123.10728204)(504.49767624,123.78553514)
\curveto(504.49767119,124.41831288)(504.37073612,124.88816208)(504.11687066,125.19508416)
\curveto(503.86299585,125.50199733)(503.54850001,125.65545615)(503.1733822,125.65546106)
\curveto(502.78310051,125.65545615)(502.4591319,125.50389189)(502.20147542,125.20076782)
\curveto(501.94381342,124.90142396)(501.8149838,124.45620395)(501.81498617,123.86510646)
}
}
{
\newrgbcolor{curcolor}{0 0 0}
\pscustom[linestyle=none,fillstyle=solid,fillcolor=curcolor]
{
\newpath
\moveto(506.83366189,122.46692477)
\lineto(508.43645553,122.71132238)
\curveto(508.50465757,122.40061368)(508.64295996,122.16379453)(508.8513631,122.0008642)
\curveto(509.05976167,121.84172048)(509.35152287,121.76214924)(509.72664758,121.76215026)
\curveto(510.13965702,121.76214924)(510.45036375,121.83793137)(510.6587687,121.98949687)
\curveto(510.79896154,122.09559061)(510.86906001,122.2376821)(510.86906432,122.41577178)
\curveto(510.86906001,122.53702151)(510.83116895,122.63743284)(510.75539101,122.71700604)
\curveto(510.67581558,122.7927862)(510.49772758,122.86288467)(510.22112647,122.92730166)
\curveto(508.9328266,123.21148247)(508.11627416,123.47103626)(507.77146668,123.70596382)
\curveto(507.29403805,124.03182402)(507.05532434,124.48462224)(507.05532484,125.06435985)
\curveto(507.05532434,125.58725223)(507.26183065,126.02678858)(507.67484437,126.38297023)
\curveto(508.08785586,126.7391406)(508.72821485,126.9172286)(509.59592327,126.91723477)
\curveto(510.42194545,126.9172286)(511.0357807,126.78271532)(511.43743086,126.51369453)
\curveto(511.83907127,126.2446622)(512.11567604,125.84680602)(512.26724601,125.3201248)
\lineto(510.76107468,125.04162519)
\curveto(510.69665567,125.2765455)(510.57350971,125.45652806)(510.39163643,125.58157341)
\curveto(510.21354459,125.70660908)(509.95777991,125.76912934)(509.6243416,125.76913436)
\curveto(509.20374772,125.76912934)(508.90251375,125.71039819)(508.7206388,125.59294074)
\curveto(508.59938523,125.50957555)(508.53875953,125.40158601)(508.53876151,125.26897181)
\curveto(508.53875953,125.15529409)(508.59180702,125.05867187)(508.69790414,124.97910487)
\curveto(508.84188805,124.87300566)(509.338261,124.72333595)(510.18702447,124.53009531)
\curveto(511.0395698,124.33684709)(511.63445952,124.10002794)(511.97169541,123.81963713)
\curveto(512.30513137,123.53545107)(512.47185205,123.13948944)(512.47185796,122.63175106)
\curveto(512.47185205,122.07853963)(512.24071656,121.60300677)(511.77845078,121.20515105)
\curveto(511.31617458,120.80729441)(510.63224086,120.60836632)(509.72664758,120.60836618)
\curveto(508.9044083,120.60836632)(508.25268199,120.775087)(507.77146668,121.10852874)
\curveto(507.29403805,121.44196974)(506.98143677,121.89476797)(506.83366189,122.46692477)
}
}
{
\newrgbcolor{curcolor}{0 0 0}
\pscustom[linestyle=none,fillstyle=solid,fillcolor=curcolor]
{
\newpath
\moveto(517.37117961,122.66585306)
\lineto(518.96260592,122.39872078)
\curveto(518.75798825,121.81519673)(518.43401964,121.36997672)(517.99069914,121.06305941)
\curveto(517.55115784,120.75993058)(516.99984284,120.60836632)(516.33675251,120.60836618)
\curveto(515.28716672,120.60836632)(514.51039989,120.95128045)(514.00644969,121.63710962)
\curveto(513.60859255,122.18652917)(513.40966446,122.87993565)(513.40966483,123.71733115)
\curveto(513.40966446,124.71765229)(513.6711128,125.50010278)(514.19401065,126.06468496)
\curveto(514.71690619,126.63304562)(515.37810527,126.9172286)(516.17760987,126.91723477)
\curveto(517.07562497,126.9172286)(517.78418788,126.61978374)(518.30330073,126.02489931)
\curveto(518.82240306,125.43379342)(519.07058953,124.52630242)(519.0478609,123.30242358)
\lineto(515.04656046,123.30242358)
\curveto(515.05792577,122.82878271)(515.18675539,122.45934483)(515.43304971,122.19410883)
\curveto(515.67933924,121.93265903)(515.98625686,121.80193486)(516.3538035,121.80193592)
\curveto(516.60388122,121.80193486)(516.81417663,121.87013878)(516.98469036,122.00654787)
\curveto(517.15519621,122.14295444)(517.28402583,122.36272262)(517.37117961,122.66585306)
\moveto(517.46211825,124.28001403)
\curveto(517.45074651,124.74228148)(517.33138966,125.09277383)(517.10404733,125.33149213)
\curveto(516.87669688,125.57399036)(516.60009211,125.69524176)(516.27423219,125.69524671)
\curveto(515.92563116,125.69524176)(515.63765906,125.5683067)(515.41031505,125.31444113)
\curveto(515.18296629,125.06056643)(515.07118765,124.71575774)(515.07497879,124.28001403)
\lineto(517.46211825,124.28001403)
}
}
{
\newrgbcolor{curcolor}{0 0 0}
\pscustom[linestyle=none,fillstyle=solid,fillcolor=curcolor]
{
\newpath
\moveto(523.12304751,126.7808268)
\lineto(523.12304751,125.50768576)
\lineto(522.03178375,125.50768576)
\lineto(522.03178375,123.07507696)
\curveto(522.03178124,122.58249079)(522.04125401,122.2945187)(522.06020208,122.21115982)
\curveto(522.08293418,122.13158712)(522.13029801,122.06527776)(522.20229372,122.01223154)
\curveto(522.27807316,121.95918278)(522.36901172,121.93265903)(522.47510965,121.93266022)
\curveto(522.62288185,121.93265903)(522.83696637,121.98381197)(523.11736384,122.08611919)
\lineto(523.25377181,120.84708013)
\curveto(522.88243564,120.68793755)(522.46184483,120.60836632)(521.9919981,120.60836618)
\curveto(521.70402353,120.60836632)(521.44446974,120.65573015)(521.21333594,120.75045782)
\curveto(520.98219875,120.84897458)(520.81168896,120.97401509)(520.70180605,121.12557973)
\curveto(520.59570989,121.28093272)(520.52182231,121.48933357)(520.4801431,121.75078293)
\curveto(520.44604018,121.93644814)(520.4289892,122.31156968)(520.42899011,122.87614868)
\lineto(520.42899011,125.50768576)
\lineto(519.69579727,125.50768576)
\lineto(519.69579727,126.7808268)
\lineto(520.42899011,126.7808268)
\lineto(520.42899011,127.9800802)
\lineto(522.03178375,128.91220133)
\lineto(522.03178375,126.7808268)
\lineto(523.12304751,126.7808268)
}
}
{
\newrgbcolor{curcolor}{0 0 0}
\pscustom[linestyle=none,fillstyle=solid,fillcolor=curcolor]
{
\newpath
\moveto(523.68004512,122.46692477)
\lineto(525.28283876,122.71132238)
\curveto(525.3510408,122.40061368)(525.48934319,122.16379453)(525.69774634,122.0008642)
\curveto(525.9061449,121.84172048)(526.1979061,121.76214924)(526.57303081,121.76215026)
\curveto(526.98604025,121.76214924)(527.29674698,121.83793137)(527.50515193,121.98949687)
\curveto(527.64534477,122.09559061)(527.71544324,122.2376821)(527.71544755,122.41577178)
\curveto(527.71544324,122.53702151)(527.67755218,122.63743284)(527.60177424,122.71700604)
\curveto(527.52219881,122.7927862)(527.34411081,122.86288467)(527.0675097,122.92730166)
\curveto(525.77920983,123.21148247)(524.96265739,123.47103626)(524.61784991,123.70596382)
\curveto(524.14042128,124.03182402)(523.90170757,124.48462224)(523.90170807,125.06435985)
\curveto(523.90170757,125.58725223)(524.10821388,126.02678858)(524.5212276,126.38297023)
\curveto(524.93423909,126.7391406)(525.57459808,126.9172286)(526.4423065,126.91723477)
\curveto(527.26832868,126.9172286)(527.88216393,126.78271532)(528.28381409,126.51369453)
\curveto(528.6854545,126.2446622)(528.96205927,125.84680602)(529.11362924,125.3201248)
\lineto(527.60745791,125.04162519)
\curveto(527.5430389,125.2765455)(527.41989294,125.45652806)(527.23801966,125.58157341)
\curveto(527.05992782,125.70660908)(526.80416314,125.76912934)(526.47072483,125.76913436)
\curveto(526.05013095,125.76912934)(525.74889698,125.71039819)(525.56702203,125.59294074)
\curveto(525.44576846,125.50957555)(525.38514276,125.40158601)(525.38514474,125.26897181)
\curveto(525.38514276,125.15529409)(525.43819025,125.05867187)(525.54428737,124.97910487)
\curveto(525.68827128,124.87300566)(526.18464423,124.72333595)(527.0334077,124.53009531)
\curveto(527.88595303,124.33684709)(528.48084275,124.10002794)(528.81807864,123.81963713)
\curveto(529.1515146,123.53545107)(529.31823528,123.13948944)(529.31824119,122.63175106)
\curveto(529.31823528,122.07853963)(529.08709979,121.60300677)(528.62483402,121.20515105)
\curveto(528.16255781,120.80729441)(527.47862409,120.60836632)(526.57303081,120.60836618)
\curveto(525.75079154,120.60836632)(525.09906522,120.775087)(524.61784991,121.10852874)
\curveto(524.14042128,121.44196974)(523.82782,121.89476797)(523.68004512,122.46692477)
}
}
{
\newrgbcolor{curcolor}{0 1 0.25098041}
\pscustom[linewidth=2.32802935,linecolor=curcolor]
{
\newpath
\moveto(581.82618442,114.33554061)
\lineto(658.65115093,114.33554061)
\lineto(658.65115093,159.73211258)
\lineto(607.43450969,159.73211258)
\lineto(607.43450969,170.2082503)
\lineto(581.82618442,170.2082503)
\lineto(581.82618442,114.33554061)
\closepath
}
}
{
\newrgbcolor{curcolor}{0 0 0}
\pscustom[linestyle=none,fillstyle=solid,fillcolor=curcolor]
{
\newpath
\moveto(596.43091863,144.17627178)
\lineto(596.43091863,145.13112757)
\lineto(601.94975773,147.46143038)
\lineto(601.94975773,146.44405428)
\lineto(597.57333537,144.64801601)
\lineto(601.94975773,142.83492675)
\lineto(601.94975773,141.81755064)
\lineto(596.43091863,144.17627178)
}
}
{
\newrgbcolor{curcolor}{0 0 0}
\pscustom[linestyle=none,fillstyle=solid,fillcolor=curcolor]
{
\newpath
\moveto(603.22858262,144.17627178)
\lineto(603.22858262,145.13112757)
\lineto(608.74742172,147.46143038)
\lineto(608.74742172,146.44405428)
\lineto(604.37099936,144.64801601)
\lineto(608.74742172,142.83492675)
\lineto(608.74742172,141.81755064)
\lineto(603.22858262,144.17627178)
}
}
{
\newrgbcolor{curcolor}{0 0 0}
\pscustom[linestyle=none,fillstyle=solid,fillcolor=curcolor]
{
\newpath
\moveto(614.09575104,141.27760243)
\curveto(613.71683568,140.95552763)(613.35118691,140.72818124)(612.99880362,140.59556258)
\curveto(612.65020221,140.46294379)(612.27508067,140.39663443)(611.87343787,140.39663429)
\curveto(611.21034175,140.39663443)(610.70070693,140.55767145)(610.34453187,140.87974585)
\curveto(609.98835491,141.20560866)(609.81026691,141.62051582)(609.81026733,142.12446857)
\curveto(609.81026691,142.42001729)(609.87657627,142.68904385)(610.00919562,142.93154906)
\curveto(610.14560283,143.17783858)(610.32179628,143.37487212)(610.5377765,143.52265026)
\curveto(610.75754353,143.67042242)(611.00383545,143.78220106)(611.276653,143.85798652)
\curveto(611.47747376,143.91103068)(611.78060227,143.96218362)(612.18603946,144.01144548)
\curveto(613.01206188,144.10995877)(613.62021347,144.22742108)(614.01049606,144.36383274)
\curveto(614.01428054,144.50402585)(614.0161751,144.59306985)(614.01617972,144.63096501)
\curveto(614.0161751,145.04776263)(613.91955288,145.34141838)(613.72631279,145.51193315)
\curveto(613.4648601,145.74306367)(613.07647669,145.85863141)(612.56116138,145.85863674)
\curveto(612.07994169,145.85863141)(611.72376568,145.77337652)(611.49263228,145.6028718)
\curveto(611.26528379,145.43614604)(611.09666855,145.13870118)(610.98678606,144.71053633)
\lineto(609.98646095,144.8469443)
\curveto(610.07739891,145.27510902)(610.22706862,145.61991771)(610.43547052,145.8813714)
\curveto(610.64387033,146.14660351)(610.9451043,146.3493207)(611.33917332,146.4895236)
\curveto(611.73323844,146.63350369)(612.18982577,146.70549671)(612.70893668,146.70550288)
\curveto(613.22425184,146.70549671)(613.64294811,146.64487101)(613.96502673,146.52362559)
\curveto(614.28709621,146.40236819)(614.52391536,146.24890938)(614.67548491,146.06324869)
\curveto(614.82704388,145.88136605)(614.93313886,145.65023056)(614.99377017,145.36984152)
\curveto(615.02786652,145.19553778)(615.0449175,144.88104194)(615.04492316,144.42635306)
\lineto(615.04492316,143.06227336)
\curveto(615.0449175,142.11120511)(615.06575759,141.50873718)(615.10744348,141.25486777)
\curveto(615.15290704,141.00478602)(615.24005649,140.76417776)(615.36889209,140.53304226)
\lineto(614.30036299,140.53304226)
\curveto(614.1942631,140.74523222)(614.12605918,140.9934187)(614.09575104,141.27760243)
\moveto(614.01049606,143.56243592)
\curveto(613.639159,143.41086863)(613.08216035,143.28203901)(612.33949843,143.17594667)
\curveto(611.91890466,143.11531832)(611.6214598,143.04711441)(611.44716296,142.97133472)
\curveto(611.27286201,142.89555015)(611.13834873,142.78377151)(611.04362272,142.63599846)
\curveto(610.9488934,142.49201031)(610.90152957,142.33097328)(610.90153108,142.1528869)
\curveto(610.90152957,141.88006961)(611.00383545,141.65272322)(611.20844901,141.47084705)
\curveto(611.41684805,141.288969)(611.71997657,141.19803045)(612.11783548,141.19803111)
\curveto(612.51189982,141.19803045)(612.86239217,141.28328534)(613.16931358,141.45379606)
\curveto(613.47622742,141.62809403)(613.70167926,141.86491319)(613.84566976,142.16425423)
\curveto(613.95554939,142.39538809)(614.01049144,142.73640768)(614.01049606,143.187314)
\lineto(614.01049606,143.56243592)
}
}
{
\newrgbcolor{curcolor}{0 0 0}
\pscustom[linestyle=none,fillstyle=solid,fillcolor=curcolor]
{
\newpath
\moveto(616.63634888,138.21979044)
\lineto(616.63634888,146.56909491)
\lineto(617.56847,146.56909491)
\lineto(617.56847,145.78474909)
\curveto(617.78823648,146.09166146)(618.03642295,146.3209024)(618.31303017,146.4724726)
\curveto(618.5896325,146.62782003)(618.92496842,146.70549671)(619.31903894,146.70550288)
\curveto(619.83435398,146.70549671)(620.28904675,146.57287798)(620.68311864,146.30764631)
\curveto(621.0771809,146.04240308)(621.37462576,145.66728154)(621.57545411,145.18228056)
\curveto(621.77627104,144.70105939)(621.87668237,144.17247903)(621.87668837,143.59653791)
\curveto(621.87668237,142.97891049)(621.76490373,142.42191184)(621.54135212,141.92554028)
\curveto(621.32157827,141.43295505)(620.99950422,141.0540444)(620.575129,140.7888072)
\curveto(620.15453347,140.5273586)(619.71120802,140.39663443)(619.24515129,140.39663429)
\curveto(618.90412834,140.39663443)(618.59721071,140.46862745)(618.3243975,140.61261358)
\curveto(618.05536848,140.75659954)(617.83370576,140.93847665)(617.65940865,141.15824545)
\lineto(617.65940865,138.21979044)
\lineto(616.63634888,138.21979044)
\moveto(617.56278634,143.5169666)
\curveto(617.56278464,142.74019678)(617.72003256,142.16614715)(618.03453056,141.79481598)
\curveto(618.34902424,141.42348228)(618.72982944,141.23781607)(619.17694731,141.23781677)
\curveto(619.63163678,141.23781607)(620.02002019,141.42916594)(620.34209872,141.81186698)
\curveto(620.6679574,142.19835456)(620.83088898,142.79513883)(620.83089394,143.60222158)
\curveto(620.83088898,144.37140712)(620.67174651,144.94735131)(620.35346605,145.33005586)
\curveto(620.03896573,145.71275081)(619.66194963,145.90410069)(619.22241663,145.90410606)
\curveto(618.78666603,145.90410069)(618.40017717,145.69948894)(618.06294889,145.2902702)
\curveto(617.72950533,144.88483105)(617.56278464,144.29373044)(617.56278634,143.5169666)
}
}
{
\newrgbcolor{curcolor}{0 0 0}
\pscustom[linestyle=none,fillstyle=solid,fillcolor=curcolor]
{
\newpath
\moveto(623.11572773,138.21979044)
\lineto(623.11572773,146.56909491)
\lineto(624.04784885,146.56909491)
\lineto(624.04784885,145.78474909)
\curveto(624.26761533,146.09166146)(624.5158018,146.3209024)(624.79240902,146.4724726)
\curveto(625.06901135,146.62782003)(625.40434727,146.70549671)(625.79841779,146.70550288)
\curveto(626.31373282,146.70549671)(626.7684256,146.57287798)(627.16249749,146.30764631)
\curveto(627.55655975,146.04240308)(627.85400461,145.66728154)(628.05483296,145.18228056)
\curveto(628.25564989,144.70105939)(628.35606122,144.17247903)(628.35606722,143.59653791)
\curveto(628.35606122,142.97891049)(628.24428257,142.42191184)(628.02073096,141.92554028)
\curveto(627.80095712,141.43295505)(627.47888307,141.0540444)(627.05450785,140.7888072)
\curveto(626.63391232,140.5273586)(626.19058686,140.39663443)(625.72453014,140.39663429)
\curveto(625.38350718,140.39663443)(625.07658956,140.46862745)(624.80377635,140.61261358)
\curveto(624.53474733,140.75659954)(624.31308461,140.93847665)(624.1387875,141.15824545)
\lineto(624.1387875,138.21979044)
\lineto(623.11572773,138.21979044)
\moveto(624.04216519,143.5169666)
\curveto(624.04216349,142.74019678)(624.19941141,142.16614715)(624.51390941,141.79481598)
\curveto(624.82840309,141.42348228)(625.20920829,141.23781607)(625.65632616,141.23781677)
\curveto(626.11101563,141.23781607)(626.49939904,141.42916594)(626.82147757,141.81186698)
\curveto(627.14733625,142.19835456)(627.31026783,142.79513883)(627.31027279,143.60222158)
\curveto(627.31026783,144.37140712)(627.15112536,144.94735131)(626.8328449,145.33005586)
\curveto(626.51834457,145.71275081)(626.14132848,145.90410069)(625.70179548,145.90410606)
\curveto(625.26604488,145.90410069)(624.87955602,145.69948894)(624.54232774,145.2902702)
\curveto(624.20888418,144.88483105)(624.04216349,144.29373044)(624.04216519,143.5169666)
}
}
{
\newrgbcolor{curcolor}{0 0 0}
\pscustom[linestyle=none,fillstyle=solid,fillcolor=curcolor]
{
\newpath
\moveto(634.98322137,144.17627178)
\lineto(629.46438227,141.81755064)
\lineto(629.46438227,142.83492675)
\lineto(633.83512096,144.64801601)
\lineto(629.46438227,146.44405428)
\lineto(629.46438227,147.46143038)
\lineto(634.98322137,145.13112757)
\lineto(634.98322137,144.17627178)
}
}
{
\newrgbcolor{curcolor}{0 0 0}
\pscustom[linestyle=none,fillstyle=solid,fillcolor=curcolor]
{
\newpath
\moveto(641.78088403,144.17627178)
\lineto(636.26204493,141.81755064)
\lineto(636.26204493,142.83492675)
\lineto(640.63278362,144.64801601)
\lineto(636.26204493,146.44405428)
\lineto(636.26204493,147.46143038)
\lineto(641.78088403,145.13112757)
\lineto(641.78088403,144.17627178)
}
}
{
\newrgbcolor{curcolor}{0 0 0}
\pscustom[linestyle=none,fillstyle=solid,fillcolor=curcolor]
{
\newpath
\moveto(590.81320507,124.99389129)
\lineto(592.40463138,124.72675901)
\curveto(592.20001371,124.14323496)(591.87604511,123.69801495)(591.4327246,123.39109765)
\curveto(590.9931833,123.08796881)(590.4418683,122.93640455)(589.77877797,122.93640441)
\curveto(588.72919218,122.93640455)(587.95242535,123.27931869)(587.44847515,123.96514785)
\curveto(587.05061801,124.5145674)(586.85168992,125.20797388)(586.85169029,126.04536939)
\curveto(586.85168992,127.04569052)(587.11313826,127.82814101)(587.63603611,128.3927232)
\curveto(588.15893165,128.96108385)(588.82013073,129.24526683)(589.61963534,129.24527301)
\curveto(590.51765043,129.24526683)(591.22621334,128.94782198)(591.74532619,128.35293754)
\curveto(592.26442852,127.76183165)(592.51261499,126.85434065)(592.48988636,125.63046181)
\lineto(588.48858592,125.63046181)
\curveto(588.49995123,125.15682095)(588.62878085,124.78738306)(588.87507517,124.52214706)
\curveto(589.1213647,124.26069726)(589.42828232,124.12997309)(589.79582896,124.12997415)
\curveto(590.04590668,124.12997309)(590.25620209,124.19817701)(590.42671582,124.3345861)
\curveto(590.59722167,124.47099267)(590.72605129,124.69076085)(590.81320507,124.99389129)
\moveto(590.90414372,126.60805226)
\curveto(590.89277197,127.07031972)(590.77341512,127.42081207)(590.5460728,127.65953036)
\curveto(590.31872234,127.90202859)(590.04211757,128.02328)(589.71625765,128.02328495)
\curveto(589.36765662,128.02328)(589.07968453,127.89634493)(588.85234051,127.64247936)
\curveto(588.62499175,127.38860466)(588.51321311,127.04379597)(588.51700425,126.60805226)
\lineto(590.90414372,126.60805226)
}
}
{
\newrgbcolor{curcolor}{0 0 0}
\pscustom[linestyle=none,fillstyle=solid,fillcolor=curcolor]
{
\newpath
\moveto(595.45675955,123.07281238)
\lineto(593.02415076,129.10886504)
\lineto(594.70083205,129.10886504)
\lineto(595.83756513,126.02831839)
\lineto(596.16721773,124.99957495)
\curveto(596.25436397,125.26102137)(596.30930601,125.43342572)(596.33204402,125.51678851)
\curveto(596.38508814,125.68729585)(596.44192474,125.85780564)(596.50255398,126.02831839)
\lineto(597.65065439,129.10886504)
\lineto(599.29323369,129.10886504)
\lineto(596.8947269,123.07281238)
\lineto(595.45675955,123.07281238)
}
}
{
\newrgbcolor{curcolor}{0 0 0}
\pscustom[linestyle=none,fillstyle=solid,fillcolor=curcolor]
{
\newpath
\moveto(601.4928117,127.26735745)
\lineto(600.04347702,127.52880606)
\curveto(600.20640802,128.112324)(600.4868019,128.54428214)(600.8846595,128.82468177)
\curveto(601.28251426,129.10506989)(601.87361487,129.24526683)(602.6579631,129.24527301)
\curveto(603.37031193,129.24526683)(603.90078683,129.16001194)(604.24938942,128.98950806)
\curveto(604.59798243,128.82278146)(604.84237979,128.60869695)(604.98258225,128.34725387)
\curveto(605.12656278,128.08958936)(605.1985558,127.6140565)(605.19856154,126.92065386)
\lineto(605.18151054,125.05641161)
\curveto(605.18150482,124.52593472)(605.20613401,124.1337622)(605.25539819,123.87989287)
\curveto(605.30843989,123.62981104)(605.4050621,123.36078448)(605.54526513,123.07281238)
\lineto(603.96520615,123.07281238)
\curveto(603.92352147,123.17890737)(603.87236854,123.33615528)(603.81174718,123.54455661)
\curveto(603.78521909,123.6392838)(603.76627355,123.70180406)(603.75491053,123.73211757)
\curveto(603.48209057,123.46687946)(603.19032937,123.26795137)(602.87962605,123.1353327)
\curveto(602.56891591,123.00271391)(602.23736909,122.93640455)(601.88498461,122.93640441)
\curveto(601.26356873,122.93640455)(600.77287944,123.10501979)(600.41291527,123.44225063)
\curveto(600.05673831,123.77948074)(599.87865031,124.20575522)(599.87865072,124.72107535)
\curveto(599.87865031,125.06209328)(599.9601161,125.3652218)(600.12304834,125.63046181)
\curveto(600.28597926,125.89948582)(600.51332564,126.10409756)(600.80508818,126.24429768)
\curveto(601.10063715,126.38828055)(601.52501707,126.51332106)(602.07822923,126.61941959)
\curveto(602.82468059,126.75961299)(603.34189363,126.89033716)(603.62986989,127.0115925)
\lineto(603.62986989,127.17073514)
\curveto(603.62986572,127.47764866)(603.55408359,127.69552229)(603.40252327,127.82435666)
\curveto(603.25095507,127.95697063)(602.96487753,128.02328)(602.5442898,128.02328495)
\curveto(602.26010373,128.02328)(602.038441,127.9664434)(601.87930094,127.85277498)
\curveto(601.72015606,127.74288612)(601.59132644,127.54774713)(601.4928117,127.26735745)
\moveto(603.62986989,125.97148174)
\curveto(603.42525397,125.90327492)(603.10128537,125.82180913)(602.6579631,125.72708412)
\curveto(602.21463445,125.63235381)(601.92476781,125.5395207)(601.7883623,125.44858452)
\curveto(601.57995912,125.30080699)(601.47575869,125.11324622)(601.4757607,124.88590165)
\curveto(601.47575869,124.66234255)(601.55911903,124.46909812)(601.72584198,124.30616778)
\curveto(601.8925604,124.14323496)(602.10475036,124.06176917)(602.3624125,124.06177016)
\curveto(602.6503817,124.06176917)(602.92509192,124.15649684)(603.18654399,124.34595343)
\curveto(603.37978469,124.48993821)(603.50671976,124.66613166)(603.56734957,124.87453432)
\curveto(603.60902564,125.01094035)(603.62986572,125.27049414)(603.62986989,125.65319647)
\lineto(603.62986989,125.97148174)
}
}
{
\newrgbcolor{curcolor}{0 0 0}
\pscustom[linestyle=none,fillstyle=solid,fillcolor=curcolor]
{
\newpath
\moveto(606.77862081,123.07281238)
\lineto(606.77862081,131.40506586)
\lineto(608.37573079,131.40506586)
\lineto(608.37573079,123.07281238)
\lineto(606.77862081,123.07281238)
}
}
{
\newrgbcolor{curcolor}{0 0 0}
\pscustom[linestyle=none,fillstyle=solid,fillcolor=curcolor]
{
\newpath
\moveto(613.97982564,123.07281238)
\lineto(613.97982564,123.97651518)
\curveto(613.76005266,123.65444023)(613.47018601,123.40057009)(613.11022484,123.21490402)
\curveto(612.75404489,123.02923766)(612.37702879,122.93640455)(611.97917542,122.93640441)
\curveto(611.57373822,122.93640455)(611.209984,123.02544855)(610.88791167,123.20353669)
\curveto(610.5658359,123.38162456)(610.33280585,123.63170559)(610.18882082,123.95378052)
\curveto(610.04483376,124.27585369)(609.97284074,124.7210737)(609.97284154,125.28944189)
\lineto(609.97284154,129.10886504)
\lineto(611.56995151,129.10886504)
\lineto(611.56995151,126.33523632)
\curveto(611.56994912,125.48647321)(611.59836741,124.96547107)(611.65520649,124.77222834)
\curveto(611.71582971,124.58277131)(611.82381925,124.43120706)(611.97917542,124.31753511)
\curveto(612.13452598,124.20764977)(612.33155952,124.15270773)(612.57027662,124.15270881)
\curveto(612.84308889,124.15270773)(613.08748626,124.22659531)(613.30346946,124.37437176)
\curveto(613.5194444,124.52593472)(613.66721955,124.71160093)(613.74679536,124.93137097)
\curveto(613.82636202,125.15492639)(613.86614764,125.69866317)(613.86615233,126.56258294)
\lineto(613.86615233,129.10886504)
\lineto(615.46326231,129.10886504)
\lineto(615.46326231,123.07281238)
\lineto(613.97982564,123.07281238)
}
}
{
\newrgbcolor{curcolor}{0 0 0}
\pscustom[linestyle=none,fillstyle=solid,fillcolor=curcolor]
{
\newpath
\moveto(618.31646151,127.26735745)
\lineto(616.86712684,127.52880606)
\curveto(617.03005783,128.112324)(617.31045171,128.54428214)(617.70830931,128.82468177)
\curveto(618.10616407,129.10506989)(618.69726468,129.24526683)(619.48161292,129.24527301)
\curveto(620.19396174,129.24526683)(620.72443665,129.16001194)(621.07303923,128.98950806)
\curveto(621.42163224,128.82278146)(621.66602961,128.60869695)(621.80623207,128.34725387)
\curveto(621.95021259,128.08958936)(622.02220562,127.6140565)(622.02221135,126.92065386)
\lineto(622.00516036,125.05641161)
\curveto(622.00515464,124.52593472)(622.02978383,124.1337622)(622.07904801,123.87989287)
\curveto(622.1320897,123.62981104)(622.22871192,123.36078448)(622.36891494,123.07281238)
\lineto(620.78885596,123.07281238)
\curveto(620.74717129,123.17890737)(620.69601835,123.33615528)(620.63539699,123.54455661)
\curveto(620.6088689,123.6392838)(620.58992337,123.70180406)(620.57856034,123.73211757)
\curveto(620.30574038,123.46687946)(620.01397918,123.26795137)(619.70327587,123.1353327)
\curveto(619.39256572,123.00271391)(619.06101891,122.93640455)(618.70863442,122.93640441)
\curveto(618.08721854,122.93640455)(617.59652925,123.10501979)(617.23656509,123.44225063)
\curveto(616.88038813,123.77948074)(616.70230012,124.20575522)(616.70230054,124.72107535)
\curveto(616.70230012,125.06209328)(616.78376591,125.3652218)(616.94669815,125.63046181)
\curveto(617.10962907,125.89948582)(617.33697546,126.10409756)(617.628738,126.24429768)
\curveto(617.92428696,126.38828055)(618.34866689,126.51332106)(618.90187905,126.61941959)
\curveto(619.64833041,126.75961299)(620.16554344,126.89033716)(620.4535197,127.0115925)
\lineto(620.4535197,127.17073514)
\curveto(620.45351554,127.47764866)(620.37773341,127.69552229)(620.22617309,127.82435666)
\curveto(620.07460489,127.95697063)(619.78852735,128.02328)(619.36793961,128.02328495)
\curveto(619.08375354,128.02328)(618.86209082,127.9664434)(618.70295076,127.85277498)
\curveto(618.54380587,127.74288612)(618.41497625,127.54774713)(618.31646151,127.26735745)
\moveto(620.4535197,125.97148174)
\curveto(620.24890379,125.90327492)(619.92493518,125.82180913)(619.48161292,125.72708412)
\curveto(619.03828427,125.63235381)(618.74841762,125.5395207)(618.61201211,125.44858452)
\curveto(618.40360893,125.30080699)(618.2994085,125.11324622)(618.29941052,124.88590165)
\curveto(618.2994085,124.66234255)(618.38276885,124.46909812)(618.54949179,124.30616778)
\curveto(618.71621022,124.14323496)(618.92840018,124.06176917)(619.18606232,124.06177016)
\curveto(619.47403151,124.06176917)(619.74874173,124.15649684)(620.0101938,124.34595343)
\curveto(620.20343451,124.48993821)(620.33036958,124.66613166)(620.39099938,124.87453432)
\curveto(620.43267545,125.01094035)(620.45351554,125.27049414)(620.4535197,125.65319647)
\lineto(620.4535197,125.97148174)
}
}
{
\newrgbcolor{curcolor}{0 0 0}
\pscustom[linestyle=none,fillstyle=solid,fillcolor=curcolor]
{
\newpath
\moveto(626.3702139,129.10886504)
\lineto(626.3702139,127.83572399)
\lineto(625.27895014,127.83572399)
\lineto(625.27895014,125.4031152)
\curveto(625.27894763,124.91052902)(625.2884204,124.62255693)(625.30736847,124.53919806)
\curveto(625.33010057,124.45962535)(625.3774644,124.39331599)(625.44946011,124.34026977)
\curveto(625.52523955,124.28722101)(625.61617811,124.26069726)(625.72227604,124.26069845)
\curveto(625.87004824,124.26069726)(626.08413276,124.3118502)(626.36453023,124.41415742)
\lineto(626.5009382,123.17511836)
\curveto(626.12960203,123.01597579)(625.70901122,122.93640455)(625.23916449,122.93640441)
\curveto(624.95118992,122.93640455)(624.69163613,122.98376838)(624.46050233,123.07849605)
\curveto(624.22936514,123.17701281)(624.05885535,123.30205333)(623.94897244,123.45361797)
\curveto(623.84287628,123.60897095)(623.7689887,123.81737181)(623.72730949,124.07882116)
\curveto(623.69320657,124.26448637)(623.67615559,124.63960791)(623.6761565,125.20418691)
\lineto(623.6761565,127.83572399)
\lineto(622.94296366,127.83572399)
\lineto(622.94296366,129.10886504)
\lineto(623.6761565,129.10886504)
\lineto(623.6761565,130.30811844)
\lineto(625.27895014,131.24023956)
\lineto(625.27895014,129.10886504)
\lineto(626.3702139,129.10886504)
}
}
{
\newrgbcolor{curcolor}{0 0 0}
\pscustom[linestyle=none,fillstyle=solid,fillcolor=curcolor]
{
\newpath
\moveto(627.48989794,129.92731285)
\lineto(627.48989794,131.40506586)
\lineto(629.08700791,131.40506586)
\lineto(629.08700791,129.92731285)
\lineto(627.48989794,129.92731285)
\moveto(627.48989794,123.07281238)
\lineto(627.48989794,129.10886504)
\lineto(629.08700791,129.10886504)
\lineto(629.08700791,123.07281238)
\lineto(627.48989794,123.07281238)
}
}
{
\newrgbcolor{curcolor}{0 0 0}
\pscustom[linestyle=none,fillstyle=solid,fillcolor=curcolor]
{
\newpath
\moveto(630.3487824,126.17609369)
\curveto(630.34878194,126.70656549)(630.47950611,127.21998942)(630.74095532,127.71636701)
\curveto(631.0024028,128.21273532)(631.37184069,128.59164597)(631.84927007,128.85310009)
\curveto(632.33048462,129.11454266)(632.86664319,129.24526683)(633.45774738,129.24527301)
\curveto(634.37091846,129.24526683)(635.11926699,128.94782198)(635.70279521,128.35293754)
\curveto(636.28631179,127.76183165)(636.57807298,127.01348312)(636.57807968,126.10788971)
\curveto(636.57807298,125.19471201)(636.28252268,124.43689072)(635.69142788,123.83442355)
\curveto(635.10411056,123.23574396)(634.36334025,122.93640455)(633.46911471,122.93640441)
\curveto(632.91590157,122.93640455)(632.38732122,123.06144506)(631.88337206,123.31152633)
\curveto(631.38320801,123.56160712)(631.0024028,123.92725589)(630.74095532,124.40847375)
\curveto(630.47950611,124.89347805)(630.34878194,125.4826841)(630.3487824,126.17609369)
\moveto(631.98567804,126.09083871)
\curveto(631.98567593,125.49215687)(632.12776743,125.03367499)(632.41195294,124.71539168)
\curveto(632.6961334,124.3971051)(633.04662575,124.23796263)(633.46343104,124.23796379)
\curveto(633.88022917,124.23796263)(634.22882697,124.3971051)(634.50922547,124.71539168)
\curveto(634.79340383,125.03367499)(634.93549533,125.49594598)(634.93550038,126.10220604)
\curveto(634.93549533,126.69330362)(634.79340383,127.1479964)(634.50922547,127.46628574)
\curveto(634.22882697,127.78456629)(633.88022917,127.94370876)(633.46343104,127.94371363)
\curveto(633.04662575,127.94370876)(632.6961334,127.78456629)(632.41195294,127.46628574)
\curveto(632.12776743,127.1479964)(631.98567593,126.68951452)(631.98567804,126.09083871)
}
}
{
\newrgbcolor{curcolor}{0 0 0}
\pscustom[linestyle=none,fillstyle=solid,fillcolor=curcolor]
{
\newpath
\moveto(643.32458968,123.07281238)
\lineto(641.7274797,123.07281238)
\lineto(641.7274797,126.15335903)
\curveto(641.72747497,126.80508226)(641.69337301,127.22567308)(641.62517372,127.41513275)
\curveto(641.55696518,127.60837284)(641.44518654,127.75804254)(641.28983747,127.86414231)
\curveto(641.13826892,127.9702325)(640.95449725,128.02328)(640.73852192,128.02328495)
\curveto(640.46191341,128.02328)(640.21372693,127.94749787)(639.99396175,127.79593833)
\curveto(639.77419058,127.64436935)(639.62262632,127.4435467)(639.53926852,127.1934698)
\curveto(639.45969475,126.94338465)(639.41990913,126.48111366)(639.41991155,125.80665544)
\lineto(639.41991155,123.07281238)
\lineto(637.82280157,123.07281238)
\lineto(637.82280157,129.10886504)
\lineto(639.30623824,129.10886504)
\lineto(639.30623824,128.22221323)
\curveto(639.83292173,128.90424725)(640.49601537,129.24526683)(641.29552113,129.24527301)
\curveto(641.64790374,129.24526683)(641.96997779,129.18085202)(642.26174425,129.05202838)
\curveto(642.55350018,128.92698189)(642.77326836,128.76594486)(642.92104943,128.56891682)
\curveto(643.07260777,128.37187779)(643.1768082,128.14832051)(643.23365103,127.89824431)
\curveto(643.2942705,127.64815845)(643.32458335,127.29008789)(643.32458968,126.82403155)
\lineto(643.32458968,123.07281238)
}
}
{
\newrgbcolor{curcolor}{0 0 0}
\pscustom[linestyle=none,fillstyle=solid,fillcolor=curcolor]
{
\newpath
\moveto(644.38743428,124.794963)
\lineto(645.99022792,125.03936061)
\curveto(646.05842996,124.72865191)(646.19673235,124.49183276)(646.40513549,124.32890244)
\curveto(646.61353406,124.16975871)(646.90529526,124.09018747)(647.28041997,124.09018849)
\curveto(647.69342941,124.09018747)(648.00413614,124.1659696)(648.21254109,124.31753511)
\curveto(648.35273393,124.42362884)(648.4228324,124.56572034)(648.42283671,124.74381001)
\curveto(648.4228324,124.86505975)(648.38494134,124.96547107)(648.3091634,125.04504428)
\curveto(648.22958797,125.12082443)(648.05149997,125.1909229)(647.77489886,125.2553399)
\curveto(646.48659899,125.5395207)(645.67004655,125.79907449)(645.32523907,126.03400206)
\curveto(644.84781044,126.35986225)(644.60909673,126.81266048)(644.60909723,127.39239809)
\curveto(644.60909673,127.91529046)(644.81560304,128.35482681)(645.22861676,128.71100846)
\curveto(645.64162825,129.06717883)(646.28198724,129.24526683)(647.14969566,129.24527301)
\curveto(647.97571784,129.24526683)(648.58955309,129.11075355)(648.99120325,128.84173276)
\curveto(649.39284366,128.57270043)(649.66944843,128.17484425)(649.8210184,127.64816303)
\lineto(648.31484707,127.36966342)
\curveto(648.25042806,127.60458373)(648.1272821,127.78456629)(647.94540882,127.90961164)
\curveto(647.76731698,128.03464731)(647.51155229,128.09716757)(647.17811399,128.0971726)
\curveto(646.75752011,128.09716757)(646.45628614,128.03843642)(646.27441119,127.92097897)
\curveto(646.15315762,127.83761378)(646.09253192,127.72962424)(646.0925339,127.59701004)
\curveto(646.09253192,127.48333232)(646.14557941,127.38671011)(646.25167653,127.30714311)
\curveto(646.39566044,127.20104389)(646.89203339,127.05137418)(647.74079686,126.85813354)
\curveto(648.59334219,126.66488532)(649.18823191,126.42806617)(649.5254678,126.14767536)
\curveto(649.85890376,125.8634893)(650.02562444,125.46752768)(650.02563035,124.9597893)
\curveto(650.02562444,124.40657786)(649.79448895,123.931045)(649.33222317,123.53318928)
\curveto(648.86994697,123.13533264)(648.18601325,122.93640455)(647.28041997,122.93640441)
\curveto(646.45818069,122.93640455)(645.80645438,123.10312524)(645.32523907,123.43656697)
\curveto(644.84781044,123.77000798)(644.53520916,124.2228062)(644.38743428,124.794963)
}
}
{
\newrgbcolor{curcolor}{0 1 0.25098041}
\pscustom[linewidth=2.32802935,linecolor=curcolor]
{
\newpath
\moveto(464.26070507,22.37839586)
\lineto(543.41370116,22.37839586)
\lineto(543.41370116,68.93896864)
\lineto(491.03304048,68.93896864)
\lineto(491.03304048,78.25108692)
\lineto(464.26070507,78.25108692)
\lineto(464.26070507,22.37839586)
\closepath
}
}
{
\newrgbcolor{curcolor}{0 0 0}
\pscustom[linestyle=none,fillstyle=solid,fillcolor=curcolor]
{
\newpath
\moveto(480.02946193,52.21910279)
\lineto(480.02946193,53.17395858)
\lineto(485.54830103,55.5042614)
\lineto(485.54830103,54.48688529)
\lineto(481.17187867,52.69084702)
\lineto(485.54830103,50.87775776)
\lineto(485.54830103,49.86038165)
\lineto(480.02946193,52.21910279)
}
}
{
\newrgbcolor{curcolor}{0 0 0}
\pscustom[linestyle=none,fillstyle=solid,fillcolor=curcolor]
{
\newpath
\moveto(486.82712592,52.21910279)
\lineto(486.82712592,53.17395858)
\lineto(492.34596502,55.5042614)
\lineto(492.34596502,54.48688529)
\lineto(487.96954267,52.69084702)
\lineto(492.34596502,50.87775776)
\lineto(492.34596502,49.86038165)
\lineto(486.82712592,52.21910279)
}
}
{
\newrgbcolor{curcolor}{0 0 0}
\pscustom[linestyle=none,fillstyle=solid,fillcolor=curcolor]
{
\newpath
\moveto(497.69429434,49.32043344)
\curveto(497.31537899,48.99835865)(496.94973021,48.77101226)(496.59734692,48.63839359)
\curveto(496.24874551,48.5057748)(495.87362397,48.43946544)(495.47198117,48.43946531)
\curveto(494.80888505,48.43946544)(494.29925023,48.60050247)(493.94307518,48.92257686)
\curveto(493.58689821,49.24843967)(493.40881021,49.66334683)(493.40881063,50.16729959)
\curveto(493.40881021,50.4628483)(493.47511957,50.73187486)(493.60773892,50.97438007)
\curveto(493.74414613,51.2206696)(493.92033958,51.41770313)(494.1363198,51.56548127)
\curveto(494.35608683,51.71325344)(494.60237875,51.82503208)(494.8751963,51.90081753)
\curveto(495.07601706,51.9538617)(495.37914558,52.00501464)(495.78458277,52.0542765)
\curveto(496.61060518,52.15278979)(497.21875677,52.27025209)(497.60903936,52.40666375)
\curveto(497.61282385,52.54685686)(497.6147184,52.63590086)(497.61472303,52.67379603)
\curveto(497.6147184,53.09059364)(497.51809618,53.38424939)(497.32485609,53.55476416)
\curveto(497.06340341,53.78589468)(496.67501999,53.90146243)(496.15970468,53.90146775)
\curveto(495.67848499,53.90146243)(495.32230898,53.81620753)(495.09117559,53.64570281)
\curveto(494.8638271,53.47897706)(494.69521186,53.1815322)(494.58532937,52.75336734)
\lineto(493.58500426,52.88977531)
\curveto(493.67594222,53.31794003)(493.82561192,53.66274872)(494.03401382,53.92420241)
\curveto(494.24241363,54.18943452)(494.5436476,54.39215172)(494.93771662,54.53235461)
\curveto(495.33178175,54.6763347)(495.78836908,54.74832772)(496.30747998,54.7483339)
\curveto(496.82279515,54.74832772)(497.24149141,54.68770202)(497.56357004,54.5664566)
\curveto(497.88563951,54.44519921)(498.12245867,54.29174039)(498.27402821,54.10607971)
\curveto(498.42558718,53.92419707)(498.53168217,53.69306157)(498.59231347,53.41267253)
\curveto(498.62640983,53.23836879)(498.64346081,52.92387296)(498.64346646,52.46918407)
\lineto(498.64346646,51.10510438)
\curveto(498.64346081,50.15403612)(498.66430089,49.55156819)(498.70598678,49.29769878)
\curveto(498.75145034,49.04761703)(498.83859979,48.80700877)(498.96743539,48.57587327)
\lineto(497.8989063,48.57587327)
\curveto(497.7928064,48.78806324)(497.72460249,49.03624971)(497.69429434,49.32043344)
\moveto(497.60903936,51.60526693)
\curveto(497.2377023,51.45369964)(496.68070365,51.32487002)(495.93804173,51.21877768)
\curveto(495.51744796,51.15814934)(495.22000311,51.08994542)(495.04570627,51.01416573)
\curveto(494.87140531,50.93838116)(494.73689203,50.82660252)(494.64216602,50.67882947)
\curveto(494.54743671,50.53484132)(494.50007288,50.3738043)(494.50007439,50.19571791)
\curveto(494.50007288,49.92290063)(494.60237875,49.69555424)(494.80699232,49.51367807)
\curveto(495.01539136,49.33180002)(495.31851987,49.24086146)(495.71637878,49.24086213)
\curveto(496.11044313,49.24086146)(496.46093548,49.32611636)(496.76785688,49.49662707)
\curveto(497.07477073,49.67092505)(497.30022256,49.9077442)(497.44421306,50.20708524)
\curveto(497.5540927,50.43821911)(497.60903474,50.77923869)(497.60903936,51.23014502)
\lineto(497.60903936,51.60526693)
}
}
{
\newrgbcolor{curcolor}{0 0 0}
\pscustom[linestyle=none,fillstyle=solid,fillcolor=curcolor]
{
\newpath
\moveto(500.23489218,46.26262146)
\lineto(500.23489218,54.61192593)
\lineto(501.16701331,54.61192593)
\lineto(501.16701331,53.8275801)
\curveto(501.38677978,54.13449248)(501.63496626,54.36373342)(501.91157347,54.51530362)
\curveto(502.1881758,54.67065104)(502.52351173,54.74832772)(502.91758225,54.7483339)
\curveto(503.43289728,54.74832772)(503.88759006,54.615709)(504.28166194,54.35047732)
\curveto(504.6757242,54.08523409)(504.97316906,53.71011255)(505.17399741,53.22511157)
\curveto(505.37481435,52.7438904)(505.47522567,52.21531005)(505.47523168,51.63936892)
\curveto(505.47522567,51.02174151)(505.36344703,50.46474285)(505.13989542,49.9683713)
\curveto(504.92012157,49.47578606)(504.59804752,49.09687542)(504.1736723,48.83163822)
\curveto(503.75307678,48.57018961)(503.30975132,48.43946544)(502.8436946,48.43946531)
\curveto(502.50267164,48.43946544)(502.19575401,48.51145846)(501.9229408,48.65544459)
\curveto(501.65391179,48.79943056)(501.43224906,48.98130767)(501.25795195,49.20107647)
\lineto(501.25795195,46.26262146)
\lineto(500.23489218,46.26262146)
\moveto(501.16132964,51.55979761)
\curveto(501.16132795,50.7830278)(501.31857587,50.20897817)(501.63307387,49.83764699)
\curveto(501.94756754,49.4663133)(502.32837274,49.28064708)(502.77549061,49.28064778)
\curveto(503.23018008,49.28064708)(503.6185635,49.47199696)(503.94064202,49.85469799)
\curveto(504.2665007,50.24118557)(504.42943228,50.83796984)(504.42943724,51.64505259)
\curveto(504.42943228,52.41423814)(504.27028981,52.99018232)(503.95200935,53.37288687)
\curveto(503.63750903,53.75558183)(503.26049293,53.9469317)(502.82095994,53.94693708)
\curveto(502.38520934,53.9469317)(501.99872048,53.74231996)(501.6614922,53.33310121)
\curveto(501.32804863,52.92766206)(501.16132795,52.33656145)(501.16132964,51.55979761)
}
}
{
\newrgbcolor{curcolor}{0 0 0}
\pscustom[linestyle=none,fillstyle=solid,fillcolor=curcolor]
{
\newpath
\moveto(506.71427103,46.26262146)
\lineto(506.71427103,54.61192593)
\lineto(507.64639216,54.61192593)
\lineto(507.64639216,53.8275801)
\curveto(507.86615863,54.13449248)(508.11434511,54.36373342)(508.39095232,54.51530362)
\curveto(508.66755465,54.67065104)(509.00289057,54.74832772)(509.3969611,54.7483339)
\curveto(509.91227613,54.74832772)(510.36696891,54.615709)(510.76104079,54.35047732)
\curveto(511.15510305,54.08523409)(511.45254791,53.71011255)(511.65337626,53.22511157)
\curveto(511.8541932,52.7438904)(511.95460452,52.21531005)(511.95461053,51.63936892)
\curveto(511.95460452,51.02174151)(511.84282588,50.46474285)(511.61927427,49.9683713)
\curveto(511.39950042,49.47578606)(511.07742637,49.09687542)(510.65305115,48.83163822)
\curveto(510.23245563,48.57018961)(509.78913017,48.43946544)(509.32307345,48.43946531)
\curveto(508.98205049,48.43946544)(508.67513286,48.51145846)(508.40231965,48.65544459)
\curveto(508.13329064,48.79943056)(507.91162791,48.98130767)(507.7373308,49.20107647)
\lineto(507.7373308,46.26262146)
\lineto(506.71427103,46.26262146)
\moveto(507.64070849,51.55979761)
\curveto(507.6407068,50.7830278)(507.79795471,50.20897817)(508.11245272,49.83764699)
\curveto(508.42694639,49.4663133)(508.80775159,49.28064708)(509.25486946,49.28064778)
\curveto(509.70955893,49.28064708)(510.09794235,49.47199696)(510.42002087,49.85469799)
\curveto(510.74587955,50.24118557)(510.90881113,50.83796984)(510.90881609,51.64505259)
\curveto(510.90881113,52.41423814)(510.74966866,52.99018232)(510.4313882,53.37288687)
\curveto(510.11688788,53.75558183)(509.73987178,53.9469317)(509.30033879,53.94693708)
\curveto(508.86458819,53.9469317)(508.47809933,53.74231996)(508.14087104,53.33310121)
\curveto(507.80742748,52.92766206)(507.6407068,52.33656145)(507.64070849,51.55979761)
}
}
{
\newrgbcolor{curcolor}{0 0 0}
\pscustom[linestyle=none,fillstyle=solid,fillcolor=curcolor]
{
\newpath
\moveto(518.58176468,52.21910279)
\lineto(513.06292557,49.86038165)
\lineto(513.06292557,50.87775776)
\lineto(517.43366427,52.69084702)
\lineto(513.06292557,54.48688529)
\lineto(513.06292557,55.5042614)
\lineto(518.58176468,53.17395858)
\lineto(518.58176468,52.21910279)
}
}
{
\newrgbcolor{curcolor}{0 0 0}
\pscustom[linestyle=none,fillstyle=solid,fillcolor=curcolor]
{
\newpath
\moveto(525.37942734,52.21910279)
\lineto(519.86058824,49.86038165)
\lineto(519.86058824,50.87775776)
\lineto(524.23132693,52.69084702)
\lineto(519.86058824,54.48688529)
\lineto(519.86058824,55.5042614)
\lineto(525.37942734,53.17395858)
\lineto(525.37942734,52.21910279)
}
}
{
\newrgbcolor{curcolor}{0 0 0}
\pscustom[linestyle=none,fillstyle=solid,fillcolor=curcolor]
{
\newpath
\moveto(475.0153492,35.36696825)
\lineto(473.44097388,35.08278498)
\curveto(473.38792187,35.39727685)(473.26667046,35.63409601)(473.0772193,35.79324316)
\curveto(472.89154892,35.95238095)(472.6490461,36.03195219)(472.34971012,36.0319571)
\curveto(471.95185051,36.03195219)(471.63356557,35.8936498)(471.39485434,35.61704953)
\curveto(471.15992726,35.34422936)(471.04246496,34.88574748)(471.04246708,34.2416025)
\curveto(471.04246496,33.52545825)(471.16182181,33.01961254)(471.400538,32.72406384)
\curveto(471.64303833,32.42851193)(471.96700694,32.28073677)(472.37244479,32.28073794)
\curveto(472.67556985,32.28073677)(472.92375632,32.36599167)(473.11700495,32.53650288)
\curveto(473.31024518,32.71080036)(473.44665302,33.00824522)(473.52622886,33.42883835)
\lineto(475.09492051,33.16170608)
\curveto(474.93198276,32.4417738)(474.61938147,31.89803702)(474.15711572,31.53049411)
\curveto(473.69483949,31.16295036)(473.07532058,30.9791787)(472.29855714,30.97917856)
\curveto(471.41569195,30.9791787)(470.71091814,31.25767803)(470.18423361,31.81467738)
\curveto(469.66133565,32.37167533)(469.3998873,33.1427585)(469.39988778,34.12792919)
\curveto(469.3998873,35.12446118)(469.6632302,35.89933346)(470.18991727,36.45254834)
\curveto(470.7166018,37.00954166)(471.42895382,37.28804098)(472.32697546,37.28804716)
\curveto(473.06205871,37.28804098)(473.64558111,37.12889851)(474.07754441,36.81061926)
\curveto(474.51328649,36.49611773)(474.82588777,36.01490121)(475.0153492,35.36696825)
}
}
{
\newrgbcolor{curcolor}{0 0 0}
\pscustom[linestyle=none,fillstyle=solid,fillcolor=curcolor]
{
\newpath
\moveto(477.42522318,35.3101316)
\lineto(475.9758885,35.5715802)
\curveto(476.1388195,36.15509815)(476.41921338,36.58705628)(476.81707098,36.86745592)
\curveto(477.21492574,37.14784404)(477.80602635,37.28804098)(478.59037458,37.28804716)
\curveto(479.30272341,37.28804098)(479.83319831,37.20278609)(480.18180089,37.03228221)
\curveto(480.5303939,36.86555561)(480.77479127,36.65147109)(480.91499373,36.39002802)
\curveto(481.05897426,36.13236351)(481.13096728,35.65683064)(481.13097302,34.96342801)
\lineto(481.11392202,33.09918576)
\curveto(481.1139163,32.56870887)(481.13854549,32.17653635)(481.18780967,31.92266702)
\curveto(481.24085137,31.67258518)(481.33747358,31.40355862)(481.4776766,31.11558653)
\lineto(479.89761762,31.11558653)
\curveto(479.85593295,31.22168151)(479.80478001,31.37892943)(479.74415866,31.58733076)
\curveto(479.71763057,31.68205795)(479.69868503,31.74457821)(479.687322,31.77489172)
\curveto(479.41450205,31.50965361)(479.12274085,31.31072552)(478.81203753,31.17810685)
\curveto(478.50132739,31.04548806)(478.16978057,30.9791787)(477.81739609,30.97917856)
\curveto(477.19598021,30.9791787)(476.70529092,31.14779394)(476.34532675,31.48502478)
\curveto(475.98914979,31.82225489)(475.81106179,32.24852937)(475.8110622,32.7638495)
\curveto(475.81106179,33.10486743)(475.89252758,33.40799595)(476.05545981,33.67323596)
\curveto(476.21839073,33.94225996)(476.44573712,34.14687171)(476.73749966,34.28707182)
\curveto(477.03304863,34.4310547)(477.45742855,34.55609521)(478.01064071,34.66219374)
\curveto(478.75709207,34.80238713)(479.27430511,34.93311131)(479.56228137,35.05436665)
\lineto(479.56228137,35.21350928)
\curveto(479.5622772,35.52042281)(479.48649507,35.73829643)(479.33493475,35.86713081)
\curveto(479.18336655,35.99974478)(478.89728901,36.06605414)(478.47670127,36.06605909)
\curveto(478.19251521,36.06605414)(477.97085248,36.00921755)(477.81171242,35.89554913)
\curveto(477.65256754,35.78566026)(477.52373792,35.59052128)(477.42522318,35.3101316)
\moveto(479.56228137,34.01425589)
\curveto(479.35766545,33.94604907)(479.03369685,33.86458328)(478.59037458,33.76985827)
\curveto(478.14704593,33.67512796)(477.85717929,33.58229485)(477.72077378,33.49135867)
\curveto(477.5123706,33.34358114)(477.40817017,33.15602037)(477.40817218,32.92867579)
\curveto(477.40817017,32.7051167)(477.49153051,32.51187227)(477.65825346,32.34894192)
\curveto(477.82497188,32.18600911)(478.03716184,32.10454332)(478.29482398,32.10454431)
\curveto(478.58279318,32.10454332)(478.85750339,32.19927098)(479.11895546,32.38872758)
\curveto(479.31219617,32.53271235)(479.43913124,32.70890581)(479.49976105,32.91730846)
\curveto(479.54143711,33.0537145)(479.5622772,33.31326829)(479.56228137,33.69597062)
\lineto(479.56228137,34.01425589)
}
}
{
\newrgbcolor{curcolor}{0 0 0}
\pscustom[linestyle=none,fillstyle=solid,fillcolor=curcolor]
{
\newpath
\moveto(482.71103185,31.11558653)
\lineto(482.71103185,39.44784001)
\lineto(484.30814182,39.44784001)
\lineto(484.30814182,31.11558653)
\lineto(482.71103185,31.11558653)
}
}
{
\newrgbcolor{curcolor}{0 0 0}
\pscustom[linestyle=none,fillstyle=solid,fillcolor=curcolor]
{
\newpath
\moveto(485.93935368,37.970087)
\lineto(485.93935368,39.44784001)
\lineto(487.53646365,39.44784001)
\lineto(487.53646365,37.970087)
\lineto(485.93935368,37.970087)
\moveto(485.93935368,31.11558653)
\lineto(485.93935368,37.15163919)
\lineto(487.53646365,37.15163919)
\lineto(487.53646365,31.11558653)
\lineto(485.93935368,31.11558653)
}
}
{
\newrgbcolor{curcolor}{0 0 0}
\pscustom[linestyle=none,fillstyle=solid,fillcolor=curcolor]
{
\newpath
\moveto(488.46858555,37.15163919)
\lineto(489.35523735,37.15163919)
\lineto(489.35523735,37.60633242)
\curveto(489.35523633,38.11406619)(489.40828382,38.49297684)(489.51437998,38.7430655)
\curveto(489.62426289,38.9931389)(489.82319098,39.19585609)(490.11116485,39.35121769)
\curveto(490.40292427,39.51035193)(490.7704676,39.58992317)(491.21379594,39.58993164)
\curveto(491.66848583,39.58992317)(492.11370584,39.52171925)(492.5494573,39.38531969)
\lineto(492.33347802,38.27132127)
\curveto(492.07960388,38.33193982)(491.83520652,38.36225267)(491.60028518,38.36225992)
\curveto(491.36914642,38.36225267)(491.20242573,38.30731062)(491.10012263,38.19743362)
\curveto(491.00160309,38.09133156)(490.95234471,37.88482525)(490.95234733,37.57791409)
\lineto(490.95234733,37.15163919)
\lineto(492.14591706,37.15163919)
\lineto(492.14591706,35.89554913)
\lineto(490.95234733,35.89554913)
\lineto(490.95234733,31.11558653)
\lineto(489.35523735,31.11558653)
\lineto(489.35523735,35.89554913)
\lineto(488.46858555,35.89554913)
\lineto(488.46858555,37.15163919)
}
}
{
\newrgbcolor{curcolor}{0 0 0}
\pscustom[linestyle=none,fillstyle=solid,fillcolor=curcolor]
{
\newpath
\moveto(493.0553037,37.970087)
\lineto(493.0553037,39.44784001)
\lineto(494.65241368,39.44784001)
\lineto(494.65241368,37.970087)
\lineto(493.0553037,37.970087)
\moveto(493.0553037,31.11558653)
\lineto(493.0553037,37.15163919)
\lineto(494.65241368,37.15163919)
\lineto(494.65241368,31.11558653)
\lineto(493.0553037,31.11558653)
}
}
{
\newrgbcolor{curcolor}{0 0 0}
\pscustom[linestyle=none,fillstyle=solid,fillcolor=curcolor]
{
\newpath
\moveto(501.5466988,35.36696825)
\lineto(499.97232348,35.08278498)
\curveto(499.91927147,35.39727685)(499.79802006,35.63409601)(499.6085689,35.79324316)
\curveto(499.42289852,35.95238095)(499.18039571,36.03195219)(498.88105973,36.0319571)
\curveto(498.48320012,36.03195219)(498.16491517,35.8936498)(497.92620394,35.61704953)
\curveto(497.69127686,35.34422936)(497.57381456,34.88574748)(497.57381669,34.2416025)
\curveto(497.57381456,33.52545825)(497.69317142,33.01961254)(497.93188761,32.72406384)
\curveto(498.17438794,32.42851193)(498.49835654,32.28073677)(498.90379439,32.28073794)
\curveto(499.20691945,32.28073677)(499.45510593,32.36599167)(499.64835456,32.53650288)
\curveto(499.84159479,32.71080036)(499.97800262,33.00824522)(500.05757847,33.42883835)
\lineto(501.62627012,33.16170608)
\curveto(501.46333236,32.4417738)(501.15073107,31.89803702)(500.68846533,31.53049411)
\curveto(500.22618909,31.16295036)(499.60667019,30.9791787)(498.82990674,30.97917856)
\curveto(497.94704155,30.9791787)(497.24226774,31.25767803)(496.71558321,31.81467738)
\curveto(496.19268525,32.37167533)(495.9312369,33.1427585)(495.93123739,34.12792919)
\curveto(495.9312369,35.12446118)(496.1945798,35.89933346)(496.72126688,36.45254834)
\curveto(497.2479514,37.00954166)(497.96030342,37.28804098)(498.85832507,37.28804716)
\curveto(499.59340831,37.28804098)(500.17693071,37.12889851)(500.60889401,36.81061926)
\curveto(501.04463609,36.49611773)(501.35723738,36.01490121)(501.5466988,35.36696825)
}
}
{
\newrgbcolor{curcolor}{0 0 0}
\pscustom[linestyle=none,fillstyle=solid,fillcolor=curcolor]
{
\newpath
\moveto(503.95657322,35.3101316)
\lineto(502.50723855,35.5715802)
\curveto(502.67016955,36.15509815)(502.95056343,36.58705628)(503.34842103,36.86745592)
\curveto(503.74627579,37.14784404)(504.3373764,37.28804098)(505.12172463,37.28804716)
\curveto(505.83407345,37.28804098)(506.36454836,37.20278609)(506.71315094,37.03228221)
\curveto(507.06174395,36.86555561)(507.30614132,36.65147109)(507.44634378,36.39002802)
\curveto(507.59032431,36.13236351)(507.66231733,35.65683064)(507.66232306,34.96342801)
\lineto(507.64527207,33.09918576)
\curveto(507.64526635,32.56870887)(507.66989554,32.17653635)(507.71915972,31.92266702)
\curveto(507.77220142,31.67258518)(507.86882363,31.40355862)(508.00902665,31.11558653)
\lineto(506.42896767,31.11558653)
\curveto(506.387283,31.22168151)(506.33613006,31.37892943)(506.27550871,31.58733076)
\curveto(506.24898061,31.68205795)(506.23003508,31.74457821)(506.21867205,31.77489172)
\curveto(505.9458521,31.50965361)(505.6540909,31.31072552)(505.34338758,31.17810685)
\curveto(505.03267743,31.04548806)(504.70113062,30.9791787)(504.34874614,30.97917856)
\curveto(503.72733025,30.9791787)(503.23664096,31.14779394)(502.8766768,31.48502478)
\curveto(502.52049984,31.82225489)(502.34241184,32.24852937)(502.34241225,32.7638495)
\curveto(502.34241184,33.10486743)(502.42387763,33.40799595)(502.58680986,33.67323596)
\curveto(502.74974078,33.94225996)(502.97708717,34.14687171)(503.26884971,34.28707182)
\curveto(503.56439867,34.4310547)(503.9887786,34.55609521)(504.54199076,34.66219374)
\curveto(505.28844212,34.80238713)(505.80565516,34.93311131)(506.09363141,35.05436665)
\lineto(506.09363141,35.21350928)
\curveto(506.09362725,35.52042281)(506.01784512,35.73829643)(505.8662848,35.86713081)
\curveto(505.7147166,35.99974478)(505.42863906,36.06605414)(505.00805132,36.06605909)
\curveto(504.72386526,36.06605414)(504.50220253,36.00921755)(504.34306247,35.89554913)
\curveto(504.18391758,35.78566026)(504.05508796,35.59052128)(503.95657322,35.3101316)
\moveto(506.09363141,34.01425589)
\curveto(505.8890155,33.94604907)(505.56504689,33.86458328)(505.12172463,33.76985827)
\curveto(504.67839598,33.67512796)(504.38852933,33.58229485)(504.25212382,33.49135867)
\curveto(504.04372064,33.34358114)(503.93952022,33.15602037)(503.93952223,32.92867579)
\curveto(503.93952022,32.7051167)(504.02288056,32.51187227)(504.18960351,32.34894192)
\curveto(504.35632193,32.18600911)(504.56851189,32.10454332)(504.82617403,32.10454431)
\curveto(505.11414322,32.10454332)(505.38885344,32.19927098)(505.65030551,32.38872758)
\curveto(505.84354622,32.53271235)(505.97048129,32.70890581)(506.03111109,32.91730846)
\curveto(506.07278716,33.0537145)(506.09362725,33.31326829)(506.09363141,33.69597062)
\lineto(506.09363141,34.01425589)
}
}
{
\newrgbcolor{curcolor}{0 0 0}
\pscustom[linestyle=none,fillstyle=solid,fillcolor=curcolor]
{
\newpath
\moveto(512.01032561,37.15163919)
\lineto(512.01032561,35.87849814)
\lineto(510.91906186,35.87849814)
\lineto(510.91906186,33.44588935)
\curveto(510.91905934,32.95330317)(510.92853211,32.66533108)(510.94748018,32.58197221)
\curveto(510.97021228,32.5023995)(511.01757611,32.43609014)(511.08957182,32.38304392)
\curveto(511.16535126,32.32999516)(511.25628982,32.30347141)(511.36238776,32.3034726)
\curveto(511.51015995,32.30347141)(511.72424447,32.35462435)(512.00464195,32.45693157)
\lineto(512.14104992,31.21789251)
\curveto(511.76971375,31.05874994)(511.34912293,30.9791787)(510.8792762,30.97917856)
\curveto(510.59130163,30.9791787)(510.33174784,31.02654253)(510.10061404,31.1212702)
\curveto(509.86947685,31.21978696)(509.69896706,31.34482747)(509.58908415,31.49639211)
\curveto(509.48298799,31.6517451)(509.40910041,31.86014596)(509.3674212,32.12159531)
\curveto(509.33331828,32.30726052)(509.3162673,32.68238206)(509.31626821,33.24696106)
\lineto(509.31626821,35.87849814)
\lineto(508.58307538,35.87849814)
\lineto(508.58307538,37.15163919)
\lineto(509.31626821,37.15163919)
\lineto(509.31626821,38.35089258)
\lineto(510.91906186,39.28301371)
\lineto(510.91906186,37.15163919)
\lineto(512.01032561,37.15163919)
}
}
{
\newrgbcolor{curcolor}{0 0 0}
\pscustom[linestyle=none,fillstyle=solid,fillcolor=curcolor]
{
\newpath
\moveto(513.13000965,37.970087)
\lineto(513.13000965,39.44784001)
\lineto(514.72711963,39.44784001)
\lineto(514.72711963,37.970087)
\lineto(513.13000965,37.970087)
\moveto(513.13000965,31.11558653)
\lineto(513.13000965,37.15163919)
\lineto(514.72711963,37.15163919)
\lineto(514.72711963,31.11558653)
\lineto(513.13000965,31.11558653)
}
}
{
\newrgbcolor{curcolor}{0 0 0}
\pscustom[linestyle=none,fillstyle=solid,fillcolor=curcolor]
{
\newpath
\moveto(515.98889411,34.21886784)
\curveto(515.98889365,34.74933964)(516.11961782,35.26276357)(516.38106703,35.75914116)
\curveto(516.64251452,36.25550947)(517.0119524,36.63442012)(517.48938178,36.89587424)
\curveto(517.97059634,37.15731681)(518.5067549,37.28804098)(519.09785909,37.28804716)
\curveto(520.01103017,37.28804098)(520.7593787,36.99059612)(521.34290692,36.39571169)
\curveto(521.9264235,35.8046058)(522.2181847,35.05625727)(522.21819139,34.15066386)
\curveto(522.2181847,33.23748616)(521.92263439,32.47966486)(521.33153959,31.8771977)
\curveto(520.74422228,31.27851811)(520.00345196,30.9791787)(519.10922642,30.97917856)
\curveto(518.55601329,30.9791787)(518.02743293,31.10421921)(517.52348377,31.35430048)
\curveto(517.02331972,31.60438127)(516.64251452,31.97003004)(516.38106703,32.4512479)
\curveto(516.11961782,32.93625219)(515.98889365,33.52545825)(515.98889411,34.21886784)
\moveto(517.62578975,34.13361286)
\curveto(517.62578765,33.53493102)(517.76787914,33.07644913)(518.05206465,32.75816583)
\curveto(518.33624511,32.43987925)(518.68673746,32.28073677)(519.10354275,32.28073794)
\curveto(519.52034088,32.28073677)(519.86893868,32.43987925)(520.14933719,32.75816583)
\curveto(520.43351555,33.07644913)(520.57560704,33.53872012)(520.57561209,34.14498019)
\curveto(520.57560704,34.73607777)(520.43351555,35.19077055)(520.14933719,35.50905989)
\curveto(519.86893868,35.82734044)(519.52034088,35.98648291)(519.10354275,35.98648778)
\curveto(518.68673746,35.98648291)(518.33624511,35.82734044)(518.05206465,35.50905989)
\curveto(517.76787914,35.19077055)(517.62578765,34.73228866)(517.62578975,34.13361286)
}
}
{
\newrgbcolor{curcolor}{0 0 0}
\pscustom[linestyle=none,fillstyle=solid,fillcolor=curcolor]
{
\newpath
\moveto(528.96470139,31.11558653)
\lineto(527.36759141,31.11558653)
\lineto(527.36759141,34.19613318)
\curveto(527.36758668,34.84785641)(527.33348473,35.26844723)(527.26528544,35.4579069)
\curveto(527.19707689,35.65114698)(527.08529825,35.80081669)(526.92994918,35.90691646)
\curveto(526.77838063,36.01300665)(526.59460896,36.06605414)(526.37863363,36.06605909)
\curveto(526.10202512,36.06605414)(525.85383865,35.99027201)(525.63407347,35.83871248)
\curveto(525.4143023,35.6871435)(525.26273804,35.48632085)(525.17938023,35.23624395)
\curveto(525.09980646,34.9861588)(525.06002084,34.52388781)(525.06002326,33.84942959)
\lineto(525.06002326,31.11558653)
\lineto(523.46291328,31.11558653)
\lineto(523.46291328,37.15163919)
\lineto(524.94634995,37.15163919)
\lineto(524.94634995,36.26498738)
\curveto(525.47303345,36.9470214)(526.13612708,37.28804098)(526.93563284,37.28804716)
\curveto(527.28801545,37.28804098)(527.6100895,37.22362617)(527.90185596,37.09480253)
\curveto(528.1936119,36.96975604)(528.41338007,36.80871901)(528.56116115,36.61169097)
\curveto(528.71271948,36.41465194)(528.81691991,36.19109466)(528.87376274,35.94101846)
\curveto(528.93438221,35.6909326)(528.96469506,35.33286204)(528.96470139,34.8668057)
\lineto(528.96470139,31.11558653)
}
}
{
\newrgbcolor{curcolor}{0 0 0}
\pscustom[linestyle=none,fillstyle=solid,fillcolor=curcolor]
{
\newpath
\moveto(530.02754599,32.83773715)
\lineto(531.63033963,33.08213476)
\curveto(531.69854167,32.77142606)(531.83684406,32.53460691)(532.04524721,32.37167659)
\curveto(532.25364577,32.21253286)(532.54540697,32.13296162)(532.92053168,32.13296264)
\curveto(533.33354112,32.13296162)(533.64424785,32.20874375)(533.8526528,32.36030925)
\curveto(533.99284564,32.46640299)(534.06294411,32.60849448)(534.06294842,32.78658416)
\curveto(534.06294411,32.9078339)(534.02505305,33.00824522)(533.94927511,33.08781843)
\curveto(533.86969968,33.16359858)(533.69161168,33.23369705)(533.41501057,33.29811405)
\curveto(532.12671071,33.58229485)(531.31015826,33.84184864)(530.96535078,34.07677621)
\curveto(530.48792215,34.4026364)(530.24920845,34.85543462)(530.24920894,35.43517224)
\curveto(530.24920845,35.95806461)(530.45571475,36.39760096)(530.86872847,36.75378261)
\curveto(531.28173996,37.10995298)(531.92209896,37.28804098)(532.78980737,37.28804716)
\curveto(533.61582955,37.28804098)(534.2296648,37.1535277)(534.63131496,36.88450691)
\curveto(535.03295537,36.61547458)(535.30956015,36.2176184)(535.46113011,35.69093718)
\lineto(533.95495878,35.41243757)
\curveto(533.89053977,35.64735788)(533.76739381,35.82734044)(533.58552053,35.95238579)
\curveto(533.40742869,36.07742146)(533.15166401,36.13994172)(532.8182257,36.13994674)
\curveto(532.39763182,36.13994172)(532.09639785,36.08121057)(531.9145229,35.96375312)
\curveto(531.79326934,35.88038793)(531.73264363,35.77239839)(531.73264561,35.63978419)
\curveto(531.73264363,35.52610647)(531.78569112,35.42948426)(531.89178824,35.34991725)
\curveto(532.03577215,35.24381804)(532.5321451,35.09414833)(533.38090857,34.90090769)
\curveto(534.23345391,34.70765947)(534.82834362,34.47084032)(535.16557951,34.19044951)
\curveto(535.49901547,33.90626345)(535.66573615,33.51030183)(535.66574207,33.00256344)
\curveto(535.66573615,32.44935201)(535.43460066,31.97381915)(534.97233489,31.57596343)
\curveto(534.51005868,31.17810679)(533.82612496,30.9791787)(532.92053168,30.97917856)
\curveto(532.09829241,30.9791787)(531.44656609,31.14589938)(530.96535078,31.47934112)
\curveto(530.48792215,31.81278212)(530.17532087,32.26558035)(530.02754599,32.83773715)
}
}
\end{pspicture}

\caption{Módulos a ser desarrollados, y sus grados de dependencia.}
\label{paquetes}
\end{figure}

Se ha definido cuatro tipos de módulos en el sistema, estos son:

\begin{description}
\item [base] Módulos que pertenecen al núcleo del sistema (estos están
representados con color rojo, en la parte superior de diagrama).
\item [middle] Módulos para creación de espacios y recursos (representados con
color morado en el diagrama).
\item [app] Módulos para administración de recursos, perfiles, y otros
(representados con verde en la parte inferior del diagrama).
\item [util] Módulos que agregan funcionalidad a otros módulos (representados
con color café en el diagrama).
\end{description}

Las modularidad establecida en el sistema representa un modelo básico de
separación y en ningún caso podría considerarse totalmente refinado (eso escapa
del alcance de los objetivos del sistema).

\section{Grupos y privilegios}

Adicionalmente a la implementación de una lógica modular para el sistema, es
deseable el manejo dinámico de permisos, de forma que un administrador pueda
definir una conjunto de funcionalidades disponibles para ciertas categorías de
usuarios.

Para esto se han establecido dos módulos:

\begin{description}
\item [Roles] Módulo para la administración de grupos de usuarios, este debe
contemplar todas las operaciones CRUD, mencionadas para un recurso, además de la
manipulación de grupos de permisos para un rol común.
\item [Privileges] Módulo para la administración de credenciales en el sistema,
se debe contemplar la creación dinámica de permisos definidos por algún módulo.
\end{description}

\section{Gestión de contenido}

Para finalizar el conjunto de requisitos que se han establecido desarrollar
para el sistema, se plantea la necesitad de creación y administración de
diversas plantillas web, además de poder definirse pequeños utilitarios
alrededor de una cierta página.

Para esto se ha definido la creación de un modulo:

\begin{description}
\item [Templates] Módulo encargado de la reenderización de contenido, además de
la administración de funcionalidad adicional a una página (widgets), y la
definición de regiones para el sistema (debe cumplir las funciones mas básicas
de un sistema de administración de contenido CMS).
\end{description}

\section{Planificación}

Una vez definidas las funcionalidades a ser desarrolladas, y las herramientas
con las que se cuenta para tal desarrollo, en esta sección se ha de definir la
planificación que se ha determinado seguir.

\subsection{Iteraciones}

Para comenzar se ha determinado realizar el desarrollo del proyecto en
iteraciones, estas están detalladas en el cuadro \ref{iteraciones}.

\begin{table}
\centering
\begin{tabular}{|c|l|p{8.0cm}|}
\hline
Iteración & Módulo & Descripción \\
\hline

\multirow{4}{*}{1} &
Usuarios (USERS) &
\multirow{4}{8cm}{Análisis, diseño, e implementación de las funciones para el
manejo de usuarios, además de la creación de datos de prueba, e implementación
de la lógica de autenticación.} \\
 &  & \\
 &  & \\
 &  & \\
\hline

\multirow{4}{*}{2} &
Paquetes (PACKAGES) &
\multirow{4}{8cm}{Análisis, diseño, implementación, evaluaciones de las
funciones que proveen modularidad, manejo de credenciales, y manejo de roles de
usuarios en el sistema.} \\
 & Privilegios (PRIVILEGES) & \\
 & Roles (ROLES) & \\
 &  & \\
\hline

\multirow{3}{*}{3} &
Rutas (ROUTES) &
\multirow{3}{8cm}{Análisis, diseño, e implementación de las funciones para el
manejo de peticiones HTTP, y gestión de contenido.} \\
 & Plantillas (TEMPLATES) & \\
 & & \\
\hline

\multirow{5}{*}{4} &
Espacios (SPACES) &
\multirow{5}{8cm}{Análisis, diseño, e implementación de las funciones de
administración de espacios virtuales, además de la creación de las funciones
generales para la adición de recursos, y funciones utilitarias.} \\
 & Áreas (AREAS) & \\
 & Gestiones (GESTIONS) & \\
 & & \\
 & & \\
\hline

\multirow{4}{*}{5} &
Materias (SUBJECTS) &
\multirow{4}{8cm}{Análisis, diseño, e implementación de los espacios virtuales
formales, de acuerdo a la estructura que se aplica en el contexto de
implantación (UMSS).} \\
 & Grupos (GROUPS) & \\
 & Equipos (TEAMS) & \\
 & & \\
\hline

\multirow{4}{*}{6} &
Comunidades (COMMUNITIES) &
\multirow{4}{8cm}{Análisis, diseño, e implementación de los espacios virtuales
informales, de acuerdo a las estructuras clásicas que pueden verse en Internet.}
\\
 & Carreras (CARRERS) & \\
 & & \\
 & & \\
\hline

\multirow{3}{*}{7} &
Evaluaciones (EVALUATIONS) &
\multirow{3}{8cm}{Análisis, diseño, e implementación de las funciones para la
evaluación y calificaciones de los estudiantes, por parte de los docentes.} \\
 & Calificaciones (CALIFICATIONS) & \\
 & Conjuntos (GROUPSETS) & \\
\hline

\multirow{3}{*}{8} &
Contactos (CONTACTS) &
\multirow{3}{8cm}{Análisis, diseño, e implementación de la lógica de red social,
es decir, la gestión de contactos entre usuarios.} \\
 & Invitaciones (INVITATIONS) & \\
 & & \\
\hline

\multirow{7}{*}{9} &
Recursos (RESOURCES) &
\multirow{7}{8cm}{Análisis, diseño, e implementación de la gestión de recursos
básicos, además de la adición de estos a espacios virtuales determinados, y la
generalización de estos para brindar la posibilidad de extender su funcionalidad
para la posterior implementación de paquetes utilitarios.} \\
 & Notas (NOTES) & \\
 & Enlaces (LINKS) & \\
 & Sugerencias (FEEDBACK) & \\
 & & \\
 & & \\
 & & \\
\hline

\multirow{3}{*}{10} &
Archivos (FILES) &
\multirow{3}{8cm}{Análisis, diseño, e implementación de los recursos básicos
extendidos, es decir, aquellos que requieren manipular archivos adjuntos.} \\
 & Imágenes (PHOTOS) & \\
 & Vídeos (VIDEOS) & \\
\hline

\multirow{4}{*}{11} &
Etiquetas (TAGS) &
\multirow{4}{8cm}{Análisis, diseño, e implementación de las funciones
utilitarias para valoraciones sobre los recursos, además de la implementación
del sistema de reputación.} \\
 & Comentarios (COMMENTS) & \\
 & Valoraciones (VALORATIONS) & \\
 & & \\
\hline
\end{tabular}
\caption{Definición de iteraciones para el proyecto.}
\label{iteraciones}
\end{table}

